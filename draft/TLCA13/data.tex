\section{Encoding Term-Indexed Datatypes} \label{sec:data}
\paragraph{Algebraic encoding using sums and products:}
Recall that our motivation is to design a foundational calculus
that can embed term-indexed datatypes. For instance, we mentioned
an encoding for the length indexed lists ({\small\tt Vec}) in \S\ref{sec:motiv}.
For a more systematic encoding of GADTs in general
\cite{Sheard04equality,crary98intensional},
we need  equality constraints and existential quantification
over term indices as well as over type constructors.
It is well known that Leibniz equality over type constructors
can be defined within System \Fw. %% \ as follows:
%% \[
%% (\stackrel{\kappa}{=}) \triangleq \l X_1^\kappa.\,\l X_2^\kappa.\,
%% 	\forall X^{\kappa\to*}.\, X X_1 -> X X_2
%% \]
Similarly, in System \Fi, we can define Leibniz equality over term indices:
\[
(\stackrel{A}{=}) \triangleq
	\l i^A.\, \l j^A.\, \forall X^{A\to*}.\, X\{i\}\to X\{j\}
\]
Then, we can encode {\small\tt Vec} as sum of the types of its two constructors,
as follows:
\[ \mathtt{Vec} \triangleq \l A^{*}.\,\l i^\texttt{Nat}.\,
	(\exists j^\texttt{Nat}.\,(\texttt{S}\;j\stackrel{A}{=}i)\times A \times X\{j\})
	+
	(\texttt{Z}\stackrel{A}{=}i)
\]
where $+$ and $\times$ are the usual impredicative encoding of sums
and products. We can encode the existental quantification over indices
($\exists$ used above) as
$ \exists i^A.B \triangleq \forall X^{*}. (\forall i^A.B -> X) -> X $,
which is similar to the usual encoding of existential quantification
over types in System $\mathsf{F}$ or \Fw.

\begin{example} Another well-known example of term-indexed datatype
	is \mbox{$\lambda$-terms} indexed by its context information
	(\texttt{C}), guaranteeing that all variables are in scope
	(\ie, no free variables).
\begin{verbatim}
   data Lam ( C: Nat -> * ) { i: Nat } where
     LVar : C {i} -> Lam C {i}
     LApp : Lam C {i} -> Lam C {i} -> Lam C {i}
     LAbs : Lam C {S i} -> Lam C {i}
\end{verbatim}
is encoded as:
\[
\mathtt{Lam} \triangleq
\!\!\!
\begin{array}[t]{l}
\l C^{\mathtt{Nat}\to*}
\l i^\mathtt{Nat}.\,\forall X^{\mathtt{Nat}\to*}.
  (\forall j^\mathtt{Nat}.\,C\s j \to X\s j)
\\[1mm]
\qquad\qquad\qquad\qquad\quad\,
 \to(\forall j^\mathtt{Nat}.\,X\s j \to X\s j \to X\s j)
\\[1mm]
\qquad\qquad\qquad\qquad\quad\,
\to(\forall j^\mathtt{Nat}.\,X\s{\mathtt S\, j} \to X\s j)
\\[1mm]
\qquad\qquad\qquad\qquad\quad\,
  \to X\s i
\end{array}
\]
For a concrete representation one can consider
$\mathtt{Lam}\,\mathtt{Fin}$ where
\begin{verbatim}
   data Fin { i: Nat } where
     FZ : Fin{S i}
     FS : Fin{i} -> Fin{S i}
\end{verbatim}
This is encoded as
\[
\mathtt{Fin}\triangleq
\!\!\!
\begin{array}[t]{l}
\l i^{\mathtt{Nat}}.\,\forall X^{\mathtt{Nat}\to*}.\,
(\forall j^\mathtt{Nat}.\, X\s{\mathtt S\, j})
	\to (\forall j^\mathtt{Nat}.\, X\s j\to X\s{\mathtt S\,j})
	\to X\s i
\end{array}
\]
\end{example}

\paragraph{Nested term indices:}
TODO


%% Although it is possible to encode existential quantification over
%% higher kinded type constructors and term indices, there is a notion
%% called \emph{Kan extension} that can capture the above use pattern of
%% equality and existentals \cite{AbeMatUus05,JohannGhani08}.
%% Kan extension is a general notion for change of indices, which is like
%% the change of variables technique to simplify in mathematical formulae.
%% In System \Fi, we can encode a right Kan extention that changes indices
%% of type $A$ into indicies of type $B$ as a type constructor
%% $\Ran_{A,B}: (A\to B) \to (A\to*) \to B\to *$, defined as follows:
%% \[
%% \Ran_{A,B}
%% \triangleq
%% \l f^{A\to B}.\,
%%   \l X^{A\to*}.\,
%%     \l j^B.
%%       \forall i^A.\,
%%       (\s j \stackrel{B}{=} \s{f\,i})
%% 	  \to X\s i
%% \]
%% Using this right Kan extension type constructor, we can encode {\small\tt{Vec}}
%% as follows:

