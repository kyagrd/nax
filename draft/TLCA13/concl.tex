\section{Summary and Ongoing Work} \label{sec:concl}
System~\Fi\ is a strongly-normalizing, logically-consistent, higher-order
polymorphic lambda calculus that was designed to support the
definition of datatypes indexed by both terms and types.
In terms of expressivity, System~\Fi\ sits between System~\Fw\ and ICC.
We designed System~\Fi\ as a tool to reason about programming
languages with term-indexed datatypes. System \Fi\ can express
a large class of term-indexed datatypes, including datatypes
with nested term-indices.

One limitation of System \Fi\ is that it cannot express type-level
data structures such as lists that contain type elements.
We hope to overcome this limitation by extending System \Fi\ 
with first-class type representations \cite{DagMcb12}, which reflect
types as term-level data (a sort of a fully reflective version of
{\small\tt TypeRep} from \S\ref{intro}).

Our goal is to to build a unified programming and reasoning system,
which supports
(1) an expressive class of datatypes including nested term-indexed datatypes
and negative datatypes,
(2) logically consistent reasoning about program properties, and
(3) Hindley--Milner-style type inference.
Towards this goal, we are developing the programming language
Nax \cite{AhnSheFioPit12} based on System \Fi.
Nax is given semantics in terms of System~\Fi.
That is, all the primitive language constructs of Nax that are not present
in \Fi\ have translations into System \Fi. Such constructs include
Mendler-style eliminators, recursive type operators, and
pattern matching.

Some language features we want to include in Nax go beyond \Fi.
One of them is a recursion scheme that guarantee normalization
due to paradigmatic use of indices in datatypes. For instance,
some recursive computations always reduce a natural number term-index
in every recursive call. Although such computations obviously terminate,
we cannot express them in System \Fi, since term-indices in are erasable
-- \Fi\ only accepts terms that are already type-correct in \Fw.
We plan to explore extensions to System \Fi\ that enable such computations
while maintailing logical consistency.

