\section{Metatheory} \label{sec:theory}

The expectation is that System \Fi\ has all the nice properties of System \Fw,
yet is more expressive (\ie, can state finer grained program properties)
because of the addition of term-indexed types.

We show some basic well-formedness properties for
the judgments of \Fi\ in \S\ref{ssec:wf}.
We prove erasure properties of \Fi, which captures the idea that indices are
erasable since they are irrelevant for reduction in \S\ref{ssec:erasure}.
We show strong normalization, logical consistence, and subject reduction for
\Fi\ by reasoning about well-known calculi related to \Fi\ in \S\ref{ssec:sn}.

\subsection{Well-formedness and Substitution Lemmas} \label{ssec:wf}

We want to show that kinding and typing derivations give
well-formed results under well-formed contexts. That is,
kinding derivations ($\Delta |- F : \kappa$) result in well-sorted kinds
($|- \kappa$) under well-formed type-level contexts ($|- \Delta$)
(Proposition \ref{prop:wfkind}), and
typing derivations ($\Delta;\Gamma |- t : A$) result in well-kinded types
($\Delta;\Gamma |- A:*$) under well-formed type and term-level contexts
(Proposition \ref{prop:wftype}).\\
\begin{minipage}{.5\linewidth} \vskip1.1ex
\begin{proposition}\hspace*{-2.5ex}
\label{prop:wfkind}
$ \inference*{ |- \Delta & \Delta |- F : \kappa}{
        \qquad |- \kappa:\square \quad} \!$\\
\end{proposition}
\end{minipage}
\begin{minipage}{.5\linewidth} \vskip1.1ex
\begin{proposition} \label{prop:wftype} \hspace{-2.5ex}
$ \inference*{ \Delta |- \Gamma & \Delta;\Gamma |- t : A}{
        \qquad \Delta |- A : * \qquad} \!\!\!$\\
\end{proposition}
\end{minipage}
We can prove these well-formedness properties
by induction over the judgment\footnote{The proof for
        Propositions \ref{prop:wfkind} and \ref{prop:wftype} are
        mutually recursive.
        So, we prove these two propositions at the same time, using a combined judgment $J$,
        which is either a kinding judgment or a typing judgment
        (\ie, $J ::= \Delta |- F : \kappa \mid \Delta;\Gamma |- t : A$).}
and using the substitution lemma below.

\begin{lemma}[substitution]\mbox{}\\[-3mm]
\label{lem:subst}
\begin{enumerate}
\item
%\begin{lemma}[type substitution]
\label{lem:tysubst}
\mbox{\rm (type substitution)}
$\inference{\Delta,X^\kappa |- F:\kappa' & \Delta |- G:\kappa}
        {\Delta |- F[G/X]:\kappa'} $
%\end{lemma}
\medskip

\item
%\begin{lemma}[index substitution]
\label{lem:ixsubst}
\mbox{\rm (index substitution)}
$ \inference{\Delta,i^A |- F:\kappa & \Delta;\cdot |- s:A}
        {\Delta |- F[s/i]:\kappa} $
%\end{lemma}
\medskip

\item
%\begin{lemma}[term substitution]
\label{lem:tmsubst}
\mbox{\rm (term substitution)}
$ \inference{\Delta;\Gamma,x:A |- t:B & \Delta;\Gamma |- s:A}
        {\Delta;\Gamma |- t[s/x]:B} $
%\end{lemma}
\end{enumerate}
\end{lemma}
These substitution lemmas are fairly standard,
comparable to substitution lemmas
in other well-known systems such as \Fw\ or ICC.


\subsection{Erasure Properties} \label{ssec:erasure}

We define a meta-operation of index erasure that projects $\Fi$-types
to $\Fw$-types.
\begin{definition}[index erasure]\label{def:ierase}
\[ \fbox{$\kappa^\circ$}
 ~~~~ ~~
 *^\circ =
 *
 ~~~~ ~~
 (\kappa_1 -> \kappa_2)^\circ =
 {\kappa_1}^\circ -> {\kappa_2}^\circ
 ~~~~ ~~
 (A -> \kappa)^\circ =
 \kappa^\circ
\]
\[ \fbox{$F^\circ$}
 ~~~~ ~~
 X^\circ =
 X
 ~~~~ ~~~~ ~~~~ ~~~~ ~~~~ ~~~~
 (A -> B)^\circ =
 A^\circ -> B^\circ
\]
\[ \qquad \qquad
 (\lambda X^\kappa.F)^\circ =
 \lambda X^{\kappa^\circ}.F^\circ
 ~~~~ ~~~
 (\lambda i^A.F)^\circ =
 F^\circ
\]
\[ \qquad \qquad
 (F\;G)^\circ =
 F^\circ\;G^\circ
 ~~~~ ~~~~ ~~~~ ~~
 (F\,\{s\})^\circ =
 F^\circ
\]
\[ \qquad \qquad
 (\forall X^\kappa . B)^\circ =
 \forall X^{\kappa^\circ} . B^\circ
 ~~~~ ~~~
 (\forall i^A . B)^\circ =
 B^\circ
\]
\[ \fbox{$\Delta^\circ$}
 ~~~~
 \cdot^\circ = \cdot
 ~~~~ ~~
 (\Delta,X^\kappa)^\circ = \Delta^\circ,X^{\kappa^\circ}
 ~~~~ ~~
 (\Delta,i^A)^\circ = \Delta^\circ
\]
\[ \fbox{$\Gamma^\circ$}
 ~~~~
 \cdot^\circ = \cdot
 ~~~~ ~~
 (\Gamma,x:A)^\circ = \Gamma^\circ,x:A^\circ
\]
\end{definition}~\\
In addition, we define another meta-operation, which selects out
all the index variable bindings from the type-level context.
We use this in Theorem \ref{thm:ierasetmctxivs}.
\begin{definition}[index variable selection]
        \[ \fbox{$\Delta^\bullet$} ~~~~ \cdot^\bullet = \cdot \qquad
        (\Delta,X^\kappa)^\bullet = \Delta^\bullet \qquad
        (\Delta,i^A)^\bullet = \Delta^\bullet,i:A
\]
\end{definition}



\begin{theorem}[index erasure on well-sorted kinds]
\label{thm:ierasesorting}
        $\inference{|- \kappa : \square}{|- \kappa^\circ : \square}$
\end{theorem}
\begin{proof}
        By induction on the sort ($\kappa$).
        \qed
\end{proof}
\begin{remark}
For any well-sorted kind $\kappa$ in \Fi,
$\kappa^\circ$ is a well-sorted kind in \Fw.
\end{remark}

\begin{theorem}[index erasure on well-formed type-level contexts]
\label{thm:ierasetyctx}
$ \inference{|- \Delta}{|- \Delta^\circ} $
\end{theorem}
\begin{proof}
        By induction on the type-level context ($\Delta$)
        and using Theorem \ref{thm:ierasesorting}.
        \qed
\end{proof}
\begin{remark}
For any well-formed type-level context $\Delta$ in \Fi,
$\Delta^\circ$ is a well-formed type-level context in \Fw.
\end{remark}

\begin{theorem}[index erasure on kind equality]\label{thm:ierasekindeq}
$ \inference{|- \kappa=\kappa':\square}
        {|- \kappa^\circ=\kappa'^\circ:\square}
$
\end{theorem}
\begin{proof}
        By induction on the kind equality derivation
        ($|- \kappa=\kappa':\square$).
        \qed
\end{proof}
\begin{remark}
For any well-sorted kind equality $|- \kappa=\kappa':\square$ in \Fi,
$\kappa^\circ$ and $\kappa'^\circ$ are the syntactically same \Fw\ kinds.
Note that no variables can appear in the erased kinds by definition of
the erasure operation on kinds.
\end{remark}

\begin{theorem}[index erasure on well-kinded type constructors]
\label{thm:ierasekinding}
\[ \inference{|- \Delta & \Delta |- F : \kappa}
                {\Delta^\circ |- F^\circ : \kappa^\circ}
\]
\end{theorem}
\begin{proof}
        By induction on the kinding derivation ($\Delta |- F : \kappa$).
        We use Theorem \ref{thm:ierasetyctx} in the ($Var$) case,
        Theorem \ref{thm:ierasekindeq} in the ($Conv$) case, and
        Theorem \ref{thm:ierasesorting} in the ($\lambda$) and ($\forall$) cases.
        \qed
\end{proof}
\begin{remark} In the theorem above, $F^\circ$ is a well-kinded type constructor in \Fw.
\end{remark}

\noindent
\begin{minipage}{.55\linewidth}
\begin{lemma}\label{lem:ieraseSubstTyVar}
$(F[G/X])^\circ = F^\circ[G^\circ/X]$
\end{lemma}
\end{minipage}
\begin{minipage}{.4\linewidth}
\begin{lemma}\label{lem:ieraseSubstIndex}
$(F[s/i])^\circ = F^\circ$
\end{lemma}
\end{minipage}
\begin{theorem}[index erasure on type constructor equality]
\label{thm:ierasetyconeq}
\[ \inference{\Delta |- F=F':\kappa}
                {\Delta^\circ |- F^\circ=F'^\circ:\kappa^\circ}
\]
\end{theorem}
\begin{proof}
        By induction on the derivation of the type constructor equality judgment
        ($\Delta |- F=F':\kappa$). We also use Proposition \ref{prop:wfkind}
        and Lemmas \ref{lem:ieraseSubstTyVar} and \ref{lem:ieraseSubstIndex}.
        \qed
\end{proof}
\begin{remark}
When $\Delta |- F=F':\kappa$ is a valid type constructor equality in \Fi,
$\Delta^\circ |- F^\circ=F'^\circ:\kappa^\circ$
is a valid type constructor equlity in \Fw.
\end{remark}

\begin{theorem}[index erasure on well-formed term-level contexts
                prepended by index variable selection]
\label{thm:ierasetmctxivs}
\[ \inference{\Delta |- \Gamma}{\Delta^\circ |- (\Delta^\bullet,\Gamma)^\circ}
\]
\end{theorem}
\begin{proof}
        By induction on the term-level context ($\Gamma$)
        and using Theorem \ref{thm:ierasekinding}.
        \qed
\end{proof}
\begin{remark}
We can also show that
$\inference{\Delta |- \Gamma}{\Delta^\circ |- \Gamma^\circ}$
and prove Corollary \ref{cor:ierasetypingifree} directly.
\end{remark}

\begin{theorem}[index erasure on well-typed terms]
\label{thm:ierasetypingall}
$ \inference{\Delta |- \Gamma & \Delta;\Gamma |- t : A}
                {\Delta^\circ;(\Delta^\bullet,\Gamma)^\circ |- t : A^\circ}
$
\end{theorem}
\begin{proof}
        By induction on the typing derivation ($\Delta;\Gamma |- t : A$).
        We also make use of Theorems
        \ref{thm:ierasesorting},
        \ref{thm:ierasekinding},
        \ref{thm:ierasetyconeq}, and
        \ref{thm:ierasetmctxivs}.
        \qed
\end{proof}
\begin{remark}
        In the theorem above, $t$ is a well typed term
        in \Fw\ as well as in \Fi.
        % TODO also say why this is useful (Nax predefeind term)
\end{remark}

\begin{corollary}[index erasure on index-free well-typed terms]
\label{cor:ierasetypingifree}
\[ \inference{ \Delta |- \Gamma & \Delta;\Gamma |- t : A}
                {\Delta^\circ;\Gamma^\circ |- t : A^\circ}
                {\enspace(\dom(\Delta)\cap\FV(t) = \emptyset)}
\]
\end{corollary}

\subsection{Strong Normalization and Logical Consistency} \label{ssec:sn}
Strong normalization is a corollary of the erasure property since we know that
System~\Fw\ is strongly normalizing. Logical consistency is immediate since
System~\Fi\ is a resriction of \emph{restricted implicit calculus}
\cite{Miquel00} or $\text{ICC}^{-}$ \cite{BarrasB08}, which are
restrictions of ICC~\cite{Miquel01} known to be logically consistent.
Subject reduction is also immediate for the same reason.

