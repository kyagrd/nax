\section{System \Fi} \label{sec:Fi}
System \Fi\ is a higher-order polymorphic lambda calculus 
designed to extend System~\Fw\ by the inclusion of term-indices.
The syntax and rules of System~\Fi\ are described in~\Figs{FiSyntax},
\ref{fig:FiTyping} and~\ref{fig:eqFi}. 
The extensions new to System~\Fi, which are not originally part of System~\Fw, 
are highlighted by \newFi{\text{grey boxes}}.
Eliding all the grey boxes from~\Figs{FiSyntax}, \ref{fig:FiTyping}
and~\ref{fig:eqFi}, one obtains a version of System~\Fw\ 
with Curry-style terms and the typing context separated into two parts
(type-level context $\Delta$ and term-level context $\Gamma$).

We assume readers to be familiar with System~\Fw\
and focus on describing the new constructs of \Fi, which appear in grey boxes.

\begin{figure}
\paragraph{Syntax:}
\qquad~\quad Term Variables \quad $x,y,z,\ldots,i,j,k,\ldots$ ~\vskip-4ex
\begin{align*}
&\text{Type Constructor Variables} & X,Y,Z,\ldots \\
&\text{Sort}
        & \square
        \\
&\text{Kinds}
        & \kappa                &~ ::= ~ *
                                \mid \kappa -> \kappa
                                \mid \newFi{A -> \kappa}
\\
&\text{Type Constructors}
        & A,B,F,G               &~ ::= ~ X
                                \mid A -> B
                                \mid \lambda X^\kappa.F
                                \mid F\,G
                                \mid \forall X^\kappa . B \\
                                &&& \qquad\qquad\qquad\quad~
                                \mid \newFi{\lambda i^A.F
                                \mid F\,\{s\}
                                \mid \forall i^A . B}
\\
&\text{Terms}
        & r,s,t                 &~ ::= ~ x \mid \lambda x.t \mid r\;s
\\
&\text{Typing Contexts}
        & \Delta                &~ ::= ~ \cdot
                                \mid \Delta, X^\kappa
                                \mid \newFi{\Delta, i^A} \\
&       & \Gamma                &~ ::= ~ \cdot
                                \mid \Gamma, x : A
\end{align*}\vskip-2ex
\paragraph{Reduction:} ~~ \fbox{$t \rightsquigarrow t'$} \\
$
 ~~~~
   \inference{}{(\lambda x.t)\,s \rightsquigarrow t[s/x]}
 ~~~~
   \inference{t \rightsquigarrow t'}{\lambda x.t \rightsquigarrow \lambda x.t'}
 ~~~~
   \inference{r \rightsquigarrow r'}{r\;s \rightsquigarrow r'\;s}
 ~~~~
   \inference{s \rightsquigarrow s'}{r\;s \rightsquigarrow r\;s'}
$
\caption{Syntax and Reduction rules of \Fi}
\label{fig:FiSyntax}
\end{figure}

\begin{figure}
\paragraph{Well-formed typing contexts:}
\[ \fbox{$|- \Delta$}
 ~~~~
   \inference{}{|- \cdot}
 ~~~
   \inference{|- \Delta & |- \kappa:\square}
             {|- \Delta,X^\kappa}
      \big( X\notin\dom(\Delta) \big)
 ~~~ \newFi{
   \inference{|- \Delta & \cdot |- A:*}
             {|- \Delta,i^A}
      \big( i\notin\dom(\Delta) \big) }
\]
\[ \fbox{$\Delta |- \Gamma$}
 ~~~~
   \inference{|- \Delta}{\Delta |- \cdot}
 ~~~
   \inference{\Delta |- \Gamma & \Delta |- A:*}
             {\Delta |- \Gamma,x:A}
      \big( x\notin\dom(\Gamma) \big)
 \qquad\qquad\qquad\qquad\qquad\qquad\qquad
\]
~\\
\paragraph{Sorting:} ~~ \fbox{$|- \kappa : \square$}
\[
  \inference[($A$)]{}{|- *:\square}
 ~~~~
   \inference[($R$)]{|- \kappa:\square & |- \kappa':\square}
                    {|- \kappa -> \kappa' : \square}
 ~~~~
   \newFi{
   \inference[($Ri$)]{\cdot |- A:* & |- \kappa:\square}
                     {|- A -> \kappa : \square} }
\]
~\\
\paragraph{Kinding:} ~~ \fbox{$\Delta |- F : \kappa$}
$ \quad
 ~~~~
   \inference[($Var$)]{X^\kappa\in\Delta & |- \Delta}
                       {\Delta |- X : \kappa}
 ~~~~
   \inference[($->$)]{\Delta |- A : * & \Delta |- B : *}
                     {\Delta |- A -> B : * }
$
\[
  \inference[($\lambda$)]{|- \kappa:\square & \Delta,X^\kappa |- F : \kappa'}
                          {\Delta |- \lambda X^\kappa.F : \kappa -> \kappa'}
 ~~~~ \quad ~~
 \newFi{
  \inference[($\lambda i$)]{\cdot |- A:* & \Delta,i^A |- F : \kappa}
                            {\Delta |- \lambda i^A.F : A->\kappa}
                    }
\]
\[ \inference[($@$)]{ \Delta |- F : \kappa -> \kappa'
                    & \Delta |- G : \kappa }
                    {\Delta |- F\,G : \kappa'}
 ~~~~
 \newFi{
   \inference[($@i$)]{ \Delta |- F : A -> \kappa
                     & \Delta;\cdot |- s : A }
                     {\Delta |- F\,\{s\} : \kappa}
             }
\]
\[ \inference[($\forall$)]{|- \kappa:\square & \Delta, X^\kappa |- B : *}
                          {\Delta |- \forall X^\kappa . B : *}
 ~~~~ \quad ~~~
 \newFi{
   \inference[($\forall i$)]{\cdot |- A:* & \Delta, i^A |- B : *}
                            {\Delta |- \forall i^A . B : *}
                    }
\]
\[ \newFi{
   \inference[($Conv$)]{ \Delta |- A : \kappa
                       & \Delta |- \kappa = \kappa' : \square }
                       {\Delta |- A : \kappa'} }
\]
~\\
\paragraph{Typing:} ~~ \fbox{$\Delta;\Gamma |- t : A$}
$ \quad
 ~~~~
 \inference[($:$)]{(x:A) \in \Gamma & \Delta |- \Gamma} 
                    {\Delta;\Gamma |- x:A}
 ~~~~ \newFi{
   \inference[($:i$)]{i^A \in \Delta & \Delta |- \Gamma} 
                     {\Delta;\Gamma |- i:A} }
$
\[
   \inference[($->$$I$)]{\Delta |- A:* & \Delta;\Gamma,x:A |- t : B}
                        {\Delta;\Gamma |- \lambda x.t : A -> B}
 ~~~~ ~~~~
   \inference[($->$$E$)]{\Delta;\Gamma |- r : A -> B & \Delta;\Gamma |- s : A}
                        {\Delta;\Gamma |- r\;s : B}
\]
\[ \inference[($\forall I$)]{|- \kappa:\square & \Delta, X^\kappa;\Gamma |- t : B}
                            {\Delta;\Gamma |- t : \forall X^\kappa.B}
                            (X\notin\FV(\Gamma))
 ~~~~ ~~~~
   \inference[($\forall E$)]{ \Delta;\Gamma |- t : \forall X^\kappa.B
                            & \Delta |- G:\kappa }
                            {\Delta;\Gamma |- t : B[G/X]}
\]
\[ \!\!\newFi{
   \inference[($\forall I i$)]{\cdot |- A:* & \Delta, i^A;\Gamma |- t : B}
                              {\Delta;\Gamma |- t : \forall i^A.B}
   \left(\begin{matrix}
                i\notin\FV(t),\\
                i\notin\FV(\Gamma)\end{matrix}\right)
 ~~
   \inference[($\forall E i$)]{ \Delta;\Gamma |- t : \forall i^A.B
                              & \Delta;\cdot |- s:A}
                              {\Delta;\Gamma |- t : B[s/i]} }
\]
\[ \inference[($=$)]{\Delta;\Gamma |- t : A & \Delta |- A = B : *}
                    {\Delta;\Gamma |- t : B}
\]
\caption{Well-formedness, Sorting, Kinding, and Typing rules of \Fi}
\label{fig:FiTyping}
\end{figure}

\begin{figure}
\paragraph{Kind equality:} ~~ \fbox{$|- \kappa=\kappa' : \square$}
%% $
%%  ~~~~
%%    \inference{}{|- * = *:\square}
%%  ~~~~
%%    \inference{ |- \kappa_1 = \kappa_1' : \square
%%              & |- \kappa_2 = \kappa_2' : \square }
%%              {|- \kappa_1 -> \kappa_2 = \kappa_1' -> \kappa_2' : \square}
%% $
%%\[
 \qquad\newFi{
   \inference{\cdot |- A=A':* & |- \kappa=\kappa':\square}
             {|- A -> \kappa = A' -> \kappa' : \square} }
%%  ~~~~
%%    \inference{|- \kappa=\kappa':\square}
%%              {|- \kappa'=\kappa:\square}
%%  ~~~~
%%    \inference{ |- \kappa =\kappa' :\square
%%              & |- \kappa'=\kappa'':\square}
%%              {|- \kappa=\kappa'':\square}
%% \]
~\\~\\
\paragraph{Type constructor equality:} ~~ \fbox{$\Delta |- F = F' : \kappa$}
\[ \inference{\Delta,X^\kappa |- F:\kappa' & \Delta |- G:\kappa}
             {\Delta |- (\lambda X^\kappa.F)\,G = F[G/X]:\kappa'}
 ~~~~ \newFi{
   \inference{\Delta,i^A |- F:\kappa & \Delta;\cdot |- s:A}
             {\Delta |- (\lambda i^A.F)\,\{s\} = F[s/i]:\kappa} }
\]
%% \[ \inference{\Delta |- X:\kappa }{\Delta |- X=X:\kappa}
%%  ~~~~
%%    \inference{\Delta |- A=A':* & \Delta |- B=B':*}{\Delta |- A-> B=A'-> B':*}
%% \]
%% \[ \inference{|- \kappa:\square & \Delta,X^\kappa |- F=F' : \kappa'}
%%              {\Delta |- \lambda X^\kappa.F=\lambda X^\kappa.F':\kappa-> \kappa'}
%%  ~~~~ \quad ~
%%  \newFi{
%%    \inference{\cdot |- A:* & \Delta,i^A |- F=F' : \kappa}
%%              {\Delta |- \lambda i^A.F=\lambda i^A.F' : A -> \kappa}
%%      }
%% \]
%% \[ \inference{\Delta |- F=F':\kappa->\kappa' & \Delta |- G=G':\kappa}
%%              {\Delta |- F\,G = F'\,G' : \kappa'}
  ~~~~ \newFi{
  \inference{\Delta |- F=F':A->\kappa & \Delta;\cdot |- s=s':A}
              {\Delta |- F\,\{s\} = F'\,\{s'\} : \kappa}
      }
%% \]
%% \[
%%    \inference{|- \kappa:\square & \Delta,X^\kappa |- B=B':*}
%%              {\Delta |- \forall X^\kappa.B=\forall X^\kappa.B':*}
%%  ~~~~ \qquad
%%  \newFi{
%%    \inference{\cdot |- A:* & \Delta,i^A |- B=B':*}
%%              {\Delta |- \forall i^A.B=\forall i^A.B':*} }
%% \]
%% \[ \inference{\Delta |- F = F' : \kappa}{\Delta |- F' = F : \kappa}
%%  ~~~~
%%    \inference{\Delta |- F = F' : \kappa & \Delta |- F' = F'' : \kappa}
%%              {\Delta |- F = F'' : \kappa}
%% \]
~\\~\\
\paragraph{Term equality:} ~~ \fbox{$\Delta;\Gamma |- t = t' : A$}
$
%% \qquad
 ~~~~
   \inference{\Delta;\Gamma,x:A |- t:B & \Delta;\Gamma |- s:A}
             {\Delta;\Gamma |- (\lambda x.t)\,s=t[s/x] : B} $
%% \[
%%    \inference{\Delta;\Gamma |- x:A}{\Delta;\Gamma |- x=x:A}
%%  ~~~~
%%    \inference{\quad~~ \Delta |- A:* \\ \Delta;\Gamma,x:A |- t=t':B}
%%              {\Delta;\Gamma |- \lambda x.t = \lambda x.t':B}
%%  ~~~~
%%    \inference{\Delta;\Gamma |- r=r':A-> B \\ \Delta;\Gamma |- s=s':A \qquad~}
%%              {\Delta;\Gamma |- r\;s=r'\;s':B}
%% \]
%% \[ \inference{|- \kappa:\square & \Delta, X^\kappa;\Gamma |- t=t' : B}
%%              {\Delta;\Gamma |- t=t' : \forall X^\kappa.B}
%%              (X\notin\FV(\Gamma))
%%  ~~~~ ~~~~
%%    \inference{ \Delta;\Gamma |- t=t' : \forall X^\kappa.B
%%              & \Delta |- G:\kappa }
%%              {\Delta;\Gamma |- t=t' : B[G/X]}
%% \]
%% \[ \newFi{
%%    \inference{\cdot |- A:* & \Delta, i^A;\Gamma |- t=t' : B}
%%              {\Delta;\Gamma |- t=t' : \forall i^A.B}
%%    \left(\begin{smallmatrix}
%%                 i\notin\FV(t),\\
%%                 i\notin\FV(t'),\\
%%                 i\notin\FV(\Gamma)\end{smallmatrix}\right)
%%  ~~~~
%%    \inference{ \Delta;\Gamma |- t=t' : \forall i^A.B
%%              & \Delta;\cdot |- s:A}
%%              {\Delta;\Gamma |- t=t' : B[s/i]} }
%% \]
%% \[ \inference{\Delta;\Gamma |- t=t':A}{\Delta;\Gamma |- t'=t:A}
%%  ~~~~
%%    \inference{\Delta;\Gamma |- t=t':A & \Delta;\Gamma |- t'=t'':A}
%%              {\Delta;\Gamma |- t=t'':A}
%% \]
%% ~\\
%% $\phantom{A}$ 
%% Some of the rules in \Fig{FiTyping} require determining if two types are equal.
%% Since types occur in kinds, and terms occur in types, the rules
%% for equality are more elaborate than the equality rules of System \Fw.
%% 
%% $\phantom{A}$
%% The first two rules of type constructor equality,
%% and the first rule of the term equality,
%% state the essence of $\beta$-equivalence.
%% The last two rules in the Kind/Type/Term equalities
%% state the symmetric and transitive nature of the rules.
%% The other rules are structural congruence rules, which follow
%% the structure of the rules in \Fig{FiTyping}.
\caption{Equality rules of \Fi\ (only the key rules are shown)}
\label{fig:eqFi}
\end{figure}

\paragraph{Kinds\!}(\Fig{FiSyntax}).\;
The key extension to \Fw\ is the addition of term-indexed arrow kinds of
the form \newFi{A -> \kappa}. This allows type constructors to have terms
as indices. The rest of the development of \Fi\ flows naturally from
this single extension.

\paragraph{Sorting\!}(\Fig{FiTyping}).\; \label{sorting}
The formation of indexed arrow kinds is
governed by the sorting rule \newFi{(Ri)}. The rule $(Ri)$ specifies that
an indexed arrow kind $A -> \kappa$ is well-sorted when $A$ has kind $*$
under the empty type-level context ($\cdot$) and $\kappa$ is well-sorted.
Requiring $A$ to be well-kinded under the empty type-level context avoids
dependent kinds (\ie, kinds depending on type-level or value level bindings).
That is, $A$ should to be a closed type of kind $*$,
which does not contain any free type variables or index variables.
For example, $(\textit{List}\,X -> *)$ is not a well-sorted kind since $X$
appears free, while $((\forall X^{*}\!.\,\textit{List}\,X) -> *)$ is a well-sorted kind.

\paragraph{Typing contexts\!}(\Fig{FiSyntax}).\;
Typing contexts are split into two parts.
Type level contexts ($\Delta$) for type-level (static) bindings,
and term-level contexts ($\Gamma$) for term-level (dynamic) bindings.
A new form of index variable binding ($i^A$) can appear in type-level contexts
in addition to the traditional type variable bindings ($X^\kappa$).
There is only one form of term-level binding ($x:A$) that appears in
term-level contexts. Note, both $x$ and $i$ represent the same
syntactic category of ``Type Variables''. The distinction between
$x$ and $i$ is only a convention for the sake of readability.

\paragraph{Well-formed typing contexts\!}(\Fig{FiTyping}).\;
A type-level context $\Delta$ is well-formed if (1) it is either empty,
or (2) extended by a type variable binding $X^\kappa$ whose kind $\kappa$ is
well-sorted under $\Delta$, or (3) extended by an index binding $i^A$
whose type $A$ is well-kinded under the empty type-level context at kind $*$.
This restriction is similar to the one that occurs in the sorting rule ($Ri$)
for term-indexed arrow kinds (see the paragraph {\textit{Sorting}}).
The consequence of this is that, in typing contexts and in sorts,
$A$ must be a closed type (not a type constructor!) without free variables.

A term-level context $\Gamma$ is well-formed under a type-level context
$\Delta$ when it is either empty or extended by a term variable binding
$x:A$ whose type $A$ is well-kinded under $\Delta$.

\paragraph{Type constructors and their kinding rules\!}(\Figs{FiSyntax} and \ref{fig:FiTyping}).\;
We extend the type constructor syntax by three constructs,
and extend the kinding rules accordingly.

\newFi{\lambda i^A.F} is the type-level abstraction over an index
(or, index abstraction). Index abstractions introduce indexed arrow kinds
by the kinding rule \newFi{(\lambda i)}. Note, the use of the new form of
context extension, $i^A$, in the kinding rule ($\lambda i$).

\newFi{F\,\{s\}} is the type-level term-index application. In contrast to
the ordinary type-level type-application ($F\,G$) where the argument ($G$) is
a type (of arbitrary kind). The argument of an term-index application ($F\,\{s\}$) is
a term ($s$). We use the curly bracket notation around an index argument
in a type to emphasize the transition from ordinary type to term, and
to emphasize that $s$ is a term-index, which is erasable. Index applications
eliminate indexed arrow kinds by the kinding rule \newFi{(@i)}.
Note, we type check
the term-index ($s$) under the current type-level context paired with
the empty term-level context ($\Delta;\cdot$) since we do not want
the term-index ($s$) to depend on any term-level bindings.
Otherwise, we would admit value dependencies in types.

\newFi{\forall i^A . B} is an index polymorphic type.
The formation of indexed polymorphic types is governed by
the kinding rule \newFi{\forall i}, which is very similar to
the formation rule ($\forall$) for ordinary polymorphic types.

In addition to the rules ($\lambda i$), ($@ i$), and ($\forall i$),
we need a conversion rule \newFi{(Conv)} at kind level. This is because
the new extension to the kind syntax $A -> \kappa$ involves types.
Since kind syntax involves types, we need more than simple structural
equality over kinds (see \Fig{eqFi}). For instance, $A -> \kappa$ and 
are $A' -> \kappa$ equivalent kinds when $A'$ and $A$ are equivalent types.
Only the key equality rules are shown in \Fig{eqFi}, and the other
structural rules (one for each sorting/kinding/typing rule) and
the congruence rules (symmetry, transitivity) are omitted.

\paragraph{Terms and their typing rules}(\Figs{FiSyntax} and \ref{fig:FiTyping}).\;
The term syntax is exactly the same as other Curry-style calclui.
We write $x$ for ordinary term variables introduced by
term-level abstractions ($\lambda x.t$).
We write $i$ for index variables introduced by
index abstractions ($\lambda i^A.F$) and by
index polymorphic types ($\forall i^A.B$). As discussed earlier,
the distinction between $x$ and $i$ is only for readability.

Since \Fi\ has index polymorphic types ($\forall i^A . B$),
we need typing rules for index polymorphism:
\newFi{(\forall I i)} for index generalization
and \newFi{(\forall E i)} for index instantiation.
These rules are similar to the type generalization ($\forall I$)
and the type instantiation ($\forall I$) rules, but involve
indices, rather than types, and have additional side conditions
compared to their type counterparts.

The additional side condition $i\notin\FV(t)$ in the ($\forall I i$) rule
prevents terms from accessing the type-level index variables introduced by
index polymorphism. Without this side condition, $\forall$-binder
would no longer behave polymorphically, but instead would behave as
a dependent function binder, which are usually denoted by $\Pi$ in
dependent type theories. Such side conditions on generalization rules
for polymorphism are fairly standard in dependent type theories that
distinguish between polymorphism (or, erasable arguments) and
dependent functions (\eg, IPTS\cite{LingerS08}, ICC\cite{Miquel01}).

The index instantiation rule ($\forall E i$) requires that
the term-index $s$, which instantiates $i$, be well-typed
in the current type-level context paired with the empty term-level context
($\Delta;\cdot$) rather than the current term-level context,
since we do not want indices to depend on term-level bindings.

In addition to the rules ($\forall I i$) and ($\forall E i$) for
index polymorphism, we need an additional variable rule \newFi{(:i)}
to access index variables already in scope. In examples
like  ($\l i^A.F\{s\}$)  and ($\forall i^A.F\{s\}$), the term
($s$) should be able to access the index variable ($i$) already in scope.
