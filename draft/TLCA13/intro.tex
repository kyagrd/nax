\section{Introduction} \label{intro}

%% Term-indices - erasable, possibly nested
%% 
%% Higher-order polymorphism
%% 
%% Strong normalization and logical consistency

We are interested in the use of indexed types to state and
maintain program properties. A type parameter
(like {\tt Int} in {\tt (List Int)}) usually tells us
something about data stored in values of that type.
A type-index (like {\tt 3} in {\tt (Vector Int 3)}) states
an inductive property of values with that type. For example,
values of type {\tt (Vector Int 3)} have three elements.

Indexed types come in two flavors: \emph{type-indexed} and \emph{term-indexed}
types. 

%{\it Type-indexing} involves higher-kinded parametric polymorphism,
%but there also exists more sophisticated types such as
%Generalized Algebraic Datatypes~(GADTs) \cite{Sheard04equality}.

An example of type-indexing is a definition of a \emph{representation type}
\cite{Crary98} using GADTs in Haskell:\vspace*{-2pt}
\begin{lstlisting}[basicstyle={\ttfamily\small},language=Haskell,mathescape]
   data TypeRep t where
     RepInt  :: TypeRep Int
     RepBool :: TypeRep Bool
     RepPair :: TypeRep a -> TypeRep b -> TypeRep (a,b)
\end{lstlisting}\vspace*{-2pt}
Here, a value of type {\tt\small (TypeRep t)} is isomorphic in shape with
the type-index {\tt t}. For example,
{\tt (RepPair RepInt RepBool) :: TypeRep (Int,Bool)}.


An example of {\it Term-indices} are datatypes with
indices ranging over data structures, such as natural numbers
(like {\tt Z}, {(\tt S Z})) or lists (like {\tt Nil} or ({\tt Cons Z Nil})). 
A classic example of a term-index is the second parameter to
the length-indexed list type {\tt Vec}~(as in ({\tt Vec Int (S Z)})).

In languages such as Haskell\footnote{see \S\ref{sec:relwork} for
        a very recent GHC extension, which enable true term-indices.}
        or OCaml \cite{GarNor11}, which support GADTs with only type-indexing,
term-indices are simulated (or faked) by reflecting data at the type-level
with uninhabited type constructors. For example,\vspace*{-2pt}
\begin{lstlisting}[basicstyle={\ttfamily\small},language=Haskell,mathescape]
   data S n
   data Z
   data Vec t n where
     Cons :: a -> Vec a n -> Vec a (S n)
     Nil  :: Vec a Z
\end{lstlisting}\vspace*{-2pt}
This simulation comes with a number of problems. First, there is no way to say
that types such as {\tt (S Int)} are ill-formed, and second the costs
associated with duplicating the constructors of data to be used
as term-indices.
Nevertheless, GADTs with ``faked" term-indices have become extremely popular
as a light-weight, type-based mechanism to raise the confidence of users
that software systems maintain important properties.

Our approach in this direction is to design a new foundational calculus,
System~\Fi, for functional programming languages with term-indexed
datatypes.  In a nutshell, System~\Fi\ is obtained by minimally extending
System~\Fw\ with type-indexed kinds.  Notably, this yields a logical
calculus that is expressive enough to embed non-dependent
\emph{term-indexed datatypes} and their eliminators. Our contributions in
this development are as follows.\vspace*{-.8ex}
\begin{itemize}
\item 
  Identifying the features that are needed in a higher-order polymorphic
  $\lambda$-calculus to embed term-indexed datatypes~(\S\ref{sec:motiv}),
  in isolation from other features normally associated with such
  calculi~(e.g., general recursion, large elimination, dependent types).
\item 
  The design of the calculus, System \Fi\ (\S\ref{sec:Fi}), and its use to
  study properties of languages with term-indexed datatypes, including
  the embedding of
  term-indexed datatypes into the calculus~(\S\ref{sec:data}) using
  Church or Mendler style encodings, and  proofs about these encodings.
  For instance, one can use System~\Fi\ to prove that
  the Mendler-style eliminators for GADTs
  \cite{AhnShe11} are normalizing.
\item 
  Showing that System~\Fi\ enjoys a simple erasure
  property (\S\ref{ssec:erasure}) 
  and inherits meta-theoretic
  results, strong normalization from \Fw\  and logical consistency from
  ICC (\S\ref{ssec:sn}).
\end{itemize}\vspace*{-3pt}

