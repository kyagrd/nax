\section{Mendler-style primitive recursion with a sized index}
\label{sec:mprsi}

In \S\ref{sec:mendler} and \S\ref{sec:HOASeval}, we discussed
Mendler-style iteration with a syntactic inverse, \lstinline{msfcata},
which is particularly useful for defining functions over
mixed-variant datatypes. We demonstrated the usefulness of
\lstinline{msfcata} by defining functions over HOAS:
\begin{itemize}
\item the string formatting function \lstinline{showExp} for
	the untyped HOAS using \lstinline{msfcata0}
	(Figure\;\ref{lst:HOASshow}) and
\item the type-preserving evaluator \lstinline{eval} for
	the simply-typed HOAS using \lstinline{msfcata1}
	(Figure\;\ref{lst:HOASeval}).
\end{itemize}

In this section, we speculate about another Mendler-style recursion scheme,
\lstinline{mprsi}, motivated by an example similar to the \lstinline{eval}
function. The name \lstinline{mprsi} stands for
Mendler-style primitive recursion with a sized index.

\begin{figure}
\lstinputlisting[
	caption={
		Two evaluators for the simply-typed $\lambda$-calculus in HOAS.
		One uses a native (Haskell) value domain (\lstinline{eval}),
		the other uses a user-defined value domain (\lstinline{veval}).
		\label{lst:HOASevalV}
		},
	firstline=4]{HOASevalV.hs}
\vspace*{-3ex}
\end{figure}

We review the \lstinline{eval} example and then compare
it to our motivating example \lstinline{veval} for \lstinline{mprsi}.
Both \lstinline{eval} and \lstinline{veval} are illustrated
in Figure\;\ref{lst:HOASevalV}. Recall that this code is written in Haskell,
following the Mendler-style conventions.
The function \lstinline{eval :: Exp t -> Id t} is
a type preserving evaluator that evaluates an HOAS expression of type
\lstinline{t} to a (Haskell) value of type \lstinline{t}.
The \lstinline{eval} function always terminates because
\lstinline{msfcata1} always terminates. Recall that \lstinline{msfcata1}
and \lstinline{Rec1} can be embedded into System~\Fw.

The motivating example \lstinline{veval :: Exp t -> Val t} is also 
a type preserving evaluator. Unlike \lstinline{eval} it evaluates to 
a user-defined value domain \lstinline{Val} of type \lstinline{t} (rather
than a Haskell value). The definition of \lstinline{veval} is similar to
\lstinline{eval} -- both of them are defined using \lstinline{msfcata1}.
The first equation of \lstinline{phi} for evaluating
the \lstinline{Lam}-expression is essentially the same as
the corresponding equation in the definition of \lstinline{eval}.
The second equation of \lstinline{phi} for evaluating
the \lstinline{App}-expression is also similar in structure to
the corresponding equation in the definition of \lstinline{eval},
but the use of \lstinline{unVal} is problematic. Note, the definition of
\lstinline{unVal} relies on pattern matching against \lstinline{In1}.
Recall that one cannot freely pattern match against a recursive value
in Mendler style. Recursive values must be analyzed (or, eliminated) by
using Mendler-style recursion schemes. It is not a problem to use
\lstinline{unId} in the definition of \lstinline{eval} because
\lstinline{Id} is non-recursive.

It is not likely that \lstinline{unVal} can be defined using any of
the existing Mendler-style recursion schemes. So, we designed
a new Mendler-style recursion scheme that can express \lstinline{unVal}.
The new recursion scheme \lstinline{mprsi} extends \lstinline{mprim} with
an additional uncast operation. Recall that \lstinline{mprim} has
two abstract operations, call and cast. So, \lstinline{mprsi} has
three abstract operations, call, cast, and uncast. In the following paragraphs,
we explain the design of \lstinline{mprsi} step-by-step.

Let us try to define \lstinline{unVal} using \lstinline{mprim1}, and see
where it falls short. \lstinline{mprim1} provides two abstract operations,
\lstinline{cast} and {call}, as you can see from its type sigaure:
\begin{lstlisting}
mprim1 :: (forall r i. (forall i. r i -> Mu1 f i)  -- cast
              -> (forall i. r i -> a i)      -- call
              -> (f r i -> a i)        ) -> Mu1 f i -> a i
\end{lstlisting}
We attempt to define \lstinline{unVal} using \lstinline{mprim1} as follows:
\begin{lstlisting}
unVal :: Mu1 V (t1 -> t2) -> (Mu1 V t1 -> Mu1 V t2)
unVal = mprim1 phi where
  phi cast call (VFun f) = ...
\end{lstlisting}
Inside the \lstinline{phi} function, we have a function
\lstinline{f :: (r t1 -> r t2)} over abstract recursive values.
We need to cast \lstinline{f} into a function over concrete recursive values
\lstinline{(Mu1 V t1 -> Mu1 V t2)}.
We should not need to use |call|, since we do not expect
to use any recursion to define |unVal|.
So the only available operation to
work with is \lstinline{cast :: (forall i.r i -> Mu1 f i)}.
Composing \lstinline{cast} with \lstinline{f}, we can get
\lstinline{(cast . f) :: (r t1 -> Mu1 V t2)}, whose codomain
\lstinline{(Mu1 V t2)} is exactly what we want. But, the domain
is still abstract \lstinline{(r t1)} rather than being concrete
\lstinline{(Mu1 V t1)}. We are stuck.

What additional abstract operation would help us complete
the definition of \lstinline{unVal}? We need an abstract operation
to cast from \lstinline{(r t1)} to \lstinline{(Mu1 V t1)}
in a contravariant position. If we had an inverse of cast,
\lstinline{uncast :: (forall i.Mu1 f i -> r i)}, we can
complete the definition of \lstinline{unVal} by composing
\lstinline{uncast}, \lstinline{f}, and \lstinline{cast}.
Observe that \lstinline{uncast . f . cast :: (Mu1 V t1 -> Mu1 V t2)}.
Thus, we can formulate \lstinline{mprsi1} with a naive type signature
as follows:
\begin{lstlisting}
mprsi1 :: (forall r i. (forall i. r i -> Mu1 f i)  -- cast
               -> (forall i. Mu1 f i -> r i)  -- uncast
               -> (forall i. r i -> a i)      -- call
               -> (f r i -> a i)        ) -> Mu1 f i -> a i

mprsi1 phi (In1 x) = phi id id (mprsi1 phi) x
\end{lstlisting}
Although the type signature above is type correct, it is too powerful.
The Mendler-style uses types to forbid non terminating computations
as ill-typed. Having both \lstinline{cast} and \lstinline{uncast} supports
the same ability as freely pattern matching over recursive values,
which can lead to non-termination as we showed in the introduction.
To recover the guarantee of termination, we need to restrict the use of
either \lstinline{cast} and \lstinline{uncast}, or both.

Let us see how this non-termination might occur. If we allowed
\lstinline{mprsi1} with the naive type signature above, we could write
an evaluator (similar to \lstinline{veval} but for an untyped HOAS),
which does not always terminate. This evaluator would diverge for terms
with self application. Typed terms use the type index to prevent
such diverging terms.

We walk through the process of defining an untyped HOAS.
The base structures of the untyped HOAS and its value domain
can be defined as follows:
\begin{lstlisting}
data ExpF_u r t = Lam_u (r t -> r t) | App_u (r t) (r t)
data V_u r t = VFun_u (r t -> r t)
\end{lstlisting}
Fixpoints of the structures above represent the untyped HOAS and
its value domain. Here, the index \lstinline{t} does not track
the types of terms but remains constant everywhere.
Why did we believe that \lstinline{veval} always terminates?
Because it evaluates a well-typed HOAS, whose type is encoded as
an index \lstinline{t} of the recursive datatype \lstinline{(Exp t)}. That is,
the use of indices as types is the key to the termination property.
Therefore, our idea is to restrict the use of the abstract operations
by enforcing constraints over their indices. We allow the use
of the abstract operations only over abstract values whose indices are
smaller in size compared to the size of the argument index.
For the \lstinline{veval} example, we define being smaller as
the structural ordering over types, that is, \lstinline{t1 < (t1 -> t2)}
and \lstinline{t2 < (t2 -> t1)}.
We have two candidates for the type signature of \lstinline{mprsi1}:
\begin{itemize}
\item Candidate 1: restrict uses of both \lstinline{cast} and \lstinline{uncast}
\begin{lstlisting}
mprsi1 :: (forall r j. (forall i. (i<j) => r i -> Mu1 f i)  --  cast
               -> (forall i. (i<j) => Mu1 f i -> r i)  --  uncast
               -> (forall i.          r i -> a i)      --  call
               -> (f r j -> a j)          ) -> Mu1 f i -> a i
\end{lstlisting}
\item Candidate 2: restrict the use of \lstinline{uncast} only
\begin{lstlisting}
mprsi1 :: (forall r j. (forall i.           r i -> Mu1 f i)  --  cast
               -> (forall i. (i<j) =>  Mu1 f i -> r i)  --  uncast
               -> (forall i.           r i -> a i)      --  call
               -> (f r j -> a j)          ) -> Mu1 f i -> a i
\end{lstlisting}
\end{itemize}
We strongly believe that the first candidate always terminate,
but it might be overly restrictive. Maybe the second candidate is
enough to guarantee termination? Both candidates allow defining
\lstinline{veval}, since one can define \lstinline{unVal}
using \lstinline{mprsi1} with either one of the candidates.
Both candiates forbid the definition of an evaluator over the untyped HOAS,
because neither support extracting functions from the untyped value domain.

We need further studies to prove termination properties of \lstinline{mprsi}.
The sized-type approach, discussed in the related work section,
seems to be relevant to showing termination of \lstinline{mprsi}.
However, existing theories on sized-types are not directly applicable to
\lstinline{mprsi} because they are focused on positive datatypes, but
not negative datatypes.


