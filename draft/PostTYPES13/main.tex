%This is a template for producing LIPIcs articles.
%See lipics-manual.pdf for further information.

\documentclass[a4paper,UKenglish]{lipics}
  %for A4 paper format use option "a4paper", for US-letter use option "letterpaper"
  %for british hyphenation rules use option "UKenglish", for american hyphenation rules use option "USenglish"
 % for section-numbered lemmas etc., use "numberwithinsect"
 
\usepackage{microtype}%if unwanted, comment out or use option "draft"

%\graphicspath{{./graphics/}}%helpful if your graphic files are in another directory

\bibliographystyle{plain}% the recommended bibstyle

% Author macros %%%%%%%%%%%%%%%%%%%%%%%%%%%%%%%%%%%%%%%%%%%%%%%%
\title{Mendler-style Recursion Schemes for Mixed-Variant Datatypes\footnote{
	This work was partially supported by someone.} }
%% \titlerunning{A Sample LIPIcs Article} %optional, in case that the title is too long; the running title should fit into the top page column

\author[1]{Ki Yung Ahn}
\author[1]{Tim Sheard}
\author[2]{Marcelo Fiore}
\affil[1]{Department of Computer Science, Portland State University\\
  Oregon, USA\\
  \texttt{\{kya,sheard\}@cs.pdx.edu}}
\affil[2]{Computer Laboratory, University of Cambridge\\
  Cambridge, UK\\
  \texttt{Marcelo.Fiore@cl.cam.ac.uk}}
\authorrunning{KY. Ahn, T. Sheard and M. Fiore} %mandatory. First: Use abbreviated first/middle names. Second (only in severe cases): Use first author plus 'et. al.'

\Copyright{Ki Yung Ahn, Tim Sheard and Marcelo Fiore}%mandatory. LIPIcs license is "CC-BY";  http://creativecommons.org/licenses/by/3.0/

\subjclass{
D.3.3 [Programming Languages]: Language Constructs and Features
--- Data types and structures, Polymorphism, and Recursion;
F.3.3 [Logics and Meanings of Programs]: Studies of Program Constructs
-- Functional constructs, Program and recursion schemes, and Type structure;
F.4.1 [Mahtematical Logic and Formal Systems]: Mathematical Logic
--- Lambda calculus and related systems
}
\keywords{
Mendler-style recursion, Higher-Order Abstract Syntax (HOAS),
mixed-variant datatypes, negative datatypes, termination}
% mandatory: Please provide 1-5 keywords
%%%%%%%%%%%%%%%%%%%%%%%%%%%%%%%%%%%%%%%%%%%%%%%%%%%%%%%%%

%Editor-only macros (do not touch as author)%%%%%%%%%%%%%%%%%%%%%%%%%%%%%%%%%%%
\serieslogo{}%please provide filename (without suffix)
\volumeinfo%(easychair interface)
  {Billy Editor, Bill Editors}% editors
  {2}% number of editors: 1, 2, ....
  {Conference title on which this volume is based on}% event
  {1}% volume
  {1}% issue
  {1}% starting page number
\EventShortName{}
\DOI{10.4230/LIPIcs.xxx.yyy.p}% to be completed by the volume editor
%%%%%%%%%%%%%%%%%%%%%%%%%%%%%%%%%%%%%%%%%%%%%%%%%%%%%%%%%

\newcommand{\cf}[0]{{cf.}}
\newcommand{\eg}[0]{{e.g.}}
\newcommand{\ie}[0]{{i.e.}}
\newcommand{\aka}[0]{{a.k.a.}}

\newcommand{\F}[0]{{\ensuremath{\mathsf{\uppercase{F}}}}}
\newcommand{\Fw}[0]{{\ensuremath{\mathsf{\uppercase{F}}_{\!\omega}}}}
\newcommand{\Fixw}[0]{{\ensuremath{\mathsf{\uppercase{F}\lowercase{ix}}_{\omega}}}}
\newcommand{\Fwhat}[0]{{\ensuremath{\mathsf{\uppercase{F}}_{\!\omega}\!\!\char`\^}}}

\newcommand{\MIt}[1]{\ensuremath{\textbf{\textit{mit}}_{#1}}}
\newcommand{\MPr}[1]{\ensuremath{\textbf{\textit{mpr}}_{#1}}}
\newcommand{\mpr}[1]{\ensuremath{\textbf{\textit{mpr}}_{#1}}}
\newcommand{\MsfIt}[1]{\ensuremath{\textbf{\textit{msfit}}_{#1}}}
\newcommand{\msfit}[1]{\ensuremath{\textbf{\textit{msfit}}_{#1}}}
\newcommand{\mprsi}[1]{\ensuremath{\textbf{\textit{mprsi}}_{#1}}}
\newcommand{\lift}[0]{\textit{lift}}
\newcommand{\In}[1]{\ensuremath{\textbf{\textit{In}}_{#1}}}
\newcommand{\inL}[0]{\textit{in}_{L}}
\newcommand{\inR}[0]{\textit{in}_{R}}

\newcommand{\textcomment}[1]{\rm\color{blue}#1}

%% \setlength{\thickmuskip}{0mu}
%% \setlength{\medmuskip}{0mu}
%% \setlength{\thinmuskip}{0mu}
\lstset{captionpos=t,
	float,
	keepspaces=false,
	abovecaptionskip=-\medskipamount,
	language=Haskell,
 	basicstyle=\it,
 	keywordstyle=\rm\bfseries,
	commentstyle=\color{blue},
	escapechar={\%},
	stringstyle=\rm\ttfamily,
	deletekeywords={Bool,Int,Integer,String,show,const,id,length,List},
 	literate=
 		{forall}{{$\forall$}}1
 		{:}{{$:$}}1
 		{;}{{$;$}}1
 		{::}{{$::$}}1
 		{->}{{$\to$}}2
 		{=}{{$=$}}1
 		{=>}{{$\Rightarrow$}}2
 		{*}{{$*$}}1
 		{\\}{{$\lambda$}}1
		{Phi0}{{Phi$_{*}$}}3
		{Phi1}{{Phi$_{*\to*}$}}5
		{Phi0'}{{Phi$_{*}^{\phantom{.}\prime}$}}3
		{Phi1'}{{Phi$_{*\to*}^{\phantom{.}\prime}$}}5
		{Mu0}{{$\mu_{*}$}}1
		{In0}{{\In{*}}}2
		{out0}{{out$_{*}$}}3
		{Mu1}{{$\mu_{*\to*}$}}3
		{In1}{{\In{*\to*}}}4
		{out1}{{out$_{*\to*}$}}5
		{Rec0}{{$\mu_{*}^\prime$}}1
		{Roll0}{{$\In{*}^\prime$}}2
		{unRoll0}{{unIn$_{*}^\prime$}}4
		{Inverse0}{{Inverse$_{*}$}}6
		{Rec1}{{$\mu_{*\to*}^\prime$}}3
		{Roll1}{{$\In{*\to*}^\prime$}}4
		{unRoll1}{{unIn$_{*\to*}^\prime$}}6
		{Inverse1}{{Inverse$_{*\to*}$}}8
 		{phi}{{$\varphi$}}1
 		{t1}{{$t_1$}}1
 		{t1}{{$t_1$}}1
 		{t2}{{$t_2$}}1
 		{e1}{{$e_1$}}1
 		{e2}{{$e_2$}}1
		{mpr}{{mpr}}3
		{mpr1}{{mpr$_{* \to *}$}}6
		{mprim}{{\mpr{}}}3
		{mprim1}{{\mpr{* \to *}}}6
		{mprsi}{{\mprsi{}}}5
		{mprsi1}{{\mprsi{* \to *}}}7
		{uncast}{{uncast}}6
		{cast}{{cast}}4
		{call}{{call}}4
		{unVal}{{un\!Val}}4
		{mcata}{{\MIt{}}}3
		{mcata0}{{\MIt{*}}}3
		{mcata1}{{\MIt{* \to *}}}5
		{msfcata}{{\msfit{}}}4
		{msfcata'}{{\msfit{}$^\prime$}}5
		{msfcata0}{{\msfit{*}}}5
		{msfcata1}{{\msfit{* \to *}}}7
 		{+}{{$+$}}1
		{++}{{$+\!\!\!+$}}2
 		{(}{{$($}}1
 		{)}{{$)$}}1
 		{[}{{$[$}}1
 		{]}{{$]$}}1
 		{\{}{{$\{$}}1
 		{\}}{{$\}$}}1
 		{|}{{$\mid$}}1
		{Exp}{{Exp}}3
		{Expr}{{Expr}}4
		{Exp_u}{{Exp$_u$}}3
		{ExpF}{{ExpF}}4
		{ExpF_u}{{ExpF$_u$}}4
		{App_u}{{App$_u$}}4
		{Lam_u}{{Lam$_u$}}4
		{V_u}{{V\!$_u$}}1
		{Val}{{Val}}3
		{VFun_u}{{VFun$_u$}}4
		{Exp'}{{Exp$\,^\prime$}}4
		{show'}{{show$\,^\prime$}}4
		{showExp}{{showExp}}7
		{const}{const}4
		{String}{String}5
		{Bool}{{Bool}}4
		{List}{{List}}3
		{Int}{{Int}}3
		{inv}{{inv}}3
		{nil}{{nil}}2
		{len}{{len}}3
		{length}{{length}}4
		{cons}{{cons}}3
		{ev}{{ev}}2
		{val}{{val}}2
		{eval}{{eval}}3
		{veval}{{veval}}4
		{vs}{{vs}}2
		{vars}{{vars}}3
		{lam}{{lam}}3
		{app}{{app}}3
		{Id}{{Id}}1
		{MkId}{{MkId}}4
		{unId}{{unId}}4
		{exp2expr}{{exp2expr}}7
		{expr2exp'}{{expr2exp$^\prime$}}7
		{exp'2expr}{{exp$^\prime$2expr}}7
		{p2r}{{p2r}}2
}


\begin{document}

\maketitle

\begin{abstract}
Some concepts, such as Higher-Order Abstract Syntax (HOAS),
are most naturally expressed by \emph{mixed-variant datatypes}
(\aka\ negative (recursive) datatypes). Unfortunately,
mixed-variant datatypes are often outlawed in formal reasoning systems
based on the Curry--Howard correspondence (\eg, Coq, Agda), because
the conventional recursion schemes (or, induction principles) supported in
such systems cannot guarantee termination for mixed-variant datatypes.

There is an alternative style of formulating recursion schemes,
known as the Mendler style, that can guarantee termination for
arbitrary datatypes. Ahn and Sheard \cite{AhnShe11} formulated
a Mendler-style recursion scheme (\msfit{}), and demonstrated
examples involving regular (\ie, non-indexed) mixed-variant datatypes
(\eg, untyped $\lambda$-calculus in HOAS). Their examples demonstrate
an advantage of the Mender style -- termination guarantee for
arbitrary datatypes, including mixed-variant ones. They proved
termination of the examples via embedding into System~\Fw.

Another advantage of the Mendler style is that recursion schemes
naturally extend to non-regular (\ie, indexed) datatypes.
Our type-preserving evaluator example for the simply-typed HOAS, defined as
a type-indexed mixed-variant datatype, demonstrates both advantages of
the Mendler style. This example has its own theoretical value, which is
a novel discovery that the simply-typed HOAS evaluation is expressible
within System~\Fw. To our knowledge, this is the first report of
a simply-typed HOAS evaluator (without translation through first-order syntax)
that is equipped with correct-by-construction proofs
(in the Curry--Howard sense) of both type-preservation and normalization.
We also develop further theoretical discussions on the \Fw-embedding of \msfit{}
and propose yet another recursion scheme (\mprsi{}), which is also useful for
mixed-variant datatypes.

Our study promotes the usefulness of the recursion schemes,
which are useful over mixed-variant datatypes. We hope our
work motivates designing logical reasoning systems that support
wider range of datatypes including mixed-variant ones.
\end{abstract}

\chapter{Introduction}\label{ch:intro}

\section{Programming and Formal Reasoning}\label{sec:intro:motiv}
In this dissertation, we contribute to answering the question:
``how does one build a seamless system where programmers can both
write (functional) programs and formally reason about those programs''.
In late 1960s, \citet{Howard69} observed that natural deduction, which is
a proof system of a formal logic, and a typed lambda calculus, which is
a model of computation, are directly related --
a proof of a proposition corresponds to a program and its type.
Since this observation, known as the Curry--Howard correspondence,
logicians and programming language researchers 
have  dreamed of
building a system in which one can both write programs
(\ie, model computation) and formally reason about (\ie, construct proofs of)
the properties (\ie, types) of those programs.

However, building a practical system that unifies programming and
formal reasoning, based on the Curry--Howard correspondence, is still
an open research problem. The gap between the conflicting
design goals of typed functional programming languages, such as ML and Haskell,
and formal reasoning systems, such as Coq and Agda, is still wide.

\begin{itemize}

\item
Programming languages are typically designed to achieve
computational expressiveness. They often sacrifice logical consistency
to achieve this goal. Programmers should be able to
conveniently express all possible computations, regardless of whether those
computations have a logical interpretation or not.

\item
Formal reasoning systems are typically designed to achieve logical consistency.
They often sacrifice computational expressiveness to achieve that goal.
Users expect that it is only possible to prove true propositions,
and it is impossible to prove falsity. They are willing
live with the difficultly (or even impossibility) to
express certain computations within the reasoning system,
to achieve logical consistency.

\end{itemize}

As a result, the recursion schemes of programming languages and
formal reasoning systems differ considerably.
Programming languages provide unrestricted general recursion,
to conveniently express computations
that may or may not terminate.
Formal reasoning systems provide induction principles for sound reasoning,
or, in the computational view, principled recursion schemes
that can only express terminating computation.

The two different design goals also lead to significant differences
in their type system as well.
Programming languages are based on \emph{recursive types},
which which place only syntactic restrictions on the definition of new types.
Programmers can express computations over a wide variety types.
In addition, most (statically typed) functional programing languages have
clear distinction between terms and types (\ie, terms do not appear in types).
Reasoning systems are usually based on \emph{inductive types},
which place semantic restrictions, accepting only type definitions that support
conventional induction principles.
In addition, most reasoning systems, based on the Curry--Howard correspondence,
allow types to depend on terms (\ie, terms can appear in types) to specify
fine grained properties.

This dissertation explores a sweet spot where one can benefit from
the advantages of both programming languages and formal reasoning systems.
That is, we design a unified language system, called Nax, which is
logically consistent while being able to conveniently express
many useful computations. We do this by placing few restriction on type definitions,
as is done in programming languages, but also provide a rich set of
non-conventional recursion schemes (or, induction principles) that
always terminate. These non-conventional recursion schemes are known as
\emph{the Mendler style}. Another major design choice in Nax is
supporting \emph{term indices} in types, a middle ground, which sits between
polymorphic types and dependent types.

In the following section, we explain what we mean by the sweet spot between programming languages
and reasoning systems. Our thesis is that the design choices we explain below
are reasonable for achieving the goal of combining programming and resoning systems.

\section{Thesis}\label{sec:intro:thesis}
Whatever design choices we make, the sweet spot should have the following features.

\begin{enumerate}[(1)]
 \item \textbf{A convenient programming} style
         supported by the major constructs of
         modern functional programming languages: 
         parametric polymorphism, recursive datatypes,
         recursive functions, and type inference,
 \item \textbf{An expressive logic}
         that can specify fine-grained program properties using types, and terms that
         witness proofs of these properties 
         (the Curry--Howard correspondence),
 \item \textbf{A small theory} based upon a minimal foundational calculus that is
         expressive enough to support the programming features, expressive
         enough to embed propostions and proofs about
         programs, and logically consistent
         to avoid paradoxical proofs in the logic, and
 \item \textbf{A simple implementation} that keeps the trusted base small.
\end{enumerate}
We claim that a language design based on \emph{Mendler-style recursion schemes}
and \emph{term-indexed types} can lead to a system that supports these four
features.

\paragraph{}
From a bird's-eye view, the following chapters back up our claim as follows:
Mendler-style recursion schemes support (1) because they are based on
parametric polymorphism and well-defined over a wide range of datatypes.
Term-indexed types support (2), because they can statically track program
properties. For instance the size of data structures can be tracked by using
a natural number term in their types.
To support (3), we design several foundational calculi, each which extends
a well known polymorphic lambda calculus with term-indexed types.
Mendler-style recursion schemes also also support (4) because their
termination is type-based -- no need for an auxiliary termination checker.

In next section, we summarize important ideas mentioned in our thesis above.

\section{Preliminary concepts}\label{sec:intro:concepts}
We give summaries of the following preliminary concepts:
Curry--Howard correspondence (\S\ref{sec:intro:concepts:CH}),
Mendler-style recursion schemes
(\S\ref{sec:intro:concepts:CH}, \S\ref{sec:intro:concepts:mendler}),
and term-indexed types (\S\ref{sec:intro:concepts:indexed}).
Further details and historical backgrounds on each of these concepts
will appear in the following chapters (see \S\ref{sec:intro:overview}
for the overview of chapter organization).

\subsection{The Curry--Howard correspondence and Normalization}
\label{sec:intro:concepts:CH}
One promising approach to designing a system that unifies
logical reasoning and programming is \emph{the Curry--Howard correspondence}.
Howard \cite{Howard69} observed that a typed model of computation
(\ie, a typed lambda calculus) gives an interpretation to a (natural deduction)
proof system (for an intuitionistic logic). More specifically, one can interpret
a type (in the lambda calculus) as a formula (in the logic) and
a term of that type, as a proof for that formula. For instance,
the typing rule for function applications (APP) in a typed lambda calculus
corresponds to Modus Ponens (MP) in a logic:
\[ \inference[(APP)]{\Gamma |- t_1 : A -> B & \Gamma |- t_2 : A}{
        \Gamma |- t_1~t_2 : B}
 ~~~~~~~~
   \inference[(MP)]{A -> B & A}{B}
\]
As you can see above, combining terms ($t_1$ and $t_2$) to build a new term
($t_1~t_2$) can be interpreted as combining proofs for formulae
($A -> B$ and $A$), to construct a proof for a new formula ($B$).
More generally, we may expect that programming (\ie, building larger terms)
corresponds to constructing larger proofs, but only when the typed lambda calculi
meets certain standards -- \emph{type soundness} and \emph{normalization}.

The Curry--Howard correspondence is a promising approach to designing a
unified system for both logical reasoning and programming. Only one language
system is needed for both the logic and the programming language. An
alternate approach is to use an external logical language to talk about
programs as the objects that the logic reasons about. In this approach, one
has the obligation to argue that the soundness of the logic, with respect to
the programming language semantics, holds.

Under the Curry--Howard correspondence, the logic is internally related to the
semantics of program -- there is no need to argue for the soundness of the
logic,  externally outside of the programming language system. The soundness
of the logic follows directly from the type soundness of the language under
the Curry--Howard correspondence.

Let us consider a proposition to be true
(or, valid) when it has a canonical (\ie, cut-free) proof.
That is, there is a program, whose type is the proposition under
consideration, and that program has a normal form. 
By type soundness, any term,
of that type, will preserve its type during the reduction steps. Thus
reduction preserves truthfulness. If we assume
that the language is normalizing (\ie, every well-typed term reduces to
a normal form), then any term of that type which is a non-canonical proof,
implies the existence of a canonical proof, which in turn implies that
the proposition specified by the type is indeed true. That is, all provable
propositions are valid (\ie, the logic is sound) when the language is
\emph{type sound} and \emph{normalizing}.

\emph{Normalization} is also essential for the consistency of the logic.
For the lambda calculus to be interpreted as a \emph{consistent} logic,
there must be no diverging terms. A diverging term (\ie, a term that does
not have a normal form) may inhabit any arbitrary type. Thus, a diverging term
can be a proof for any proposition under the Curry--Howard correspondence.
General purpose functional programming languages (\eg, Haskell and ML), that
support unrestricted general recursion, cannot be interpreted as a consistent
logic, since they allow diverging terms (\ie, non-terminating programs).
For example, a diverging Haskell definition $\textit{loop} = \textit{loop}$
can be given an arbitrary type such as
$\textit{loop}\mathrel{::}\textit{Bool}$,
$\textit{loop}\mathrel{::}\textit{Int} -> \textit{Bool}$,
and even $\textit{loop}\mathrel{::}\forall a. a$, which is a proof of false.


Therefore, useful logical reasoning systems based on the Curry--Howard
correspondence (\eg, Coq, Agda) never support language features that can
lead to diverging terms. For example, in both Coq and Agda,
unrestricted general recursion (at term level) is not supported. 
Instead, these logical reasoning systems
often provide principled recursion schemes over recursive types that are
guaranteed to normalize. 

Recursive types (\ie, recursion at type level)
can also lead to diverging terms when they are not restricted carefully.
Many of the conventional logical reasoning systems, based on
Curry--Howard correspondence, restrict recursive types in a way,
which is not an ideal design choice, if one's goal is a unified system for
logic and programming. Our approach explores another design space not yet
completely explored. We introduce both approaches to restricting recursive
types to ensure normalization in the following two subsections.


\subsection{Restriction on recursive types for normalization}
\label{sec:intro:concpets:recursive}
We have argued that normalization is essential for logical reasoning systems
based on the Curry--Howard correspondence. One challenge to the successful
design of such systems is how to restrict recursion at the type level
so that normalization of terms is preserved. 
There are two different
design choices illustrated in Figure~\ref{fig:approaches}. 
The conventional approach restricts the formation
of recursive types (\ie, the restriction is in datatype definition), and
the Mendler-style approach restricts the elimination
of the values of recursive types (\ie, the restriction is in pattern matching).

\begin{figure}
{\centering
\begin{tabular}{p{3cm}|p{12.5cm}}
\parbox{3cm}{~~Functional\\programming\\$~~~~$language} &
\parbox{12.5cm}{
 kinding:~
  \inference[($\mu$-form)]{\Gamma |- F : * -> *}{\Gamma |- \mu F : *} \\
 \\
 typing:\quad
  \inference[($\mu$-intro)]{\Gamma |- t : F (\mu F)}{\Gamma |- \In~t:\mu F} ~~~~
  \inference[($\mu$-elim)]{\Gamma |- t : \mu F}{\Gamma |- \unIn~t : F (\mu F)}\\
 \\
 reduction:
  \inference[(\unIn-\In)]{}{\unIn~(\In~t) \rightsquigarrow t}
} \\
\\ \hline\hline
\parbox{3cm}{$~$Conventional\\$~~~$approach for\\consistent logic} &
\parbox{12.5cm}{$\phantom{a}$\\
 kinding:~
  \inference[($\mu$-form$^{+}$)]{ \Gamma |- F : * -> * 
                           & \mathop{\mathsf{positive}}(F)}
                           {\Gamma |- \mu F : *} \\
 \\
 typing:~
  \text{{\small($\mu$-intro)} and {\small($\mu$-elim)}
                same as functional language} \\
  \[\inference[(\It)]{\Gamma |- t : \mu F & \Gamma |- \varphi : F A -> A}
                     {\Gamma |- \It~\varphi~t : A}\]
 reduction:~ \text{{\small(\unIn-\In)} same as functional language}
  \[\inference[(\It-\In)]{}{\It~\varphi~(\In~t) \rightsquigarrow
                            \varphi~(\textsf{map}_F~(\It~\varphi)~t)}\]
}
\\ \hline
\parbox{3cm}{Mendler-style\\$~~$approach for\\consistent logic} &
\parbox{12.5cm}{$\phantom{a}$\\
 kinding:~ \text{{\small($\mu$-form)} same as functional language} \\
 \\
 typing:~
  \text{{\small($\mu$-intro)} same as functional language}
  \[\inference[(\MIt)]
     { \Gamma |- t : \mu F &
       \Gamma |- \varphi : \forall X . (X -> A) -> F X -> A}
     {\Gamma |- \MIt~\varphi~t : A} \]
 reduction:~
  \inference[(\MIt-\In)]
     {}
     {\MIt~\varphi~(\In~t) \rightsquigarrow \varphi~(\MIt~\varphi)~t}
}
\end{tabular} }
\caption{Two different approaches to designing a logic
         (in contrast to functional languages)}
\label{fig:approaches}
\end{figure}

\paragraph{Recursive types in functional programming languages.}
Let us start with a review of the theory of recursive types used
in functional programming languages. Here, the term
language is not expected to be normalizing, so restrictions are few.

Just as we can capture the essence of unrestricted general recursion at term
level, by a fix point operator (usually denoted by \textsf{Y} or \textsf{fix}),
we can capture the essence of recursive types by the
use of fix point operator, $\mu$, at type level. 
The rules for the formation {\small($\mu$-form)},
introduction {\small($\mu$-intro)}, and elimination {\small($\mu$-elim)} of
the recursive type operator $\mu$ are described in Figure \ref{fig:approaches}.
We also need a reduction rule {\small(\unIn-\In)}, which relates \In,
the data constructor for recursive types, and \unIn, the destructor for
recursive types, at the term level.

Surprisingly (if you hadn't known), the recursive {\em type} operator, $\mu$,
as described in Figure \ref{fig:approaches}, is already powerful enough to
express non-terminating programs, even without introducing the general recursive
{\em term} operator, \textsf{fix}, to the language. We illustrate this below.
First a short reminder of how a fixpoint at the term level operates.
The typing rule and the reduction rule for \textsf{fix} can be given as follows:
\[ \text{typing:}~ \inference{\Gamma |- f : A -> A}{\textsf{fix}\,f : A}
 \qquad\qquad
   \text{reduction}:~ \textsf{fix}\,f \rightsquigarrow f(\textsf{fix}\,f)
\]
We can actually implement \textsf{fix}, using $\mu$, as follows
(using some Haskell-like syntax):
\begin{align*}
& \textbf{data}~T\;a\;r = C\;(r -> a) \quad
          \texttt{-}\texttt{-}~\text{\small a non-recursive datatype} \\
& w \,:\, \mu(T\;a) -> a ~~ \quad
          \texttt{-}\texttt{-}~\text{\small an encoding of the untyped
                                     $(\lambda x.x\;x)$
                                     in a typed language}~ \\
& w = \lambda x . \,\textbf{case}~\unIn~x~\textbf{of}~C\;f -> f\;x \\
& \textsf{fix} \,:\, (a -> a) -> a \quad
          \texttt{-}\texttt{-}~\text{\small an encoding of 
                                     $(\lambda f.(\lambda x.f(x\;x))\,
                                                 (\lambda x.f(x\;x)))$} \\
& \textsf{fix} = \lambda f. (\lambda x. f (w\;x))\,(\In(C(\lambda x. f (w\;x))))
\end{align*}

Thus, to avoid the loss of termination guarantees, we need to alter the rules
for $\mu$, in someways, to ensure a consistent logic. One way, is to restrict
the rule {\small $\mu$-form}; the other way, is to restrict the rule
{\small $\mu$-elim}. Once we decide which of these two alterations of the
rules we will use, the design of principled recursion combinators (\eg, \It\
for the former and \MIt\ for the latter) follows from that choice.

\paragraph{Recursive types in the conventional approach to consistent logic.}
In the conventional approach, the formation (\ie, datatype definition) of
recursive types is restricted, but arbitrary elimination (\ie, pattern matching)
over the values of recursive types is allowed. In particular, the formation of
negative recursive types is restricted. Only positive recursive types are
supported. Thus, in Figure \ref{fig:approaches}, we have a restricted version of
the formation rule {\small($\mu$-form$^{+}$)} has an additional condition that
require $F$ to be positive. The other rules {\small($\mu$-intro)},
{\small($\mu$-elim)}, and {\small(\unIn-\In)} remain the same as for
functional languages. Since we have restricted the recursive types
at the type level and we do not have general recursion at the term level,
the language is indeed normalizing. However, we can neither write
interesting (\ie, recursive) programs that involves recursive types nor
inductively reason about those programs, unless we have principled recursion
schemes that are guaranteed to normalize. One such recursion scheme is called
iteration (\aka\ catamorphism). The typing rules for the conventional iteration
\It\ are illustrated in Figure \ref{fig:approaches}. Note, we have the typing
rule {\small(\It)} and the reduction rule {\small(\It-\In)} for \It\,
in addition to the rules for the recursive type operator $\mu$.

\paragraph{Recursive types in the Mendler-style approach to consistent logic.}
In the Mendler-style approach, we allow arbitrary formation
(\ie, datatype definition) of recursive types, but we restrict
the elimination (\ie, pattern matching) over the values of recursive types. 
The formation rule {\small($\mu$-form)} remains the same as
for functional languages. That is, we can define arbitrary recursive types,
both positive and negative. However, we no longer have the elimination
rule {\small($\mu$-elim)}. That is, we are not allowed to pattern match over
the values of recursive types in the normal fashion. We can only pattern match
over the values of recursive types through the Mendler-style recursion
combinators. The rules for the Mendler-style iteration combinator \MIt\
are illustrated in Figure \ref{fig:approaches}.
Note, there are no rules for \unIn\ in the Mendler-style approach.
The typing rule {\small($\mu$-elim)} is replaced by {\small(\MIt)} and
the reduction rule {\small(\unIn-\In)} is replaced by {\small(\MIt-\In)}.
More precisely, the typing rule {\small \MIt} is both an elimination rule
for recursive types and a typing rule for the Mendler-style iterator.
You can think of the rule {\small(\MIt)} as replacing both the elimination rule
{\small($\mu$-elim)} and the typing rule for conventional iteration
{\small(\It)}, but in a safe way that guarantees normalization.

\subsection{Justification of the Mendler-style as a design choice.}
\label{sec:intro:concepts:mendler}
We choose to base our approach to the design of a seamless synthesis of both
logic and programming on the Mendler-style. It restricts the elimination (\ie,
pattern matching) over values of recursive types, rather restricting the
formation (\ie, datatype definition) of recursive types (a more conventional
approach). The impact of this design choice is that it enables the logic to
include all datatype definitions that are used in functional programming
languages.

Functional programming promotes ``functions as first class values''.
It is natural to pass functions as arguments and embed functions into
(recursive) datatypes. If embedding functions in datatypes is allowed,
we can embed a function whose domain is the very type we are defining.
For example, the recursive datatype definition
$\mathbf{data}~T = C\;(T -> \textit{A})$ in Haskell is such a recursive
datatype definition. Such datatypes are called negative recursive datatypes
since the recursive occurrence $T$ appears in a negative position.
We say that $T$ is in a negative position, since $(T -> A)$ is analogous to
$(\neg T \land A)$ when we think of $->$ as a logical implication. There exist
both interesting and useful examples in functional programming involving
negative datatypes. In \S\ref{sec:msf}, we illustrate that
the Mendler-style recursion scheme we discovered can express
interesting examples involving negative datatypes.

Recall that the motivation of this dissertation research
(quoting again from \S\ref{sec:intro:motiv})
is to contribute to answering the question of {\em how does one build a
seamless system where programmers can both write (functional) programs and
formally reason about those programs}. Under the Curry--Howard correspondence,
to formally reason about a program, the logic needs to refer to the type of
the program, since the type, interpreted as a proposition, describes its
properties. Since the Mendler-style approach does not restrict recursive
datatype definitions, we can directly refer to the types of programs that use
negative recursive types.

The Mendler style is a promising approach to building a unified system because
all the recursive types (both positive and negative) are definable and
the recursion schemes over those types are normalizing.
%% As we mentioned previously, the Mendler-style iteration
%% (\MIt) always terminate for both positive and negative recursive types.
%% There exist other families of Mendler-style recursion combinators,
%% which also guarantee for negative recursive types, and more useful
%% than \MIt\ over negative datatypes.
Although the conventional approach is widely followed
in the design of formal reasoning systems (\eg, Coq, Agda), it cannot directly
refer to programs that use non-positive recursive types.One may object that
it is possible to indirectly model negative recursive types
in the conventional style, via alternative equivalent encodings
which map negative recursive types into positive ones. But, such
encodings do not align with our motivation towards a seamless unified
system for both programming and reasoning. It is undesirable to require
programmers to significantly change their programs just to reason about them.
If the change is unavoidable, it should be kept small. That is,
the changed program should syntactically resemble the original program,
which the programmer would usually write in a functional programming language.
In Chapter 3, we show a number of examples of programs written in
the Mendler style that look more close to the programs written using
general recursion than the programs written in the conventional style.

%% Throughout this dissertation,
%% we show that the Mendler-style recursion schemes are
%% indeed useful and well-behaved induction principles.

\subsection{Term-indexed types, type inference, and datatypes}
\label{sec:intro:concepts:indexed}
One of the most frequently asked questions about our design choices for Nax,
regarding term-indexed types, is ``why not dependent types?". Our answer
is that a moderate extension to the polymorphic calculus is a better candidate
than a dependently typed calculus as the basis for a practical programming
system. Recall, that we hope to design a unified system for programming
as well as reasoning. Language designs based on indexed types can
benefit from existing compiler technology and type inference algorithms
for functional programming languages. In addition, theories for
term-indexd datatypes are simpler than theories for full-fledged
dependent datatypes, because term-indexd datatypes can be encoded as
functions (using Church-like encodings).

The implementation technology for functional programming languages based on
polymorphic calculi is quite mature. There exist industrial
strength implementations, such as the Glasgow Haskell Compiler (GHC),
whose intermediate core language is an extension of \Fw.
Our term-indexed calculi described in Part \ref{part:Calculi} are closely
related to \Fw\ by an index-erasure property. The hope is that
our implementation can benefit from these technologies.

Type inference algorithms for functional programming languages are often
based on certain restrictions of the Curry-style polymorphic lambda calculi.
These restrictions are designed to avoid higher-order unification during
type inference.
We develop a conservative extension of Hindley--Milner type inference for
Nax (Chapter \ref{ch:naxTyInfer}). This is possuble because Nax is based on our
term-indexed calculi (Part \ref{part:Calculi}). Dependently typed languages,
on the other hand, are often based on bidirectional type checking, which
requires annotations on top level definitions, rather than
Hindley--Milner-style type inference.

In dependent type theories, datatypes are usually supported as primitive
constructs with axioms, rather than as functional encodings
(\eg, Church encodings). One can give functional encodings for datatypes
in a dependent type theory, but one soon realizes that the induction principles
(or, dependent eliminators) for those datatypes cannot be derived within
the pure dependent calculi \cite{Geuvers01}.
So, dependently typed reasoning systems support datatypes as primitives.
For instance, Coq is based on Calculus of Inductive Constructions, which
extends Calculus of Constructions \cite{CoqHue86} with dependent datatypes
and their induction principles.

In contrast, in polymorphic type theories, all imaginable datatypes
within the calculi have functional encodings (\eg, Church encodings).
For instance, \Fw\ need not introduce datatypes as primitive constructs,
since \Fw\ can embed all imaginable datatypes, including non-regular
recursive datatypes with type indices. 

Another reason to use term-indexed calculi, rather than dependent type theories,
is to extend the application of Mendler-style recursion schemes,
which are well-understood in the context of \Fw.
Researchers have thought about (though not published)\footnote{
     Tarmo Uustalu described this on a whiteboard
     when we met with him at the University of Cambridge in 2011.
     We discuss this in Chapter \ref{ch:relwork}.}
Mendler-style primitive recursion over dependently-typed functions
over positive datatypes (\ie, datatypes that have a map), but not for
negative (or, mixed-variant) datatypes. In our term-indexed calculi,
we can embed Mendler-style recursion schemes (just as we embedded them in \Fw)
that are also well-defined for negative datatypes.

\section{Contributions}\label{sec:intro:contrib}
This dissertation makes contributions in several areas.
\begin{itemize}
\item[1.]
It organizes and expands the realm of \emph{Mendler-style recursion schemes}
(Part~\ref{part:Mendler}, \ie, Chapter \ref{ch:mendler})

\item[2.] It establishes a meta-theories for \emph{term-indexed types}
        (Part~\ref{part:Calculi}),

\item[3.] It designs a practical language (with an implementation)
        \emph{in the sweet spot} between programming and logical reasoning
        (Part~\ref{part:Nax}), and

\item[4.] It identifies several interesting open problems related to above.
\end{itemize}

\subsection{Contributions related to the Mendler style}
We organize a hierarchy of Mendler-style recursion schemes in two dimensions.
The first dimension is the abstract operations they support. For instance,
the Mendler-style iteration (\MIt) supports a single abstract operation
the recursive call. All the other Mendler-style recursion schemes
support the recursive call and an additional set of abstract operations. 
The second dimension is over the kind of the datatypes they operate over.
For example, \texttt{Nat} has kind $*$, while \texttt{Vec}
has kind $* -> \mathtt{Nat} -> *$. Each recursion scheme is actually a
family of recursion combinators sharing the same term definition
(\ie, uniformly defined) but with different type signatures at each kind.

We expand the realm of Mendler-style recursion schemes in several ways.
First, we report on a new recursion scheme $\MsfIt$, which is useful
for negative datatypes.  Second, we study the termination behaviors
of Mendler-style recursion schemes. Some recursion schemes (\eg, \MIt, \MsfIt)
always terminate for any recursive type, while others (\eg, \McvPr) only
terminate for certain classes of recursive types. Third, we extend
all Mendler-style recursion schemes to be expressive over term-indexed types.
The Mendler style has been studied in the context of \Fw\ (and several
extensions) which can express {\bf type}-indexed types. To extend Mendler-style
recursion schemes to be expressive over {\bf term}-indexed types, we report on
several theories for calculi (\Fi\ and \Fixi) that support term indices.
This is another important area of our contribution.

We provide examples that illustrate when each recursion scheme is useful
in Chapter \ref{ch:mendler}. The most interesting example among them is
the type-preserving evaluator for a simply-typed HOAS (\S\ref{sec:evalHOAS}),
which involves negative datatypes with indices.
This example is our novel discovery, which implies that
a type-preserving evaluator for a simply-typed HOAS
can be expressed within \Fw.

In addition, we develop a better understanding of some existing
Mendler-style recursion schemes. For instance, the existence of
Mendler-style course-of-values recursion (\McvPr) is reported
in the literature, but the calculus that can embed \McvPr\ was unknown.
We embed Mendler-style course-of-values recursion into \Fixi\ 
(or into \Fixw\ \cite{AbeMat04}, when we do not consider term-indices).

\subsection{Contributions to the theory of Term-Indexed Types}
Mendler-style recursion schemes have been studies in the context of
polymorphic lambda calculi. For instance, \citet{AbeMatUus03} embedded 
Mendler-style iteration (\MIt) into \Fw\ and \citet{AbeMat04} embedded
Mendler-style primitive recursion (\MPr) into \Fixw. These calculi
support type-indexed types.

To extend the realm of Mendler-style recursion schemes to include
term-indexed types, we extended \Fw\ and \Fixw\ to support term indices.
In Part \ref{part:Calculi}, we present our new calculi
\Fi\ (Chapter \ref{ch:fi}), which extends \Fw\ with term indices, and
\Fixi\ (Chapter \ref{ch:fixi}), which extends \Fixw\ with term indices.
These calculi have an erasure property that states that well-typed terms
in each calculus are also well typed terms (when erased) in the 
underlying calculus. For instance, any well typed term in \Fi\ is also
a well-typed term in \Fw, and there are no additional well-typed terms
in \Fi\ that are not well-typed in \Fw.

Our new calculi, \Fi\ and \Fixi, are strongly normalizing and
logically consistent. We show strong normalization and logical consistency
using the erasure properties. That is, strong normalization and
logical consistency of \Fi\ and \Fixi\ are inherited from \Fw\ and \Fixw.
Since \Fi\ and \Fixi\ are strong normalizing and logically  consistent,
the Mendler-style recursion schemes that can be embedded into these calculi
are adequate for logical reasoning as well as programming.

\subsection{Contributions in the design of the Nax language}
We design and implement a prototypical language Nax that explores
the sweet spot between programming oriented systems and logic oriented systems.
The language features supported by Nax provide the advantages
of both programming oriented systems and logic oriented systems.
Nax supports both term- and type-indexed datatypes,
rich families of Mendler-style recursion combinators,
and a conservative extension of Hindley--Milner type inference.
We designed Nax so that its foundational theory and
implementation framework could be kept simple.

Term- and type-indexed datatypes can express fine grained program properties
via the Curry--Howard correspondence, as in logic oriented systems. Although
not as flexible as full-fledged dependent types, indexed datatypes can
still express program invariants, such as type preserving compilation
(\S\ref{sec:example}), and size invariants on data structures.
Index types can simulate much of what
dependent types can do using singleton types. Since Nax has only erasable
indices, the foundational theory can be kept simple, and it supports
features that have the advantages of programming oriented systems 
(\eg, type inference, arbitrary recursive datatypes).

Adopting Mendler style provides merits of both programming oriented systems
and logic oriented systems. Since Mendler style is elimination based, one can
define all recursive datatypes usually supported in functional programming
languages. In addition, the programs written using Mendler-style recursion
combinators look more similar to the programs written using general recursion
than programs written in Squiggol style.
Since Nax supports only the well-behaved (\ie, strongly normalizing)
Mendler-style recursion combinators, it is safe to construct proofs using them.
In addition, Mendler-style recursion combinators are naturally well-defined
over indexed datatypes, which are essential to express fine-grained program
properties. Mendler style provides type based termination, that is, termination
is a by-product of type checking. Thus, it makes the implementation framework
simple since we do not need extra termination checking theories or algorithm.

Hindley--Milner-style type inference is familiar 
to functional programmers.
Nax can infer types for all programs that involve only regular datatypes,
which are already inferable in Hindley--Milner, without any type annotation.
Nax requires programs involving indexed datatypes to annotate their eliminators
by index transformers, which specify the relation between the input type index
and the result type. Eliminators of non-recursive datatypes are case expressions
and eliminators of recursive datatypes are Mendler-style recursion combinators.

\subsection{Contributions identifying open problems}
We identify several open problems alongside the contributions mentioned
in previews subsections. We will discuss the details of these open problems
in the future work chapter (Chapter \ref{ch:futwork}).
Here, we briefly introduce two of them.

\paragraph{Handling different interpretations of $\mu$ in one language system:}
Nax provides multiple recursion schemes (or, induction principles) used
to describe different kinds of recursive computations over recursive datatypes.
These recursion schemes are all motivated by concrete examples, which explains
the need for multiple schemes. It is more convenient to express various kinds of
recursive computations in Nax, by choosing a recursion scheme that fits
the structure of the computation, than in those systems that provide
only one induction scheme. However, there is theoretical difficulty
handling multiple interpretations of the recursive type operator $\mu$
in one language system.

Recall that we can embed datatypes as functional encodings in
our indexed type theory. Recursive datatypes and their recursion schemes in Nax
are embedded using Mendler-style encodings.
In Mendler style, one encodes the recursive type operator $\mu$
and its eliminator (the recursion scheme) as a pair.
So, there are several different encodings of $\mu$,
one for each recursion scheme. Some recursion schemes subsume others
(\ie, the more expressive one can simulate the other).

It would have been easy to describe the theory for Nax if we had
one most powerful recursion scheme that subsumes all the others,
which leads to a single interpretation of $\mu$. Unfortunately, we know of
no Mendler-style recursion scheme that subsumes all the other recursion schemes
in Nax. For instance, iteration (\MIt) can be subsumed by either 
iteration with a syntactic inverse (\MsfIt) or primitive recursion (\MPr).
But, there is no known recursion scheme that can subsume both \MsfIt\ and \MPr.

However, we strongly believe that it is okay to apply \MsfIt\ to
the result of \MPr\ (when \MPr\ produces a recursive value) and vice versa.
Intuitively, the different interpretations of $\mu$ only matter during
the internal computation of the recursion scheme. That is, one may consider
that (recursive) values resulting from different recursion schemes
share a common abstract representation of $\mu$.
The theoretical justification for this is still ongoing work.

\paragraph{Deriving positivity (or monotonicity) from polarized kinds:}
One can extend the kind syntax of arrow kinds in \Fw\ with polarities
($p\kappa_1 -> \kappa_2$ where the polarity $p$ is either $+$, $-$, or $0$)
to track whether a type constructor argument is used in
covariant (positive), contra-variant (negative), or
mixed-variant (both positive and negative) positions.
It is still an open problem whether it is possible to derive monotonicity
(\ie, the  existence of a map) for a type constructor from its polarized kind,
without examining the type constructor definition.

We identified a useful application for a solution to this open problem.
We discovered an embedding of Mendler-style course-of-values recursion in
a polarized system for positive (or monotone) type constructors.
That is, once you can show the existence of a map for a datatype,
course-of-values recursion always terminates.
However, in a practical language system, it is not desirable to burden users
with the manual derivation for every datatype on which they might want to
perform course-of-values recursion. If the type system can automatically
categorize datatypes that have maps from their polarized kinds,
this burden can be alleviated.


\section{Methodology and Overview}\label{sec:intro:overview}
This dissertation consist of five parts:
Part \ref{part:Prelude} (Prelude),
Part \ref{part:Mendler} (The Mendler style),
Part \ref{part:Calculi} (Term-indexed lambda calculi),
Part \ref{part:Nax} (The Nax language), and
Part \ref{part:Postlude} (Postlude).
The three parts in the middle, excluding the prelude and postlude parts,
describes the three steps of our approach. First, we experiment new ideas on
Mendler-style recursion schemes driven from concrete examples
using Haskell (with some GHC extensions), which is a functional language
based on an extension of \Fw\ (Part \ref{part:Mendler}). Second, we develop
theories (\ie, lambda calculi) for term-indexed datatypes to prove that
the Mendler-style recursion schemes are well-defined over indexed datatypes
and have the expected termination behavior. Lastly, we design a language system
with practical features that embodies our new ideas in and is based on the
theory we developed. Figure~\ref{fig:overview} summarizes the organization of
key concepts throughout the dissertation.

\begin{figure}
TODO \\

STLC\\
\F\
\Fw\    \Fi\ \\
\Fixw\  \Fixi\ \\
Hindley--Milner Nax type inference \\

TODO \\
make arrow diagrams
TODO \\
TODO \\
TODO \\
TODO \\
TODO \\
\caption{Summary of key concepts}
\label{fig:overview}
\end{figure}

\paragraph{Part \ref{part:Prelude} (Prelude)}\hspace{-1em} opens
the dissertation by giving an introduction (Chapter \ref{ch:intro}),
which you are currently reading, followed by
reviews on several well-known typed lambda calculi (Chapter \ref{ch:poly}).
In Chapter \ref{ch:poly}, we review
the simply-typed lambda calculus (STLC) (\S\ref{sec:stlc}),
System \F\ (\S\ref{sec:f}),
System \Fw\ (\S\ref{sec:fw}), and
the Hindley--Milner type system (\S\ref{sec:hm}).

From \S\ref{sec:stlc} to \ref{sec:fw}, we prove strong normalization using
saturated sets for each of the three calculi:
STLC (no polymorphism), System \F\ (polymorphism over types), and
System \Fw\ (polymorphism over type constructors).
The normalization proof on later sections extends upon
the normalization proof of the previous section,
as the calculus extends its feature of polymorphism.
We use the strong normalization of System \Fw\ to show that
our term-indexed lambda calculi in Part \ref{part:Calculi} are
strongly normalizing. Readers familiar with strong normalization proofs
on these calculi may skip or quickly skim over these sections.
It is worth noticing two stylistic choices in our formalization of
System \F\ and \Fw: (1) terms are in Curry style and
(2) typing contexts in System \F\ and \Fw\ are divided into two parts
    (one for type variables and the other for term variables).
This prepares readers for our formalization of the term-indexed calculi
in Part \ref{part:Calculi}, which have Curry-style terms and
typing contexts divided into two parts.

In \S\ref{sec:hm}, we review the type inference algorithm for
the Hindley--Milner type system (\S\ref{sec:hm}).
The Hindley--Milner type system (HM) is a restriction of System~\F,
which makes it possible to infer types without any type annotation on terms.
Later in Part~\ref{part:Nax} Chapter \ref{ch:naxTyInfer},
we formulate a conservative extension of HM, which restricts
the term-indexed calculus System \Fi\ (Chapter \ref{ch:fi}) in a similar manner.

\paragraph{Part \ref{part:Mendler} (the Mendler style)}\hspace{-1em} introduces
the concept of Mendler-style recursion schemes (Chapter \ref{ch:mendler})
using examples written in Haskell (with some GHC extensions). So, the readers
of Chapter \ref{ch:mendler} need no background knowledge on typed lambda calculi
but only some familiarity to functional programming. We explain the concepts of
a number of Mendler-style recursion schemes, their termination properties, and
how one recursion scheme is related to another, in an intuitive manner by using
examples written in Haskell. We also provide semi-formal proofs of termination
for some of the recursion schemes (\MIt\ and \MsfIt) by embedding the those
recursion schemes into \Fw\ fragment of Haskell. More formal and general
proof by embedding into our term-indexed lambda calculi comes later in
Part \ref{part:Calculi}.

The Mendler-style recursion schemes discussed in Chapter \ref{ch:mendler}
include iteration (\MIt), iteration with syntactic inverse (\MsfIt),
primitive recursion (\MPr), course-of-values iteration (\McvIt),
and course-of-values recursion (\McvPr). Among them, \MsfIt\ is a
Mendler-style recursion scheme we discovered.
There are even more Mendler-style recursion schemes, which are not
discussed in Chapter \ref{ch:mendler} -- we give pointers to them later in
the related work chapter (Chapter \ref{ch:relwork} in Part \ref{part:Postlude}).

\paragraph{Part \ref{part:Calculi} (term-indexed lambda calculi)}\hspace{-1em}
establishes theories for term-indexed types.
We formalized two term-indexed lambda calculi,
System \Fi\ (Chapter \ref{ch:fi}) and System \Fixi\ (Chapter \ref{ch:fixi}),
which are extensions of polymorphic calculi with term indices.
System \Fi\ extends System \Fw\ with term indices and
System \Fixi\ extends System \Fixw\ \cite{AbeMat04} with term indices.

We prove both strong normalization and logical consistency of
these term-indexed calculi using their index erasure property.
The index erasure property of a term-indexed calculus
projects a typing in the term-index calculi into
the polymorphic calculus it was extended from.
That is, all well-typed terms in \Fi\ and \Fixi\ are
also well-typed typed terms in \Fw\ and \Fixw.
That is, our term-indexed calculi, \Fi\ and \Fixi,
inherits strong normalization and logical consistency
from the polymorphic calculi, \Fw\ and \Fixw.

By embedding those recursion schemes into our term-indexed lambda calculi,
we prove that Mendler-style recursion schemes are well-defined and
terminates over term-indexed datatypes  For instance,
\MIt\ and \MsfIt\ can be embedded into System \Fi,
and, \MPr\ and \McvPr\ can be embedded into System \Fixi.

\paragraph{Part \ref{part:Nax} (the Nax language)}\hspace{-1em} consist of
three chapters.
First, we introduce the features of Nax (Chapter \ref{ch:naxFeatures})
in a tutorial format using small Nax code snippet examples.
Next, we discuss the design principles of the type system (Chapter \ref{ch:nax})
in comparison to two other systems: Haskell's datatype promotion and Agda.
We develop the discussion In Chapter \ref{ch:nax} along
a larger and more practical example Nax programs:
a type preserving interpreter and a stack safe compiler.
Lastly, we discuss type inference in Nax (Chapter \ref{ch:naxTyInfer}),
which is a conservative extension of Hindley--Milner type system (HM).
That is, any program, whose type can be inferred in HM, can also be
inferred its type in Nax without any annotation. Programs involving
term- or type-indexed datatypes, which are not supported in HM, needs
some annotation to infer their types in Nax. Annotations are only
required on three syntactic entities (datatype declarations, case expressions,
and Mendler-style recursion combinators) and nowhere else.

\paragraph{Part \ref{part:Postlude} (Postlude)}\hspace{-1em} closes
the dissertation by summarizing
  related work (Chapter~\ref{ch:relwork}),
  future work (Chapter~\ref{ch:futwork}), and
  conclusions (Chapter~\ref{ch:concl}).


\section{Mendler-style recursion schemes}
\label{sec:mendler}
In this section, we introduce basic concepts of two Mendler-style recursion
schemes: iteration (\lstinline{mcata}\,) and iteration with syntactic inverses
(\lstinline{msfcata}\,). Further details on Mendler-style recursion schemes,
including these two and more,
can be found in \cite{AhnShe11,AbeMatUus05,UusVen00,AbeMat04}.

\begin{figure}
\begin{lstlisting}[
	caption={Mendler-style iteration (\MIt{}) and
		Mendler-style iteration with syntactic inverses (\msfit{})
		at kind $*$ and $*\to *$ transcribed in Haskell
		\label{lst:reccomb}}]
data Mu0   (f::(* -> *))               = In0   (f (Mu0   f)  )
data Mu1 (f::(* -> *) -> (* -> *)) i = In1 (f (Mu1 f) i)

type Phi0   f a = forall r . (    r   -> a  )  -- %\textcomment{abstract recursive call%}
                    -> (f r   -> a  )
type Phi1 f a = forall r i.(forall i.r i -> a i)  -- %\textcomment{abstract recursive call%}
                    -> (f r i -> a i)

mcata0   :: Phi0   f a ->  Mu0  f    -> a
mcata1 :: Phi1 f a ->  Mu1 f i -> a i

mcata0   phi (In0   x) = phi (mcata0   phi) x
mcata1 phi (In1 x) = phi (mcata1 phi) x


data Rec0   f a   = Roll0   (f (Rec0 f a)    ) | Inverse0   a
data Rec1 f a i = Roll1 (f (Rec1 f a) i) | Inverse1 (a i)

type Phi0'   f a = forall r . (     a  -> r a  )  -- %\textcomment{abstract recursive call%}
                    -> (     r a  -> a  )  -- %\textcomment{abstract inverse%}
                    -> (f (r a)   -> a  )
type Phi1' f a = forall r i.(forall i.a i -> r a i)  -- %\textcomment{abstract recursive call%}
                    -> (forall i.r a i -> a i)  -- %\textcomment{abstract inverse%}
                    -> (f (r a) i -> a i)

msfcata0   :: Phi0'   f a -> (forall a. Rec0   f a  ) -> a
msfcata1 :: Phi1' f a -> (forall a. Rec1 f a i) -> a i

msfcata0  phi r = msfcata phi r where
  msfcata phi (Roll0   x)      = phi Inverse0  (msfcata phi)  x
  msfcata phi (Inverse0 z)    = z

msfcata1  phi r = msfcata phi r  where
  msfcata  :: Phi1' -> Rec1 f a i -> a i
  msfcata phi (Roll1 x)     = phi Inverse1 (msfcata phi)  x
  msfcata phi (Inverse1 z) = z
\end{lstlisting}
\vspace*{-3ex}
\end{figure}

Each Mendler-style recursion scheme is described by a pair:
a fixpoint and its constructors, and, the recursion scheme itself.
In Listing~\ref{lst:reccomb}, we illustrate the two recursion schemes,
\lstinline{mcata} and \lstinline{msfcata}, using Haskell. We follow
a certain style of Haskell, where we restrict the use of certain langauge
features and some of the definitions we introduce. We will explain such
stylistic restrictions as we discuss the details of Listing~\ref{lst:reccomb}.

A Mendler-style recursion scheme is characterized by
the set of abstract operations it supports. The types of
these abstract operations are evident in the type sigature
of the recursion scheme. In Listing~\ref{lst:reccomb},
we made this more evident by factoring out the first agument types
as the type synonyms prefixed by \lstinline{Phi}.
Mendler-style recursion schemes take two arguments:
the first is a function\footnote{By convention,
	we denote the function as $\varphi$. Guess why
	the type synonyms are prefixed by \lstinline{Phi}.}
that describes the computation, and,
the second is a recursive value to compute over.
One describes the computation by defining the function, which is to be passed
as the first argument, in terms of the abstract operations supported by
the recursion scheme.

\subsection{Mendler-style iteration}
Mendler-style iteration (\MIt{}) operates on recursive types constructed by
the fixpoint $\mu$. The fixpoint $\mu$ is indexed by a kind. We describe
$\mu$ at kind $*$ and $*\to*$ using Haskell in Listing~\ref{lst:reccomb}.
There are two stylistic restrictions on the Haskell code
following Mendler style:
\begin{itemize}
\item Recursion is allowed only by the fixpoint at type-level and
	by the recursion scheme at term-level.
\lstinline{Mu0} expects a non-recursive base structure \lstinline{f : * -> *}
to construct a recursive type \lstinline{(Mu0 f : *)}. \lstinline{Mu1} expects
a non-recursive base structure \lstinline{f : (* -> *) -> (* -> *)}
to construct a recursive type constructor \lstinline{(Mu1 f  : * -> *)},
which expects one type index (\lstinline{i :: *}). We do not rely on
recursive datatype definition natively supported by Haskell, except for
defining $\mu$. We do not rely on function definitions using general recursion
either, except for defining \msfit{}.

\item Elimination of recursive values is only allowed via the recursion scheme.
One is allowed to freely introduce recursive values using \In{}-constructors,
but not allowed to freely eliminate (\ie, pattern match against \In{})
those recursive values. One can only eliminate recursive values via
the recursion scheme. Note that \lstinline{mcata0} and \lstinline{mcata1}
are defined using pattern matching against \In{*} and \In{*\to*}.
Pattern matching against them elsewhere is prohibited.
\end{itemize}

The type synonyms \lstinline{Phi0} and \lstinline{Phi1} stands for
the first argument types of \lstinline{mcata0} and \lstinline{mcata1}.
These type synonyms indicate that Mendler-style iteration supports
one abstract operation: abstract recursive call.
The type (constructor) variable $r$ stands for abstract recursive values.
Since $r$ universally quantified within \lstinline{Phi0} and \lstinline{Phi1},
functions of either \lstinline{Phi0}-type or \lstinline{Phi1}-type must be
parametric over $r$ (\ie, must not rely on examining specifics of $r$-values).
In \lstinline{Phi0}, \lstinline{(r -> a)} is the type for
an abstract recursive call, which computes an answer of type \lstinline{a}
from the abstract recursive type \lstinline{r}. One uses an handle to
this abstract recursive call when implementing a function of type
\lstinline{f r -> a}, which computes en answer (\lstinline{a})) from
\lstinline{f}-structures filled with abstract recursive values (\lstinline{r}).
Similarly, \lstinline{(forall i.r i -> a i)} in \lstinline{Phi1} is the type
for an abstract recursive call, which is an index preserving function that
computes an indexed answer \lstinline{(a i)} from an indexed recursive value
\lstinline{(r i)}. In the Haskell definitions of \lstinline{mcata0} and
\lstinline{mcata1}, these abstract operations are implemented by Haskell's
natively supported recursive function call. Note that the first arguments to
\lstinline{phi} in the definitions of \lstinline{mcata0} and \lstinline{mcata1}
are \lstinline{(mcata0 phi)} and \lstinline{(mcata1 phi)}.

Uses of Mendler-style recursion schemes are best explained by examples.
Listing~\ref{lst:Len} is a well-known example of a list length function
defined in terms of \lstinline{mcata0}. The recursive type for lists
\lstinline{(List a)} is defined as a fixpoint of \lstinline{(L a)},
where \lstinline{L} is the base structure for lists. The data constructors
of \lstinline{List}, \lstinline{nil} and \lstinline{cons}, are defined
in terms of \lstinline{In0} and the data constructors of \lstinline{L}.
We define \lstinline{length} by applying \lstinline{mcata0} to
the \lstinline{phi} function, The function \lstinline{phi} is defined
by two equations, one for the \lstinline{N}-case and the other for
the \lstinline{C}-case. When the list is empty (\lstinline{N}-case),
the \lstinline{phi} function simply returns 0. When the list has an
element (\lstinline{N}-case), we first compute the length of the tail
(\ie, the list excluding the head, that is, the first element) by
applying the abstract recursive call \lstinline{(len :: r -> Int)}\footnote{
	Here, the answer type is \lstinline{Int}. }
to the (abstract) tail \lstinline{(xs :: r)},\footnote{
	Note that \lstinline{C x xs :: L p r} since \lstinline{xs :: r}.}
and, then, we add 1 to the length of the tail \lstinline{(len xs)}.

\begin{figure}
\lstinputlisting[
	caption={List length example using \lstinline{mcata0} \label{lst:Len}},
	firstline=3]{Len.hs}
\vspace*{-3ex}
\end{figure}


\subsection{Mendler-style iteration with syntactic inverses}
Mendler-style iteration with syntactic inverses (\msfit{}) operates on
recursive types constructed by the fixpoint $\mu'$. The fixpoint $\mu'$
is a non-standard fixpoint additionally parameterized by the answer type ($a$)
and has two constructors $\In{}'$ and \textit{Inverse}. $\In{}'$-constructors
are analogous to \In{}-constructors of $\mu$. \textit{Inverse}-constructors
hold answers to be computed by \msfit{}. For example,\footnote{
	In fact, this example is ill typed, because \msfit{} expects
	its second argument type to be parametric over
	(\ie, does not rely on specifics of) the answer type.
	This example is just to illustrate the intuitive idea.}
the result of computing \lstinline{ msfcata0  phi (Inverse0 5) }
is \lstinline{5} regardless of \lstinline{phi}.
The stylistic restrictions on the Haskell code involving \msfit{} are:
\begin{itemize}
\item Recursion is only allowed by the fixpoint at type-level ($\mu'$)
	and by the recursion scheme (\msfit{}) at term-level.
We do not rely on recursive datatype definitions and function definitions
general recursion
natively supported in Haskell.
\item Elimination of recursive values are only allowed via the recursion scheme.
One is allowed to freely construct recursive values using $\In{}'$-constructors,
but not allowed to freely eliminate (\ie, pattern match against $\In{}'$) them.
Pattern matching against \textit{Inverse} is also restricted.
\end{itemize}
These restrictions are similar to the stylistic restrictions involving \MIt{}.

The abstract operations supported by \msfit{} are evident
in the first argument type -- \lstinline{Phi0'} and \lstinline{Phi1'}
are the type synonyms for the first argument types of \lstinline{msfcata0}
and \lstinline{msfcata1}. Note that the abstract recursive type $r$ is also
additionally parameterized by the answer type $a$ in the type sigatures
of \lstinline{msfcata0} and \lstinline{msfcata1}, since $\mu'$ is additionaly
parameterized by $a$. In addition to the abstract recursive call, \msfit{}
also supports the abstract inverse operation. Note that the types for
abstract inverse (\lstinline{(a -> r a)} and \lstinline{(a i -> r a i)})
are indeed the types for inverse functions of abstract recursive call
(\lstinline{(r a -> a)} and \lstinline{(r a i -> a i)}). Instead of using
actual inverse functions to compute inverse images from answer values
during computation, one can hold intermediate answer values, whose inverse
images are irrelevant, inside \textit{Inverse}-constructors during
the computation using \msfit{}.

The type signature of \msfit{} expects the second argument to be
parametric over the answer type. Note the second argument types
\lstinline{(forall a. Rec0 f a)} and \lstinline{(forall a. Rec1 f a i)}
in the type signatures of \lstinline{msfcata0} and \lstinline{msfcata1}. 
Using \textit{Inverse} to construct recursive values elsewhere is, in a way,
prohibited due to the second argument type of \msfit{}. Using \textit{Inverse}
to construct concrete recursive values makes the answer type specific.
For example, \lstinline{(Inverse0 5) :: Rec0 f Int}, whose answer type
made specific to \lstinline{Int}, cannot be passed to \lstinline{msfcata0}
its second argument. The constructor \textit{Inverse} is only intended to
define \msfit{} and its first argument (\lstinline{phi}). One can indirectly
access \textit{Inverse} via the abstract inverse operation supported by
\msfit{}. Note, in the Haskell definitions of \lstinline{msfcata0} and
\lstinline{msfcata1}, the second arguments to \lstinline{phi} are
\lstinline{Inverse0} and \lstinline{Inverse0}. That is, the abstract inverse
operation is implemented by the \textit{Inverse}-constructor.

We adopt the HOAS formatting
\begin{figure}
\lstinputlisting[
	caption={Formatting an untyped HOAS expression into a \lstinline{String}
		\label{lst:HOASshow} (adopted from \cite{AhnShe11}) },
	firstline=5]{HOASshow.hs}
\vspace*{-3ex}
\end{figure}
TODO Write text about the showExp example TODO

%% Using general recursion, one would have defined
%% the datatype |Exp_g :: * -> *| that corresponds to |Exp|
%% as follows, using Haskell's native recursive datatype definition.
%% \begin{code}
%% data Exp_g t where
%%   Lam_g :: (Exp_g a -> Exp_g b) -> Exp_g (a -> b)
%%   App_g :: Exp_g (a -> b) -> Exp_g a -> Exp_g b
%% \end{code}



\section{Type-preserving evaluation of the simply-typed HOAS}
\label{sec:HOASeval}
We can write an evaluator for a simply-typed HOAS in a simple manner,
as illustrated in Listing\;\ref{lst:HOASeval}.
We first define the simply-typed HOAS as a recursive indexed datatype
\lstinline{Exp  :: * -> *}. We take the fixpoint using \lstinline{Rec1}
(the fixpoint with a syntactic inverse). This fixpoint is taken over
a non recursive base structure \lstinline{ExpF  :: (* -> *) -> (* -> *)}.
Note that \lstinline{ExpF} is an indexed type. So expressions will be indexed
by their type. Using \lstinline{Rec1} the fixpoint of any structure is also
parameterized by the type of the answer. The use of the \lstinline{msfcata1}
requires that \lstinline{Exp} should be parametric in this answer type
(by defining \lstinline{type Exp  t =  forall a. Exp' a}). 
%% just as we did
%% in the untyped HOAS formatting example in Listing\;\ref{lst:HOASshow}.

\begin{figure}
\lstinputlisting[
	caption={Simply-typed HOAS evaluation using \lstinline{msfcata1}
		\label{lst:HOASeval}},
        firstline=4]{HOASeval.hs}
\vspace*{-3ex}
\end{figure}

The definition of \lstinline{eval} specifies how to evaluate
an HOAS expression to a host-language value (\ie, Haskell) wrapped by
the identity type (\lstinline{Id}\,). In the description below, we ignore
the wrapping (\lstinline{MkId}\,) and unwrapping (\lstinline{unId}\,) of
\lstinline{Id} by completely dropping them from the description.
See the Listing\;\ref{lst:HOASeval} (where they are not omitted)
if you care about these details. We discuss the evaluation for each of
the constructors of \lstinline{Exp}:
\begin{itemize}
	\item Evaluating an HOAS abstraction (\lstinline{Lam f}\,) lifts
		an object-language function \lstinline{(f)} over \lstinline{Exp}
		into a host-language function over values:
		\lstinline{(\v -> ev (f(inv v)))}.
		In the body of this host-language lambda abstraction,
		the inverse of the (host-language) argument value \lstinline{v}
		is passed to the object-language function \lstinline{f}.
		The resulting HOAS expression \lstinline{(f(inv v))} is
		evaluated by the recursive caller (\lstinline{ev}) to
		obtain a host-language value.

	\item Evaluating an HOAS application \lstinline{(App f x)} lifts
		the function \lstinline{f} and argument \lstinline{x} to
		host-language values \lstinline{(ev f}) and \lstinline{(ev x)},
		and uses host-language application to compute
		the resulting value. Note that the host-language application
		\lstinline{((ev f) (ev x))} is type correct since
		\lstinline{ev f :: a -> b}\, and \lstinline{ev x :: a},
		thus the resulting value has type \lstinline{b}.
\end{itemize}
We know that \lstinline{eval} indeed terminates since \lstinline{Rec1} and
\lstinline{msfcata1} can be embedded into System~\Fw\ in manner similar to
the embedding of \lstinline{Rec0} and \lstinline{msfcata0} into System~\Fw.

Listing\;\ref{lst:HOASeval} highlights two advantages of the Mendler style over
conventional style in one example. This example shows that the Mendler-style
iteration with syntactic inverses is useful for both \textit{negative} and
\textit{indexed} datatypes. \lstinline{Exp} in Listing\;\ref{lst:HOASeval} has
both negative recursive occurrences and type indices.

The \lstinline{showExp} example in Listing\;\ref{lst:HOASshow},
which we discussed in the previous section, has appeared in the work
of Fegaras and Sheard \cite{FegShe96}, written in conventional style.
So, the \lstinline{showExp} example, only shows that the Mendler style is
as expressive as the conventional style (although it is
perhaps syntactically more pleasant than the conventional style).
However, it is not obvious how one could extend the conventional-style
counterpart over indexed datatypes.

In contrast, \lstinline{msfcata} is uniformly defined over indexed datatypes of
arbitrary kinds. Both \lstinline{msfcata1}, used in the \lstinline{eval},
and \lstinline{msfcata0}, used in the \lstinline{showExp}, have exactly
the same syntatctic definition, differing only in their type signatures,
as lillustrated in Listing\;\ref{lst:reccomb}.



\section{$\mu'$ is a subtype of $\mu$}\label{sec:murec}
We discussed the usefulness of \msfit{} by the illustrating examples on HOAS.
If one is to design a language based on Mendler-style recursion schemes,
one would want to support as many useful recursion schemes avilaible,
including \MIt{} and \msfit{}. One issue in such design is that we have
two different fixpoints $\mu$ and $\mu'$. The standard fixpoint $\mu$
does not come with syntactic inverses while $\mu'$ comes with
its syntactic inverse. It would be a bad design choice to have those
two fixpoints unrelated in the language semantics and let users deal
with them manually. We would like to apply as many recursion schemes to
one recursive value without having to manually convert from one another.

We discovered that $\mu'$ is a subtype of $\mu$. Listing\;\ref{lst:rec2mu}
illustrates a mapping from \lstinline{(forall a.Rec0 ExpF  a)}
to \lstinline{(Mu0 ExpF)} implemented using \lstinline{msfcata0},
where \lstinline{ExpF} is a base structure for the untyped HOAS.
Since we have two fixpoints, \lstinline{Rec0} and \lstinline{Mu0},
we can define two recursive datatypes from the base structure \lstinline{ExpF}.
One is \lstinline{Exp} defined as \lstinline{(forall a.Rec0 ExpF  a)} and
the other is \lstinline{Expr} defined as \lstinline{Mu0 ExpF}.
The function \lstinline{exp2expr  :: Exp -> Expr}\, implements the mapping from
\lstinline{Rec0}-based HOAS expressions to \lstinline{Mu0}-based
HOAS expressions. Note, \lstinline{exp2expr}\, is defined
using \lstinline{msfcata0}.  Since there exists an embedding of
\lstinline{Mu0} and \lstinline{msfcata0} into System~\Fw\ \cite{AhnShe11},
\lstinline{exp2expr}\, is admissible in System~\Fw. However, it is not likley
that we can embed a coercion function for an arbitrary base structure $f$,
\lstinline{mu2rec :: (forall a.Rec0 f a) -> Mu0 f}, in System~\Fw.\footnote{
	The discussions in \S\ref{sec:theory} on the embedding of \msfit{}
	suggests why the \lstinline{mu2rec} is not likely to be embedded
	in System~\Fw, but its specific instances, such as
	\lstinline{exp2expr}, can be embedded in System~\Fw.}
We conjecture that it should be possible to derive a coercion function
from $\mu'$-values to $\mu$-values when given a specific instance of
the base structure. Therefore, when designing a language based on Mendler-style
recursion schemes, we may support coercion from $\mu'$-values to $\mu$-values.

\begin{figure}
\lstinputlisting[
	caption={Coercoin from $\mu'$-values to $\mu$-values
		using \lstinline{msfcata0} \label{lst:rec2mu} },
	firstline=4]{Exp2Expr.hs}
\vspace*{-3ex}
\end{figure}

\begin{figure}
\begin{lstlisting}[
	caption={An incomplete attempt to convert from  from $\mu$-values
		to $\mu'$-values \label{lst:mu2rec} } ]
msfcata'  :: Phi0' f a -> Rec0 f a -> a
msfcata' phi (Roll0 x)     = phi Inverse0 (msfcata' phi)  x
msfcata' phi (Inverse0 z) = z

exp'2expr :: Exp' Expr -> Expr  -- %\textcomment{i.e, %} Rec0 E (Mu0 E) -> Mu0 E
exp'2expr = msfcata' phi where
  phi inv p2r (Lam f)    = In0(Lam((\x -> p2r (f (inv x)))))
  phi inv p2r (App e1 e2)  = In0(App (p2r e1) (p2r e2))

expr2exp' :: Expr -> Exp' Expr  -- %\textcomment{i.e., %} Mu0 E -> Rec0 E (Mu0 E)
expr2exp' (In0(Lam f))   = Roll0 (Lam (\x->expr2exp' (f (exp'2expr x))))
expr2exp' (In0(App e1 e2)) = Roll0 (App (expr2exp' e1) (expr2exp' e2))
\end{lstlisting}
\vspace*{-3ex}
\end{figure}

Coercion the other way around, from $\mu$-values to $\mu'$-values,
is not likely to be possible in general, but might be possible
only when the answer type of the $\mu'$-values (\eg, \lstinline{a}
in \lstinline{Rec0 ExpF  a}) has been monomorphically instantiated
to the final answer value. That is, we attempted to convert from
\lstinline{Exp' Expr} to \lstinline{Expr}, rather than from \lstinline{Exp}
(\ie, \lstinline{forall a.Exp' a}) to \lstinline{Expr}.\footnote{
	Also note that \lstinline{a} in \lstinline{(Rec0 ExpF  a)}
	in the type signature of $\msfit{}'$
	is not quantified. {c.f.} (\lstinline{(forall a.Rec0 f a)}
	in the type signature of $\msfit{*}$.}
We illustrate this idea in Listing~\ref{lst:mu2rec}, which is still
an incomplete attempt since there is no termination guarantee for
\lstinline{expr2exp'}. Note that \lstinline{expr2exp'} is not defined
using a Mendler-style recursion scheme but using general recursion.

Let us contemplate on why the coercion from \lstinline{(forall a.Rec0 ExpF  a)}
to \lstinline{(Mu0 ExpF)} exists, but the coercion the other way is hard
(perhaps impossible) to find. We believe that \lstinline{msfcata0} can
express more functions than \lstinline{mcata0} (e.g., \lstinline{showHOAS}
in Listing\;\ref{lst:HOASshow}). Then, it may be the case that values of
\lstinline{(forall a.Rec0 f a)} is in fact more restrictive than
the values of \lstinline{(Mu0 f)}. The additional expressiveness of
\lstinline{msfcata0} may be a compensation for the restrictions on
the value of \lstinline{(forall a.Rec0 f a)}. In summary, we believe
that \lstinline{(forall a.Rec0 f a)} is a subset of \lstinline{(Mu0 f)}.


\section{Metatheory}
\label{sec:theory}

The expectation is that System \Fi\ has all the nice properties of System \Fw, yet
is more expressive because of the addition of term indexed types.

Before we prove subjected reduction and strong normalization for \Fi, we need to
check some basic well-formedness properties for the derivations of \Fi.
We want to show that the sorting, kinding, and typing derivations give
well-formed results under well-formed contexts. That is, sorting derivations
result in well-formed sorts (Proposition \ref{prop:wfsort}),
kinding derivations result in well-sorted kinds under well-formed
type level contexts (Proposition \ref{prop:wfkind}), and typing derivations
result in well-kinded types under well-formed type and term level contexts
(Proposition \ref{prop:wftype}).

Since the definitions of sorting, kinding, and typing rules are
mutually recursive, these three properties are considered as
one big property (illustrated below) in order to be more rigorous abouts
the induction principle used in the proof.
\begin{proposition*}[The big well-formedness property of \Fi,
		roughly\footnote{Technically,
    this is not yet completely rigorous since there are three more forms of
    judgments in the mutually recursive definition. The \emph{kind equality},
    \emph{type considered equality}, and \emph{term equality} rules are part of
    the mutually recursive definition along with the sorting, kinding, and
    typing rules. So, the complete description of the big well-formedness
    property will consist of six cases, which correspond to
    Proposition \ref{prop:wfsort}, Proposition \ref{prop:wfkind},
    Proposition \ref{prop:wftype}, Lemma \ref{lem:wfeqkind},
    Lemma \ref{lem:wfeqtype}, and Lemma \ref{lem:wfeqterm}.  }  ]~
\begin{quote}
\begin{itemize}
\item[case] \fbox{$ |- \kappa : \square $}\qquad\quad
 $ \inference{ |- \kappa : \mathfrak{s} }{ \mathfrak{s}=\square } $\\
 \qquad\qquad (Proposition \ref{prop:wfsort})
\item[case] \fbox{$ \Delta |- F : \kappa $}\qquad
 $ \inference{ |- \Delta & \Delta |- F : \kappa}{ |- \kappa:\square } $\\
 \qquad\qquad (Proposition \ref{prop:wfkind})
\item[case] \fbox{$ \Delta;\Gamma |- t:A $}\quad
 $ \inference{ \Delta |- \Gamma & \Delta;\Gamma |- t : A}{ \Delta |- A : * } $\\
 (Proposition \ref{prop:wftype})
\end{itemize}
\end{quote}
\end{proposition*}\noindent
The big well-formedness property has one of the three forms --
\fbox{$ |- \kappa : \square $} (sorting),
\fbox{$ \Delta |- F : \kappa $} (kinding), and
\fbox{$ \Delta;\Gamma |- t:A $} typing.
That is, a derivation for a judgment of either sorting, kinding, or typing
results in either a well-formed sort (when it is a sorting judgment),
a well-sorted kind (when it is a kinding judgment), or
a well-kinded type (when it is a typing judgment),
under well-formed contexts for the judgment (no context for sorting judgments,
$\Delta$ for kinding judgments, and $\Delta;\Gamma$ for typing judgments).

We can prove the big well-formedness property of \Fi\ by induction on
the derivation of a judgment, which can be any one of the three forms.
Here, we illustrate the proof for the three propositions as if they were
separate proofs. Because it provides a more intuitive proof sketch, during 
the proof description, the proof for each proposition
references the other properties (which are yet another application of the
induction hypothesis of the big well-formedness property).  So, when we say ``by induction''
during the proofs, what we really mean is the induction hypothesis of
the big well-formedness property.

\begin{proposition}[sorting derivations result in well-formed sorts]
\label{prop:wfsort}
\[ \inference{ |- \kappa : \mathfrak{s} }{ \mathfrak{s}=\square } \]
\end{proposition}
\begin{proof}Obvious since $\square$ is the only sort in \Fi.\end{proof}

\begin{proposition}[kinding derivations under well-formed contexts
	       	result in well-sorted kinds]
\label{prop:wfkind}
\[ \inference{ |- \Delta & \Delta |- F : \kappa}{ |- \kappa:\square }
\]
\end{proposition}
\begin{proof} By induction on the derivation.
\begin{itemize}
\item[case] ($Var$)
	Trivial by the second well-formedness rule of $\Delta$.
\item[case] ($Conv$)
	By induction and Lemma \ref{lem:wfeqkind}.
\item[case] ($\lambda$) \\
	By induction and Proposition \ref{prop:wfsort} we know
	that $|- \kappa:\square$.\\
	By the second well-formedness rule of $\Delta$,
	we know that $|- \Delta,X^\kappa$ since we already know
	that $|- \kappa:\square$ and $|- \Delta$ from the property statement.\\
	By induction, we know that $|- \kappa':\square$
	since we already know that $|- \Delta,X^\kappa$ and
	that $\Delta,X^\kappa|- F:\kappa'$ from induction hypothesis.\\
	By the sorting rule ($R$), we know that $|- \kappa -> \kappa':\square$
	since we already know that $|- \kappa:\square$ and $|- \kappa':\square$.
\item[case] ($@$)
	By induction, easy.
\item[case] ($\lambda i$)\\
	By induction and Proposition \ref{prop:wftype} we know
	that $\cdot|- A:*$.
	By the third well-formedness rule of $\Delta$,
	we know that $|- \Delta,i^A$ since we already know that $\cdot|- A:*$ and
	that $|- \Delta$ from the property statement.\\
	By induction, we know that $|- \kappa:\square$
	since we already know that $|- \Delta,i^A$ and
	that $\Delta,i^A|- F:\kappa$ from the induction hypothesis.\\
	By the sorting rule ($Ri$), we know that $|- A -> \kappa:\square$
	since we already know that $\cdot |- A:*$ and $|- \kappa:\square$.
\item[case] ($@i$)
	By induction and Proposition \ref{prop:wftype}, easy.
\item[case] ($->$)
	Trivial since $|- * : \square$.
\item[case] ($\forall$)
	Trivial since $|- * : \square$.
\item[case] ($\forall i$)
	Trivial since $|- * : \square$.
\end{itemize}
\end{proof}

The basic structure of the proof for the following proposition on typing
derivations is similar to above. So, we illustrate the proof for most of
the cases, which can be done by applying the induction hypothesis, rather
bravely. We elaborate more on interesting cases ($\forall E$) and ($\forall Ei$)
which involve substitutions in the types resulting from the typing judgments.
\begin{proposition}[typing derivations under well-formed contexts result in
	well-kinded types]
\label{prop:wftype}
\[ \inference{ \Delta |- \Gamma & \Delta;\Gamma |- t : A}{ \Delta |- A : * }
\]
\end{proposition}
\begin{proof} By induction on the derivation.
\begin{itemize}
\item[case] ($:$)
	Trivial by the second well-formedness rule of $\Gamma$.
\item[case] ($:i$)
	Trivial by the third the well-formedness rule of $\Delta$.
\item[case] ($=$)
	By induction and Lemma \ref{lem:wfeqtype}.
\item[case] ($->$$I$)
	By induction and well-formedness of $\Gamma$.
\item[case] ($->$$E$)
	By induction.
\item[case] ($\forall I$)
	By induction and well-formedness of $\Delta$.
\item[case] ($\forall E$)\\
	By induction we know that $\Delta |- \forall X^\kappa.B : *$.\\
	By the kinding rule ($\forall$), which is the only kinding rule
	able to derive $\Delta |- \forall X^\kappa.B : *$, we know
	that $\Delta,X^\kappa |- B : *$.\\
	Then, we use the type substitution lemma (Lemma \ref{lem:tysubst}).
\item[case] ($\forall Ii$)
	By induction and well-formedness of $\Delta$.
\item[case] ($\forall Ei$)\\
	By induction we know that $\Delta |- \forall i^A.B : *$.\\
	By the kinding rule ($\forall i$), which is the only kinding rule
	able to derive $\Delta |- \forall i^A.B : *$, we know
	that $\Delta,i^A |- B : *$.\\
	Then, we use the index substitution lemma (Lemma \ref{lem:ixsubst}).
\end{itemize}
\end{proof}

\begin{lemma}[kind equality is well-sorted]\label{lem:wfeqkind}
$ \inference{|- \kappa = \kappa':\square}
	{|- \kappa:\square \quad |- \kappa':\square} $
\end{lemma}
\begin{proof}
	By induction on the derivation of kind equality
	and using the sorting rules.
\end{proof}

\begin{lemma}[type constructor equality is well-kinded]\label{lem:wfeqtype}
\[ \inference{\Delta |- F = F':\kappa}
	{\Delta |- F:\kappa \quad \Delta |- F':\kappa}
\]
\end{lemma}
\begin{proof}
	By induction on the derivation of type constructor equality
	and using the kinding rules.

	Also use the type substitution lemma (Lemma \ref{lem:tysubst})
	and the index substitution lemma (Lemma \ref{lem:ixsubst}).
\end{proof}

\begin{lemma}[term equality is well-typed]\label{lem:wfeqterm}
\[ \inference{\Delta,\Gamma |- t = t':A}
	{\Delta,\Gamma |- t:A \quad \Delta,\Gamma |- t':A}
\]
\end{lemma}
\begin{proof}
	By induction on the derivation of term equality
	and using the typing rules.

	Also use the term substitution lemma (Lemma \ref{lem:tmsubst}).
\end{proof}

The proofs for the three lemmas above are straightforward
once we have dealt with the interesting cases for the equality rules
involving substitution. We can prove those interesting cases
by applying the substitution lemmas. The other cases fall into two
categories: firstly, the equality rules following the same structure of
the sorting, kinding, and typing rules; and secondly, the reflexive
rules and the transitive rules. The proof for the equality rules
following the same structure of the sorting, kinding, and typing rules
can be proved by induction and applying the corresponding
sorting, kinding, and typing rules. The proof for the reflexive rules
and the transitive rules can be proved simply by induction.

\begin{lemma}[type substitution]\label{lem:tysubst}
$ \inference{\Delta,X^\kappa |- F:\kappa' & \Delta |- G:\kappa}
	{\Delta |- F[G/X]:\kappa'} $
\end{lemma}

\begin{lemma}[index substitution]\label{lem:ixsubst}
$ \inference{\Delta,i^A |- F:\kappa & \Delta;\cdot |- s:A}
	{\Delta |- F[s/i]:\kappa} $
\end{lemma}

\begin{lemma}[term substitution]\label{lem:tmsubst}
$ \inference{\Delta;\Gamma,x:A |- t:B & \Delta;\cdot |- s:A}
	{\Delta,\Gamma |- t[s/x]:B} $
\end{lemma}

\subsection{Erasure}
\paragraph{}
\begin{definition}[index erasure]\label{def:ierase}
\[ \fbox{$\kappa^\circ$}
 ~~~~ ~~
 *^\circ =
 *
 ~~~~ ~~
 (\kappa -> \kappa')^\circ =
 \kappa^\circ -> \kappa^\circ
 ~~~~ ~~
 (A -> \kappa)^\circ =
 \kappa^\circ
\]
\[ \fbox{$F^\circ$}
 ~~~~
 X^\circ =
 X
 ~~~~ ~~~~
 (A -> B)^\circ =
 A^\circ -> B^\circ
\]
\[ \qquad
 (\lambda X^\kappa.F)^\circ =
 \lambda X^{\kappa^\circ}.F^\circ
 ~~~~ ~~~~
 (\lambda i^A.F)^\circ =
 F^\circ
\]
\[ \qquad
 (F\;G)^\circ =
 F^\circ\;G^\circ
 ~~~~ ~~~~ ~~~~ ~~~~ ~~
 (F\,\{s\})^\circ =
 F^\circ
\]
\[ \qquad
 (\forall X^\kappa . B)^\circ =
 \forall X^{\kappa^\circ} . B^\circ
 ~~~~ ~~~~
 (\forall i^A . B)^\circ =
 B^\circ
\]
\[ \fbox{$\Delta^\circ$}
 ~~~~
 \cdot^\circ = \cdot
 ~~~~ ~~
 (\Delta,X^\kappa)^\circ = \Delta^\circ,X^{\kappa^\circ}
 ~~~~ ~~
 (\Delta,i^A)^\circ = \Delta^\circ
\]
\[ \fbox{$\Gamma^\circ$}
 ~~~~
 \cdot^\circ = \cdot
 ~~~~ ~~~~
 (\Gamma,x:A)^\circ = \Gamma^\circ,x:A^\circ
\]
\end{definition}

\begin{theorem}[index erasure on well-sorted kinds]
\label{thm:ierasesorting}
	$\inference{|- \kappa : \square}{|- \kappa^\circ : \square}$
\end{theorem}
\begin{proof}
	By induction on the sorting derivation.
\end{proof}
\begin{remark}
For any well-sorted kind $\kappa$ in \Fi,
$\kappa^\circ$ is a kind in \Fw.
\end{remark}

\begin{theorem}[index erasure on well-formed type level contexts]
\label{thm:ierasetyctx}
\[ \inference{|- \Delta}{|- \Delta^\circ} \]
\end{theorem}
\begin{proof}
	By induction on the derivation for well-formed type level context
	and using Theorem \ref{thm:ierasesorting}.
\end{proof}
\begin{remark}
For any well-formed type level context $\Delta$ in \Fi,
$\Delta^\circ$ is a well-formed type level context in \Fw.
\end{remark}

\begin{theorem}[index erasure on kind equality]\label{thm:ierasekindeq}
$ \inference{|- \kappa=\kappa':\square}
	{|- \kappa^\circ=\kappa'^\circ:\square}
$
\end{theorem}
\begin{proof}
	By induction on the kind equality judgement.
\end{proof}
\begin{remark}
For any well-sorted kind equality $|- \kappa=\kappa':\square$ in \Fi,
$|- \kappa^\circ=\kappa'^\circ:\square$ is a well-sorted kind equality in \Fw.
\end{remark}

The three theorems above on kinds are rather simple to prove since there is
no need to consider mutual recursion in the definition of kinds due to
the erasure operation on kinds. Recall that the erasure operation on kinds
discards the type ($A$) appearing in the index arrow type ($A -> \kappa$).
So, there is no need to consider the types appearing in kinds
and the index terms appearing in those types, after the erasure.\\

\begin{theorem}[index erasure on well-kinded type constructors]
\label{thm:ierasekinding}
\[ \inference{|- \Delta & \Delta |- F : \kappa}
		{\Delta^\circ |- F^\circ : \kappa^\circ}
\]
\end{theorem}
\begin{proof}
	By induction on the kinding derivation.
\begin{itemize}
\item[case] ($Var$)
	Use Theorem \ref{thm:ierasetyctx}.

\item[case] ($Conv$)
	By induction and using Theorem \ref{thm:ierasekindeq}.

\item[case] ($\lambda$)
	By induction and using Theorem \ref{thm:ierasesorting}.

\item[case] ($@$)
	By induction.

\item[case] ($\lambda i$)

	We need to show that
	$\Delta^\circ |- (\lambda i^A.F)^\circ : (A -> \kappa)^\circ$,
	which simplifies to $\Delta^\circ |- F^\circ : \kappa^\circ$
	by Definition \ref{def:ierase}.

	By induction, we know that
	$(\Delta,i^A)^\circ |- F^\circ : \kappa^\circ $,
	which simplifies $\Delta^\circ |- F^\circ : \kappa^\circ$
	by Definition \ref{def:ierase}.

\item[case] ($@ i$)

	We need to show that
	$\Delta^\circ |- (F\;\{s\})^\circ : \kappa^\circ$,
	which simplifies to $\Delta^\circ |- F^\circ : \kappa^\circ$
	by Definition \ref{def:ierase}.

	By induction we know that
	$\Delta^\circ |- F^\circ : (A -> \kappa)^\circ$,
	which simplifies to $\Delta^\circ |- F^\circ : \kappa^\circ$
	by Definition \ref{def:ierase}.

\item[case] ($->$)
	By induction.

\item[case] ($\forall$)

	We need to show that
	$\Delta^\circ |- (\forall X^\kappa.B)^\circ : *^\circ$,
	which simplifies to
	$\Delta^\circ |- \forall X^{\kappa^\circ}.B^\circ : *$
	by Definition \ref{def:ierase}.

	Using Theorem \ref{thm:ierasesorting}, we know that
	$|- \kappa^\circ : \square$.

	By induction we know that
	$(\Delta,X^\kappa)^\circ |- B^\circ : *^\circ$,
	which simplifies to
	$\Delta^\circ,X^{\kappa^\circ} |- B^\circ : *$
	by Definition \ref{def:ierase}.

	Using the kinding rule ($\forall$), we get exactly
	what we need to show:
	$\Delta^\circ |- \forall X^{\kappa^\circ}.B^\circ : *$.

\item[case] ($\forall i$)

	We need to show that
	$\Delta^\circ |- (\forall i^A.B)^\circ : *^\circ$,
	which simplifies to $\Delta^\circ |- B^\circ : *$
	by Definition \ref{def:ierase}.

	By induction we know that
	$(\Delta,i^A)^\circ |- B^\circ : *^\circ$,
	which simplifies $\Delta^\circ |- B^\circ : *$
	by Definition \ref{def:ierase}.

\end{itemize}
\end{proof}

\begin{theorem}[index erasure on type constructor equality]
\[ \inference{\Delta |- F=F':\kappa}
		{\Delta^\circ |- F^\circ=F'^\circ:\kappa^\circ}
\]
\label{thm:ierasetyconeq}
\end{theorem}
\begin{proof}
By induction on the derivation of type constructor equality.

Most of the cases are done by applying the induction hypothesis
and sometimes using Proposition \ref{prop:wfkind}.

The only interesting cases, which are worth elaborating, are
the equality rules involving substitution.
There are two such rules.

\paragraph{}
  $\inference{\Delta,X^\kappa |- F:\kappa' & \Delta |- G:\kappa}
             {\Delta |- (\lambda X^\kappa.F)\,G = F[G/X]:\kappa'}$ \\

We need to show
$ \Delta^\circ |- ((\lambda X^\kappa.F)\,G)^\circ = (F[G/X])^\circ : \kappa'^\circ $,
which simplifies to 
$ \Delta^\circ |- (\lambda X^{\kappa^\circ}.F^\circ)\,G^\circ = (F[G/X])^\circ : \kappa'^\circ $
by Definition \ref{def:ierase}.

By induction, we know that $(\Delta,X^\kappa)^\circ |- F^\circ : \kappa'^\circ$,
which simplifies to $\Delta^\circ,X^{\kappa^\circ} |- F^\circ : \kappa'^\circ$.
by Definition \ref{def:ierase}.

Using the kinding rule ($\lambda$), we get
$\Delta^\circ |- \lambda X^{\kappa^\circ} F^\circ : \kappa^\circ -> \kappa'^\circ$.

Using the kinding rule ($@$), we get
$\Delta^\circ |- (\lambda X^{\kappa^\circ} F^\circ) G^\circ : \kappa^\circ$.

Using the very equality rule of this case,\\ we get 
$\Delta^\circ |- (\lambda X^{\kappa^\circ} F^\circ) G^\circ =  F^\circ[G^\circ/X] : \kappa^\circ$.

All we need to check is $([G/X]F)^\circ = F^\circ[G^\circ/X]$,
which is easy to see.

\paragraph{}
  $\inference{\Delta,i^A |- F:\kappa & \Delta;\cdot |- s:A}
             {\Delta |- (\lambda i^A.F)\,\{s\} = F[s/i]:\kappa}$ \\

By induction we know that $\Delta^\circ |- F^\circ : \kappa^\circ$.

The erasure of the left hand side of the equality is\\
$((\lambda i^A.F)\,\{s\})^\circ = (\lambda i^A.F)^\circ = F^\circ$.

All we need to show is $(F[s/i])^\circ = F^\circ$, which is obvious
since index variables can only occur in index terms and index terms
are always erased. Recall the index erasure over type constructors in
Definition \ref{def:ierase}, in particular, $(\lambda i^A.F)^\circ=F^\circ$,
$(F\{s\})^\circ=F^\circ$, and $(\forall i^A.B)^\circ=B^\circ$.
\end{proof}

The proofs for the two theorems above on type constructors need not consider
mutual recursion in the definition of type constructors due to
the erasure operation. Recall that the erasure operation on type constructors
discards the index term ($s$) appearing in the index application $(F\;\{s\})$.
So, there is no need to consider the index terms appearing in the types after
the erasure.

\begin{theorem}[index erasure on well-formed term level contexts]
\label{thm:ierasetmctx}
\[ \inference{\Delta |- \Gamma}{\Delta^\circ |- \Gamma^\circ} \]
\end{theorem}
\begin{proof}
By induction on $\Gamma$.
\begin{itemize}
\item[case] ($\Gamma=\cdot$) It trivially holds.
\item[case] ($\Gamma = \Gamma',x:A$),
we know that  $\Delta |- \Gamma'$ and $\Delta |- A:*$
by the well-formedness rules
and that $\Delta^\circ |- \Gamma'^\circ$ by induction.

From $\Delta |- A:*$, we know that $\Delta^\circ |- A^\circ :*$
by Theorem \ref{thm:ierasekinding}.

We know that $\Delta^\circ |- \Gamma'^\circ,x:A^\circ$
from $\Delta^\circ |- \Gamma'^\circ$ and $\Delta^\circ |- A^\circ :*$
by the well-formedness rules.

Since $\Gamma'^\circ,x:A^\circ = (\Gamma',x:A)^\circ = \Gamma^\circ$
by definition, we know that $\Delta^\circ |- \Gamma^\circ$.
\end{itemize}
\end{proof}

\begin{theorem}[index erasure on index-free well-typed terms]
\label{thm:ierasetypingifree}
For any term $t$ that does not contain any index variables,
\[ \inference{\Delta |- \Gamma & \Delta;\Gamma |- t : A}
		{\Delta^\circ;\Gamma^\circ |- t : A^\circ}
\]
\end{theorem}
\begin{proof} By induction on the typing derivation.
	Interesting cases are the index related rules ($:i$), ($\forall Ii$),
	and ($\forall Ei$). Proofs for the other cases are straightforward
	by induction and applying other erasure theorems corresponding to
	the judgment forms.
\begin{itemize}
\item[case] ($:$)
	By Theorem \ref{thm:ierasetmctx}, we know that
	$\Delta^\circ|- \Gamma^\circ$ when $\Delta|- \Gamma$.
	By definition of erasure on term-level context, we know that
	$x:A^\circ \in \Gamma^\circ$ when $x:A \in \Gamma$.
\item[case] ($:i$)
	Vacuously true since we assume that $t$ does not contain
	any index variables.
\item[case] ($->$$I$)
	By Theorem \ref{thm:ierasekinding}, we know that $\circ |- A^\circ:*$.
	By induction, we know that
	$\Delta^\circ;\Gamma^\circ,x:A^\circ |- t^\circ : B^\circ$.
	Applying the ($->$$I$) rule to what we know, we have
	$\Delta^\circ;\Gamma^\circ |- \l x.t^\circ : A^\circ -> B^\circ$.
\item[case] ($->$$E$)
	Straightforward by induction.
\item[case] ($\forall I$)
	By Theorem \ref{thm:ierasesorting}, we know that
	$|- \kappa^\circ:\square$.
	By induction, we know that
	$\Delta^\circ,X^{\kappa^\circ};\Gamma^\circ |- t : B^\circ$.
	Applying the ($\forall I$) rule to what we know, we have
	$\Delta^\circ;\Gamma^\circ |- t : \forall X^{\kappa^\circ}.B^\circ$.
\item[case] ($\forall E$)
	By induction, we know that
	$\Delta^\circ;\Gamma^\circ |- t : \forall X^{\kappa^\circ}.B^\circ$.
	By Theorem \ref{thm:ierasekinding}, we know that
	$\Delta^\circ |- G^\circ : \kappa^\circ$.
	Applying the ($\forall E$) rule, we have
	$\Delta^\circ;\Gamma^\circ |- t : B^\circ[G^\circ / X]$.
\item[case] ($\forall Ii$)
	By Theorem \ref{thm:ierasekinding}, we know that $\circ |- A^\circ:*$.
	By induction, we know that $\Delta^\circ;\Gamma^\circ |- t : B^\circ$,
	which is what we want since $(\forall i^A.B)^\circ = B^\circ$.
\item[case] ($\forall Ei$)
	By induction, we know that $\Delta^\circ;\Gamma^\circ |- t : B^\circ$,
	which is what we want since $(B[s/i])^\circ = B^\circ$.
\item[case] ($=$)
	By Theorem \ref{thm:ierasetyconeq} and induction.
\end{itemize}
\end{proof}

\begin{definition}[index variable selection]
\[ \circ^\bullet = \circ \qquad
	(\Delta,X^A)^\bullet = \Delta^\bullet \qquad
	(\Delta,i^A)^\bullet = \Delta^\bullet,i^A
\]
\end{definition}\noindent
The index variable selection operation ($^\bullet$) selects
all the index variable bindings from the type level context.
\begin{theorem}[index erasure on well-formed term level contexts
		prepended by index variable selection]
\label{thm:ierasetmctxivs}
\[ \inference{\Delta |- \Gamma}{\Delta^\circ |- (\Delta^\bullet,\Gamma)^\circ}
\]
\end{theorem}
\begin{proof}
Straightforward by Theorem \ref{thm:ierasetmctx} and the typing rule ($:i$).
\end{proof}

\begin{theorem}[index erasure on well-typed terms]
\label{thm:ierasetypingall}
\[ \inference{\Delta |- \Gamma & \Delta;\Gamma |- t : A}
		{\Delta^\circ;(\Delta^\bullet,\Gamma)^\circ |- t : A^\circ}
\]
\end{theorem}
\begin{proof}
	The proof is almost the same as as Theorem \ref{thm:ierasetypingifree},
	except for the ($:i$) case. The proof for the ($:i$) case is easy
	since $i^A \in \Delta^\bullet$ when $i^A \in \Delta$ by definition of
	the index variable selection operation. The indices from $\Delta$
	being prepended to $\Gamma$ do not affect the proof for the other cases.
\end{proof}

%% \begin{theorem}[index erasure on term equality]
%% \[ \inference{\Delta;\Gamma |- t=t':A}
%%  	{\Delta^\circ;\Gamma^\circ |- t=t':A^\circ}
%% \]
%% \end{theorem}

\subsection{TODO about void type instantiation}
Why did we bother to design \Fi? What is different from a Curry-style
dependent calculus with implicit arguments such as ICC? The following rule
is the instantiation 
\[
\inference{\Gamma,x:A |- t : B}{\Gamma |- t : \forall x^A.B }~(x\notin\FV(t))
\]
Consider when $A=Void$ and $B=\forall i^{Void}.NeverEverVoid\,\{i\}$.
In the calculus above, we can instantiate $i$ with $y$ provided that
$(y:Void)\in\Gamma$. It is a void type instantiation, Uh-Oh ...

In \Fi\ we cannot instantiate $B$ with any of the term variables
since index instantiation can not refer to the term-level context
but only refer to the type-level context $\Delta$. Recall
\[
	\inference[($\forall E i$)]{ \Delta;\Gamma |- t : \forall i^A.B & \textcolor{red}{\Delta;\cdot |- s:A} }
	                           {\Delta;\Gamma |- t : B[s/i]}
			   \]

\begin{proposition}[anti-dependency on arrow kinds]
\[ \inference{ |- \Delta,X^\kappa
             & \Delta,X^\kappa |- F : \kappa' }
             { X\notin\FV(\kappa') }
\]
\end{proposition}
\begin{proof}
	By Proposition \ref{prop:wfkind}, $|- \kappa'$.
	Note that $|- \kappa'$ does not involve any type level context.

	Therefore, $X$ cannot appear free in $\kappa'$.
\end{proof}

\begin{proposition}[anti-dependency on indexed arrow kinds]
\[ \inference{ |- \Delta,i^A
             & \Delta,i^A |- F : \kappa }
             { i\notin\FV(\kappa) }
\]
\end{proposition}
\begin{proof}
	By Proposition \ref{prop:wfkind}, $|- \kappa'$.
	Note that $|- \kappa'$ does not involve any type level context.

	Therefore, $i$ cannot appear free in $\kappa'$.
\end{proof}

\begin{proposition}[anti-dependency on arrow types]
\[ \inference{ \Delta |- \Gamma,x:A
             & \Delta;\Gamma,x:A |- t : B }
             { x\notin\FV(B) }
\]
\end{proposition}
\begin{proof}
	By Proposition \ref{prop:wftype}, $\Delta |- B$.
	Note that $\Delta |- \kappa'$ does not involve any term level context.

	Therefore, $x$ cannot appear free in $B$.
\end{proof}


\begin{remark} Our system is more strong??? than anti-dependency on arrow types
TODO
\end{remark}



\section{Related Work}\label{sec:relwork}
TODO

\subsection{Mendler-style primitive recursion}\label{sec:relwork:mpr}
TODO

TODO
we should introduce Mendler-style primitive recursion (\mpr{}) here
because we are going to discuss \mprsi{}
TODO

\subsection{Type-based termination and sized types}\label{sec:relwork:sized}
\emph{Type-based termination} (coined by Barthe and others \cite{BartheFGPU04})
stands for approaches that integrate termination into type checking,
as opposed to syntactic approaches (\ie, conventional approach) that
reason about termination over untyped term structures.
The Mendler-style approach is, of course, type-based. In fact, the idea of
type-based termination was inspired by Mendler \cite{Mendler87,Mendler91}.
In the Mendler style, we know that well-typed functions defined using
Mendler-style recursion schemes always terminate.  This guarantee flows
from the design of the recursion scheme, where the use of higher-rank 
polymorphic types in the abstract operations enforce the invariants
necessary for termination.

Abel \cite{abel06phd,Abel12talkFICS} summarizes the advantages of
type-based termination as follows:
\textbf{communication} (programmers can think using types),
\textbf{certification} (types are machine checkable certificates),
\textbf{a simple theoretical justification}
        (no additional complication for termination other than type checking),
\textbf{orthogonality} (only small parts of the language are affected,
        \eg, principled recursion schemes instead of general recursion),
\textbf{robustness} (type system extensions are less likely to
                        disrupt termination checking),
\textbf{compositionality}\footnote{This is not listed in Abel's thesis,
                                but comes from his invited talk in FICS 2012.}
        (one needs only types, not the code, for checking the termination), and
\textbf{higher-order functions and higher-kinded datatypes}
        (works well even for higher-order functions and non-regular datatypes,
        as a consequence of compositionality).
In his dissertation \cite{abel06phd} (Section 4.4) on sized types,
Abel views the Mendler-style approach as enforcing size restrictions
using higher-rank polymorphism as follows:
\begin{itemize}
\item The abstract recursive type $r$ in Mendler style corresponds to
        $\mu^\alpha F$ in his sized-type system (System~\Fwhat),
        where the sized type
        for the value being passed in corresponds to $\mu^{\alpha+1} F$.
\item The concrete recursive type $\mu F$ in Mendler style corresponds to
        $\mu^\infty F$ since there is no size restriction.
\item By subtyping, a type with a smaller size index can be cast to
        the same type with a larger size index.
\end{itemize}
The same intuition holds for the termination behaviors
of Mendler-style recursion schemes over positive datatypes.
Mendler-style recursion schemes terminate, for positive datatypes,
because $r$-values are direct subcomponents of the value being eliminated.
They are always smaller than the value being passed in.
Types enforce that recursive calls are only well typed,
when applied to smaller subcomponents.

Abel's System~\Fwhat\ can express primitive recursion quite naturally
using subtyping. The casting operation $(r \to \mu F)$ in Mendler-style
primitive recursion corresponds to an implicit conversion by subtyping
from $\mu^\alpha F$ to $\mu^\infty F$ because $\alpha \leq \infty$.

System~\Fwhat\ \cite{abel06phd} is closely related to
System~\Fixw\ \cite{AbeMat04}. Both of these systems are base on
equi-recursive fixpoint types over positive base structures.
Both of these systems are able to embed (or simulate) Mendler-style
primitive recursion (which is based on iso-recursive types) via
the encoding \cite{Geu92} of arbitrary base structures into
positive base structures.
%% In \S\ref{sec:fixi:data}, we rely on
%% the same encoding, denoted by $\Phi$, when embedding \MPr\ into System~\Fixw.

Abel's sized-type approach provides good intuitions why 
certain recursion schemes terminate over positive datatypes.
But, it does not give a good intuition of whether or not
those recursion schemes would terminate for negative datatypes,
unless there is an encoding that can translate negative datatypes into
positive datatypes. For primitive recursion, this is possible (as we
mentioned above). However, for our recursion scheme \MsfIt, which is
especially useful over negative datatypes, we do not know of an encoding
that can map the inverse augmented fixpoints into positive fixpoints.
So, it is not clear whether Abel's the sized type approach based on
positive equi-recursive fixpoint types can provide a good intuition
for the termination behavior of \MsfIt. In the following section,
we will discuss another Mendler-style recursion scheme (\mprsi{}), which
is also useful over negative datatypes and has a termination property
(not proved yet) based on the size of the index in the datatype.



\section{Mendler-style primitive recursion with a sized index}
\label{sec:mprsi}

In \S\ref{sec:mendler} and \S\ref{sec:HOASeval}, we discussed
Mendler-style iteration with a syntactic inverse, \lstinline{msfcata},
which is particularly useful for defining functions over
mixed-variant datatypes. We demonstrated the usefulness of
\lstinline{msfcata} by defining functions over HOAS:
\begin{itemize}
\item the string formatting function \lstinline{showExp} for
	the untyped HOAS using \lstinline{msfcata0}
	(Figure\;\ref{lst:HOASshow}) and
\item the type-preserving evaluator \lstinline{eval} for
	the simply-typed HOAS using \lstinline{msfcata1}
	(Figure\;\ref{lst:HOASeval}).
\end{itemize}

In this section, we speculate about another Mendler-style recursion scheme,
\lstinline{mprsi}, motivated by an example similar to the \lstinline{eval}
function. The name \lstinline{mprsi} stands for
Mendler-style primitive recursion with a sized index.

\begin{figure}
\lstinputlisting[
	caption={A simply-typed HOAS evaluation via a user-defined value domain.
		\label{lst:HOASevalV}
		},
	firstline=4]{HOASevalV.hs}
\vspace*{-3ex}
\end{figure}

We review the \lstinline{eval} example and then compare
it to our motivating example \lstinline{veval} for \lstinline{mprsi}.
Both \lstinline{eval} and \lstinline{veval} are illustrated
in Figure\;\ref{lst:HOASevalV}. Recall that this code is written in Haskell,
following the Mendler-style conventions.
The function \lstinline{eval :: Exp t -> Id t} is
a type preserving evaluator that evaluates an HOAS expression of type
\lstinline{t} to a (Haskell) value of type \lstinline{t}.
The \lstinline{eval} function always terminates because
\lstinline{msfcata1} always terminates. Recall that \lstinline{msfcata1}
and \lstinline{Rec1} can be embedded into System~\Fw.

The motivating example \lstinline{veval :: Exp t -> Val t} is also 
a type preserving evaluator. Unlike \lstinline{eval} it evaluates to 
a user-defined value domain \lstinline{Val} of type \lstinline{t} (rather
than a Haskell value). The definition of \lstinline{veval} is similar to
\lstinline{eval} -- both of them are defined using \lstinline{msfcata1}.
The first equation of \lstinline{phi} for evaluating
the \lstinline{Lam}-expression is essentially the same as
the corresponding equation in the definition of \lstinline{eval}.
The second equation of \lstinline{phi} for evaluating
the \lstinline{App}-expression is also similar in structure to
the corresponding equation in the definition of \lstinline{eval},
but the use of \lstinline{unVal} is problematic. Note, the definition of
\lstinline{unVal} relies on pattern matching against \lstinline{In1}.
Recall that one cannot freely pattern match against a recursive value
in Mendler style. Recursive values must be analyzed (or, eliminated) by
using Mendler-style recursion schemes. It is not a problem to use
\lstinline{unId} in the definition of \lstinline{eval} because
\lstinline{Id} is non-recursive.

It is not likely that \lstinline{unVal} can be defined using any of
the existing Mendler-style recursion schemes. So, we designed
a new Mendler-style recursion scheme that can express \lstinline{unVal}.
The new recursion scheme \lstinline{mprsi} extends \lstinline{mprim} with
an additional uncast operation. Recall that \lstinline{mprim} has
two abstract operations, call and cast. So, \lstinline{mprsi} has
three abstract operations, call, cast, and uncast. In the following paragraphs,
we explain the design of \lstinline{mprsi} step-by-step.

Let us try to define \lstinline{unVal} using \lstinline{mprim1}, and see
where it falls short. \lstinline{mprim1} provides two abstract operations,
\lstinline{cast} and {call}, as you can see from its type sigaure:
\begin{lstlisting}
mprim1 :: (forall r i. (forall i. r i -> Mu1 f i)  -- cast
              -> (forall i. r i -> a i)      -- call
              -> (f r i -> a i)        ) -> Mu1 f i -> a i
\end{lstlisting}
We attempt to define \lstinline{unVal} using \lstinline{mprim1} as follows:
\begin{lstlisting}
unVal :: Mu1 V (t1 -> t2) -> (Mu1 V t1 -> Mu1 V t2)
unVal = mprim1 phi where
  phi cast call (VFun f) = ...
\end{lstlisting}
Inside the \lstinline{phi} function, we have a function
\lstinline{f :: (r t1 -> r t2)} over abstract recursive values.
We need to cast \lstinline{f} into a function over concrete recursive values
\lstinline{(Mu1 V t1 -> Mu1 V t2)}.
We should not need to use \lstinline{call}, since we do not expect
to use any recursion to define \lstinline{unVal}.
So the only available operation to
work with is \lstinline{cast :: (forall i.r i -> Mu1 f i)}.
Composing \lstinline{cast} with \lstinline{f}, we can get
\lstinline{(cast . f) :: (r t1 -> Mu1 V t2)}, whose codomain
\lstinline{(Mu1 V t2)} is exactly what we want. But, the domain
is still abstract \lstinline{(r t1)} rather than being concrete
\lstinline{(Mu1 V t1)}. We are stuck.

What additional abstract operation would help us complete
the definition of \lstinline{unVal}? We need an abstract operation
to cast from \lstinline{(r t1)} to \lstinline{(Mu1 V t1)}
in a contravariant position. If we had an inverse of cast,
\lstinline{uncast :: (forall i.Mu1 f i -> r i)}, we can
complete the definition of \lstinline{unVal} by composing
\lstinline{uncast}, \lstinline{f}, and \lstinline{cast}.
Observe that \lstinline{uncast . f . cast :: (Mu1 V t1 -> Mu1 V t2)}.
Thus, we can formulate \lstinline{mprsi1} with a naive type signature
as follows:
\begin{lstlisting}
mprsi1 :: (forall r i. (forall i. r i -> Mu1 f i)  -- cast
               -> (forall i. Mu1 f i -> r i)  -- uncast
               -> (forall i. r i -> a i)      -- call
               -> (f r i -> a i)        ) -> Mu1 f i -> a i

mprsi1 phi (In1 x) = phi id id (mprsi1 phi) x
\end{lstlisting}
Although the type signature above is type correct, it is too powerful.
The Mendler-style uses types to forbid non terminating computations
as ill-typed. Having both \lstinline{cast} and \lstinline{uncast} supports
the same ability as freely pattern matching over recursive values,
which can lead to non-termination as we showed in the introduction.
To recover the guarantee of termination, we need to restrict the use of
either \lstinline{cast} and \lstinline{uncast}, or both.

Let us see how this non-termination might occur. If we allowed
\lstinline{mprsi1} with the naive type signature above, we could write
an evaluator (similar to \lstinline{veval} but for an untyped HOAS),
which does not always terminate. This evaluator would diverge for terms
with self application. Typed terms use the type index to prevent
such diverging terms.

We walk through the process of defining an untyped HOAS.
The base structures of the untyped HOAS and its value domain
can be defined as follows:
\begin{lstlisting}
data ExpF_u r t = Lam_u (r t -> r t) | App_u (r t) (r t)
data V_u r t = VFun_u (r t -> r t)
\end{lstlisting}
Fixpoints of the structures above represent the untyped HOAS and
its value domain. Here, the index \lstinline{t} does not track
the types of terms but remains constant everywhere.
Why did we believe that \lstinline{veval} always terminates?
Because it evaluates a well-typed HOAS, whose type is encoded as
an index \lstinline{t} of the recursive datatype \lstinline{(Exp t)}. That is,
the use of indices as types is the key to the termination property.
Therefore, our idea is to restrict the use of the abstract operations
by enforcing constraints over their indices. We allow the use
of the abstract operations only over abstract values whose indices are
smaller in size compared to the size of the argument index.
For the \lstinline{veval} example, we define being smaller as
the structural ordering over types, that is, \lstinline{t1 < (t1 -> t2)}
and \lstinline{t2 < (t2 -> t1)}.
We have two candidates for the type signature of \lstinline{mprsi1}:
\begin{itemize}
\item Candidate 1: restrict uses of both \lstinline{cast} and \lstinline{uncast}
\begin{lstlisting}
mprsi1 :: (forall r j. (forall i. (i<j) => r i -> Mu1 f i)  --  cast
               -> (forall i. (i<j) => Mu1 f i -> r i)  --  uncast
               -> (forall i.          r i -> a i)      --  call
               -> (f r j -> a j)          ) -> Mu1 f i -> a i
\end{lstlisting}
\item Candidate 2: restrict the use of \lstinline{uncast} only
\begin{lstlisting}
mprsi1 :: (forall r j. (forall i.           r i -> Mu1 f i)  --  cast
               -> (forall i. (i<j) =>  Mu1 f i -> r i)  --  uncast
               -> (forall i.           r i -> a i)      --  call
               -> (f r j -> a j)          ) -> Mu1 f i -> a i
\end{lstlisting}
\end{itemize}
We strongly believe that the first candidate always terminate,
but it might be overly restrictive. Maybe the second candidate is
enough to guarantee termination? Both candidates allow defining
\lstinline{veval}, since one can define \lstinline{unVal}
using \lstinline{mprsi1} with either one of the candidates.
Both candiates forbid the definition of an evaluator over the untyped HOAS,
because neither support extracting functions from the untyped value domain.

We need further studies to prove termination properties of \lstinline{mprsi}.
The sized-type approach, discussed in the related work section,
seems to be relevant to showing termination of \lstinline{mprsi}.
However, existing theories on sized-types are not directly applicable to
\lstinline{mprsi} because they are focused on positive datatypes, but
not negative datatypes.



\section{Summary}\label{sec:summary}
My dissertation will consist of ten chapters.
Tentative titles of the chapters are listed below:
\begin{quote}
\begin{enumerate}[1.]
\item Introduction
\item Polymorphic type systems (System \textsf{F}, System \Fw, Hindley-Milner)
\item Mendler style recursion schemes
\item System \Fi
\item System \Fixi
\item The Nax language
\item Case studies
\item Related work
\item Future work
\item Conclusion
\end{enumerate}
\end{quote}

In the introduction, I will state my thesis and motivation for my work.
Also, I plan to give a summary on the history of typed programming languages
and formal reasoning. in particular, I will cover the subjects including
Curry-Howard correspondence, recursive types, and induction.

Then, I will review System \textsf{F}, System \Fw, and the Hindley-Milner
type system, in order to lead the discussions in the following sections
on Mendler style recursion schemes, System \Fi, System \Fixi, and
the Nax langauge.

In the case studies section, I will demonstrate larger examples of Nax programs
to demonstrate the expressiveness of Nax. One of the examples I have in mind
is normalization by evaluation.

In thre related work section, I will provide further details on what I have
summeraized in the introduction secction, and also discuss some recent work
related to my approach.

Then, I will discuss future work, which will mainly be discussion on further
extensions to the Nax language, and their ramifications to
the target calculi and the type inference algorithm.

Finally, I will summarize and conclude my thesis.

\paragraph{Achivements, progress, and plans:}
TODO



\section*{Acknowledgements}
%% \subparagraph*{Acknowledgements}
I want to thank \dots

\appendix
\section{TODO Appendix Section}
TODO

\bibliography{main}

\end{document}
