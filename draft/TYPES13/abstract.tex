\documentclass[a4paper]{easychair} % A4 is needed for the abstract book

%\documentclass[a4paper, debug]{easychair} 
% can be used to better see overfull boxes

\usepackage{enumerate}

\bibliographystyle{plain}

%\newtheorem{thm}{Theorem}   % no such environments are predefined

\title{Type-Based Termination of\\ Term-Indexed Types}
\titlerunning{Type-Based Termination of Term-Indexed Types}
\author{
Ki Yung Ahn\inst{1}
\and
Tim Sheard\inst{1}
 \and
TODO\inst{2}
 \and
TODO\inst{2}
}
\institute{
  Portland State University,\thanks{Funded by XXX project.} \\
  Portland, OR, USA
\and
  University of Cambridge, \\
  Cambridge, UK
}
\authorrunning{Author1 and Author2}

\newcommand{\cf}[0]{{cf.}}
\newcommand{\eg}[0]{{e.g.}}
\newcommand{\ie}[0]{{i.e.}}
\newcommand{\aka}[0]{{a.k.a.}}

\newcommand{\F}[0]{{\ensuremath{\mathsf{\uppercase{F}}}}}
\newcommand{\Fw}[0]{{\ensuremath{\mathsf{\uppercase{F}}_{\!\omega}}}}

\begin{document}
\maketitle

%\begin{abstract}
%\end{abstract}
% abstracts of abstracts are not compulsory

The context of our work is the Nax programming language project.
We are developing a unified programming and reasoning system,
called Nax, with the following design goals:\vspace*{-1ex}
\begin{enumerate}[(1)]
 \item supports major constructs of modern functional programming languages,
 such as parametric polymorphism, recursive datatypes, and type inference,
 \vspace*{-1.2ex}
 \item can specify fine-grained program properties as types and
 prove them by writing a program (Curry--Howard correspondence),
 \vspace*{-1.2ex}
 \item is based on a minimal foundational calculus
 that is expressible enough to embed all the language constructs in (1)
 and also logically consistent to avoid paradoxical proofs in (2),
 \vspace*{-1.2ex}
 \item a simple implementation infrastructure that keeps the trusted base small.
\end{enumerate}
Our approach twoards these goals is to put together
\emph{Mendler-style recursion schemes}
and \emph{term-indexed datatypes},
while finding an appropriate foundational calculus.
Term-indexed datatypes are necessary to support (2), for insance,
statically specifying size of a list by a natural number index.
Mendler-style recursion schemes support (1) since they are based
on parametric polymorphism and well-defined over wide range of datatypes,
and also support (4) since their termination is type-based --
no other termination checking infrastructure is necessary.

In this abstract, we first overview advantages of adopting Mendler-style,
and then, introduce a new interesting Mendler-style recursion scheme
motivated from the term-indeices that goven the termination
behavior of their datatypes.
\vspace*{-.5ex}
\paragraph{Advantages of adopting Mendler-style\!\!\!}
is that it allows formation of
any recursive datatype, while providing rich set of
principled eliminators that ensure well-behaved use.
Certain concepts, such as Higher-Order Abstract Syntax
(HOAS) are most succintly defined as mixed-variant datatypes.
However, most existing reasoning systems based on
Curry--Howard correspondence, unfortunatly, restrict
formation of those recursive datatypes, even though ill-behaved
use were never attempted. As a result, one is forced to devise
certain tricks to encode concepts like HOAS within the restricted
forms of datatypes in such systems.

We believe it is worthwile to allow formation of all the recursive datatypes
(\eg, non-strictly positive, mixed-variant, nested)
usually available in functional languages. For instance,
in Haskell, we can define a HOAS for the untyped lambda-calculus
as folows.
\begin{verbatim}
    data Exp = Abs (Exp -> Exp) | App Exp Exp
\end{verbatim}
Even if we assume all functions contained in \texttt{Abs} are non-recursive,
trying to evaluate these HOAS expressions may still cause problems in logial reasoning,
since the untyped lambda calculus has diverging terms. However, there are
many well-behaved computations on \texttt{Exp}. For instance, examining whether
an HOAS expression is \texttt{Abs} or \texttt{App}, and, converting an HOAS expression
to first-order synax are examples of terminating computation on \texttt{Exp}.
In order to capture these well-behaved computations, Ahn and Sheard \cite{AhnShe11}
formalized a new Mendler-style recursion scheme.
\vspace*{-1ex}
\paragraph{Term-indices that govern termination behavir of their datatypes\!\!}
motivate a new Mendler-style recursion scheme.
Consider yet another HOAS datatype below for the Simply-Typed Lambda-Calculus (STLC)
defiend in Nax-like syntax,\footnote{curly braces emphasize
  term-indices used in types (\texttt{Exp\{t1\}}),
  and types used in kinds (\texttt{\{Ty\}\;->\;*}).}
where HOAS expressions (\texttt{Exp}) are
statically indexed by the terms that represent STLC-types (\texttt{Ty}).
\begin{verbatim}
    data Ty : * where    Iota : Ty
                         Arr  : Ty -> Ty -> Ty

    data Exp : {Ty} -> * where   Abs : (Exp{t1} -> Exp{t2}) -> Exp{Arr t1 t2}
                                 App : Exp{Arr t1 t2} -> Exp{t1} -> Exp{t2}
\end{verbatim}

TODO in consideration to submit to TYPES 2013

% create the bibliography
\bibliography{main}   % refers to main.bib
\end{document}