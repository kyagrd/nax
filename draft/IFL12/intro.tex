\section{Introduction}
During the past decade, the functional programming community achieved
partial success in their goal of maintaining fine-grained properties
by moderate extensions to the type system of functional languages
\cite{CheHin03,CheHin02,Xi03}.
This approach is often called \emph{``lightweight''}\footnote{e.g.,
  \url{http://okmij.org/ftp/Computation/lightweight-dependent-typing.html} }
since using a full blown proof assistant to maintain similar properties
is likely to require much more effort (heavyweight).

The Generalized Algebraic Data Type (GADT) extension implemented
in the Glasgow Haskell Compiler (GHC) has made this approach
widely avialibe to everyday functional programming tasks.
OCaml encodings of GADTs has been reported \cite{ManStu09}
and recent versions of OCaml supports GADTs natively \cite{GarNor11}.

Unfortunately, implementations supporting a lightweight approach (\eg, GHC)
lack \textbf{correctness guarantees} and \textbf{type inference} in general.
In addition, those implementations often lack support term indexing,
so \textbf{term indices are faked} (or, simulated) by additional
type structure replicating the requisite term structure.
A recent GHC extension, datatype promotion,

\begin{itemize}
\item Nax is strongly normalizing, and logically consistent, so the
type of Nax programs can be given logical interpretations as propositions.

\item Nax supports type inference. Types appear only in data declarations
and index transformers attached to mendler operators. Index transformers
are the ``minimal"  annotation necessary to support type inference over GADTs.

\item Nax programs are expressive and concise. Nax programs
are similar in size to their Haskell and Agda equivalents, yet they
still retain logicality and type inference. We don't pay for the
complexity of two level types and mendler morphisms.

\end{itemize}

TODO contribution of Nax

TODO contribution of this paper

