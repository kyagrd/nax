\section{System \Fi}
\label{sec:Fi}
System \Fi\ is a higher-order polymorphic lambda calculus with term indices.
In other words, System \Fi\ is an extension of System \Fw\ by term indices.
The complete syntax and rules of \Fi\ are described in \Fig{Fi} and
\Fig{eqFi}. The syntax and rules highlighted by \newFi{\text{grey boxes}}
are the extensions new to \Fi, which are not originally part of \Fw.
That is, the system we obtain by excluding all the grey boxes from \Fig{Fi}
and \Fig{eqFi} is a version of \Fw. In particular, it is a version of \Fw\
with Curry-style terms and typing contexts separated into two parts
(type level and term level).

Terms are Curry style. That is, term level abstractions are unannotated
($\lambda x.t$), and type generalizations ($\forall I$) and type instantiations
($\forall E$) are implicit at term level. A Curry style calculus generally has
advantages over its Church style couterpart when reasoning about properties of
reduction. For instnace, the Church-Rosser property naturally holds for all
of $\beta$-, $\eta$-, and $\beta\eta$-reduction in Curry style, but
it may not hold in Church style due to the presense of annotations in
abstractions \cite{Miquel01}.\footnote{In Church style, the Church-Rosser
	property Does not hold in the strict sense
	(\ie, $\alpha$-equivalence over Church-style terms),
	but may hold under certain approximations such as
	modulo ignoring the annotations in abstractions.}

Type constructors remain in Church style. That is, type level abstractions are
annotated by kinds ($\lambda X^\kappa.F$). Choosing type constructors
to be Church style makes it visually more evident to which kind
a type constructor belongs. But, the choice of style for type constructors
is not as crucial as the choice of style for terms, since the syntax and
kinding rules are essentially a simply typed lambda calculus (STLC)
at type level. Chosing type constructors to be Curry style would not
make a big diffrecne for \Fi. When type constructors are also in Curry style,
the type instantiation rule ($\forall E$) becomes slightly more complicated,
though.

Typing contexts are separated into type level contexts ($\Delta$) and
term level contexts ($\Gamma$). TODO explain why in \Fw\ it may simply
a matter of taste having one context is more desriable considering
relation to other systems blah but we have a problem in \Fi\ ..
keep track of different
TODO don't explain everything here we can point to where we explain this

Although it is imaginable to formalize a system, which distinguishes
index variables from ordinary term variables using more general concepts
(\eg, capability, modality), we think that spliting the tying context into
two parts is a more simple solution just for the purose of distinguishing
index variables from ordinary term variables.

We expect readers to be familiar to \Fw\
and focus on describing new constructs of \Fi, which are in grey boxes.

\begin{figure*}
\paragraph{Syntax:}
\begin{align*}
\!\!\!\!\!\!\!\!&\text{Kinds}
 	& \kappa		&~ ::= ~ *
				\mid \kappa -> \kappa
				\mid \newFi{A -> \kappa}
\\
\!\!\!\!\!\!\!\!&\text{Type Constructors}
	& A,B,F,G		&~ ::= ~ X
				\mid A -> B
				\mid \lambda X^\kappa.F
				\mid F\,G
				\mid \forall X^\kappa . B
				\mid \newFi{\lambda i^A.F
				\mid F\,\{s\}
				\mid \forall i^A . B}
\\
\!\!\!\!\!\!\!\!&\text{Terms}
	& r,s,t			&~ ::= ~ x \mid \lambda x.t \mid r\;s
				\mid \newFi{i}
\\
\!\!\!\!\!\!\!\!&\text{Typing Contexts}
	& \Delta		&~ ::= ~ \cdot
				\mid \Delta, X^\kappa
				\mid \newFi{\Delta, i^A} \\
&	& \Gamma		&~ ::= ~ \cdot
				\mid \Gamma, x : A
\end{align*}

\paragraph{Well-formed typing contexts:}
\[ \fbox{$|- \Delta$}
 ~~~~ ~~~~
   \inference{}{|- \cdot}
 ~~~~
   \inference{|- \Delta & |- \kappa:\square}
             {|- \Delta,X^\kappa}
      \big( X\notin\dom(\Delta) \big)
 ~~~~ \newFi{
   \inference{|- \Delta & \cdot |- A:*}
             {|- \Delta,i^A}
      \big( i\notin\dom(\Delta) \big) }
\]
\[ \fbox{$\Delta |- \Gamma$}
 ~~~~
   \inference{|- \Delta}{\Delta |- \cdot}
 ~~~~
   \inference{\Delta |- \Gamma & \Delta |- A:*}
             {\Delta |- \Gamma,x:A}
      \big( x\notin\dom(\Gamma) \big)
\]
~\\
\paragraph{Sorting:} \fbox{$|- \kappa : \square$}
$ \qquad
 ~~~~
  \inference[($A$)]{}{|- *:\square}
 ~~~~
   \inference[($R$)]{|- \kappa:\square & |- \kappa':\square}
                    {|- \kappa -> \kappa' : \square}
 ~~~~
   \newFi{
   \inference[($Ri$)]{\cdot |- A:* & |- \kappa:\square}
                     {|- A -> \kappa : \square} }
$
~\\ ~\\
\paragraph{Kinding:} \fbox{$\Delta |- F : \kappa$}
$ \quad
 ~~~~
   \inference[($Var$)]{X^\kappa\in\Delta & |- \Delta}
                       {\Delta |- X : \kappa}
 ~~~~
   \inference[($->$)]{\Delta |- A : * & \Delta |- B : *}
                     {\Delta |- A -> B : * }
$
\[
  \inference[($\lambda$)]{|- \kappa:\square & \Delta,X^\kappa |- F : \kappa'}
                          {\Delta |- \lambda X^\kappa.F : \kappa -> \kappa'}
 ~~~~
   \inference[($@$)]{ \Delta |- F : \kappa -> \kappa'
                    & \Delta |- G : \kappa }
                    {\Delta |- F\,G : \kappa'}
 ~~~~
   \inference[($\forall$)]{|- \kappa:\square & \Delta, X^\kappa |- B : *}
                          {\Delta |- \forall X^\kappa . B : *}
\]
\[ \newFi{
  \inference[($\lambda i$)]{\cdot |- A:* & \Delta,i^A |- F : \kappa}
                            {\Delta |- \lambda i^A.F : A->\kappa}
 ~~~~
   \inference[($@i$)]{ \Delta |- F : A -> \kappa
                     & \Delta;\cdot |- s : A }
                     {\Delta |- F\,\{s\} : \kappa}
 ~~~~
   \inference[($\forall i$)]{\cdot |- A:* & \Delta, i^A |- B : *}
                            {\Delta |- \forall i^A . B : *} }
\]
\[ \newFi{
   \inference[($Conv$)]{ \Delta |- A : \kappa
                       & \Delta |- \kappa = \kappa' : \square }
                       {\Delta |- A : \kappa'} }
\]
~\\
\paragraph{Typing:} \fbox{$\Delta;\Gamma |- t : A$}
$ \qquad
 ~~~~
   \inference[($:$)]{x:A \in \Gamma & \Delta |- \Gamma} 
                    {\Delta;\Gamma |- x:A}
 ~~~~ \newFi{
   \inference[($:i$)]{i^A \in \Delta & \Delta |- \Gamma} 
                     {\Delta;\Gamma |- i:A} }
$
\[
   \inference[($->$$I$)]{\Delta |- A:* & \Delta;\Gamma,x:A |- t : B}
                        {\Delta;\Gamma |- \lambda x.t : A -> B}
 ~~~~ ~~~~
   \inference[($->$$E$)]{\Delta;\Gamma |- r : A -> B & \Delta;\Gamma |- s : A}
                        {\Delta;\Gamma |- r\;s : B}
\]
\[ \inference[($\forall I$)]{|- \kappa:\square & \Delta, X^\kappa;\Gamma |- t : B}
                            {\Delta;\Gamma |- t : \forall X^\kappa.B}
			    (X\notin\FV(\Gamma))
 ~~~~ ~~~~
   \inference[($\forall E$)]{ \Delta;\Gamma |- t : \forall X^\kappa.B
                            & \Delta |- G:\kappa }
                            {\Delta;\Gamma |- t : [G/X]B}
\]
\[ \newFi{
   \inference[($\forall I i$)]{\cdot |- A:* & \Delta, i^A;\Gamma |- t : B}
                              {\Delta;\Gamma |- t : \forall i^A.B}
   \left(\begin{matrix}
		i\notin\FV(t),\\
		i\notin\FV(\Gamma)\end{matrix}\right)
 ~~~~
   \inference[($\forall E i$)]{ \Delta;\Gamma |- t : \forall i^A.B
                              & \Delta;\cdot |- s:A}
                              {\Delta;\Gamma |- t : [s/i]B} }
\]
\[ \inference[($=$)]{\Delta;\Gamma |- t : A & \Delta |- A = B : *}
                    {\Delta;\Gamma |- t : B}
\]
~\\
\paragraph{Reduction:} \fbox{$t \rightsquigarrow t'$}
$ 
 ~~~~
   \inference{}{(\lambda x.t)\,s \rightsquigarrow [s/x]t}
 ~~~~
   \inference{t \rightsquigarrow t'}{\lambda x.t \rightsquigarrow \lambda x.t'}
 ~~~~
   \inference{r \rightsquigarrow r'}{r\;s \rightsquigarrow r'\;s}
 ~~~~
   \inference{s \rightsquigarrow s'}{r\;s \rightsquigarrow r\;s'}
$
~\\ ~\\
\caption{Syntax, Typing rules, and Reduction rules of \Fi}
\label{fig:Fi}
\end{figure*}

\begin{figure*}
\paragraph{Kind equality:} \fbox{$|- \kappa=\kappa' : \square$}
$ \quad
 ~~~~
   \inference{}{|- * = *:\square}
 ~~~~
   \inference{ |- \kappa_1 = \kappa_1' : \square
             & |- \kappa_2 = \kappa_2' : \square }
             {|- \kappa_1 -> \kappa_2 = \kappa_1' -> \kappa_2' : \square}
 ~~~~ \newFi{
   \inference{\cdot |- A=A':* & |- \kappa=\kappa':\square}
             {|- A -> \kappa = A' -> \kappa' : \square} }
$
\[ \inference{|- \kappa=\kappa':\square}
             {|- \kappa'=\kappa:\square}
 ~~~~
   \inference{ |- \kappa =\kappa' :\square
             & |- \kappa'=\kappa'':\square}
             {|- \kappa=\kappa'':\square}
\]
~\\
\paragraph{Type constructor equality:} \fbox{$\Delta |- F = F' : \kappa$}
$ \quad
 ~~~~
   \inference{\Delta,X^\kappa |- F:\kappa' & \Delta |- G:\kappa}
             {\Delta |- (\lambda X^\kappa.F)\,G = [G/X]F:\kappa'}
 ~~~~ \newFi{
   \inference{\Delta,i^A |- F:\kappa & \Delta;\cdot |- s:A}
             {\Delta |- (\lambda i^A.F)\,\{s\} = [s/i]F:\kappa} }
$
\[ \inference{\Delta |- X:\kappa }{\Delta |- X=X:\kappa}
 ~~~~
   \inference{\Delta |- A=A':* & \Delta |- B=B':*}{\Delta |- A-> B=A'-> B':*}
\]
\[ \inference{|- \kappa:\square & \Delta,X^\kappa |- F=F' : \kappa'}
             {\Delta |- \lambda X^\kappa.F=\lambda X^\kappa.F':\kappa-> \kappa'}
 ~~~~
   \inference{\Delta |- F=F':\kappa->\kappa' & \Delta |- G=G':\kappa}
             {\Delta |- F\,G = F'\,G' : \kappa'}
 ~~~~
   \inference{|- \kappa:\square & \Delta,X^\kappa |- B=B':*}
             {\Delta |- \forall X^\kappa.B=\forall X^\kappa.B':*}
\]
\[ \newFi{
   \inference{\cdot |- A:* & \Delta,i^A |- F=F' : \kappa}
             {\Delta |- \lambda i^A.F=\lambda i^A.F' : A -> \kappa}
 ~~~~
   \inference{\Delta |- F=F':A->\kappa & \Delta;\cdot |- s=s':A}
             {\Delta |- F\,\{s\} = F'\,\{s'\} : \kappa}
 ~~~~
   \inference{\cdot |- A:* & \Delta,i^A |- B=B':*}
             {\Delta |- \forall i^A.B=\forall i^A.B':*} }
\]
\[ \inference{\Delta |- F = F' : \kappa}{\Delta |- F' = F : \kappa}
 ~~~~
   \inference{\Delta |- F = F' : \kappa & \Delta |- F' = F'' : \kappa}
             {\Delta |- F = F'' : \kappa}
\]
~\\
\paragraph{Term equality:} \fbox{$\Delta;\Gamma |- t = t' : A$}
$ \qquad
 ~~~~
   \inference{\Delta;\Gamma,x:A |- t:B & \Delta;\Gamma |- s:A}
             {\Delta;\Gamma |- (\lambda x.t)\,s=[s/x]t : B}
 ~~~~
   \inference{\Delta;\Gamma |- x:A}{\Delta;\Gamma |- x=x:A}
$
\[ \inference{\Delta |- A:* & \Delta;\Gamma,x:A |- t=t':B}
             {\Delta;\Gamma |- \lambda x.t = \lambda x.t':B}
 ~~~~
   \inference{\Delta;\Gamma |- r=r':A-> B & \Delta;\Gamma |- s=s':A}
             {\Delta;\Gamma |- r\;s=r'\;s':B}
\]
\[ \inference{|- \kappa:\square & \Delta, X^\kappa;\Gamma |- t=t' : B}
             {\Delta;\Gamma |- t=t' : \forall X^\kappa.B}
	     (X\notin\FV(\Gamma))
 ~~~~ ~~~~
   \inference{ \Delta;\Gamma |- t=t' : \forall X^\kappa.B
             & \Delta |- G:\kappa }
             {\Delta;\Gamma |- t=t' : [G/X]B}
\]
\[ \newFi{
   \inference{\cdot |- A:* & \Delta, i^A;\Gamma |- t=t' : B}
             {\Delta;\Gamma |- t=t' : \forall i^A.B}
   \left(\begin{smallmatrix}
		i\notin\FV(t),\\
		i\notin\FV(t'),\\
		i\notin\FV(\Gamma)\end{smallmatrix}\right)
 ~~~~
   \inference{ \Delta;\Gamma |- t=t' : \forall i^A.B
             & \Delta;\cdot |- s:A}
             {\Delta;\Gamma |- t=t' : [s/x]B} }
\]
\[ \inference{\Delta;\Gamma |- t=t':A}{\Delta;\Gamma |- t'=t:A}
 ~~~~
   \inference{\Delta;\Gamma |- t=t':A & \Delta;\Gamma |- t'=t'':A}
             {\Delta;\Gamma |- t=t'':A}
\]
~\\
\caption{Equality rules of \Fi}
\label{fig:eqFi}
\end{figure*}

\subsection{The constructs new to \Fi}

\paragraph{Typing contexts}
Typing contexts are split into two parts:
type level contexts ($\Delta$) for type level bindings
and term level contexts ($\Gamma$) for term level bindings.
We have a new form of index variable bindings ($i^A$) that can appear in
type level contexts in addition to type variable bindings ($X^\kappa$).
There is only one form of term level binding ($x:A$) that can appear in
term level contexts.

A type level context $\Delta$ is well-formed when it is either empty,
extended by a type variable binding $X^\kappa$ whose kind $\kappa$ is
well-sorted under $\Delta$, or extended by an index binding $i^A$
whose type $A$ is well-kinded under the empty type level context.
A similar restriction occurs in the sorting rule ($Ri$), for indexed arrow kinds,
we require $A$ of $i^A$ to be well-kinded in the empty type level context
($\cdot$), since no bindings are introduced at kind level in \Fi. The consequence of
this is that, in typing contexts and in sorts, $A$ must be type constructor with no
type variables.

A term level context $\Gamma$ is well-formed under a type level context
$\Delta$ when it is either empty or extended by a term variable binding
$x:A$ whose type $A$ is well-kinded under $\Delta$.


\paragraph{Kinds and their sorting rules}
We extend the kind syntax of \Fw\ by indexed arrow kinds of the form
\newFi{A -> \kappa}. The formation of indexed arrow kinds is
governed by the sorting rule \newFi{(Ri)}. The rule $(Ri)$ specifies that
an indexed arrow kind $A -> \kappa$ is well-sorted when $A$ is well-kinded
under the empty type level context ($\cdot$) and $\kappa$ is well-sorted.
We avoid dependent kinds (\ie, kinds depending on type level or value level
bindings) by requiring $\cdot |- A$. The type $A$ appearing in
the index arrow kind $A -> \kappa$ must be well-kinded under
the empty type level context ($\cdot$), since no bindings are
introduced at kind level in \Fi.

\paragraph{Type constructors and their kinding rules}
We extend the type constructor syntax by three constructs,
and extend the kinding rules accordingly for these new constructs.

\newFi{\lambda i^A.F} is the type level abstraction over an index
(or, index abstraction). Index abstractions introduce indexed arrow kinds
by the kinding rule \newFi{(\lambda i)}. Note, we have a new form of binding
$i^A$ in the kinding rule ($\lambda i$).
We will explain what this binding means when we discuss contexts.

\newFi{F\,\{s\}} is the type level index application. In contrast to
the ordinary type level application ($F\,G$) whose argument being applied is
a type constructor ($G$), the argument of the index application ($F\,\{s\}$) is
a term ($s$). We use the curly bracket notation around the index argument to
emphasize the distinction from ordinary type arguments and also to emphasize
that $s$ is an index term, which is erasable. Index applications eliminate
indexed arrow kinds by the kinding rule \newFi{(@i)}. Note, we type check
the index term ($s$) under the current type level context paired with
the empty term level context ($\Delta;\cdot$) since we do not want
the index term ($s$) to depend on any term level bindings.

\newFi{\forall i^A . B} is an index polymorphic type.
The formation of indexed polymorphic types is governed by
the kinding rule \newFi{\forall i}, which is very similar to
the formation rule ($\forall$) for ordinary polymorphic types.

In addition to the rules ($\lambda i$), ($@ i$), and ($\forall i$),
we need a conversion rule \newFi{(Conv)} at kind level. This is because
the new extension to the kind syntax $A -> \kappa$ involves types.
Since kind syntax involves types, we need more than simple structural
equality over kinds. The equality over kinds is the usual structural equality
extended by type constructor equality when comparing indexed arrow kinds
(see \Fig{eqFi}).

\paragraph{Terms and their typing rules}
The term syntax is extended by index variables (\newFi{i}), since terms used
as indices may have index variables as well as term variables
(\eg, $\lambda i^A.F\{(\lambda x.x\;i)\,(\lambda x.x)\}$).
The term variables ($x$) are introduced from
term level abstractions ($\lambda x.t$).
The index variables ($i$) are introduced from
index abstractions ($\lambda i^A.F$) and
index polymorphic types ($\forall i^A.B$). However, the distinction between
$x$ and $i$ is in fact only a convention for the convenience of readability.
An equivalent but more succinct description of the system would be possible
by having only $x$ for both term and index variables instead of having two
kinds of variables $x$ and $i$.

Since \Fi\ has index polymorphic types ($\forall i^A . B$),
we need typing rules for index polymorphism:
\newFi{(\forall I i)} for index generalization
and \newFi{(\forall E i)} for index instantiation.

The index generalization rule ($\forall I i$) is similar to
the type generalization rule ($\forall I$), except the additional
side condition $\big(i\notin\FV(t)\big)$. This side condition prevents
terms from accessing the type level index variables introduced by index
polymorphism. Otherwise, without this side condition, $\forall$-binder
would be no longer behave as polymorphism but powerful enough to behave as
dependent functions, which are usually denoted by the $\Pi$-binder in
dependent type theories. The side condition on generalization rules
for polymorphism is fairly standard in dependently typed languages supporting
distinctions between polymorphism (or, erasable arguments) and
dependent functions (\eg, IPTS[TODO cite Nathans' thesis], ICC[TODO cite]).
The rule ($\forall I$) for ordinary type generalization rule does not need
a side condition because type variables cannot appear in the syntax of terms.

The index instantiation rule ($\forall E i$) is similar to
the type instantiation rule ($\forall E i$), except that
we type check the index term $s$ to be instantiated for $i$
in the current type level context paired with the empty term level context
($\Delta;\cdot$) rather than the current term level context.
Since index terms are at type level, they should not depend on
term level bindings.

In addition to the rules ($\forall I i$) and ($\forall E i$) for
index polymorphism, we need an additional variable rule \newFi{(:i)}
to be able to access the index variables already in scope. Terms ($s$) used
at type level in index applications ($F\{s\}$) should be able to access
index variables already in scope. For example, $\lambda i^A.F\{i\}$ should be
well-kinded under a context where $F$ is well-kinded,
justified by the derivation in Figure \ref{fig:ivarexample}.
\begin{figure*}
\[ \inference[($\lambda i$)]
      { \cdot |- A:* &
	\inference[($@i$)]{ \Delta, i^A |- F : A -> \kappa
                          & \inference[($:i$)]{ i^A\in \Delta,i^A
                                              & \Delta |- \cdot }
                                              {\Delta,i^A;\cdot |- i:A}
                          }
                          {\Delta, i^A |- F\{i\} : \kappa} }
      {\Delta |- \lambda i^A.F\{i\} :\kappa}
\]
\caption{Kinding derivation for an index abstraction}
\label{fig:ivarexample}
\end{figure*}






