\documentclass{article}
\usepackage{listings}
\usepackage{url}
\usepackage[fleqn]{amsmath}
\usepackage{amssymb}
\usepackage{semantic}
\usepackage{todonotes}
\usepackage{framed}

\definecolor{grey}{rgb}{0.8,0.8,0.8}

\newcommand{\Fi}{\ensuremath{\mathsf{F}_i}}
\newcommand{\Fw}{\ensuremath{\mathsf{F}_\omega}}

\newcommand{\dom}{\mathop{\mathsf{dom}}}
\newcommand{\FV}{\mathop{\mathrm{FV}}}

\newcommand{\newFi}[1]{\colorbox{grey}{\ensuremath{#1}}}

\newcommand{\email}[1]{\texttt{\small #1}}

\title{System \Fi: a higher-order polymorphic $\lambda$-calculus
	with erasable term indices \vspace*{-5pt}}

\author{
 $\begin{matrix}_\text{grad student}\\ ~ \\ ~ \end{matrix}\!\!\!\!\!\!\!\!$
 $\begin{matrix}\text{Ki Yung Ahn} \\ \email{kya@cs.pdx.edu} \\
   \text{\small ACM member no. 4262877}   \end{matrix}$
 $\!\!\!\!\begin{matrix}_\text{advised by}\\ ~ \\ ~ \end{matrix}$
 $\begin{matrix}\text{Tim Sheard} \\ \email{sheard@cs.pdx.edu} \\ ~ \end{matrix}
 \qquad$
 \vspace*{5pt} \\
 \small Department of Computer Science \\
 \small Portland State University \\ \small PO Box 751, Portland, Oregon, 97201
}

\date{\small \vspace*{-12pt}
% (An extended abstract submitted to ICFP 2012 Student Research Competition)
% \hrule
 }


\begin{document}
\maketitle \vspace*{-15pt}
%% Clearly state the problem being addressed
%% and explain the reasons for seeking a solution to this problem.
My goal is to design a calculus, as simple as possible, yet expressive enough
to embed non-dependent \emph{term-indexed datatypes} and their eliminators.
It is well known that datatypes can be embedded into polymorphic lambda calculi
(e.g., \cite{AbeMatUus03}).

In System \textsf{F}, we can embed \emph{regular datatypes},
such as homogeneous lists:\vspace*{1pt}\\
\fbox{$
\begin{array}{ll}
\text{Haskell:} & \texttt{data List a = Nil | Cons a (List a)} \\
\text{System \textsf{F}:} & \texttt{{List}}\:\: A\:\:\triangleq\:\:\forall X.X\to(A\to X\to X)\to X \qquad\qquad\qquad\qquad\quad
\end{array}$}\vspace*{1pt}\\
Note the use of the universally quantified type variable $X$
and the regularity of $(\texttt{List a})$ in the datatype definition.

In System \Fw, we can embed \emph{type-indexed datatypes}, which include
datatypes that are not regular. For example, we can embed powerlists with
heterogeneous elements where an element of type \texttt{a} is followed by
an element of type \texttt{(a,a)}:\vspace*{1pt}\\
\fbox{$
\begin{array}{ll}
\text{Haskell:} & \texttt{data Powl a = PNil | PCons a (Powl(a,a))} \\
\text{System \textsf{F}:} & \texttt{{Powl}}\:\:\triangleq\:\:
\lambda A^{*}.\forall X^{*\to*}.X\, A\to(A\to X(A\times A)\to X\, A)\to X\, A
\end{array}$}\vspace*{1pt}\\
Note the non-regular occurrence \texttt{(Powl(a,a))} and
the use of the type constructor variable $X$ universally quantifying over
type constructors of kind $* -> *$.\\
\begin{center}\vspace*{-14pt}
\textbf{A Motivating Example: length-indexed lists} \vspace*{-8pt}
\end{center}
\definecolor{shadecolor}{rgb}{1,0.9,0.7}
\begin{snugshade}\vspace{-7pt}
\begin{lstlisting}[basicstyle={\small\ttfamily},language=Haskell]
data Nat = Z | S n
data Vec (a:*) {i:N} where
  VNil  : Vec a {Z}
  VCons : a -> Vec a {i} -> Vec a {S i}
\end{lstlisting}\vspace*{-10pt}
\end{snugshade}\vspace*{-15pt}
\begin{snugshade}\noindent
$
\texttt{{Vec}}\:\:\triangleq\:\:\lambda A^{*}.\lambda i^{\texttt{{N}}}.
\forall X^{\texttt{{Nat}}\to*}.
X\{\texttt{{Z}}\}\to
(\forall i^{\texttt{{Nat}}}.A\to X\{i\}\to X\{\texttt{{S}}\; i\})\to X\{i\}
$
\end{snugshade}\vspace*{-5pt}
What extensions to \Fw\ do we need to embed datatypes that are indexed by
terms (\texttt{Z}, \texttt{i}, \texttt{S i}), as well as types (\texttt{a}),
such as length-indexed lists (\verb|Vec a {i}|)? From the motivating example
above, we learn that the calculus would need four additional constructs:
index arrow kinds ($\texttt{{Nat}\ensuremath{\to}*}$),
index abstraction ($\lambda i^{\texttt{{Nat}}}.\cdots$),
index application ($X\{i\}$), and
index polymorphism ($\forall i^{\texttt{{Nat}}}.\cdots$).

\begin{center}
\textbf{
 System \Fi ~= Curry-style System \Fw\ + \newFi{\text{\{erasable term indices\}}}}
 \vspace*{-7pt}
\end{center}
\begin{framed}\vspace{-6pt}\noindent
$
\text{Terms}~~ r,s,t ~ ::= ~ x \mid \lambda x.t \mid r\;s \mid \newFi{i}
\qquad
\text{Kinds}~~ \kappa ~ ::= ~ *
                                \mid \kappa -> \kappa
                                \mid \newFi{A -> \kappa}$ \vspace*{2pt} \\
$
\text{Type Constructors}~~~
        A,B,F,G               ~ ::= ~ X
                                \mid A -> B
                                \mid \lambda X^\kappa.F
                                \mid F\,G
                                \mid \forall X^\kappa . B$\\
$~\qquad\qquad\qquad\qquad\qquad\qquad\qquad\qquad\qquad\qquad\quad
\mid \newFi{\lambda i^A.F
                                \mid F\,\{s\}
                                \mid \forall i^A . B}$ \\
$\text{Typing Contexts}
\qquad \Delta                ~ ::= ~ \cdot
                                \mid \Delta, X^\kappa
                                \mid \newFi{\Delta, i^A}
\qquad\qquad \Gamma                ~ ::= ~ \cdot
                                \mid \Gamma, x : A 
$\vspace*{3pt}
\hrule ~ \\ \vspace*{-20pt}
\begin{align*}
 \!\!\!\!\!\!\!\!\text{Index Erasure}\quad
& \qquad (A\to\kappa)^{\circ}=\kappa^{\circ} \qquad \qquad
(\Delta,i^A)^\circ = \Delta^\circ
\\
& (\lambda i^{A}.F)^{\circ}=F^{\circ} \qquad
(F\,\{s\})^{\circ}\;=F^{\circ} \qquad
(\forall i^{A}.F)^{\circ}=F^{\circ} \qquad
\end{align*}\vspace*{-25pt}
\end{framed}\vspace*{-5pt}
We extend the syntax of the Curry-style \Fw\ based on the observations from
the motivating example. The constructs new to \Fi\ are highlighted grey.

The index erasure operation ($^\circ$) erases all the index related constructs
at type level, but does not erase anything from terms. The key erasure rules are
illustrated above, and all other rules left out are contextual rules that
delegates the erasure operation structurally. Since index erasure is completely
predetermined by the syntax, there is no need for any extra analysis
to determine erasable fragments. System \Fi\ enjoys a simple erasure property:
if $\Gamma;\Delta|- t:A$ then $\Delta^\circ;\Gamma^\circ|- t:A^\circ$
when $t$ is index free.
Moreover, if the original judgment $\Gamma;\Delta|- t:A$ is derivable in \Fi,
then the erased judgment $\Gamma^\circ;\Delta^\circ|- t:A^\circ$ should be
derivable in \Fw.

The key mechanism to ensure this erasure property comes from splitting
the typing context into two parts. When we go inside the curly braces
to type check a term index ($s$) in a type application ($F\,\{s\}$), we keep
the type-level context $\Delta$ but forget the term-level context $\Gamma$,
as in the kinding rule below: \vspace{1pt}\\
$~ \qquad ~ \qquad ~ \qquad ~ \qquad ~ \qquad ~ \qquad ~ \qquad ~ \qquad
\inference[($@i$)]{ \Delta |- F : A -> \kappa & \Delta;\cdot |- s : A }
                  {\Delta |- F\,\{s\} : \kappa}  
$\vspace*{3pt}

We proved that the index erasure property mentioned above. Strong normalization
is a corollary of the erasure property since we know that System \Fw\ is
strongly normalizing. Logical consistency is immediate since System \Fi\
is a strict subset of the \emph{restricted implicit calculus} \cite{Miquel00}.
\vspace{-3pt}
\begin{center}
\textbf{
 Conclusions and Future work
 \vspace*{-9pt}
}
\end{center}
My contributions are:\vspace*{-6pt}
\begin{itemize}
\item identifying the features needed for $\lambda$-calculi 
    to embed term-indexed datatypes, in isolation with other
    requirements (e.g., large elimination), \vspace*{-7pt}
\item design of a calculus useful for studying properties of
    term-indexed datatypes
    (e.g., we can use \Fi\ to give formal proof that the eliminators
    for GADTs appearing in our last year's ICFP paper \cite{AhnShe11} are
    indeed nomralizing), \vspace*{-7pt}
\item proof that the calculus enjoys a simple erasure property
    and inherits metatheoretic results from well-known calculi.
\end{itemize}\vspace*{-3pt}
We are exploring whether Leibniz equality over indices
(i.e., $s_1=s_2$ encoded as $\forall X^{A -> *}.X\{s_1\} -> X\{s_2\}$)
may help us express functions whose domains are restricted by term-indices
(e.g., \verb|vtail :: Vec a {S n} -> Vec a n|). We wonder what extension
we need to enable large eliminations (i.e., computing types from term-indices).
We are also developing a programming language Nax, which supports
type inference with little annotation, based on System \Fi.
% \paragraph{Acknowledgments:} Thanks to Tim Sheard, Andy Pitts, Marcelo Fiore,
% Stephanie Weirich for their enlightening suggestions and feedback.

\bibliographystyle{plain}
\bibliography{main}


\end{document}
