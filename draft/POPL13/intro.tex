\section{Introduction}
\label{sec:intro}

%% theoretical core language for languages supporting
%% non-regular datatypes (\eg, nested datatypes) and
%% indexed datatypes (\eg, GADTs), but not fully dependent types.
%% 
%% embed recursive datatypes by Church encoding and show strong normalization
%% for the logical fragment of such a language

%%%%%%\subsection{Motivation and development}

We wish to incorporate dependent types into ordinary programming
languages. We are interested in two kinds of dependent types.
Full dependency, where the type of a function can depend upon the value
of its run-time parameters, and static dependency, where the type
of a function can depend only upon static (or compile-time) parameters.
Static dependency is sometimes referred to as indexed typing. The first
is very expressive, while the second is often much easier to learn and 
use especially for those who are familiar to functional programming languages
like Haskell or ML.
Indexed types come in two flavors: type-indexed and term-indexed
types. Type indexing includes parametric polymorphism, but it also includes
more sophisticated typing as found in Generalized Algebraic Datatypes~(GADTs).
An example of type indexing using GADTs is a type representation:\vspace*{-2pt}
\begin{lstlisting}[basicstyle={\ttfamily\small},language=Haskell,mathescape]
data TypeRep t where
  Int :: TypeRep Int
  Bool:: TypeRep Bool
  Pair:: TypeRep a -> TypeRep b -> TypeRep(a,b)
\end{lstlisting}\vspace*{-2pt}
Here, a value of type {\tt (TypeRep t)} is isomorphic in ``shape" with
the type {\tt t}. For example {\tt (Pair Int Bool)} is isomorphic in shape
with its type {\tt (Int,Bool)}.

On the other hand, term-indexed types include indices that range over data
structures, such as Natural Numbers~(like {\tt Z}, {(\tt S
Z})) or Lists~(like {\tt Nil} or ({\tt Cons Z Nil})). 
The classic example of a term index is the second
parameter to the length-indexed list type {\tt Vec}~(as in ({\tt Vec Int
(S Z)})).  In languages such as Haskell, which support GADTs with type indexing,
term-indices are not first-class;
they are ``faked'' by reflecting data at the type level with uninhabited
type constructors (see \S\ref{sec:relwork} for a very recent GHC extension,
	which enable term-indices be first class).
For example,\vspace*{-2pt}
\begin{lstlisting}[basicstyle={\ttfamily\small},language=Haskell,mathescape]
data Succ n
data Zero
\end{lstlisting}\vspace*{-8pt}
\begin{lstlisting}[basicstyle={\ttfamily\small},language=Haskell,mathescape]
data Vector t n where
  Cons :: a -> Vector a n -> Vector a (Succ n)
  Nil  :: Vector a Zero
\end{lstlisting}\vspace*{-2pt}
This comes with a number of problems. First, there is no way to say
that types such as {\tt (Succ Int)} are ill-formed, and second the costs
associated with duplicating the constructor functions of data to be used
as term-indices.
Nevertheless, ``faked" term-indexed GADTs have become extremely
popular as a light-weight, type-based mechanism
to raise the confidence of users that software systems maintain important
properties.  

A salient example is Guillemette's thesis~\cite{guillemetteThesis}
encoding the classic paper by~\citet{tal-toplas} completely
in Haskell. This impressive system embeds a multi-stage compiler, from
System~F all the way to typed assembly language using indexed datatypes
(many of them ``faked" term-indices) to
show that every stage preserves type information.
%
As such, it provides confidence but no guarantees.  Indeed, since in
Haskell the non-terminating computation can be assigned any type, it is in
principle possible that the type-preservation property is a consequence of
a non-terminating computation in the program code.

This drawback is absent in approaches based on strongly normalizing
logical calculi; like, for instance, System~\Fw, the higher-order
polymorphic lambda calculus, which is rich enough to express a wide
collection of data structures.  Unfortunately, the \emph{term-indexed
datatypes} that are necessary to support Guillemette's system are not
known to be expressible in System~\Fw.

In his CompCert system, Leroy~\cite{Leroy-Compcert-CACM} showed that the
much richer logical Calculus of Inductive Constructions~(CIC), which
constitutes the basis of the Coq proof assistant, is expressive enough to
guarantee type preservation (and more) between compiler stages.  This
approach, however, comes at a cost.  Programmers must learn to use both
dependent types and a new programming paradigm, programming by code
extraction.

Some natural questions thus arise: Is there an expressive system
supporting term-indexed types, say, sitting somewhere in between
System~\Fw\ and fully dependent calculi? 
If only term-indexed types are needed to maintain properties
of interest, is there a language one can use?
Can one program, rather than extract code? 
The goal of this paper is to develop the theory necessary to begin
answering these and related questions.  

Our approach in this direction is to design a new foundational calculus,
System~\Fi, for functional programming languages with term-indexed
datatypes.  In a nutshell, System~\Fi\ is obtained by minimally extending
System~\Fw\ with type-indexed kinds.  Notably, this yields a logical
calculus that is expressive enough to embed non-dependent
\emph{term-indexed datatypes} and their eliminators. Our contributions in
this development are as follows.\vspace*{-2pt}
\begin{itemize}
\item 
  Identifying the features that are needed for a higher-order polymorphic
  $\lambda$-calculus to embed term-indexed datatypes~(\S\ref{sec:motiv}),
  in isolation from other features normally associated with such
  calculi~(e.g., general recursion, large elimination, dependent types).
  \vspace*{-1pt}
\item 
  The design of the calculus, System \Fi\ (\S\ref{sec:Fi}), and its use to
  study properties of languages with term-indexed datatypes, by embedding
  these into the calculus~(\S\ref{sec:data}).  For instance, one can use
  System~\Fi\ to prove that the Mendler-style eliminators
  for GADTs of~\cite{AhnShe11} are normalizing.
  \vspace*{-1pt}
\item 
  Showing that System~\Fi\ enjoys a simple erasure
  property %~(\S\ref{ssec:erasure}) 
  and inherits meta-theoretic
  results~(strong normalization and logical consistency) from well-known
  calculi~(System~\Fw\ and ICC) that enclose System~\Fi~%(\S\ref{ssec:sn}).
(\S\ref{sec:theory}).
\end{itemize}\vspace*{-3pt}

\section
%%%%%%\subsection
{From System~\Fw\ to System~\Fi, and back}%{Motivation}
\label{sec:motiv}

It is well known that datatypes can be embedded into polymorphic lambda
calculi by means of functional encodings~(e.g.,~\cite{AbeMatUus03}), such
as the Church and Boehm-Berarducci encodings.

In System~\textsf{F}, for instance, one can embed \emph{regular
datatypes}~\cite{BoehmBerarducci}, like homogeneous lists:
\[\!\!\!\!\!\!\!\!\!
\begin{array}{ll}
\text{Haskell:} & \texttt{data List a = Cons a (List a) | Nil} \\
\text{System \textsf{F}:} & 
\texttt{{List}}\:\: A\:\:\triangleq\:\:
\forall X.
(A\to X\to X) \to X \to X ~~\; 
\end{array}
\]
In such regular datatypes, constructors have algebraic structure that
directly translates into polymorphic operations on abstract types as
encapsulated by universal quantification.

In the more expressive System \Fw, one can encode more general
\emph{type-indexed datatypes} that go beyond the algebraic class.
For example, one can embed powerlists with
heterogeneous elements in which an element of type \texttt{a} is followed by
an element of the product type \texttt{(a,a)}:
\[
\begin{array}{ll}
\text{Haskell:} & \!\!\!\!\texttt{data Powl a = 
	PCons a (Powl(a,a))
	| 
	PNil 
} \\
\text{System \Fw:} & \!\!\!\!\texttt{{Powl}}\:\triangleq\:
\lambda A^{*}.\forall X^{*\to*}.\\ \qquad
& \qquad 
(A\to X(A\times A)\to X A)
\to 
X A
\to XA
\end{array}
\]
Note the non-regular occurrence (\texttt{Powl(a,a)}) in the definition of
(\texttt{Powl a}), and the use of universal quantification over
higher-order kinds.

%\begin{figure}\noindent
%\definecolor{shadecolor}{rgb}{1,0.9,0.7}
%\!\!\!\!\!A functional language supporting term-indexed datatypes: \vspace*{-4.5pt}
%\begin{lstlisting}[basicstyle={\ttfamily},language=Haskell]
%data Nat = Z | S n
%data Vec (a:*) {i:N} where
%  VNil  : Vec a {Z}
%  VCons : a -> Vec a {i} -> Vec a {S i}
%\end{lstlisting}\noindent
%\!\!\!System \Fi: \vspace*{-7pt}
%\begin{multline*}\!\!\!\!\!\!\!
%\texttt{{Vec}}\:\:\triangleq\:\:\lambda A^{*}.\lambda i^{\texttt{{N}}}.
%\forall X^{\texttt{{Nat}}\to*}.\\
%X\{\texttt{{Z}}\}\to
%(\forall i^{\texttt{{Nat}}}.A\to X\{i\}\to X\{\texttt{{S}}\; i\})\to X\{i\}
%\end{multline*}\vspace*{-10pt}
%\caption{A Motivating Example: length-indexed lists}
%\label{fig:motiv}
%\end{figure}

What about term-indexed datatypes?  What extension to System~\Fw\ is
needed to embed these, as well as type-indexed ones?  Our answer is
System~\Fi.

In a functional language supporting term-indexed datatypes, we envisage
that the classic example of homogeneous length-indexed lists would be
defined along the following lines:\vspace{-5pt}
\begin{lstlisting}[basicstyle={\ttfamily},language=Haskell]
 data Nat = S Nat | Z 

 data Vec (a:*) {i:Nat} where
   VCons : a -> Vec a {i} -> Vec a {S i}
   VNil  : Vec a {Z}
\end{lstlisting}~\vspace{-15pt}\\ \noindent
Here the type constructor~{\tt Vec} is defined to admit parameterisation
by both type and term indices.  For instance, the type 
(\verb|Vec (List Nat) {S (S Z)}|) is that of two-dimensional
vectors of lists of natural numbers.  By design, our syntax directly
reflects the different type and term indexing by indicating the latter in
curly brackets.  This feature has been directly transferred from
System~\Fi, where it is used as a mechanism for guaranteeing the static
nature of term indexing.

The encoding of the vector datatype in System~\Fi\ is as follows: 
\begin{equation*}\label{FiVecType}
\texttt{{Vec}}
\triangleq
\!\!\!
\begin{array}[t]{l}
\lambda A^\mathtt{*}.\lambda
i^{\texttt{{Nat}}}.  \forall X^{\texttt{{Nat}}\to\mathtt{*}}.\\[1mm]
\quad 
  (\forall j^{\texttt{{Nat}}}.A\to X\{j\}\to X\{\texttt{{S}}\; j\})
  \to X\{\texttt{{Z}}\}
    \to X\{i\}
\end{array}
\end{equation*}
where $\texttt{{Nat}}$, $\mathtt Z$, and $\mathtt S$ respectively encode
the Natural Numbers, zero and successor.
Without going into the details of the formalism, which are given in the
next section, one sees that such a calculus incorporating term-indexing
structure needs four additional constructs.
\begin{enumerate}
\item 
  Type-indexed kinding~($A\to\kappa$) (as in $(\texttt{{Nat}\ensuremath{\to}*})$
  in the example above) where the compile-time nature of term-indexing
  will be reflected in the enforcement that $A$ be a closed
  type~(rule~$(Ri)$ in \Fig{Fi}).

\item 
  Term-index abstraction~$\lambda i^A.F$~(as $\lambda
  i^{\texttt{{Nat}}}.\cdots$ in the example above) for constructing (or
  introducing) type-indexed kinds (rule~$(\lambda i)$ in
  \Fig{Fi}).  

\item 
  Term-index application~$F\{s\}$ (as $X\{{\tt Z}\}$, $X\{j\}$, and
  $X\{\texttt{S}\;j\}$ in the example above) for destructing (or
  eliminating) type-indexed kinds, where the compile-time nature of
  indexing will be reflected in the enforcement that the index be
  statically typed% in that it does not depend on run-time parameters
~(rule~$(@i)$ in \Fig{Fi}) .

\item 
  Term-index polymorphism~$\forall i^A.B$ (as $\forall
  j^{\texttt{{Nat}}}.\cdots$ in the example above) where the compile-time
  nature of polymorphic term-indexing will be reflected in the enforcement
  that the variable~$i$ be static of closed type~$A$~(rule~$(\forall
  Ii)$ in \Fig{Fi}).
\end{enumerate}

As exemplified above, System~\Fi\ maintains a clear-cut separation between
higher-order kinding and term indexing.  This adds a level of abstraction
to System~\Fw\ and yields types that in addition to structural invariants
also keep track of indexing invariants.  Being static, all term-index
information can be erased.  This projects System~\Fi\ into System~\Fw\
fixing the latter.  For instance, the erasure of the \Fi-type~\texttt{Vec}
is the \Fw-type~\texttt{List}, the erasure of which (when regarded as an
\Fi-type that is) is in turn itself.  Since, as already mentioned, typing
in System~\Fi\ imposes structural and indexing constraints on terms one
expects that the structural projection from System~\Fi\ to System~\Fw\
provided by index erasure preserves typing.  This is established in
\S\ref{sec:theory} and used to deduce the strong normalization of
System~\Fi.

