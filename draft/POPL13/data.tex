\section{Embedding datatypes and their eliminators} \label{sec:data}
We demonstrate some examples of embedding datatypes into System \Fi.
%% TODO cite some paper that does this with System Fw or System F
We first illustrate embeddings for both non-recursive datatypes and
recursive datatypes, where we use Church encodings \cite{Church33}
for data constructors (\S\ref{ssec:embedChurch}). Then, we illustrate
a more involved embedding for the recursive datatypes based on two-level types
(\S\ref{ssec:embedTwoLevel}).

\subsection{Embedding datatypes using Church-encoded terms}
\label{ssec:embedChurch}
\begin{figure}
\begin{align*}
&\!\!\!\!\!\!\mathtt{Bool} &=~& \forall X.X -> X -> X \\
&\!\!\!\!\!\!\mathtt{true}  &\!\!\!:~~& \texttt{Bool} ~~=~ \l x_1.\l x_2. x_1 \\
&\!\!\!\!\!\!\mathtt{false} &\!\!\!:~~& \texttt{Bool} ~~=~ \l x_1.\l x_2. x_2 \\
&\!\!\!\!\!\!\mathtt{elim_{Bool}} &\!\!\!:~~& \texttt{Bool} -> \forall X.X -> X -> X \\
&	&=~& \l x.\l x_1. \l x_2. x\;x_1\,x_2 \qquad
(\textbf{if}~x~\textbf{then}~x_1~\textbf{else}~x_2)
\end{align*}\vspace*{-19pt} \\ \vspace*{-4pt}
\rule{\linewidth}{.4pt}
\begin{align*}
&\!\!\!\!\!\!A_1\times A_2 &=~& \forall X. (A_1 -> A_2 -> X) -> X \\
&\!\!\!\!\!\!\mathtt{pair} &\!\!\!:~~&
	\forall A_1^{*}.\forall A_2^{*}.A_1\times A_2
	~~=~ \l x_1.\l x_2.\l x'.x'\,x_1\,x_2 \\
&\!\!\!\!\!\!\mathtt{elim_{(\times)}} &\!\!\!:~~&
	\forall A_1^{*}.\forall A_2^{*}.A_1\times A_2 ->
	\forall X. (A_1 -> A_2 -> X) -> X \\
	& &=~& \l x.\l x'.x\;x' \\
 &&&\!\!\!\!\!\!\!\!\text{(by passing appropriate values to $x'$, we get}\\
 &&&\!\!\!\!\textit{fst} = \l x.x(\l x_1.\l x_2.x_1),~
            \textit{snd} = \l x.x(\l x_1.\l x_2.x_2) ~)
\end{align*} \vspace*{-19pt} \\ \vspace*{-4pt}
\rule{\linewidth}{.4pt}
\begin{align*}
&\!\!\!\!\!\!A_1+A_2 &=~&\forall X^{*}. (A_1 -> X) -> (A_2 -> X) -> X \\
&\!\!\!\!\!\!\mathtt{inl} &\!\!\!:~~& \forall A_1^{*}.\forall A_2^{*}.A_1-> A_1+A_2
	~~=~ \l x. \l x_1. \l x_2 . x_1\,x \\
&\!\!\!\!\!\!\mathtt{inr} &\!\!\!:~~& \forall A_1^{*}.\forall A_2^{*}.A_2-> A_1+A_2
	~~=~ \l x. \l x_1. \l x_2 . x_2\,x \\
&\!\!\!\!\!\!\mathtt{elim_{(+)}} &\!\!\!:~~&
	\forall A_1^{*}.\forall A_2^{*}.(A_1+A_2) -> \\
	&&& \forall X^{*}. (A_1 -> X) -> (A_2 -> X) -> X \\
	& &=~& \l x.\l x_1. \l x_2. x\;x_1\,x_2 \\
	&&&			(\textbf{case}~x~\textbf{of}~
				\{\mathtt{inl}~x' -> x_1\;x';
				  \mathtt{inr}~x' -> x_2\;x'\})
\end{align*}~\vspace*{-10pt}
\caption{Embedding non-recursive datatypes}
\label{fig:churchnonrec}
\end{figure}
\begin{figure}
\begin{align*}
&\!\!\!\!\!\!\mathtt{List} &\!\!\!\!\!=~& \l A^{*}.\forall X^{*}.(A-> X-> X)-> X-> X
	\\
&\!\!\!\!\!\!\texttt{cons} &\!\!\!\!\!:~~& \forall A^{*}.A-> \mathtt{List}\,A-> \mathtt{List}\,A \\
& & & \qquad~\qquad~\quad\, =~\l x_a.\l x.\l x_c.\l x_n. x_c\,x_a\,(x\;x_c\,x_n) \\
&\!\!\!\!\!\!\mathtt{nil} &\!\!\!\!\!:~~& \forall A^{*}.\texttt{List}\,A
~~=~ \l x_c.\l x_n.\l x_n \\
&\!\!\!\!\!\!\mathtt{elim_{List}} &\!\!\!\!:~~& \forall A^{*}.\texttt{List}\,A ->
	\forall X^{*}.(A -> X -> X) -> X -> X \\
& &\!\!\!\!\!=~& \l x.\l x_c. \l x_n.x\;x_c\,x_n\qquad
	\text{(\textit{foldr} $x_z$ $x_c$ $x~$ in Haskell)}
\end{align*}\vspace*{-19pt} \\ \vspace*{-4pt}
\rule{\linewidth}{.4pt}
\begin{align*}
&\!\!\!\!\!\!\mathtt{Powl} &\!\!\!\!\!=~& \l A^{*}.\\
&&&\forall X^{*-> *}.(A-> X(A\times A)-> XA)-> XA -> XA \\
&\!\!\!\!\!\!\texttt{pcons} &\!\!\!\!\!:~~& \forall A^{*}.A-> \mathtt{Powl}(A\times A)-> \mathtt{Powl}\,A \\
&&& \qquad~\qquad~\quad\, ~=~ \l x_a.\l x.\l x_c.\l x_n. x_c\,x_a\,(x\;x_c\,x_n) \\
&\!\!\!\!\!\!\mathtt{pnil} &\!\!\!\!\!:~~& \forall A^{*}.\texttt{Powl}\,A
~~~=~ \l x_c.\l x_n.\l x_n \\
&\!\!\!\!\!\!\mathtt{elim_{Powl}} &\!\!\!\!:~~& \forall A^{*}.\texttt{Powl}\,A -> \\
&&& \forall X^{*-> *}.(A -> X(A\times A) -> XA) -> XA -> XA \\
& &\!\!\!\!\!=~& \l x.\l x_c. \l x_n.x\;x_c\,x_n
\end{align*}\vspace*{-19pt} \\ \vspace*{-4pt}
\rule{\linewidth}{.4pt}
\begin{align*}
&\!\!\!\!\!\!\mathtt{Vec} &\!\!\!\!\!\!\!\!=~& \l A^{*}.\l i^{\mathtt{Nat}}.\\
&&&	\forall X^{\mathtt{Nat}-> *}.
	(\forall i^\mathtt{Nat}.A-> X\{i\}-> X\{\mathtt{succ}\,i\}) ->  \\
&&& \qquad~\qquad X\{\texttt{zero}\} -> X\{i\} \\
 &\!\!\!\!\!\!\texttt{vcons} &\!\!\!\!\!\!\!\!:~& \forall A^{*}.\forall i^\mathtt{Nat}.A-> \mathtt{Vec}\,A\,\{i\}-> \mathtt{Vec}\,A\,\{\mathtt{succ}\,i\} \\
&&&\;\qquad\qquad\quad =~ \l x_a.\l x.\l x_c.\l x_n. x_c\,x_a\,(x\;x_c\,x_n) \\
&\!\!\!\!\!\!\mathtt{vnil} &\!\!\!\!\!\!\!\!:~& \forall A^{*}.\texttt{Vec}\,A\,\{\mathtt{zero}\} 
~~~=~ \l x_c.\l x_n.x_n \\
&\!\!\!\!\!\!\mathtt{elim_{Vec}} &\!\!\!\!\!\!\!\!:~& \forall A^{*}.\forall i^\mathtt{Nat}.\texttt{Vec}\,A\,\{i\} -> \\
&&& \forall X^{\mathtt{Nat}-> *}.(\forall i^\mathtt{Nat}.A -> X\{i\} -> X\{\mathtt{succ}\,i\}) -> \\
&&& \qquad~\qquad X\{\mathtt{zero}\} -> X\{i\} \\
& &\!\!\!\!\!=~& \l x.\l x_c. \l x_n.x\;x_c\,x_n
\end{align*} ~\vspace*{-14pt}
\caption{Embedding recursive datatypes}
\label{fig:churchrec}
\end{figure}
\citet{Church33} demonstrated an embedding of natural numbers into
the untyped $\lambda$-calculus, which he invented, in order to argue
that the $\lambda$-calculus  expressive enough for the foundation of
logic and arithmetic. Church encoded the data constructors of natural numbers,
successor and zero, as higher-order functions,
$\mathtt{succ}=\l x.\l x_s.\l x_z.x_s(x\,x_s x_z)$ and
$\mathtt{zero}=\l x_s.\l x_z.x_z$.
The heart of the Church encoding is that a value is encoded as its elimination.
The bound variables $x_s$ and $x_z$ stands for the operations needed for
eliminating the successor case and the zero case. The Church encodings of
successor and zero states that: to eliminate $\mathtt{succ}\,x$, apply $x_s$
to the elimination of the predecessor $(x\,x_s x_z)$; and,
to eliminate $\mathtt{zero}$, just return $x_z$.
Since values themselves are eliminators,
eliminator can be defined as applying the value itself to the needed operations
for each data constructor case. For instance, we can define an eliminator
for natural numbers as $\mathtt{elim_{Nat}}=\l x.\l x_s.\l x_z.x\,x_s x_z$,
which is just an $\eta$-expansion of the identity function $\l x.x$.
Church encoded natural numbers are typable in polymorphic $\lambda$-calculi,
such as System \Fw, as follows:\vspace*{-2pt}
\begin{align*}
&\texttt{Nat} &=~& \forall X^{*}.(X -> X) -> X -> X \qquad\qquad\qquad\\
&\texttt{succ} &\!\!\!:~~& \texttt{Nat} -> \texttt{Nat}
	~~ =~ \l x.\l x_s.\l x_z.x_s(x\,x_s x_z) \\
&\texttt{zero} &\!\!\!:~~& \texttt{Nat} \qquad\quad\,
	~~ =~ \l x_s.\l x_z.x_z \\
&\mathtt{elim_{Nat}} &\!\!\!:~~& \texttt{Nat} -> \forall X^{*}.(X -> X)-> X-> X \\
& &=~& \l x.\l x_s.\l x_z.x\,x_s x_z
\end{align*}~\vspace*{-13pt}

Similarly, other datatypes are also embeddable into
polymorphic $\lambda$-calculi in this fashion.
Embeddings of some well-known non-recursive datatypes are illustrated
in Figure \ref{fig:churchnonrec}, and embeddings of the list-like
recursive datatypes, which we discussed earlier as motivating examples
(\S\ref{sec:motiv}), are illustrated in Figure \ref{fig:churchrec}.
Note that the term encodings for the constructors and eliminators of
the list-like datatypes in Figure \ref{fig:churchrec} are exactly the same.
For instance, the term encodings for \texttt{nil}, \texttt{pnil}, and
\texttt{vnil} coincide as $\l x_s.\l x_z.x_z$.

\subsection{Embedding recursive datatypes as two-level types}
\label{ssec:embedTwoLevel}
We can divide a recursive datatype definition into two levels,
by factoring out the recursive type operator, which weaves in the recursion
to the datatype definition, and a non-recursive base structure, which describes
the shape (\ie, number of data constructors and their types) of the datatype.
We can program with two-level types in functional languages that support
higher-order polymorphism\footnote{a.k.a. higher-kinded polymorphism,
	or type-constructor polymorphism}, such as Haskell, as illustrated
in Figure \ref{fig:twoleveltypes}. The function $\mathtt{mit}_\kappa$ describes
the Mendler-style iteration\footnote{An iteration is a principled recursion
	scheme guaranteed to terminate for any well-founded input.
	Also known as fold, or catamorphism.} for the recursive types
defined by $\mu_\kappa$. The use of two-level types has been recognized as
a useful functional programming pearl \cite{Sheard04} since two-level types
separate the two concerns of (1) recursing deeper into recursive subcomponents
and (2) handling different cases by the shape of the base structure. 
Although it is possible to write programs using two level datatypes, one could
not expect logical consistency in such general purpose functional languages.

\begin{figure}\vspace*{-8pt}
\begin{lstlisting}[basicstyle={\ttfamily\small},language=Haskell,mathescape]
newtype $\mu_{*}$ (f :: * -> *)
  = In$_{*}$ (f ($\mu_{*}$ f))

data ListF (a::*) (r::*)
  = Cons a r       | Nil

type List a = $\mu_{*}$ (ListF a)
cons x xs = In$_{*}$ (Cons x xs)
nil       = In$_{*}$ Nil

mit$_{*}$ :: ($\forall$ r.(r->x) -> f r -> x) -> Mu0 f -> x
mit$_{*}$ phi (In$_{*}$ z) = phi (mit$_{*}$ phi) z

newtype $\mu_{(*-> *)}$ (f :: (*->*) -> (*->*)) (a::*)
  = In$_{(*-> *)}$ (f (Mu$_{(*-> *)}$ f)) a

data PowlF (r::*->*) (a::*)
  = PCons a (r(a,a)) | PNil

type Powl a = $\mu_{(*-> *)}$ PowlF a
pcons x xs = In$_{(*-> *)}$ (PCons x xs)
pnil       = In$_{(*-> *)}$ PNil

mit$_{(*-> *)}$ :: ($\forall$ r a.($\forall$a.r a->x a) -> f r a -> x a)
        -> $\mu_{(*-> *)}$ f a -> x a
mit$_{(*-> *)}$ phi (In$_{(*-> *)}$ z) = phi (mit$_{(*-> *)}$ phi) z

-- above is Haskell (with some GHC extensions)
-- below is Haskell-ish psudocode

newtype $\mu_{(\mathtt{Nat}-> *)}$ (f::(Nat->*)->(Nat->*)) {n::Nat}
  = In$_{(\mathtt{Nat}-> *)}$ (f ($\mu_{(\mathtt{Nat}-> *)}$ f)) {n}

data VecF (a::*) (r::Nat->*) {n::Nat} where
  VCons :: a -> r n -> VecF a r {S n}
  VNil  :: VecF a r {Z}

type Vec a {n::Nat} = $\mu_{(\mathtt{Nat}-> *)}$ (VecF a) {n}
vcons x xs = In$_{(\mathtt{Nat}-> *)}$ (VCons x xs)
vnil       = In$_{(\mathtt{Nat}-> *)}$ VNil

mit$_{(\mathtt{Nat}-> *)}$::($\forall$ r n.($\forall$n.r{n}->x{n})->f r {n}->x{n})
        -> $\mu_{(\mathtt{Nat}-> *)}$ f {n} -> x{n}
mit$_{(\mathtt{Nat}-> *)}$ phi (In$_{(\mathtt{Nat}-> *)}$ z) = phi (mit$_{(\mathtt{Nat}-> *)}$ phi) z
\end{lstlisting}
\caption{2-level types and their Mendler-style iterators in Haskell}
\label{fig:twoleveltypes}
\end{figure}

Interestingly, there exists an embedding of the recursive type operator
$\mu_\kappa$, its data constructor $\mathtt{In}_\kappa$, and
the Mendler-style iterator $\mathtt{mit}\kappa$ at each kind $\kappa$
into a higher-order polymorphic $\lambda$-calculus, as illustrated
in Figure \ref{fig:mu}. In addition to illustrating the general form of
embedding $\mu_\kappa$, we also fully expand the embeddings for some
instances ($\mu_{*}$, $\mu_{*->*}$, $\mu_{\mathtt{Nat}->*}$), which is
used in Figure \ref{fig:twoleveltypes}. This embedding allows arbitrary
type- and term-indexed recursive datatypes be embeddable into System \Fi,
so that we can reason about these datatypes in a logically consistent calculus.
However, it is important to note that there does not exist an embedding of
arbitrary destruction (or, pattern matching away) of the $\mathtt{In}_\kappa$
constructor. It is known that having arbitrary recursive datatypes together with
the ability to arbitrarily destruct (or, unroll) the values of recursive types
is powerful enough to define non-terminating computation in a type safe way,
which leads to logical inconsistency.

\begin{figure*}
\begin{multline*} \text{notation:}\quad
   \boldsymbol{\l}\mathbb{I}^\kappa.F =
	\lambda I_1^{\kappa_1}.\cdots.\lambda I_n^{\kappa_n}.F \qquad
   \boldsymbol{\forall}\mathbb{I}^\kappa.B =
	\forall I_1^{\kappa_1}.\cdots.\forall I_n^{\kappa_n}.B \qquad
   F\mathbb{I} = F I_1 \cdots I_n \qquad
   F \stackrel{\kappa}{\pmb{\pmb{->}}} G =
	\boldsymbol{\forall}\mathbb{I}^\kappa.F\mathbb{I} -> G\mathbb{I} \\
\begin{array}{lll}
\text{where}
 	& \kappa = \kappa_1 -> \cdots -> \kappa_n -> * & \text{and} ~~~
 	\text{$I_i$ is an index variable ($i_i$) when $\kappa_i$ is a type,}
 		\\
 	& \mathbb{I}\,=I_1,\;\dots\;\dots\;,\;I_n& \qquad~\qquad
 	\text{a type constructor variable ($X_i$) otherwise.}
\end{array}
\end{multline*} ~ \vspace*{-5pt}
\hrule  \vspace*{-2pt}
\begin{align*}
&\mu_\kappa &\!\!\!\!\!~:~~& (\kappa -> \kappa) -> \kappa
  \qquad\qquad\qquad\qquad\quad
  = \l F^{\kappa -> \kappa}.\boldsymbol{\l}\mathbb{I}^\kappa.
  \forall X^\kappa.
  (\forall {X_r}^{\!\!\kappa}.
  	(X_r \stackrel{\kappa\;}{\pmb{\pmb{->}}} X) ->
	(F X_r \stackrel{\kappa\;}{\pmb{\pmb{->}}} X)) -> X\mathbb{I} \\
&\mu_{*} &\!\!\!\!\!~:~~& (* -> *) -> * 
 \qquad\qquad\qquad\qquad\quad~
 = \l F^{* -> *}.\phantom{\boldsymbol{\l}\mathbb{I}^\kappa.}
 \forall X^{*}.(\forall {X_r}^{\!\!*}.(X_r -> X) -> (F\,X_r -> X)) -> X \\
&\mu_{*-> *} &\!\!\!\!\!~:~~& ((* -> *) -> (* -> *)) -> (* -> *) \\
&            &\!\!\!\!\!=~&
  \l F^{(*-> *) -> (*-> *)}.\l X_1^{*}.
   \forall X^{* -> *}.(\forall {X_r}^{\!\!* -> *}.
   (\forall X_1^{*}.X_r X_1 -> X X_1) -> (\forall X_1^{*}.F\,X_r X_1 -> X X_1)) -> X X_1 \\
  &\mu_{\mathtt{Nat}-> *} &\!\!\!\!\!~:~~& ((\mathtt{Nat} -> *) -> (\mathtt{Nat} -> *)) -> (\mathtt{Nat} -> *) \\
&            &\!\!\!\!\!=~&
  \l F^{(\mathtt{Nat}-> *) -> (\mathtt{Nat}-> *)}.\l i_1^\mathtt{Nat}.
  \forall X^{\mathtt{Nat} -> *}.(\forall {X_r}^{\!\!\mathtt{Nat} -> *}.
  (\forall i_1^\mathtt{Nat}.X_r i_1 -> X i_1) -> (\forall i_1^\mathtt{Nat}.F\,X_r i_1 -> X i_1)) -> X i_1 \qquad\qquad
\end{align*}
\begin{align*}
\mathtt{In}_{\kappa} \,~\,&~~:~ \forall F^{\kappa-> \kappa}.
	F(\mu_\kappa F) \stackrel{\kappa\;}{\pmb{\pmb{->}}} \mu_\kappa F
&&=~ \l x_r. \l x_\varphi.x_\varphi\,(\mathtt{mit}_\kappa~x_\varphi)\,x_r
	\qquad~\qquad~\qquad~\qquad~\quad \\
\mathtt{mit}_\kappa &~~:~ \forall F^{\kappa-> \kappa}.\forall X^\kappa.
	(\forall {X_r}^{\!\!\kappa}.
	 (X_r \stackrel{\kappa\;}{\pmb{\pmb{->}}} X) ->
	 (F X_r \stackrel{\kappa\;}{\pmb{\pmb{->}}} X) ) ->
	(\mu_\kappa F \stackrel{\kappa\;}{\pmb{\pmb{->}}} X)
&&=~ \l x_\varphi.\l x_r.x_r\,x_\varphi
\end{align*} ~ \vspace*{-10pt}
\caption{Embedding of the recursive operators ($\mu_\kappa$),
	their data constructors ($\mathtt{In}_\kappa$),
	and the Mendler-style iterators ($\mathtt{mit}_\kappa$).}
\label{fig:mu}
\end{figure*}

\subsection{Leibniz index equality}
\label{Leibniz}

The quantification over type-indexed kinding available in System~\Fi\
allows the definition of \emph{Leibniz-equality type} constructors
$\Eq_A: A\to A\to *$ on closed types~$A$.  These are as follows
\[
\Eq_A 
\triangleq
\l i^A.\, \l j^A.\, 
  \LEq_A\s i \s j \times \LEq_A\s j \s i 
\]
where
\[
\LEq_A 
\triangleq
\l i^A.\, \l j^A.\, \forall X^{A\to*}.\, X\{i\}\to X\{j\}
\enspace.
\]
Note that $\Delta\vdash F \s s \s t = F\s{s'}\s{t'}:*$ for $F=\Eq_A,\LEq_A$
and all $\Delta;\cdot\vdash s=s':A$, $\Delta;\cdot\vdash t=t':A$.

As a basic property, one has that the types 
\[\begin{array}{l}
\forall i^A.\,\LEq_A\s{i}\s{i}
\\[1mm]
\forall i^A.\,\forall j^A.\,\forall k^A.\,
  \LEq_A\s{i}\s{j}\to\LEq_A\s{j}\s{k}\to\LEq_A\s{i}\s{k}
\\[1mm]
\forall i^A.\,\forall j^A.\,
  \LEq_A\s{i}\s{j}\to \forall f^{A\to B}.\, \LEq_B\s{f\,i}\s{f\,j}
\end{array}\]
are inhabited.  Hence Leibniz equality is a congruence, in that the types
\[\begin{array}{l}
\forall i^A.\,\Eq_A\s{i}\s{i}
\\[1mm]
\forall i^A.\,\forall j^A.\,
  \Eq_A\s{i}\s{j}\to\Eq_A\s{j}\s{i}
\\[1mm]
\forall i^A.\,\forall j^A.\,\forall k^A.\,
  \Eq_A\s{i}\s{j}\to\Eq_A\s{j}\s{k}\to\Eq_A\s{i}\s{k}
\\[1mm]
\forall i^A.\,\forall j^A.\,
  \Eq_A\s{i}\s{j}\to
  \forall f^{A\to B}.\,
  \Eq_B\s{f\,i}\s{f\,j}
\end{array}\]
are inhabited.

In applications, the types~$\LEq_A$ are useful in constraining the
term-indexing of datatypes as parameterised by coercions.  A general such
construction is given by the type constructors $\Ext_{A,B}: (A\to B) \to
(A\to*) \to B\to *$ (in spirit right Kan extensions, see
\eg~\cite{AbeMatUus03}) defined as 
\[
\Ext_{A,B}
\triangleq
\l f^{A\to B}.\,
  \l X^{A\to*}.\,
    \l j^B.
      \forall i^A.\,
        \LEq_B \s j \s{f\,i}
	  \to X\s i
\]
for closed types~$A$ and $B$.  
%
It follows that, %Here, 
for closed $t:A\to B$, $F:A\to*$, and $s:B$, a closed term $u:
(\Ext_{A,B}\ \s t\ F) \s s$ is a polymorphic function that, for every
closed $r:A$, given a generic coercion $\forall X^{B\to*}.\, X\s s \to
X\s{t\,r}$ provides output of type $F\s r$.  In fact, %particular, 
$\forall f^{A\to B}.\,\forall X^{A\to*}.\,\forall i^A.\,
(\Ext_{A,B}\ \s f\ X) \s{f\,i}\to X\s i$ is inhabited by 
$\l f.\,f(\l x.\,x)$.
%
%(We note the interesting fact that the type 
%$\forall X^{A\to*}.\,\forall j^A.\,\forall i^A.\, 
%   X\s i \to (\Ext_{A,A}\ \s{\lambda x.\,x}\ X) \s i$
%is inhabited by a retraction.)

\begin{quote}
{\bf\em
\noindent\hrulefill begin cut \hrulefill
\begin{center}
THIS IS UNFORTUNATELY FLAWED
\end{center}

As a concrete example to which the same principle applies, consider the
type constructor
\[
\mathtt{Fin}\triangleq
\!\!\!
\begin{array}[t]{l}
\l i^{\mathtt{Nat}}.\,\forall X^{\mathtt{Nat}\to*}.\,
\\[1mm]
	\quad (\forall j^\mathtt{Nat}.\,\LEq_\mathtt{Nat}\s i\s{\mathtt
		S\,j}\to X\s i)
\\[1mm]
	\quad\quad \to (\forall j^\mathtt{Nat}.\, X\s j\to X\s{\mathtt S\,j})
\\[1mm]
	\quad\quad\quad\to X\s i
\end{array}
\]
For it, one derives constructors
\[\begin{array}{lcl}
\l x.\,\l x_0.\, \l x_1.\, x_0\, x
& : & 
  \forall i^{\mathtt{Nat}}\,\forall j^{\mathtt{Nat}}.\,
    \LEq_{\mathtt{Nat}}\s i \s{\mathtt S\,j} \to \mathtt{Fin}\s i
\\[1mm]
!!!???
& : & 
  \forall i^{\mathtt{Nat}}.\,
    \mathtt{Fin}\s i \to \mathtt{Fin}\s{\mathtt S\, i}
\end{array}\]
so that 
\[\begin{array}{l}
\l x_0.\, \l x_1.\, x_0\, (\l x.\, x)
: \mathtt{Fin}\s{\mathtt S\, t}
\\[1mm]
!!!???
: \mathtt{Fin}\s t \to \mathtt{Fin}\s{\mathtt S\, t}
\end{array}\]
for all closed $t:\mathtt{Nat}$.

\noindent\hrulefill end cut \hrulefill
}
\end{quote}

\begin{quote}
{\bf\em
\noindent\hrulefill begin cut \hrulefill
\begin{center}
WHERE SHOULD THIS GO, IF ANYWHERE, NOW???
\end{center}

The datatype of \mbox{$\lambda$-terms} in context 
\begin{verbatim}
data Lam ( C: Nat -> * ) { i: Nat } where
  LVar : C{i} -> Lam{i}
  LApp : Lam{i} -> Lam{i} -> Lam{i}
  LAbs : Lam{S i} -> Lam{i}
\end{verbatim}
is encoded as:
\[
\mathtt{Lam} \triangleq
\!\!\!
\begin{array}[t]{l}
\l C^{\mathtt{Nat}\to*}
\l i^\mathtt{Nat}.\,\forall X^{\mathtt{Nat}\to*}.
\\[1mm]
\quad
  (\forall j^\mathtt{Nat}.\,C\s j \to X\s j)
\\[1mm]
\quad\quad
 \to(\forall j^\mathtt{Nat}.\,X\s j \to X\s j \to X\s j)
\\[1mm]
\quad\quad\quad
\to(\forall j^\mathtt{Nat}.\,X\s{\mathtt S\, j} \to X\s j)
\\[1mm]
\quad\quad\quad\quad
  \to X\s i
\end{array}
\]

\noindent\hrulefill end cut \hrulefill
}
\end{quote}
