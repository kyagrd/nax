\section{Related work}
\label{sec:relwork}
Among theoretical calculi, System~\Fi\ is most closely related to
Curry-style System~\Fw \cite{AbeMatUus03,AbeMatUus05,GHR93}
and Implicit Calculus of Constructions (ICC) \cite{Miquel01}.
All terms typable in Curry-style System \Fw\ are typable in System \Fi\ 
with the same type, and all terms typable in \Fi\ are typable in ICC
with the same type.\footnote{The $*$ kind in \Fw\ and \Fi\ corresponds
	to \textsf{Set} in ICC.}
We have discussed that we can derive strong normalization, logical consistency,
and subject reduction of System~\Fi, from System~\Fw\ and a subset of ICC.
In fact, ICC is more than just an extension of System~\Fi, 
%as described in our work, 
with dependent types and stratified universes. ICC includes
$\eta$-reduction
and the extensionality typing rule. We do not foresee any problem of adding
$\eta$-reduction and the extensionality typing rule to System~\Fi. Although
System~\Fi\ accepts less terms than ICC, it enjoys simpler
erasure properties (Theorem \ref{thm:ierasetypingifree} and
Theorem \ref{thm:ierasetypingall}), which ICC cannot provide
due to its support for full dependent types. In System \Fi, index terms
appearing in types (\eg,~$s$ in $F\{s\}$) are always erasable.
\citet{LingerS08} formalized a more generic framework than ICC, which describes
the erasure on arbitrary Church-style calculi~(EPTS) and Curry-style
calculi~(IPTS), but only consider $\beta$-equivalence for type conversion.

In~\S\ref{ssec:rationale}, we have mentioned that Curry-style calculi enjoy
better reduction properties (\eg,$\beta\eta$-reduction is Church-Rosser)
than Church-style calculi. \citet{Nederpelt73} showed a counterexample to
the Church-Rosser property for $\beta\eta$-reduction of Church-style terms.
\citet{Geuvers92} proved that $\beta\eta$-reduction is Church-Rosser
in functional PTSs, which is a certain class of Church-style calculi.
\citet{Seldin08} discusses the relation between the Church-style typing
and the Curry-style typing.

In a more practical setting aimed towards language implementation,
\citet{YorgeyWCJVM12}, inspired by \citet{SHE}, designed a language extension
to Haskell, promoting datatypes to be used as kinds. For instance, \texttt{Bool}
is promoted to a kind (\ie, $\texttt{Bool}:\square$) and its data constructors
\texttt{True} and \texttt{False} are promoted to type level. To support this
in GHC, they extended System $F_{\!C}$ (GHC's intermediate language, or,
GHC Core), naming the extended GHC Core as System $F_{\!C}^\uparrow$.
The key difference between $F_{\!C}^\uparrow$ and \Fi\ is in the extension
to the kind syntax, as illustrated below: \vspace*{-2pt}
\[\qquad\quad
\begin{array}{ll}
F_{\!C}^\uparrow\,\textbf{kinds}: &
\kappa ::= * \mid \kappa -> \kappa \mid F \vec{\kappa} \mid \mathcal{X} \mid \forall \mathcal{X}.\kappa \mid \cdots \\
\,\Fi\,\,\textbf{kinds}: &
\kappa ::= * \mid \kappa -> \kappa \mid A -> \kappa \phantom{A^{A^A}}
\end{array}  
\] ~\vspace*{-6pt}\\
In $F_{\!C}^\uparrow$, any type constructor ($F$) is promoted to 
kind level and becomes a kind when fully applied to other kinds
($F\vec\kappa$). In \Fi, on the other hand, a type can only appear
at the domain of an index arrow kind ($A-> \kappa$). This seemingly small
difference to the kind syntax makes $F_{\!C}^\uparrow$ to develop into
a drastically more expressive language than \Fi. The promotion of
a type constructor, for instance, $\texttt{List}:* -> *$ to a kind constructor
$\texttt{List}:\square-> \square$ enables type-level data structures
such as $\mathtt{[Nat,Bool,Nat-> Bool]:List\,*}$. Having type-level
data structure motivates type-level computation over promoted data,
which they make it possible by extending the type families\footnote{
	A GHC extension to define type-level functions in Haskell.}.
In addition, the promotion of polymorphic types naturally motivates
kind polymorphism ($\forall \mathcal{X}.\kappa$), which is known to
break strong normalization and cause logical inconsistency \cite{Girard72}.
For the purpose of extending a functional programming language,
inconsistency is not an issue. However, for our interest of studying
term-indexed datatypes in a logically consistent calculi, we need
a more conservative approach, as in System~\Fi, starting from smallest possible
extension that maintains normalization and consistency.

Literature on handling type equality constraints in the type systems supporting
GADTs is vast. We list a few of them. System~$F_{\!C}$~\cite{Sulzmann07} would
arguably be the most influential system, being the intermediate language of
the Glasgow Haskell Compiler (GHC). \citet{VytWei10} proved parametricity of
System $\mathrm{R}_\omega$ \cite{Crary98}, an extension to Curry-style
System~\Fw
with the type-representation datatype and its primitive recursor, so that
one may derive \emph{free theorems} \cite{Wadler89free} in the presence of
type equalities.

Another possible approach for utilizing term-indexed datatypes for lightweight
verified programming is to restrict from full-spectrum dependent types, rather
than to extend from non-de\-pend\-ent types. \citet{Swamy11} developed $F^{*}$,
a language for secure distributed programming with value dependent types.
Terms appearing in dependent types in $F^{*}$ are restricted to first-order
values, similarly to value restriction in ML type inference, but even more
limited since function values are not allowed. By taming the dependent type
with such a restriction, they were able to have a usable programming language
and self-certify~\cite{Strub12} their compiler by implementing $F^{*}$
type checker in $F^{*}$.

