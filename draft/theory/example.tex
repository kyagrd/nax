
\section{Nax by Example}\label{sec:bg}

In this section we introduce programming in our implementation of Nax by providing
examples. An example usually consists of several parts.

\begin{itemize}
\item Introducing data definitions to describe the data of interest. Recursive
data is introduced in two stages. We must be careful to separate parameters
from indices when using indices to describe static properties of data.

\item Introduce macros, either by explicit defintion, or by automatic derivation
to limit the amount of explicit notation that must be supplied by the programmer.

\item Write a series of definitions that describe how the data is to be
manipulated. Deconstruction of recursive data can only be performed with Mendler style operators
to ensure strong normalization.

\end{itemize}

\subsection{Two level types}
Non recurisive datatypes are introduced by the \verb+data+ declaration.
The data declaration can include arguments. The kind and separation of
arguments into parameters and a indices is infered. For example, 
the three no recursive data types, \verb+Bool+, \verb+Either+, and 
\verb+Maybe+, familiar to many functional programmers, are introduced
in a manner almost identical manner to traditional functional languages.
{\small
\begin{verbatim}   
data Bool = False | True
data Either a b = Left a | Right b
data Maybe a = Nothing | Just a
\end{verbatim}}
Kind inference infers the following kinds:
(\verb+Bool: *+),  (\verb+Either: * -> * -> *+), and 
(\verb+Maybe: * -> *+), where \verb+*+ is the kind of ordinary types that
classify values.

To introduce a recursive type, we first introduce a non recursive datatype
that uses a parameter where the usual recursive components occur. By design,
normal parameters of the introduced type are written first (\verb+a+ in \verb+L+)
and the parameter to stand for the recursive component is written last
(the \verb+r+ of \verb+Nat+, and the \verb+r+ of \verb+L+).
{\small
\begin{verbatim}
data N r = Zero | Succ r  
data L a r = Nil | Cons a r       
\end{verbatim}}

Kind inference infers the following kinds:
(\verb+N: * -> *+),  (\verb+L: * -> * -> *+). 
The recursive type is just the fixpoint of a (perhaps partially applied)
non recursive type. Thus the traditional natural numbers are typed by \verb+Mu[*] N+
and the traditional lists with components of type \verb+a+ are typed by
\verb+Mu[*] (L a)+.  Note that the \verb+Mu+ type operator is itself specialized with a kind argument inside
square brackets (\verb+[ _ ]+). The fixpoint operator (\verb+Mu[k] f+) is well formed only
if the operand \verb+f+ has kind \verb+k -> *+. 

We may codify this once and for all by introducing
a type synonym (macro).

{\small
\begin{verbatim}
synonym Nat = Mu[*] N
synonym List a = Mu[*] (L a)     
\end{verbatim}}
Note that both \verb+N+ and \verb+(L a)+
have kind \verb+*+, so both of our synonyms are well formed.

\subsection{Creating values}

Values of a particular data type are created by use of constructor functions. For
example \verb+True+ and \verb+False+ are constructor (constants) of type \verb+Bool+.
(\verb+Left 4+) is a value of type (\verb+Either Int a+).


Non recursive
data

\subsubsection{Constructor functions, macros, and macro derivation}

\subsection{Mendler operators for non-indexed types}

\subsection{Types with static indices}\label{sec:bg:ixty}


\subsection{Mendler operators for indexed types}

A representative example of a term
indexed type is the length indexed list type (often called the vector type).
A regular polymorphic list type ($\sfList\;a$) is parametrized by a type
parameter ($a$), which can be instantiated to a specific type, while the
vector type ($\sfVec\;a\;\{n\}$) has an additional term index ($n$), which
can be instantiated to a specific natural number value. The curly brackets
($\{\cdots\}$) around $n$ is to syntactically distinguish term indices from
other type arguments. 

\subsection{Recursive types of unrestricted polarity but restricted elimination}
\label{sec:bg:recty}
TODO

\subsection{Mendler style iteration and recursion combinators}
\label{sec:bg:Mendler}


\section{Examples}\label{sec:examples}

