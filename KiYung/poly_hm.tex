\section{The Hindley-Milner type system} \label{sec:hm}

\citet{Hindley69} discovered that there exists a unique principal type scheme
for an object in a combinatory logic. \citet{Milner78} rediscovered this
in the setting of a polymorphic lambda calculus, while he was devising
an algorithm, called the algorithm $W$, which infers a most general
type scheme (\aka\ principal type scheme) for a Curry-style term.
\citet{Damas85} developed detailed theories on Milner's polymorphic
lambda calculus and the type inference algorithm $W$. The type system
for the Milner's polymorphic lambda calculus \cite{Milner78,DamMil82,Damas85}
is also known as the Hindley-Milner type system (HM),
Damas-Hindley-Milner type system (DHM), or let-polymorphism.

The Hindley-Milner type system (HM) is illustrated in Figure \ref{fig:hm}.
\begin{figure}
\begin{singlespace}
\small
\begin{align*}
&\textbf{Term}&
t,s&~::= ~ x          
    ~  | ~ \l x    . t 
    ~  | ~ t ~ s       
    ~  | ~ \<let> x=s \<in> t
\\
&\textbf{Type}&
A,B&~::= ~ A -> B
    ~  | ~ \iota
    ~  | ~ X
\\
&\textbf{Type scheme}&
\sigma&~::= ~ \forall X.\sigma
       ~  | ~ A
\\
&\textbf{Typing context}&
\Gamma&~::= ~ \cdot 
       ~  | ~ \Gamma, x:\sigma \quad (x\notin \dom(\Gamma))
\end{align*}
\[ \textbf{Type scheme ordering} \quad \framebox{$\sigma \sqsubseteq \sigma'$}\]
\[ \inference{X_1',\dots,X_m'\notin\FV(\forall X_1\dots X_n.A)}
             {\forall X_1\dots X_n.A \;\sqsubseteq\;
	      \forall X_1'\dots X_m'.\,A[B_1/X_1]\cdots[B_n/X_n]} \]
$\!\!\!\!\!\!\!\!\!\!$
\begin{align*}
&\textbf{Delcarative typing rules}&\quad
&\textbf{Syntax-directed typing rules}
	\\
& \qquad\framebox{$\Gamma |- t : \sigma$}
&
&~\qquad\framebox{$\Gamma |-s t : A$}
	\\
& \inference[\sc Var]{x:\sigma \in \Gamma}{\Gamma |- x:\sigma} &
& \inference[\sc Var$_s$]{x:\sigma \in \Gamma & \sigma \sqsubseteq A}
 	                 {\Gamma |-s x:A} \\
& \inference[\sc Abs]{\Gamma,x:A |- t : B}{\Gamma |- \l x   .t : A -> B} &
& \inference[\sc Abs$_s$]{\Gamma,x:A |-s t:B}{\Gamma |-s \l x   .t : A -> B} \\
& \inference[\sc App]{\Gamma |- t : A -> B & \Gamma |- s : A}
		     {\Gamma |- t~s : B} &
& \inference[\sc App$_s$]{\Gamma |-s t : A -> B & \Gamma |-s s : A}
		         {\Gamma |-s t~s : B} \\
& \inference[\sc Let]{\Gamma |- s : \sigma & \Gamma,x:\sigma |- t : B}
		     {\Gamma |- \<let> x=s \<in> t : B} &
& \inference[\sc Let$_s$]
            {\Gamma |-s s : A & \Gamma,x:\overline{\Gamma}(A) |-s t : B}
	    {\Gamma |-s \<let> x=s \<in> t : B} \\
& \inference[\sc Inst]{\Gamma |- t : \sigma & \sigma \sqsubseteq \sigma'}
		      {\Gamma |- t : \sigma'} &
&\quad\qquad \begin{smallmatrix}\overline{\Gamma}(A)=\forall\vec{X}.A&
			 ~\text{where}~\vec{X}=\FV(A)\setminus\FV(\Gamma)
		 \end{smallmatrix}
		 \\
& \inference[\sc Gen]{\Gamma |- s : \sigma & X \notin\FV(\Gamma)}
		     {\Gamma |- t : \forall X.\sigma}
\end{align*}
\end{singlespace}
\caption{Milner's polymorphic lambda calculus}
\label{fig:hm}
\end{figure}



Terms include the usual Curry-style terms ($x$, $\l x.t$, and $t\;s$)
and the let-terms ($\<let> x=s \<in> t$). A let term $\<let> x=s \<in> t$
may be considered as a syntactic sugar for $(\l x.t)\,s$
while reduction.\footnote{The reduction rules for the terms are exactly
	the same as the reduction rules for Curry-style terms
	in the previous sections, treating the let-term as a syntactic sugar.}
However, let-terms does make a significance in the typing rules
since a let-term can introduce polymorphic type schemes to the typing context
(\rulename{Let}). We will discuss further details on the typing of let-term
later on, when we explain the typing rules.

Types (or, monotypes) include the types of the STLC ($A -> B$ and $\iota$)
and type variables ($X$). Type schemes (or, polytypes) are like
the (polymorphic) types of System \F, but universal quantifications
are restricted to top level. Type schemes are either universal quantifications
over type schemes or types. Note, it is only possible to quantify
over type schemes but not types. Typing contexts ($\Gamma$) keep track of
each bound variable and its type scheme ($x:\sigma$).

The ordering between two type schemes $\sigma \sqsubseteq \sigma'$,
defined in Figure \ref{fig:hm}, means that $\sigma$ is more general
than or equivalent to $\sigma'$. The ordering relation $\sqsubseteq$
comes from \citet{DamMil82}, which is also known as generic instantiation:
$\sigma'$ is called a generic instance of $\sigma$
when $\sigma \sqsubseteq \sigma'$. The shorthand notation
$\forall X_1\dots X_n.A$ stands for consecutive universal quantification
of $n$ variables. For instance, $\forall X_1\,X_2\,X_3.A$
is a shorthand for $\forall X_1.\forall X_2.\forall X_3.A$.

The two type schemes $\sigma$ and $\sigma'$ are equivalent
(\ie, $\sigma \sqsubseteq \sigma'$ and $\sigma' \sqsubseteq \sigma$)
when they are $\alpha$-equivalent (\eg, $\forall X.X -> X$ is
equivalent to $\forall X'. X' -> X$). In fact, $\alpha$-equivalence is
a special case of the type scheme ordering rule,
where $n=m$ and $B_i=X_i'$ for each $i$ from $1$ to $n$.

The type scheme ordering rule also enables instantiation of type variables
in type schemes. For instance,
$ \forall X_1\,X_2.X_1->X_2 \sqsubseteq
  \forall X_2.\iota\to X_2 \sqsubseteq \iota\to\iota $.
In such cases of $\sigma\sqsubseteq\sigma'$, we can call $\sigma'$ an instance,
as well as generic instance, of $\sigma$. For example, $\iota\to\iota$ is
an instance of $\forall X_2.\iota\to X_2$ and $\forall X_1\,X_2.X->X_2$, and,
$\iota\to\iota$ and $\forall X_2.\iota\to X_2$ are instances of
$\forall X_1\,X_2.X_1->X_2$.

More generally, we say $\sigma'$ is a generic instance of $\sigma$
when $\sigma \sqsubseteq \sigma'$, since the relation $\sqsubseteq$
is more than $\alpha$-equivalence and initiation.
The type scheme ordering rule allows quantising newly introduced
variables in $\sigma'$, which do no appear free in $\sigma$.
For example,
consider the two generic instances of $\forall X.X -> X$ below:
\begin{align*}
\forall X.X -> X \sqsubseteq\;\,& (X'-> X')-> (X'-> X') \\
\forall X.X -> X \sqsubseteq\;\,& \forall X'.(X'-> X')-> (X'-> X')
\end{align*}
The former, $(X'-> X')-> (X'-> X')$, is an instance of $\forall X.X -> X$
instantiating $X$ to $(X'-> X')$. However, the latter,
$\forall X'.(X'-> X')-> (X'-> X')$, is not an instance
but a generic instance of $\forall X.X-> X$ because
the newly introduced variable $X'$ is universally quantified.

Readers who are not familiar with HM may wonder what is different between
$(X'-> X')-> (X'-> X')$ and $\forall X'.(X'-> X')-> (X'-> X')$.
The difference is that a higher order function of a monomorphic type
$(X'-> X')-> (X'-> X')$ can only be applied to functions of the same type
in one program, but a higher order function of a polymorphic type scheme
$\forall X'.(X'-> X')-> (X'-> X')$ can be applied to functions of many
different types in one program as long as their domain and range are the same.
For example, consider a typing context $\Gamma$ such that\footnote{
	For an intuitive explanation, we assume \texttt{int} and \texttt{string}
	to be existing ground types although our formal definition of HM
	in Figure \ref{fig:hm} only has the void type $\iota$ as
	the ground type for simplicity.} \vspace*{-1em}\\ 
\begin{minipage}{.3\linewidth}
\begin{align*}
\!\!\!\!\!\!\!\!\!\!
\textit{square} : \;\quad~\texttt{int} -> \texttt{int} ~\quad\; \in \Gamma \\
\!\!\!\!\!\!\!\!\!\!
\textit{revstr} : \texttt{string} -> \texttt{string} \in \Gamma
\end{align*}
\end{minipage}
\begin{minipage}{.4\linewidth}
\begin{align*}
\textit{Id}_{->}^{\,\textit{mon}} : \quad (X'-> X')-> (X'-> X') \quad \in \Gamma \\
\textit{Id}_{->}^{\,\textit{poly}} : \forall X'.(X'-> X')-> (X'-> X') \in \Gamma
\end{align*}
\end{minipage} \vspace*{1em} \\
Under the typing context $\Gamma$, It is possible to apply
$\textit{Id}_{->}^{\,\textit{mon}}$,
the monomorphic identity function over endofunctions,
to each of \textit{square} and \textit{revstr} as below,
as long as we do not try to apply $\textit{Id}_{->}^{\,\textit{mon}}$ to both
of them in the same program.\footnote{A program is just a term, but it sounds
like a more practical example.} For example,
\[ \Gamma |-
	(\textit{Id}_{->}^{\,\textit{mon}}\;\textit{square}) :
	\texttt{int} -> \texttt{int}
\]
\[ \Gamma |-
	(\textit{Id}_{->}^{\,\textit{mon}}\;\textit{revstr}) :
	\texttt{string} -> \texttt{string}
\]
However, it is impossible to derive type of a program, which tries to
apply an error when we try to apply $\textit{Id}_{->}^{\,\textit{mon}}$
to both \textit{square} and \textit{revstr} in one program.
\begin{align*}
\Gamma |- \;
& \dots (\textit{Id}_{->}^{\,\textit{mon}}\;\textit{square}) \dots \\
& \dots (\textit{Id}_{->}^{\,\textit{mon}}\;\textit{revstr}) \dots
~:~ \text{this is a type error}
\end{align*}
On the other hand, we can apply $\textit{Id}_{->}^{\,\textit{poly}}$,
the polymorphic identity function over endofunctions, to both
\textit{square} and \textit{revstr} even in one program.
\[ \Gamma |-
	(\textit{Id}_{->}^{\,\textit{poly}}\;\textit{square}) :
	\texttt{int} -> \texttt{int}
\]
\[ \Gamma |-
	(\textit{Id}_{->}^{\,\textit{poly}}\;\textit{revstr}) :
	\texttt{string} -> \texttt{string}
\]
\begin{align*}
\Gamma |-\;
& \dots (\textit{Id}_{->}^{\,\textit{poly}}\;\textit{square}) \dots \\
& \dots (\textit{Id}_{->}^{\,\textit{poly}}\;\textit{revstr}) \dots
~:~ \text{this can be type correct}
\end{align*}

Two sets of typing rules are illustrated in Figure \ref{fig:hm}:
declarative typing rules on the left and the syntax directed typing rules
on the right.

The declarative typing rules deduce a type scheme for a given term under
a typing context ($\Gamma |- t : \sigma$). The type scheme ($\sigma$)
deduced for the given term ($t$) under the typing ($\Gamma$) may not
be unique. For example,
\begin{align*}
	&\cdot |- \l x.x: \iota -> \iota \\
	&\cdot |- \l x.x: X -> X \\
	&\cdot |- \l x.x: (X -> X) -> (X -> X) \\
	&\cdot |- \l x.x: \forall X.X -> X \\
	&\cdot |- \l x.x: \forall X.(X -> X) -> (X -> X) \\
	&\cdot |- \l x.x: \forall X_1 X_2.(X_1 -> X_2) -> (X_1 -> X_2) \\
	&\quad \vdots
\end{align*}
This is expected since terms of HM are Curry style. Recall that
the uniqueness of typing property do not hold for lambda calculi
with Church-style terms. The first three rules \rulename{Var}, \rulename{Abs},
and \rulename{App} are fairly standard. The \rulename{Var} rule deduces
the type scheme of a variable according to the type scheme binding
of the variable in the typing context. The type schemes deduced by
the rules \rulename{Abs} and \rulename{App} are restricted to the form
of types\footnote{Recall type types are subset of type schemes.}
since the domain and range of functions types are restricted to types.
The \rulename{Let} rule enables polymorphic type schemes to be introduced
into the typing context. The \rulename{Inst} rule deduces a generic instance
($\sigma'$) of any deducible type scheme ($\sigma$). The \rulename{Inst} rule
is essential when variables with polymorphic type schemes appear in
the rules \rulename{Abs} and \rulename{App} rules. For instance,
when $t$ is a variable with polymorphic type scheme binding in $\Gamma$,
we need to instantiate the type scheme into a type since \rulename{Abs}
and \rulename{App} is restricted to deduce types. A typical usage of
the \rulename{Inst} rule is illustrated below:
\[
\inference[\sc Abs]
  {\inference[Inst]
	{\inference[Var]
		{x':\forall X'.X' -> X' \in x':\forall X'.X' -> X',x:X}
		{x':\forall X'.X' -> X',x:X|- x':\forall X'.X'-> X'}
	}
  	{x':\forall X'.X' -> X',x:X |- x':X' -> X'}}
  {x':\forall X'.X' -> X' |- \l x.x' : X -> (X' -> X')}
\]
The \rulename{Gen} rule deduces a generalization (\ie,
polymorphic quantification) of a deducible type scheme, as long as
the quantified variable does not appear free in the typing context.
The \rulename{Gen} is essential for the \rulename{Let} rule to be useful.
For instance, consider that $s$ is a function that may be polymorphic,
such as the identity function $\l x'.x'$. Recall that the \rulename{Abs} can
only deduce a type without any universal quantification, such as $X -> X$.
Here, we can use the \rulename{Gen} rule to generalize $X -> X$ to
$\forall X.X -> X$, provided that $X$ does not appear free in
the typing context, in order to make $\sigma$ appearing in
the \rulename{Let} rule a polymorphic type scheme, as below:
\[
\inference[\sc Let]
  { \inference[\sc Gen]
	  {\inference[\sc Abs]{\vdots}{\Gamma|- \l x'.x' : X -> X}}
	  {\Gamma|- \l x'.x':\forall X.X-> X}
  & \Gamma,x:\forall X.X-> X |- t : B}
  {\Gamma |- \<let> x = \l x'.x' \<in> t : B}
\]

The syntax directed typing rules deduce a type, rather than a type scheme,
for a given term under a typing context ($\Gamma |- t : A$). These rules
are syntax directed since there is only one rule to apply for each syntactic
cateogry of terms. Therefore, the typing is unique (up to change of
type variables) for the given term and the typing context. For example,
$\l x.x$ has a unique typing with respect to the syntax directed typing rules,
as below: \[ \cdot |- \l x.x: X -> X \]
The syntax directed typing rules are based on the observation
that the \rulename{Inst} and \rulename{Gen} in the delcarting typing rules
are only necessary at the \rulename{Var} and \rulename{Let} rules, respectively.
That is, we only need to apply the \rulename{Inst} rule to the conclusion of
the \rulename{Var} rule, and, we only need to apply the \rulename{Gen} rule to
the first premise ($\Gamma |- s:\sigma$) of the \rulename{Let} rule.
The \rulename{Var$_s$} rule can be understood as a merge of
\rulename{Var} and \rulename{Inst} in one rule.
The \rulename{Abs$_s$} rule and the \rulename{App$_s$} rule remains the same as
their counterparts in the declarative typing rules.
The \rulename{Let$_s$} rule can be understood as a merge of
\rulename{Let} and \rulename{Gen} in one rule.
The notation $\overline{\Gamma}(A)$ appearing in the \rulename{Let$_s$} is
a closure of type $A$ with respect to $\Gamma$. That is, $\overline{\Gamma}(A)$
generalizes $A$ all the free type variables of $A$ except the free variables of
$\Gamma$. The free type variables of $\Gamma$ is defined as:
$\FV(\Gamma) = \bigcup_{x:\sigma\in \Gamma} \FV(\sigma)$.

Syntax directed typing rules are sound (Theorem \ref{thm:sdHMsound})
and complete (Corollary \ref{cor:sdHMcomplete} with respect to
the declarative typing rules.
\begin{theorem}[$|-s$ is sound with respect to $|-$]
$ \inference{\Gamma |-s t : A}{\Gamma |- t : A} $
\label{thm:sdHMsound}
\end{theorem}
\begin{proof}
We will just give an key idea for the proof
since the soundness is rather obvious.
All we need to do is transform any given derivation for $|-s$
into a derivation for $|-$, which is straightforward.

Recall that the \rulename{Var$_s$} rule can be understood as a merge of
\rulename{Var} and \rulename{Inst}. Thus, we can transform any derivation
step using \rulename{Var$_s$} rule into two steps of derivation using
the \rulename{Var} rule, and then applying the \rulename{Inst} rule
to the conclusion of the \rulename{Var} rule.

The \rulename{Abs$_s$} rule and the \rulename{App$_s$} rules are mapped
to the \rulename{Abs} rule and the \rulename{App} rule, respectively.

Recall that the \rulename{Let$_s$} rule can be understood as a merge of
\rulename{Let} and \rulename{Gen}. We can transform any derivation step
using the \rulename{Let$_s$} rule into a series of \rulename{Gen} rules
applied to the first premise of the \rulename{Let} rule, and then applying
the \rulename{Let} rule. Since the definition of the closure
$\overline{\Gamma}(A)$ appearing in the \rulename{Let$_s$} rule
generalizes only the free type variables of $A$, which do not
appear free in $\Gamma$, the condition $X\notin\FV(\Gamma)$ appearing
in the \rulename{Gen} rule holds.
\end{proof}

\begin{theorem}[$|-s$ deduces a most general type]
$ \inference
	{\Gamma |- t : \sigma}
	{\Gamma |-s t : A ~\land~ \overline{\Gamma}(A)\sqsubseteq\sigma} $
\end{theorem}
\begin{proof}
TODO
\end{proof}

\begin{corollary}[$|-s$ is complete with respect to $|-$]
$ \inference{\Gamma |- t : \sigma}{\Gamma |-s t : A} $
\label{cor:sdHMcomplete}
\end{corollary}

The syntax directed rule and the algorithm $W$ is sound and complete
with respect to the syntax directed typing rules.

TODO\\
TODO\\
TODO\\
TODO\\
TODO\\
TODO\\
TODO\\
TODO\\
TODO\\
TODO\\

\begin{figure}
\begin{singlespace}
\[ \inference[\sc Var$_W$]
	{x:\forall X_1\dots X_n.A\in\Gamma \\
	 X_1',\dots,X_n'~\text{fresh}}
        {W(\Gamma,x) ~> (\emptyset,A[X_1'/X_1]\cdots[X_n'/X_n])}
\]
\[ \inference[\sc Abs$_W$]
	{X~\text{fresh} \\
	 W(\Gamma,x:X,t) ~> (S_1,A)}
	{W(\Gamma,x) ~> (S_1,S_1(X -> A))}
\]

\[ \inference[\sc App$_W$]
	{W(\Gamma,s) ~> (S_1,A_1) \\
	 W(S_1 \Gamma,t) ~> (S_2,A_2) \\
	 X~\text{fresh} \\
	 \unify(S_2 A_1,A_2 -> X) ~> S_3 }
	{W(\Gamma,s\;t) ~> (S_3\circ S_2\circ S_1,S_3 X)}
\]

\[ \inference[\sc Let$_W$]
	{W(\Gamma,s) ~> (S_1,A_1) \\
	 W(S_1(\Gamma,x:\overline{S_1\Gamma}(A_1)),t) ~> (S_2,A_2) }
	{W(\Gamma,\<let> x=s \<in> t) ~> (S_2\circ S_1,A_2)}
\]
\end{singlespace}
\caption{The type inference algorithm $W$}
\label{fig:algW}
\end{figure}

There exists a unification algorithm such that


\subsection{Unification}
A single substitution

\begin{figure}
\begin{singlespace}
\textbf{Unification}
\[ \unify(A_1,A_2) = U(\{A_1 =?= A_2\}) \]
\begin{align*}
U(\{A -> B =?= \iota\}\uplus Q) & ~> \text{fail} \\
U(\{\iota =?= A -> B\}\uplus Q) & ~> \text{fail} \\
U(\{\iota =?= \iota\}\uplus Q) & ~> U(Q) \\
U(\{x =?= x\}\uplus Q) & ~> U(Q) \\
U(\{A_1 -> B_1 =?= A_2 -> B_2\}\uplus Q)
	& ~> U(\{A_1 =?= A_2, B_1 =?= B_2\}\uplus Q) \\
U(\{x =?= y\}\uplus Q) & ~> U(\{y =?= x\}\uplus Q) \<when> x>y \\
U(\{A =?= y\}\uplus Q) & ~> U(\{y =?= A\}\uplus Q) \\
                            & \<when> A ~\text{is not a variable} \\
U(\{x =?= A\}\uplus Q) & ~> \text{fail}~ \<when> x\notin\FV(A) \\
U(\{x =?= A\}\uplus Q) & ~> U(\{x =?= A\}\uplus Q[A/x]) \\
			    & \<when> x\in\FV(Q) \\
U(\{x_1 =?= A_1,\dots,x_n =?= A_n\})
	& ~> \{x_1\mapsto A_1,\dots,x_n\mapsto A_n\} \\
	& \<when> x_1,\dots,x_n ~\text{are distinct, and,} \\
	& \qquad\quad x_i\notin\FV(A_j) ~\text{for all $i$ and $j$} \\
\end{align*}

\textbf{Unifier composition}
\begin{quote}
Let $S_1$ and $S_2$ be unifiers. That is,\\
for any $x\in \dom(S_1)$, $x$ does not occur in range of $S_2$, and,\\
for any $x\in \dom(S_1)$, $x$ does not occur in range of $S_2$.

Then, the composition of two unifiers $S_1$ and $S_2$ is defined as follows:
\end{quote}
\[ S_1 \circ S_2 = U(\{ x =?= A \mid x\mapsto A \in S_1 \cup S_2 \}) \]

\end{singlespace}
\caption{The unification algorithm and the composition of unifiers}
\label{fig:algU}
\end{figure}

\cite{Robinson65} TODO

\begin{proposition}[unification of identical types] $U(A,A)=\emptyset$
\label{prop:unifyidentical}
\end{proposition}
\begin{proof}
	%% TODO Easy by induction on the structure of type $A$.
\end{proof}
Note, that unification of identical types always suceeds.

\begin{lemma}[substitution composition is idempotent] \label{lem:compident}
	~\\ \indent
	$S\circ S = S$ \quad (when $\circ$ succeeds)
\end{lemma}
\begin{proof}
We prove by induction on the size of the substitution $S$.
The size of $S$ is defined as the size of its domain (\ie, $|S| = |\dom(S)|$).

When $S$ is empty, then it is trivial since
$\emptyset \circ \emptyset = \emptyset$
by the first equation of the substitution composition in Figure \ref{fig:algU}.

When $S$ is non-empty, we should apply the third equation
since the domains of the left- and right-hand side of the composition
obviously share the same variables. Recall the third equation of
the substitution composition in Figure \ref{fig:algU}:
\begin{align*}
(\{x\mapsto A_1\}\uplus S_1)\circ
(\{x\mapsto A_2\}\uplus S_2)
	&= (S_1\circ S_2) \circ (\{x\mapsto S A_1\}\uplus S) \\
	&\<where> S=U(A_1,A_2)
\end{align*}
Note that $\{x\mapsto A_1\}\uplus S_1 = \{x\mapsto A_2\}\uplus S_2$ in our case.
So, is easy to see that $A_1=A_2$ and $S_1=S_2$ Let us give a new name $S'$
for the substitutions $S_1$ and $S_2$ (\ie, $S'=S_1=S_2$) and a new name $A$
for the two types $A_1$ and $A_2$ (\ie, $A=A_1=A_2$).
By Proposition \ref{prop:unifyidentical}, we know that $S=\emptyset$
and $A=A_1=A_2$ in the above equation.  And, by induction,
we know that $S'\circ S' = S'$ (when composition succeeds).
Using what we know, we can simplify the above equation as follows:
\begin{align*}
(\{x\mapsto A\}\uplus S')\circ
(\{x\mapsto A\}\uplus S') &= S'\circ S' \circ \{x\mapsto A\} \\
			  &= S' \circ \{x\mapsto A\} \\
			  &= S' \uplus \{x\mapsto A\} \\
			  &= \{x\mapsto A\} \uplus S'
\end{align*}
The last two steps of the above simplification rely on the fact that
$x\notin\dom(S')$ (thus, using the second equation of $\circ$)
and $\uplus$ is commutative.

Therefore, $S\circ S = S$ (when composition succeeds).
\end{proof}

\begin{proposition}[composition of identical well-formed substitutions succeds]
	\label{prop:compident}
\begin{quote} $S \circ S$ suceeds when $S$ is well-formed \end{quote}
\end{proposition}
\begin{proof}
As discussed in the proof of the previous lemma,
the computation of $S \circ S$ will only involve
the first and thrid equations of the defintion of $\circ$.
The only source of possible failure is in the third equation calling on $U$.
However, we know that $U$ always suceeds when unifying identical types
(Proposition \ref{prop:unifyidentical}).
\end{proof}

From now on, $S_1 \circ S_2 = S$ means that the composition succeeds and
returns $S$. Similarly, $U(A_1,A_2)=S$ means that the unification succeeds
and returns $S$.

\begin{theorem}[substitution composition is idempotent] ~
	\begin{quote} $S\circ S = S$ for any well-formed $S$ \end{quote}
\end{theorem}
\begin{proof}
	By Lemma \ref{lem:compident} and Proposition \ref{prop:compident}.
\end{proof}

\begin{theorem}[unification is commutative] $ U(A_1,A_2) = U(A_2,A_1) $
	\label{thm:commU}
\end{theorem}
\begin{proof} Easy by induction on the structure of $A_1$ and $A_2$.

The base cases (\ie, $U(\iota,\iota)$, $U(x,x)$, $U(x_1,x_2)$,
			$U(x_1,A_2)$, $U(A_1,x_2)$) are trivial.

In the inductive case $U(A_1 -> B_1, A_2 -> B_2)$, we know that
$U(A_1,A_2) = U(A_2,A_1)$ and $U(B_1,A_2) = U(B_2,B_1)$ by induction.
Therefore,
\begin{align*}
U(A_1 -> B_1, A_2 -> B_2)
	&= U(A_1,A_2) \circ U(B_1,B_2) \\
	&= U(A_2,A_1) \circ U(B_2,B_1) = U(A_2 -> B_2, A_1 -> B_1)
\end{align*}
\end{proof}
TODO have we proved that both sides fail at the same time in the inductive case?



A substitution is well-formed when no variables of its domain
occurs in its range (\ie, $\dom(S)\cap\FV(\ran(S))=\emptyset$).

\begin{proposition}[substitution composition preserves well-formedness] ~
\begin{quote}
	Let $S_1$ and $S_2$ be well-formed substitutions.\\
	If $S_1\circ S_2=S$ then $S$ is also well-formed.
\end{quote}
\end{proposition}
\begin{proof}
	TODO
\end{proof}

\begin{proposition}[$U$ produces well-formed substitutions] ~
\begin{quote}
	If $U(A_1,A_2)=S$ then $S$ is well-formed.
\end{quote}
\end{proposition}
\begin{proof}
	TODO
\end{proof}

\begin{lemma}
Let $S$, $\{x\mapsto A'\}$, and $\{x\mapsto A\}$ be well-formed substitutions.
	\[ (S \circ \{x\mapsto A'\}) \circ \{x\mapsto A\} =
	S \circ (\{x\mapsto A'\} \circ \{x\mapsto A\}) \]
\end{lemma}
\begin{proof}
TODO
\end{proof}

\begin{lemma} \label{lem:assocsinglesubst}
Let $S_1$, $S_2$, and $\{x\mapsto A\}$ be well-formed substitutions.
\[ (S_1\circ S_2) \circ \{x\mapsto A\} = S_1 \circ (S_2 \circ \{x\mapsto A\}) \]
\end{lemma}
\begin{proof}
By induction on the size of $S_2$.

When $S_2=\emptyset$, trivial.

When $S_2$ is non-empty.

If $x\notin \dom(S_2)$, let $S_2=\{x'\mapsto A'\}\uplus S_2'$.
\begin{align*}
 &~ (S_1\circ S_2) \circ \{x\mapsto A\} \\
=&~ (S_1\circ(\{x'\mapsto A'\}\uplus S_2')) \circ \{x\mapsto A\} \\
=&~ (S_1\circ(S_2'\uplus\{x'\mapsto A'\})) \circ \{x\mapsto A\} \\
=&~ (S_1\circ(S_2'\circ\{x'\mapsto A'\})) \circ \{x\mapsto A\} \\
=&~ ((S_1\circ S_2')\circ\{x'\mapsto A'\}) \circ \{x\mapsto A\}
	& \text{by induction} \\
=&~ (S_1\circ S_2')\circ(\{x'\mapsto A'\} \circ \{x\mapsto A\})
	& \text{by Lemma TODO} \\
=&~ TODO \\
=&~ S_1 \circ (S_2'\circ (\{x'\mapsto A'\} \circ \{x\mapsto A\})) \\
=&~ S_1 \circ ((S_2'\circ \{x'\mapsto A'\}) \circ \{x\mapsto A\}) \\
=&~ S_1 \circ ((S_2'\uplus \{x'\mapsto A'\}) \circ \{x\mapsto A\}) \\
=&~ S_1 \circ ((\{x'\mapsto A'\}\uplus S_2') \circ \{x\mapsto A\}) \\
=&~ S_1 \circ (S_2 \circ \{x\mapsto A\})
\end{align*}


If $x\in \dom(S_2)$, let $S_2=\{x\mapsto A'\}\uplus S_2'$.
\begin{align*}
 &~ (S_1\circ S_2) \circ \{x\mapsto A\} \\
=&~ (S_1\circ(\{x\mapsto A'\}\uplus S_2')) \circ \{x\mapsto A\} \\
=&~ (S_1\circ(S_2'\uplus\{x\mapsto A'\})) \circ \{x\mapsto A\} \\
=&~ (S_1\circ(S_2'\circ\{x\mapsto A'\})) \circ \{x\mapsto A\} \\
=&~ ((S_1\circ S_2')\circ\{x\mapsto A'\}) \circ \{x\mapsto A\}
	& \text{by induction} \\
=&~ (S_1\circ S_2')\circ(\{x\mapsto A'\} \circ \{x\mapsto A\})
	& \text{by Lemma \ref{TODO}} \\
=&~ (S_1\circ S_2')\circ \{x\mapsto A''\} \\
=&~ S_1 \circ (S_2'\circ \{x\mapsto A''\} )
	& \text{by induction} \\
=&~ S_1 \circ (S_2'\circ (\{x\mapsto A'\} \circ \{x\mapsto A\})) \\
=&~ S_1 \circ ((S_2'\circ \{x\mapsto A'\}) \circ \{x\mapsto A\})
	& \text{by Lemma \ref{TODO}} \\
=&~ S_1 \circ ((S_2'\uplus \{x\mapsto A'\}) \circ \{x\mapsto A\}) \\
=&~ S_1 \circ ((\{x\mapsto A'\}\uplus S_2') \circ \{x\mapsto A\}) \\
=&~ S_1 \circ (S_2 \circ \{x\mapsto A\})
\end{align*}


\end{proof}

\begin{theorem}[substitution composition is associative]
	\[(S_1\circ S_2) \circ S_3 = S_1 \circ (S_2 \circ S_3)
	\qquad\text{($S_1, S_2, S_3$ are well-formed)}
	\]
\end{theorem}
\begin{proof}
By induction on the size of $S_3$.

When $S_3=\emptyset$, trivial.

When $S_3$ is non-empty, let $S_3=\{x\mapsto A\}\uplus S_3'$.
\begin{align*}
(S_1\circ S_2) \circ S_3
	&= (S_1\circ S_2) \circ (\{x\mapsto A\}\uplus S_3') \\
	&= (S_1\circ S_2) \circ (\{x\mapsto A\} \circ S_3') \\
	&= ((S_1\circ S_2) \circ \{x\mapsto A\}) \circ S_3'
	& \text{by induction} \\
	&= (S_1\circ(S_2 \circ \{x\mapsto A\})) \circ S_3'
	& \text{by Lemma \ref{lem:assocsinglesubst}} \\
	&= S_1\circ((S_2 \circ \{x\mapsto A\}) \circ S_3')
	& \text{by induction} \\
	&= S_1\circ(S_2 \circ(\{x\mapsto A\} \circ S_3'))
	& \text{by induction} \\
	&= S_1 \circ (S_2 \circ S_3)
\end{align*}
\end{proof}


\begin{lemma}\label{lem:commsinglesinglesubst}
$ \{x\mapsto A\}\circ\{x'\mapsto A'\} = \{x'\mapsto A'\}\circ\{x\mapsto A\} $
\end{lemma}
\begin{proof}
When either $\{x\mapsto A\}$ or $\{x'\mapsto A'\}$ is not well-formed,
\ie, either $x$ occurs in $A$ or $x'$ occurs in $A'$,
both sides of the equation above fail. The failure will rise from
the occurs check on a single substitution on types (Figure \ref{fig:algU}).
It is easy to check this proceeding by the second equation of $\circ$
for each side. We leave this as an exercise to the reader.

When $x$ occurs in $A'$ and $x'$ occurs $A$ at the same time,
both sides of the equation above fail as well. The failure will rise from
the occurs check on a single substitution on types (Figure \ref{fig:algU}).
It is easy to check this proceeding by the second equation of $\circ$.
We leave this as an exercise to the reader.

When neither $x$ nor $x'$ occur in $A$ and $A'$, it is easy to see
that this lemma holds, by using the second equation of $\circ$.
\begin{align*}
\{x\mapsto A\}\circ\{x'\mapsto A'\}
&= \{x\mapsto A[\{x\mapsto A\} A'/x']\}\uplus\{x'\mapsto\{x\mapsto A\} A'\} \\
&= \{x\mapsto A\}\uplus\{x'\mapsto A'\} \\
&= \{x\mapsto A'\}\uplus\{x\mapsto A\} \\
&= \{x'\mapsto A'[\{x'\mapsto A'\}A/x]\}\uplus\{x\mapsto\{x'\mapsto A'\}A\} \\
&= \{x'\mapsto A'\}\circ\{x\mapsto A\}
\end{align*}

The other two cases to consider are:
\begin{itemize}
\item[(1)] $x$ occurs in $A'$ but $x'$ does not occur in $A$
\begin{align*}
\{x\mapsto A\}\circ\{x'\mapsto A'\}
&= \{x\mapsto A[\{x\mapsto A\} A'/x']\}\uplus\{x'\mapsto\{x\mapsto A\} A'\} \\
&= \{x\mapsto A\}\uplus\{x'\mapsto A'[A/x]\} \\
&= \{x'\mapsto A'[A/x]\}\uplus\{x\mapsto A\} \\
&= \{x'\mapsto A'[\{x'\mapsto A'\}A/x]\}\uplus\{x\mapsto\{x'\mapsto A'\}A\} \\
&= \{x'\mapsto A'\}\circ\{x\mapsto A\}
\end{align*}
\item[(2)] $x'$ occurs in $A$ but $x$ does not occur in $A'$
\begin{align*}
\{x\mapsto A\}\circ\{x'\mapsto A'\}
&= \{x\mapsto A[\{x\mapsto A\} A'/x']\}\uplus\{x'\mapsto\{x\mapsto A\} A'\} \\
&= \{x\mapsto A[A'/x']\}\uplus\{x'\mapsto A'\} \\
&= \{x'\mapsto A'\}\uplus\{x\mapsto A[A'/x']\} \\
&= \{x'\mapsto A'[\{x'\mapsto A'\}A/x]\}\uplus\{x\mapsto\{x'\mapsto A'\}A\} \\
&= \{x'\mapsto A'\}\circ\{x\mapsto A\}
\end{align*}
\end{itemize}
\end{proof}

\begin{lemma}\label{lem:commsingletonsubst}
	$S \circ \{x\mapsto A\} = \{x\mapsto A\} \circ S
	\quad (x \notin \dom(S))$
\end{lemma}
\begin{proof} By induction on the size of $S$.

When $S=\emptyset$ it holds trivially.

When $S$ is non-empty, that is, let $S=\{x'\mapsto A'\}\uplus S'$,
\begin{align*}
S\circ\{x\mapsto A\}
&= (\{x'\mapsto A'\}\uplus S') \circ \{x\mapsto A\} \\
&= (\{x'\mapsto A'\}\circ S') \circ \{x\mapsto A\} \\
&= \{x'\mapsto A'\} \circ (S' \circ \{x\mapsto A\})
& \text{by associativity of $\circ$} \\
&= \{x'\mapsto A'\} \circ (\{x\mapsto A\} \circ S')
& \text{by induction} \\
&= (\{x'\mapsto A'\} \circ \{x\mapsto A\}) \circ S'
& \text{by associativity of $\circ$} \\
&= (\{x\mapsto A\} \circ \{x'\mapsto A\}') \circ S'
& \text{by Lemma \ref{lem:commsinglesinglesubst}} \\
&= \{x\mapsto A\} \circ (\{x'\mapsto A\}' \circ S')
& \text{by associativity of $\circ$} \\
&= \{x\mapsto A\} \circ S
\end{align*}
\end{proof}

\begin{theorem}[substitution composition is commutative]
\[ S_1\circ S_2 = S_2\circ S_1 \qquad\text{($S_1$ and $S_2$ are well-formed)} \]
\end{theorem}
\begin{proof}
By induction on the size of $S_2$.

When $S_2=\emptyset$ it holds trivially.

When $S_2$ is non-empty, let $S_2=\{x\mapsto A_2\}\uplus S_2'$,
\begin{align*}
S_1 \circ S_2
	&= S_1 \circ (\{x\mapsto A_2\}\uplus S_2') \\
	&= S_1 \circ (\{x\mapsto A_2\}\circ S_2') \\
	&= (S_1 \circ \{x\mapsto A_2\}) \circ S_2'
	& \text{by associativity of $\circ$} \\
	&= (\{x\mapsto A_2\}\circ S_1) \circ S_2'
	& \text{by Lemma \ref{lem:commsingletonsubst}} \\
	&= \{x\mapsto A_2\}\circ (S_1 \circ S_2')
	& \text{by associativity of $\circ$} \\
	&= \{x\mapsto A_2\}\circ (S_2' \circ S_1)
	& \text{by induction} \\
	&= (\{x\mapsto A_2\}\circ S_2') \circ S_1
	& \text{by associativity of $\circ$} \\
	&= (\{x\mapsto A_2\}\uplus S_2') \circ S_1 \\
	&= S_2 \circ S_1
\end{align*}
\end{proof}

\begin{lemma} $(S \circ S') A = (S \circ S')(S' A)$
	\label{lem:dupsubstR}
\end{lemma}
\begin{proof}
	TODO
\end{proof}

\begin{lemma} $(S \circ S') A = (S \circ S')(S A)$
	\label{lem:dupsubstL}
\end{lemma}
\begin{proof}
	TODO
\end{proof}


\begin{theorem}[$U$ is sound] \label{prop:soundU}
$ \inference{U(A_1,A_2)=S}{S A_1 = S A_2} $
\end{theorem}
\begin{proof}
By induction on the structure of $A_1$ and $A_2$.
%% By induction on the computation of $U$.
%% Since we assume $U(A_1,A_2)=S$, that is, $U$ succeeds and returns $S$,
%% we know that the computation of $U(A_1,A_2)$ is finite.
%% Therefore, we can induct on the computation of $U$.

The base cases (\ie, $U(\iota,\iota)$, $U(x,x)$, $U(x_1,x_2)$,
			$U(x_1,A_2)$, $U(A_1,x_2)$) are trivial.

In the inductive case $U(A_1 -> B_1, A_2 -> B_2)$, we know that
\[ \inference{U(A_1,A_2)=S' }{S'  A_1 = S'  A_2} ~\text{and}~
   \inference{U(B_1,B_2)=S''}{S'' B_1 = S'' B_2} ~\text{by induction.}
\]
Then, $U(A_1 -> B_1, A_2 -> B_2)=S'\circ S''$ by definition.
\begin{align*}
 &~(S'\circ S'')(A_1 -> B_1) \\
=&~ (S'\circ S'') A_1 -> (S'\circ S'') B_1 \\
=&~ (S'\circ S'')(S' A_1) -> (S'\circ S'')(S'' B_1)
 & \text{by Lemma \ref{lem:dupsubstR} and Lemma \ref{lem:dupsubstL}} \\
=&~ (S'\circ S'')(S' A_2) -> (S'\circ S'')(S'' B_2)
 & \text{by what we know from induction} \\
=&~ (S'\circ S'') A_2 -> (S'\circ S'') B_2
 & \text{by Lemma \ref{lem:dupsubstR} and Lemma \ref{lem:dupsubstL}} \\
=&~ (S'\circ S'')(A_2 -> B_2)
\end{align*}
Therefore, $U(A_1 -> B_1, A_2 -> B_2)=S'\circ S''$ is a unifier.
\end{proof}

\begin{theorem}[$U$ is complete] \label{prop:completeU}
$ \inference{S A_1 = S A_2}{\exists S'.\, U(A_1,A_2)\circ S' = S} $
\end{theorem}
\begin{proof}
	TODO
\end{proof}


