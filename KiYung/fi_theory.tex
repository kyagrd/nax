\section{Metatheory}\label{sec:fi:theory}
The expectation is that System \Fi\ has all the nice properties of System \Fw,
yet is more expressive because of the addition of term-indexed types.

We show some basic well-formedness properties for
the judgments of \Fi\ in \S\ref{ssec:fi:wf}.
We prove erasure properties of \Fi, which captures the idea that indices are
erasable since they are irrelevant for reduction in \S\ref{ssec:fi:erasure}.
We show strong normalization, logical consistence, and subject reduction for
\Fi\ by reasoning about well-known calculi related to \Fi\ in \S\ref{ssec:fi:sn}.

\subsection{Well-formedness properties and substitution lemmas}
\label{ssec:fi:wf}
We want to show that kinding and typing derivations give
well-formed results under well-formed contexts. That is,
kinding derivations ($\Delta |- F : \kappa$) result in well-sorted kinds
($|- \kappa$) under well-formed type-level contexts ($|- \Delta$)
(Proposition \ref{prop:wfkind}), and
typing derivations ($\Delta;\Gamma |- t : A$) result in well-kinded types
($\Delta;\Gamma |- A:*$) under well-formed type and term-level contexts
(Proposition \ref{prop:wftype}).

\begin{proposition}
\label{prop:wfkind}
$ \inference*{ |- \Delta & \Delta |- F : \kappa}{
	\qquad |- \kappa:\square \quad} $
\end{proposition}

\begin{proposition}
\label{prop:wftype}
$ \inference*{ \Delta |- \Gamma & \Delta;\Gamma |- t : A}{
	\qquad \Delta |- A : * \qquad} $
\end{proposition}

We can prove these well-formedness properties
by induction over the judgment, and using 
the well-formness lemmas on equalities
(Lemmas~\ref{lem:wfeqkind}, \ref{lem:wfeqtype}, and \ref{lem:wfeqterm})
and the substitution lemma (Lemma~\ref{lem:subst}).
The proof for Propositions \ref{prop:wfkind} and \ref{prop:wftype}
are mutually inductive.  So, we prove these two propositions
at the same time, using a combined judgment $J$,
which is either a kinding judgment or a typing judgment
(\ie, $J ::= \Delta |- F : \kappa \mid \Delta;\Gamma |- t : A$).
See Appendix \ref{app:proofsFi} for the detailed proofs of the
two propositions above.

\begin{lemma}[kind equality is well-sorted]\label{lem:wfeqkind}
$ \inference{|- \kappa = \kappa':\square}
	{|- \kappa:\square \quad |- \kappa':\square} $
\end{lemma}
\begin{proof}
	By induction on the derivation of kind equality
	and using the sorting rules.
\end{proof}

\begin{lemma}[type constructor equality is well-kinded]\label{lem:wfeqtype}
\[ \inference{\Delta |- F = F':\kappa}
	{\Delta |- F:\kappa \quad \Delta |- F':\kappa}
\]
\end{lemma}
\begin{proof}
	By induction on the derivation of type constructor equality
	and using the kinding rules.
	Also use the type substitution lemma
	(Lemma~\ref{lem:subst}(\ref{lem:tysubst}))
	and the index substitution lemma
	(Lemma~\ref{lem:subst}(\ref{lem:ixsubst})).  
\end{proof}

\begin{lemma}[term equality is well-typed]\label{lem:wfeqterm}
\[ \inference{\Delta,\Gamma |- t = t':A}
	{\Delta,\Gamma |- t:A \quad \Delta,\Gamma |- t':A}
\]
\end{lemma}
\begin{proof}
	By induction on the derivation of term equality
	and using the typing rules.
	Also use the term substitution lemma
	(Lemma~\ref{lem:subst}(\ref{lem:tmsubst})).
\end{proof}

The proofs for the three lemmas above are straightforward
once we have dealt with the interesting cases for the equality rules
involving substitution. We can prove those interesting cases
by applying the substitution lemmas. The other cases fall into two
categories: firstly, the equality rules following the same structure of
the sorting, kinding, and typing rules; and secondly, the reflexive
rules and the transitive rules. The proof for the equality rules
following the same structure of the sorting, kinding, and typing rules
can be proved by induction and applying the corresponding
sorting, kinding, and typing rules. The proof for the reflexive rules
and the transitive rules can be proved simply by induction.

\begin{lemma}[substitution]\mbox{}\\[-3mm]
\label{lem:subst}
\begin{enumerate}
\item
%\begin{lemma}[type substitution]
\label{lem:tysubst}
\mbox{\rm (type substitution)}
$\inference{\Delta,X^\kappa |- F:\kappa' & \Delta |- G:\kappa}
	{\Delta |- F[G/X]:\kappa'} $
%\end{lemma}
\medskip

\item
%\begin{lemma}[index substitution]
\label{lem:ixsubst}
\mbox{\rm (index substitution)}
$ \inference{\Delta,i^A |- F:\kappa & \Delta;\cdot |- s:A}
	{\Delta |- F[s/i]:\kappa} $
%\end{lemma}
\medskip

\item
%\begin{lemma}[term substitution]
\label{lem:tmsubst}
\mbox{\rm (term substitution)}
$ \inference{\Delta;\Gamma,x:A |- t:B & \Delta;\Gamma |- s:A}
	{\Delta,\Gamma |- t[s/x]:B} $
%\end{lemma}
\end{enumerate}
\end{lemma}
The substitution lemma is fairly standard, comparable to substitution lemmas
in other well-known systems such as \Fw\ or ICC.

\subsection{Erasure properties}
\label{ssec:fi:erasure}

We define a meta-operation of index erasure that projects $\Fi$-types
to $\Fw$-types.

\begin{definition}[index erasure]\label{def:ierase}
\[ \fbox{$\kappa^\circ$}
 ~~~~ ~~
 *^\circ =
 *
 ~~~~ ~~
 (\kappa_1 -> \kappa_2)^\circ =
 {\kappa_1}^\circ -> {\kappa_2}^\circ
 ~~~~ ~~
 (A -> \kappa)^\circ =
 \kappa^\circ
\]
\[ \fbox{$F^\circ$}
 ~~~~
 X^\circ =
 X
 ~~~~ ~~~~
 (A -> B)^\circ =
 A^\circ -> B^\circ
\]
\[ \qquad
 (\lambda X^\kappa.F)^\circ =
 \lambda X^{\kappa^\circ}.F^\circ
 ~~~~ ~~~~
 (\lambda i^A.F)^\circ =
 F^\circ
\]
\[ \qquad
 (F\;G)^\circ =
 F^\circ\;G^\circ
 ~~~~ ~~~~ ~~~~ ~~~~ ~~
 (F\,\{s\})^\circ =
 F^\circ
\]
\[ \qquad
 (\forall X^\kappa . B)^\circ =
 \forall X^{\kappa^\circ} . B^\circ
 ~~~~ ~~~~
 (\forall i^A . B)^\circ =
 B^\circ
\]
\[ \fbox{$\Delta^\circ$}
 ~~~~
 \cdot^\circ = \cdot
 ~~~~ ~~
 (\Delta,X^\kappa)^\circ = \Delta^\circ,X^{\kappa^\circ}
 ~~~~ ~~
 (\Delta,i^A)^\circ = \Delta^\circ
\]
\[ \fbox{$\Gamma^\circ$}
 ~~~~
 \cdot^\circ = \cdot
 ~~~~ ~~~~
 (\Gamma,x:A)^\circ = \Gamma^\circ,x:A^\circ
\]
\end{definition}

\begin{example}\label{PathologicalExample}
The meta-operation of index erasure simply discards all indexing
information.  The effect of this on most datatypes is to project the
indexing invariants while retaining the type structure.  
%
This is clearly seen for the vector type constructor~$\mathtt{Vec}$ whose
index erasure is the list type constructor~$\mathtt{List}$,
see~\Fig{churchrec}.
%
One can however build pathological examples.  For instance, the
type $\mathtt P_A \triangleq \forall i^A.\,\forall j^A.\, \LEq_A\s i \s j$
has index erasure $\mathtt{Unit} \triangleq \forall X^\mathtt{*}.\,X\to
X$.
\end{example}

\begin{theorem}[index erasure on well-sorted kinds]
\label{thm:ierasesorting}
	$\inference{|- \kappa : \square}{|- \kappa^\circ : \square}$
\end{theorem}
\begin{proof}
	By induction on the sorting derivation.
\end{proof}
\begin{remark}
For any well-sorted kind $\kappa$ in \Fi,
$\kappa^\circ$ is a kind in \Fw.
\end{remark}

\begin{theorem}[index erasure on well-formed type level contexts]
\label{thm:ierasetyctx}
\[ \inference{|- \Delta}{|- \Delta^\circ} \]
\end{theorem}
\begin{proof}
	By induction on the derivation for well-formed type level context
	and using Theorem \ref{thm:ierasesorting}.
\end{proof}
\begin{remark}
For any well-formed type level context $\Delta$ in \Fi,
$\Delta^\circ$ is a well-formed type level context in \Fw.
\end{remark}

\begin{theorem}[index erasure on kind equality]\label{thm:ierasekindeq}
$ \inference{|- \kappa=\kappa':\square}
	{|- \kappa^\circ=\kappa'^\circ:\square}
$
\end{theorem}
\begin{proof}
	By induction on the kind equality judgement.
\end{proof}
\begin{remark}
For any well-sorted kind equality $|- \kappa=\kappa':\square$ in \Fi,
$|- \kappa^\circ=\kappa'^\circ:\square$ is a well-sorted kind equality in \Fw.
\end{remark}

The three theorems above on kinds are rather simple to prove since there is
no need to consider mutual recursion in the definition of kinds due to
the erasure operation on kinds. Recall that the erasure operation on kinds
discards the type ($A$) appearing in the index arrow type ($A -> \kappa$).
So, there is no need to consider the types appearing in kinds
and the index terms appearing in those types, after the erasure.\\

\begin{theorem}[index erasure on well-kinded type constructors]
\label{thm:ierasekinding}
\[ \inference{|- \Delta & \Delta |- F : \kappa}
		{\Delta^\circ |- F^\circ : \kappa^\circ}
\]
\end{theorem}
\begin{proof}
	By induction on the kinding derivation.
\begin{itemize}
\item[case] ($Var$)
	Use Theorem \ref{thm:ierasetyctx}.

\item[case] ($Conv$)
	By induction and using Theorem \ref{thm:ierasekindeq}.

\item[case] ($\lambda$)
	By induction and using Theorem \ref{thm:ierasesorting}.

\item[case] ($@$)
	By induction.

\item[case] ($\lambda i$)
	We need to show that
	$\Delta^\circ |- (\lambda i^A.F)^\circ : (A -> \kappa)^\circ$,
	which simplifies to $\Delta^\circ |- F^\circ : \kappa^\circ$
	by Definition \ref{def:ierase}.

	By induction, we know that
	$(\Delta,i^A)^\circ |- F^\circ : \kappa^\circ $,
	which simplifies $\Delta^\circ |- F^\circ : \kappa^\circ$
	by Definition \ref{def:ierase}.

\item[case] ($@ i$)
	We need to show that
	$\Delta^\circ |- (F\;\{s\})^\circ : \kappa^\circ$,
	which simplifies to $\Delta^\circ |- F^\circ : \kappa^\circ$
	by Definition \ref{def:ierase}.

	By induction we know that
	$\Delta^\circ |- F^\circ : (A -> \kappa)^\circ$,
	which simplifies to $\Delta^\circ |- F^\circ : \kappa^\circ$
	by Definition \ref{def:ierase}.

\item[case] ($->$)
	By induction.

\item[case] ($\forall$)
	We need to show that
	$\Delta^\circ |- (\forall X^\kappa.B)^\circ : *^\circ$,
	which simplifies to
	$\Delta^\circ |- \forall X^{\kappa^\circ}.B^\circ : *$
	by Definition \ref{def:ierase}.

	Using Theorem \ref{thm:ierasesorting}, we know that
	$|- \kappa^\circ : \square$.

	By induction we know that
	$(\Delta,X^\kappa)^\circ |- B^\circ : *^\circ$,
	which simplifies to
	$\Delta^\circ,X^{\kappa^\circ} |- B^\circ : *$
	by Definition \ref{def:ierase}.

	Using the kinding rule ($\forall$), we get exactly
	what we need to show:
	$\Delta^\circ |- \forall X^{\kappa^\circ}.B^\circ : *$.

\item[case] ($\forall i$)
	We need to show that
	$\Delta^\circ |- (\forall i^A.B)^\circ : *^\circ$,
	which simplifies to $\Delta^\circ |- B^\circ : *$
	by Definition \ref{def:ierase}.

	By induction we know that
	$(\Delta,i^A)^\circ |- B^\circ : *^\circ$,
	which simplifies $\Delta^\circ |- B^\circ : *$
	by Definition \ref{def:ierase}.
\end{itemize}\qedhere
\end{proof}

\begin{theorem}[index erasure on type constructor equality]
\label{thm:ierasetyconeq}
\[ \inference{\Delta |- F=F':\kappa}
		{\Delta^\circ |- F^\circ=F'^\circ:\kappa^\circ}
\]
\end{theorem}\begin{proof}
By induction on the derivation of type constructor equality.

Most of the cases are done by applying the induction hypothesis
and sometimes using Proposition \ref{prop:wfkind}.

The only interesting cases, which are worth elaborating on, are the
equality rules involving substitution.  There are two such rules.

\paragraph{}
  $\inference{\Delta,X^\kappa |- F:\kappa' & \Delta |- G:\kappa}
             {\Delta |- (\lambda X^\kappa.F)\,G = F[G/X]:\kappa'}$ \\

We need to show
$ \Delta^\circ |- ((\lambda X^\kappa.F)\,G)^\circ = (F[G/X])^\circ : \kappa'^\circ $,
which simplifies to 
$ \Delta^\circ |- (\lambda X^{\kappa^\circ}.F^\circ)\,G^\circ = (F[G/X])^\circ : \kappa'^\circ $
by Definition \ref{def:ierase}.

By induction, we know that $(\Delta,X^\kappa)^\circ |- F^\circ : \kappa'^\circ$,
which simplifies to $\Delta^\circ,X^{\kappa^\circ} |- F^\circ : \kappa'^\circ$
by Definition \ref{def:ierase}.

Using the kinding rule ($\lambda$), we get
$\Delta^\circ |- \lambda X^{\kappa^\circ}. F^\circ : \kappa^\circ -> \kappa'^\circ$.

Using the kinding rule ($@$), we get
$\Delta^\circ |- (\lambda X^{\kappa^\circ}. F^\circ) G^\circ : \kappa'^\circ$.

Using the very same equality rule of this case,\\ we get 
$\Delta^\circ |- (\lambda X^{\kappa^\circ} F^\circ) G^\circ =
F^\circ[G^\circ/X] : \kappa'^\circ$.

All we need to check is $(F[G/X])^\circ = F^\circ[G^\circ/X]$,
which is easy to see.

\paragraph{}
  $\inference{\Delta,i^A |- F:\kappa & \Delta;\cdot |- s:A}
             {\Delta |- (\lambda i^A.F)\,\{s\} = F[s/i]:\kappa}$ \\

By induction we know that $\Delta^\circ |- F^\circ : \kappa^\circ$.

The erasure of the left hand side of the equality is\\
$((\lambda i^A.F)\,\{s\})^\circ = (\lambda i^A.F)^\circ = F^\circ$.

All we need to show is $(F[s/i])^\circ = F^\circ$, which is obvious
since index variables can only occur in index terms and index terms
are always erased. Recall the index erasure over type constructors in
Definition \ref{def:ierase}; in particular, $(\lambda i^A.F)^\circ=F^\circ$,
$(F\{s\})^\circ=F^\circ$, and $(\forall i^A.B)^\circ=B^\circ$.
\end{proof}
\begin{remark}
For any well-kinded type constructor equality $\Delta |- F=F':\kappa$ in \Fi,
$\Delta^\circ|- F^\circ=F'^\circ:\kappa^\circ$ is
a well-kinded type constructor equality in \Fw.
\end{remark}

The proofs for the two theorems above on type constructors need not consider
mutual recursion in the definition of type constructors due to
the erasure operation. Recall that the erasure operation on type constructors
discards the index term ($s$) appearing in the index application $(F\;\{s\})$.
So, there is no need to consider the index terms appearing in the types after
the erasure.

\begin{theorem}[index erasure on well-formed term level contexts]
\label{thm:ierasetmctx}
\[ \inference{\Delta |- \Gamma}{\Delta^\circ |- \Gamma^\circ} \]
\end{theorem}
\begin{proof}
By induction on $\Gamma$.
\begin{itemize}
\item[case] ($\Gamma=\cdot$) It trivially holds.
\item[case] ($\Gamma = \Gamma',x:A$),
we know that  $\Delta |- \Gamma'$ and $\Delta |- A:*$
by the well-formedness rules
and that $\Delta^\circ |- \Gamma'^\circ$ by induction.

From $\Delta |- A:*$, we know that $\Delta^\circ |- A^\circ :*$
by Theorem \ref{thm:ierasekinding}.

We know that $\Delta^\circ |- \Gamma'^\circ,x:A^\circ$
from $\Delta^\circ |- \Gamma'^\circ$ and $\Delta^\circ |- A^\circ :*$
by the well-formedness rules.

Since $\Gamma'^\circ,x:A^\circ = (\Gamma',x:A)^\circ = \Gamma^\circ$
by definition, we know that $\Delta^\circ |- \Gamma^\circ$.
\end{itemize}\vspace*{-10pt}
\end{proof}

\begin{theorem}[index erasure on index-free well-typed terms]
\label{thm:ierasetypingifree}
\[ \inference{ \Delta |- \Gamma & \Delta;\Gamma |- t : A}
		{\Delta^\circ;\Gamma^\circ |- t : A^\circ}
		{\enspace(\dom(\Delta)\cap\FV(t) = \emptyset)}
\]
\end{theorem}
\begin{proof} By induction on the typing derivation.
	Interesting cases are the index related rules ($:i$), ($\forall Ii$),
	and ($\forall Ei$). Proofs for the other cases are straightforward
	by induction and applying other erasure theorems corresponding to
	the judgment forms.
\begin{itemize}
\item[case] ($:$)
	By Theorem \ref{thm:ierasetmctx}, we know that
	$\Delta^\circ|- \Gamma^\circ$ when $\Delta|- \Gamma$.
	By definition of erasure on term-level context, we know that
	$(x:A^\circ) \in \Gamma^\circ$ when $(x:A) \in \Gamma$.
\item[case] ($:i$)
	Vacuously true since $t$ does not contain any index variables
        (\ie, $\dom(\Delta)\cap\FV(t) = \emptyset$).
\item[case] ($->$$I$)
	By Theorem \ref{thm:ierasekinding}, we know that $\cdot |- A^\circ:*$.
	By induction, we know that
	$\Delta^\circ;\Gamma^\circ,x:A^\circ |- t^\circ : B^\circ$.
	Applying the ($->$$I$) rule to what we know, we have
	$\Delta^\circ;\Gamma^\circ |- \l x.t^\circ : A^\circ -> B^\circ$.
\item[case] ($->$$E$)
	Straightforward by induction.
\item[case] ($\forall I$)
	By Theorem \ref{thm:ierasesorting}, we know that
	$|- \kappa^\circ:\square$.
	By induction, we know that
	$\Delta^\circ,X^{\kappa^\circ};\Gamma^\circ |- t : B^\circ$.
	Applying the ($\forall I$) rule to what we know, we have
	$\Delta^\circ;\Gamma^\circ |- t : \forall X^{\kappa^\circ}.B^\circ$.
\item[case] ($\forall E$)
	By induction, we know that
	$\Delta^\circ;\Gamma^\circ |- t : \forall X^{\kappa^\circ}.B^\circ$.
	By Theorem \ref{thm:ierasekinding}, we know that
	$\Delta^\circ |- G^\circ : \kappa^\circ$.
	Applying the ($\forall E$) rule, we have
	$\Delta^\circ;\Gamma^\circ |- t : B^\circ[G^\circ / X]$.
\item[case] ($\forall Ii$)
	By Theorem \ref{thm:ierasekinding}, we know that $\cdot |- A^\circ:*$.
	By induction, we know that $\Delta^\circ;\Gamma^\circ |- t : B^\circ$,
	which is what we want since $(\forall i^A.B)^\circ = B^\circ$.
\item[case] ($\forall Ei$)
	By induction, we know that $\Delta^\circ;\Gamma^\circ |- t : B^\circ$,
	which is what we want since $(B[s/i])^\circ = B^\circ$.
\item[case] ($=$)
	By Theorem \ref{thm:ierasetyconeq} and induction.
\end{itemize}\qedhere
\end{proof}

\begin{example}\label{PathologicalExampleContinued}
The theorem yields that the pathological type~$\mathtt P_A$
of~Example~\ref{PathologicalExample} is not inhabited, as it is impossible
to have both $t:\mathtt P_A$ and $t:(\mathtt P_A)^\circ=\mathtt{Unit}$.
It follows as a corollary that the implication of
Theorem~\ref{thm:ierasetypingifree} does not admit a converse.

In this context for $A=\mathtt{Void}$, note that even though one has
%the open typing 
$i^\mathtt{Void};\cdot\vdash\lambda x.\,i:\forall
j^{\mathtt{Void}}.\,\forall X^{\mathtt{Void}\to*}.\, X\s i\to X\s j$, 
this open term %this derivation 
cannot be closed by rule~$(\forall Ii)$ because of its side
condition.  This is in stark contrast to what is possible in calculi with
full type dependency. In System \Fi, the index variables
in type level context~$\Delta$ cannot appear dynamically at term level.
Conversely, term variables in the term level context~$\Gamma$ cannot be
used for instantiation of index polymorphic types (rule $(\forall Ei)$).

%% Similar considerations to the above show that $\LEq_A$ is not symmetric,
%% in that the type {\small\rm(Symmetric)} in~\S\ref{Leibniz}
\end{example}

We introduce an index variable selection meta-operation that selects all
the index variable bindings from the type level context.

\begin{definition}[index variable selection]
\[ \cdot^\bullet = \cdot \qquad
	(\Delta,X^\kappa)^\bullet = \Delta^\bullet \qquad
	(\Delta,i^A)^\bullet = \Delta^\bullet,i:A
\]
\end{definition}

\begin{theorem}[index erasure on well-formed term level contexts
		prepended by index variable selection]
\label{thm:ierasetmctxivs}
\[ \inference{\Delta |- \Gamma}{\Delta^\circ |- (\Delta^\bullet,\Gamma)^\circ}
\]
\end{theorem}
\begin{proof}
Straightforward by Theorem \ref{thm:ierasetmctx} and the typing rule ($:i$).
\end{proof}

The following result is the appropriate version of
Theorem~\ref{thm:ierasetypingifree} without the side condition therein.

\begin{theorem}[index erasure on well-typed terms]
\label{thm:ierasetypingall}
\[ \inference{\Delta |- \Gamma & \Delta;\Gamma |- t : A}
		{\Delta^\circ;(\Delta^\bullet,\Gamma)^\circ |- t : A^\circ}
\]
\end{theorem}
\begin{proof}
	The proof is almost the same as that of
	Theorem~\ref{thm:ierasetypingifree}, except for the ($:i$) case.
	The proof for the rule~($:i$) case is easy
	since $(i:A) \in \Delta^\bullet$ when $i^A \in \Delta$ by definition of
	the index variable selection operation. The indices from $\Delta$
	being prepended to $\Gamma$ do not affect the proof for the other cases.
\end{proof}

%% \begin{theorem}[index erasure on term equality]
%% \[ \inference{\Delta;\Gamma |- t=t':A}
%%  	{\Delta^\circ;\Gamma^\circ |- t=t':A^\circ}
%% \]
%% \end{theorem}

\subsection{Strong normalization and logical consistency}
\label{ssec:fi:sn}
\index{strong normalization!System Fi@System \Fi}
Strong normalization is a corollary of the erasure property since we know that
System~\Fw\ is strongly normalizing.

Logical consistency is immediate since
System~\Fi\ is a strict subset of the \emph{restricted implicit calculus}
\cite{Miquel00}, which is in turn a restriction of ICC~\cite{Miquel01}.
Subject reduction is also immediate for the same reason.
%% \marginpar{\framebox{\bf\em State these results formally in a theorem?}}

We can also give a more direct proof of logical consistency by shoing that
the void type $\forall X^{*}.X$ is uninhabited in \Fi. By type erasure
no more terms inhabit \Fi-types than the corresponding \Fw-types.
Since we already know that the void type $\forall X^{*}.X$
is uninhabited in \Fw, it must be the case that the void type
is uninhabited in \Fi.

\begin{comment}
\subsection{No \texttt{Void} type instantiation from dynamic values}
\label{NoVoid}

There is an interesting difference between \Fi\ and a Curry-style
dependent calculus with implicit arguments such as ICC, regarding
the instantiation of uninhabited type. Consider the 
instantiation rule\marginpar{\framebox{\bf\em Instantiation!?}}
of ICC, shown below:
\[
\inference{\Gamma,x:A |- t : B}{\Gamma |- t : \forall x^A.B }~(x\notin\FV(t))
\]
When $A=\mathtt{Void}$ and $B=\forall i^\mathtt{Void}.\mathtt{NeverEverVoid}\{i\}$,
we can instantiate $i$ with $y$, according to the rule above,
provided that $(y:\mathtt{Void})\in\Gamma$. Note that, in ICC, it is possible to
instantiate a universally quantified term variable $x$ of an uninhabited type
from a possibly dynamic term $y$.

\marginpar{\framebox{\bf\em Sorry, I don't quite understand.  Needs to be
		improved.}}

In System~\Fi, one cannot instantiate $B$ with any of the term variables since
index instantiation cannot refer to the term-level context~($\Gamma$)
but only refers to the type-level context~($\Delta$) --
recall the ($\forall E i$) rule in Figure \ref{fig:fi}.
%\[
%\inference[($\forall E i$)]
%{ \Delta;\Gamma |- t : \forall i^A.B & \Delta;\cdot |- s:A }
%	                           {\Delta;\Gamma |- t : B[s/i]}
%\]
Note that it is still possible to instantiate uninhabited type from
index variables introduced at type level~(\ie, when $j^\mathtt{Void}\in\Delta$).
%%%However, such variables are only introduced within a pathological
%%%type constructor definition, as in Examples~\ref{PathologicalExample}
%%%and~\ref{PathologicalExampleContinued}.
\texttt{Void} type instantiation is localized inside type constructor
definition.  It is assured that function definitions at term level will
never cause \texttt{Void} type instantiation, even when some of the
function arguments have uninhabited type.



\begin{proposition}[anti-dependency on arrow kinds]
\[ \inference{ |- \Delta,X^\kappa
             & \Delta,X^\kappa |- F : \kappa' }
             { X\notin\FV(\kappa') }
\]
\end{proposition}
\begin{proof}
	By Proposition \ref{prop:wfkind}, $|- \kappa'$.
	Note that $|- \kappa'$ does not involve any type level context.

	Therefore, $X$ cannot appear free in $\kappa'$.
\end{proof}

\begin{proposition}[anti-dependency on indexed arrow kinds]
\[ \inference{ |- \Delta,i^A
             & \Delta,i^A |- F : \kappa }
             { i\notin\FV(\kappa) }
\]
\end{proposition}
\begin{proof}
	By Proposition \ref{prop:wfkind}, $|- \kappa'$.
	Note that $|- \kappa'$ does not involve any type level context.
	Therefore, $i$ cannot appear free in $\kappa'$.
\end{proof}

\begin{proposition}[anti-dependency on arrow types]
\[ \inference{ \Delta |- \Gamma,x:A
             & \Delta;\Gamma,x:A |- t : B }
             { x\notin\FV(B) }
\]
\end{proposition}
\begin{proof}
	By Proposition \ref{prop:wftype}, $\Delta |- B:*$.
	Note that $\Delta |- \kappa'$ does not involve any term level context.
	Therefore, $x$ cannot appear free in $B$.
\end{proof}


\begin{remark} Our system is more strong??? than anti-dependency on arrow types
TODO
\end{remark}

\end{comment}

