\section{System \F} \label{sec:f}
\begin{figure}
\begin{singlespace}
\begin{minipage}{.46\textwidth}
	\begin{center}Church-style\end{center}
\def\baselinestretch{0}
\small
\begin{align*}
\textbf{term syntax} \\
t,s ::= &~ x           & \text{variable}    \\
      | &~ \l(x:A) . t & \text{abstraction} \\
      | &~ t ~ s       & \text{application} \\
      | &~ \L X    . t & \text{type abstraction} \\
      | &~ t [A]       & \text{type application} \\
\\
\textbf{type syntax} \\
A,B ::= &~ X           & \text{variable type}   \\
      | &~ A -> B      & \text{arrow type} \\
      | &~ \forall X.B & \text{forall type}   \\
\end{align*}
\[ \textbf{kinding \& typing contexts} \]\vspace*{-1em}
\begin{align*}\quad
\Delta ::= &~ \cdot \\
	 | &~ \Delta, x:A & (X\notin \dom(\Delta)) \\
\Delta;\Gamma ::= &~ \Delta;\cdot \\
	        | &~ \Delta,X;\Gamma    & (X\notin \ran(\Gamma)) \\
	        | &~ \Delta;\Gamma, x:A & (x\notin \dom(\Gamma)) \\
\end{align*}
\[ \textbf{kinding rules} \quad \framebox{$ \Delta |- A $} \]\vspace*{-1em}
\begin{align*}
& \inference[\sc TVar]{X \in \Delta}{\Delta |- X} \\
& \inference[\sc TArr]{\Delta |- A & \Delta |- B}{\Delta |- A -> B} \\
& \inference[\sc TAll]{\Delta,X |- B}{\Delta |- \forall X.B} \\
\end{align*}
\[ \textbf{typing rules} \quad \framebox{$ \Delta;\Gamma |- t : A $ } \]
\vspace*{-1em}
\begin{align*}
& \inference[\sc Var]{x:A \in \Gamma}{\Delta;\Gamma |- x:A} \\
& \inference[\sc Abs]{\Delta |- A & \Delta;\Gamma,x:A |- t : B}
	             {\Delta;\Gamma |- \l(x:A).t : A -> B} \\
& \inference[\sc App]{\Delta;\Gamma |- t : A -> B & \Delta;\Gamma |- s : A}
		     {\Delta;\Gamma |- t~s : B} \\
& \inference[\sc TyAbs]{\Delta,X;\Gamma |- t : B}
		       {\Delta;\Gamma |- \L X.t : \forall X.B} \\
& \inference[\sc TyApp]{\Delta;\Gamma |- t : \forall X.B & \Delta |- A}
		       {\Delta;\Gamma |- t[A] : B[A/X]}
\end{align*}
\end{minipage}
\begin{minipage}{.46\textwidth}
	\begin{center}Curry-style\end{center}
\def\baselinestretch{0}
\small
\begin{align*}
\textbf{term syntax} \\
t,s ::= &~ x           \\
      | &~ \l x    . t \\
      | &~ t ~ s       \\
      \phantom{|} &~ \\
      \phantom{|} &~ \\
\\
\textbf{type syntax} \\
A,B ::= &~ X \\
      | &~ A -> B \\
      | &~ \forall X . B \\
\end{align*}
\[ \textbf{kinding \& typing contexts} \]\vspace*{-1em}
\begin{align*}\quad
\Delta ::= &~ \cdot \\
	 | &~ \Delta, x:A & (X\notin \dom(\Delta)) \\
\Delta;\Gamma ::= &~ \Delta;\cdot \\
	        | &~ \Delta,X;\Gamma    & (X\notin \ran(\Gamma)) \\
	        | &~ \Delta;\Gamma, x:A & (x\notin \dom(\Gamma)) \\
\end{align*}
\[ \textbf{kinding rules} \quad \framebox{$ \Delta |- A $}\]\vspace*{-1em}
\begin{align*}
& \inference[\sc TVar]{X \in \Delta}{\Delta |- X} \\
& \inference[\sc TArr]{\Delta |- A & \Delta |- B}{\Delta |- A -> B} \\
& \inference[\sc TAll]{\Delta,X |- B}{\Delta |- \forall X.B} \\
\end{align*}
\[ \textbf{typing rules} \quad \framebox{$ \Delta;\Gamma |- t : A $ } \]
\vspace*{-1em}
\begin{align*}
& \inference[\sc Var]{x:A \in \Gamma}{\Delta;\Gamma |- x:A} \\
& \inference[\sc Abs]{\Delta |- A & \Delta;\Gamma,x:A |- t : B}
		     {\Delta;\Gamma |- \l x   .t : A -> B} \\
& \inference[\sc App]{\Delta;\Gamma |- t : A -> B & \Delta;\Gamma |- s : A}
		     {\Delta;\Gamma |- t~s : B} \\
& \inference[\sc TyAbs]{\Delta,X;\Gamma |- t : B}
		       {\Delta;\Gamma |- t : \forall X.B} \\
& \inference[\sc TyApp]{\Delta;\Gamma |- t : \forall X.B & \Delta |- A}
		       {\Delta;\Gamma |- t : B[A/X]}
\end{align*}
\end{minipage}
~\\
\caption{System \F\ in Church-style and Curry-style}
\label{fig:f}
\end{singlespace}
\end{figure}


TODO explain what extension we have made from STLC to F

TODO encoding of polymorphic regular recursive datatypes is possible

TODO compare curry style and church style

Since the terms of the Curry-style System \F\ is identical to the terms of
the Curry-style STLC, the reduction rules for Curry-style System \F\ is exactly
the same as the reduction rules for the Curry-style STLC (repeated below).
\vspace*{-.7em}
\begin{align*}
& \inference[\sc RedBeta]{}{(\l x   .t)~s --> t[s/x]} \\
& \inference[\sc RedAbs]{t --> t'}{\l x   .t --> \l x   .t'} \\
& \inference[\sc RedApp1]{t --> t'}{t~s --> t'~s} \\
& \inference[\sc RedApp2]{s --> s'}{t~s --> t~s'}
\end{align*}

The reduction rules for the Church-style System \F\ has all the reduction rules
for the Church-style STLC, and, in addition, three more reduction rules
(\rulename{RedTy}, \rulename{RedTyAbs}, and \rulename{RedTyApp}) involving
type abstractions and type applications. The following is the list of
reduction rules for the Church-style System \F:
\begin{align*}
& \inference[\sc RedBeta]{}{(\l(x:A).t)~s --> t[s/x]}
&&\inference[\sc RedTy]{}{(\L X   .t)[A] --> t[A/X]} \\
& \inference[\sc RedAbs]{t --> t'}{\l x   .t --> \l x   .t'}
&&\inference[\sc RedTyAbs]{t --> t'}{\L X   .t --> \L X   .t'} \\
& \inference[\sc RedApp1]{t --> t'}{t~s --> t'~s}
&&\inference[\sc RedTyApp]{t --> t'}{t[A] --> t'[A]} \\
& \inference[\sc RedApp2]{s --> s'}{t~s --> t~s'}
\end{align*}


\subsection{Subject reduction and strong normalization}\label{sec:f:srsn}
We can prove subject reduction for the Curry-style System \F\ in a similar
fashion to the proof of subject reduction for the Curry-style STLC,
by induction on the derivation of the reduction rules.

\begin{theorem}[subject reduction]
$\inference{\Delta;\Gamma |- t : A  & t --> t'}{\Delta;\Gamma |- t' : A}$
\end{theorem}
Proof for all other cases except for the \rulename{RedBeta} rule, are simply
done by applying the induction hypothesis. The only complication we have,
compared to STLC, is that the typing rules are not syntax directed 

The most interesting case of the proof is, of course,
the \rulename{RedBeta} rule.


Other cases

\begin{lemma}[substitution]
$ \inference{\Delta;\Gamma,x:A |- t : B  & \Delta;\Gamma |- s : A}
	{\Delta;\Gamma |- t[s/x] : B} $
\end{lemma}

Church-style

\paragraph{Strong normalization}

\begin{figure}
\begin{singlespace}
\begin{description}
\item[Interpretation of types] as saturated sets of normalizing terms
	whose free type variables are substituted according to
	the given type valuation ($\xi$):
\begin{align*}
[| X |]_\xi           &= \xi(X) \\ 
[| A -> B |]_\xi      &= [|A|]_\xi -> [|B|]_\xi \\
[| \forall X.B |]_\xi &= [|B|]_\xi \qquad\qquad\qquad (X\notin\dom(\xi))
\end{align*}
\item[Interpretation of kinding and typing contexts]
		       as sets of type and term valuations ($\xi$ and $\rho$):
\begin{align*}
[| \Delta        |] &= \dom(\Delta) -> \SAT \\
[| \Delta;\Gamma |] &= \{ \xi;\rho \mid \xi\in[|\Delta|], \rho\in[|\Gamma|]_\xi \} \\
[| \Gamma        |]_\xi\ &= \{ \rho : \dom(\Gamma) -> \SN \mid \rho(x)=[|\Gamma(x)|]_\xi ~\text{for all}~x\in\dom(\Gamma) \}
\end{align*}
\item[Interpretation of terms] as terms themselves whose free variables are
	substitued according to the given pair of valuations ($\xi$;$\rho$):
\begin{align*}
[| x      |]_{\xi;\rho} &= \rho(x) \\
[| \l x.t |]_{\xi;\rho} &= \l x . [|t|]_{\xi;\rho} \qquad (x\notin\dom(\rho)) \\
[| t ~ s  |]_{\xi;\rho} &= [| t |]_{\xi;\rho} ~ [| s |]_{\xi;\rho}
\end{align*}
\end{description}
\caption[Interpretation of System \F\ for proving strong normalization]
	{Interpretation of types, typing contexts, and terms of System \F\
         for the proof of strong normalization}
\label{fig:interpF}
\end{singlespace}
\end{figure}

\begin{theorem}[soundness of typing]
$ \inference{\Delta;\Gamma|- t:A & \xi;\rho \in [|\Delta;\Gamma|]}
	    {[|t|]_{\xi;\rho} \in [|A|]_\xi} $
\end{theorem}
\begin{proof}
\end{proof}

\begin{corollary}[strong normalization]
	$\inference{\Delta;\Gamma |- t : A}{t \in \SN}$
\end{corollary}

