\section{System \F} \label{sec:f}
\begin{figure}
\begin{singlespace}
\begin{minipage}{.46\textwidth}
	\begin{center}Church-style\end{center}
\def\baselinestretch{0}
\small
\begin{align*}
\textbf{term syntax} \\
t,s ::= &~ x           & \text{variable}    \\
      | &~ \l(x:A) . t & \text{abstraction} \\
      | &~ t ~ s       & \text{application} \\
      | &~ \L X    . t & \text{type abstraction} \\
      | &~ t [A]       & \text{type application} \\
\\
\textbf{type syntax} \\
A,B ::= &~ X           & \text{variable type}   \\
      | &~ A -> B      & \text{arrow type} \\
      | &~ \forall X.B & \text{forall type}   \\
\end{align*}
\[ \textbf{kinding \& typing contexts} \]\vspace*{-1em}
\begin{align*}\quad
\Delta ::= &~ \cdot \\
	 | &~ \Delta, x:A & (X\notin \dom(\Delta)) \\
\Gamma ::= &~ \cdot \\
	 | &~ \Gamma, x:A & (x\notin \dom(\Gamma)) \\
\end{align*}
\[ \textbf{kinding rules} \quad \framebox{$ \Delta |- A $} \]\vspace*{-1em}
\begin{align*}
& \inference[\sc TVar]{X \in \Delta}{\Delta |- X} \\
& \inference[\sc TArr]{\Delta |- A & \Delta |- B}{\Delta |- A -> B} \\
& \inference[\sc TAll]{\Delta,X |- B}{\Delta |- \forall X.B} \\
\end{align*}
\[ \textbf{typing rules} \quad \framebox{$ \Delta;\Gamma |- t : A $ } \]
\vspace*{-1em}
\begin{align*}
& \inference[\sc Var]{x:A \in \Gamma}{\Delta;\Gamma |- x:A} \\
& \inference[\sc Abs]{\Delta |- A & \Delta;\Gamma,x:A |- t : B}
	             {\Delta;\Gamma |- \l(x:A).t : A -> B} \\
& \inference[\sc App]{\Delta;\Gamma |- t : A -> B & \Delta;\Gamma |- s : A}
		     {\Delta;\Gamma |- t~s : B} \\
& \inference[\sc TyAbs]{\Delta,X;\Gamma |- t : B}
		       {\Delta;\Gamma |- \L X.t : \forall X.B} ~
		       (X\notin\ran(\Gamma)) \\
& \inference[\sc TyApp]{\Delta;\Gamma |- t : \forall X.B & \Delta |- A}
		       {\Delta;\Gamma |- t[A] : B[A/X]}
\end{align*}
\end{minipage}
\begin{minipage}{.46\textwidth}
	\begin{center}Curry-style\end{center}
\def\baselinestretch{0}
\small
\begin{align*}
\textbf{term syntax} \\
t,s ::= &~ x           \\
      | &~ \l x    . t \\
      | &~ t ~ s       \\
        &~\phantom{| \L X}  \\
        &~\phantom{| t [A]\vspace*{.1em}} \\
\\
\textbf{type syntax} \\
A,B ::= &~ X \\
      | &~ A -> B \\
      | &~ \forall X . B \\
\end{align*}
\[ \textbf{kinding \& typing contexts} \]\vspace*{-1em}
\begin{align*}\quad
\Delta ::= &~ \cdot \\
	 | &~ \Delta, x:A & (X\notin \dom(\Delta)) \\
\Gamma ::= &~ \cdot \\
	 | &~ \Gamma, x:A & (x\notin \dom(\Gamma)) \\
\end{align*}
\[ \textbf{kinding rules} \quad \framebox{$ \Delta |- A $}\]\vspace*{-1em}
\begin{align*}
& \inference[\sc TVar]{X \in \Delta}{\Delta |- X} \\
& \inference[\sc TArr]{\Delta |- A & \Delta |- B}{\Delta |- A -> B} \\
& \inference[\sc TAll]{\Delta,X |- B}{\Delta |- \forall X.B} \\
\end{align*}
\[ \textbf{typing rules} \quad \framebox{$ \Delta;\Gamma |- t : A $ } \]
\vspace*{-1em}
\begin{align*}
& \inference[\sc Var]{x:A \in \Gamma}{\Delta;\Gamma |- x:A} \\
& \inference[\sc Abs]{\Delta |- A & \Delta;\Gamma,x:A |- t : B}
		     {\Delta;\Gamma |- \l x   .t : A -> B} \\
& \inference[\sc App]{\Delta;\Gamma |- t : A -> B & \Delta;\Gamma |- s : A}
		     {\Delta;\Gamma |- t~s : B} \\
& \inference[\sc TyAbs]{\Delta,X;\Gamma |- t : B}
		       {\Delta;\Gamma |- t : \forall X.B} ~
		       (X\notin\ran(\Gamma)) \\
& \inference[\sc TyApp]{\Delta;\Gamma |- t : \forall X.B & \Delta |- A}
		       {\Delta;\Gamma |- t : B[A/X]}
\end{align*}
\end{minipage}
~\\
\caption{System \F\ in Church-style and Curry-style}
\label{fig:f}
\end{singlespace}
\end{figure}

\begin{figure}
\paragraph{Reduction rules for the Church-style System \F}
\begin{align*}
& \inference[\sc RedBeta]{}{(\l(x:A).t)~s --> t[s/x]}
&&\inference[\sc RedTy]{}{(\L X   .t)[A] --> t[A/X]} \\
& \inference[\sc RedAbs]{t --> t'}{\l x   .t --> \l x   .t'}
&&\inference[\sc RedTyAbs]{t --> t'}{\L X   .t --> \L X   .t'} \\
& \inference[\sc RedApp1]{t --> t'}{t~s --> t'~s}
&&\inference[\sc RedTyApp]{t --> t'}{t[A] --> t'[A]} \\
& \inference[\sc RedApp2]{s --> s'}{t~s --> t~s'}
\end{align*}
\paragraph{Reduction rules for the Curry-style System \F}
\begin{align*}
& \inference[\sc RedBeta]{}{(\l x   .t)~s --> t[s/x]} \\
& \inference[\sc RedAbs]{t --> t'}{\l x   .t --> \l x   .t'} \\
& \inference[\sc RedApp1]{t --> t'}{t~s --> t'~s} \\
& \inference[\sc RedApp2]{s --> s'}{t~s --> t~s'}
\end{align*}
\caption{Reduction rules for System \F}
\label{fig:redf}
\end{figure}

System \F\ extends the type syntax of STLC with type variables ($X$) and
forall types ($\forall X.B$), which enables us to express polymorphic types.
However, System \F\ does not have a dedicated syntax for ground types,
such as the void type $\iota$ in STLC, since we can populate types from
forall types such as $\forall X.X$. (this type is, in fact, an encoding of
the void type. We will discuss that System \F\ can encode certain class of
datatypes in \S\ref{sec:f:data}). In System \F, not all types constructed by
the type syntax make sense, unlike in STLC. Since we have type variables
in System \F, we need to make sure that types are well-kinded. That is,
we should make share that all the type variables appearing in types are
properly bound by universal quantifiers ($\forall$). For instance, consider
two types $\forall X.X$ and $\forall X.X'$. Under the empty kinding context,
$\forall X.X$ is well-kinded since $X$ is bounded by $\forall$, but
$\forall X.X'$ is ill-kinded since $X'$ is an unbound type variable.
So, we have kinding rules to determine whether a type is well-kinded.
In the kinding  rules and typing rules, we use kinding context ($\Delta$)
to keep track of bound type variables. The complete syntax, kinding rules,
and typing rules of System \F, for both Church style and Curry style, are
illustrated in Figure \ref{fig:f}. The reduction rules are shown
in Figure \ref{fig:redf} separately due to the space limitation.

Similarly to STLC, the term syntax of the abstraction differ between
the two styles. The Church-style System \F\ has type annotation in
the abstraction but the Curry-style System \F\ has not.
Furthermore, the Church-style System \F\ has additional syntax
for type abstractions and type applications. The differences in
typing rules and reduction rules between the two styles follow
from this difference in the term syntax.

The typing rules \rulename{Var}, \rulename{Abs}, and \rulename{App} are
pretty much the same as in STLC except that we carry around the kinding context
($\Delta$) along with the typing context ($\Gamma$). What is new in System \F\
are the typing rules for type abstractions (\rulename{TyAbs}) and
type applications (\rulename{TyApp}), which enable us to introduce
forall types and instantiate forall types to a specific type.
In Church style, uses of these two rules, \rulename{TyAbs} and
\rulename{TyAbs}, are guided by the term syntax of type abstractions
($\L X.t$) and type applications ($t[A]$). So, the typing rules of
the Church-style STLC are syntax directed. In Curry style, on the contrary,
there are no term syntax to guide the uses of \rulename{TyAbs} and
\rulename{TyAbs}. So, the typing rules of the Curry-style STLC are
not syntax directed.

The reduction rules for the Church-style System \F\ has all the reduction rules
for the Church-style STLC, and, in addition, three more reduction rules
(\rulename{RedTy}, \rulename{RedTyAbs}, and \rulename{RedTyApp}) involving
type abstractions and type applications.

The reduction rules for the Curry-style System \F\
is exactly the same as the reduction rules for the Curry-style STLC
(Figure \ref{fig:stlc}) since the terms of the Curry-style System \F\
and the terms of the Curry-style STLC are identical.

\subsection{Encoding datatypes in System \F}
\label{sec:f:data}
TODO encoding of polymorphic regular recursive datatypes is possible


\subsection{Subject reduction and strong normalization}\label{sec:f:srsn}
We discuss two important properties of System \F, which holds in both
Church style and Curry style -- \emph{subject reduction} (\aka\
\emph{type preservation}) and \emph{strong normalization}.

\subsubsection*{Subject reduction}
The subject reduction theorem for System \F\ can be stated as follows:
\begin{theorem}[subject reduction]
$\inference{\Delta;\Gamma |- t : A  & t --> t'}{\Delta;\Gamma |- t' : A}$
\end{theorem}
We can prove subject reduction for the System \F\, in a similar fashion
to the proof of subject reduction for STLC,
by induction on the derivation of the reduction rules.

In Church style, proof for all other cases except for the rules
\rulename{RedBeta} and \rulename{RedTy} are simply done by applying
the induction hypothesis. Since the typing rules in Church style are
syntax directed, there is no ambiguity on which typing rule to be used
in the derivation for a certain judgment. For the \rulename{RedBeta} case,
we use the substitution lemma. For proving the \rulename{RedTy} case,
we use the type substitution lemma. The substitution lemma and
the type substitution lemma are stated below:
\begin{lemma}[substitution]
$ \inference{\Delta;\Gamma,x:A |- t : B  & \Delta;\Gamma |- s : A}
	{\Delta;\Gamma |- t[s/x] : B} $
\end{lemma}
\begin{lemma}[type substitution]
$ \inference{\Delta,X;\Gamma |- t : B  & \Delta |- A}
	{\Delta;\Gamma |- t[A/X] : B[A/X]} ~ (X\notin\ran(\Gamma))$
\end{lemma}

In Curry style, the most interesting case is the \rulename{RedBeta} rule,
which we the substitution lemma to prove. The other rules are basically
done by applying the induction hypothesis, but there is a little complication
in the proof, compared to the proof in Church style, since the typing rules
are not syntax directed. Although we have less rules to consider than
the Church-style STLC, we need to deal with the ambiguity of which rule
to apply for a typing judgement. For example,

An alternative way to prove subject reduction for the Curry-style STLC is
by translating the subject reduction property of the Curry-style STLC into
the subject reduction property of the Church-style STLC. That is, we extract
a Church-style term from a typing derivation in Curry style. It is not
difficult to see that each typing derivation in Curry style corresponds to
a unique Church-style term, and, that a reduction step in Curry style
corresponds to one or more reduction steps in Church style.\footnote{
It is not always one step to one step since the reduction rules
{\sc RedTyAbs} and {\sc RedTyApp} in Church style correspond to
zero reduction step in Curry style.}

\subsubsection*{Strong Normalization}
\begin{figure}
\begin{singlespace}
\begin{description}
\item[Interpretation of types] as saturated sets of normalizing terms
	whose free type variables are substituted according to
	the given type valuation ($\xi$):
\begin{align*}
[| X |]_\xi           &= \xi(X) \\ 
[| A -> B |]_\xi      &= [|A|]_\xi -> [|B|]_\xi \\
[| \forall X.B |]_\xi &= \bigcap_{\mathcal{A}\in\SAT} [|B|]_{\xi[X:=\mathcal{A}]} \qquad\qquad\qquad (X\notin\dom(\xi))
\end{align*}
\item[Interpretation of kinding and typing contexts]
		       as sets of type and term valuations ($\xi$ and $\rho$):
\begin{align*}
[| \Delta        |] &= \dom(\Delta) -> \SAT \\
[| \Delta;\Gamma |] &= \{ \xi;\rho \mid \xi\in[|\Delta|], \rho\in[|\Gamma|]_\xi \} \\
[| \Gamma        |]_\xi\ &= \{ \rho \in \dom(\Gamma) -> \SN \mid \rho(x)=[|\Gamma(x)|]_\xi ~\text{for all}~x\in\dom(\Gamma) \}
\end{align*}
\item[Interpretation of terms] as terms themselves whose free variables are
	substituted according to the given pair of type and term valuations
	($\xi$;$\rho$):
\begin{align*}
[| x      |]_{\xi;\rho} &= \rho(x) \\
[| \l x.t |]_{\xi;\rho} &= \l x . [|t|]_{\xi;\rho} \qquad (x\notin\dom(\rho)) \\
[| t ~ s  |]_{\xi;\rho} &= [| t |]_{\xi;\rho} ~ [| s |]_{\xi;\rho}
\end{align*}
\end{description}
\caption[Interpretation of System \F\ for proving strong normalization]
	{Interpretation of types, typing contexts, and terms of System \F\
         for the proof of strong normalization}
\label{fig:interpF}
\end{singlespace}
\end{figure}
For the proof of strong normalization, we use the same proof strategy of
interpreting types as saturated sets of normalizing terms as in the proof of
strong normalization of STLC in \S\ref{sec:stlc:srsn}. The interpretation of
types, contexts, and terms of System \F\ are illustrated in
Figure \ref{fig:interpF}. Since we have type variables we need a type valuation
($\xi$), which maps type variables to interpretations of types. So, the
interpretation of types are indexed by the type valuation ($\xi$), and
the interpretation of terms are indexed by the pair of term and type valuations
($\xi;\rho$). A type valuation $\xi$ is a function from $\dom(\Delta)$,
the set of bound type variables in $\Delta$, to $\SAT$, the set of all
saturated sets.

Any type interpretation is a saturated set. Since $\xi$ maps a type variable
to a saturated set, $[|X|]_\xi \in \SAT$. We know $[|A -> B|]_\xi \in \SAT$
since saturated sets are closed under the arrow operation ($->$), as we
mentioned in \S\ref{sec:stlc:srsn}. $[|\forall X.B|]_\xi \in \SAT$ since
it is known that saturated sets are closed under set indexed intersection.

The Proof of strong normalization amounts to proving the following theorem:
\begin{theorem}[soundness of typing]
$ \inference{\Delta;\Gamma|- t:A & \xi;\rho \in [|\Delta;\Gamma|]}
	    {[|t|]_{\xi;\rho} \in [|A|]_\xi} $
\end{theorem}
\begin{proof}
We prove by induction on the typing derivation ($\Delta;\Gamma|- t:A$).
\paragraph{Case (\rulename{Var})}
It is trivial to show that
$ \inference{\Delta;\Gamma |- x:A & \xi;\rho \in [|\Delta;\Gamma|]}
	{[|x|]_{\xi;\rho} \in [|A|]_\xi} $.

We know that $x:A \in \Gamma$ from the \rulename{Var} rule.
So, $[|x|]_{\xi;\rho} =\rho(x)\in[|\Gamma(x)|]_\xi = [|A|]_\xi$.

\paragraph{Case (\rulename{Abs})}
We need to show that
$ \inference{\Delta;\Gamma |- \l x.t : A -> B & \xi;\rho \in [|\Delta;\Gamma|]}
	{[|\l x.t|]_{\xi;\rho} \in [|A -> B|]_\xi} $.

By induction, we know that
$ \inference{\Delta;\Gamma,x:A |- t : B & \xi';\rho' \in [|\Delta;\Gamma,x:A|]}
	     {[|t|]_{\xi';\rho'} \in [|B|]_\xi} $.

Since this holds for all $\xi';\rho' \in [|\Delta;\Gamma,x:A|]$, it also holds
for particular $\xi';\rho'$ such that $\xi'=\xi$ and
$\rho' = \rho[x \mapsto s]$ for any $s \in [|A|]_\xi' = [|A|]_\xi$.
Since saturated sets are closed under normalizing weak head expansion,
$(\l x.[|t|]_{\xi;\rho})~s \in[|B|]_\xi$ for any $s\in[|A|]_\xi$.
Therefore, $\l x.[|t|]_{\xi;\rho}$ is obviously in the very set,
which we wanted it to be in. That is,
\[
[|\l x.t|]_{\xi;\rho} = \l x.[|t|]_{\xi;\rho}
\in \{t\in \SN \mid t~s\in[|B|] ~\text{for all}~ s\in[|A|]\} 
= [|A -> B|]_\xi
\]

\paragraph{Case (\rulename{App})}
We need to show that
$ \inference{\Delta;\Gamma |- t~s : B & \xi;\rho\in[|\Delta;\Gamma|]}{[|t~s|]_{\xi;\rho} \in [|B|]_\xi} $.

By induction we know that
\[
\inference{\Delta;\Gamma |- t : A -> B & \xi;\rho\in[|\Delta;\Gamma|]}{[|t|]_{\xi;\rho} \in [|A -> B|]_\xi}
\qquad
\inference{\Delta;\Gamma |- s : A & \xi;\rho\in[|\Delta;\Gamma|]}{[|s|]_{\xi;\rho} \in [|A|]_\xi}
\]
Then, it is straightforward to see that $[|t~s|]_{\xi;\rho}\in[|B|]_\xi$
by definition of $[|A -> B|]_\xi$.

\paragraph{Case (\rulename{TyAbs})}
We need to show that
$ \inference{\Delta;\Gamma |- t : \forall X.B & \xi;\rho\in[|\Delta;\Gamma|]}
	{[|t|]_{\xi;\rho} \in [|\forall X.B|]_\xi} $

By induction, we know that
\[ \inference{\Delta,X;\Gamma |- t : B & \xi';\rho'\in[|\Delta,X;\Gamma|]}
	{[|t|]_{\xi';\rho'} \in [|B|]_{\xi'}} ~
	(X\notin\ran(\Gamma))
\]
Since this holds for all $\xi';\rho' \in [|\Delta,X;\Gamma|]$ where
$X\notin\ran(\Gamma)$, it also holds for particular subset such that
$\xi' = \xi[X:=\mathcal{A}]$ and $\rho'=\rho$ for all $\mathcal{A}\in\SAT$.
That is,
\[ [|t|]_{\xi[X:=\mathcal{A}];\rho} \in [|B|]_{\xi[X:=\mathcal{A}]}
   \quad \text{for all}~\mathcal{A}\in\SAT \]
From $X\notin\ran(\Gamma)$, we know that
$[|t|]_{\xi[X:=\mathcal{A}];\rho} = [|t|]_{\xi;\rho}$
because $\rho$ is independent of what $X$ maps to.
So, we know that
\[ [|t|]_{\xi;\rho} \in [|B|]_{\xi[X:=\mathcal{A}]}
	\quad \text{for all}~\mathcal{A}\in\SAT \]
By set theoretic definition, this is exactly what we wanted to show:
\[ [|t|]_{\xi;\rho} \in
	\bigcap_{\mathcal{A}\in\SAT} [|B|]_{\xi[X:=\mathcal{A}]}
	= [|\forall X.B|]_\xi
\]

\paragraph{Case (\rulename{TyApp})}
We need to show that
$ \inference{\Delta;\Gamma |- t : B[A/X] & \xi;\rho\in[|\Delta;\Gamma|]}
	{[|t|]_{\xi;\rho} \in [|B[A/X]|]_\xi} $

By induction, we know that
$ \inference{\Delta;\Gamma |- t : \forall X.B & \xi';\rho'\in[|\Delta;\Gamma|]}
	{[|t|]_{\xi';\rho'} \in [|\forall X.B|]_{\xi'}}
$.

Since this holds for all $\xi';\rho' \in [|\Delta,\Gamma|]$,
it also holds for $\xi'=\xi$ and $\rho'=\rho$. Then, we are done:
$ [|t|]_{\xi;\rho} \in [|\forall X.B|]_{\xi}
	= \bigcap_{\mathcal{A}\in\SAT} [|B|]_{\xi[X:=\mathcal{A}]}
	\subset [|B|]_{\xi[X:=[|A|]_\xi]} = [|B[A/X]|]_\xi
$.
\end{proof}
\begin{corollary}[strong normalization]
	$\inference{\Delta;\Gamma |- t : A}{t \in \SN}$
\end{corollary}
Once we have proved the soundness of typing with respect to interpretation,
it is easy to see that STLC is strongly normalizing by giving a trivial term
interpretation $\rho(x) = x$ for all the free variables.
Note that $[|t|]_{\xi;\rho} = t \in [|A|]_\xi \subset \SN$
under the trivial interpretation.

