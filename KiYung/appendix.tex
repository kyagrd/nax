\chapter{The Proof for Completeness of $W$}
\label{app:completeWproof}
Proof of Theorem \ref{thm:completeW}:
~\\ $~$ $~$ For any $\Gamma$ and $t$,
there exist $S'$, where $\dom(S')\subseteq\FV(\Gamma)$, and $A'$ such that
\[ \inference{ S'\Gamma |-s t : A' }{
        W(\Gamma,t) ~> (S,A_W) ~\land~
        \exists R . \left(
                S'\Gamma = R(S\Gamma) \,\land\,
                R(\overline{S\Gamma}(A_W))\sqsubseteq A' \right) }
\]
\begin{proof}
By induction on recursive call step of the algorithm $W$.
\begin{itemize}
\item[case]($x$)
        From the \rulename{Var$_s$} rule, we know that
        $S'\sigma \in S'\Gamma$, where $\sigma\in\Gamma$,
        and $S'\sigma\sqsubseteq A'$.
        By definition of $\sqsubseteq$, $A'$ has the form
        $S'A[B_1/X_1]\cdots[B_n/X_n]$ where $\sigma = \forall X_1\dots X_n.A$.

        From \rulename{Var$_W$} rule, we know that
        $S = \emptyset $ and $A_W = A[X_1'/X_1]\cdots[X_n'/X_n]$
        where $X_1',\dots,X_n'$ are fresh.

        Let $R=S'$, then, we are done.

\item[case]($\l x.t$)
        We want to show that \vspace*{-2em}
        \begin{singlespace}
        \[\inference{S'\Gamma |-s \l x.t : A -> B}{
        \begin{matrix} W(\Gamma,\l x.t) \\ ~> (S,S X -> B_W) \end{matrix}
        ~\land~
        \exists R.
                \left(\begin{matrix}
                        S'\Gamma=R(S\Gamma) ~~ \land \\
                        R(\overline{S\Gamma}(SX -> B_W))\sqsubseteq A -> B
                \end{matrix}\right) } \]
        \end{singlespace}

        Without loss of generality, we can choose $A = X$,
        since we can choose $S'$ accordingly such that $S'X = A$.
        Then, we have \vspace*{-2em}
        \begin{singlespace}
        \[\inference{S'\Gamma |-s \l x.t : S'X -> B}{
        \begin{matrix} W(\Gamma,\l x.t) \\ ~> (S,S X -> B_W) \end{matrix}
        ~\land~
        \exists R.
                \left(\begin{matrix}
                        S'\Gamma=R(S\Gamma) ~~ \land \\
                        R(\overline{S\Gamma}(SX -> B_W))\sqsubseteq S'X -> B
                \end{matrix}\right) } \]
        \end{singlespace}

        By induction, we know that  \vspace*{-2em}
        \begin{singlespace}
        \[\inference{S'(\Gamma,x:X) |-s t : B}{
        W((\Gamma, x:X),t) ~> (S,B_W) ~\land~
        \exists R.
                \left(\begin{matrix}
                        S'(\Gamma,x:X)=R(S(\Gamma,x:X)) \\ \land~\;
                        R(\overline{S(\Gamma,x:X)}(B_W))\sqsubseteq B
                \end{matrix}\right) } \]
        \end{singlespace}

        By Proposition \ref{prop:Abssrev},
        $S'\Gamma |- \l x.t : S'X -> B$ is sufficient to assume that
        $S'(\Gamma,x:X) |-s t:B$.

        By applying \rulename{Abs$_W$} rule to
        $W((\Gamma, x:X),t) ~> (S,B_W)$ where $X$ fresh,
        we have $W(\Gamma,\l x.t) ~> (S,S X -> B_W)$.

        From $S'(\Gamma,x:X)=R(S(\Gamma,x:X))$, we know that
        $S'\Gamma = R(S\Gamma)$ and $S'X = R(S X)$.

        If we can show that
        $R(\overline{S\Gamma}(S X -> B_W)) \sqsubseteq S'X -> B$,
        we are done. Since
        $R(\overline{S\Gamma}(S X -> B_W)) =
        R(\overline{S\Gamma}(S X)) -> R(\overline{S\Gamma}(B_W))$,
        what we need to show are
        $R(\overline{S\Gamma}(S X))\sqsubseteq S'X$ and
        $R(\overline{S\Gamma}(B_W)) \sqsubseteq B$.
        The former is true by Proposition \ref{prop:substclosure} and
        the facts that $S'X = R(S X)$ and
        $X\notin \dom(\Gamma)$ since $X$ is fresh:
        $R(\overline{S\Gamma}(S X)) =
        R(S(\overline{\Gamma}(X))) = R(S(X)) = S'X
        \sqsubseteq S' X$.
        The latter is true since $ R(\overline{S\Gamma}(B_W)) 
        \sqsubseteq R(\overline{S(\Gamma,x:X)}(B_W)) \sqsubseteq B$.

\item[case]($t\;s$)
        We want to show that \vspace*{-2em}
        \begin{singlespace}
        \[\inference{S'\Gamma |-s t\;s : B}{
        \begin{matrix} W(\Gamma,t\;s) ~\qquad~ \\
                ~> (S_3\circ S_2\circ S_1,S_3 X) \end{matrix}
        ~\land~
        \exists R.
                \left(\begin{matrix}
                        S'\Gamma=R((S_3\circ S_2\circ S_1)\Gamma) ~~ \land \\
                        R(\overline{(S_3\circ S_2\circ S_1)\Gamma}(S_3 X))
                        \sqsubseteq B
                \end{matrix}\right) } \]
        \end{singlespace}
        Note that we can use $S_3 X$ instead of $(S_3\circ S_2\circ S_1) X$
        since $X\notin\dom(S_2)\cup\dom(S_1)$ because
        $X$ has been picked fresh after $S_2$ and $S_1$ has been computed
        in \rulename{App$_W$} rule. So, $(S_3\circ S_2\circ S_1) X = S_3 X$.
        
        Since $S'\Gamma=R((S_3\circ S_2\circ S_1)\Gamma)$,
        we can replace $S'\Gamma$ with $S''((S_3\circ S_2\circ S_1)\Gamma)$
        without loss of generality. Then, what we want to show is
        \vspace*{-2em}
        \begin{singlespace}
        \begin{equation}
        \inference{S''((S_3\circ S_2\circ S_1)\Gamma) |-s t\;s : B'}{
        \begin{matrix} W(\Gamma,t\;s) ~\qquad~ \\
                ~> (S_3\circ S_2\circ S_1,S_3 X) \end{matrix}
        ~\land~
        \exists R.
        \left(\begin{matrix}
                S''((S_3\circ S_2\circ S_1)\Gamma)=R((S_3\circ S_2\circ S_1)\Gamma)
                \\ \land ~~
                R(\overline{(S_3\circ S_2\circ S_1)\Gamma}(S_3 X)) \sqsubseteq B'
        \end{matrix}\right) }
        \label{WcompleteWhatWeWantToShow}
        \end{equation}
        \end{singlespace}

        By induction and (\rulename{App$_s$}), we know that \vspace*{-2em}
        \begin{singlespace}
        \begin{align}
        \label{WcompleteAppsCase1}
        \inference{S_1'\Gamma |-s t : A_t}{
        \begin{matrix} W(\Gamma,t) \\ ~> (S_1,A_1) \end{matrix}
        ~\land~ 
        \exists R_1.
                \left(\begin{matrix}
                        S_1'\Gamma=R_1(S_1\Gamma) ~~\land\\
                        R_1(\overline{S_1\Gamma}(A_1))
                        \sqsubseteq A_t
                \end{matrix}\right) }
        \\
        \label{WcompleteAppsCase2}
        \inference{S_2'(S_1\Gamma) |-s s : A_s}{
        \begin{matrix} W(S_1\Gamma,s) \\ ~> (S_2,A_2) \end{matrix}
        ~\land~
        \exists R_2.
                \left(\begin{matrix}
                        S_2'(S_1\Gamma)=R_2(S_2(S_1\Gamma)) ~~\land\\
                        R_2(\overline{S_2(S_1\Gamma)}(A_2))
                        \sqsubseteq A_s
                \end{matrix}\right) }
        \end{align}
        \end{singlespace}

        From $S_1'\Gamma = R_1(S_1\Gamma)$ in the conclusion of 
        (\ref{WcompleteAppsCase1}), we can replace $S_1'\Gamma$
        with $S_2'(S_1\Gamma)$ in (\ref{WcompleteAppsCase1})
        without loss of generality, as follows: \vspace*{-2em}
        \begin{singlespace}
        \begin{align*}
        \inference{S_2'(S_1\Gamma) |-s t : A_t}{
        \begin{matrix} W(\Gamma,t)  \\~> (S_1,A_1) \end{matrix}
        ~\land~ \exists R_1.
        \left(\begin{matrix}
                S_2'(S_1\Gamma)=R_1(S_1\Gamma) ~~\land\\
                R_1(\overline{S_1\Gamma}(A_1))
                \sqsubseteq A_t
        \end{matrix}\right) }
        \end{align*}
        \end{singlespace}
        From $S_2'(S_1\Gamma)=R_1(S_1\Gamma)$, $R_1$ must be a substitution
        equivalent to $S_2'$ for all free type variables of $S_1\Gamma$.
        That is, $\dom(S_2')\subseteq\dom(R_1)$ and $S_2'X = R_1 X$ for
        any $X\in\dom(S_2')$. Note that $S_2'(\overline{S_1\Gamma}(A_1))
        \sqsubseteq R_1(\overline{S_1\Gamma}(A_1))$.
        So, we can choose $R_1=S_2'$ without loss of generality, as follows:
        \vspace*{-2em}
        \begin{singlespace}
        \begin{align*}
        \inference{S_2'(S_1\Gamma) |-s t : A_t}{
        \begin{matrix} W(\Gamma,t)  \\~> (S_1,A_1) \end{matrix}
        ~\land~
        \left(\begin{matrix}
                S_2'(S_1\Gamma)=S_2'(S_1\Gamma) ~~\land\\
                S_2'(\overline{S_1\Gamma}(A_1))
                \sqsubseteq A_t
        \end{matrix}\right) }
        \end{align*}
        \end{singlespace}
        Removing the trivial equation $S_2'(S_1\Gamma)=S_2'(S_1\Gamma)$
        from above, we have
        \begin{equation}
        \inference{ S_2'(S_1\Gamma) |-s t : A_t }{
        W(\Gamma,t) ~> (S_1,A_1) ~\land~
        S_2'(\overline{S_1\Gamma}(A_1)) \sqsubseteq A_t }
        \label{WcompleteAppsCase1'}
        \end{equation}

        Similarly, from (\ref{WcompleteAppsCase1}), we know that
        \begin{equation}
        \inference{ S_3'(S_2(S_1\Gamma)) |-s s : A_s}{
                W(S_1\Gamma,s) ~> (S_2,A_2) \land
                S_3'(\overline{S_2(S_1\Gamma)})A_2))\sqsubseteq A_s }
        \label{WcompleteAppsCase2'}
        \end{equation}

        We can choose $S_3'=S'''\circ S_3$ and $S_2' = S'''\circ S_3 \circ S_2$.
	Here, we rely on the fact that $S_3$ is a most general unifier.
	Recall that $\unify(A,B)$ succeeds when the two types $A$ and $B$
	are unifiable and the resulting subsitutiion is a most general unifier
	for those two types. If $S_3$ were not a most general unifier, it might
	make the closures of $A_1$ and $A_2$ too specific so that $\sqsubseteq$
	relations \ref{WcompleteAppsCase2'} no longer hold.
	So our choice $S_3'=S'''\circ S_3$ for \ref{WcompleteAppsCase2'} is
	a most probable candidate -- that is, nothing else could work if
	this choice doesn't work. The choice $S_2' = S'''\circ S_3 \circ S_2$
	for \ref{WcompleteAppsCase1'} is made accordingly
	to match $S_3'=S'''\circ S_3$.

        Then, by the syntax drived typing rule (\rulename{App$_s$}),
        $A_t = A_s -> B'$. Thus, the premises of (\ref{WcompleteAppsCase1'})
        and (\ref{WcompleteAppsCase2'}) are sufficient to assume the premise
        of what we want to prove, by Proposition \ref{prop:Appsrev}.
        Note that left-hand sides of the logical conjuctions in the conclusions,
        $W(\Gamma,t) ~> (S_1,A_1)$ and $W(S_1\Gamma,s) ~> (S_2,A_2)$, cocincides
        with the recursive call in the $W$ algorithm (\rulename{App$_W$}),
        since we are proving by induction on the recursive call step of
        the algorithm $W$. All we need to check is that the right-hand sides
        of ($\land$) in the conclusions of (\ref{WcompleteAppsCase1'}) and
        (\ref{WcompleteAppsCase2'}) are neccessary conditions for
        the right-hand side of ($\land$) in the conclusion of
        what we want to prove.

        Consider the right-hand side of $(\land$) in
        the conclusion of (\ref{WcompleteAppsCase1'}), replacing $S_2'$ with
        our choice of $S_2' = S'''\circ S_3\circ S_2$:
        \[ (S'''\circ S_3\circ S_2)(\overline{S_1\Gamma}(A_1))
        \sqsubseteq A_s -> B' \]
        We can replace $A_1$ in terms of $A_2$ and $X$ as follows:
        \begin{align*}
          & (S'''\circ S_3\circ S_2)(\overline{S_1\Gamma}(A_1))
                \\
        =~& S'''(\overline{S_3(S_2(S_1\Gamma))}(S_3(S_2 A_1)))
        &\quad& \text{by Proposition \ref{prop:substclosure}}
        \\
        =~& S'''(\overline{S_3(S_2(S_1\Gamma))}(S_3 A_2 -> S_3 X))
        &\quad& \text{by the unification used in (\rulename{App$_W$})}
        \\
        =~& S'''(\overline{S_3(S_2(S_1\Gamma))}(S_3 A_2 -> S_3 X)) \\
        =~& S'''(\overline{(S_3\circ S_2 \circ S_1)\Gamma}(S_3 A_2 -> S_3 X))
        &\quad& \quad\sqsubseteq\quad A_s -> B'
        \end{align*}
        Since closure operation and substitutions distribute over ($->$),
        we have
        \begin{equation}
                S'''(\overline{(S_3\circ S_2 \circ S_1)\Gamma}(S_3 A_2)) \sqsubseteq A_s
                \;\land\;
                S'''(\overline{(S_3\circ S_2 \circ S_1)\Gamma}(S_3 X) \sqsubseteq B'
                \label{WcompleteAppsAlmostWhatWeWant}
        \end{equation}

        Consider the right-hand side of $(\land$) in
        the conclusion of (\ref{WcompleteAppsCase2'}):
        \begin{align*}
          & (S'''\circ S_3)(\overline{S_2(S_1\Gamma)}(A_2))
                \\
        =~& S'''(\overline{S_3(S_2(S_1\Gamma))}(S_3 A_2)))
        && \text{by Proposition \ref{prop:substclosure}}
                \\
        =~& S'''(\overline{(S_3\circ S_2 \circ S_1)\Gamma}(S_3 A_2))
                && \quad\sqsubseteq\quad A_s
        \end{align*}
        Note that above is exaclty the same as the left-hand side of ($\land$)
        in \ref{WcompleteAppsAlmostWhatWeWant}, which is expected due to
        the nature the unification.

        We are done by choosing $S''=S'''$ and $R=S'''$
        in what we want to show (\ref{WcompleteWhatWeWantToShow}).
        Consider the right-hand side of $(\land)$ in the conclusion,
        replacing both $S''$ and $R$ with $S'''$:
        \[
        \left(\begin{matrix}
                S'''((S_3\circ S_2\circ S_1)\Gamma)=S'''((S_3\circ S_2\circ S_1)\Gamma)
                \\ \land ~~
                S'''(\overline{(S_3\circ S_2\circ S_1)\Gamma}(S_3 X)) \sqsubseteq B'
        \end{matrix}\right)
        \]
        Note that left-hand side of ($\land$) is trivially true and
        the right-hand side exactly matches the right-hand side of
        (\ref{WcompleteAppsAlmostWhatWeWant}).

\item[case]($\<let> x=s \<in> t$)
        We want to show that \vspace*{-2em}
        \begin{singlespace}
        \[\inference{S'\Gamma |-s \<let> x=s \<in> t : A_2'}{
        \begin{matrix} W(\Gamma,\<let> x=s \<in> t) \\
                ~> (S_2\circ S_1,A_2)
        \end{matrix}
        ~\land~
        \exists R.
                \left(\begin{matrix}
                        S'\Gamma=R((S_2\circ S_1)\Gamma) ~~ \land \\
                        R(\overline{(S_2\circ S_1)\Gamma}(A_2)) \sqsubseteq A_2'
                \end{matrix}\right) } \]
        \end{singlespace}

        By induction, we know that \vspace*{-2em}
        \begin{singlespace}
        \begin{align}
        \inference{S_1'\Gamma |-s s : A_1'}{
        W(\Gamma,s) ~> (S_1,A_1)
        ~\land~
        \exists R_1.
                \left( S_1'\Gamma=R_1(S_1\Gamma) \;\land\;
                        R_1(\overline{S_1\Gamma}(A_1)) \sqsubseteq A_1'
                \right) }
        \label{WcompleteLetsCase1}
        \\
        \inference{S_2'(S_1\Gamma,x:\overline{S_1\Gamma}(A_1)) |-s t : A_2'}{
        \begin{smallmatrix} W((S_1\Gamma,x:\overline{S_1\Gamma}(A_1)),t) \\
                ~> (S_2,A_2)
        \end{smallmatrix}
        ~\land~
        \exists R_2.
                \left(\begin{smallmatrix}
                        S_2'(S_1\Gamma,x:\overline{S_1\Gamma}(A_1))=R_2(S_2(S_1\Gamma,x:\overline{S_1\Gamma}(A_1))) \\ \land ~~
                        R_2(\overline{S_2(S_1\Gamma,x:\overline{S_1\Gamma}(A_1))}(A_2))
                        \sqsubseteq A_2'
                \end{smallmatrix}\right) }
        \label{WcompleteLetsCase2}
        \end{align}
        \end{singlespace}

        From $S_1'\Gamma = R_1(S_1\Gamma)$ in the conclusion of 
        (\ref{WcompleteLetsCase1}), we can replace $S_1'\Gamma$
        with $S_2'(S_1\Gamma)$ in (\ref{WcompleteLetsCase1})
        without loss of generality, as follows: \vspace*{-2em}
        \begin{singlespace}
        \[
        \inference{ S_2'(S_1\Gamma) |-s s : A_1' }{
        W(\Gamma,s) ~> (S_1,A_1)
        ~\land~ \exists R_1.
                \left( S_2'(S_1\Gamma)=R_1(S_1\Gamma) \;\land\;
                        R_1(\overline{S_1\Gamma}(A_1)) \sqsubseteq A_1'
                \right) }
        \]
        \end{singlespace}

        From $S_2'(S_1\Gamma)=R_1(S_1\Gamma)$, $R_1$ must be a substitution
        equivalent to $S_2'$ for all free type variables of $S_1\Gamma$.
        That is, $\dom(S_2')\subseteq\dom(R_1)$ and $S_2'X = R_1 X$ for
        any $X\in\dom(S_2')$. Note that $S_2'(\overline{S_1\Gamma}(A_1))
        \sqsubseteq R_1(\overline{S_1\Gamma}(A_1))$.
        So, we can choose $R_1=S_2'$ without loss of generality, as follows:
        \vspace*{-2em}
        \begin{singlespace}
        \[
        \inference{ S_2'(S_1\Gamma) |-s s : A_1' }{
        W(\Gamma,s) ~> (S_1,A_1)
        ~\land~
                \left( S_2'(S_1\Gamma)=S_2'(S_1\Gamma) \;\land\;
                        S_2'(\overline{S_1\Gamma}(A_1)) \sqsubseteq A_1'
                \right) }
        \]
        \end{singlespace}
        Removing the trivial equation $S_2'(S_1\Gamma)=S_2'(S_1\Gamma)$
        from above, we have \vspace*{-2em}
        \begin{singlespace}
        \[
        \inference{ S_2'(S_1\Gamma) |-s s : A_1' }{
        W(\Gamma,s) ~> (S_1,A_1) ~\land~
        S_2'(\overline{S_1\Gamma}(A_1)) \sqsubseteq A_1' }
        \]
        \end{singlespace}
        Using above and Lemma \ref{lem:genGamma}, we have
        \[
        \inference{ S_2'(S_1\Gamma),x:A_1' |-s t : A_2' &
           \inference{
                \inference{ S_2'(S_1\Gamma) |-s s : A_1' }{
                S_2'(\overline{S_1\Gamma}(A_1)) \sqsubseteq A_1'} }{
                        S_2'(S_1\Gamma),x:S_2'(\overline{S_1\Gamma}(A_1))
                        \sqsubseteq
                        S_2'(S_1\Gamma),x:A_1' } }{
           S_2'(S_1\Gamma),x:S_2'(\overline{S_1\Gamma}(A_1)) |-s t : A_2' }
        \]
        which can be summarized as
        \[
        \inference{ S_2'(S_1\Gamma),x:A_1' |-s t : A_2' &
                    S_2'(S_1\Gamma) |-s s : A_1' }{
           S_2'(S_1\Gamma),x:S_2'(\overline{S_1\Gamma}(A_1)) |-s t : A_2' }
        \]
        By Proposition \ref{prop:Letsrev}, we have
        \begin{align} \label{usingLetsrev1}
        \inference{
           \inference{ S_2'(S_1\Gamma) |-s \<let> x=s \<in> t : A_2' }{
                   \exists A_1'.\left(
                S_2'(S_1\Gamma),x:A_1' |-s t : A_2' ~\land~
                S_2'(S_1\Gamma) |-s s : A_1' \right) } }{
           S_2'(S_1\Gamma),x:S_2'(\overline{S_1\Gamma}(A_1)) |-s t : A_2' }
        \end{align}
        Note that the assumption of (\ref{usingLetsrev1}),
        $S_2'(S_1\Gamma) |-s \<let> x=s \<in> t : A_2'$,
        implies both the assumption of (\ref{WcompleteLetsCase1})
        instantiated by $S_1'=S_2'\circ S_1$
        and the assumption (\ref{WcompleteLetsCase2}).
        So, we can merge the conclusion of (\ref{WcompleteLetsCase1}) and
        the conclusion of (\ref{WcompleteLetsCase2})
        instantiated by $S_1'=S_2'\circ S_1$
        in order to synthesize what we want to prove.

        Applying \rulename{Let$_W$} rule to left-hand arguments of $\land$
        in the conclusions of (\ref{WcompleteLetsCase1}) and
        (\ref{WcompleteLetsCase2}), we get
        $W(\Gamma,\<let> x=s \<in> t) ~> (S_2\circ S_1,A_2)$.

        Let $R_2=R$ in the right-hand side in the conclusion of
        (\ref{WcompleteLetsCase2}). Then, we get
        $\exists R.\left(  S_2'(S_1\Gamma)=R((S_2\circ S_1)\Gamma) ~\land~
        R(\overline{(S_2\circ S_1)\Gamma}(A_2)) \sqsubseteq A_2' \right)$
        by similar steps we took for the case (\rulename{Abs$_s$}).

        In summary, we get \vspace*{-2em}
        \begin{singlespace}
        \[\inference{S_2'(S_1\Gamma) |-s \<let> x=s \<in> t : A_2'}{
        \begin{matrix} W(\Gamma,\<let> x=s \<in> t) \\
                ~> (S_2\circ S_1,A_2)
        \end{matrix}
        ~\land~
        \exists R.
                \left(\begin{matrix}
                        S_2'(S_1\Gamma)=R((S_2\circ S_1)\Gamma) ~~ \land \\
                        R(\overline{(S_2\circ S_1)\Gamma}(A_2)) \sqsubseteq A_2'
                \end{matrix}\right) } \]
        \end{singlespace}
        which is almost exactly what we want to prove,
        except that $S_2'(S_1\Gamma)$ is used in place of $S'\Gamma$.

        Without loss of generality, we can use $S_2'(S_1\Gamma)$
        instead of $S'\Gamma$. By Proposition \ref{prop:Letsrev},
        $S'\Gamma |-s \<let> x=s \<in> t : A_2'$ implies
        $S'\Gamma |-s s : A_1'$ for some $A_1'$.
        Applying (\ref{WcompleteLetsCase1}) to 
        $S'\Gamma |-s s : A_1'$ with $S_1'=S'$,
        we have $S'\Gamma = R_1(S_1\Gamma)$ for some $R_1$.

\vspace*{-2em}
\end{itemize}
\end{proof}


\chapter{Proofs in the metatheory of System \Fi}\label{app:proofsFi}
This appendix contains proofs of propositions in \S\ref{sec:fi:theory}.

\paragraph{}
Proof of Proposition \ref{prop:wfkind}:
\[ \inference{ |- \Delta & \Delta |- F : \kappa}{ |- \kappa:\square }
\]
\begin{proof} By induction on the derivation.
\begin{itemize}
\item[case] ($Var$)
	Trivial by the second well-formedness rule of $\Delta$.
\item[case] ($Conv$)
	By induction and Lemma~\ref{lem:wfeqkind}.
\item[case] ($\lambda$)
	By induction, we know that $|- \kappa:\square$.\\
	By the second well-formedness rule of $\Delta$,
	we know that $|- \Delta,X^\kappa$ since we already know
	that $|- \kappa:\square$ and $|- \Delta$ from the property statement.\\
	By induction, we know that $|- \kappa':\square$
	since we already know that $|- \Delta,X^\kappa$ and
	that $\Delta,X^\kappa|- F:\kappa'$ from induction hypothesis.\\
	By the sorting rule ($R$), we know that $|- \kappa -> \kappa':\square$
	since we already know that $|- \kappa:\square$ and $|- \kappa':\square$.
\item[case] ($@$)
	By induction, easy.
\item[case] ($\lambda i$)
	By induction we know that $\cdot|- A:*$.
	By the third well-formedness rule of $\Delta$, we know that
	$|- \Delta,i^A$ since we already know that $\cdot|- A:*$ and
	that $|- \Delta$ from the property statement.\\
	By induction, we know that $|- \kappa:\square$
	since we already know that $|- \Delta,i^A$ and
	that $\Delta,i^A|- F:\kappa$ from the induction hypothesis.\\
	By the sorting rule ($Ri$), we know that $|- A -> \kappa:\square$
	since we already know that $\cdot |- A:*$ and $|- \kappa:\square$.
\item[case] ($@i$)
	By induction and Proposition \ref{prop:wftype}, easy.
\item[case] ($->$)
	Trivial since $|- * : \square$.
\item[case] ($\forall$)
	Trivial since $|- * : \square$.
\item[case] ($\forall i$)
	Trivial since $|- * : \square$.\qedhere
\end{itemize}
\end{proof}

The basic structure of the proof for Proposition \ref{prop:wftype}
on typing derivations is similar to above. So, we illustrate the proof
for most of the cases, which can be done by applying the induction hypothesis,
rather briefly. We elaborate more on interesting cases ($\forall E$) and
($\forall Ei$) which involve substitutions in the types resulting from
the typing judgments.

\paragraph{}
Proof of Proposition \ref{prop:wftype}:
\[ \inference{ \Delta |- \Gamma & \Delta;\Gamma |- t : A}{ \Delta |- A : * }
\]
\begin{proof} By induction on the derivation.
\begin{itemize}
\item[case] ($:$)
	Trivial by the second well-formedness rule of $\Gamma$.
\item[case] ($:i$)
	Trivial by the third the well-formedness rule of $\Delta$.
\item[case] ($=$)
	By induction and Lemma \ref{lem:wfeqtype}.
\item[case] ($->$$I$)
	By induction and well-formedness of $\Gamma$.
\item[case] ($->$$E$)
	By induction.
\item[case] ($\forall I$)
	By induction and well-formedness of $\Delta$.
\item[case] ($\forall E$)
	By induction we know that $\Delta |- \forall X^\kappa.B : *$.\\
	By the kinding rule ($\forall$), which is the only kinding rule
	able to derive $\Delta |- \forall X^\kappa.B : *$, we know
	that $\Delta,X^\kappa |- B : *$.\\
	Then, we use the type substitution lemma
	(Lemma~\ref{lem:subst}(\ref{lem:tysubst})).
\item[case] ($\forall Ii$)
	By induction and well-formedness of $\Delta$.
\item[case] ($\forall Ei$)
	By induction we know that $\Delta |- \forall i^A.B : *$.\\
	By the kinding rule ($\forall i$), which is the only kinding rule
	able to derive $\Delta |- \forall i^A.B : *$, we know
	that $\Delta,i^A |- B : *$.\\
	Then, we use the index substitution lemma
	(Lemma~\ref{lem:subst}(\ref{lem:ixsubst})).\qedhere
\end{itemize}
\end{proof}



\chapter{TODO Yet Another Appendix}


