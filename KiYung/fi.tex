\chapter{System \Fi}\label{ch:fi} TODO

I am developing a typed lambda calculus called System \Fi\, which is an
extension of System \Fw\ with term indexed types. Term indexed types
naturally support the Curry-Howard isomorphism, as a type constructor can be
seen as a relation between one or more terms.  This kind of expressiveness is
available in dependently typed languages (where there is no distinction between
types and terms), and is also encodable using GADTs in functional languages
such as Haskell.  GADTs in Haskell are often indexed by uninhabited types
which are isomorphic to terms. The type indices used in GADTs in Haskell are
often really intended to be values at term level, but are just simulations of
such terms at type level.  For example one might introduce uninhabited type
constructors \verb+Zero+ and \verb+Succ+ in order to simulate natural numbers
when defining length indexed lists, as follows:
\begin{verbatim}
data Zero
data Succ x

data Vector a n where
  Nil:: Vector a Zero
  Cons:: a -> Vector a n -> Vector a (Succ n)
\end{verbatim}  

Within System \Fi, we can rigorously define types indexed by terms,
while keeping the term level and the type level cleanly separated.
We can also argue about the properties of the Mendler-style combinators
over indexed types.  You can find a summary of \Fi\ in \S\ref{ch:fi}.

TODO
