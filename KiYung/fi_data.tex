\section{Embedding datatypes and Mendler-style iterators}\label{sec:fi:data}

System \Fi\ can express a rich collection of datatypes.
%% TODO cite some paper that does this with System Fw or System F
First, we illustrate embeddings for both non-recursive and
recursive datatypes using Church encodings \cite{Church33} to define
data constructors (\S\ref{ssec:embedChurch}). Second, we illustrate
a more involved embedding for recursive datatypes based on two-level types
(\S\ref{ssec:embedTwoLevel}). Lastly, we discuss an encoding of equality over
term indices (\S\ref{Leibniz}).

\subsection{Embedding datatypes using Church-encoded terms}
\label{ssec:embedChurch}
\begin{figure}
\begin{singlespace}
\begin{align*}
&\!\!\!\!\!\!\mathtt{Bool} &=~& \forall X.X -> X -> X \\
&\!\!\!\!\!\!\mathtt{true}  &\!\!\!:~~& \texttt{Bool} ~~=~ \l x_1.\l x_2. x_1 \\
&\!\!\!\!\!\!\mathtt{false} &\!\!\!:~~& \texttt{Bool} ~~=~ \l x_1.\l x_2. x_2 \\
&\!\!\!\!\!\!\mathtt{elim_{Bool}} &\!\!\!:~~& \texttt{Bool} -> \forall X.X -> X -> X \\
&	&=~& \l x.\l x_1. \l x_2. x\;x_1\,x_2 \qquad
(\textbf{if}~x~\textbf{then}~x_1~\textbf{else}~x_2)
\end{align*}\vspace*{-19pt} \\ \vspace*{-4pt}
\rule{\linewidth}{.4pt}
\begin{align*}
&\!\!\!\!\!\!A_1\times A_2 &=~& \forall X. (A_1 -> A_2 -> X) -> X \\
&\!\!\!\!\!\!\mathtt{pair} &\!\!\!:~~&
	\forall A_1^{*}.\forall A_2^{*}.A_1\times A_2
	~~=~ \l x_1.\l x_2.\l x'.x'\,x_1\,x_2 \\
&\!\!\!\!\!\!\mathtt{elim_{(\times)}} &\!\!\!:~~&
	\forall A_1^{*}.\forall A_2^{*}.A_1\times A_2 ->
	\forall X. (A_1 -> A_2 -> X) -> X \\
	& &=~& \l x.\l x'.x\;x' \\
 &&&\!\!\!\!\!\!\!\!\text{(by passing appropriate values to $x'$, we get}\\
 &&&\!\!\!\!\texttt{fst} = \l x.x(\l x_1.\l x_2.x_1),~
            \texttt{snd} = \l x.x(\l x_1.\l x_2.x_2) ~)
\end{align*} \vspace*{-19pt} \\ \vspace*{-4pt}
\rule{\linewidth}{.4pt}
\begin{align*}
&\!\!\!\!\!\!A_1+A_2 &=~&\forall X^{*}. (A_1 -> X) -> (A_2 -> X) -> X \\
&\!\!\!\!\!\!\mathtt{inl} &\!\!\!:~~& \forall A_1^{*}.\forall A_2^{*}.A_1-> A_1+A_2
	~~=~ \l x. \l x_1. \l x_2 . x_1\,x \\
&\!\!\!\!\!\!\mathtt{inr} &\!\!\!:~~& \forall A_1^{*}.\forall A_2^{*}.A_2-> A_1+A_2
	~~=~ \l x. \l x_1. \l x_2 . x_2\,x \\
&\!\!\!\!\!\!\mathtt{elim_{(+)}} &\!\!\!:~~&
	\forall A_1^{*}.\forall A_2^{*}.(A_1+A_2) -> \\
	&&& \forall X^{*}. (A_1 -> X) -> (A_2 -> X) -> X \\
	& &=~& \l x.\l x_1. \l x_2. x\;x_1\,x_2 \\
	&&&			(\textbf{case}~x~\textbf{of}~
				\{\mathtt{inl}~x' -> x_1\;x';
				  \mathtt{inr}~x' -> x_2\;x'\})
\end{align*}~\vspace*{-10pt}
\end{singlespace}
\caption{Embedding non-recursive datatypes}
\label{fig:churchnonrec}
\end{figure}
\begin{figure}
\begin{singlespace}
\begin{align*}
&\!\!\!\!\!\!\mathtt{List} &\!\!\!\!\!=~& \l A^{*}.\forall X^{*}.(A-> X-> X)-> X-> X
	\\
&\!\!\!\!\!\!\texttt{cons} &\!\!\!\!\!:~~& \forall A^{*}.A-> \mathtt{List}\,A-> \mathtt{List}\,A \\
& & & \qquad~\qquad~\quad\, =~\l x_a.\l x.\l x_c.\l x_n. x_c\,x_a\,(x\;x_c\,x_n) \\
&\!\!\!\!\!\!\mathtt{nil} &\!\!\!\!\!:~~& \forall A^{*}.\texttt{List}\,A
~~=~ \l x_c.\l x_n.\l x_n \\
&\!\!\!\!\!\!\mathtt{elim_{List}} &\!\!\!\!:~~& \forall A^{*}.\texttt{List}\,A ->
	\forall X^{*}.(A -> X -> X) -> X -> X \\
& &\!\!\!\!\!=~& \l x.\l x_c. \l x_n.x\;x_c\,x_n\qquad
	\text{(\textit{foldr} $x_z$ $x_c$ $x~$ in Haskell)}
\end{align*}\vspace*{-19pt} \\ \vspace*{-4pt}
\rule{\linewidth}{.4pt}
\begin{align*}
&\!\!\!\!\!\!\mathtt{Powl} &\!\!\!\!\!=~& \l A^{*}.\\
&&&\forall X^{*-> *}.(A-> X(A\times A)-> XA)-> XA -> XA \\
&\!\!\!\!\!\!\texttt{pcons} &\!\!\!\!\!:~~& \forall A^{*}.A-> \mathtt{Powl}(A\times A)-> \mathtt{Powl}\,A \\
&&& \qquad~\qquad~\quad\, ~=~ \l x_a.\l x.\l x_c.\l x_n. x_c\,x_a\,(x\;x_c\,x_n) \\
&\!\!\!\!\!\!\mathtt{pnil} &\!\!\!\!\!:~~& \forall A^{*}.\texttt{Powl}\,A
~~~=~ \l x_c.\l x_n.\l x_n \\
&\!\!\!\!\!\!\mathtt{elim_{Powl}} &\!\!\!\!:~~& \forall A^{*}.\texttt{Powl}\,A -> \\
&&& \forall X^{*-> *}.(A -> X(A\times A) -> XA) -> XA -> XA \\
& &\!\!\!\!\!=~& \l x.\l x_c. \l x_n.x\;x_c\,x_n
\end{align*}\vspace*{-19pt} \\ \vspace*{-4pt}
\rule{\linewidth}{.4pt}
\begin{align*}
&\!\!\!\!\!\!\mathtt{Vec} &\!\!\!\!\!\!\!\!=~& \l A^{*}.\l i^{\mathtt{Nat}}.\\
&&&	\forall X^{\mathtt{Nat}-> *}.
	(\forall i^\mathtt{Nat}.A-> X\{i\}-> X\{\mathtt{S}\,i\}) ->  \\
&&& \qquad~\qquad X\{\texttt{Z}\} -> X\{i\} \\
 &\!\!\!\!\!\!\texttt{vcons} &\!\!\!\!\!\!\!\!:~& \forall A^{*}.\forall i^\mathtt{Nat}.A-> \mathtt{Vec}\,A\,\{i\}-> \mathtt{Vec}\,A\,\{\mathtt{S}\,i\} \\
&&&\;\qquad\qquad\quad =~ \l x_a.\l x.\l x_c.\l x_n. x_c\,x_a\,(x\;x_c\,x_n) \\
&\!\!\!\!\!\!\mathtt{vnil} &\!\!\!\!\!\!\!\!:~& \forall A^{*}.\texttt{Vec}\,A\,\{\mathtt{Z}\} 
~~~=~ \l x_c.\l x_n.x_n \\
&\!\!\!\!\!\!\mathtt{elim_{Vec}} &\!\!\!\!\!\!\!\!:~& \forall A^{*}.\forall i^\mathtt{Nat}.\texttt{Vec}\,A\,\{i\} -> \\
&&& \forall X^{\mathtt{Nat}-> *}.(\forall i^\mathtt{Nat}.A -> X\{i\} -> X\{\mathtt{S}\,i\}) -> \\
&&& \qquad~\qquad X\{\mathtt{Z}\} -> X\{i\} \\
& &\!\!\!\!\!=~& \l x.\l x_c. \l x_n.x\;x_c\,x_n
\end{align*} ~\vspace*{-14pt}
\end{singlespace}
\caption{Embedding recursive datatypes}
\label{fig:churchrec}
\end{figure}
\citet{Church33} invented an embedding of the natural numbers into
the untyped $\lambda$-calculus, which he used to argue
that the $\lambda$-calculus was expressive enough for the foundation of
logic and arithmetic. Church encoded the data constructors of natural numbers,
successor and zero, as higher-order functions,
$\mathtt{succ}=\l x.\l x_s.\l x_z.x_s(x\,x_s x_z)$ and
$\mathtt{zero}=\l x_s.\l x_z.x_z$.
The heart of the Church encoding is that a value is encoded as an elimination function.
The bound variables $x_s$ and $x_z$ (of both $\mathtt{succ}$ and $\mathtt{zero}$) stand for the operations needed to
eliminate the successor case and the zero case respectively. The Church encodings of
successor states: to eliminate $\mathtt{succ}\,x$, ``apply $x_s$
to the elimination of the predecessor $(x\,x_s x_z)$"; and,
to eliminate $\mathtt{zero}$, just ``return $x_z$".
Since values {\it are} elimination functions, the
eliminator can be defined as applying the value itself to the needed operations. One
for each of the data constructors. 
For instance, we can define an eliminator
for the natural numbers as $\mathtt{elim_{Nat}}=\l x.\l x_s.\l x_z.x\,x_s x_z$.
This is just an $\eta$-expansion of the identity function $\l x.x$.
The Church encoded natural numbers are typable in a polymorphic $\lambda$-calculi,
such as System \Fw, as follows:\vspace*{-2pt}
\begin{align*}
&\texttt{Nat} &=~& \forall X^{*}.(X -> X) -> X -> X \qquad\qquad\qquad\\
&\texttt{S} &\!\!\!:~~& \texttt{Nat} -> \texttt{Nat}
	~~ =~ \l x.\l x_s.\l x_z.x_s(x\,x_s x_z) \\
&\texttt{Z} &\!\!\!:~~& \texttt{Nat} \qquad\quad\,
	~~ =~ \l x_s.\l x_z.x_z \\
&\mathtt{elim_{Nat}} &\!\!\!:~~& \texttt{Nat} -> \forall X^{*}.(X -> X)-> X-> X \\
& &=~& \l x.\l x_s.\l x_z.x\,x_s x_z
\end{align*}~\vspace*{-13pt}

In a Similar fashion, other datatypes are also embeddable into
polymorphic $\lambda$-calculi.
Embeddings of some well-known non-recursive datatypes are illustrated
in Figure \ref{fig:churchnonrec}, and embeddings of the list-like
recursive datatypes, which we discussed as motivating examples
in the beginning of this chapter, are illustrated in Figure \ref{fig:churchrec}.
Note that the term encodings for the constructors and eliminators of
the list-like datatypes in Figure \ref{fig:churchrec} are exactly the same.
For instance, the term encodings for \texttt{nil}, \texttt{pnil}, and
\texttt{vnil} are all the same term: $\l x_s.\l x_z.x_z$. The nil and cons terms
capture the linear nature of lists, so they are the same for all list like structures.
But, the types differ, capturing different invariants about lists -- shape of the elements ({\tt Powl}), and
length of the list ({\tt Vec}).

\subsection{Embedding recursive datatypes as two-level types}
\label{ssec:embedTwoLevel}
We can divide a recursive datatype definition into two parts --
a recursive type operator and a base structure. The operator ``weaves" recursion
into the datatype definition, and the base structure describes
its shape (\ie, number of data constructors and their types).
One can program with two-level types in any functional language that supports
higher-order polymorphism\footnote{a.k.a. higher-kinded polymorphism,
	or type-constructor polymorphism}, such as Haskell. 
In Figure \ref{fig:twoleveltypes}, we illustrate this by giving an example of a two level definition
for ordinary lists (all the other types in this paper have similar definitions).

The use of two-level types has been recognized as
a useful functional programming pearl \cite{Sheard04}, since two-level types
separate the two concerns of (1) recursion on recursive sub components
and (2) handling different cases (by pattern matching over the shape of the (non-recursive) base structure).
An advantage of such an approach, is that a single eliminator can be defined once for
all datatypes of the same kind. For example, the function $\mathtt{mit}_\kappa$ describes
Mendler-style iteration\footnote{An iteration is a principled recursion
	scheme guaranteed to terminate for any well-founded input.
	Also known as fold, or catamorphism.} for the recursive types
defined by $\mu_\kappa$. Although it is possible to write programs using two level datatypes
in a general purpose functional language, one could
not expect logical consistency in such systems.

\begin{figure}\vspace*{-8pt}
\begin{singlespace}
\begin{lstlisting}[basicstyle={\ttfamily\small},language=Haskell,mathescape]
newtype $\mu_{*}$ (f :: * -> *) = In$_{*}$ (f ($\mu_{*}$ f))

data ListF (a::*) (r::*) = Cons a r | Nil

type List a = $\mu_{*}$ (ListF a)
cons x xs = In$_{*}$ (Cons x xs)
nil       = In$_{*}$ Nil

mit$_{*}$ :: ($\forall$ r.(r->x) -> f r -> x) -> Mu0 f -> x
mit$_{*}$ phi (In$_{*}$ z) = phi (mit$_{*}$ phi) z

newtype $\mu_{(*-> *)}$ (f :: (*->*) -> (*->*)) (a::*)
  = In$_{(*-> *)}$ (f (Mu$_{(*-> *)}$ f)) a

data PowlF (r::*->*) (a::*) = PCons a (r(a,a)) | PNil

type Powl a = $\mu_{(*-> *)}$ PowlF a
pcons x xs = In$_{(*-> *)}$ (PCons x xs)
pnil       = In$_{(*-> *)}$ PNil

mit$_{(*-> *)}$ :: ($\forall$ r a.($\forall$a.r a->x a) -> f r a -> x a)
        -> $\mu_{(*-> *)}$ f a -> x a
mit$_{(*-> *)}$ phi (In$_{(*-> *)}$ z) = phi (mit$_{(*-> *)}$ phi) z

-- above is Haskell (with some GHC extensions)
-- below is Haskell-ish pseudocode

newtype $\mu_{(\mathtt{Nat}-> *)}$ (f::(Nat->*)->(Nat->*)) {n::Nat}
  = In$_{(\mathtt{Nat}-> *)}$ (f ($\mu_{(\mathtt{Nat}-> *)}$ f)) {n}

data VecF (a::*) (r::Nat->*) {n::Nat} where
  VCons :: a -> r n -> VecF a r {S n}
  VNil  :: VecF a r {Z}

type Vec a {n::Nat} = $\mu_{(\mathtt{Nat}-> *)}$ (VecF a) {n}
vcons x xs = In$_{(\mathtt{Nat}-> *)}$ (VCons x xs)
vnil       = In$_{(\mathtt{Nat}-> *)}$ VNil

mit$_{(\mathtt{Nat}-> *)}$::($\forall$ r n.($\forall$n.r{n}->x{n})->f r {n}->x{n})
        -> $\mu_{(\mathtt{Nat}-> *)}$ f {n} -> x{n}
mit$_{(\mathtt{Nat}-> *)}$ phi (In$_{(\mathtt{Nat}-> *)}$ z) = phi (mit$_{(\mathtt{Nat}-> *)}$ phi) z
\end{lstlisting}
\end{singlespace}
\caption{2-level types and their Mendler-style iterators in Haskell}
\label{fig:twoleveltypes}
\end{figure}

Interestingly, there exist embeddings of the recursive type operator
$\mu_\kappa$, its data constructor $\mathtt{In}_\kappa$, and
the Mendler-style iterator $\mathtt{mit}_\kappa$ for each kind $\kappa$
into the higher-order polymorphic $\lambda$-calculus \Fi, as illustrated
in Figure \ref{fig:mu}. In addition to illustrating the general form of
embedding $\mu_\kappa$, we also fully expand the embeddings for some
instances ($\mu_{*}$, $\mu_{*->*}$, $\mu_{\mathtt{Nat}->*}$), which are
used in Figure \ref{fig:twoleveltypes}. These embeddings support the embedding
of arbitrary type- and term-indexed recursive datatypes into System \Fi.
Thus we can reason about these datatypes in a logically consistent calculus.

However, it is important to note that there does not exist an embedding of the
arbitrary destruction (or, pattern matching away) of the $\mathtt{In}_\kappa$
constructor. It is known that combining arbitrary recursive datatypes with
the ability to destruct (or, unroll) their values
is powerful enough to define non-terminating computations in a type safe way,
leading to logical inconsistency. Some systems maintain consistency by restricting
which recursive datatypes can be defined, but allow arbitrary unrolling. In System
\Fi, we can define any datatype, but restrict unrolling to Mendler style operators
definable in \Fi. Such operators are quite expressive, capturing at least
iteration, primitive recursion, and courses of values recursion.

\begin{landscape}
\begin{figure}
\begin{singlespace}
\begin{multline*} \text{notation:}\quad
   \boldsymbol{\l}\mathbb{I}^\kappa.F =
	\lambda I_1^{K_1}.\cdots.\lambda I_n^{K_n}.F \qquad
   \boldsymbol{\forall}\mathbb{I}^\kappa.B =
	\forall I_1^{K_1}.\cdots.\forall I_n^{K_n}.B \qquad
   F\mathbb{I} = F I_1 \cdots I_n \qquad
   F \stackrel{\kappa}{\pmb{\pmb{->}}} G =
	\boldsymbol{\forall}\mathbb{I}^\kappa.F\mathbb{I} -> G\mathbb{I} \\
\begin{array}{lll}
\text{where}
 	& \kappa = K_1 -> \cdots -> K_n -> * & \text{and} ~~~
 	\text{$I_i$ is an index variable ($i_i$) when $K_i$ is a type,}
 		\\
 	& \mathbb{I}\,=I_1,\;\dots\;\dots\;,\;I_n& \qquad~\qquad
 	\text{a type constructor variable ($X_i$) otherwise
		(\ie, $K_i=\kappa_i$).}
\end{array}
\end{multline*} ~ \vspace*{-5pt}
\hrule  \vspace*{-2pt}
\begin{align*}
&\mu_\kappa &\!\!\!\!\!~:~~& (\kappa -> \kappa) -> \kappa
  \qquad\qquad\qquad\qquad\quad
  = \l F^{\kappa -> \kappa}.\boldsymbol{\l}\mathbb{I}^\kappa.
  \forall X^\kappa.
  (\forall {X_r}^{\!\!\kappa}.
  	(X_r \karrow{\kappa} X) ->
	(F X_r \karrow{\kappa} X)) -> X\mathbb{I} \\
&\mu_{*} &\!\!\!\!\!~:~~& (* -> *) -> * 
 \qquad\qquad\qquad\qquad\quad~
 = \l F^{* -> *}.\phantom{\boldsymbol{\l}\mathbb{I}^\kappa.}
 \forall X^{*}.(\forall {X_r}^{\!\!*}.(X_r -> X) -> (F\,X_r -> X)) -> X \\
&\mu_{*-> *} &\!\!\!\!\!~:~~& ((* -> *) -> (* -> *)) -> (* -> *) \\
&            &\!\!\!\!\!=~&
  \l F^{(*-> *) -> (*-> *)}.\l X_1^{*}.
   \forall X^{* -> *}.(\forall {X_r}^{\!\!* -> *}.
   (\forall X_1^{*}.X_r X_1 -> X X_1) -> (\forall X_1^{*}.F\,X_r X_1 -> X X_1)) -> X X_1 \\
  &\mu_{\mathtt{Nat}-> *} &\!\!\!\!\!~:~~& ((\mathtt{Nat} -> *) -> (\mathtt{Nat} -> *)) -> (\mathtt{Nat} -> *) \\
&            &\!\!\!\!\!=~&
  \l F^{(\mathtt{Nat}-> *) -> (\mathtt{Nat}-> *)}.\l i_1^\mathtt{Nat}.
  \forall X^{\mathtt{Nat} -> *}.(\forall {X_r}^{\!\!\mathtt{Nat} -> *}.
  (\forall i_1^\mathtt{Nat}.X_r i_1 -> X i_1) -> (\forall i_1^\mathtt{Nat}.F\,X_r i_1 -> X i_1)) -> X i_1 \qquad\qquad
\end{align*}
\begin{align*}
\mathtt{In}_{\kappa} \,~\,&~~:~ \forall F^{\kappa-> \kappa}.
	F(\mu_\kappa F) \karrow{\kappa} \mu_\kappa F
&&=~ \l x_r. \l x_\varphi.x_\varphi\,(\mathtt{mit}_\kappa~x_\varphi)\,x_r
	\qquad~\qquad~\qquad~\qquad~\quad \\
\mathtt{mit}_\kappa &~~:~ \forall F^{\kappa-> \kappa}.\forall X^\kappa.
	(\forall {X_r}^{\!\!\kappa}.
	 (X_r \karrow{\kappa} X) ->
	 (F X_r \karrow{\kappa} X) ) ->
	(\mu_\kappa F \karrow{\kappa} X)
&&=~ \l x_\varphi.\l x_r.x_r\,x_\varphi
\end{align*} ~ \vspace*{-10pt}
\end{singlespace}
\caption{Embedding of the recursive operators ($\mu_\kappa$),
	their data constructors ($\mathtt{In}_\kappa$),
	and the Mendler-style iterators ($\mathtt{mit}_\kappa$) in \Fi.}
\label{fig:mu}
\end{figure}
\end{landscape}

\begin{example}
The datatype of \mbox{$\lambda$-terms} in context 
\begin{verbatim}
data Lam ( C: Nat -> * ) { i: Nat } where
  LVar : C{i} -> Lam{i}
  LApp : Lam{i} -> Lam{i} -> Lam{i}
  LAbs : Lam{S i} -> Lam{i}
\end{verbatim}
is encoded as:
\[
\mathtt{Lam} \triangleq
\!\!\!
\begin{array}[t]{l}
\l C^{\mathtt{Nat}\to*}
\l i^\mathtt{Nat}.\,\forall X^{\mathtt{Nat}\to*}.
\\[1mm]
\quad
  (\forall j^\mathtt{Nat}.\,C\s j \to X\s j)
\\[1mm]
\quad\quad
 \to(\forall j^\mathtt{Nat}.\,X\s j \to X\s j \to X\s j)
\\[1mm]
\quad\quad\quad
\to(\forall j^\mathtt{Nat}.\,X\s{\mathtt S\, j} \to X\s j)
\\[1mm]
\quad\quad\quad\quad
  \to X\s i
\end{array}
\]
For a concrete representation one can consider
$\mathtt{Lam}\,\mathtt{Fin}$ where
\begin{verbatim}
data Fin { i: Nat } where
  FZ : Fin{S i}
  FS : Fin{i} -> Fin{S i}
\end{verbatim}
This is encoded as
\[
\mathtt{Fin}\triangleq
\!\!\!
\begin{array}[t]{l}
\l i^{\mathtt{Nat}}.\,\forall X^{\mathtt{Nat}\to*}.\,
\\[1mm]
\quad 
(\forall j^\mathtt{Nat}.\, X\s{\mathtt S\, j})
	\to (\forall j^\mathtt{Nat}.\, X\s j\to X\s{\mathtt S\,j})
\\[1mm]
	\qquad\to X\s i
\end{array}
\]
\end{example}
\subsection{Leibniz index equality}
\label{Leibniz}

The quantification over type-indexed kinding available in System~\Fi\
allows the definition of \emph{Leibniz-equality type} constructors
$\Eq_A: A\to A\to *$ on closed types~$A$, defined as follows:
\[\begin{array}{l}
\Eq_A \triangleq
	\l i^A.\, \l j^A.\, 
	\LEq_A\s i \s j \times \LEq_A\s j \s i
\enspace,
\\[1mm]
\text{where } 
\LEq_A \triangleq
	\l i^A.\, \l j^A.\, \forall X^{A\to*}.\, X\{i\}\to X\{j\}
\enspace.
\end{array}\]
For $F_A\in\{\Eq_A,\LEq_A\}$, 
%\[\begin{array}{c}
%\Delta;\cdot\vdash s=s':A\enspace\quad \Delta;\cdot\vdash t=t':A
%\\ \hline \\[-3mm]
%\Delta\vdash F_A \s s \s t = F_A\s{s'}\s{t'}:*
%\end{array}\] 
%and, as further basic properties, 
observe that the following types are
inhabited: 
\[\!\!\!\!\begin{array}{l}
\text{\small(Reflexive)} 
~~\,
\forall i^A.\,F_A\s{i}\s{i}
\\[1mm]
\text{\small(Transitive)} 
~~
\forall i^A.\,\forall j^A.\,\forall k^A.\,
  F_A\s{i}\s{j}\to F_A\s{j}\s{k}\to F_A\s{i}\s{k}
\\[1mm]
\text{\small(Logical)}
\quad~\, \forall i^A.\,\forall j^A.\, 
F_A\s{i}\s{j}\to \forall f^{A\to B}.\, F_B\s{f\,i}\s{f\,j}
\\[1mm]
\qquad\qquad~~~
\forall f^{A\to B}.\,\forall g^{A\to B}.\, 
F_{A\to B}\s{f}\s{g}\to 
\forall i^A.\, F_B\s{f\,i}\s{g\,i}
\end{array}\]
Hence Leibniz equality is a congruence; in that, in addition to the above
one also has the inhabitation of the type
\[\!\!\!\!\begin{array}{l}
\text{\small(Symmetric)} 
\quad
\forall i^A.\,\forall j^A.\,
  \Eq_A\s{i}\s{j}\to\Eq_A\s{j}\s{i}
\end{array}\qquad\qquad\qquad\qquad\qquad\]
%This is not so for the type 
%\begin{equation}\label{NonSymmetry}
%\forall i^A.\,\forall j^A.\, \LEq_A\s{i}\s{j}\to\LEq_A\s{j}\s{i}
%\enspace.
%\end{equation}
%(Cf.~Example~\ref{PathologicalExampleContinued} in~\S\ref{sec:theory}.)

In applications, the types~$\LEq_A$ are useful in constraining the
term-indexing of datatypes. %as parameterised by coercions.  
A general such construction is given by the type constructors $\Ran_{A,B}:
(A\to B) \to (A\to*) \to B\to *$.  These are defined as 
\[
\Ran_{A,B}
\triangleq
\l f^{A\to B}.\,
  \l X^{A\to*}.\,
    \l j^B.
      \forall i^A.\,
        \LEq_B \s j \s{f\,i}
	  \to X\s i
\]
%for closed types~$A$ and $B$, 
and are in spirit right Kan extensions, a notion that is being extensively
used in programming,~\eg~\cite{AbeMatUus05,JohannGhani08}. %,Hinze}.  .  
%%
%It follows that, %Here, 
%%for closed $t:A\to B$, $F:A\to*$, and $s:B$, a closed term $u:
%%(\Ext_{A,B}\ \s t\ F) \s s$ is a polymorphic function that, for every
%%closed $r:A$, given a generic coercion $\forall X^{B\to*}.\, X\s s \to
%%X\s{t\,r}$ provides output of type $F\s r$.  In particular, 
%the type
%$\forall f^{A\to B}.\,\forall X^{A\to*}.\,\forall i^A.\,
%(\Ext_{A,B}\ \s f\ X) \s{f\,i}\to X\s i$ is inhabited by 
%$\l f.\,f(\l x.\,x)$.
%%
%%(We note the interesting fact that the type 
%%$\forall X^{A\to*}.\,\forall j^A.\,\forall i^A.\, 
%%   X\s i \to (\Ext_{A,A}\ \s{\lambda x.\,x}\ X) \s i$
%%is inhabited by a retraction.)
%
One of their usefulness comes from the fact that the following type
is inhabited by a section
\[
\!\!\!
\begin{array}{l}
%\l h.\,\l y.\,\l g.\, h(g\, y):
\forall Y^{B\to*}.\,\forall X^{A\to*}.\,\forall f^{A\to B}.\,
\\
\big(\forall i^A.\,Y\s{f\,i}\to X\s i\big)
\to
\big(\forall j^B.\,Y\s j\to (\Ran_{A,B}\s f X)\s j\big)
\end{array}\]
This allows one to represent functions from input datatypes with
constrained indices as plain indexed functions, and vice versa.  For
instance, by means of the iterators of the previous section one can define
a vector tail function of type 
\[
\forall X^{*}.\,\forall j^{\mathtt{Nat}}.\,\mathtt{Vec}\, X\, \s j \to
\big(\Ran_{\mathtt{Nat},\mathtt{Nat}}\,\s{\mathtt S}(\mathtt{Vec}\,
X)\big)\s j 
\]
and retract it to one of type
\[
\forall X^{*}.\,\forall i^{\mathtt{Nat}}.\,\mathtt{Vec}\, X\, \s{\mathtt
S\, i} \to \mathtt{Vec}\, X\,\s i
\enspace.
\]
%
Analogously, one can use an iterator to define a single-variable
capture-avoiding substitution function of type
\[\begin{array}{l}
\forall i^{\mathtt{Nat}}.\,
(\mathtt{Lam}\,\mathtt{Fin})\s i
\\[1mm]
\quad
\to
\big(\Ran_{\mathtt{Nat},\mathtt{Nat}}
\s{\mathtt S}
(\lambda j^{\mathtt{Nat}}.\,
\mathtt{Lam}\,\mathtt{Fin}\s j
\to
\mathtt{Lam}\,\mathtt{Fin}\s j)\big)
\s i
\end{array}\]
and then retract it to one of type 
\[
\forall i^{\mathtt{Nat}}.\,
(\mathtt{Lam}\,\mathtt{Fin})\s{\mathtt S\, i}
\to
(\mathtt{Lam}\,\mathtt{Fin})\s{i}
\to
(\mathtt{Lam}\,\mathtt{Fin})\s{i}
\enspace.
\]

Type constructors ${\Lan_{A,B}:(A\to B)\to (A\to*)\to B\to *}$, which are
in spirit left Kan extensions, permit the encoding of functions of type
$(\forall i^{A}.\, F\s i\to G\s{t\,i})$, for ${F:A\to*}$, ${G:B\to*}$, and
${t:A\to B}$, as functions of type $(\forall j^{B}.\, (\Lan_{A,B}\s t F)\s
j\to G\s{j})$.  Left Kan extensions are dual to right Kan extensions, but
to define them as such one needs existential and product types.  In
formalisms without them, these have to be encoded.  This can be done as
follows: 
\[
\Lan_{A,B}
\triangleq
\!\!\!
\begin{array}[t]{l}
\l f^{A\to B}.\,
\l X^{A\to*}.\,
\l j^{B}.\, 
\\[1mm]
\quad
\forall Z^{*}.\,
  (\forall i^A.\, \LEq_B\s{f\,i}\s{j}\to X\s i\to Z)\to Z
\end{array}
\]
The type 
\[\begin{array}{l}
\forall X^{A\to*}.\,
\forall Y^{B\to*}.\,
\forall f^{A\to B}.\,
\\[1mm]
(\forall i^{A}.\, 
X\s i\to Y\s{f\,i})
\to
(\forall j^{B}.\, 
(\Lan_{A,B}\s f X)\s j\to Y\s{j})
\end{array}\]
is thus inhabited by a section, providing a retractable coercion between
the two functional representations.

Left Kan extensions come with a canonical section of type
$\forall f^{A\to B}.\,\forall X^{A\to *}.\,\forall i^A.\, X\s i \to
(\Lan_{A,B}\s f X)\s{f\,i}$ that, according to a reindexing function
$t:A\to B$, embeds an \mbox{$A$-indexed} type $F$ (at index $s$) into
the $B$-indexed type $\Lan_{A,B}\s t F$ (at index $t\,s$).  For instance, the
type constructor $\mathtt{Lan}_{A,A\times A}\s{\l x.\,\mathtt{pair}\,x\,x}$
embeds arrays of types into matrices along the diagonal; while the type
constructors $\mathtt{Lan}_{A\times A,A}\s{\mathtt{fst}}$ and
$\mathtt{Lan}_{A\times A,A}\s{\mathtt{snd}}$ respectively encapsulate matrices
of types as arrays by columns and by rows.

