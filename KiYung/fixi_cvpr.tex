\section{Embedding course-of-values primitive recursion}
\label{sec:fixi:cv}

\lstset{language=Haskell,
	basicstyle=\ttfamily\small,
%	keywordstyle=\color{ta4chameleon},
%	emph={List,Int,Bool},
	commentstyle=\color{grey},
	literate =
		{forall}{{$\forall$}}1
%		{|}{{$\mid\;\,$}}1
%		{=}{{\textcolor{ta3chocolate}{$=\,\;$}}}1
		{::}{{$:\!\,:$}}1
		{->}{{$\to$}}1
		{=>}{{$\Rightarrow$}}1
		{\\}{{$\lambda$}}1
	}


Figure~\ref{fig:embedMcvPr} and Figure~\ref{fig:unInExamples}
illustrates embeddings of the iso-recursive operator ($\mu^{+}_\kappa$) and
the course-of-values primitive recursor ($\McvPr_\kappa$), and
the roll and unroll operations ($\In_F$ and $\unIn_F$) over
a fairly large class of positive base structures ($F$) in \Fixi.
The embeddings of $\mu^{+}_\kappa$, $\McvPr_\kappa$, and $\In_F$
are very similar to the embeddings of $\mu_\kappa$, $\MPr_\kappa$,
and $\In_\kappa$ for the primitive recursion in the previous section,
but there should be an additional embedding for the unroll operation $\unIn_F$
for course-of-values primitive recursion.

\afterpage{ %%%%%%%%%%%%%%%%%%%%%%% begin afterpage
\begin{landscape}
\begin{figure}
\begin{singlespace}
\begin{align*}
\mu^{+}_\kappa &\;:~ 0(+\kappa -> \kappa) -> \kappa \\
\mu^{+}_\kappa &\triangleq
\l X_{\!F}^{0(+\kappa -> \kappa)}.\fix(\Phi^{+}_\kappa\,X_{\!F})\\
\Phi^{+}_\kappa &\;:~ 0(+\kappa -> \kappa) -> +\kappa -> \kappa \\
\Phi^{+}_\kappa &\triangleq \l X_{\!F}^{0(+\kappa -> \kappa)}.
\l X_c^{+\kappa}.\boldsymbol{\l}\mathbb{I}^\kappa.
\forall X^\kappa.
(\forall X_r^\kappa. (X_r \karrow{\kappa} X_{\!F}\,X_r)
		-> (X_r \karrow{\kappa} X_c)
		-> (X_r \karrow{\kappa} X)
		-> (X_{\!F}\,X_r \karrow{\kappa} X) ) -> X\,\mathbb{I}\\
~\\
\McvPr_\kappa &\;:~
	\forall X_{\!F}^{+\kappa-> \kappa}.\forall X^\kappa.
	(\forall {X_r}^{\!\!\kappa}.
	 (X_r \karrow{\kappa} X_{\!F}\,X_r) ->
	 (X_r \karrow{\kappa} \mu^{+}_\kappa X_{\!F}) ->
	 (X_r \karrow{\kappa} X) ->
	 (X_{\!F}\,X_r \karrow{\kappa} X) ) ->
	 (\mu^{+}_\kappa X_{\!F} \karrow{\kappa} X) \\
\McvPr_\kappa &\triangleq \l s.\l r.r\;s\\
~\\
\In_F &\;:~ F(\mu^{+}_\kappa F) \karrow{\kappa} \mu^{+}_\kappa F\\
\In_F &\triangleq
	\l t. \l s. s~\unIn_F\;\textit{id}\;\,(\McvPr_\kappa\;s)\;\,t \\
\end{align*}\vskip -2.5ex
Provided that there exists
$\unIn_{F} : \mu^{+}_\kappa F \karrow{\kappa} F(\mu^{+}_\kappa F)$
for the base structure $F:+\kappa -> \kappa$, 
such that $\unIn_F(\In_F\;t) -->+ t$
where the reduction is constant regardless of $t$
(steps may vary between each base structure $F$ though).
\end{singlespace} \vskip-3.5ex
\[\text{See Figure \ref{fig:unInExamples} for
embeddings of unroll operations ($\unIn_F$) for
some well-known positive base structures ($F$).}
\]
\caption{Embedding of the recursive type operators ($\mu^{+}_\kappa$),
	the Mendler-style course-of-values primitive recursors
	($\McvPr_\kappa$), and the roll operation ($\In_F$) in \Fixi,
        provided that the embedding of $\unIn_F$ exists}
\label{fig:embedMcvPr}
\end{figure}

\begin{figure}
\[\!\!\!\!\!\!\!
\begin{array}{llcll}
	& \text{\textbf{Regular datatypes}} \\
N &\!\!\!\triangleq \l X^{+*}.X + \textsf{Unit} &\qquad&
\unIn_N &\!\!\!\triangleq \McvPr_{*} (\l\_.\l\textit{cast}.\l\_.
	\l x. x\;(\texttt{InL}\circ\textit{cast})\;\texttt{InR})
	\\
L &\!\!\!\triangleq \l X_a^{+*}.\l X^{+*}.(X_a\times X) + \textsf{Unit} &&
\unIn_{(LA)} &\!\!\!\triangleq \McvPr_{*} (\l\_.\l\textit{cast}.\l\_.
	\l x. x\;(\texttt{InL}\circ(\textit{id}\times cast))\;\texttt{InR})
	\\
R &\!\!\!\triangleq \l X_a^{+*}.\l X^{+*}.(X_a\times \texttt{List} X) -> X &&
\unIn_{(RA)} &\!\!\!\triangleq \McvPr_{*} (\l\_.\l\textit{cast}.\l\_.
	\l x. x\;(\textit{id}\times \textit{fmap}_\texttt{List}\;cast) )
	\\
& \text{\textbf{Type-indexed datatypes}} \phantom{G^{G^{G^{G^{G^G}}}}}\\
P &\!\!\!\triangleq \l X^{+* -> *}.\l X_a^{+*}.
	X_a \times X(X_a \times X_a) + \textsf{Unit} &&
\unIn_P &\!\!\!\triangleq \McvPr_{+* -> *} (\l\_.\l\textit{cast}.\l\_.
	\l x. x \;(\texttt{InL}\circ(\textit{id}\times\textit{cast}))
		\;\texttt{InR})
	\\
B &\!\!\!\triangleq \l X^{+* -> *}.\l X_a^{+*}.
	X_a \times X(X\,X_a) + \textsf{Unit} &&
\unIn_B &\!\!\!\triangleq \McvPr_{+* -> *} (\l\_.\l\textit{cast}.\l\_.
 	\l x. x \;(\texttt{InL}\circ
 		(\textit{id}\times
 			(\textit{cast}\circ\textit{fmap}\;\textit{cast})))
 		\;\texttt{InR})
	\\
	& \text{\textbf{Term-indexed datatypes}} \phantom{G^{G^{G^{G^{G^G}}}}}
\end{array}
\]\vskip-4.5ex
\[
V \triangleq \l X_a^{*}.\l X^{\texttt{Nat} -> *}.\l i^{\texttt{Nat}}.
(\exists j^\texttt{Nat}.((i=\texttt{succ}\,j) \times X_a \times X\{j\})) +
(i=\texttt{zero})
\]
\[
\begin{array}{lll}
\texttt{VCons} &\!\!\!\triangleq \l x_a.\l x.
	\texttt{InL}(\mathtt{Ex_{Nat}}(\mathtt{Eq_{\,Nat}},x_a,x))
& : \;
\forall X_a^{*}. \forall X^{\texttt{Nat} -> *}. \forall i^\texttt{Nat}.
	X_a -> X\,\{i\} -> V\,X_a\,X\,\{\texttt{succ}\,i\}
	\\
\texttt{VNil} &\!\!\!\triangleq \texttt{InR}~\mathtt{Eq_{\,Nat}}
& : \;
\forall X_a^{*}. \forall X^{\texttt{Nat} -> *}. V\,X_a\,X\,\{\texttt{zero}\}
\end{array}
\]
\[
\unIn_{(V\,A)} \triangleq \McvPr_{\texttt{Nat} -> *}(\l\_.\l\textit{cast}.\l\_.
\l x. x \;(\texttt{InL}\circ
		(\textit{id}\times\textit{id}\times\textit{cast}))
	\;\texttt{InR})
\]
The notation $\exists j^A.B$ is a shorthand for $\exists_A(\l j^A.B)$
where $\exists_A$ is defined in Figure~\ref{fig:fixiNonRecData}.\\
$\mathtt{Ex_{A}} : \forall F^{A -> *}.\exists_A F$ and
$\mathtt{Eq_{A}} : \forall i^A.\forall j^A.(i=j)$ are
the data constructors of the existential type and the equality type.
\[\text{
See Figure \ref{fig:embedMcvPr} for the embedding of the Mendler-style
course-of-values primitive recursor ($\McvPr_\kappa$)}
\]
\caption{Embeddings of unroll operator ($\unIn_F$)
	for some well-known positive base structures ($F$).}
\label{fig:unInExamples}
\end{figure}

\end{landscape}
} %%%%%%%%%%%%%%%%%%%%%%% end of afterpage
%% [basicstyle={\ttfamily\small},language=Haskell,mathescape]

The embedding of unroll operations for some well-known positive datatypes
are illustrated in Figure~\ref{fig:unInExamples}. The idea is to use
$\McvPr_\kappa$ to define $\unIn_F$ for the base structure
$F:+\kappa -> \kappa$ without using the abstract recursive call operation
in order to be constant time. To define the unroll operation, we map
non-recursive components ($X_a$) as they are using \textit{id} and map
abstract recursive components ($X_r$) to concrete recursive components
($\mu^{+}_\kappa F$) using the abstract \textit{cast} operation provided
by the $\McvPr_\kappa$ combinator. We can embed unroll operations
for regular datatypes such as natural numbers (the base $N$) and lists
(the base $L$), type-indexed datatypes such as powerlists (the base $P$),
and term-indexed datatypes such as vectors (the base $V$) in this way.
The embeddings of $\unIn_N$ and $\unIn_L$ are self explanatory.
In fact, we can generalize \lstinline$mcvpr0 (\ _ cast _ ->  fmap cast)$
For intuitive understanding of the embedding of $\unIn_P$ and $\unIn_B$,
we provide transcriptions into Haskell in Figures~\ref{fig:HaskellunInP}
and \ref{fig:HaskellunInB}.
(see Figure~\ref{fig:HaskellFunctor1} for the defintions of \texttt{Mu1} and
\texttt{mcvpr1}). To embed unrolling operation for indexed datatypes we would
often need existential types (Figure~\ref{fig:fixiNonRecData}) and
equality types. We can encode equality types in \Fixi\ as a Leibniz equality
over indices, \ie, $(i=j) \triangleq \forall F^{A -> *}.F\{i\} -> F\{j\}$,
as discussed in \S\ref{Leibniz}.

\begin{figure}
\begin{singlespace}
\begin{lstlisting}
newtype Mu0 f = In0 { unIn0 :: f(Mu0 f) }

mcvpr0 :: Functor f => (forall r. (r -> f r) ->
                           (r -> Mu0 f) ->
                           (r -> a) ->
                           (f r -> a) )
       -> Mu0 f -> a
mcvpr0 phi = phi unIn0 id (mcvpr0 phi) . unIn0

newtype Mu1 f i = In1 { unIn1 :: f(Mu1 f)i }

mcvpr1 :: Functor1 f =>
         (forall r i'. Functor r => (forall i. r i -> f r i) ->
                              (forall i. r i -> Mu1 f i) ->
                              (forall i. r i -> a i) ->
                              (f r i' -> a i') )
       -> Mu1 f i -> a i
mcvpr1 phi = phi unIn1 id (mcvpr1 phi) . unIn1

class Functor1 h where
  fmap1  :: Functor f => (forall i j. (i -> j) -> f i -> g j)
                     -> (a -> b) -> h f a -> h g b
  -- fmap1 h = fmap1' (h id)

  fmap1' :: Functor f => (forall i. f i -> g i)
                     -> (a -> b) -> h f a -> h g b
  fmap1' h = fmap1 (\f -> h . fmap f)

instance (Functor1 h, Functor f) => Functor (h f) where
  fmap f = fmap1' id
        -- fmap1 (\f -> id . fmap f)

instance Functor (f (Mu1 f)) => Functor (Mu1 f) where
  fmap f = In1 . fmap f . unIn1
\end{lstlisting}
\end{singlespace}
\caption{$\mu_{*}$, $\McvPr_{*}$, and $\mu_{* -> *}$, $\McvPr_{* -> *}$
	transcribed into Haskell}
\label{fig:HaskellFunctor1}
\end{figure}

\begin{figure}
\begin{lstlisting}
data N r   = S r   | Z  deriving Functor
type Nat = Mu0 N
data L a r = C a r | N  deriving Functor
type List a = Mu0 (L a)
data R a r = F a [r]    deriving Functor -- relies on (Functor [])
type Rose a = Mu0 (R a)

unInN :: Mu0 N -> N(Mu0 N)
unInN = mcvpr0 (\ _ cast _ ->  fmap cast)
unInL :: Mu0(L a) -> (L a) (Mu0(L a))
unInL = mcvpr0 (\ _ cast _ ->  fmap cast)
unInR :: Mu0(R a) -> (R a) (Mu0(R a))
unInR = mcvpr0 (\ _ cast _ ->  fmap cast)
\end{lstlisting}
\label{fig:HaskellunInRegular}
\caption{Embedings of $\unIn_N$, $\unIn_{(L A)}$, $\unIn_{(R A)}$ into Haskell}
\end{figure}

\begin{figure}
\begin{singlespace}
\begin{lstlisting}
data P r i = PC i (r (i,i)) | PN
type Powl i = Mu1 P i

instance Functor1 P where
  fmap1 h f (PC x a) = PC (f x) (h (\(i,j) -> (f i,f j)) a)
  fmap1 _ _ PN = PN

unInP :: Mu1 P i -> P(Mu1 P) i
unInP = mcvpr1 (\ _ cast _ -> fmap1' cast id)
  -- mcvpr1 phi where
  --   phi _ cast _ (PC x xs) = PC x (cast xs)
  --   phi _ cast _ PN = PN
\end{lstlisting}
\end{singlespace}
\caption{Embedding of $\unIn_P$ in Figure~\ref{fig:unInExamples}
	transcribed into Haskell}
\label{fig:HaskellunInP}
\end{figure}


\begin{figure}
\begin{singlespace}
\begin{lstlisting}
data B r i = BC i (r (r i)) | BN
type Bush i = Mu1 B i

instance Functor1 B where
  fmap1 h f (BC x a) = BC (f x) (h (h f) a)
  fmap1 _ _ BN = BN

unInB :: Mu1 B i -> B (Mu1 B) i
unInB = mcvpr1 (\ _ cast _ -> fmap1' cast id)
  -- mcvpr1 phi where
  --   phi _ cast _ (BC x xs) = BC x (cast (fmap cast xs))
  --   phi _ cast _ BN = BN
\end{lstlisting}
\end{singlespace}
\caption{Embedding of $\unIn_B$ in Figure~\ref{fig:unInExamples}
	transcribed into Haskell}
\label{fig:HaskellunInB}
\end{figure}


However, not all datatypes seem to have embeddings of constant time 
unroll operations in this way, as in Figure~\ref{fig:embedMcvPr} and
Figure~\ref{fig:unInExamples}. For instance, the embeddings of
unroll operations for indirectly recursive datatypes such as
the rose tree datatype (the base $R$) are not constant due to the
use of $\textit{fmap}_\textit{List}$, which is obviously not constant
function. What we can do is to prove a meta-property that 
($\textit{fmap}_\texttt{List}\;\textit{cast}) : \texttt{List}(X_r\,X_a)
-> \texttt{List}(\mu^{+}_\kappa R\,X_a)$ can be safely optimized
to the constant time identity function since the value of \textit{cast}
is given as \textit{id} by definition of $\McvPr_\kappa$. But, that does
not mean that we have a constant time embedding of $\unIn_R$ within \Fixi.

For similar reasons, embeddings of unroll operations for
\emph{truly nested datatypes} \cite{AbeMatUus05} such as bushes
(the base $B$) are not likely to be constant either.
In the embedding of $\unIn_B$, we use
$\textit{fmap}\;\textit{cast}: (X_r(X_r\,X_a)) -> (X_r(\mu^{+}_{* -> *}B\,X_a))$
in order to cast the inner abstract recursive type $X_r\,X_a$ into
the concrete recursive type $\mu^{+}_{* -> *}B\,X_a$, before we do the outer
$\textit{cast} : (X_r(\mu^{+}_{* -> *}B\,X_a)) ->
                 (\mu^{+}_{* -> *}B(\mu^{+}_{* -> *}B\,X_a))$.
But this time, there is yet another subtlety other than worrying about
whether $\unIn_B$ is constant time. Note, we boldly assumed that
the abstract recursive type $r$ has an \textit{fmap} operation
(In Haskell, \texttt{{\bf\ttfamily Functor} r}). Previously, in the embedding
of $\unIn_R$ for the rose tree (Figure~\ref{fig:unInExamples}), we used
\textit{fmap} for \texttt{List}, which is a well known type to have
\textit{fmap}. In case of $\unIn_B$, we only know the kind of $r : +* -> *$
but nothing else since $r$ is abstract. So, we should rely on a more general
property that \textit{fmap} is well-defined for all covariant type constructors
of kind $+* -> *$. \KYA{TODO someone should have proved this? if so cite.
	but what about higher kinds?}

Note that the definition of \texttt{unInN}, \texttt{unInL}, \texttt{unInR}
in Figure~\ref{fig:HaskellunInRegular} are of the same form:
\lstinline$mvcvp0 (\ _ cast _ ->  fmap cast)$.  Also, note that
the definition of \texttt{unInP} in Figure~\ref{fig:HaskellunInP}
and \texttt{unInB} in Figure~\ref{fig:HaskellunInP} are of the same form:
\lstinline$mvcvp1 (\ _ cast _ ->  fmap1 cast id)$.
TODO We conjecture that ...

% Embeddings of unroll operations for term-indexed datatypes (e.g. vectors)
% are intuitively more simple than type-indexed datatypes (e.g. powerlists,
% bushes). Due to the erasure property, existence of unroll
% We will discuss the conditions when we can further NO, it's the other way
% around

\KYA{ TODO cite Monotone Inductive and Coinductive Constructors of Rank 2  - Matthes CSL paper }

\KYA{ TODO state a theorem about this in metatheory and
	cite some previous work on this maybe? }
\KYA{ TODO P do not need Functor requirement but $+* -> *$ kind already
        implies that so it will be true anyway}
\KYA{ TODO someone must have proved free theorems for higher-kinded cases? }


%% Apart from the limitations of constant-time undefinability of $\unIn_F$
%% discussed above, the embeddings illustrated in Figure~\ref{fig:unInExamples}
%% are not in spirit of Mendler-style. Note that the embeddings of $\unIn_F$ are
%% polytypic (different term encodings for each different $F$) rather than
%% polymorphic (one uniform term encoding whose type is polymorphic over $F$).
%% Recall that the key advantage of Mendler-style comes from being polymorhpic.
%% 
%% Fortunately, there does exists more proper Mendler-style embeddings
%% of the course-of-values combinators over arbitrary positive datatypes
%% using both iteration and coiteration schemes \cite{TODO}. Since coiteration
%% is embeddable in \Fi\ and co(-primitive-)recursion is embeddable in \Fixi,
%% the result directly applies without extending our calculi. However,
%% to our knowledge, course-of-values combinators over higher-kinded
%% type constructors (\ie, type constructors other than kind $*$) has not been
%% well studied enough, even in that setting of using both iteration/recursion
%% and coiteration/corecursion. That is, course-of-values combinators for
%% regular indirect recursive datatypes are very likely to be embeddable in
%% a calculus like \Fi\ or \Fixi\ directly applying the known results, but
%% we may need further investigation to assure ourselves for the behavior of
%% course-of-values combinators over higher-kinded datatypes.
%% 
%% We leave the search for embeddings for arbitrary positive datatypes,
%% including indirectly recursive datatypes and truly nested datatypes,
%% as future work, since coiteration and corecursion are out of the scope of
%% this dissertation.



