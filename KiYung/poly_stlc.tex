\section{The simply-typed lambda calculus}\label{sec:stlc}
\begin{figure}
\begin{singlespace}
\begin{minipage}{.46\textwidth}
	\begin{center}Church-style\end{center}
\def\baselinestretch{0}
\small
\begin{align*}
\textbf{term syntax} \\
t,s ::= &~ x           & \text{variable}    \\
      | &~ \l(x:A) . t & \text{abstraction} \\
      | &~ t ~ s       & \text{application} \\
\\
\textbf{type syntax} \\
A,B ::= &~ A -> B  & \text{arrow type} \\
      | &~ \iota   & \text{ground type}   \\
\end{align*}
\[ \textbf{typing context} \]\vspace*{-1em}
\begin{align*}\quad
\Gamma ::= &~ \cdot \\
	 | &~ \Gamma, x:A \quad (x\notin \dom(\Gamma)) \\
\end{align*}
\[ \textbf{typing rules}
	\qquad \framebox{$\Gamma |- t : A$} \]
\vspace*{-1em}
\begin{align*}
& \inference[\sc Var]{x:A \in \Gamma}{\Gamma |- x:A} \\
& \inference[\sc Abs]{\Gamma,x:A |- t : B}
		     {\Gamma |- \l(x:A).t : A -> B} \\
& \inference[\sc App]{\Gamma |- t : A -> B & \Gamma |- s : A}
		     {\Gamma |- t~s : B} \\
\end{align*}
\[ \textbf{reduction rules}
	\quad \framebox{$t --> t'$} \]
\vspace*{-1em}
\begin{align*}
& \inference[\sc RedBeta]{}{(\l(x:A).t)~s --> t[s/x]} \\
& \inference[\sc RedAbs]{t --> t'}{\l(x:A).t --> \l(x:A).t'} \\
& \inference[\sc RedApp1]{t --> t'}{t~s --> t'~s} \\
& \inference[\sc RedApp2]{s --> s'}{t~s --> t~s'} \\
\end{align*}
\end{minipage}
\begin{minipage}{.46\textwidth}
	\begin{center}Curry-style\end{center}
\def\baselinestretch{0}
\small
\begin{align*}
\textbf{term syntax} \\
t,s ::= &~ x           \\
      | &~ \l x    . t \\
      | &~ t ~ s       \\
\\
\textbf{type syntax} \\
A,B ::= &~ A -> B \\
      | &~ \iota  \\
\end{align*}
\[ \textbf{typing context} \]\vspace*{-1em}
\begin{align*}\quad
\Gamma ::= &~ \cdot \\
	 | &~ \Gamma, x:A \quad (x\notin \dom(\Gamma)) \\
\end{align*}
\[ \textbf{typing rules}
	\qquad \framebox{$\Gamma |- t : A$} \]
\vspace*{-1em}
\begin{align*}
& \inference[\sc Var]{x:A \in \Gamma}{\Gamma |- x:A} \\
& \inference[\sc Abs]{\Gamma,x:A |- t : B}
		     {\Gamma |- \l x   .t : A -> B} \\
& \inference[\sc App]{\Gamma |- t : A -> B & \Gamma |- s : A}
		     {\Gamma |- t~s : B} \\
\end{align*}
\[ \textbf{reduction rules}
	\quad \framebox{$t --> t'$} \]
\vspace*{-1em}
\begin{align*}
& \inference[\sc RedBeta]{}{(\l x   .t)~s --> t[s/x]} \\
& \inference[\sc RedAbs]{t --> t'}{\l x   .t --> \l x   .t'} \\
& \inference[\sc RedApp1]{t --> t'}{t~s --> t'~s} \\
& \inference[\sc RedApp2]{s --> s'}{t~s --> t~s'} \\
\end{align*}
\end{minipage}
~\\
\caption{Simply-typed lambda calculus in Church-style and Curry-style}
\label{fig:stlc}
\end{singlespace}
\end{figure}
We illustrate two styles of the simply-typed lambda calculus (STLC)
in Figure \ref{fig:stlc}. In both styles, a term can either be a variable,
an abstraction (\aka\ lambda term), or an application; and, a type can
either be an arrow type or a ground type. The distinction between
the two style comes from whether the abstraction has type annotation
in the term syntax. The differences in typing rules and reduction rules
between the two styles follow from this difference in the term syntax.
Both styles share exactly the same type syntax.

We first discuss two important properties of STLC
(subject reduction and strong normalization), which hold
in both Curry style and Church style (\S\ref{sec:stlc:srsn}).
Then, we discuss distinctive characteristics of each style
(\S\ref{sec:stlc:church},\S\ref{sec:stlc:curry}).
Finally, we motivate the discussion of polymorphic type systems
by reviewing the limitations of STLC (\S\ref{sec:stlc:topoly}).

\paragraph{Remark on the ground type:}
Before we discuss other properties of STLC, I'd like to make a remark on
the type syntax, in particular, on the ground type ($\iota$).
There would be no question for arrow types, which are types for functions,
since abstractions represent (certain implementations of) functions.
We need some ground types to populate the types. Otherwise, if there weren't
any ground types, we won't have any types\footnote{If we allow infinite
types, then it is possible to do away with ground types. There exist
exotic lambda calculi with infinite types, but rather uncommon.}
-- then, such version of STLC will be very uninteresting since there cannot
be any well-typed terms satisfying the typing judgement according to
the typing rules. Here, I provide the most simple ground type ($\iota$),
which does not inhabit any term (\aka\ the void type). Note that there
exist no term of type $\iota$. We can have functions over $\iota$,
such as ($\l(x:\iota).x$), the identity on $\iota$.\footnote{Here,
	I present examples in Church-style since it is more succinct than
	writing typing judgments (\eg, $(\l x.t) : \iota -> \iota$)
	in Curry-style.  But, the remark on the ground type $\iota$
	holds the same for Curry style as well.}
However, there exist no term, to which we cannot apply ($\l(x:\iota).x$).
What we can do is to apply higher-order functions
(\eg, $\l(x_{f}:\iota -> \iota).\l(x_a:\iota).x_f~x_a$, which expects
an argument of type $\iota -> \iota$) to the functions over $\iota$.

When people use STLC to model a programming language (with simple types),
they usually provide richer set of ground types other than the void type
(\eg, unit, boolean, natural numbers). In such versions of STLC with further
extensions to the type syntax, they also need to extend the term syntax
by providing normal terms for the ground types (\eg, \textsf{true} and
\textsf{false} for booleans) and eliminators (\eg, if-then-else expression for
booleans) that can examine the normal terms. Here, having just the void type
is good enough for my purpose of leading up the discussion for
the polymorphic type systems, without complicating the term syntax.

\subsection{Subject reduction and strong normalization}\label{sec:stlc:srsn}
We discuss two important properties of STLC, which holds in both
Church style and Curry style -- \emph{subject reduction} (\aka\
\emph{type preservation}) and \emph{strong normalization}.

\paragraph{Subject reduction} is a property that reduction preserves type,
as stated below:
\begin{theorem}[subject reduction]
	$\inference{\Gamma |- t : A  & t --> t'}{\Gamma |- t' : A}$
\end{theorem}
That is, when a well typed term takes a reduction step, then the reduce term
also has the same type. We can prove subject reduction (\aka\ type preservation)
by induction on the derivation of the reduction rules.
The only interesting case is the \rulename{RedBeta} rule. Proof for all
the other rules are simply done by applying the induction hypothesis.
Proof for the \rulename{RedBeta} rule amounts to proving the substitution lemma:
\begin{lemma}[substitution]
$ \inference{\Gamma,x:A |- t : B  & \Gamma |- s : A}{\Gamma |- t[s/x] : B} $
\end{lemma}
Proof of the substitution lemma is a straightforward induction on
the derivation of the typing judgement.

When people use STLC to model a programming language,
they usually consider another property called \emph{progress},
which states that well-typed terms are either values or
able to take an evaluation step. An evaluation is a reduction strategy
(\ie, certain subset of the reduction relation), which is often deterministic.
Values are terms that meet certain syntactic criteria, which is intended not
to take further evaluation steps. In such a setting, type safety means
subject reduction together with progress. However, in a calculus considering
reductions to normal terms, rather than evaluations to values, type safety is
no more than subject reduction since normal terms are irreducible by definition.

\paragraph{Strong normalization} is another important property of STLC,
when we intend to consider terms of STLC as a proof of a propositional logic
by the Curry-Howard correspondence. The proof strategy we present here is
to define the set of strongly normalizing terms, which may or may not be
well-typed, and show that all well-typed terms belong to that set
(TODO cite Matthes' CSL 04 paper on \Fixw\ and others).
We shall continue the discussion on strong normalization using
the Curry-style term syntax, but this proof strategy also works well
for the Church-style STLC (and, even generalizes to more complicated systems
such as Calculus of Constructions.
TODO cite Geuvers ``a short and flexible proof of String Normalization for CC'').
TODO We will basically use the same strategy generalized for 

A straightforward definition of the strongly normalizing set of terms would be:
\[
\inference{s_1,\ldots,s_n\in\SN}{x~s_1~\cdots~s_n \in \SN}
\qquad
\inference{t \in \SN}{\l x.t \in \SN}
\qquad
\inference{t' \in \SN & t --> t'}{t \in \SN}
\]
That is:
Variables and applications of a variable to strongly normalizing terms are
in $\SN$; abstractions are in $\SN$ when their bodies are in $\SN$; and
terms that reduce to $\SN$ are also in $\SN$.
Relying on the fact that normal order reduction (\ie, reduce outermost
leftmost redex first) of a term always leads to a normal form if the term
has a normal form, we alter the last part of inductive definition
more syntactically, and yet defining the same set $\SN$, as follows:
\[
\inference{s_1,\ldots,s_n\in\SN}{x~s_1~\cdots~s_n \in \SN}
\quad
\inference{t \in \SN}{\l x.t \in \SN}
\quad
\inference{t[s/x]~s_1~\cdots~s_n\in \SN & s\in\SN}
	{(\lambda x.t)~s~s_1~\cdots~s_n \in \SN}
\]

A set $\mathcal{A}$ is saturated when it is closed under adding
strongly normalizing neutral terms and strongly normalizing weak head expansion:
\[
\inference{s_1,\ldots,s_n\in\SN}{x~s_1~\cdots~s_n \in \mathcal{A}}
\qquad\qquad\qquad\qquad\quad
\inference{t[s/x]~s_1~\cdots~s_n\in \mathcal{A} & s\in\SN}
	  {(\l x.t)~s~s_1~\cdots~s_n \in \mathcal{A}}
\]
We can easily observe that $\SN$ is a saturated set by definition
since we can get the first and last part of the inductive definition for $\SN$
when $\mathcal{A}=\SN$. We can define an arrow operator over saturated sets
\[ \mathcal{A} -> \mathcal{B} = \{ t \in \SN \mid t~s \in \mathcal{B} ~\text{for all}~ s \in \mathcal{A} \} \]
which is also known to be saturated when both $\mathcal{A}$ and $\mathcal{B}$
are saturated.

\begin{figure}
\begin{singlespace}
\begin{description}
\item[Interpretation of types] as saturated sets of normalizing terms:
\begin{align*}
[| \iota  |] &= \bot \qquad\qquad (\text{the minimal saturated set}) \\
[| A -> B |] &= [| A |] -> [| B |]
\end{align*}

\item[Interpretation of typing contexts] as sets of valuations ($\rho$):
\[ [| \Gamma |] =
	\{\; \rho\,\in\,\dom(\Gamma) -> \SN ~\mid~
             \rho(x) \in [|\Gamma(x)|] ~\text{for all}~x\in\dom(\Gamma) \;\}
\]

\item[Interpretation of terms] as terms themselves whose free variables are
	substituted according to the given valuation ($\rho$):
\begin{align*}
[|x|]_\rho      &= \rho(x) \\
[|\l x.t|]_\rho &= \l x.[|t|]_\rho  \qquad (x\notin\dom(\rho)) \\
[|t~s|]_\rho    &= [|t|]_\rho~[|s|]_\rho
\end{align*}
\end{description}
\caption[Interpretation of STLC for proving strong normalization]
	{Interpretation of types, typing contexts, and terms of STLC
         for the proof of strong normalization}
\label{fig:interpSTLC}
\end{singlespace}
\end{figure}

We interpret types as saturated subsets (\ie, subsets that are saturated) of
$\SN$ as in Figure \ref{fig:interpSTLC}. We interpret the void type as
the minimal saturated set ($\bot$), which is saturated from the empty set.
We choose the symbol $\bot$ since saturated sets form a complete lattice
under the subset relation as the partial order. We may denote $\SN$ as $\top$
since it is the maximal element of the lattice. Note that $\bot$,
or $[|\iota|]$, does not include any abstraction since $\iota$ is not
a type of a function. Arrow types ($A -> B$)are interpreted as the arrow over
the interpretation of the domain type and the range type ($[|A|] -> [|B|]$).

We interpret a typing context as a set of valuations which map
each variable of the typing context to a term, which belongs to interpretation
of its type. That is, if $x : A \in \Gamma$ then any $\rho \in \Gamma$ should
satisfy that $\rho(x) \in [|A|]$.

The Proof of strong normalization amounts to proving the following theorem:
\begin{theorem}[soundness of typing]
$ \inference{\Gamma|- t:A & \rho \in [|\Gamma|]}{[|t|]_\rho \in [|A|]} $
\end{theorem}
\begin{proof}
We prove by induction on the typing derivation ($\Gamma|- t:A$).
\paragraph{}
For variables, it is trivial to show that
$ \inference{\Gamma |- x:A & \rho \in [|\Gamma|]}{[|x|]_\rho \in [|A|]} $.

Due to the \rulename{Var} rule, we know that $x:A \in \Gamma$.
So, $[|x|]_\rho =\rho(x)\in[|\Gamma(x)|] = [|A|]$.

\paragraph{}
For abstractions, we need to show that
$ \inference{\Gamma |- \l x.t : A -> B & \rho \in [|\Gamma|]}
	     {[|\l x.t|]_\rho \in [|A -> B|]} $.

Since $[|\l x.t|]_\rho = \l x.[|t|]_\rho$ and
$[|A -> B|] = \{ t\in \SN \mid t~s\in[|B|]~\text{for all}~s\in[|A|] \}$,
what we need to show is equivalent to the following:
\[ \inference{\Gamma |- \l x.t : A -> B & \rho \in [|\Gamma|]}
	     {\l x.[|t|]_\rho \in
		\{ t\in \SN \mid t~s\in[|B|] ~\text{for all}~ s\in[|A|] \} }
\]
By induction, we know that:
$ \inference{\Gamma,x:A |- t : B & \rho' \in [|\Gamma,x:A|]}
	     {[|t|]_{\rho'} \in [|B|]} $.

Since this holds for all $\rho' \in [|\Gamma,x:A|]$, it also holds
for particular $\rho' = \rho[x \mapsto s]$ for any $s \in [|A|]$.
So, $[|t|]_{\rho[x\mapsto s]} = ([|t|]_\rho)[s/x]\in[|B|]$ for any $s\in[|A|]$.
Since saturated sets are closed under normalizing weak head expansion,
$(\l x.[|t|]_\rho)~s \in[|B|]$ for any $s\in[|A|]$.
Therefore, $\l x.[|t|]_\rho$ is obviously in the very set,
which we wanted it to be in (\ie,
$\l x.[|t|]_\rho\in\{t\in \SN \mid t~s\in[|B|] ~\text{for all}~ s\in[|A|]\}$).

\paragraph{}
For applications, we need to show that
$ \inference{\Gamma |- t~s : B & \rho\in[|\Gamma|]}{[|t~s|]_\rho \in [|B|]} $.

By induction we know that
$
\inference{\Gamma |- t : A -> B & \rho\in[|\Gamma|]}{[|t|]_\rho \in [|A -> B|]}
\qquad
\inference{\Gamma |- s : A & \rho\in[|\Gamma|]}{[|s|]_\rho \in [|A|]}
$.

Then, it is straightforward to see that $[|t~s|]_\rho\in[|B|]$
by definition of $[|A -> B|]$.\\
\end{proof}

\begin{corollary}[strong normalization]
	$\inference{\Gamma |- t : A}{t \in \SN}$
\end{corollary}
Once we have proved the soundness of typing with respect to interpretation,
it is easy to see that STLC is strongly normalizing by giving a trivial
interpretation such that $\rho(x) = x$ for all the free variables. Note that
$[|t|]_\rho = t \in [|A|] \subset \SN$ under the trivial interpretation.



\subsection{Characteristics of the Church-style STLC}\label{sec:stlc:church}
In Church style, the variable ($x$) in an abstraction
($\l(x:A).t$) has a type annotation ($A$). Intuitively, we may think of
the abstraction ($\l(x:A).t$) as a function that expects an argument of
the type ($A$) specified by the type annotation.

There are some interesting properties that hold in Church style,
but not in Curry style. Here, I discuss two of them:
\begin{itemize}
\item \emph{Uniqueness of typing}
$\qquad\inference{\Gamma |- t : A & \Gamma |- t : A'}{A = A'} $

\item \emph{Type equivalence between well-typed $\beta$-equivalent terms}
\begin{align*}
& \inference{t =_{\beta} t' & \Gamma |- t : A & \Gamma |- t' : A'}{A = A'} \\
& \text{where $=_{\beta}$ is the reflexive symmetric transitive closure of $-->$.}
\end{align*}
\end{itemize}

\emph{Uniqueness of typing}, described in the first item above,
holds in the Church-style STLC.  More specifically, given
a well-formed typing context $\Gamma$ and a term $t$ as input,
if the term is well-typed, that is, $\Gamma |- t : A$ for some type $A$,
then that $A$ is the unique such type. We can prove this by induction on
the derivation of the typing judgment.
For variables, it trivially holds since the variables appearing in
the typing contexts are unique.
For abstractions, we use induction on the derivation.
To use the induction hypothesis we should make sure that
the typing context ($\Gamma,x:A$) and the term $t$ of the premise
is uniquely determined. It is easy to see that they are uniquely determined
since all the peaces appearing in the input (\ie, the typing context and
the term) of the premise (in particular, $\Gamma$, $A$, and $t$) are
part of the input (in particular the term $\l(x:A).t$) of the conclusion.
Therefore, by induction hypothesis, $B$ is uniquely determined, and,
as a consequence, $A -> B$ is uniquely determined.
For applications, it is easy to show by induction for each of the premise.
This proof describes the essence of the type reconstruction algorithm for
the Church-style STLC.


Proving \emph{type equivalence between well-typed $\beta$-equivalent terms},
described in the second item above, amounts to proving \emph{type equivalence
between terms before and after well-typed $\beta$-reducion
(or, $\beta$-expanion)},
described below:
\begin{align}
\inference{t --> t' & \Gamma |- t : A & \Gamma |- t' : A'}{A = A'}
	\label{eqn:welltypedarrow}
\end{align}
That is, when a well-typed term ($t$) reduces to
another well-typed term ($t'$) in single step ($t --> t'$),
the types of those two terms are identical ($A=A'$).
Since the claim above is symmetric, we can also say: when a well-typed term
($t'$) expands to another well-typed term ($t$) in single step ($t' <-- t$),
the types of those two terms are identical ($A'=A$).
Let us break down the claim (\ref{eqn:welltypedarrow}) above into two parts:
\[
\inference{t --> t' & \Gamma |- t : A}
          {\Gamma |- t' : A' ~~\text{implies}\quad A = A'} \qquad
	\begin{smallmatrix}
		\text{type equivalence between terms} \\
  		\text{before and after well-typed $\beta$-reduction}
	\end{smallmatrix}
\]
\[
\inference{t --> t' & \Gamma |- t' : A'}
          {\Gamma |- t : A \quad\text{implies}\quad A = A'} \qquad
	\begin{smallmatrix}
		\text{type equivalence between terms} \\
  		\text{before and after well-typed $\beta$-expansion}
	\end{smallmatrix}
\]~\vspace*{-3em}\\

We know that the former (type equivalence between terms before and after
well-typed $\beta$-reduction) holds because we already know of
a stronger property, subject reduction, which we discussed earlier
(repeated below) that it is one of the properties that commonly hold
in both Church style and Curry style.
\begin{align*}
\inference{\Gamma |- t : A  & t --> t'}{\Gamma |- t' : A}
 &\qquad \text{subject reduction, or type preservation}
\end{align*}

\begin{figure}
\begin{singlespace}
\begin{tabular}{lp{7cm}}
$t <-- (\l(x:\iota).x)~t$ &
relying on the property of the void type
that $\iota$ is not inhabited by any term
\\ ~ \\
$t <-- (\l(x:\iota -> \iota).t)~(\l(x:\iota).x~x)$ &
without relying on the property of $\iota$ : \par
apply a constant function to an already ill-typed term (self application)
\end{tabular}
\end{singlespace}
\caption{Examples of $\beta$-expansions to ill-typed terms}
\label{ill-typed_expand}
\end{figure}

So, what is interesting about the Church style STLC, in contrast to
the Curry-style STLC, is the latter part of the claim that terms before and
after well-typed $\beta$-expansion must always be of same type. Unlike
the former part of the claim on $\beta$-reduction, where reduced term
is always well-typed (corollary of subject reduction), we really need
the condition that expanded term is well-typed ($\Gamma |- t : A$).
This is because a well-typed term can be expanded to a ill-typed term.
In fact, we can always expand any well-typed term to ill-typed terms
(see Figure \ref{ill-typed_expand}).

The proof of (\ref{eqn:welltypedarrow}) is very simple.
Recall that we are that $t --> t'$, $\Gamma |- t : A$,
and $\Gamma |- t' : A'$; and we want to show $A = A'$.
We prove by using \emph{subject reduction} and \emph{uniqueness of typing},
as follows:
\[ \inference[(\emph{uniqueness of typing})]
	{ \inference[(\emph{subject reduction})\!\!\!]
		{t --> t' & \Gamma |- t : A}
		{\Gamma |- t' : A \phantom{a_f}} 
	& \Gamma |- t' : A' }
	{ A = A' }
\]

\subsection{Characteristics of the Curry-style STLC}\label{sec:stlc:curry}
In Curry style, there is no annotation on the variable in
an abstraction. Since the variable binding in the abstraction is no longer
fixed to a specific type, \emph{uniqueness of typing} does not hold,
unlike in Church style. For instance, the identity function ($\l x.x$)
could have one of any type that has of the form $A -> A$, such as:
\begin{quote}\vspace*{-1em}
\begin{singlespace}
$\Gamma |- \l x.x : \iota -> \iota$ \\
$\Gamma |- \l x.x : (\iota -> \iota) -> (\iota -> \iota)$ \\
$\Gamma |- \l x.x : (\iota -> (\iota -> \iota)) -> (\iota -> (\iota -> \iota))$ \\
$\Gamma |- \l x.x : ((\iota -> \iota) -> \iota) -> ((\iota -> \iota) -> \iota)$ \\
$\Gamma |- \l x.x : ((\iota -> \iota) -> (\iota -> \iota)) -> ((\iota -> \iota) -> (\iota -> \iota))$ \\
$~~~~ \vdots $
\end{singlespace}
\end{quote}
So, we read the typing judgment $\Gamma |- t : A$ in Curry style as
\begin{quote}
$t$ \emph{can have type} $A$ under the typing context $\Gamma$,
\end{quote}
unlike in Church style where we read the typing judgment as
\begin{quote}
$t$ \emph{has the type} $A$ under the typing context $\Gamma$.
\end{quote}
However, we don't consider the Curry-style STLC to be
a polymorphic type system since the typing for a variable
under a well-formed context is still unique. That is,
\[ \inference{\Gamma |- x : A & \Gamma |- x : A'}{A = A'} \]

Type equivalence between terms before and after well-typed $\beta$-reduction
(or $\beta$-expansion) does not hold either, unlike in Church style.
This is expected since uniqueness of typing does not hold in Curry style.
Even when a well-typed term reduces to another well-typed term, they may be
given different types.\footnote{Subject reduction still holds for
the Curry-style STLC since we can choose to give the same type to
the reduced term as the type before the reduction.}
Consider the reduction $(\l x'.x')(\l x.x) --> \l x.x$.
We can give different types to the term before the reduction and the
term after the reduction. For example,
\[\setpremisesend{.1em} 
\inference[\sc App]
 {
   \inference[\sc Lam]
     { \inference[\sc Var]
         {x' : \iota -> \iota ~\in~ \cdot, x' : \iota -> \iota}
         {\cdot,x' : \iota -> \iota |- x': \iota -> \iota}
     }
     {\cdot |- \l x'.x' : (\iota -> \iota) -> (\iota -> \iota)}
 &
   \inference[\sc Lam]
     {\inference[\sc Var]{x:\iota ~\in~ \cdot,x:\iota}
                         {\cdot,x:\iota |- x:\iota} \phantom{x'}}
     {\cdot |- \l x.x : \iota -> \iota \phantom{(x')}}
 }
 {\cdot |- (\l x'.x')(\l x.x) : \iota -> \iota}
\]
\[\qquad\qquad\qquad\quad
\inference[\sc Lam]
  {\inference[\sc Var]{x:\iota->\iota ~\in~ \cdot,x:\iota->\iota}
                      {\cdot,x:\iota->\iota |- x:\iota->\iota} }
  {\cdot |- \l x.x : (\iota -> \iota) -> (\iota -> \iota)}
\]

%% does the set of possible types get larger after reduction even when
%% the only ground type is the void type? I am not sure. So, I won't
%% elaborate on this.


\subsection{Motivation for polymorphic type systems}\label{sec:stlc:topoly}

%%%% need to rewrite this
%%%%%%%%%%%% Well-typed expansion does not preserve types in the Curry style.
%%%%%%%%%%%%%%%%%%%%%%%%%%%%%
%% Type preservation for $\beta$-expansion does not to hold either,
%% unlike in Church style. More fundamentally, well-typedness is not preserved
%% for inverse $\beta$-reduction. The reason for this is exactly because
%% uniqueness of typing does not hold in Curry style. Consider the following
%% $\beta$-reduction:
%% \[(\l x_I. (x_I~s_1)~(x_I~s_2)) (\l x.x) --> ((\l x.x)~s_1)~((\l x.x)~s_2)\]
%% This is a perfectly fine reduction according to the reduction rules for the
%% Curry-style STLC in Figure \ref{fig:stlc}. However, there exist $s_1$ and
%% $s_2$ such that the right-hand side is well-typed but the left-hand side
%% is not.
%% 
%% The left-hand side cannot be well-typed, regardless of $s_1$ and $s_2$.
%% The variable $x_I$ appearing in the abstraction must have a type of the form
%% $A -> A$, since the abstraction is applied to the identity term $(\l x.x)$.
%% In the body of the abstraction, which is an application
%% $(x_I~s_1)~(x_I~s_2)$, we have a problem. Since $x_I$ has an arrow type
%% that expects an argument of type $A$ and returns a result of type $A$,
%% both $(x_I~s_1)$ and $(x_I~s_2)$ have the same type $A$.  Thus,
%% the application $(x_I~s_1)~(x_I~s_2)$ cannot be well-typed, since
%% the function $(x_I~s_1)$ applied to the argument $(x_I~s_2)$ of type $A$
%% must have type $A -> A$. But, we know that $(x_I~s_1)$ has type $A$.
%% The fact that uniqueness of typing does not hold won't help us resolve
%% the contradicting typing requirements of $(x_I~s_1)$ here, because
%% the typing of a variable is still unique under a well-formed context.
%% There is no way to satisfy both $\Gamma |- x_I : A -> A$ and 
%% $\Gamma |- x_I : (A -> A) -> (A -> A)$ even in the Curry-style STLC.
%% 
%% There exist $s_1$ and $s_2$ such that the right-hand side is well-typed.
%% Actually, there are many such $s_1$ and $s_2$. I will just give two
%% instances of them. Try to justify them yourself. They are easy exercises.
%% One instance is to let $s_1 = (\l x_I . \l x . x_I~x)$, which is
%% a higher-order function, and $s_2 = (\l x. x)$. Since uniqueness typing
%% does not hold, we are able to find appropriate types for each occurrence of
%% ($\l x.x$) appearing in the right-hand side. Another instance is having
%% both of them to be the identity term, that is, let $s_1 = (\l x . x)$ and
%% $s_2 = (\l x . x)$ as well.

