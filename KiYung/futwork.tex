\chapter{Future work}\label{ch:futwork}
We summarize some ongoing and future work in this chapter:
designing a new Mendler-style recursion scheme
useful for negative datatypes \S\ref{sec:futwork:mprsi},
different fixpoint types \S\ref{sec:futwork:mu},
deriving monotonicity from polarized kinds \S\ref{sec:futwork:kindpoly}, and
kind polymorphism and kind inference \S\ref{sec:futwork:kindpoly}.

\input{futwork_mprsi} %% sec:futwork:mprsi

\input{futwork_mu} %% sec:futwork:mu

\section{Monotonicity from polarized kinds}\label{sec:futwork:mon}
We first review the summary of discussions in \S\ref{sec:fixi:cv}
and then list the future work on monotonicity and polarized kinds.

\subsection*{Summary of the discussions in \S\ref{sec:fixi:cv}}
In \S\ref{sec:fixi:cv}, we embedded Mendler-style course-of-values recursion
into System \Fixi\ assuming monotonicity. Recall that kinds are polarized in
System \Fixi. For instance, $F: p* -> *$ is a type constructor that expects
a type argument, whose polarity is $p$, and returns a type. We discussed that,
for a regular recursive datatype ($\mu_{*} F$), monotonicity amounts to
its base structure ($F:p* -> *$) being a functor. When $F$ is a functor,
there exists $\textit{fmap}_F : \forall X^{*} Y^{*}. (X -> Y) -> F X -> F Y$,
which satisfies the desired properties of a functor.

We can generalizes the concept of ``being a functor''
to type constructors of arbitrary kinds, and such
type constructors are called \emph{monotone}.
A \emph{monotonicity witness} is a generalization of $\textit{fmap}_F$,
which witnesses $F$ being a functor, and its type is called \emph{monotonicity},
denoted by $\textit{mon}_{\kappa}F$. For example, monotonicity for $F$
at kind $*$ is denoted by $mon_{*}F$, thus, $\textit{fmap}_F : mon_{*} F$.
More generally, when the type constructor has more than one argument,
there can be more than one notion of monotonicity.
For example, consider $F : p_1\kappa_1 -> p_2\kappa_2 -> *$.
We say
that $F$ is monotone on its first argument
when $(X_1 -> X_2)$ implies $(F X_1 Y -> F X_2 Y)$, and,
that $F$ is monotone on its second argument
when $(Y_1 -> Y_2)$ implies $(F X Y_1 -> F X Y_2)$.
One possible notion of monotonicity for $F$ is requiring only the first
argument to be monotone. Another possible notion is requiring both of
the arguments to be monotone.

We discussed in \S\ref{sec:fixi:cv} that there are more than one notion of
monotonicity witness at hinger kinds. For a non-regular recursive type
($\mu_{p* -> *} F$), where $F : p_r(p* -> *) -> (p* -> *)$, there are
two different notions of monotonicity.\vspace*{-1.5em}
\begin{singlespace}
\begin{align*}
\textit{mon}_{p* -> *}F =~&
	\forall G_1^{p* -> *}.\forall G_2^{p* -> *}.
	(\forall X. G_1 X -> G_2 X) -> (\forall X.F\,G_1 X -> F\,G_2 X)
	\\[1mm]
\textit{mon}_{p* -> *}'F =~&
	\forall G_1^{p* -> *}.\forall G_2^{p* -> *}.
		mon_{*} G_1 -> \\ & \qquad\qquad\qquad\quad
		(\forall X^{*}. G_1 X -> G_2 X) -> \\ & \qquad\qquad\qquad\quad
		\forall X_1^{*}.\forall X_2^{*}.
		(X_1 -> X_2) -> F\,G_1 X_1 -> F\,G_2 X_2
\end{align*}
\end{singlespace}
The former, $\textit{mon}_{p* -> *}F$, requires $F$ to be monotone
on its first argument, which is the recursive argument.
We discussed that $\textit{mon}_{p* -> *}F$ is sufficient
for the embedding of \McvPr\ over non-truly nested datatypes,
such as powerlists.

The latter, $\textit{mon}_{p* -> *}F$, requires $F$ to be monotone
on both arguments (\ie, both the recursive argument and the index argument).
We discussed that we need this stronger notion of monotonicity
to embed \McvPr\ over truly nested datatypes, such as bushes,
whose index involves the recursive argument in its definition.

\subsection*{Future work deriving monotonicity from polarized kinds.}
According to the embedding of $\McvPr_\kappa$ in \S\ref{sec:fixi:cv},
one needs to witness monotonicity of $F$ to ensure that $\McvPr_\kappa$
always terminates for $(\mu_\kappa F)$-values. That is, one must show
the existence of $mon_{\kappa}F$ with the desired properties to use
$\McvPr_\kappa$ with termination guarantee. However, it is not desirable
to require programmers to manually derive $mon_{\kappa}F$ for each $F$.
It more desirable for the language implementation to automatically
derive monotonicity witness for $F$. It would be even better if
the language type system can guarantee existence of monotonicity witness
by examining the polarized kind of $F$, rather than actually deriving
monotonicity for $F$ by examining its definition.

For System \F, it is known that $mon_{*}F$ exists for any positive $F$
(\ie, $F:+* -> *$ if given a polarized kind) \cite{Mat99}. However,
it is still an open question whether any $F:+* -> *$ is monotone
in higher-order polymorphic calculi, such as \Fixi. In \S\ref{sec:fixi:cv},
We proved that $mon_{*}F$ exists for a certain class of $F:+* -> *$,
and proof for any $F:+* -> *$ in \Fixi\ is left for future work.

Every for datatype

In \S\ref{sec:fixi:cv},
we proved that $F:+* -> *$
In $mon_{*}F$ exists

Appropriate monotonicity

TODO \\ 
TODO \\ 
TODO \\ 
TODO \\ 
TODO \\ 
TODO \\ 
TODO \\ 
TODO \\ 
TODO \\ 
TODO \\ 
TODO \\ 
TODO \\ 
TODO \\ 
TODO \\ 
TODO \\ 


\section{Kind polymorphism and kind inference}\label{sec:futwork:kindpoly}
TODO


TODO \\ TODO \\ TODO \\ TODO \\ TODO \\ TODO \\ TODO \\ TODO \\ TODO \\ 
TODO \\ TODO \\ TODO \\ TODO \\ TODO \\ TODO \\ TODO \\ TODO \\ TODO \\ 
TODO \\ TODO \\ TODO \\ TODO \\ TODO \\ TODO \\ TODO \\ TODO \\ TODO \\ 
TODO \\ TODO \\ TODO \\ TODO \\ TODO \\ TODO \\ TODO \\ TODO \\ TODO \\ 
TODO \\ TODO \\ TODO \\ TODO \\ TODO \\ TODO \\ TODO \\ TODO \\ TODO \\ 
TODO \\ TODO \\ TODO \\ TODO \\ TODO \\ TODO \\ TODO \\ TODO \\ TODO \\ 



