\chapter{Future work}\label{ch:futwork}
\index{Mendler-style}
We summarize some ongoing and future work in this chapter:
designing a new Mendler-style recursion scheme
useful for negative datatypes \S\ref{sec:futwork:mprsi},
different fixpoint types \S\ref{sec:futwork:mu},
deriving monotonicity from polarized kinds \S\ref{sec:futwork:kindpoly}, and
kind polymorphism and kind inference \S\ref{sec:futwork:kindpoly}.

\input{futwork_mprsi} %% sec:futwork:mprsi

\input{futwork_mu} %% sec:futwork:mu

\section{Monotonicity from polarized kinds}\label{sec:futwork:mon}
\index{monotonicity}
\index{polarized kind}
\index{kind!polarized}
We first review the summary of discussions in \S\ref{sec:fixi:cv}
and then list the future work on monotonicity and polarized kinds.

\subsection*{Summary of the discussions in \S\ref{sec:fixi:cv}}
\index{Mendler-style!course-of-values recursion}
\index{System Fixi@System \Fixi}
In \S\ref{sec:fixi:cv}, we embedded Mendler-style course-of-values recursion
into System \Fixi\ assuming monotonicity. Recall that kinds are polarized in
System \Fixi. For instance, $F: p* -> *$ is a type constructor that expects
a type argument, whose polarity is $p$, and returns a type. We discussed that,
for a regular recursive datatype ($\mu_{*} F$), monotonicity amounts to
its base structure ($F:p* -> *$) being a functor. When $F$ is a functor,
there exists $\textit{fmap}_F : \forall X^{*} Y^{*}. (X -> Y) -> F X -> F Y$,
which satisfies the desired properties of a functor.

We can generalizes the concept of ``being a functor''
to type constructors of arbitrary kinds, and such
type constructors are called \emph{monotone}.
\index{monotonicity!witness}
A \emph{monotonicity witness} is a generalization of $\textit{fmap}_F$,
which witnesses $F$ being a functor, and its type is called \emph{monotonicity},
denoted by $\textit{mon}_{\kappa}F$. For example, monotonicity for $F$
at kind $*$ is denoted by $mon_{*}F$, thus, $\textit{fmap}_F : mon_{*} F$.
More generally, when the type constructor has more than one argument,
there can be more than one notion of monotonicity.
For example, consider $F : p_1\kappa_1 -> p_2\kappa_2 -> *$.
We say
that $F$ is monotone on its first argument
when $(X_1 -> X_2)$ implies $(F X_1 Y -> F X_2 Y)$, and,
that $F$ is monotone on its second argument
when $(Y_1 -> Y_2)$ implies $(F X Y_1 -> F X Y_2)$.
One possible notion of monotonicity for $F$ is requiring only the first
argument to be monotone. Another possible notion is requiring both of
the arguments to be monotone.

We discussed in \S\ref{sec:fixi:cv} that there are more than one notion of
monotonicity witness at hinger kinds. For a non-regular recursive type
($\mu_{p* -> *} F$), where $F : p_r(p* -> *) -> (p* -> *)$, there are
two different notions of monotonicity.\vspace*{-1.5em}
\begin{singlespace}
\begin{align*}
\textit{mon}_{p* -> *}F =~&
	\forall G_1^{p* -> *}.\forall G_2^{p* -> *}.
	(\forall X. G_1 X -> G_2 X) -> (\forall X.F\,G_1 X -> F\,G_2 X)
	\\[1mm]
\textit{mon}_{p* -> *}'F =~&
	\forall G_1^{p* -> *}.\forall G_2^{p* -> *}.
		mon_{*} G_1 -> \\ & \qquad\qquad\qquad\quad
		(\forall X^{*}. G_1 X -> G_2 X) -> \\ & \qquad\qquad\qquad\quad
		\forall X_1^{*}.\forall X_2^{*}.
		(X_1 -> X_2) -> F\,G_1 X_1 -> F\,G_2 X_2
\end{align*}
\end{singlespace}
The former, $\textit{mon}_{p* -> *}F$, requires $F$ to be monotone
on its first argument, which is the recursive argument.
We discussed that $\textit{mon}_{p* -> *}F$ is sufficient
for the embedding of \McvPr\ over non-truly nested datatypes,
such as powerlists.
\index{powerlist}

The latter, $\textit{mon}_{p* -> *}F$, requires $F$ to be monotone
on both arguments (\ie, both the recursive argument and the index argument).
\index{truly nested datatype}
\index{datatype!nested}
\index{bush}
We discussed that we need this stronger notion of monotonicity
to embed \McvPr\ over truly nested datatypes, such as bushes,
whose index involves the recursive argument in its definition.

\subsection*{Future work on deriving monotonicity from polarized kinds.}
\index{monotonicity}
\index{polarized kind}
\index{kind!polarized}
According to the embedding of $\McvPr_\kappa$ in \S\ref{sec:fixi:cv},
one needs to witness monotonicity of $F$ to ensure that $\McvPr_\kappa$
always terminates for $(\mu_\kappa F)$-values. That is, one must show
the existence of $mon_{\kappa}F$ with the desired properties to use
$\McvPr_\kappa$ with termination guarantee. However, it is not desirable
to require programmers to manually derive $mon_{\kappa}F$ for each $F$.
It more desirable for the language implementation to automatically
derive monotonicity witness for $F$. It would be even better if
the language type system can guarantee existence of monotonicity witness
by examining the polarized kind of $F$, rather than actually deriving
monotonicity for $F$ by examining its definition.

For System \F, it is known that $mon_{*}F$ exists for any positive $F$
(\ie, $F:+* -> *$ if given a polarized kind) \cite{Mat99}. However,
it is still an open question whether any $F:+* -> *$ is monotone
in higher-order polymorphic calculi, such as \Fixi. In \S\ref{sec:fixi:cv},
We proved that $mon_{*}F$ exists for a certain class of $F:+* -> *$,
and proof that $mon_{*}F$ exists for any $F:+* -> *$ in \Fixi\ is
left for future work.

We discussed that there are two notions of monotonicity at kind $p* -> *$,
one ($\textit{mon}_{p* -> *}F$) for non-truly nested datatypes and
the other ($\textit{mon'}_{p* -> *}F$) for truly nested datatypes.
We conjecture that $\textit{mon}_{p* -> *}F$ exists for any non-truly nested
$F:+(p* -> *) -> (p* -> *)$, and, that $\textit{mon'}_{+* -> *}F$ exists
for any $F:+(+* -> *) -> (+* -> *)$. Proofs for such conjectures
at higher kinds are also future work.

\section{Kind polymorphism and kind inference}\label{sec:futwork:kindpoly}
\index{kind polymorphism}
\index{kind inference}
We support rank-1 polymorphism at kind level in our Nax implementation.
(\eg, \textit{Path} in Figure~\ref{fig:glist} on p\pageref{fig:glist},
      \textit{Env} in Figure~\ref{fig:env} on p\pageref{fig:env}).
However, our theories (System \Fi\ and \Fixi) do not have kind polymorphism.
We strongly believe that rank-1 polymorphism at kind level do not affect
normalisation and consistency, but needs further investigation to confirm
our belief.

In the Nax implementation, types are mostly inferred but kinds are always
annotated. For example, we must annotate $\kappa$ in $\mu_{[\kappa]}$ and
$\In_{[\kappa]}$, in addition to the kind annotations in datatype declarations
($\mathbf{data}\;F : \kappa\;\mathbf{where} \cdots$).
We believe we only need kind annotations in datatype declarations.
We can omit the kind annotations in $\mu$ and because $\mu$ is always
followed by a type constructor, $\mu F$, and we can always infer
the kind of $F$. Similarly, we might be able omit or simplify
the kind annotation in $\In$, because $\In$ is always followed by a term,
$\In t$, and we can infer the type of $t$, and we might have
enough information to infer the kind for $\In$.

\index{kind inference}
\index{fixpoint derivation}
We are also working on a new implementation of Nax with better syntax
that support better kind inference and non-ambiguous fixpoint derivation.
In our new implementation, the kind is inferred for $\mu$ without
any annotation. We have not found a good way to completely infer the kind
for $\In$, but we found out that it is enough to specify the arity of the kind.
That is, write $\In_n$ instead of $\In_{[\kappa]}$ where $n$ is the arity of
$\kappa$. For example, in the new syntax, we write $\In_3$ instead of
$\In_{[* -> * -> * -> *]}$, $\In_{[* -> (* -> *) -> * -> *]}$, or
$\In_{[* -> \{\texttt{Nat}\} -> * -> *]}$, which is much more succinct,
especially for larger arities. Another change to the syntax is on
the datatype declaration of the GADT form. The syntax in our dissertation
($\mathbf{data}\;F : \kappa\;\mathbf{where} \cdots$) can be ambiguous
to derive fixpoint of $F$. For example, when $F : (* -> *) -> * -> *$,
we can either take fixpoint of $\mu F : * -> *$, or $\mu (F\;t) : *$
for some $t : *$. In the current syntax, we simply choose the longest match,
that is $\mu F : * -> *$. In the new syntax, we change the datatype declaration
syntax, similar to the syntax of Agda's, to clearly distinguish
parameter arguments from index arguments.
\index{parameter}
\index{index}
For instance,
($\mathbf{data}\;F_1\;r: * -> * \;\mathbf{where} \cdots$) or
($\mathbf{data}\;F_0\;t\;r: * \;\mathbf{where} \cdots$), where
parameter arguments ($t$, $r$) appear on the left of the colon ($:$).
Then, we can derive fixpoints without ambiguity, always on the last
parameter argument, for instance, we would derive $\mu F_1 : * -> *$
and $\mu(F_0\;t): *$.

