\section{System \Fixi} \label{sec:fixi:def}
The syntax and rules of System~\Fi\ are described in
Figures \ref{fig:Fixi}, \ref{fig:Fixi2} and~\ref{fig:eqFixi}.
The extensions new to System~\Fixi\ that are not original parts of
either System~\Fw\ or System~\Fixw\ are highlighted by either
\dbox{dashed boxes} or \newFi{\text{grey boxes}}, respectively.

The extensions that not originally part of System~\Fixw\ are highlighted
by \newFi{\text{grey boxes}}. Those extensions support term indexing.
Eliding all the grey boxes from Figures~\ref{fig:Fixi}, \ref{fig:Fixi2},
and~\ref{fig:eqFixi}, one obtains a version of System~\Fixw\ with
typing contexts separated into two parts.\footnote{
	The original description of \Fixw \cite{AbeMat04} has
	one combined typing context.}

The extensions that are not originally part of System \Fw\ but
present in System~\Fixw\ are highlighted by \dbox{dashed boxes}.
Those extensions support equi-recursive types. Eliding all the dashed boxes,
as well as all the grey boxes, from Figures~\ref{fig:Fixi}, \ref{fig:Fixi2},
and~\ref{fig:eqFixi}, one obtains the Curry-style System \Fw\ with
typing contexts separated into two parts.


\begin{figure}\begin{singlespace}
	\small
\paragraph{Syntax:}
\begin{align*}
\!\!\!\!\!\!\!\!&\text{Sort}
 	& \square
	\\
\!\!\!\!\!\!\!\!&\text{Term Variables}
 	& x,i
\\
\!\!\!\!\!\!\!\!&\text{Type Constructor Variables}
 	& X
\\
\!\!\!\!\!\!\!\!&\text{\dbox{Polarities}}
	& \dbox{$p$} &~ ::= + \mid - \mid 0
\\
\!\!\!\!\!\!\!\!&\text{Kinds}
 	& \kappa		&~ ::= ~ *
				\mid \dbox{$p\kappa$} -> \kappa
				\mid \newFi{A -> \kappa}
\\
\!\!\!\!\!\!\!\!&\text{Type Constructors}
	& A,B,F,G		&~ ::= ~ X
				\mid A -> B
				\mid \dbox{$\fix F$} \\ &&& ~\quad
				\mid \lambda \dbox{$X^{p\kappa}$}.F
				\mid F\,G
				\mid \forall X^\kappa . B \\ &&& ~\quad
				\mid \newFi{\lambda i^A.F
				\mid F\,\{s\}
				\mid \forall i^A . B}
\\
\!\!\!\!\!\!\!\!&\text{Terms}
	& r,s,t			&~ ::= ~ x \mid \lambda x.t \mid r\;s
\\
\!\!\!\!\!\!\!\!&\text{Typing Contexts}
	& \Delta		&~ ::= ~ \cdot
				\mid \Delta,\dbox{$X^{p\kappa}$}
				\mid \newFi{\Delta, i^A} \\
&	& \Gamma		&~ ::= ~ \cdot
				\mid \Gamma, x : A
\end{align*}
\paragraph{Reduction:} \fbox{$t \rightsquigarrow t'$}
\[ 
   \inference{}{(\lambda x.t)\,s \rightsquigarrow t[s/x]}
 ~~~~
   \inference{t \rightsquigarrow t'}{\lambda x.t \rightsquigarrow \lambda x.t'}
 ~~~~
   \inference{r \rightsquigarrow r'}{r\;s \rightsquigarrow r'\;s}
 ~~~~
   \inference{s \rightsquigarrow s'}{r\;s \rightsquigarrow r\;s'}
\]
~\\
\end{singlespace}
\caption{Syntax and Reduction rules of \Fixi.}
\label{fig:Fixi}
\end{figure}

\begin{figure}\begin{singlespace}\small
\paragraph{Well-formed typing contexts:}
\[ \fbox{$|- \Delta$}
 ~~~~
   \inference{}{|- \cdot}
 ~~~
   \inference{|- \Delta & |- \kappa:\square}{|- \Delta,\dbox{$X^{p\kappa}$}}
      \big( X\notin\dom(\Delta) \big)
 ~~~ \newFi{
   \inference{|- \Delta & \cdot |- A:*}{|- \Delta,i^A}
      \big( i\notin\dom(\Delta) \big) }
\]
$ \fbox{$\Delta |- \Gamma$}
 ~~~~
   \inference{|- \Delta}{\dbox{$0\Delta$} |- \cdot}
 ~~~~
   \inference{\Delta |- \Gamma & \Delta |- A:*}{
              \Delta |- \Gamma,x:A}
      \big( x\notin\dom(\Gamma) \big)
$ \vskip1ex ~
\paragraph{Sorting:} \fbox{$|- \kappa : \square$}
$ \quad
  \inference[($A$)]{}{|- *:\square}
 ~~~
   \inference[($R$)]{|- \kappa:\square & |- \kappa':\square}{
		     |- \dbox{$p\kappa$} -> \kappa' : \square}
 ~~~
   \newFi{
   \inference[($Ri$)]{\cdot |- A:* & |- \kappa:\square}{
                      |- A -> \kappa : \square} }
$
~\\
\paragraph{Kinding:} \fbox{$\Delta |- F : \kappa$}
$ \qquad
   \inference[($Var$)]{\dbox{$X^{p\kappa}$}\in\Delta & |- \Delta}{
 		       \Delta |- X : \kappa}
		\, \dbox{$(p\in \{+,0\})$} $
\[
   \inference[($->$)]{\dbox{$-\Delta$} |- A : * & \Delta |- B : *}{
                      \Delta |- A -> B : * }
 \qquad \qquad \dbox{
   \inference[($\fix$)]{\Delta |- F : +\kappa -> \kappa}{
		      \Delta |- \fix F : \kappa } }
\]
\[
  \inference[($\lambda$)]{ |- \kappa:\square
                         & \Delta,\dbox{$X^{p\kappa}$} |- F:\kappa'}{
  	\Delta |- \lambda \dbox{$X^{p\kappa}$}.F : \dbox{$p\kappa$} -> \kappa'}
 ~~~~ \qquad
 \newFi{
  \inference[($\lambda i$)]{\cdot |- A:* & \Delta,i^A |- F : \kappa}{
			    \Delta |- \lambda i^A.F : A->\kappa} }
\]
\[
  \inference[($@$)]{ \Delta |- F : \dbox{$p\kappa$} -> \kappa'
		    & \dbox{$p\Delta$} |- G : \kappa }{
                     \Delta |- F\,G : \kappa'}
 ~~~~
 \newFi{
   \inference[($@i$)]{ \Delta |- F : A -> \kappa
   		     & \dbox{$0\Delta$\!};\cdot |- s : A }{
		      \Delta |- F\,\{s\} : \kappa} }
\]
\[
   \inference[($\forall$)]{ |- \kappa:\square
   			  & \Delta,\dbox{$X^{0\kappa}$} |- B : *}{
                           \Delta |- \forall X^\kappa . B : *}
 ~~~~ \qquad ~\,
	\newFi{
   \inference[($\forall i$)]{\cdot |- A:* & \Delta, i^A |- B : *}{
                             \Delta |- \forall i^A . B : *} }
\]
\[ \newFi{
   \inference[($Conv$)]{ \Delta |- A : \kappa
                       & \Delta |- \kappa = \kappa' : \square }{
                        \Delta |- A : \kappa'} }
\]
~\\
\paragraph{Typing:} \fbox{$\Delta;\Gamma |- t : A$}
$ \qquad
 ~~~~
 \inference[($:$)]{(x:A) \in \Gamma & \Delta |- \Gamma}{
                   \Delta;\Gamma |- x:A}
 ~~~~ \newFi{
   \inference[($:i$)]{i^A \in \Delta & \Delta |- \Gamma}{
                      \Delta;\Gamma |- i:A} }
$
\[
   \inference[($->$$I$)]{\Delta |- A:* & \Delta;\Gamma,x:A |- t : B}{
                         \Delta;\Gamma |- \lambda x.t : A -> B}
 ~~~~ ~~~~
   \inference[($->$$E$)]{\Delta;\Gamma |- r : A -> B & \Delta;\Gamma |- s : A}{
                         \Delta;\Gamma |- r\;s : B}
\]
\[ \inference[($\forall I$)]{|- \kappa:\square
			    & \Delta,\dbox{$X^{0\kappa}$\!};\Gamma |- t : B}{
                             \Delta;\Gamma |- t : \forall X^\kappa.B}
			    (X\notin\FV(\Gamma))
 ~~~~ ~~~~
   \inference[($\forall E$)]{ \Delta;\Gamma |- t : \forall X^\kappa.B
                            & \Delta |- G:\kappa }{
                             \Delta;\Gamma |- t : B[G/X]}
\]
\[ \newFi{
   \inference[($\forall I i$)]{\cdot |- A:* & \Delta, i^A;\Gamma |- t : B}{
                               \Delta;\Gamma |- t : \forall i^A.B}
   \left(\begin{matrix}
		i\notin\FV(t),\\
		i\notin\FV(\Gamma)\end{matrix}\right)
 ~~~~
   \inference[($\forall E i$)]{ \Delta;\Gamma |- t : \forall i^A.B
                              & \Delta;\cdot |- s:A}{
                               \Delta;\Gamma |- t : B[s/i]} }
\]
\[ \inference[($=$)]{\Delta;\Gamma |- t : A & \Delta |- A = B : *}{
                     \Delta;\Gamma |- t : B}
\]
\end{singlespace}
\caption{Sorting, Kinding, and Typing rules of \Fixi.}
\label{fig:Fixi2}
\end{figure}

\begin{figure}\begin{singlespace}\small
\paragraph{Kind equality:} \fbox{$|- \kappa=\kappa' : \square$}
$ \quad
 ~~~~
   \inference{}{|- * = *:\square} $
\[
   \inference{ |- \kappa_1 = \kappa_1' : \square
             & |- \kappa_2 = \kappa_2' : \square }{
   |- \dbox{$p\kappa_1$}-> \kappa_2 = \dbox{$p\kappa_1'$}-> \kappa_2' : \square}
 ~~~~ \newFi{
   \inference{\cdot |- A=A':* & |- \kappa=\kappa':\square}{
              |- A -> \kappa = A' -> \kappa' : \square} }
\]
\[ \inference{|- \kappa=\kappa':\square}{
              |- \kappa'=\kappa:\square}
 ~~~~
   \inference{ |- \kappa =\kappa' :\square
             & |- \kappa'=\kappa'':\square}{
              |- \kappa=\kappa'':\square}
\]
~
\paragraph{Type constructor equality:} \fbox{$\Delta |- F = F' : \kappa$}
$\qquad \dbox{$
  \inference{\Delta|- F:+\kappa-> \kappa}{\Delta|- \fix F=F(\fix F):\kappa}$ } $
\[
   \inference{ \Delta,\dbox{$X^{p\kappa}$} |- F:\kappa'
   	     & \dbox{$p\Delta$} |- G:\kappa}{
	      \Delta |- (\lambda X^{p\kappa}.F)\,G = F[G/X]:\kappa'}
 ~~~~ \newFi{
   \inference{ \Delta,i^A |- F:\kappa
             & \dbox{$0\Delta$\!};\cdot |- s:A}{
              \Delta |- (\lambda i^A.F)\,\{s\} = F[s/i]:\kappa} }
\]
\[ \inference{\Delta |- X:\kappa }{\Delta |- X=X:\kappa}
 ~~~~
   \inference{-\Delta |- A=A':* & \Delta |- B=B':*}{\Delta |- A-> B=A'-> B':*}
 ~~~~
   \inference{\Delta |- F=F' : +\kappa -> \kappa}{
	      \Delta |- \fix F = \fix F' : \kappa}
\]
\[
   \inference{ |- \kappa:\square
   	     & \Delta,\dbox{$X^{p\kappa}$} |- F=F' : \kappa'}{
   	\Delta |- \lambda \dbox{$X^{p\kappa}$}.F
		= \lambda \dbox{$X^{p\kappa}$}.F': \dbox{$\kappa$} -> \kappa'}
 ~~~~ ~
 \newFi{
   \inference{\cdot |- A:* & \Delta,i^A |- F=F' : \kappa}{
	      \Delta |- \lambda i^A.F=\lambda i^A.F' : A -> \kappa} }
\]
\[\!\!\!\!\!\!\!\!\!\!\!\!
   \inference{ \Delta |- F=F' : \dbox{$p\kappa$} -> \kappa'
   	     & \dbox{$p\Delta$} |- G=G':\kappa}{
              \Delta |- F\,G = F'\,G' : \kappa'}
 ~~~~
 \newFi{
   \inference{ \Delta |- F=F': A -> \kappa
             & \dbox{$0\Delta$\!};\cdot |- s=s':A}{
	      \Delta |- F\,\{s\} = F'\,\{s'\} : \kappa} }
\]
\[
   \inference{ |- \kappa:\square
   	     & \Delta,\dbox{$X^{0\kappa}$} |- B=B':*}{
              \Delta |- \forall X^\kappa.B=\forall X^\kappa.B':*}
 ~~~~ \quad
 \newFi{
   \inference{\cdot |- A:* & \Delta,i^A |- B=B':*}{
              \Delta |- \forall i^A.B=\forall i^A.B':*} }
\]
\[ \inference{\Delta |- F = F' : \kappa}{\Delta |- F' = F : \kappa}
 ~~~~
   \inference{\Delta |- F = F' : \kappa & \Delta |- F' = F'' : \kappa}{
              \Delta |- F = F'' : \kappa}
\]
~
\paragraph{Term equality:} \fbox{$\Delta;\Gamma |- t = t' : A$}
\[
   \inference{\Delta;\Gamma,x:A |- t:B & \Delta;\Gamma |- s:A}{
              \Delta;\Gamma |- (\lambda x.t)\,s=t[s/x] : B}
 ~~~~
   \inference{\Delta;\Gamma |- x:A}{\Delta;\Gamma |- x=x:A}
\]
\[ \inference{\Delta |- A:* & \Delta;\Gamma,x:A |- t=t':B}{
              \Delta;\Gamma |- \lambda x.t = \lambda x.t':B}
 ~~~~
   \inference{\Delta;\Gamma |- r=r':A-> B & \Delta;\Gamma |- s=s':A}{
              \Delta;\Gamma |- r\;s=r'\;s':B}
\]
\[ \inference{ |- \kappa:\square
	     & \Delta, \dbox{$X^{0\kappa}$\!};\Gamma |- t=t' : B}{
              \Delta;\Gamma |- t=t' : \forall X^\kappa.B}
	     (X\notin\FV(\Gamma))
 ~~~~ ~~~~
   \inference{ \Delta;\Gamma |- t=t' : \forall X^\kappa.B
             & \Delta |- G:\kappa }{
              \Delta;\Gamma |- t=t' : B[G/X]}
\]
\[ \newFi{
   \inference{\cdot |- A:* & \Delta, i^A;\Gamma |- t=t' : B}{
              \Delta;\Gamma |- t=t' : \forall i^A.B}
   \left(\begin{smallmatrix}
		i\notin\FV(t),\\
		i\notin\FV(t'),\\
		i\notin\FV(\Gamma)\end{smallmatrix}\right)
 ~~~~
   \inference{ \Delta;\Gamma |- t=t' : \forall i^A.B
             & \Delta;\cdot |- s:A}{
              \Delta;\Gamma |- t=t' : B[s/i]} }
\]
\[ \inference{\Delta;\Gamma |- t=t':A}{\Delta;\Gamma |- t'=t:A}
 ~~~~
   \inference{\Delta;\Gamma |- t=t':A & \Delta;\Gamma |- t'=t'':A}{
              \Delta;\Gamma |- t=t'':A}
\]
\end{singlespace}
\caption{Equality rules of \Fixi.}
\label{fig:eqFixi}
\end{figure}

The grey-boxed extensions for term-indexing are essentially the same as
those grey-boxed extensions in System \Fi\ (\S\ref{sec:fi:fi}). Hence, we will
only focus our description on the dashed-box extensions regarding polarities
(\S\ref{ssec:fixi:def:polarity}) and equi-recursive types
(\S\ref{ssec:fixi:def:equirec}).

\subsection{Polarities} \label{ssec:fixi:def:polarity}
Polarities track how type constructor variables are used.
A polarity ($p$) is either covariant ($+$), contravariant ($-$), or
avariant ($0$). When a type variable is bound, its polarity is 
made explicit both at its binding site, and in the context.
The avariant polarity ($0$) means that a variable can be used both
covariantly and contravariantly\footnote{
	The word ``invariant'' is sometimes used (see \cite{AbeMat04}),
	but we think this notation is quite misleading,
	The polarity $0$ means that the system \emph{does not care} about
	that variable's polarity, rather than indicating some
	unchanging set of properties about the variable's polarity.}
We prefix a kind  by a polarity (\ie, $p\kappa$) to specify the variable's
kind and polarity. For example,
\[
\begin{array}{rlrl}
X_1^{-*},X_2^{+*} |- \!\!\! & X_1 -> X_2 : *
	& ~~~\text{justifies} & \l X_1^{-*}.\l X_2^{+*}.X_1 -> X_2 \\
X_1^{0*},X_2^{0*} |- \!\!\! & X_1 -> X_2 : *
	& ~~~\text{also justifies} & \l X_1^{0*}.\l X_2^{0*}.X_1 -> X_2 \\
X^{0*} |- \!\!\! & X\phantom{_.} -> X\phantom{_2} : *
	& ~~~\text{justifies} & \l X^{0*}.X -> X
\end{array}
\]
Note that we can replace $+$ and $-$ in the first example with $0$
as in the second example, since the variables of avariant polarity can be used
in any position that is both in a covariant and contravariant position.
In the third example, the polarity of $X$ can be neither $+$ nor $-$, but must
must be $0$, since $X$ appears in both covariant and contravariant positions.

\paragraph{Syntax using polarized kinds:}
The kind syntax is polarized. That is, the domain kind ($\kappa$) of
an arrow kind ($p\kappa -> \kappa'$) must be prefixed by its polarity ($p$).
Type abstractions ($\l X^{p\kappa}.F$) in the type syntax are annotated by
polarity-prefixed kinds ($p\kappa$). Type constructor variables ($X$) bound
in the type-level contexts ($\Delta$) are likewise annotated by
polarity-prefixed kinds ($p\kappa$). Note the syntax for extending
the type-level context $\Delta,X^{p\kappa}$ in Figure~\ref{fig:Fixi}.
The kinding rule ($\lambda$) exploits all these three uses of polarized kinds
-- in type abstractions, in kind arrows, and in type-level contexts.

\paragraph{Polarity operation on type-level context ($p\Delta$):}
The kinding judgment $\Delta |- F : \kappa$ assumes that $F$ is in covariant
positions. This is why the ($Var$) rule requires the polarity of $X$ to be
either $+$ or $0$ but not $-$. To judge well-kindedness of type constructors
in contravariant positions (\eg, $A$ in $A -> B$), we should invert
the polarities of all the type constructor variables in the context.
This idea of inverting polarities in the context is captured by the $-\Delta$
operation in the kinding rule ($->$). More generally, the well-kindedness $F$
expected to be used as $p$-polarity can be determined by the judgement
$p\Delta |- F : \kappa$, where $p\Delta$ operation is defined as:
\begin{itemize}
	\item when $p$ is either $+$ or $-$\vspace*{-2ex}
		\[
		\begin{array}{lcl}
		p~\cdot &=& \cdot \\
		p(\Delta,X^{p'\kappa}) &=& p\Delta,X^{(pp')\kappa} \\
		p(\Delta,i^A) &=& p\Delta,i^A
		\end{array}
		\quad\left(\text{$pp'$ is the usual sign product}~~
		\begin{smallmatrix}
		+ p' & = & p' \\
		- +  & = & -  \\
		- -  & = & +  \\
		- 0  & = & 0
		\end{smallmatrix}\right)
	\]
	\item when $p=0$\vspace*{-2ex}
		\[
		\begin{array}{lcl}
		0~\cdot &=& \cdot \\
		0(\Delta,X^{0\kappa}) &=& 0\Delta,X^{0\kappa} \\
		0(\Delta,X^{p\kappa}) &=& 0\Delta \qquad(p\neq 0)\\
		0(\Delta,i^A) &=& 0\Delta,i^A
		\end{array}
		\]
\end{itemize}
Note the use of $p\Delta$ operation in the kinding rule ($@$) in order to
determine the well-kindedness of $G$ expected to be used as $p$-polarity
by the type constructor $F : p\kappa -> \kappa'$ being applied to $G$.

\paragraph{Where polarities are irrelevant (\ie, avariant):}
Polarities are irrelevant (\ie, avariant) for universally quantified variables
and indices as well as in the typing rules. This is because the sole purpose
of tracking polarities in \Fixi\ is to ensure that we only take
the equi-recursive fixpoint over covariant type constructors, as in
the kinding rule ($\mu$).
Note that we can only take fixpoints over type constructors of covariant
arrow kinds whose domain and codomain coincide ($+\kappa -> \kappa$).
We can never take fixpoints over forall types (or universal quantification)
and type constructor that expect an index because they are not of arrow kinds.
Forall types are always of kind $*$ and type constructors that expect an index
are of arrow kinds ($A -> \kappa$). So, we give universally quantified variables
avariant polarity ($X^{0\kappa}$ in the ($\forall$) rule) and nullify polarities
when type checking indices ($0\Delta$ in the ($@i$) rule). For similar reasons,
we assume that type-level contexts are nullified in the typing rules;
note $0\Delta$ in the well-formedness condition for $\Delta |- \Gamma$
in Figure~\ref{fig:Fixi2}. That is, we always type check under nullified
type-level context for all terms in general as well as for indices appearing
in type applications. As a result, the typing rules of \Fixi\ have no dashed-box
extensions except for $X^{0\kappa}$ in the generalization rule ($\forall$$I$)
where we introduce a universally quantified type constructor variable.

\subsection{Equi-recursive type operator $\fix$}
\label{ssec:fixi:def:equirec}
\index{fixpoint!equi-recursive}
\index{fixpoint!iso-recursive}
\Fixi\ provides the equi-recursive type operator $\fix$.
The kinding rule ($\fix$) in Figure~\ref{fig:Fixi2} is similar to
the ($\mu$) rule of System \Fi\ (see Figure~\ref{fig:Fi2} in \S\ref{sec:fi:fi}),
but requires base structure $F$ to be covariant (or positive), that is,
$F : +\kappa -> \kappa$. This restriction on the polarity of $F$ is caused by
the equi-recursive nature of $\fix$, that is, $\fix F=F(\fix F)$, 
described by the first type constructor equality rule inside a dashed box
in Figure~\ref{fig:eqFixi}. Restricting the polarity of the base structure,
to which $\fix$ can be applied, is necessary to maintain strong normalization.
Adding equi-recursive types without restricting the polarity breaks
the strong normalization because it amounts to having both formation and
elimination of arbitrary iso-recursive types.

Note that there is no explicit term syntax that guides the conversion between
$\fix F$ and $F(\fix F)$, unlike in iso-recursive\footnote{
	In this dissertation, all the other fixpoint type operators
	except $\fix$ are iso-recursive.}
systems where $\In$ and $\mathsf{unIn}$ are term syntaxes that explicitly guide
rolling (from $\mu F$ to $F(\mu F)$) and unrolling (from $F(\mu F)$ to $\mu F$).
Because $\fix F=F(\fix F)$ is given definitionally (\ie, by the equality rule
definition), the type constructor conversion rule ($Conv$) can silently
roll (from $F(\fix F)$ to $\fix F$) and unroll (from $F(\fix F)$ to $\fix F$)
the recursive types, just as it can silently $\beta$-convert type constructors.

In the following section, we review how iso-recursive type operator
$\mu_\kappa$ and its $\In_\kappa$ constructor, which is well-behaved
(\ie, strongly normalizes) for base structures of arbitrary polarity,
can be embedded into \Fixi\ defined in terms of the equi-recursive
type operator $\fix$ that is only well-behaved for covariant base structures.

