\section{Introduction: dividing the problem space of typed formal reasoning systems}
\label{sec:intro}
\newcommand{\IND}{\ensuremath{\mathsf{IND}}}
\newcommand{\REC}{\ensuremath{\mathsf{REC}}}
\newcommand{\INDbot}{\ensuremath{\mathsf{IND}_\bot}}
\newcommand{\RECbot}{\ensuremath{\mathsf{REC}_\bot}}

\begin{figure}
\begin{tabular}{p{12em}p{10em}|p{10em}}
             & ~\newline normalizing terms & possibly\newline non-normalizing terms \\
\begin{center}\centering
inductive types
\end{center} & \begin{center}\Large\IND\end{center} & \begin{center}\Large\INDbot\end{center} \\ \hline
\begin{center}
~~~~recursive types\newline
(possibly non-inductive)
\end{center} & \begin{center}\Large\REC\end{center} & \begin{center}\Large\RECbot\end{center}
\end{tabular}
\caption{The four problem spaces categorized by inductiveness and normalization}
\label{fig:probspace}
~\\
\framebox{
\begin{minipage}{\textwidth}
\IND\ consists of the normalizing terms with inductive types.
Terms of the simply typed lambda calculus, extended with finite ground types
(\eg unit, boolean) or (more generally) inductive types with primitive recursion
(\eg natural number, list, tree) also belong to \IND.\\

\INDbot\ consists of possibly non-normalizing terms with inductive types.
That is, both normalizing and non-normalizing terms with inductive types
are in \INDbot. Thus, any term belonging to \IND also belongs to \INDbot\
(\ie $\IND\subset\INDbot$) by definition. A typical way of extending
a language, to break the \IND\ containment, is to add unrestricted
\emph{general recursion} to the language. For example,
the simply typed lambda calculus extended with general recursion
(\ie by adding a Y-combinator) have non-normalizing (or, non-terminating) terms,
which then belong to \INDbot\ but not \IND.\\

\REC\ consists of normalizing terms of recursive types, which are possibly
not inductive. That is, some recursive types correspond to inductive types,
while others do not. So, \IND\ is contained in \REC\ (\ie $\IND\subset\REC$)
by definition. Terms of System \F\ belong to \REC, since System \F\ is powerful
enough to encode all recursive types, while remaining strongly normalizing.
Although the fact that there exist sound languages contained in \REC, such
languages have often been overlooked in the design of formal reasoning systems.\\

\RECbot\ consists of possibly non-normalizing terms with recursive types,
which are possibly not inductive. Unlike languages with inductive types,
languages that allow unrestricted use of recursive types can have
non-normalizing terms, even without unrestricted general recursion.
For example, the simply typed lambda calculus extended with recursive types
can have non-normalizing terms (\ie can express non-terminating computations).
Note, it is much easier to introduce non-termination in the presence of
recursive types (where one doesn't need general recursion) than
in the presence of inductive types (where one needs general recursion
to introduce non-termination). For this reason, \REC\ has often been neglected
and \RECbot\ has mainly been considered for reasoning about languages
with recursive types.
\end{minipage}
}
\end{figure}

My proposed thesis is:
\begin{quote}
The two important concepts in typed formal reasoning systems (or,
typed programming languages), \emph{inductiveness of types}
(\ie well defined as logical propositions)
and \emph{normalization of terms} (\ie canonical values and total functions),
can be and must be considered separately.
\end{quote}

We can categorize the problem space of typed formal reasoning systems (\eg LCF,
HOL, Coq, Twelf) by the inductiveness of types and the normalization of terms.
There are four combinations (\IND, \INDbot, \REC, \RECbot)
in this two dimensional space as shown in Figure\;\ref{fig:probspace}.
The upper two (\IND, \INDbot) categorize inductive types and
the lower two (\REC, \RECbot) categorize recursive types.
The left two (\IND, \REC) only allow total functions and
the right two (\INDbot, \RECbot) allow partial functions.
These four fragments are related by inclusion: the upper ones are included in
the lower ones ($\IND\subset\REC$, $\INDbot\subset\RECbot$) and the left ones
are included in the right ones ($\IND\subset\INDbot$, $\REC\subset\RECbot$).
And, as a consequence of transitivity, it is the case that $\IND\subset\RECbot$.
Note, \IND\ is the smallest and \RECbot is the largest of the four divisions
of the problem space. As we move to a smaller space, we are less expressive,
but we have more properties guaranteed. As we move to a larger space, we are
more expressive but have fewer properties guaranteed. For instance, in \IND,
we can not use general recursion to express terms, but have a normalization
guarantee; whereas, in \INDbot, we can use general recursion but no longer
have a normalization guarantee.
 
An ideal general purpose formal reasoning system should be able to shift
gears smoothly between these four spaces. If we can smoothly move around
from one space to another, we can take advantage of the benefits of both
the smaller space and the larger space. That is, we can express terms
in flexible ways allowed by the larger space, yet also depend upon
the desired properties guaranteed by the smaller space. For example,
consider the following diagram of function compositions, which commute:
That is, $f = g \circ f' \circ h$.
\[ \xymatrix{A_1 \ar[r]^f \ar[d]_h & A_2 \\
             B_1 \ar[r]_{f'}       & B_2 \ar[u]^g} \]
Let this diagram lie in \REC. That is,
all the functions ($f,h,g,f'$) are total and all the types ($A_1,A_2,B_1,B_2)$
are recursive types, which may or may not be inductive.
Then I expect the following property to hold:
If $A_1$ and $A_2$ are inductive, then $f$ is in \IND.

The motivation for the diagram above and its related property is the following.
We want to define $f$, which may be definable in \IND\ but the definition
is likely to be annoying since we have limited expressiveness in \IND.
It may be easier and more natural to define $f'$ in \REC, which is a function
over non-inductive recursive types $B_1$ and $B_2$, where the meaning of $f'$
corresponds to $f$. Then, we would be able to define $f$ as a composition of
$f'$ with (hopefully simple) mapping functions, $g$ and $h$. Later on,
we will introduce a well known example of Normalization by Evaluation
(\S\ref{sssec:NbE}), which directly correspond to this diagram.

More generally, I expect that a function $f$ in \REC\ can also be considered
to be in \IND\ when domain and range of $f$ are inductive types.
We can also write this in the style of typing judgement as follows:
\[\inference[\REC-\IND($\to$)]{ \Gamma |-_\REC f:A\to B
                      & \Gamma |-_\IND A:\star
                      & \Gamma |-_\IND B:\star }
                      { \Gamma |-_\IND f:A\to B } \]
Note, $A\to B$ is a inductive type when $A$ and $B$ are inductive. Thus,
even more generally, I expect any term $e$ in \REC\ can also be considered
to be in \IND\ when its type is inductive:
\[\inference[\REC-\IND]{ \Gamma |-_\REC e:T
                       & \Gamma |-_\IND T:\star }
                       { \Gamma |-_\IND e:T } \]
The typing rule for going the other way (\ie considering a term of \IND\
as a term of \REC) is trivial by the inclusion relation between \IND\ and \REC:
\[\inference[\IND-\REC]{ \Gamma |-_\IND e:T }
                       { \Gamma |-_\REC e:T } \]
There would be similar properties between other pairs of the four fragments,
which are in an inclusion relation (\eg \IND\ and \INDbot,
\INDbot\ and \RECbot). In my thesis work, I will identify those properties
and categorize prior work by the inclusion relation they depend upon.
From my preliminary research, it seems that the relation between
\IND\ and \INDbot\ has been most heavily studied and implemented in practice.
Various \emph{termination analysis} methods found in proof assistants,
which are based on inductive types, can be categorized as relating
\IND\ and \INDbot. The studies on \emph{strictly positive types} and
\emph{monotonicity} relates \IND\ and \REC. Although there exists some
theoretical studies relating \REC\ and \RECbot, few formal reasoning system
make use of those studies in practice. Thus, in my dissertation, I will put
more focus on relating \REC\ and \RECbot\ among the other inclusion relations.
More detailed plans for my dissertation work will be presented
in \S\ref{sec:plan}, after reviewing the background literature on
inductive types and recursive types in \S\ref{sec:bg}
and summarizing our preliminary work in \S\ref{sec:prelim}.

%% TODO thesis should be made more clear in the last summary sentence/paragraph
