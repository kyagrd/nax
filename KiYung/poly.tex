\chapter{Polymorphic type systems}\label{ch:poly}
We review the simply-typed lambda calculus (\S\ref{sec:stlc}),
a non-polymorphic type system, and
a series of well-known polymorphic type systems:
System~\F\ (\S\ref{sec:f}), System~\Fw\ (\S\ref{sec:fw}),
and the Hindley-Milner type system (\S\ref{sec:hm}).
We review them because \Fi\ (Chapter \ref{ch:fi}),
\Fixi\ (Chapter \ref{ch:fixi}), and the Nax language (Chapter \ref{ch:nax})
in later chapters are extensions of these systems. We assume the reader has
some familiarity with lambda calculi, at least with the untyped lambda calculus.

Readers who are not interested in reviewing these type systems,
either because not being interested in theories
or because already having expert-level understanding of the subject,
may skip this chapter.

One of the purposes of this chapter is illustrating
the strong normalization theorem for less common formulations of
the polymorphic lambda calculi. System~\F\ and System~\Fw\ are more often
formulated in Church style and with a single typing context. Here,
we illustrate them in Curry style and their typing rules with two
typing contexts, because our indexed type theories, System~\Fi\ and
System~\Fixi, in Part~\ref{part:Calculi} are formulated in such ways.

Another purpose of this chapter is familiarize the readers with
functional encodings of datatypes in polymorphic type systems
(see \S\ref{sec:f:data} and \S\ref{sec:fw:data}).

\section{The simply-typed lambda calculus}\label{sec:stlc}
\begin{figure}
\begin{singlespace}
\begin{minipage}{.46\textwidth}
	\begin{center}Church-style\end{center}
\def\baselinestretch{0}
\small
\begin{align*}
\textbf{term syntax} \\
t,s ::= &~ x           & \text{variable}    \\
      | &~ \l(x:A) . t & \text{abstraction} \\
      | &~ t ~ s       & \text{application} \\
\\
\textbf{type syntax} \\
A,B ::= &~ A -> B  & \text{arrow type} \\
      | &~ \iota   & \text{ground type}   \\
\end{align*}
\[ \textbf{typing context} \]\vspace*{-1em}
\begin{align*}\quad
\Gamma ::= &~ \cdot \\
	 | &~ \Gamma, x:A \quad (x\notin \dom(\Gamma)) \\
\end{align*}
\[ \textbf{typing rules}
	\qquad \framebox{$\Gamma |- t : A$} \]
\vspace*{-1em}
\begin{align*}
& \inference[\sc Var]{x:A \in \Gamma}{\Gamma |- x:A} \\
& \inference[\sc Abs]{\Gamma,x:A |- t : B}
		     {\Gamma |- \l(x:A).t : A -> B} \\
& \inference[\sc App]{\Gamma |- t : A -> B & \Gamma |- s : A}
		     {\Gamma |- t~s : B} \\
\end{align*}
\[ \textbf{reduction rules}
	\quad \framebox{$t --> t'$} \]
\vspace*{-1em}
\begin{align*}
& \inference[\sc RedBeta]{}{(\l(x:A).t)~s --> t[s/x]} \\
& \inference[\sc RedAbs]{t --> t'}{\l(x:A).t --> \l(x:A).t'} \\
& \inference[\sc RedApp1]{t --> t'}{t~s --> t'~s} \\
& \inference[\sc RedApp2]{s --> s'}{t~s --> t~s'} \\
\end{align*}
\end{minipage}
\begin{minipage}{.46\textwidth}
	\begin{center}Curry-style\end{center}
\def\baselinestretch{0}
\small
\begin{align*}
\textbf{term syntax} \\
t,s ::= &~ x           \\
      | &~ \l x    . t \\
      | &~ t ~ s       \\
\\
\textbf{type syntax} \\
A,B ::= &~ A -> B \\
      | &~ \iota  \\
\end{align*}
\[ \textbf{typing context} \]\vspace*{-1em}
\begin{align*}\quad
\Gamma ::= &~ \cdot \\
	 | &~ \Gamma, x:A \quad (x\notin \dom(\Gamma)) \\
\end{align*}
\[ \textbf{typing rules}
	\qquad \framebox{$\Gamma |- t : A$} \]
\vspace*{-1em}
\begin{align*}
& \inference[\sc Var]{x:A \in \Gamma}{\Gamma |- x:A} \\
& \inference[\sc Abs]{\Gamma,x:A |- t : B}
		     {\Gamma |- \l x   .t : A -> B} \\
& \inference[\sc App]{\Gamma |- t : A -> B & \Gamma |- s : A}
		     {\Gamma |- t~s : B} \\
\end{align*}
\[ \textbf{reduction rules}
	\quad \framebox{$t --> t'$} \]
\vspace*{-1em}
\begin{align*}
& \inference[\sc RedBeta]{}{(\l x   .t)~s --> t[s/x]} \\
& \inference[\sc RedAbs]{t --> t'}{\l x   .t --> \l x   .t'} \\
& \inference[\sc RedApp1]{t --> t'}{t~s --> t'~s} \\
& \inference[\sc RedApp2]{s --> s'}{t~s --> t~s'} \\
\end{align*}
\end{minipage}
~\\
\caption{Simply-typed lambda calculus in Church style and Curry style}
\label{fig:stlc}
\end{singlespace}
\end{figure}
We illustrate two styles of the simply-typed lambda calculus (STLC)
in Figure \ref{fig:stlc}. The left column of the figure describes
the Church-style STLC and the right column describes the Curry-style STLC.

A term can be either a variable, an abstraction (\aka\ lambda term), or
an application. The distinction between the two style comes from whether
the abstraction has type annotation in the term syntax. Abstractions in
Church style have the form $\lambda(x:A).t$ with type annotation $A$ on
the variable $x$ bound in $t$. Abstractions in Curry have the form
$\lambda x.t$ without any type annotation. The differences in typing rules
and reduction rules between the two styles follow from this distinction
in the term syntax.

A type can be either an arrow type or a ground type.
The type syntax is exactly the same in both styles.
%% TODO discuss about void type

Typing rules are the rules to derive (or prove) typing judgments.
A typing judgment $\Gamma |- t : A$ means that the term $t$ has type $A$
under the typing context $\Gamma$. We say $t$ is well-typed
(or, $t$ is a well-typed term) under $\Gamma$ when we can derive (or prove)
a typing judgment $\Gamma |- t : A$ for some $A$ according to the typing rules.
There are three typing rules --
one typing rule for each item of the term syntax.
Therefore, the typing rules of the STLC are syntax directed.
That is, there is always only one rule to choose for the typing derivation
by examining the shape of the term.
 
The reduction rules in Figure \ref{fig:stlc} describes $\beta$-reduction.
The \rulename{RedBeta} rule describes the key concept $\beta$-reduction on
the $\beta$-redex. A $\beta$-redex is an application of an abstraction to
another term. The other three rules describes the idea that a redex may
appear in subterms although the term itself is not a redex. The reduction
rules of the STLC are virtually the same as the reduction rules of
the untyped lambda calculus.

We first discuss two important properties of STLC
(subject reduction and strong normalization), which hold
in both Curry style and Church style (\S\ref{sec:stlc:srsn}).
Then, we discuss distinctive characteristics of each style
(\S\ref{sec:stlc:church},\S\ref{sec:stlc:curry}).
Finally, we motivate the discussion of polymorphic type systems
by reviewing the limitations of STLC (\S\ref{sec:stlc:topoly}).

\paragraph{Remark on the ground type:}
Before we discuss other properties of STLC, I would like to make a remark on
the type syntax, in particular, on the ground type ($\iota$). There should be
no questions regarding arrow types as they are types for functions.
For instance, abstractions have arrow types. We need some ground types
in order to populate the types. Otherwise, if there were no ground types,
we would not have any types\footnote{If we allow infinite types,
	then it is possible to populate types without ground types.
	There exist exotic lambda calculi with infinite types, but
	rather uncommon.}
Then, such version of STLC will be very uninteresting since there cannot
be any well-typed terms because there are no types to begin with. Here,
our version of STLC comes with the most simple ground type ($\iota$),
which does not inhabit any term (\aka\ the void type). Note that there exist
no term of type $\iota$. We can have functions over $\iota$, such as
the identity over the ground type ($\l(x:\iota).x$).\footnote{Here,
	I present examples in Church style since it is more succinct than
	writing typing judgments (\eg, $(\l x.t) : \iota -> \iota$)
	in Curry style.  But, the remark on the ground type $\iota$
	holds the same for Curry style as well.}
However, there exist no term, to which we cannot apply ($\l(x:\iota).x$).
What we can have are higher-order functions
(\eg, $\l(x_{f}:\iota -> \iota).\l(x_a:\iota).x_f~x_a$, which expects
an argument of type $\iota -> \iota$) applicable to the functions over $\iota$.

When people use STLC to model a programming language (with simple types),
they usually provide richer set of ground types other than the void type
(\eg, unit, boolean, natural numbers). In such versions of STLC with further
extensions to the type syntax, they also need to extend the term syntax
by providing normal terms for the ground types (\eg, \textsf{true} and
\textsf{false} for booleans) and eliminators (\eg, if-then-else expression for
booleans) that can examine the normal terms. Here, having just the void type
is good enough for my purpose of leading up the discussion for
the polymorphic type systems, without complicating the term syntax.

\subsection{Subject reduction and strong normalization}\label{sec:stlc:srsn}
We discuss two important properties of STLC, which holds in both
Church style and Curry style -- \emph{subject reduction} (\aka\
\emph{type preservation}) and \emph{strong normalization}.

\paragraph{Subject reduction} is a property that reduction preserves type,
as stated below:
\begin{theorem}[subject reduction]
	$\inference{\Gamma |- t : A  & t --> t'}{\Gamma |- t' : A}$
\end{theorem}
That is, when a well-typed term takes a reduction step, then the reduce term
also has the same type. We can prove subject reduction (\aka\ type preservation)
by induction on the derivation of the reduction rules.
The only interesting case is the \rulename{RedBeta} rule. Proof for all
the other rules are simply done by applying the induction hypothesis.
Proof for the \rulename{RedBeta} rule amounts to proving the substitution lemma:
\begin{lemma}[substitution]
$ \inference{\Gamma,x:A |- t : B  & \Gamma |- s : A}{\Gamma |- t[s/x] : B} $
\end{lemma}
Proof of the substitution lemma is a straightforward induction on
the derivation of the typing judgement.

When people use STLC to model a programming language,
they usually consider another property called \emph{progress},
which states that well-typed terms are either values or
able to take an evaluation step. An evaluation is a reduction strategy
(\ie, certain subset of the reduction relation), which is often deterministic.
Values are terms that meet certain syntactic criteria, which is intended not
to take further evaluation steps. In such a setting, type safety means
subject reduction together with progress. However, in a calculus considering
reductions to normal terms, rather than evaluations to values, type safety is
no more than subject reduction since normal terms are irreducible by definition.

\paragraph{Strong normalization} is another important property of STLC,
when we intend to consider terms of STLC as a proof of a propositional logic
by the Curry-Howard correspondence. The proof strategy we present here is
to define the set of strongly normalizing terms, which may or may not be
well typed, and show that all well-typed terms belong to that set
We shall continue the discussion on strong normalization using
the Curry-style term syntax, but this proof strategy also works well
for the Church-style STLC\footnote{This proof strategy generallize well
	to more complicated systems such as System \F, System \Fw, and
	even for dependently-typed calculi, such as
	the Calculus of Constructions\cite{Geuvers94}.}.
In fact, this strategy originates from Girard's strong normalization proof
for System \F\ using reducibility candidates \cite{Gir71}, and later rephrased
using Tait's satruated sets \cite{Tait75}. In particular, I adopt
the notations from the work of \citet{AbeMat04}, which includes
a strong normalization proof for an extension of \Fw\ using saturated sets.
The strong normalization proofs for System \F\ (\S\ref{sec:f}) and
\Fw\ (\S\ref{sec:fw}) in this chapter, and also for System \Fi\
(Chapter \ref{ch:fi}) and \Fixi\ (Chapter \ref{ch:fixi}) in later chapters,
are also based on the same strategy using saturated sets, with gradually
increasing complexity for the interpretation of types in each of those systems.

A straightforward inductive definition of $\SN$,
the set of strongly normalizing terms, would be:
\[
\inference{s_1,\ldots,s_n\in\SN}{x~s_1~\cdots~s_n \in \SN}
\qquad
\inference{t \in \SN}{\l x.t \in \SN}
\qquad
\inference{t' \in \SN & t --> t'}{t \in \SN}
\]
That is, variables and applications of a variable to a strongly normalizing term
are in $\SN$, abstractions are in $\SN$ when their bodies are in $\SN$,
and terms that reduce to $\SN$ are also in $\SN$. Relying on the fact that
normal order reduction (\ie, reduce the outermost leftmost redex first) always
leads to a normal form if a normal form exists, we can alter the last rule of
the inductive definition above more syntactically, yet defining the same set
$\SN$, as follows:
\[
\inference{s_1,\ldots,s_n\in\SN}{x~s_1~\cdots~s_n \in \SN}
\quad
\inference{t \in \SN}{\l x.t \in \SN}
\quad
\inference{t[s/x]~s_1~\cdots~s_n\in \SN & s\in\SN}
	{(\lambda x.t)~s~s_1~\cdots~s_n \in \SN}
\]

A set $\mathcal{A}$ is saturated when it is closed under adding
strongly normalizing neutral terms and strongly normalizing weak head expansion:
\[
\inference{s_1,\ldots,s_n\in\SN}{x~s_1~\cdots~s_n \in \mathcal{A}}
\qquad\qquad\qquad\qquad\quad
\inference{t[s/x]~s_1~\cdots~s_n\in \mathcal{A} & s\in\SN}
	  {(\l x.t)~s~s_1~\cdots~s_n \in \mathcal{A}}
\]
We can easily observe that $\SN$ is a saturated set by definition
since we can get the first and last part of the inductive definition for $\SN$
when $\mathcal{A}=\SN$. We can define an arrow operator over saturated sets
(or, saturated-set arrow) as follows:
\[ \mathcal{A} -> \mathcal{B} = \{ t \in \SN \mid t~s \in \mathcal{B} ~\text{for all}~ s \in \mathcal{A} \} \]
It is known that $\mathcal{A} -> \mathcal{B}$ is saturated
when both $\mathcal{A}$ and $\mathcal{B}$ are saturated.

\begin{figure}
\begin{singlespace}
\begin{description}
\item[Interpretation of types] as saturated sets of normalizing terms:
\begin{align*}
[| \iota  |] &= \bot \qquad\qquad (\text{the minimal saturated set}) \\
[| A -> B |] &= [| A |] -> [| B |]
\end{align*}

\item[Interpretation of typing contexts] as sets of valuations ($\rho$):
\[ [| \Gamma |] =
	\{\; \rho\,\in\,\dom(\Gamma) -> \SN ~\mid~
             \rho(x) \in [|\Gamma(x)|] ~\text{for all}~x\in\dom(\Gamma) \;\}
\]

\item[Interpretation of terms] as terms themselves whose free variables are
	substituted according to the given valuation ($\rho$):
\begin{align*}
[|x|]_\rho      &= \rho(x) \\
[|\l x.t|]_\rho &= \l x.[|t|]_\rho  \qquad (x\notin\dom(\rho)) \\
[|t~s|]_\rho    &= [|t|]_\rho~[|s|]_\rho
\end{align*}
\end{description}
\caption[Interpretation of STLC for proving strong normalization]
	{Interpretation of types, typing contexts, and terms of STLC
         for the proof of strong normalization}
\label{fig:interpSTLC}
\end{singlespace}
\vspace*{.5em}\hrule
\end{figure}

We interpret types as saturated subsets (\ie, subsets that are saturated) of
$\SN$ as in Figure \ref{fig:interpSTLC}. We interpret the void type as
the minimal saturated set ($\bot$), which is saturated from the empty set.
We choose the symbol $\bot$ since saturated sets form a complete lattice
under the subset relation as the partial order. We may denote $\SN$ as $\top$
since it is the maximal element of the lattice. Note that $\bot$,
or $[|\iota|]$, does not include any abstraction since $\iota$ is
not a type of a function. Arrow types ($A -> B$) are interpreted as
the saturated-set arrow over the interpretations of the domain type
and the range type ($[|A|] -> [|B|]$).

We interpret a typing context ($\Gamma$) as a set of valuations ($\rho$).
For every variable binding in the typing context ($x:A \in \Gamma$),
a valuation should map the variable ($x$) to a term that belongs to
the interpretation of its desried type ($[|A|]$). That is, if
$x : A \in \Gamma$ then any $\rho \in [|\Gamma|]$ should satisfy
that $\rho(x) \in [|A|]$.

The Proof of strong normalization amounts to proving the following theorem:
\begin{theorem}[soundness of typing]
$ \inference{\Gamma|- t:A & \rho \in [|\Gamma|]}{[|t|]_\rho \in [|A|]} $
\end{theorem}
\begin{proof}
We prove by induction on the typing derivation ($\Gamma|- t:A$).
\paragraph{}
For variables, it is trivial to show that
$ \inference{\Gamma |- x:A & \rho \in [|\Gamma|]}{[|x|]_\rho \in [|A|]} $.

Due to the \rulename{Var} rule, we know that $x:A \in \Gamma$.
So, $[|x|]_\rho =\rho(x)\in[|\Gamma(x)|] = [|A|]$.

\paragraph{}
For abstractions, we need to show that
$ \inference{\Gamma |- \l x.t : A -> B & \rho \in [|\Gamma|]}
	     {[|\l x.t|]_\rho \in [|A -> B|]} $.

Since $[|\l x.t|]_\rho = \l x.[|t|]_\rho$ and
$[|A -> B|] = \{ t\in \SN \mid t~s\in[|B|]~\text{for all}~s\in[|A|] \}$,
what we need to show is equivalent to the following:
\[ \inference{\Gamma |- \l x.t : A -> B & \rho \in [|\Gamma|]}
	     {\l x.[|t|]_\rho \in
		\{ t\in \SN \mid t~s\in[|B|] ~\text{for all}~ s\in[|A|] \} }
\]
By induction, we know that:
$ \inference{\Gamma,x:A |- t : B & \rho' \in [|\Gamma,x:A|]}
	     {[|t|]_{\rho'} \in [|B|]} $.

Since this holds for all $\rho' \in [|\Gamma,x:A|]$, it also holds
for particular $\rho' = \rho[x \mapsto s]$ for any $s \in [|A|]$.
So, $[|t|]_{\rho[x\mapsto s]} = ([|t|]_\rho)[s/x]\in[|B|]$ for any $s\in[|A|]$.
Since saturated sets are closed under normalizing weak head expansion,
$(\l x.[|t|]_\rho)~s \in[|B|]$ for any $s\in[|A|]$.
Therefore, $\l x.[|t|]_\rho$ is obviously in the very set,
which we wanted it to be in (\ie,
$\l x.[|t|]_\rho\in\{t\in \SN \mid t~s\in[|B|] ~\text{for all}~ s\in[|A|]\}$).

\paragraph{}
For applications, we need to show that
$ \inference{\Gamma |- t~s : B & \rho\in[|\Gamma|]}{[|t~s|]_\rho \in [|B|]} $.

By induction we know that
$
\inference{\Gamma |- t : A -> B & \rho\in[|\Gamma|]}{[|t|]_\rho \in [|A -> B|]}
\qquad
\inference{\Gamma |- s : A & \rho\in[|\Gamma|]}{[|s|]_\rho \in [|A|]}
$.

Then, it is straightforward to see that $[|t~s|]_\rho\in[|B|]$
by definition of $[|A -> B|]$.\\
\end{proof}

\begin{corollary}[strong normalization]
	$\inference{\Gamma |- t : A}{t \in \SN}$
\end{corollary}
Once we have proved the soundness of typing with respect to interpretation,
it is easy to see that STLC is strongly normalizing by giving a trivial
interpretation such that $\rho(x) = x$ for all $x\in\dom(\Gamma)$. Note that
$[|t|]_\rho = t \in [|A|] \subset \SN$ under the trivial interpretation.



\subsection{Characteristics of the Church-style STLC}\label{sec:stlc:church}
In Church style, the variable ($x$) in an abstraction
($\l(x:A).t$) has a type annotation ($A$). Intuitively, we may think of
the abstraction ($\l(x:A).t$) as a function that expects an argument of
the type ($A$) specified by the type annotation.

There are some interesting properties that hold in Church style,
but not in Curry style. Here, we discuss two of them:
\begin{itemize}
\item \emph{Uniqueness of typing}
$\qquad\inference{\Gamma |- t : A & \Gamma |- t : A'}{A = A'} $

\item \emph{Type equivalence between well-typed $\beta$-equivalent terms}
\begin{align*}
& \inference{t =_{\beta} t' & \Gamma |- t : A & \Gamma |- t' : A'}{A = A'} \\
& \text{where $=_{\beta}$ is the reflexive symmetric transitive closure of $-->$.}
\end{align*}
\end{itemize}

\emph{Uniqueness of typing}, described in the first item above,
holds in the Church-style STLC.  More specifically, given
a well-formed typing context $\Gamma$ and a term $t$ as input,
if the term is well typed, that is, $\Gamma |- t : A$ for some type $A$,
then that $A$ is the unique such type. We can prove this by induction on
the derivation of the typing judgment.
For variables, it trivially holds since the variables appearing in
the typing contexts are unique.
For abstractions, we use induction on the derivation.
To use the induction hypothesis we should make sure that
the typing context ($\Gamma,x:A$) and the term $t$ of the premise
is uniquely determined. It is easy to see that they are uniquely determined
since all the peaces appearing in the input (\ie, the typing context and
the term) of the premise (in particular, $\Gamma$, $A$, and $t$) are
part of the input (in particular the term $\l(x:A).t$) of the conclusion.
Therefore, by induction hypothesis, $B$ is uniquely determined, and,
as a consequence, $A -> B$ is uniquely determined.
For applications, it is easy to show by induction for each of the premise.
This proof describes the essence of the type reconstruction algorithm for
the Church-style STLC.


Proving \emph{type equivalence between well-typed $\beta$-equivalent terms},
described in the second item above, amounts to proving \emph{type equivalence
between terms before and after well-typed $\beta$-reducion
(or, $\beta$-expanion)},
described below:
\begin{align}
\inference{t --> t' & \Gamma |- t : A & \Gamma |- t' : A'}{A = A'}
	\label{eqn:welltypedarrow}
\end{align}
That is, when a well-typed term ($t$) reduces to
another well-typed term ($t'$) in single step ($t --> t'$),
the types of those two terms are identical ($A=A'$).
Since the claim above is symmetric, we can also say: when a well-typed term
($t'$) expands to another well-typed term ($t$) in single step ($t' <-- t$),
the types of those two terms are identical ($A'=A$).
Let us break down the claim (\ref{eqn:welltypedarrow}) above into two parts:
\[
\inference{t --> t' & \Gamma |- t : A}
          {\Gamma |- t' : A' ~~\text{implies}\quad A = A'} \qquad
	\begin{smallmatrix}
		\text{type equivalence between terms} \\
  		\text{before and after well-typed $\beta$-reduction}
	\end{smallmatrix}
\]
\[
\inference{t --> t' & \Gamma |- t' : A'}
          {\Gamma |- t : A \quad\text{implies}\quad A = A'} \qquad
	\begin{smallmatrix}
		\text{type equivalence between terms} \\
  		\text{before and after well-typed $\beta$-expansion}
	\end{smallmatrix}
\]~\vspace*{-3em}\\

We know that the former (type equivalence between terms before and after
well-typed $\beta$-reduction) holds because we already know of
a stronger property, subject reduction, which we discussed earlier
(repeated below) that it is one of the properties that commonly hold
in both Church style and Curry style.
\begin{align*}
\inference{\Gamma |- t : A  & t --> t'}{\Gamma |- t' : A}
 &\qquad \text{subject reduction, or type preservation}
\end{align*}

\begin{figure}
\begin{singlespace}
\begin{tabular}{lp{7cm}}
$t <-- (\l(x:\iota).x)~t$ &
relying on the property of the void type
that $\iota$ is not inhabited by any term
\\ ~ \\
$t <-- (\l(x:\iota -> \iota).t)~(\l(x:\iota).x~x)$ &
without relying on the property of $\iota$ : \par
apply a constant function to an already ill-typed term (self application)
\end{tabular}
\end{singlespace}
\caption{Examples of $\beta$-expansions to ill-typed terms}
\label{ill-typed_expand}
\hrule
\end{figure}

So, what is interesting about the Church style STLC, in contrast to
the Curry-style STLC, is the latter part of the claim that terms before and
after well-typed $\beta$-expansion must always be of same type. Unlike
the former part of the claim on $\beta$-reduction, where reduced term
is always well-typed (corollary of subject reduction), we really need
the condition that expanded term is well-typed ($\Gamma |- t : A$).
This is because a well-typed term can be expanded to a ill-typed term.
In fact, we can always expand any well-typed term to ill-typed terms
(see Figure \ref{ill-typed_expand}).

The proof of the claim (\ref{eqn:welltypedarrow}) is very simple.
Recall that we want to show that $A = A'$,
assuming $t --> t'$, $\Gamma |- t : A$ and $\Gamma |- t' : A'$.
We can prove it by using \emph{subject reduction} and
\emph{uniqueness of typing} as follows:
\[ \inference[(\emph{uniqueness of typing})]
	{ \inference[(\emph{subject reduction})\!\!\!]
		{t --> t' & \Gamma |- t : A}
		{\Gamma |- t' : A \phantom{a_f}} 
	& \Gamma |- t' : A' }
	{ A = A' }
\]

\subsection{Characteristics of the Curry-style STLC}\label{sec:stlc:curry}
In Curry style, there is no annotation on the variable in an abstraction.
Since the variable binding in the abstraction is no longer fixed to
a specific type, \emph{uniqueness of typing} does not hold,
unlike in Church style. For instance, the identity function ($\l x.x$)
could have one of any type that has of the form $A -> A$, such as:
\begin{quote}\vspace*{-1em}
\begin{singlespace}
$\Gamma |- \l x.x : \iota -> \iota$ \\
$\Gamma |- \l x.x : (\iota -> \iota) -> (\iota -> \iota)$ \\
$\Gamma |- \l x.x : (\iota -> (\iota -> \iota)) -> (\iota -> (\iota -> \iota))$ \\
$\Gamma |- \l x.x : ((\iota -> \iota) -> \iota) -> ((\iota -> \iota) -> \iota)$ \\
$\Gamma |- \l x.x : ((\iota -> \iota) -> (\iota -> \iota)) -> ((\iota -> \iota) -> (\iota -> \iota))$ \\
$~~~~ \vdots $
\end{singlespace}
\end{quote}
So, we read the typing judgment $\Gamma |- t : A$ in Curry style as
\begin{quote}
$t$ \emph{can have type} $A$ under the typing context $\Gamma$,
\end{quote}
unlike in Church style where we read the typing judgment as
\begin{quote}
$t$ \emph{has the type} $A$ under the typing context $\Gamma$.
\end{quote}
However, we do not consider the Curry-style STLC to be
a polymorphic type system since the typing for a variable
under a well-formed context is still unique. That is,
\[ \inference{\Gamma |- x : A & \Gamma |- x : A'}{A = A'} \]

Type equivalence between terms before and after well-typed $\beta$-reduction
(or $\beta$-expansion) does not hold either, unlike in Church style.
This is expected since uniqueness of typing does not hold in Curry style.
Even when a well-typed term reduces to another well-typed term, they may be
given different types.\footnote{Subject reduction still holds for
the Curry-style STLC since we can choose to give the same type to
the reduced term as the type before the reduction.}
Consider the reduction $(\l x'.x')(\l x.x) --> \l x.x$.
We can give different types to the term before the reduction and the
term after the reduction. For example,
\[\setpremisesend{.1em} 
\inference[\sc App]
 {
   \inference[\sc Lam]
     { \inference[\sc Var]
         {x' : \iota -> \iota ~\in~ \cdot, x' : \iota -> \iota}
         {\cdot,x' : \iota -> \iota |- x': \iota -> \iota}
     }
     {\cdot |- \l x'.x' : (\iota -> \iota) -> (\iota -> \iota)}
 &
   \inference[\sc Lam]
     {\inference[\sc Var]{x:\iota ~\in~ \cdot,x:\iota}
                         {\cdot,x:\iota |- x:\iota} \phantom{x'}}
     {\cdot |- \l x.x : \iota -> \iota \phantom{(x')}}
 }
 {\cdot |- (\l x'.x')(\l x.x) : \iota -> \iota}
\]
\[\qquad\qquad\qquad\quad
\inference[\sc Lam]
  {\inference[\sc Var]{x:\iota->\iota ~\in~ \cdot,x:\iota->\iota}
                      {\cdot,x:\iota->\iota |- x:\iota->\iota} }
  {\cdot |- \l x.x : (\iota -> \iota) -> (\iota -> \iota)}
\]

%% does the set of possible types get larger after reduction even when
%% the only ground type is the void type? I am not sure. So, I won't
%% elaborate on this.

\paragraph{}
\KYA{TODO there should be a story on type inference and principal types
	to motivate the HM type system}

\subsection{Motivations for polymorphic type systems}\label{sec:stlc:topoly}
TODO
 %% \section{The simply-typed lambda calculus} \label{sec:stlc}
\section{System \F} \label{sec:f}
\begin{figure}
\begin{singlespace}
\begin{minipage}{.46\textwidth}
	\begin{center}Church style\end{center}
\def\baselinestretch{0}
\small
\begin{align*}
\textbf{term syntax} \\
t,s ::= &~ x           & \text{variable}    \\
      | &~ \l(x:A) . t & \text{abstraction} \\
      | &~ t ~ s       & \text{application} \\
      | &~ \L X    . t & \text{type abstraction} \\
      | &~ t [A]       & \text{type application} \\
\\
\textbf{type syntax} \\
A,B ::= &~ X           & \text{variable type}   \\
      | &~ A -> B      & \text{arrow type} \\
      | &~ \forall X.B & \text{forall type}   \\
\end{align*}
\[ \textbf{kinding \& typing contexts} \]\vspace*{-1em}
\begin{align*}\quad
\Delta ::= &~ \cdot \\
	 | &~ \Delta, X & (X\notin \dom(\Delta)) \\
\Gamma ::= &~ \cdot \\
	 | &~ \Gamma, x:A & (x\notin \dom(\Gamma)) \\
\end{align*}
\[ \textbf{kinding rules} \quad \framebox{$ \Delta |- A $} \]\vspace*{-1em}
\begin{align*}
& \inference[\sc TVar]{X \in \Delta}{\Delta |- X} \\
& \inference[\sc TArr]{\Delta |- A & \Delta |- B}{\Delta |- A -> B} \\
& \inference[\sc TAll]{\Delta,X |- B}{\Delta |- \forall X.B} \\
\end{align*}
\[ \textbf{typing rules} \quad \framebox{$ \Delta;\Gamma |- t : A $ } \]
\vspace*{-1em}
\begin{align*}
& \inference[\sc Var]{x:A \in \Gamma}{\Delta;\Gamma |- x:A} \\
& \inference[\sc Abs]{\Delta |- A & \Delta;\Gamma,x:A |- t : B}
	             {\Delta;\Gamma |- \l(x:A).t : A -> B} \\
& \inference[\sc App]{\Delta;\Gamma |- t : A -> B & \Delta;\Gamma |- s : A}
		     {\Delta;\Gamma |- t~s : B} \\
& \inference[\sc TyAbs]{\Delta,X;\Gamma |- t : B}
		       {\Delta;\Gamma |- \L X.t : \forall X.B} ~
		       (X\notin\FV(\Gamma)) \\
& \inference[\sc TyApp]{\Delta;\Gamma |- t : \forall X.B & \Delta |- A}
		       {\Delta;\Gamma |- t[A] : B[A/X]}
\end{align*}
\end{minipage}
\begin{minipage}{.46\textwidth}
	\begin{center}Curry style\end{center}
\def\baselinestretch{0}
\small
\begin{align*}
\textbf{term syntax} \\
t,s ::= &~ x           \\
      | &~ \l x    . t \\
      | &~ t ~ s       \\
        &~\phantom{| \L X}  \\
        &~\phantom{| t [A]\vspace*{.1em}} \\
\\
\textbf{type syntax} \\
A,B ::= &~ X \\
      | &~ A -> B \\
      | &~ \forall X . B \\
\end{align*}
\[ \textbf{kinding \& typing contexts} \]\vspace*{-1em}
\begin{align*}\quad
\Delta ::= &~ \cdot \\
	 | &~ \Delta, X & (X\notin \dom(\Delta)) \\
\Gamma ::= &~ \cdot \\
	 | &~ \Gamma, x:A & (x\notin \dom(\Gamma)) \\
\end{align*}
\[ \textbf{kinding rules} \quad \framebox{$ \Delta |- A $}\]\vspace*{-1em}
\begin{align*}
& \inference[\sc TVar]{X \in \Delta}{\Delta |- X} \\
& \inference[\sc TArr]{\Delta |- A & \Delta |- B}{\Delta |- A -> B} \\
& \inference[\sc TAll]{\Delta,X |- B}{\Delta |- \forall X.B} \\
\end{align*}
\[ \textbf{typing rules} \quad \framebox{$ \Delta;\Gamma |- t : A $ } \]
\vspace*{-1em}
\begin{align*}
& \inference[\sc Var]{x:A \in \Gamma}{\Delta;\Gamma |- x:A} \\
& \inference[\sc Abs]{\Delta |- A & \Delta;\Gamma,x:A |- t : B}
		     {\Delta;\Gamma |- \l x   .t : A -> B} \\
& \inference[\sc App]{\Delta;\Gamma |- t : A -> B & \Delta;\Gamma |- s : A}
		     {\Delta;\Gamma |- t~s : B} \\
& \inference[\sc TyAbs]{\Delta,X;\Gamma |- t : B}
		       {\Delta;\Gamma |- t : \forall X.B} ~
		       (X\notin\FV(\Gamma)) \\
& \inference[\sc TyApp]{\Delta;\Gamma |- t : \forall X.B & \Delta |- A}
		       {\Delta;\Gamma |- t : B[A/X]}
\end{align*}
\end{minipage}
~\\
\caption{System \F\ in Church style and Curry style}
\label{fig:f}
\end{singlespace}
\end{figure}

\begin{figure}
\paragraph{Reduction rules for the Church-style System \F}
\begin{align*}
& \inference[\sc RedBeta]{}{(\l(x:A).t)~s --> t[s/x]}
&&\inference[\sc RedTy]{}{(\L X   .t)[A] --> t[A/X]} \\
& \inference[\sc RedAbs]{t --> t'}{\l x   .t --> \l x   .t'}
&&\inference[\sc RedTyAbs]{t --> t'}{\L X   .t --> \L X   .t'} \\
& \inference[\sc RedApp1]{t --> t'}{t~s --> t'~s}
&&\inference[\sc RedTyApp]{t --> t'}{t[A] --> t'[A]} \\
& \inference[\sc RedApp2]{s --> s'}{t~s --> t~s'}
\end{align*}
\paragraph{Reduction rules for the Curry-style System \F}
\begin{align*}
& \inference[\sc RedBeta]{}{(\l x   .t)~s --> t[s/x]} \\
& \inference[\sc RedAbs]{t --> t'}{\l x   .t --> \l x   .t'} \\
& \inference[\sc RedApp1]{t --> t'}{t~s --> t'~s} \\
& \inference[\sc RedApp2]{s --> s'}{t~s --> t~s'}
\end{align*}
\caption{Reduction rules for System \F}
\label{fig:redf}
\end{figure}

System \F\ extends the type syntax of STLC with type variables ($X$)
and forall types ($\forall X.B$), which enable us to express polymorphic types
(see Figure \ref{fig:f}). However, System \F\ does not have a dedicated syntax
for ground types, such as the void type $\iota$ in STLC. In System \F, we can
populate types from forall types such as $\forall X.X$. This type is, in fact,
an encoding of the void type. We shall see that large class of datatypes are
encodable in System \F (\S\ref{sec:f:data})

Unlike in STLC, not all types constructed by the type syntax of System \F\
make sense. Since we have type variables in System \F, we need to
make sure that types are well-kinded. That is, we should make sure
that all the type variables appearing in types are properly bound by
universal quantifiers ($\forall$). For instance, consider the two types
$\forall X.X$ and $\forall X.X'$. Under the empty kinding context,
$\forall X.X$ is well-kinded since $X$ is bounded by $\forall$, but
$\forall X.X'$ is ill-kinded since $X'$ is an unbound type variable.
The kinding rules determine whether a type is well-kinded.
In the kinding rules and typing rules, the kinding context ($\Delta$)
keeps track of the bound type variables. The complete syntax, kinding rules,
and typing rules of System \F are illustrated in Figure \ref{fig:f}.
The left column describes the Church-style System \F\ and the right
column describes the Curry-style System \F. The reduction rules are
shown separately in Figure \ref{fig:redf}.

As in STLC, the term syntax for abstractions differs between the two styles.
The Church-style System \F\ has type annotations in abstractions but
the Curry-style System \F\ does not. Furthermore, the Church-style System \F\
has additional syntax for type abstractions and type applications. The syntax
for type abstractions ($\L X.t$) makes it explicit that the type of the term
should be generalized to a forall type. The syntax for type applications
($t[A]$) makes it explicit that the type of the term should be instantiated to
a specific type from a forall type. On the contrary, the Curry-style System \F\
has neither type abstractions nor type type applications in the term syntax.
So, it is implicit in Curry style where types are generalized and instantiated.
The differences in typing rules and reduction rules between the two styles
follow from this difference in the term syntax.

The typing rules \rulename{Var}, \rulename{Abs}, and \rulename{App} are
pretty much the same as in STLC except that we carry around the kinding context
($\Delta$) along with the typing context ($\Gamma$). What is new in System \F\
are the typing rules for type abstractions (\rulename{TyAbs}) and
type applications (\rulename{TyApp}), which enable us to introduce
forall types and instantiate forall types to a specific type.
In Church style, the use of these two rules \rulename{TyAbs} and
\rulename{TyAbs} are guided by the term syntax of type abstractions
($\L X.t$) and type applications ($t[A]$). So, the typing rules of
the Church-style System \F\ are syntax directed. In Curry style,
on the contrary, there are no term syntax to guide the use of the rules
\rulename{TyAbs} and \rulename{TyApp}. So, the typing rules of
the Curry-style System \F\ are not syntax directed.

The reduction rules for the Church-style System \F\ includes all
the reduction rules for the Church-style STLC. In addition,  there
are three more reduction rules (\rulename{RedTy}, \rulename{RedTyAbs},
and \rulename{RedTyApp}) involving type abstractions and type applications.

The reduction rules for the Curry-style System \F\ are exactly the same as
the reduction rules for the Curry-style STLC (Figure \ref{fig:stlc}) since
the terms of the Curry-style System \F\ are identical to 
the terms of the Curry-style STLC.

\subsection{Encoding datatypes in System \F}
\label{sec:f:data}
System \F\ is  powerful enough to encode a fairly large class of datatypes
within its type system. Encodings of well-known datatypes are listed in
Table \ref{tbl:dataF}. In System \F, we can encode non-recursive datatypes
that are either simply-typed (\eg, void, unit, and booleans)
or parametrized (\eg, pairs and sums).
More interestingly, we can also encode recursive datatypes
that are either simply-typed (natural numbers) or parametrized (lists).
All of these datatypes are classified as \emph{regular datatypes}.\footnote{
	We discuss the concept of regular datatypes,
	in contrast to non-regular datatypes, in \S\ref{sec:fw:data}. }
All regular datatypes that are not mutually recursive are encodable
in System \F. Encodings of mutually recursive datatypes require
more expressive type systems such as System \Fw\ (\S\ref{sec:fw}).

\begin{table}
\begin{tabular}{p{15mm}|lp{92mm}}
	\hline
void
& encoding of type	& $\textit{Void} = \forall X.X$ \\
& constructor		& \\
& eliminator		& $\l x.x$
	\\\hline
unit
& encoding of type	& $\textit{Unit} = \forall X.X -> X$	\\
& constructor		& $\mathtt{Unit} = \l x.x$ \\
& eliminator		& $\l x.\l x'.x\;x'$
	\\\hline
booleans
& encoding of type	& $\textit{Bool} = \forall X.X -> X -> X$ \\
& constructors		& $\mathtt{True} = \l x_1.\l x_2. x_1$,\quad
			$\mathtt{False} = \l x_1.\l x_2. x_2$ \\
& eliminator		& $\l x.\l x_1. \l x_2. x\;x_1\,x_2$ \qquad
			(\textbf{if} $x$ \textbf{then} $x_1$ \textbf{else} $x_2$)
	\\\hline
pairs
& encoding of type	& $ A_1\times A_2 = \forall X. (A_1 -> A_2 -> X) -> X$ \\
& constructor		& $\mathtt{Pair} = \l x_1.\l x_2.\l x'.x'\,x_1\,x_2$ \\
& eliminator		& $\l x.\l x'.x\;x'$ \par
			(by passing appropriate values to $x'$, we get\par
			~~$\textit{fst} = \l x.x(\l x_1.\l x_2.x_1)$,
			$\textit{snd} = \l x.x(\l x_1.\l x_2.x_2)$ )
	\\\hline
sums
& encoding of type	& $A_1+A_2 = \forall X. (A_1 -> X) -> (A_2 -> X) -> X$ \\
& constructors		& $\mathtt{Inl} = \l x. \l x_1. \l x_2 . x_1\,x$,\quad
			$\mathtt{Inr} = \l x. \l x_2. \l x_2 . x_2\,x$ \\
& eliminator		& $\l x.\l x_1. \l x_2. x\;x_1\,x_2$ \par
			(\textbf{case} $x$ \textbf{of}
				\{$\mathtt{Inl}~x' -> x_1\;x'$;
				  $\mathtt{Inr}~x' -> x_2\;x'$\})
	\\\hline
natural
& encoding of type	& $\textit{Nat} = \forall X. (X -> X) -> X -> X$ \\
numbers
& constructors		& $\mathtt{Succ} = \l x. \l x_s. \l. x_z. x_s (x\;x_s\,x_z)$,\par
			$\mathtt{Zero} = \l x_s. \l x_z. x_z$ \\
& eliminator		& $\l x.\l x_z. \l x_s.x\;x_s\,x_z$ \quad
			(iteration on natural num.)
	\\\hline
lists
& encoding of type	& $\textit{List}\;A = \forall X. (A -> X -> X) -> X -> X$ \\
& constructors		& $\mathtt{Cons} = \l x_a.\l x.\l x_c.\l x_n. x_c\,x_a\,(x\;x_c\,x_n)$,\par
			$\mathtt{Nil}\;\, = \l x_c.\l x_n.\l x_n$ \\
& eliminator		& $\l x.\l x_c. \l x_n.x\;x_c\,x_n$ \quad
			(\textit{foldr} $x_z$ $x_c$ $x$ in Haskell)
	\\\hline
\end{tabular}
\caption{Church encodings of regular datatypes being well-typed in System \F}
\label{tbl:dataF}
\end{table}

\citet{Church41} devised an encoding for natural numbers
in the untyped lambda calculus, based on the idea that the natural number $n$
is represented by a higher-order function ($\l x_s.\l x_z.x_s^n~x_z$), which
applies the first argument ($x_s$) $n$ times to the second argument ($x_z$).
Such an encoding of natural numbers is called Church numerals, named after
Alonzo Church. More generally, term encodings of the objects of datatypes
based on similar a idea are called Church encodings. In System \F,
these Church encoded terms can be well-typed by encoding the datatype
as a polymorphic type of System \F, as illustrated in Table \ref{tbl:dataF}.
Such encodings for datatypes are called impredicative encodings
since it rely on the impredicative polymorphism of System \F.

Encodings of types, constructors, and eliminators for
several well-known datatypes are listed in Table \ref{tbl:dataF}.
We use the Curry-style System \F\ since the constructors and the eliminators
are exactly the same as the Church encodings in the untyped lambda calculus.
If we were to use the Church-style System \F, we would need to adjust
the constructors and the eliminators by adding type abstractions and
type applications in appropriate places. For example,
the constructor for $\textit{Unit}$ would be
$\mathtt{Unit} = \L X.\l x:X.x$ and the eliminator would be
$\l(x:\textit{Unit}).\L X.x[X]\;x'$.

Constructors produce objects of a datatype. Nullary constructors
(\aka\ constants) are objects by themselves. For example,
$\mathtt{Unit}$ (or, $\l x.x$) is a unit object,
$\mathtt{True}$ (or, $\l x_1.\l x_2. x_1$) is a boolean object,
$\mathtt{Zero}$ (or, $\l x_s. \l x_z. x_z$) is a natural number, and
$\mathtt{Nil}$ (or, $\l x_c.\l x_n.\l x_n$) is a list.
That is,
\[
|- \mathtt{Unit}:\textit{Unit} \qquad
|- \mathtt{True}:\textit{Bool} \qquad
|- \mathtt{Zero}:\textit{Nat} \qquad
|- \mathtt{Nil}:\forall X_a.\textit{List}\;X_a
\]
where $\textit{Unit}$ is a shorthand notation (or, type synonym)
for $\forall X.X -> X$, $Bool$ is for $\forall X.X -> X -> X$, and so on,
as described in Figure \ref{tbl:dataF}.
%%% say it is a function
Other (non-nullary) constructors expect some arguments to produce objects.
For example, $\mathtt{Pair}$ expects two arbitrary arguments to produce a pair,
$\mathtt{Succ}$ expects a natural number argument to produce another
natural number, and $\mathtt{Cons}$ expects a new element and a list as
arguments to produce another list. That is,
\begin{align*}
& |- \mathtt{Pair} : \forall X_1. \forall X_2. X_1 -> X_2 -> X_1\times X_2
&& |- \mathtt{Succ} : \textit{Nat} -> \textit{Nat} \\ &
|- \mathtt{Cons} : \forall X_a. X_a -> \mathit{List}\;X_a -> \mathit{List}\;X_a
\end{align*}
where ${X_1 \times X_2}$, $\mathit{Nat}$, and $\mathit{List\,X_a}$
are shorthand notations for encodings of the datatypes,
as described in Figure \ref{tbl:dataF}.
%%% maybe type syn ???

We can deduce the number of constructors for a datatype and the types
of those constructors from the impredicative encoding of the datatype.
The general form for the encodings of the simply-typed datatypes is:
\[D = \forall X. A_1 -> \cdots -> A_n -> X
	\qquad\text{where}~~ A_i = A_{i1} -> \cdots -> A_{ik} -> X \]
From the encoding of type above, we can deduce the following facts:
\begin{itemize}
\item $n$ is the number of constructors,
\item $k$ is the arity of the $i$th constructor, and
\item the type of the $i$th constructor is $A_i[D/X]$.
\end{itemize}
Note, $D$ is a shorthand notation for the entire encoding of the type.
So, $A_i[D/X]$ expands to $A_i[(\forall X. A_1 -> \cdots -> A_n -> X)/ X]$.
Here, the type variable $X$ in $A_i$ is substituted by a polymorphic type
$D$, or $(\forall X. \cdots)$. Recall that $X$ in $A_i$ comes from
the variable universally quantified in $D$. In System \F, we are able to
substitute the universally quantified type variable $X$ with
the very polymorphic type $D$ quantifying $X$. For this ability,
we say ``System \F\ is \emph{impredicative}''. Impredicative encodings
of datatypes rely on this impredicative nature (or, impredicativity)
of System \F.

Similarly, the general form for the encodings of the parametrized datatypes is
$D\,X_1 \cdots X_k = \forall X. A_1 -> \cdots -> A_n -> X$. Then,
the number of constructors is $n$ and the type of the $i$th constructor
is $\forall X_1.\cdots\forall X_n.A_i[D/X]$.

Eliminators consume objects of a datatype for computation.
An eliminator for a datatype expects an object of the datatype
as its first argument followed by arguments of computations
to be performed for each of the constructors. For instance, the eliminator
for void ($\l x.x$) expects only one argument since void has
no constructor, the eliminator for unit ($\l x.\l x'.x\;x'$) expect
two arguments since unit has one constructor, and the eliminator for booleans
($\l x.\l x_1. \l x_2. x\;x_1\,x_2$) expect three arguments since there are
two boolean constructors.

Eliminators examine the shape of the object (\ie, by which constructor it is
constructed) to perform the computation, which corresponds to the shape
of the object. For instance, the eliminator for booleans amounts to the
well-known if-then-else expression.
For recursive types, computations are performed recursively due to
the definition of the constructors that expect recursive arguments.
For instance, note that $(x~x_s~x_z)$ appearing in the definition of
$\mathtt{Succ}$ coincides with the body of the eliminator for natural numbers.
Eliminators for recursive types are also known as iterators or folds.

The impredicative encoding of a datatype specifies what it needs to eliminate
an object of the datatype. Recall the general form for the encodings of
the simply-typed datatypes:
\[D = \forall X. A_1 -> \cdots -> A_n -> X
	\qquad\text{where}~~ A_i = A_{i1} -> \cdots -> A_{ik} -> X \]
We can understand this encoding as follows:
\begin{quote}
To compute the result of type $X$ from an object of type $D$,
we need $n$ small computations, whose types are $A_1,\dots,A_n$.
When the object is constructed by $i$th constructor, we use the $i$th small
computation, whose type is $A_i$, that is, $A_{i1} -> \cdots -> A_{ik} -> X$.
This small computation gathers all the $k$ arguments supplied to
the $i$th constructor for the object construction, in order to
compute the result from those argument.
\end{quote}

For constants, the eliminator simply returns the argument being passed
to handle the constant, as it is. For example, the unit eliminator
$(\l x .\l x'.x\;x')$ will return what is passed into $x'$. That is,
\[   (\l x .\l x'.x\;x')~\mathtt{Unit}~s
 --> (\l x'.\mathtt{Unit}\;x')~s
 --> \mathtt{Unit}~s
 --> s
\] since $\texttt{Unit}=\l x.x$.
Similarly, the boolean eliminator $(\l x.\l x_1.\l x_2.x~x_1\;x_2)$
simply returns $x_1$ when $x$ is $\mathtt{True}$
and returns $x_2$ when $x$ is $\mathtt{False}$,
due to the definition of $\mathtt{True} = \l x_1.\l x_2. x_1$
and $\mathtt{False} = \l x_1.\l x_2. x_2$.

For non-nullary constructors, the argument being passed to the eliminator
to handle the constructor must be a function that collects the argument used
for the object construction. The pair eliminator $(\l x.\l x'.x\;x')$ expects
the argument $x'$ be of type $X_1 -> X_2 -> X$ where $X$ is the result type
you want to compute. For example, you may pass an addition function 
($\textit{Nat} -> \textit{Nat} -> \textit{Nat}\,$) to $x'$ to compute
the sum of the first element and the second element of a pair of
natural numbers ($\textit{Nat}\times\textit{Nat}\,$). We can define
selector functions $\mathit{fst}$ and $\mathit{snd}$ for pairs by
providing an appropriate
argument for $x'$ as described in Table \ref{tbl:dataF}.

The key idea behind Church encodings is that objects are defined by
how they will be eliminated. That is, the Church encoded objects
are, in fact, eliminators. Readers familiar with lambda calculi may have
noticed that all the eliminators in Table \ref{tbl:dataF} are
$\eta$-expansions of the identity function. The formulation of eliminators
in Table \ref{tbl:dataF} is just to emphasize how many arguments
each eliminator expects.

\subsection{Subject reduction and strong normalization}\label{sec:f:srsn}
We discuss two important properties of System \F, which holds in both
Church style and Curry style -- \emph{subject reduction} (\aka\
\emph{type preservation}) and \emph{strong normalization}.

\subsubsection*{Subject reduction}
The subject reduction theorem for System \F\ can be stated as follows:
\begin{theorem}[subject reduction]
$\inference{\Delta;\Gamma |- t : A  & t --> t'}{\Delta;\Gamma |- t' : A}$
\end{theorem}
We can prove subject reduction for System \F\, in a similar fashion
to the proof of subject reduction for STLC,
by induction on the derivation of the reduction rules.

In Church style, proof for all other cases except for the rules
\rulename{RedBeta} and \rulename{RedTy} are simply done by applying
the induction hypothesis. Since the typing rules in Church style are
syntax directed, there is no ambiguity on which typing rule to be used
in the derivation for a certain judgment. For the \rulename{RedBeta} case,
we use the substitution lemma. For proving the \rulename{RedTy} case,
we use the type substitution lemma. The substitution lemma and
the type substitution lemma are stated below:
\begin{lemma}[substitution]
$ \inference{\Delta;\Gamma,x:A |- t : B  & \Delta;\Gamma |- s : A}
	{\Delta;\Gamma |- t[s/x] : B} $
\end{lemma}
\begin{lemma}[type substitution]
$ \inference{\Delta,X;\Gamma |- t : B  & \Delta |- A}
	{\Delta;\Gamma |- t[A/X] : B[A/X]} ~ (X\notin\FV(\Gamma))$
\end{lemma}

In Curry style, the most interesting case is the \rulename{RedBeta} rule,
which we the substitution lemma to prove. The other rules are basically
done by applying the induction hypothesis, but there is a little complication
in the proof, compared to the proof in Church style, since the typing rules
are not syntax directed. Although we have less rules to consider than
the Church-style STLC, we need to deal with the ambiguity of which rule
to apply for a typing judgement. The ambiguity is due to the rules
\rulename{TyAbs} and \rulename{TyApp}.

An alternative way to prove subject reduction for the Curry-style STLC is
by translating the subject reduction property of the Curry-style STLC into
the subject reduction property of the Church-style STLC. That is, we extract
a Church-style term from a typing derivation in Curry style. It is not
difficult to see that each typing derivation in Curry style corresponds to
a unique Church-style term, and, that a reduction step in Curry style
corresponds to one or more reduction steps in Church style.\footnote{
It is not always one step to one step since the reduction rules
{\sc RedTyAbs} and {\sc RedTyApp} in Church style correspond to
zero reduction step in Curry style.}

\subsubsection*{Strong Normalization}
\begin{figure}
\begin{singlespace}
\begin{description}
\item[Interpretation of types] as saturated sets of normalizing terms
	whose free type variables are substituted according to
	the given type valuation ($\xi$):
\begin{align*}
[| X |]_\xi           &= \xi(X) \\ 
[| A -> B |]_\xi      &= [|A|]_\xi -> [|B|]_\xi \\
[| \forall X.B |]_\xi &= \bigcap_{\mathcal{A}\in\SAT} [|B|]_{\xi[X\mapsto\mathcal{A}]} \qquad\qquad\qquad (X\notin\dom(\xi))
\end{align*}
\item[Interpretation of kinding and typing contexts]
       as sets of type valuations and term valuations ($\xi$ and $\rho$):
\begin{align*}
[| \Delta        |] &= \dom(\Delta) -> \SAT \\
[| \Delta;\Gamma |] &= \{ \xi;\rho \mid \xi\in[|\Delta|], \rho\in[|\Gamma|]_\xi \} \\
[| \Gamma        |]_\xi\ &= \{ \rho \in \dom(\Gamma) -> \SN \mid \rho(x)=[|\Gamma(x)|]_\xi ~\text{for all}~x\in\dom(\Gamma) \}
\end{align*}
\item[Interpretation of terms]
	as terms themselves whose free variables are substituted according to
	the given pair of type and term valuation ($\xi$;$\rho$):
\begin{align*}
[| x      |]_{\xi;\rho} &= \rho(x) \\
[| \l x.t |]_{\xi;\rho} &= \l x . [|t|]_{\xi;\rho} \qquad (x\notin\dom(\rho)) \\
[| t ~ s  |]_{\xi;\rho} &= [| t |]_{\xi;\rho} ~ [| s |]_{\xi;\rho}
\end{align*}
\end{description}
\caption[Interpretation of System \F\ for proving strong normalization]
	{Interpretation of types, kinding and typing contexts, and terms
		of System \F\ for the proof of strong normalization}
\label{fig:interpF}
\end{singlespace}
%% \vspace*{.3em}\hrule
\end{figure}
To prove strong normalization of System \F, we use the same proof strategy
as in the proof of strong normalization of the STLC in \S\ref{sec:stlc:srsn}.
That is, we interpret types as saturated sets of normalizing terms, which
may or may not be well typed. The interpretation of types, contexts, and
terms of System \F\ are illustrated in Figure \ref{fig:interpF}. Since we
have type variables, we need a type valuation ($\xi$), which maps
the type variables to interpretations of types. So, the interpretation of types
are indexed by the type valuation ($\xi$), and the interpretation of terms are
indexed by the pair of term and type valuations ($\xi;\rho$). A type valuation
$\xi$ is a function from $\dom(\Delta)$, the set of type variables bound in
$\Delta$, to $\SAT$, the set of all saturated sets.

Any type interpretation is a saturated set. Since $\xi$ maps a type variable
to a saturated set, $[|X|]_\xi \in \SAT$. We know $[|A -> B|]_\xi \in \SAT$
since saturated sets are closed under the arrow operation ($->$), as we
mentioned in \S\ref{sec:stlc:srsn}. $[|\forall X.B|]_\xi \in \SAT$ since
it is known that saturated sets are closed under set indexed intersection.

The proof of strong normalization amounts to proving the following theorem:
\begin{theorem}[soundness of typing]
$ \inference{\Delta;\Gamma|- t:A & \xi;\rho \in [|\Delta;\Gamma|]}
	    {[|t|]_{\xi;\rho} \in [|A|]_\xi} $
\end{theorem}
\begin{proof}
We prove by induction on the typing derivation ($\Delta;\Gamma|- t:A$).
\paragraph{Case (\rulename{Var})}
It is trivial to show that
$ \inference{\Delta;\Gamma |- x:A & \xi;\rho \in [|\Delta;\Gamma|]}
	{[|x|]_{\xi;\rho} \in [|A|]_\xi} $.

We know that $x:A \in \Gamma$ from the \rulename{Var} rule.
So, $[|x|]_{\xi;\rho} =\rho(x)\in[|\Gamma(x)|]_\xi = [|A|]_\xi$.

\paragraph{Case (\rulename{Abs})}
We need to show that
$ \inference{\Delta;\Gamma |- \l x.t : A -> B & \xi;\rho \in [|\Delta;\Gamma|]}
	{[|\l x.t|]_{\xi;\rho} \in [|A -> B|]_\xi} $.

By induction, we know that
$ \inference{\Delta;\Gamma,x:A |- t : B & \xi';\rho' \in [|\Delta;\Gamma,x:A|]}
	     {[|t|]_{\xi';\rho'} \in [|B|]_\xi} $.

Since this holds for all $\xi';\rho' \in [|\Delta;\Gamma,x:A|]$, it also holds
for particular $\xi';\rho'$ such that $\xi'=\xi$ and
$\rho' = \rho[x \mapsto s]$ for any $s \in [|A|]_\xi' = [|A|]_\xi$.
Since saturated sets are closed under normalizing weak head expansion,
$(\l x.[|t|]_{\xi;\rho})~s \in[|B|]_\xi$ for any $s\in[|A|]_\xi$.
Therefore, $\l x.[|t|]_{\xi;\rho}$ is obviously in the set,
which we wanted it to be in, \ie,
\[ [|\l x.t|]_{\xi;\rho} = \l x.[|t|]_{\xi;\rho}
   \in \{t\in \SN \mid t~s\in[|B|] ~\text{for all}~ s\in[|A|]\} 
 = [|A -> B|]_\xi \]

\paragraph{Case (\rulename{App})}
We need to show that
$ \inference{\Delta;\Gamma |- t~s : B & \xi;\rho\in[|\Delta;\Gamma|]}{[|t~s|]_{\xi;\rho} \in [|B|]_\xi} $.

By induction we know that
\[
\inference{\Delta;\Gamma |- t : A -> B & \xi;\rho\in[|\Delta;\Gamma|]}{[|t|]_{\xi;\rho} \in [|A -> B|]_\xi}
\qquad
\inference{\Delta;\Gamma |- s : A & \xi;\rho\in[|\Delta;\Gamma|]}{[|s|]_{\xi;\rho} \in [|A|]_\xi}
\]
Then, it is straightforward to see that $[|t~s|]_{\xi;\rho}\in[|B|]_\xi$
by definition of $[|A -> B|]_\xi$.

\paragraph{Case (\rulename{TyAbs})}
We need to show that
$ \inference{\Delta;\Gamma |- t : \forall X.B & \xi;\rho\in[|\Delta;\Gamma|]}
	{[|t|]_{\xi;\rho} \in [|\forall X.B|]_\xi} $

By induction, we know that
\[ \inference{\Delta,X;\Gamma |- t : B & \xi';\rho'\in[|\Delta,X;\Gamma|]}
	{[|t|]_{\xi';\rho'} \in [|B|]_{\xi'}} ~
	(X\notin\FV(\Gamma))
\]
Since this holds for all $\xi';\rho' \in [|\Delta,X;\Gamma|]$ where
$X\notin\FV(\Gamma)$, it also holds for particular subset such that
$\xi' = \xi[X\mapsto\mathcal{A}]$ and $\rho'=\rho$ for all $\mathcal{A}\in\SAT$.
That is,
\[ [|t|]_{\xi[X\mapsto\mathcal{A}];\rho} \in [|B|]_{\xi[X\mapsto\mathcal{A}]}
   \quad \text{for all}~\mathcal{A}\in\SAT \]
From $X\notin\FV(\Gamma)$, we know that
$[|t|]_{\xi[X\mapsto\mathcal{A}];\rho} = [|t|]_{\xi;\rho}$
because $\rho$ is independent of what $X$ maps to.
So, we know that
\[ [|t|]_{\xi;\rho} \in [|B|]_{\xi[X\mapsto\mathcal{A}]}
	\quad \text{for all}~\mathcal{A}\in\SAT \]
By set theoretic definition, this is exactly what we wanted to show:
\[ [|t|]_{\xi;\rho} \in
	\bigcap_{\mathcal{A}\in\SAT} [|B|]_{\xi[X\mapsto\mathcal{A}]}
	= [|\forall X.B|]_\xi
\]

\paragraph{Case (\rulename{TyApp})}
We need to show that
$ \inference{\Delta;\Gamma |- t : B[A/X] & \xi;\rho\in[|\Delta;\Gamma|]}
	{[|t|]_{\xi;\rho} \in [|B[A/X]|]_\xi} $

By induction, we know that
$ \inference{\Delta;\Gamma |- t : \forall X.B & \xi';\rho'\in[|\Delta;\Gamma|]}
	{[|t|]_{\xi';\rho'} \in [|\forall X.B|]_{\xi'}} $.

Since this holds for all $\xi';\rho' \in [|\Delta,\Gamma|]$,
it also holds for $\xi'=\xi$ and $\rho'=\rho$. Then, we are done:
$ [|t|]_{\xi;\rho} \in [|\forall X.B|]_{\xi}
	= \bigcap_{\mathcal{A}\in\SAT} [|B|]_{\xi[X\mapsto\mathcal{A}]}
	\subseteq [|B|]_{\xi[X\mapsto[|A|]_\xi]} = [|B[A/X]|]_\xi $.
\end{proof}

\begin{corollary}[strong normalization]
	$\inference{\Delta;\Gamma |- t : A}{t \in \SN}$
\end{corollary}
Once we have proved the soundness of typing with respect to interpretation,
it is easy to see that STLC is strongly normalizing by giving a trivial term
interpretation $\rho(x) = x$ for all the free variables.
Note that $[|t|]_{\xi;\rho} = t \in [|A|]_\xi \subset \SN$
under the trivial interpretation.

    %% \section{System \F}                        \label{sec:f}
\section{System \Fw} \label{sec:fw}
System \Fw\ \cite{Gir72} extends the type syntax of System \F\ with lambda types
and application types (see Figure \ref{fig:fw}). Lambda types ($\l X^\kappa.F$)
and application types ($F\;G$), at the type level, are analogous to lambda terms
and applications at the term level. Type constructors are like functions, but
at the type level. Type constructors are categorized by kinds, just as terms
are categorized by types. Type constructors of kind $*$ are just \emph{types},
and do not expect any arguments. Type constructors that expect an argument
have arrow kinds ($\kappa -> \kappa'$). A type constructor of kind
$\kappa -> \kappa'$ expects another type constructor of kind $\kappa$
as an argument to produce yet another type constructor of kind $\kappa'$,
just as a function of type $A -> B$ expects another term of type $A$
as an argument, to produce yet another term of type $B$. By convention,
$A$ and $B$ stand for types (\ie, type constructors of kind $*$),
while $F$ and $G$ stand for type constructors or arbitrary kinds.

We can think of System \F\ as a restriction of System \Fw, where we only
allow types of kind $*$. That is, all the type variables appearing in
well-kinded types in System \F\ are of the star kind. Since there exists only
one kind ($*$) in System \F, the kinding rules of System \F\ need only to make
sure that type variables are bound in $\Delta$.

Since the kind structure of System \Fw\ is richer than the kind structure of
System \F, we need to keep track of the kind of the type variables in
the kinding context ($\Delta$). So, the kinding context is extended by
a type variable annotated by its kind ($X^\kappa$). The kinding rules of
System \Fw\ keep track of the kinds of type constructors as well as making
sure that the type variables are bound in $\Delta$.

The kinding rules, for the type syntax inherited from System \F:
\rulename{TVar}, \rulename{TArr}, \rulename{TAll} are similar to
their counterparts in System \F, except for this additional kinding annotation.
The kinding rules \rulename{TLam} and \rulename{TApp} state when the extensions
(lambda types and application types) to System \F\ are well-kinded.

The typing rules of System \Fw\ are almost identical to the typing rules of
System \F, except for one new rule \rulename{Conv}. The \rulename{Conv} rule
supports conversion between equivalent types.

\begin{itemize}
\item In the STLC, types are equal when they are syntactically identical.
\item In System \F, types are equal when they are $\alpha$-equivalent
(\ie, up to change of bound type variable names). For example,
$\forall X.X$ and $\forall X'.X'$ are considered to be same types in System \F.

\item
In System \Fw, we expect a richer notion of equality which incorporates
the notion of $\beta$-equivalence at the type level, since the type syntax of
System~\Fw\ has the structure of the STLC at the type level.
For instance, we want $(\lambda X^{*}.X) A = A$. 
\end{itemize}

The equality rules over the type constructors of System \Fw\ are
illustrated in Figure \ref{fig:eqtyfw}. The \rulename{EqTBeta} rule
describes the essence of $\beta$-equivalence.
Other rules describe the structural nature of equality (\rulename{EqTVar},
\rulename{EqTArr}, \rulename{EqTAll}, \rulename{EqTLam}, \rulename{EqTApp})
and transitivity of equality (\rulename{EqTTrans}).

%% syntax directed formalism???
%% http://pauillac.inria.fr/~herbelin/talks/cic.ps
%% there is a slide the refers to other paper and say that CIC is okay
%% since CIC is full. Is Fw also?

The complete syntax, kinding rules, and typing rules of System \Fw\
are illustrated in Figure \ref{fig:fw}. Since lambda binders exist
at both the term and type levels in System \Fw, we also have a choice of
either Church style (kind annotations on lambda types) or
Curry style (no kind annotations on lambda types) for the type syntax.
In this section (and, also throughout the dissertation), we consider
only the Church style type syntax, that is, the explicitly annotated kinds
on type-level lambdas. So, the distinction between the Church style and
the Curry style is only on the type syntax. The Church-style System \Fw,
on the left column of Figure \ref{fig:fw}, is a version of System \Fw\ with
the Church-style term syntax. The Curry-style System \Fw, on the right column,
is a version of System \Fw\ with the Curry-style term syntax.

The reduction rules of System \Fw\ (Figure \ref{fig:redfw}) are almost identical
to the reduction rules of System \F\ since the term syntax of System \Fw\ is
almost identical to the term syntax of System \F. Reduction rules are defined
only on the structure of terms, usually ignoring types.

The Church-style term syntax of System \Fw\ differs from the term syntax of
System \F\ in the type-abstraction ($\L X^\kappa.t$) terms. The difference is
the kind annotation on the type variable appearing in the System \Fw.
The Church-style term syntax is exactly the same as the term syntax of
System \F.

\begin{figure}
\begin{singlespace}
\begin{minipage}{.46\textwidth}
        \begin{center}Church style\end{center}\vspace*{-1em}
\def\baselinestretch{0}
\small
\begin{align*}
\textbf{term syntax} \\
t,s ::= &~ x               & \text{variable}    \\
      | &~ \l(x:A) . t     & \text{abstraction} \\
      | &~ t ~ s           & \text{application} \\
      | &~ \L X^\kappa . t & \text{type abstraction} \\
      | &~ t [G]           & \text{type application} \\
\textbf{type syntax} \\
F,G,A,B ::= &~ X                  & \text{variable type} \\
          | &~ A -> B             & \text{arrow type} \\
          | &~ \forall X^\kappa.B & \text{forall type}   \\
          | &~ \l X^\kappa.F      & \text{lambda type}   \\
          | &~ F ~ G              & \text{application type}   \\
\textbf{kind syntax} \\
\kappa ::= &~ \kappa -> \kappa' & \text{arrow kind} \\
         | &~ *                 & \text{star kind}   \\
\end{align*}
\[ \textbf{kinding rules} \quad \framebox{$ \Delta |- F:\kappa $} \]\vspace*{-1em}
\begin{align*}
& \inference[\sc TVar]{X^\kappa \in \Delta}{\Delta |- X:\kappa} \\
& \inference[\sc TArr]{\Delta |- A:* & \Delta |- B:*}{\Delta |- A -> B:*} \\
& \inference[\sc TAll]{\Delta,X^\kappa |- B:*}
                      {\Delta |- \forall X^\kappa.B:*} \\
& \inference[\sc TLam]{\Delta,X^\kappa |- F:\kappa'}
                      {\Delta |- \l X^\kappa.F:\kappa -> \kappa'} \\
& \inference[\sc TApp]{\Delta |- F : \kappa -> \kappa' & \Delta |- G : \kappa}
                      {\Delta |- F ~ G : \kappa'} \\
\end{align*}
\[ \textbf{typing rules} \quad \framebox{$ \Delta;\Gamma |- t : A $ } \]
\vspace*{-1em}
\begin{align*}
& \inference[\sc Var]{x:A \in \Gamma}{\Delta;\Gamma |- x:A} \\
& \inference[\sc Abs]{\Delta |- A:* & \Delta;\Gamma,x:A |- t : B}
                     {\Delta;\Gamma |- \l(x:A).t : A -> B} \\
& \inference[\sc App]{\Delta;\Gamma |- t : A -> B & \Delta;\Gamma |- s : A}
                     {\Delta;\Gamma |- t~s : B} \\
& \inference[\sc TyAbs]{\Delta,X^\kappa;\Gamma |- t : B}
                       {\Delta;\Gamma |- \L X^\kappa.t : \forall X^\kappa.B}
	 \begin{smallmatrix}(X\notin\FV(\Gamma))\end{smallmatrix} \\
& \inference[\sc TyApp]{\Delta;\Gamma |- t : \forall X^\kappa.B & \Delta |- G:\kappa}
                       {\Delta;\Gamma |- t[G] : B[G/X]} \\
& \inference[\sc Conv]{\Delta;\Gamma |- t : A & \Delta |- A = A' : *}
                      {\Delta;\Gamma |- t : A'}
\end{align*}
\end{minipage}
\begin{minipage}{.46\textwidth}
        \begin{center}Curry style\end{center}\vspace*{-1em}
\def\baselinestretch{0}
\small
\begin{align*}
\textbf{term syntax} \\
t,s ::= &~ x           \\
      | &~ \l x    . t \\
      | &~ t ~ s       \\
      \phantom{|} &~ \\
      \phantom{|} &~ \\
\textbf{type syntax} \\
F,G,A,B ::= &~ X                  \\
          | &~ A -> B             \\
          | &~ \forall X^\kappa.B \\
          | &~ \l X^\kappa.F      \\
          | &~ F ~ G              \\
\textbf{kind syntax} \\
\kappa ::= &~ \kappa -> \kappa' \\
         | &~ *                 \\
\end{align*}
\[ \textbf{kinding rules} \quad \framebox{$ \Delta |- F:\kappa$}\]\vspace*{-1em}
\begin{align*}
& \inference[\sc TVar]{X^\kappa \in \Delta}{\Delta |- X:\kappa} \\
& \inference[\sc TArr]{\Delta |- A:* & \Delta |- B:*}{\Delta |- A -> B:*} \\
& \inference[\sc TAll]{\Delta,X^\kappa |- B:*}{\Delta |- \forall X^\kappa.B:*} \\
& \inference[\sc TLam]{\Delta,X^\kappa |- F:\kappa'}
                      {\Delta |- \l X^\kappa.F:\kappa -> \kappa'} \\
& \inference[\sc TApp]{\Delta |- F : \kappa -> \kappa' & \Delta |- G : \kappa}
                      {\Delta |- F ~ G : \kappa'} \\
\end{align*}
\[ \textbf{typing rules} \quad \framebox{$ \Delta;\Gamma |- t : A $ } \]
\vspace*{-1em}
\begin{align*}
& \inference[\sc Var]{x:A \in \Gamma}{\Delta;\Gamma |- x:A} \\
& \inference[\sc Abs]{\Delta |- A:* & \Delta;\Gamma,x:A |- t : B}
                     {\Delta;\Gamma |- \l x   .t : A -> B} \\
& \inference[\sc App]{\Delta;\Gamma |- t : A -> B & \Delta;\Gamma |- s : A}
                     {\Delta;\Gamma |- t~s : B} \\
& \inference[\sc TyAbs]{\Delta,X^\kappa;\Gamma |- t : B}
                       {\Delta;\Gamma |- t : \forall X^\kappa.B}
	 \begin{smallmatrix}(X\notin\FV(\Gamma))\end{smallmatrix} \\
& \inference[\sc TyApp]{\Delta;\Gamma |- t : \forall X^\kappa.B & \Delta |- G:\kappa}
                       {\Delta;\Gamma |- t : B[G/X]} \\
& \inference[\sc Conv]{\Delta;\Gamma |- t : A & \Delta |- A = A' : *}
                      {\Delta;\Gamma |- t : A'}
\end{align*}
\end{minipage}
~\\
\caption{System \Fw\ in Church style and Curry style}
\label{fig:fw}
\end{singlespace}
\end{figure}

%% \begin{figure}
%% \begin{singlespace}
%% \begin{minipage}{.46\textwidth}
%%         \begin{center}Curry-Church-style\end{center}
%% \def\baselinestretch{0}
%% \small
%% \begin{align*}
%% \textbf{term syntax} \\
%% t,s ::= &~ x           & \text{variable}    \\
%%       | &~ \l x    . t & \text{abstraction} \\
%%       | &~ t ~ s       & \text{application} \\
%% \textbf{type syntax} \\
%% F,G,A,B ::= &~ X                  & \text{variable type}    \\
%%           | &~ A -> B             & \text{arrow type}       \\   
%%           | &~ \forall X^\kappa.B & \text{forall type}      \\
%%           | &~ \l X^\kappa.F      & \text{lambda type}      \\
%%           | &~ F ~ G              & \text{application type} \\
%% \textbf{kind syntax} \\
%% \kappa ::= &~ \kappa -> \kappa' & \text{arrow kind} \\
%%          | &~ *                 & \text{star kind}   \\
%% \end{align*}
%% \[ \textbf{kinding rules} \quad \framebox{$ \Delta |- F:\kappa $} \]\vspace*{-1em}
%% \begin{align*}
%% & \inference[\sc TVar]{X^\kappa \in \Delta}{\Delta |- X:\kappa} \\
%% & \inference[\sc TArr]{\Delta |- A:* & \Delta |- B:*}{\Delta |- A -> B:*} \\
%% & \inference[\sc TAll]{\Delta,X^\kappa |- B:*}{\Delta |- \forall X^\kappa.B:*} \\
%% & \inference[\sc TLam]{\Delta,X^\kappa |- F:\kappa'}
%%                       {\Delta |- \l X^\kappa.F:\kappa -> \kappa'} \\
%% & \inference[\sc TApp]{\Delta |- F : \kappa -> \kappa' & |- G : \kappa}
%%                       {\Delta |- F ~ G : \kappa'} \\
%% \end{align*}
%% \[ \textbf{typing rules} \quad \framebox{$ \Delta;\Gamma |- t : A $ } \]
%% \vspace*{-1em}
%% \begin{align*}
%% & \inference[\sc Var]{x:A \in \Gamma}{\Delta;\Gamma |- x:A} \\
%% & \inference[\sc Abs]{\Delta |- A:* & \Delta;\Gamma,x:A |- t : B}
%%                      {\Delta;\Gamma |- \l x   .t : A -> B} \\
%% & \inference[\sc App]{\Delta;\Gamma |- t : A -> B & \Delta;\Gamma |- s : A}
%%                      {\Delta;\Gamma |- t~s : B} \\
%% & \inference[\sc TyAbs]{\Delta,X^\kappa;\Gamma |- t : B}
%%                        {\Delta;\Gamma |- t : \forall X^\kappa.B} \\
%% & \inference[\sc TyApp]{\Delta;\Gamma |- t : \forall X^\kappa.B & \Delta |- G:\kappa}
%%                        {\Delta;\Gamma |- t : B[G/X]} \\
%% & \inference[\sc Conv]{\Delta;\Gamma |- t : A & \Delta |- A = A' : *}
%%                       {\Delta;\Gamma |- t : A'}
%% \end{align*}
%% \end{minipage}
%% \begin{minipage}{.46\textwidth}
%%         \begin{center}Curry-Curry-style\end{center}
%% \def\baselinestretch{0}
%% \small
%% \begin{align*}
%% \textbf{term syntax} \\
%% t,s ::= &~ x           \\
%%       | &~ \l x    . t \\
%%       | &~ t ~ s       \\
%% \textbf{type syntax} \\
%% F,G,A,B ::= &~ X                  \\
%%           | &~ A -> B             \\   
%%           | &~ \forall X.B \\
%%           | &~ \l X.F      \\
%%           | &~ F ~ G              \\
%% \textbf{kind syntax} \\
%% \kappa ::= &~ \kappa -> \kappa'  \\
%%          | &~ *                  \\
%% \end{align*}
%% \[ \textbf{kinding rules} \quad \framebox{$ \Delta |- F:\kappa $} \]\vspace*{-1em}
%% \begin{align*}
%% & \inference[\sc TVar]{X^\kappa \in \Delta}{\Delta |- X:\kappa} \\
%% & \inference[\sc TArr]{\Delta |- A:* & \Delta |- B:*}{\Delta |- A -> B:*} \\
%% & \inference[\sc TAll]{\Delta,X^\kappa |- B:*}{\Delta |- \forall X.B:*} \\
%% & \inference[\sc TLam]{\Delta,X^\kappa |- F:\kappa'}
%%                       {\Delta |- \l X.F:\kappa -> \kappa'} \\
%% & \inference[\sc TApp]{\Delta |- F : \kappa -> \kappa' & |- G : \kappa}
%%                       {\Delta |- F ~ G : \kappa'} \\
%% \end{align*}
%% \[ \textbf{typing rules} \quad \framebox{$ \Delta;\Gamma |- t : A $ } \]
%% \vspace*{-1em}
%% \begin{align*}
%% & \inference[\sc Var]{x:A \in \Gamma}{\Delta;\Gamma |- x:A} \\
%% & \inference[\sc Abs]{\Delta |- A:* & \Delta;\Gamma,x:A |- t : B}
%%                      {\Delta;\Gamma |- \l x   .t : A -> B} \\
%% & \inference[\sc App]{\Delta;\Gamma |- t : A -> B & \Delta;\Gamma |- s : A}
%%                      {\Delta;\Gamma |- t~s : B} \\
%% & \inference[\sc TyAbs]{\Delta,X^\kappa;\Gamma |- t : B}
%%                        {\Delta;\Gamma |- t : \forall X.B} \\
%% & \inference[\sc TyApp]{\Delta,X^\kappa;\Gamma |- t : B & \Delta |- G:\kappa}
%%                        {\Delta;\Gamma |- t : B[G/X]} \\
%% & \inference[\sc Conv]{\Delta;\Gamma |- t : A & \Delta |- A = A' : *}
%%                       {\Delta;\Gamma |- t : A'}
%% \end{align*}
%% \end{minipage}
%% ~\\
%% \caption{System \Fw\ in Curry-Church-style and Curry-Curry-style}
%% \label{fig:fw2}
%% \end{singlespace}
%% \end{figure}

\begin{figure}
\paragraph{Reduction rules for the Church-style System \Fw}
\begin{align*}
& \inference[\sc RedBeta]{}{(\l(x:A).t)~s --> t[s/x]}
&&\inference[\sc RedTy]{}{(\L X   .t)[A] --> t[A/X]} \\
& \inference[\sc RedAbs]{t --> t'}{\l x   .t --> \l x   .t'}
&&\inference[\sc RedTyAbs]{t --> t'}{\L X^\kappa   .t --> \L X^\kappa   .t'} \\
& \inference[\sc RedApp1]{t --> t'}{t~s --> t'~s}
&&\inference[\sc RedTyApp]{t --> t'}{t[A] --> t'[A]} \\
& \inference[\sc RedApp2]{s --> s'}{t~s --> t~s'}
\end{align*}
\paragraph{Reduction rules for the Curry-style System \Fw}
\begin{align*}
& \inference[\sc RedBeta]{}{(\l x   .t)~s --> t[s/x]} \\
& \inference[\sc RedAbs]{t --> t'}{\l x   .t --> \l x   .t'} \\
& \inference[\sc RedApp1]{t --> t'}{t~s --> t'~s} \\
& \inference[\sc RedApp2]{s --> s'}{t~s --> t~s'}
\end{align*}
\caption{Reduction rules for System \Fw}
\label{fig:redfw}
\end{figure}

\begin{figure}
\begin{align*}
& \inference[\sc EqTBeta]
        {\Delta,X^\kappa |- F : \kappa -> \kappa' & \Delta |- G : \kappa}
        {\Delta |- (\l X^\kappa.F)\;G = F[G/X] : \kappa'} \\
& \inference[\sc EqTVar]{X^\kappa \in \Delta}{\Delta |- X = X : \kappa} \\
& \inference[\sc EqTArr]{\Delta |- A=A':* & \Delta |- B=B':*}
                        {\Delta |- A -> B=A' -> B':*} \\
& \inference[\sc EqTAll]{\Delta,X^\kappa |- B=B':*}
                        {\Delta |- \forall X^\kappa.B=B':*} \\
& \inference[\sc EqTLam]
        {\Delta,X^\kappa |- F=F':\kappa'}
        {\Delta |- \l X^\kappa.F=\l X^\kappa.F':\kappa -> \kappa'} \\
& \inference[\sc EqTApp]
        {\Delta |- F=F' : \kappa -> \kappa' & \Delta |- G=G' : \kappa}
        {\Delta |- F ~ G = F' ~ G' : \kappa'} \\
& \inference[\sc EqTTrans]
        {\Delta |- F=F' : \kappa & \Delta |- F'=F'' : \kappa}
        {\Delta |- F=F'' : \kappa'}
\end{align*}
\caption{Type constructor equality rules for System \Fw}
\label{fig:eqtyfw}
\end{figure}

%% \paragraph{From Curry-Church-style to Curry-Curry-style.}
%% It is also possible to have a version of \Fw\ where the type syntax is also in Curry style.
%% That is, the forall type ($\forall X.B$) and the lambda type ($\l X.B$)
%% do not have kind annotations. We call this version of \Fw, where both terms
%% and types are unannotated, the Curry-Curry-style \Fw.
%% In Figure \ref{fig:fw2}, the Curry-Curry-style \Fw\ (right) is laid out
%% side-by-side to the Curry-Church-style \Fw\ (left).
%% 
%% Since we changed the type syntax of forall types and lambda types in
%% the Curry-Curry-style \Fw\ (\ie, removed the kind annotations from
%% the type constructor syntax), we need to adjust the kinding rules
%% involving forall types and lambda types. The rules we need to adjust are
%% the kinding rules \rulename{TAll} and \rulename{TLam}, and
%% the typing rules \rulename{TyAbs} and \rulename{TyApp}.
%% 
%% For the kinding rules, \rulename{TAll} and \rulename{TLam}, we simply need to
%% drop the kind annotations appearing in the forall type and
%% the lambda type in each rule. In Figure \ref{fig:fw2}, you can see that
%% that the difference between the left (Curry-Church-style) and
%% right (Curry-Curry-style) is the mssing kind annotations in these two rules.
%% 
%% How should we adjust the typing rules \rulename{TyAbs} and \rulename{TyApp}
%% for the Curry-Curry-style \Fw? A naive approach might be to adjust
%% the kind annotations (just as we did for the kinding rules) to
%% the changes below:
%% \begin{align*}
%% & \inference[\sc TyAbs]{\Delta,X^\kappa;\Gamma |- t : B}
%%                        {\Delta;\Gamma |- t : \forall X.B}
%% & \inference[\sc TyApp]{\Delta;\Gamma |- t : \forall X.B & \Delta |- G:\kappa}
%%                        {\Delta;\Gamma |- t : B[G/X]}
%% \end{align*}
%% The \rulename{TyAbs} rule above is fine. However, the \rulename{TyApp} rule
%% above is problematic because it fails to require that both $X$ and
%% $G$ have kind $\kappa$. To ensure that $X$ is of kind $\kappa$, we need to
%% adjust the \rulename{TyApp} rule as follows:
%% \begin{align*}
%% & \phantom{ \inference[\sc TyAbs]{\Delta,X^\kappa;\Gamma |- t : B}
%%                                  {\Delta;\Gamma |- t : \forall X.B} }
%% & \inference[\sc TyApp]{\Delta,X^\kappa;\Gamma |- t : B & \Delta |- G:\kappa}
%%                         {\Delta;\Gamma |- t : B[G/X]}
%% \end{align*}
%% Note, the first premise ($\Delta,X^\kappa;\Gamma |- t : B$) of
%% the \rulename{TyApp} rule is exactly the same as the premise of
%% the \rulename{TyAbs} rule.
%% 
%% \paragraph{Church-Curry-style \Fw}
%% Although not illustrated in Figures \ref{fig:fw} and \ref{fig:fw2},
%% we can imagine yet another version of \Fw\ with annotated terms
%% and unannotated types -- namely, the Church-Curry-style \Fw.
%% Since we make no use of this style in the thesis, we leave the details
%% as an exercise for the reader.


\subsection{Encodings of datatypes in System \Fw}
\label{sec:fw:data}
In System \Fw\ we can encode all the datatypes encodable in System \F\ (see
\S\ref{sec:f:data}) and more. In addition to the obvious type constructors,
one can encode indexed types, nested types, and even fixpoint operators
over type constructors.
\begin{itemize}

\item \emph{Type constructors} for polymorphic datatypes
can be encoded using lambda types that abstract over types.


\item \emph{Non-regular datatypes}, or \emph{nested datatypes}, can be encoded
using forall types that are polymorphic over type constructors.

\item With higher-kinded type constructors, we can even encode
the recursive type operator $\mu$ in System \Fw\ by abstracting
over non-star type constructors.
\end{itemize}

This additional expressive power comes from two different uses of
type-level constructs other than types of kind $*$.
\begin{itemize}

\item  \emph{Higher-kinded polymorphism} is the ability to
	universally quantify over both type constructors
	of arbitrary kinds.

\item 
\emph{Type constructors of higher kinds} or \emph{higher-kinded
type constructors} are type constructors that expect type constructors
as their arguments.
\end{itemize}

In fact we combine these two to define 
a family of kind-indexed recursive type operators $\mu_\kappa$ using both
higher-kinded type constructors and higher-kinded polymorphism.



\paragraph{Type constructors for polymorphic datatypes} expect
other types as arguments to produce a datatype. We can encode these 
type constructors in System \Fw. For example, the shorthand notations
(or, type synonyms) in \S\ref{sec:f:data}, such as $(\times)$ for pair types
and $(+)$ for sum types, can be encoded as as follows:\footnote{Here,
        I used a Haskell-ish notation of turning a infix binary operator
        into a prefix binary operator by surrounding the operator in parenthesis
        (\eg, $(+)\;X_1\,X_2 = X_1 + X_2$). I also annotated the kinds of
        the type constructors after the colon (:).}
\begin{align*}
(\times) &= \l X_1^{*}.\l X_2^{*}.(X_1 -> X_2 -> X) -> X &:~ * -> * -> * \\
     (+) &= \l X_1^{*}.\l X_2^{*}.(X_1 -> X) -> (X_2 -> X) -> X &:~ * -> * -> *
\end{align*}
Type constructors for polymorphic recursive datatypes are encodable as well.
For instance, we can encode the constructor \textit{List}
for the polymorphic list datatype:
\begin{align*}
        \textit{List} &= \l X_a^{*}.\forall X^{*}.(X_a -> X -> X) -> X -> X
        &:~ * -> *
\end{align*}
In System \F, type constructors, such as $(\times)$, $(+)$, and
\textit{List} are meta-level concepts (or, shorthand notations, macros)
that cannot be encoded within the type system of System \F.
In System \Fw, these datatype constructors are encodable as type constructors,
which are ordinary constructs of System \Fw.

\paragraph{Higher-kinded datatype constructors} that expect
not only types but also type constructors of arbitrary kinds as arguments
are encodable in System \Fw\ as well. For example, we can encode
\textit{Flip}, which flips the order of the first and second arguments of
a binary type constructor (\ie, $(\textit{Flip}\;F)\,A_1\,A_2 = F\,A_2\,A_1$),
and \textit{Comp}, which composes two unary type constructors
(\ie, $(\textit{Comp}\;F_1\,F_2)\,A = F_1\,(F_2\,A)$), as follows:
\begin{align*}
\textit{Flip} &= \l X_{\!f}^{*-> *-> *}.\l X_1^{*}.\l X_2^{*}.X_{\!f}\,X_2\;X_1
                     &:\;& (* -> * -> *) -> * -> * -> * \\
\textit{Compose} &= \l X_{\!f}^{*-> *}.\l X_g^{*-> *}.\l X^{*}.X_{\!f}\,(X_g\,X)
                     &:\;& (* -> *) -> (* -> *) -> * -> *
\end{align*}

\paragraph{Higher-kinded polymorphism} is the ability universally quantify
over type constructors as well as types. That is, we can have
$\forall X^\kappa.B$ where $\kappa$ is not the star kind.
We can encode \emph{non-regular (recursive) datatypes} in System \Fw\ using
higher-kinded polymorphism.

We mentioned that we can encode \emph{regular (recursive) datatypes}
in Systerm \F\ (\S\ref{sec:f:data}), but have not discussed
what regular datatypes are. A representative example of a regular datatype
is the polymorphic list type ($\forall X_a.\textit{List}\,X_a$).
We say that the polymorphic list type is regular since its recursive
component, the tail, has exactly the same type. That is, for any
non-empty list of type $\textit{List}\,A$, its tail must be of type $\textit{List}\,A$.
Many other well-known recursive datatypes are also regular (\eg, binary trees).

But, one can imagine a non-regular twist to the regular polymorphic list type
by insisting the recursive components (\ie, tails) have different
type arguments from the list they are part of. For instance, we may insist
that a list-like datatype of type ($\textit{Powl}\;A$) must have its tail be
of type $(\textit{Powl}\,(A\times A))$. That is, if the first element is
an integer (\eg, $1$), then the second element must be a pair of integers
(\eg, $(2,3)$), and the third elment must be a pair of pair of integers
(\eg, $((4,5),(6,7))$), and so on. We can depict an example of this list-like
datatype with three elements as: $[1,\,(2,3),\,((4,5),(6,7))]$.
This is a representative example of a non-regular datatype called powerlists.
Such datatypes are also called \emph{nested datatypes} %% TODO cite
since the the type constructor is applied to ever increasing complex arguments
(here they are nested, but one can imagine even richer kinds of complexity)
as we step further inside the recursive components.

We can encode the type constructor \textit{Powl} for powerlists using
higher-kinded polymorphism of System \Fw, as follows
(\cf\ encoding of \textit{List}):
\begin{align*}
\textit{Powl} &= \l X_a^{*}.\forall X^{* -> *}&.&
        (X_a -> X(X_a\times X_a) & -> & X\,X_a) & -> & X\,X_a & -> & X\,X_a \\
\textit{List} &= \l X_a^{*}.\forall X^{*}&.&
        (X_a -> X & -> & X) & -> & X & -> & X
\end{align*}
Unlike the encoding of \textit{List}, where $X$ is polymorphic over types
of kind $*$, the universally quantified variable $X$ in the encoding of
\textit{Powl} is polymorphic over constructors of kind $* -> *$.
Intuitively, $X$ in the list encoding corresponds to $\textit{List}\;X_a$
(\ie, the type constructor \textit{List} applied to its uniform argument $X_a$),
and, $X$ in the powerlist encoding corresponds to \textit{Powl} without
being applied to its argument so that it may be applied to a non-regular
argument (\eg, $X(X_a\times X_a)$). See \S\ref{mendler_nonreg} for more
examples and discussions on non-regular datatypes.

\paragraph{The recursive type operator $\mu$} builds a recursive type
($\mu F$) from a non-recursive base structure ($F:* -> *$).
Theories on recursive datatypes are often formulated in terms of
the recursive type operator $\mu$, which satisfies the property
that $\mu F = F (\mu F)$ for any $F: * -> *$. A recursive datatype ($\mu F$)
is built from its base structure ($F$) by applying the recursive operator.
For example, the natural number datatype can be built from the base structure
$F = \l X_r^{*}.X_r + \textit{Unit}$. Intuitively, we can understand this
base structure as a specification for natural numbers: a natural number is
either a successor of a recursive object ($X_r$)
or zero encoded as the unit object (\textit{Unit}).
From this base structure, we can define
$\textit{Nat} = \mu(\l X_r^{*}.X_r + \textit{Unit})$.
Let us write down the desired property of $\mu$ for $\textit{Nat}$.
\begin{align*}
\mu(\l X_r^{*}.X_r + \textit{Unit}) &=
(\l X_r^{*}.X_r + \textit{Unit})(\mu(\l X_r^{*}.X_r + \textit{Unit})) \\
\textit{Nat} &= (\l X_r^{*}.X_r + \textit{Unit})\,\textit{Nat} \\
\textit{Nat} &= \textit{Nat}\, + \textit{Unit}
\end{align*}
Note, the simplified last equation looks very similar to
the recursive datatype definitions for unary natural numbers
in functional languages, such as Haskell: \[ \qquad\qquad
\textbf{data}~\textit{Nat} = \texttt{Succ}~\textit{Nat}\,\mid\,\texttt{Zero} \]
See Chapter \ref{ch:mendler}
for more of the Haskell examples on recursive datatypes and $\mu$.

Although recursive datatypes are encodable in System \F\ (\S\ref{sec:f:data}),
extensions of System \F\ with $\mu$ have been studied %% TODO cite
since one can reason about the properties of recursive datatypes more
uniformly by factoring out the recursion at the type level as
the fixpoint $\mu$. In System \Fw, we can encode $\mu$ using
higher-kinded type constructors and higher-kinded polymorphism as follows:
\[
\mu =
 \l X_{\!f}^{* -> *}.
 \forall X'^{*}.(\forall X_r^{*}.(X_r -> X') -> X_{\!f}\,X_r -> X') -> X'
 ~:\; (* -> *) -> *
\]
Let us intuitively derive above the encoding of $\mu$ starting from
the impredicative encoding of natural numbers:
\begin{align*}
\textit{Nat}
        &= \forall X^{*} . (X -> X) -> X -> X \\
        &\cong \forall X^{*} . (X -> X) -> (\textit{Unit} -> X) -> X 
                &(\because \textit{Unit} -> X \cong X) \\
        &\cong \mu(\l X_r^{*}.\,X_r + \textit{Unit})
                & (\text{to show})
\end{align*}
We want to show that the impredicative encoding of natural numbers is
equivalent to the natural number type defined using $\mu$. We need to turn
the impredicative encoding of natural numbers into a non-recursive
base structure by abstract away the recursive component, which is
the underlined part below. That is, we replace the underlined $X$
with a new variable $X_r$:
\begin{align*}
\forall X^{*} . (\underline{X}\; -> X) -> (\textit{Unit} -> X) -> X \\
\forall X^{*} . (X_r -> X) -> (\textit{Unit} -> X) -> X
\end{align*}
Recall that
$X_r +\textit{Unit} = \forall X^{*} . (X_r -> X) -> (\textit{Unit} -> X) -> X$.
Recall that the idea behind the impredicative encoding is that we can eliminate
an object of the datatype into an arbitrary result type $X$. If we are to
encode datatypes constructed by $\mu$, we apply this idea of
impredicative encoding in two layers: for the base structure and for $\mu$.
We already know how to encode the base structure: with the encoding above,
we can eliminate to an arbitrary result type $X$. For $\mu$, we introduce
yet another variable $X'$ so that we can eliminate to an arbitrary result
type $X'$. Thus, the encoding for natural number type constructed using $\mu$
would be of the following form:
\[ \forall X'^{*}.(\;\dots\;\dots\;\dots\;\dots\;\dots (X_r + \textit{Unit}) -> X') -> X' \]
Since the recursive type contains the base structure, we would be able to
eliminate the recursive type, given that we know how to eliminate
the base structure $((X_r + \textit{Unit}) -> X')$.
However, this is not yet complete because we do not know how to eliminate $X_r$.
So, we require that we should also know how to eliminate $X_r$, as follows:
\[ \forall X'^{*}.
        (\forall X_r^{*}.(X_r -> X') -> (X_r + \textit{Unit}) -> X') -> X' \]
We can derive the encoding for $\mu$ (repeated below)
so that $\mu(\l X_r^{*}.X_r + \textit{Unit})$ is equivalent to above.
\[
\mu =
 \l X_{\!f}^{* -> *}.
 \forall X'^{*}.(\forall X_r^{*}.(X_r -> X') -> X_{\!f}\,X_r -> X') -> X'
 ~:\; (* -> *) -> *
\]
Note that $X_r$ is also universally quantified
in $(\forall X_r^{*}.(X_r -> X') -> X_{\!f}\,X_r -> X')$
locally.
See Chapter \ref{ch:mendler} %% TODO update the fwd ref to a section later
for an intuitive explanation for why $X_r$ should be universally quantified.

The (data) constructor for the recursive type operator $\mu$ is called $\In$
and the eliminator is called $\MIt$. The encodings of $\In$ and $\MIt$ as
Curry-style terms are as follows:
\[ \In = \l x_r. \l x_\varphi.x_\varphi\,(\MIt~x_\varphi)\,x_r
\qquad\qquad \MIt = \l x_\varphi.\l x_r.x_r\,x_\varphi \]
These ($\mu$, $\In$, and $\MIt$) are, in fact, encodings for
Mendler-style iteration, which will be discussed in \S\ref{sec:proof}.
\index{Mendler-style}

\paragraph{A kind-indexed family of recursive type operators $\mu_\kappa$:}
The recursive type operator $\mu : (* -> *) -> *$ discussed so far can only
construct (non-mutually recursive) regular datatypes. For example,
\begin{align*}
\textit{Nat} &= \mu(\l X^{*}.X + \textit{Unit}) \\
\textit{List} &= \l X_a^{*}.\mu(\l X^{*}.(X_a\times X) + \textit{Unit})
\end{align*}
More generally, there is a family of recursive type operators
$\mu_\kappa : (\kappa -> \kappa) -> \kappa$ for each kind $\kappa$.
The $\mu$, which we discussed above, is $\mu_{*} : (* -> *) -> *$.
We can construct \textit{Powl}, which is a non-regular datatype, using another
recursive typer operator $\mu_{* -> *} : ((*-> *) -> (*-> *)) -> (*-> *) $
as follows (\cf\ \textit{List}).
\begin{align*}
\textit{Powl} &= \mu_{* -> *}(\l X^{* -> *}.\l X_a^{*}.
                        (X_a\times X(X_a\times X_a)) + \textit{Unit}) \\
\textit{List} &= \l X_a^{*}.\mu_{*}(\l X^{*}.(X_a\times X) + \textit{Unit})
\end{align*}
Note the difference where $X_a$ is bound in the definitions of \textit{Powl}
and \textit{List}. The encodings of $\mu_{*}$ and $\mu_{*-> *}$ in System \Fw\ 
are shown below:
\begin{align*}
\mu_{*} &=
 \l X_{\!f}^{* -> *}.\forall X'^{*}.
 (\forall X_r^{*}.(X_r -> X') -> (X_{\!f}\,X_r -> X')) -> X' \\
\mu_{* -> *} &=
 \l X_{\!f}^{(*-> *) -> (*-> *)}.\l X_a^{*}.\\&\qquad\quad \forall X'^{*-> *}.
 \big(\forall X_r^{*-> *}.
        (\forall X_a^{*}.X_r\,X_a -> X'\,X_a) -> \\&\qquad\qquad\qquad\qquad\qquad~
        (\forall X_a^{*}.X_{\!f}\,X_r\,X_a -> X'\,X_a)\big) -> X'\,X_a
\end{align*}
The genreal form for the encoding of $\mu_\kappa$
is as follows:
\begin{align*}
\mu_{\kappa} &=
 \l X_{\!f}^{\kappa -> \kappa}.\l \vec{X}^{\vec{\kappa}}.
 \forall X'^{*-> *}.
 \big(\forall X_r^{\kappa -> \kappa}.
 (\forall \vec{X}^{\vec{\kappa}}.X_r\,\vec{X} -> X'\,\vec{X}) -> \\
 &\qquad\qquad\qquad\qquad\qquad\qquad\qquad\quad
 (\forall \vec{X}^{\vec{\kappa}}.X_{\!f}\,X_r\,\vec{X} -> X'\,\vec{X})
 \big) -> X'\,\vec{X}
\end{align*}
where $\vec{X}$ denotes a sequence of $n$ variables
such that $n=0$ when $\kappa = *$, otherwise, $n = |\vec{\kappa}|$ when
$\kappa = \vec{\kappa} -> * = \kappa_1 -> \cdots -> \kappa_n -> *$.\footnote{
        $\kappa$ always end up with $*$ when it is an arrow kind
        since $->$ is right associative by convention.}
That is, we can simply erase all the $\l \vec{X}^{\vec{\kappa}}$,
$\forall \vec{X}^{\vec{\kappa}}$, and $\vec{X}$ from above when
$\kappa = *$, otherwise, $\l \vec{X}^{\vec{\kappa}}$ stands for
$\l X_1^{\kappa_1}.\cdots.\l X_n^{\kappa_n}$,
$\forall \vec{X}^{\vec{\kappa}}$ stands for
$\forall X_1^{\kappa_1}.\cdots.\forall X_n^{\kappa_n}$,
and $F\,\vec{X}$ stands for $F\,X_1\cdots X_n$
when $\kappa = \vec{\kappa} -> * = \kappa_1 -> \cdots -> \kappa_n -> *$.

The (data) constructor for the recursive type operator $\mu_\kappa$ is
called $\In_\kappa$ and the eliminator is called $\MIt_\kappa$.
The encodings of $\In_\kappa$ and $\MIt_\kappa$ as Curry-style terms are
exactly the same as for $\In$ and $\MIt$ for the star kind:
\[ \In_\kappa = \l x_r. \l x_\varphi.x_\varphi\,(\MIt~x_\varphi)\,x_r
\qquad\qquad \MIt_\kappa = \l x_\varphi.\l x_r.x_r\,x_\varphi \]
These ($\mu_\kappa$, $\In_\kappa$, and $\MIt_\kappa$) are, in fact,
encodings for Mendler-style iteration in \Fw, which will discussed in
\S\ref{sec:fi:data}.
\index{Mendler-style}


\subsection{Strong normalization}\label{sec:fw:srsn}
Here, we will take the \emph{subject reduction} (Theorem\;\ref{thm:srfw})
(\aka\ \emph{type preservation}) for granted,\footnote{
	The proof for subject reduction of System~\Fw\ is similar to
	the proof for the subject reduction of System~\F.}
and focus our discussion on the \emph{strong normalization}
of System~\Fw.

\begin{theorem}[subject reduction]\label{thm:srfw}
$\inference{\Delta;\Gamma |- t : A  & t --> t'}{\Delta;\Gamma |- t' : A}$
\end{theorem}

\begin{figure}
\begin{singlespace}
\begin{description}
\item[Interpretation of kinds] as pointwise generalization of $\SAT$
        \[ [| \kappa |] = \SAT_\kappa \]
\item[Interpretation of type constructors]
        as function spaces over saturated sets of normalizing terms
        whose free type variables are substituted according to
        the given type constructor valuation ($\xi$):
\begin{align*}
[| X |]_\xi      &= \xi(X) \\ 
[| A -> B |]_\xi &= [|A|]_\xi -> [|B|]_\xi \\
[| \forall X^\kappa . B |]_\xi
        &= \bigcap_{\mathcal{F}\in[|\kappa|]} [|B|]_{\xi[X\mapsto\mathcal{F}]}
                \qquad\qquad\qquad (X\notin\dom(\xi)) \\
[| \l X^\kappa . F |]_\xi
        &= \bbl(\mathcal{G} \in [|\kappa|]) . [|F|]_{\xi[X\mapsto\mathcal{G}]}
                \qquad\quad~ (X\notin\dom(\xi)) \\
[| F \; G |]_\xi &= [|F|]_\xi ( [|G|]_\xi )
\end{align*}
\item[Interpretation of kinding and typing contexts]
        as sets of type constructor valuations and term valuations
        ($\xi$ and $\rho$):
\begin{align*}
[| \Delta        |] &= \{ \xi \in \dom(\Delta) -> \bigcup_{\kappa} [|\kappa|] \mid \xi(x)\in[|\Delta(x)|] ~\text{for all}~x\in\dom(\Delta) \} \\
[| \Delta;\Gamma |] &= \{ \xi;\rho \mid \xi\in[|\Delta|], \rho\in[|\Gamma|]_\xi \} \\
[| \Gamma        |]_\xi\ &= \{ \rho \in \dom(\Gamma) -> \SN \mid \rho(x)=[|\Gamma(x)|]_\xi ~\text{for all}~x\in\dom(\Gamma) \}
\end{align*}
\item[Interpretation of terms]
        as terms themselves whose free variables are substituted according to
        the given pair of type constructor and term valuations
        ($\xi$;$\rho$):
\begin{align*}
[| x      |]_{\xi;\rho} &= \rho(x) \\
[| \l x.t |]_{\xi;\rho} &= \l x . [|t|]_{\xi;\rho} \qquad (x\notin\dom(\rho)) \\
[| t ~ s  |]_{\xi;\rho} &= [| t |]_{\xi;\rho} ~ [| s |]_{\xi;\rho}
\end{align*}
\end{description}
\caption[Interpretation of System \Fw\ for proving strong normalization]
        {Interpretation of type constructors, kinding and typing contexts,
                and terms of System \Fw\ for the proof of strong normalization}
\label{fig:interpFw}
\end{singlespace}
\end{figure}
To prove strong normalization of System \F, we use the same proof strategy
as in the proof of strong normalization of System \F\ (\S\ref{sec:f:srsn}).
That is, we interpret types as saturated sets of normalizing terms as we did
for System \F. However, we need to generalize the interpretation of types to
the interpretation of type constructors. In the strong normalization proof of
System \F, we had a complete lattice $(\SAT,\subseteq)$. We generalize from
$(\SAT,\subseteq)$, which is for kind $*$ only, to
$(\SAT_\kappa,\sqsubseteq_\kappa)$ for an arbitrary kind $\kappa$,
as follows:
\begin{itemize}
\item The set $\SAT_\kappa$ is a generalization of $\SAT$ such that
\begin{align*}
        \SAT_{*} &= \SAT \\
        \SAT_{\kappa -> \kappa'} &= \SAT_\kappa -> \SAT_{\kappa'}
        \qquad \text{(\ie, functions from $\SAT_\kappa$ to $\SAT_\kappa'$)}.
\end{align*}

\item The relation $\sqsubseteq_\kappa$ is
        a pointwise generalization of $\subseteq$ such that
\begin{align*}
\mathcal{A} \sqsubseteq_{*} \mathcal{A'} &= \mathcal{A} \subseteq \mathcal{A'}\\
\mathcal{F} \sqsubseteq_{\kappa -> \kappa'} \mathcal{F'} &=
        \mathcal{F}(\mathcal{G}) \sqsubseteq_{\kappa'} \mathcal{F'}(\mathcal{G})
        ~\text{for all}~\mathcal{G}\in\SAT_\kappa
\end{align*}
\end{itemize}
It is easy to see that $(\SAT_\kappa,\sqsubseteq_\kappa)$ forms
a complete lattice by induction on $\kappa$. For kind $\star$,
it is obvious since we already know that $(\SAT,\subseteq)$ forms
a complete lattice. For an arrow kind $\kappa -> \kappa'$, we know that
$(\SAT_{\kappa'},\sqsubseteq_{\kappa})$ forms a complete lattice by induction.
It is easy to see that for every two element $\mathcal{F}_1, \mathcal{F}_2
\in \SAT{\kappa'},\sqsubseteq_{\kappa}$ there exist a greatest lower bound
($\mathcal{F}_1 \wedge \mathcal{F}_2$) and a greatest upper bound 
($\mathcal{F}_1 \vee \mathcal{F}_2$), defined pointwisely as follows:
\begin{align*}
(\mathcal{F}_1 \wedge \mathcal{F}_2)(\mathcal{G}) &=
   \mathcal{F}_1(\mathcal{G}) \wedge \mathcal{F}_2(\mathcal{G})
        ~\text{for all}~\mathcal{G}\in\SAT_\kappa \\
(\mathcal{F}_1 \vee \mathcal{F}_2)(\mathcal{G}) &=
   \mathcal{F}_1(\mathcal{G}) \vee \mathcal{F}_2(\mathcal{G})
        ~\text{for all}~\mathcal{G}\in\SAT_\kappa
\end{align*}
The top and bottom elements at an arrow kind $\bot_{\kappa -> \kappa'}$
are also defined pointwisely. Let $\bot_{\kappa -> \kappa'}$ be
the constant function that that always return $\bot_{\kappa'}$
(the bottom element at $\kappa'$, and, let $\top_{\kappa -> \kappa'}$ be
the constant function that that always return $\top_{\kappa'}$
(the top element of the lattice at $\kappa'$). It is easy to see that
$\bot_{\kappa -> \kappa'}$ and $\top_{\kappa -> \kappa'}$
are the bottom and top elements at kind $\kappa -> \kappa'$
by definition of $\sqsubseteq_{\kappa -> \kappa'}$.

Then, we can give interpretation of kind $\kappa$ as $\SAT_\kappa$.
That is, $[| \kappa |] = \SAT_\kappa$. An interpretation of a type constructor
of kind $\kappa$ should be a member of $[| \kappa |]$, \ie, $\SAT_\kappa$.
The interpretation of kinds, type constructors, contexts, and
terms of System \Fw\ are illustrated in Figure \ref{fig:interpFw}. 

We use the Curry-style System \Fw\ to present the strong normalization
proof. It is more convenient to interpret terms in Curry style since
the Curry-style terms syntax is simpler than the Church-style term syntax.
It is more convenient to interpret type constructors in Curry style since
the kind annotation makes it clear how to interpret the bound type variable $X$
in forall types and lambda types (\ie, for $X^\kappa$ choose from $[|\kappa|]$).

The proof of strong normalization amounts to proving the following theorem:
\begin{theorem}[soundness of typing]
$ \inference{\Delta;\Gamma|- t:A & \xi;\rho \in [|\Delta;\Gamma|]}
            {[|t|]_{\xi;\rho} \in [|A|]_\xi} $
\end{theorem}
\begin{proof}
We prove by induction on the typing derivation ($\Delta;\Gamma |- t:A$).

The cases for \rulename{Var}, \rulename{Abs}, and \rulename{App} are pretty
much the same as the strong normalization proof for System \F.
The cases for \rulename{TyAbs} and \rulename{TyApp} is almost the same
as the strong normalization proof for System \F, except that the type variable
can be of some kind $\kappa$ other than just the star kind.
We need to consider one more rule \rulename{Conv}, which is new in System \Fw.
Let us elaborate on the three cases of
\rulename{TyAbs} and \rulename{TyApp}, and \rulename{Conv}.

\paragraph{Case (\rulename{TyAbs})}
We need to show that
$ \inference{\Delta;\Gamma |- t : \forall X.B & \xi;\rho\in[|\Delta;\Gamma|]}
        {[|t|]_{\xi;\rho} \in [|\forall X^\kappa.B|]_\xi} $

By induction, we know that
\[ \inference{\Delta,X^\kappa;\Gamma |- t:B & \xi';\rho'\in[|\Delta,X;\Gamma|]}
        {[|t|]_{\xi';\rho'} \in [|B|]_{\xi'}} ~
        (X\notin\FV(\Gamma))
\]
Since this holds for all $\xi';\rho' \in [|\Delta,X^\kappa;\Gamma|]$ where
$X\notin\FV(\Gamma)$, it also holds for particular subset such that
$\xi' = \xi[X\mapsto\mathcal{F}]$ and $\rho'=\rho$ for all $\mathcal{F}\in[|\kappa|]$.
That is,
\[ [|t|]_{\xi[X\mapsto\mathcal{F}];\rho} \in [|B|]_{\xi[X\mapsto\mathcal{F}]}
        \quad \text{for all}~\mathcal{F}\in[|\kappa|] \]
From $X\notin\FV(\Gamma)$, we know that
$[|t|]_{\xi[X\mapsto\mathcal{F}];\rho} = [|t|]_{\xi;\rho}$
because $\rho$ is independent of what $X$ maps to.
So, we know that
\[ [|t|]_{\xi;\rho} \in [|B|]_{\xi[X\mapsto\mathcal{F}]}
        \quad \text{for all}~\mathcal{F}\in[|\kappa|] \]
By set theoretic definition, this is exactly what we wanted to show:
\[ [|t|]_{\xi;\rho} \in
        \bigcap_{\mathcal{F}\in[|\kappa|]} [|B|]_{\xi[X\mapsto\mathcal{F}]}
        = [|\forall X^\kappa.B|]_\xi
\]

\paragraph{Case (\rulename{TyApp})}
We need to show that
$ \inference{\Delta;\Gamma |- t : B[G/X] & \xi;\rho\in[|\Delta;\Gamma|]}
        {[|t|]_{\xi;\rho} \in [|B[G/X]|]_\xi} $.

By induction, we know that
$ \inference
        { \Delta;\Gamma |- t : \forall X^\kappa.B
        & \xi';\rho'\in[|\Delta;\Gamma|] }
        {[|t|]_{\xi';\rho'} \in [|\forall X^\kappa.B|]_{\xi'}} $.

Since this holds for all $\xi';\rho' \in [|\Delta,\Gamma|]$,
it also holds for $\xi'=\xi$ and $\rho'=\rho$. Then, we are done:
$ [|t|]_{\xi;\rho} \in [|\forall X^\kappa.B|]_{\xi}
        = \bigcap_{\mathcal{G}\in[|\kappa|]} [|B|]_{\xi[X\mapsto\mathcal{G}]}
        \subseteq [|B|]_{\xi[X\mapsto[|G|]_\xi]} = [|B[G/X]|]_\xi $.

\paragraph{Case (\rulename{Conv})}
We need to show that
$ \inference{\Delta;\Gamma |- t : A' & \xi;\rho\in[|\Delta;\Gamma|]}
        {[|t|]_{\xi;\rho} \in [|A'|]_\xi} $

By induction we know that 
$ \inference{\Delta;\Gamma |- t : A & \xi;\rho\in[|\Delta;\Gamma|]}
        {[|t|]_{\xi;\rho} \in [|A|]_\xi} $

If we can show that $[|A|]_\xi = [|A'|]_\xi$, we are done.

To show that $[|A|]_\xi = [|A'|]_\xi$,
we use the soundness of type constructor equality lemma
(Lemma \ref{lem:fw:soundtyeq}).
\end{proof}

\begin{corollary}[strong normalization]
        $\inference{\Delta;\Gamma |- t : A}{t \in \SN}$
\end{corollary}

\begin{lemma}[soundness of type equality] \label{lem:fw:soundtyeq}
$ \inference{\Delta |- F = F' : \kappa & \xi\in[|\Delta|]}
        {[|F|]_\xi = [|F'|]_\xi} $
\end{lemma}
\begin{proof}
The only interesting case is the \rulename{EqTBeta} rule.
The \rulename{EqTVar} is trivial and all other rules are handled by induction.
Let us elaborate on the \rulename{EqTBeta} case.

\paragraph{Case (\rulename{EqTBeta})} We need to show that
\[ \inference
        { \Delta |- (\l X^\kappa.F)\;G = F[G/X] : \kappa' & \xi\in[|\Delta|] }
        { [| (\l X^\kappa.F)\;G |]_\xi = [| F[G/X] |]_\xi }
\]

By applying the soundness of kinding lemma (Lemma \ref{lem:fw:soundki})
to the premises, we know that
\[ \inference
        {\Delta,X^\kappa |- F : \kappa -> \kappa' & \xi'\in[|\Delta,X^\kappa|]}
        {[|F|]_{\xi'} \in [|\kappa'|]}
\quad\text{and}\quad
   \inference{\Delta |- G : \kappa & \xi\in[|\Delta|]}{[|G|]_\xi \in [|\kappa|]}
\]

Since it should hold for arbitrary $\xi'$, it should also hold for
a particular $\xi'$ such that $\xi'=\xi[X\mapsto\mathcal{G}]$
for any $\mathcal{G} \in [|\kappa|]$. Therefore, we can rewrite
the left-hand side of the conclusion, which what we wanted to show,
into the right-hand side as follows:
\begin{align*}
[| (\l X^\kappa.F)\;G |]_\xi
&=[|(\l X^\kappa.F)|]_\xi ([|G|]_\xi) \\
&=(\bbl(\mathcal{G}\in[|\kappa|]).[|F|]_{\xi[X\mapsto\mathcal{G}]})([|G|]_\xi)\\
&=[|F|]_{\xi[X\mapsto[|G|]_\xi]} \\
&=[|F[G/X]|]_\xi
\end{align*}
\end{proof}

System \Fw\ has a richer kind structure than System \F, which only has one and
the only the star kind. So, the interpretation of type constructors would only
be well-defined when the type constructors are well-kinded. For example,
the interpretation of a type constructor application $[|F~G|]_\xi$
would only make sense when $[|F|]_\xi\in[|\kappa -> \kappa'|]$
and $[|G|]_\xi\in[|\kappa|]$ for some $\kappa$ and $\kappa'$.
The soundness of kinding lemma below states the property that
well-kinded type constructors indeed have well-defined interpretation.
\begin{lemma}[soundness of kinding]  \label{lem:fw:soundki}
$ \inference{\Delta |- F : \kappa & \xi\in[|\Delta|]}{[|F|]_\xi\in[|\kappa|]} $
\end{lemma}
\begin{proof}
We prove by induction on the kinding judgment.
\paragraph{Case (\rulename{TVar})}
Straightforward by definition of $[|\Delta|]$.

$[|X|]_\xi=\xi(X) \in [|\kappa|]$
since $\xi(X)\in[|\kappa|]$ for any $\xi\in[|\Delta|]$
when $X^\kappa \in [|\Delta|]$.

\paragraph{Case (\rulename{TArr})} By induction, straightforward.

\paragraph{Case (\rulename{TAll})}
We need to show that
$ \inference{\Delta |- \forall X^\kappa.B:* & \xi\in[|\Delta|]}
        {[|\forall X^\kappa.B|]_\xi\in[|*|]} $.

By induction, we know that
$ \inference{\Delta,X^\kappa |- B:* & \xi'\in[|\Delta,X^\kappa|]}
        {[|B|]_{\xi'} \in [|*|]} $.

Since it should hold for any $\xi'$, it also holds for
$\xi'=\xi[X\mapsto\mathcal{G}]$ for any $\mathcal{G}\in[|\kappa|]$.

Therefore,
$  [|\forall X^\kappa.B|]_\xi
 = \bigcap_{\mathcal{G}\in[|\kappa|]}[|B|]_{\xi[X\mapsto\mathcal{G}]}\in[|*|] $.

\paragraph{Case (\rulename{TLam})}
We need to show that
$ \inference{\Delta |- \l X^\kappa.B:* & \xi\in[|\Delta|]}
        {[|\forall X.B|]_\xi\in[|*|]} $.

By induction, we know that 
$ \inference{\Delta,X^\kappa |- F:\kappa' & \xi'\in[|\Delta,X^\kappa|]}
        {[|F|]_{\xi'} \in [|\kappa'|]} $.

Since it should hold for any $\xi$, it also holds for
$\xi'=\xi[X\mapsto\mathcal{G}]$ for any $\mathcal{G}\in[|\kappa|]$.

Therefore,
$  [|\l X^\kappa.F|]_\xi
 = \bbl(\mathcal{G}\in[|\kappa|]).[|F|]_{\xi[X\mapsto\mathcal{G}]}
        \in [|\kappa -> \kappa'|]$.

\paragraph{Case (\rulename{TApp})} By induction, straightforward.
\end{proof}

   %% \section{System \Fw}                       \label{sec:fw}
\section{The Hindley-Milner type system} \label{sec:hm}

The Hindly-Milner type system (HM),
\aka\ Dammas-Hindly-Milner type system (DHM),
\cite{Hindley69,Milner78,DamMil82,Damas85} infers the most general type scheme
(\aka\ the principal type scheme) for a Curry-style term.
A type scheme is
%%% TODO explain what a type scheme is maybe I should have
%%% already expained this in the previous section of STLC systm F
%%% recall the reader on the 

\citet{Hindley69} discovered that there exists a unique principal type scheme
for an object in a combinatory logic. \citet{Milner78} rediscovered this
in the setting of a polymorphic lambda calculus, while he was devising
an algorithm, called the algorithm $W$, which infers a type scheme for
a Curry-style term. \citet{Damas85} developed detailed theories on
Milner's polymorphic lambda calculus (\aka\ let-polymorhpism) and
the type inferecne algorithm $W$.


\begin{figure}
\begin{singlespace}
\small
\begin{align*}
&\textbf{term}&
t,s&~::= ~ x          
    ~  | ~ \l x    . t 
    ~  | ~ t ~ s       
    ~  | ~ \<let> x=s \<in> t
\\
&\textbf{type}&
A,B&~::= ~ A -> B
    ~  | ~ \iota
    ~  | ~ X
\\
&\textbf{type scheme}&
\sigma&~::= ~ \forall X.\sigma
       ~  | ~ A
\\
&\textbf{typing context}&
\Gamma&~::= ~ \cdot 
       ~  | ~ \Gamma, x:\sigma \quad (x\notin \dom(\Gamma))
\end{align*}
\[ \textbf{Type scheme ordering} \quad \framebox{$\sigma \sqsubseteq \sigma'$}\]
\[ \inference{X_1',\dots,X_n'\notin\FV(\forall X_1\dots X_n.A)}
             {\forall X_1\dots X_n.A \;\sqsubseteq\;
	      \forall X_1'\dots X_m'.\,A[B_1/X_1]\cdots[B_n/X_n]} \]
$\!\!\!\!\!\!\!\!\!\!$
\begin{align*}
&\textbf{Delcarative typing rules}&\quad
&\textbf{Syntax-directed typing rules}
	\\
& \inference[\sc Var]{x:\sigma \in \Gamma}{\Gamma |- x:\sigma} &
& \inference[\sc Var$_s$]{x:\sigma \in \Gamma & \sigma \sqsubseteq A}
 	                 {\Gamma |- x:A} \\
& \inference[\sc Abs]{\Gamma,x:A |- t : B}{\Gamma |- \l x   .t : A -> B} &
& \inference[\sc Abs$_s$]{\Gamma,x:A |- t : B}{\Gamma |- \l x   .t : A -> B} \\
& \inference[\sc App]{\Gamma |- t : A -> B & \Gamma |- s : A}
		     {\Gamma |- t~s : B} &
& \inference[\sc App$_s$]{\Gamma |- t : A -> B & \Gamma |- s : A}
		         {\Gamma |- t~s : B} \\
& \inference[\sc Let]{\Gamma |- s : \sigma & \Gamma,x:\sigma |- t : B}
		     {\Gamma |- \<let> x=s \<in> t : B} &
& \inference[\sc Let$_s$]{\Gamma |- s : A & \Gamma,x:\Gen(\Gamma,A) |- t : B}
		         {\Gamma |- \<let> x=s \<in> t : B} \\
& \inference[\sc Inst]{\Gamma |- t : \sigma & \sigma \sqsubseteq \sigma'}
		      {\Gamma |- t : \sigma'} &
&\quad\qquad \begin{smallmatrix}\Gen(\Gamma,A)=\forall\vec{X}.A&
			 ~\text{where}~\vec{X}=\FV(A)\setminus\FV(\Gamma)
		 \end{smallmatrix}\\
& \inference[\sc Gen]{\Gamma |- s : \sigma & X \notin\FV(\Gamma)}
		     {\Gamma |- t : \forall X.\sigma}
\end{align*}

\end{singlespace}
\caption{The Hindily-Milner type system}
\label{fig:hm}
\end{figure}

\begin{figure}
TODO
\caption{The type inference algorithm $W$}
\label{fig:algW}
\end{figure}
   %% \section{The Hindley-Milner type system}   \label{sec:hm}

