\section{The Hindley--Milner type system} \label{sec:hm}
\index{Hindley--Milner}
\citet{Hindley69} demonstrated the existence of a unique principal type scheme
for every object in a combinatory logic. \citet{Milner78} rediscovered this fact
in the setting of a polymorphic lambda calculus. He was devising
an algorithm, called algorithm $W$, which infers a most general
type scheme (\aka\ principal type scheme) for a Curry-style lambda term.
\index{type scheme}
\citet{Damas85} (a student of Milner) published detailed theories about
Milner's polymorphic lambda calculus and the type inference algorithm $W$.
This type system for Milner's polymorphic lambda calculus \cite{Milner78,DamMil82,Damas85}
is also known as the Hindley--Milner type system (HM),
the Damas--Hindley--Milner type system (DHM), or let-polymorphic type system.
\index{polymorphic lambda calculus!Milner's}
\index{let-polymorphic}
\index{type inference}

The syntax of Milner's polymorphic lambda calculus and its typing rules
are illustrated in Figure \ref{fig:hm}. The type inference algorithm $W$
is illustrated in Figure \ref{fig:algW}. We discuss each of these
figures separately --- the syntax in
\S\ref{sec:hm:syntax}, the typing rules in \S\ref{sec:hm:dectyrule} and
\S\ref{sec:hm:syntyrule}, and the inference algorithm in \S\ref{sec:hm:W}.
We provide two equivalent sets of typing rules (we prove this in \S\ref{sec:hm:syntyrule}).
\index{typing rules!declarative}
The declarative typing rules (\S\ref{sec:hm:dectyrule}) are suited for
reasoning about the soundness of typing.
\index{typing rules!syntax-directed}
The syntax-directed typing rules (\S\ref{sec:hm:syntyrule}) are suited
for reasoning about the properties of the type inference algorithm $W$
(\S\ref{sec:hm:W}).

\begin{figure}
\begin{singlespace}\small
\small
\begin{align*}
&\textbf{Term}&
t,s&~::= ~ x          
    ~  | ~ \l x    . t 
    ~  | ~ t ~ s       
    ~  | ~ \<let> x=s \<in> t
\\
&\textbf{Type (or, monotype)}&
A,B&~::= ~ A -> B
    ~  | ~ \iota
    ~  | ~ X
\\
&\textbf{Type scheme (or, polytype)}&
\sigma&~::= ~ \forall X.\sigma
       ~  | ~ A
\\
&\textbf{Typing context}&
\Gamma&~::= ~ \cdot 
       ~  | ~ \Gamma, x:\sigma \quad (x\notin \dom(\Gamma))
\end{align*}
\[ \textbf{Type scheme ordering (or, generic instantiation)} \quad \framebox{$\sigma \sqsubseteq \sigma'$}\]
\[ \qquad \inference[\sc GenInst]{X_1',\dots,X_m'\notin\FV(\forall X_1\dots X_n.A)}
             {\forall X_1\dots X_n.A \;\sqsubseteq\;
              \forall X_1'\dots X_m'.\,A[B_1/X_1]\cdots[B_n/X_n]} \]
$\!\!\!\!\!\!\!\!\!\!$
\begin{align*}
&\textbf{Declarative typing rules}&\quad
&\textbf{Syntax-directed typing rules}
        \\
& \qquad\framebox{$\Gamma |- t : \sigma$}
&
&~\qquad\framebox{$\Gamma |-s t : A$}
        \\
& \inference[\sc Var]{x:\sigma \in \Gamma}{\Gamma |- x:\sigma} &
& \inference[\sc Var$_s$]{x:\sigma \in \Gamma & \sigma \sqsubseteq A}
                         {\Gamma |-s x:A} \\
& \inference[\sc Abs]{\Gamma,x:A |- t : B}{\Gamma |- \l x   .t : A -> B} &
& \inference[\sc Abs$_s$]{\Gamma,x:A |-s t:B}{\Gamma |-s \l x   .t : A -> B} \\
& \inference[\sc App]{\Gamma |- t : A -> B & \Gamma |- s : A}
                     {\Gamma |- t~s : B} &
& \inference[\sc App$_s$]{\Gamma |-s t : A -> B & \Gamma |-s s : A}
                         {\Gamma |-s t~s : B} \\
& \inference[\sc Let]{\Gamma |- s : \sigma & \Gamma,x:\sigma |- t : B}
                     {\Gamma |- \<let> x=s \<in> t : B} &
& \inference[\sc Let$_s$]
            {\Gamma |-s s : A & \Gamma,x:\overline{\Gamma}(A) |-s t : B}
            {\Gamma |-s \<let> x=s \<in> t : B} \\
& \inference[\sc Inst]{\Gamma |- t : \sigma & \sigma \sqsubseteq \sigma'}
                      {\Gamma |- t : \sigma'} &
&\quad\qquad \begin{smallmatrix}\overline{\Gamma}(A)=\forall\vec{X}.A&
                         ~\text{where}~\vec{X}=\FV(A)\setminus\FV(\Gamma)
                 \end{smallmatrix}
                 \\
& \inference[\sc Gen]{\Gamma |- t : \sigma}
                     {\Gamma |- t : \forall X.\sigma} ~ (X \notin\FV(\Gamma))
\end{align*}
\end{singlespace}
\caption{Milner's polymorphic lambda calculus.}
\label{fig:hm}
\end{figure}

\subsection{Syntax}\label{sec:hm:syntax}
The syntax of terms includes the usual Curry-style terms
($x$, $\l x.t$, and $t\;s$) and let-terms ($\<let> x=s \<in> t$).
A let term, $\<let> x=s \<in> t$, is semantically equivalent to
$(\l x.t)\,s$. That is, $\<let> x=s \<in> t$ is a syntactic sugar
for $(\l x.t)\,s$ when we think about reduction.\footnote{
	The reduction rules for
        the terms of HM are exactly the same as the reduction rules for
        Curry-style terms in the previous sections, once we desugar
        all the let terms.}
However, a let-term ($\<let> x=s \<in> t$) is assigned
a significantly different type than its semantic equivalent ($(\l x.t)\,s$).
The typing rules support the introduction of
a polymorphic type scheme for $x$ into the typing context when typing the let-term's
body $t$. We will discuss many
further details of typing let-terms (the \rulename{Let} rule) when
we explain the typing rules.
\index{type scheme}

\index{monotype}
\index{polytype}
The syntax of types (or, monotypes) includes all the types in the STLC
($A -> B$ and $\iota$) and type variables ($X$).
The syntax of type schemes (or, polytypes) are similar to
the polymorphic types of System \F, but universal quantification must appear
only at the top level. Syntactically, type schemes are either universal quantifications over
other type schemes ($\forall X.\sigma$) or (mono)types ($A$).
Typing contexts ($\Gamma$) keep track of each term variable and
its associated type scheme ($x:\sigma$).

The ordering between two type schemes $\sigma \sqsubseteq \sigma'$,
defined in Figure \ref{fig:hm}, means that $\sigma$ is more general
than or equivalent to $\sigma'$. The ordering relation $\sqsubseteq$
comes from \citet{DamMil82}, and is also known as generic instantiation ---
$\sigma'$ is called a generic instance of $\sigma$
when $\sigma \sqsubseteq \sigma'$. The shorthand notation
$\forall X_1\dots X_n.A$ stands for consecutive universal quantification
of $n$ variables. For instance, $\forall X_1\,X_2\,X_3.A$
is a shorthand for $\forall X_1.\forall X_2.\forall X_3.A$.

Two type schemes $\sigma$ and $\sigma'$ are equivalent
when $\sigma \sqsubseteq \sigma'$ and $\sigma' \sqsubseteq \sigma$.
This coincides with $\alpha$-equivalence (\eg, $\forall X.X -> X$ is
equivalent to $\forall X'. X' -> X'$). In fact, we can derive
$\alpha$-equivalence as a special case of the type scheme ordering rule \rulename{GenInst} (Figure \ref{fig:hm}),
where $n=m$ and $B_i=X_i'$ for each $i$ from $1$ to $n$.

\index{generic instance}
The usual \empty{instantiation} (\ie, substitution of quantified variables
with monotypes) is a special case of generic instantiation. For example,
consider the instantiations of $\forall X_1\,X_2.X_1 -> X_2$ below:
\[ (\forall X_1\,X_2.X_1 -> X_2) \sqsubseteq
   (\forall X_2.\iota -> X_2) \sqsubseteq (\iota -> \iota)
\] 
Here, we instantiate the quantified $X_1$ with $\iota$,
and then, instantiate the quantified $X_2$ with $\iota$.
In such cases of $\sigma\sqsubseteq\sigma'$, we can call $\sigma'$ an instance
(as well as a generic instance) of $\sigma$. For example, (1) $\iota\to\iota$ is
an instance of $\forall X_2.\iota\to X_2$ and an instance of $\forall X_1\,X_2.X->X_2$; and (2)
both $\iota\to\iota$ and $\forall X_2.\iota\to X_2$ are instances of
$\forall X_1\,X_2.X_1->X_2$.

The relation $\sqsubseteq$
is more than $\alpha$-equivalence and instantiation, since
the type scheme ordering rule allows quantifying newly introduced
variables in $\sigma'$, which do not appear free in $\sigma$.
For example, consider the two generic instances of $\forall X.X -> X$ below:
\begin{align*}
\forall X.X -> X \sqsubseteq\;\,& (X'-> X')-> (X'-> X') \\
\forall X.X -> X \sqsubseteq\;\,& \forall X'.(X'-> X')-> (X'-> X')
\end{align*}
The former, $(X'-> X')-> (X'-> X')$, is an instance of $\forall X.X -> X$
instantiating $X$ to $(X'-> X')$. However, the latter,
$\forall X'.(X'-> X')-> (X'-> X')$, is not an instance
but a generic instance of $\forall X.X-> X$ because
the newly introduced variable $X'$ is universally quantified.
\index{generic instance}

There is a difference between the (mono)type $(X'-> X')-> (X'-> X')$,
where $X'$ is free, and the type scheme $\forall X'.(X'-> X')-> (X'-> X')$,
where $X'$ is universally quantified.
A function of the monomorphic type $(X'-> X')-> (X'-> X')$
can only be applied to functions of the same type in one program, but
a function of the polymorphic type scheme $\forall X'.(X'-> X')-> (X'-> X')$
can be applied to functions of many different types in one program.
For example, consider a typing context $\Gamma$ with four term-variables such that\footnote{
        For an intuitive explanation, we assume \texttt{int} and \texttt{string}
        to be existing ground types although our formal definition of HM
        in Figure \ref{fig:hm} only has the void type $\iota$ as
        the ground type for simplicity.} \vspace*{-1em}\\ 
\begin{minipage}{.3\linewidth}
\begin{align*}
\!\!\!\!\!\!\!\!\!\!
\textit{square} : \;\quad~\texttt{int} -> \texttt{int} ~\quad\; \in \Gamma \\
\!\!\!\!\!\!\!\!\!\!
\textit{revstr} : \texttt{string} -> \texttt{string} \in \Gamma
\end{align*}
\end{minipage}
\begin{minipage}{.4\linewidth}
\begin{align*}
\textit{Id}_{->}^{\,\textit{mon}} : \quad (X'-> X')-> (X'-> X') \quad \in \Gamma \\
\textit{Id}_{->}^{\,\textit{poly}} : \forall X'.(X'-> X')-> (X'-> X') \in \Gamma
\end{align*}
\end{minipage} \vspace*{1em} \\
The four function names, with the types assigned as above,
are available in the context. Under this typing context $\Gamma$,
it is possible to apply $\textit{Id}_{->}^{\,\textit{mon}}$,
the monomorphic identity function over endofunctions,
to either \textit{square} or \textit{revstr} (as we do below),
as long as we do not try to apply $\textit{Id}_{->}^{\,\textit{mon}}$ to both
of them in the same program.\footnote{A program is just a term, but it sounds
like a more practical example.} For example, we note the different types
for each application, and consider the different type that
$\textit{Id}_{->}^{\,\textit{mon}}$ must have inside each term.
\[ \Gamma |-
        (\textit{Id}_{->}^{\,\textit{mon}}\;\textit{square}) :
        \texttt{int} -> \texttt{int}
\]
\[ \Gamma |-
        (\textit{Id}_{->}^{\,\textit{mon}}\;\textit{revstr}) :
        \texttt{string} -> \texttt{string}
\]
It is impossible to derive a type for a program that applies
$\textit{Id}_{->}^{\,\textit{mon}}$ to both \textit{square} and \textit{revstr} 
in one program, since there is no solution for the inconsistent equations
$X'=\mathtt{int}$ and $X'=\mathtt{string}$.
\begin{align*}
\Gamma |- \;
& \dots (\textit{Id}_{->}^{\,\textit{mon}}\;\textit{square}) \dots \\
& \dots (\textit{Id}_{->}^{\,\textit{mon}}\;\textit{revstr}) \dots
~:~ \text{this is a type error}
\end{align*}
On the other hand, we can apply $\textit{Id}_{->}^{\,\textit{poly}}$,
the polymorphic identity function over endofunctions, to both
\textit{square} and \textit{revstr} in the same program, since
the universally quantified type variable $X'$ can be instantiated
to many different types including \texttt{int} and \texttt{string}.
\[ \Gamma |-
        (\textit{Id}_{->}^{\,\textit{poly}}\;\textit{square}) :
        \texttt{int} -> \texttt{int}
\]
\[ \Gamma |-
        (\textit{Id}_{->}^{\,\textit{poly}}\;\textit{revstr}) :
        \texttt{string} -> \texttt{string}
\]
\begin{align*}
\Gamma |-\;
& \dots (\textit{Id}_{->}^{\,\textit{poly}}\;\textit{square}) \dots \\
& \dots (\textit{Id}_{->}^{\,\textit{poly}}\;\textit{revstr}) \dots
~:~ \text{this can be type correct}
\end{align*}

To discover why the first must be ill-typed, and the second can be well-typed
we must look at the details of the typing rules.

\subsection{Declarative typing rules}\label{sec:hm:dectyrule}
\index{typing rules!declarative}
The declarative typing rules deduce a type scheme for a given term under
a typing context ($\Gamma |- t : \sigma$). The type scheme ($\sigma$)
deduced for the given term ($t$) under the typing context ($\Gamma$) may not
be unique. For example,
\begin{align*}
        &\cdot |- \l x.x: \iota -> \iota \\
        &\cdot |- \l x.x: X -> X \\
        &\cdot |- \l x.x: (X -> X) -> (X -> X) \\
        &\cdot |- \l x.x: \forall X.X -> X \\
        &\cdot |- \l x.x: \forall X.(X -> X) -> (X -> X) \\
        &\cdot |- \l x.x: \forall X_1 X_2.(X_1 -> X_2) -> (X_1 -> X_2) \\
        &\quad \vdots
\end{align*}
This is expected since terms of HM are Curry style. Recall that
the uniqueness of typing does not hold for lambda calculi with
Curry-style terms.

The first three declarative rules
-- \rulename{Var}, \rulename{Abs}, and \rulename{App} --
in Figure \ref{fig:hm} are fairly standard.
The \rulename{Var} rule deduces the type scheme of
a variable according to the type scheme binding of the variable in
the typing context. Note that the type schemes deduced by the rules
\rulename{Abs} and \rulename{App} are restricted to the form of
(mono)types\footnote{Recall that (mono)types are subset of type schemes.}
since the domain and range of function ($\rightarrow$) types are restricted to (mono)types.

%%%%% TODO TODO TODO more about this later <- fix ... we discuss this in ...
The \rulename{Let} rule can introduce polymorphic type schemes into
the typing context (we discuss more about this shortly, in the next page).
The most interesting rules are the \rulename{Inst} rule and
the \rulename{Gen} rule. %% Let's consider these one at a time.

\begin{itemize}

\item {\bf The \rulename{Inst} rule} deduces a generic instance
($\sigma'$) of any type scheme ($\sigma$). The \rulename{Inst} rule
is essential when variables with polymorphic type schemes appear in
the rules \rulename{Abs} and \rulename{App}. For instance,
when $t$ is a variable with a polymorphic type scheme in $\Gamma$,
we need to instantiate the type scheme into a type since \rulename{Abs}
and \rulename{App} are restricted to deduce (mono)types. A typical usage
(where $\Gamma = x':\forall X'.X' -> X'$) of
the \rulename{Inst} rule is illustrated below:
\[
\inference[\sc Abs]
  {\inference[Inst]
        {\inference[Var]
                {x':\forall X'.X' -> X' \in \Gamma,x:X}
                {\Gamma,x:X|- x':\forall X'.X'-> X'}
        }
        {\Gamma,x:X |- x':X'' -> X''}}
  {\Gamma |- \l x.x' : X -> (X'' -> X'')}
\]

\item {\bf The \rulename{Gen} rule} deduces a generalization (\ie,
universal quantification) of a type scheme, as long as
the quantified variable does not appear free in the typing context.
The \rulename{Gen} is essential for the \rulename{Let} rule to be useful.
For instance, consider that $s$ is a function that may be polymorphic,
such as the identity function $\l x'.x'$. We want to bind this function
in a let term, $\<let> x=\l x'.x' \<in> t$, and use $x$ as a polymorphic
function in $t$ (\ie, extend the typing context with $x:\forall X.X-> X$).
However, the \rulename{Abs} rule can only deduce a function type without
any universal quantification, such as $\Gamma |- \l x'.x' : X -> X$.
Here, we can use the \rulename{Gen} rule to generalize $X -> X$ to
$\forall X.X -> X$, provided that $X$ does not appear free in
the typing context $\Gamma$, as below:
\[
\inference[\sc Let]
  { \inference[\sc Gen]
          {\inference[\sc Abs]{\vdots}{\Gamma|- \l x'.x' : X -> X}}
          {\Gamma|- \l x'.x':\forall X.X-> X}
  & \Gamma,x:\forall X.X-> X |- t : B}
  {\Gamma |- \<let> x = \l x'.x' \<in> t : B}
\]

\end{itemize}

\paragraph{}
The soundness of typing is obvious
once we observe that HM is a restriction of the Curry-style System \F\ (\ie,
if $\Gamma|- t:\sigma$ in HM, then $\Delta;\Gamma|- t:\sigma$ in System \F).
The terms in HM are exactly the same as the terms in the Curry-style System \F,
if we consider the let-term as a syntactic sugar. Both types (or, monotypes)
and type schemes (or, polytypes) in HM are restrictions of types in System \F.
The declarative typing rules (of Figure \ref{fig:hm}) are also a restriction of the typing rules
in System \F. The rules \rulename{Var}, \rulename{Abs}, \rulename{App},
and \rulename{Gen} in HM are virtually the same as their counterparts
in System \F.\footnote{The names of corresponding rules
        in HM and System \F\ are the same (\rulename{Var}, \rulename{Abs},
        \rulename{App}), except for the \rulename{Gen} rule.
        The \rulename{Gen} rule in HM corresponds to
        the \rulename{TyAbs} rule in System \F.}
Thus, we only need ensure that the \rulename{Let} rule and
the \rulename{Inst} rule in HM are admissible in System \F.

A single derivation step of \rulename{Let} in HM corresponds to
two derivation steps in System \F\
involving the \rulename{App} and \rulename{Abs} rules.
Let us start from the \rulename{Let} rule in HM, quoted verbatim from
Figure \ref{fig:hm}:
\[\inference[\sc Let]{\Gamma |- s : \sigma & \Gamma,x:\sigma |- t : B}
                     {\Gamma |- \<let> x=s \<in> t : B} 
\]
Recall that a let-term, $\<let> x=s \<in> t$ is semantically equivalent to
$(\l x.t)\;s$. We first desugar the let-term into an application of
an abstraction $(\l x.t)$ to the body of the local definition ($s$).
Then, we can simply apply the \rulename{App} rule and the \rulename{Abs} rule
in System~\F, as below:
\label{hm:LetAdmissibleFw}
\[
 \inference[\sc App]
        { \inference[\sc Abs]
                { \Delta |- \sigma
                & \Delta;\Gamma,x:\sigma |- t : B}
                {\Delta;\Gamma |- \l x.t : A -> B}
        & \Delta;\Gamma |- s : \sigma }
        {\Delta;\Gamma |- (\l x.t)\;s}
\]


A singe derivation step of \rulename{Inst} in HM corresponds to
multiple derivation steps in System \F\ involving
the rules \rulename{TyAbs} and \rulename{TyApp} rules.
Since the \rulename{Inst} rule refers to the generic instantiation relation
$\sqsubseteq$, the shape of $\sigma$ and $\sigma'$ in the \rulename{Inst} rule
must match the left- and right-hand sides of $\sqsubseteq$
in the generic instantiation rule, as below:
\begin{singlespace}
\begin{multline*}
  \inference[\sc Inst]{\Gamma |- t : \sigma & \sigma \sqsubseteq \sigma'}
                      {\Gamma |- t : \sigma'}
        \quad\text{where}~
        \begin{array}{rl}
                \sigma =& \forall X_1\dots X_n.A ~~~~~~~~~~~~~~~~~~~~~~ \\
               \sigma' =& \forall X_1'\dots X_m'.\,A[B_1/X_1]\cdots[B_n/X_n]
        \end{array}
	\\ \text{such that} ~ \phantom{\frac{\frac{A}{\frac{A}{A}}}{A}}
   \inference{X_1',\dots,X_m'\notin\FV(\forall X_1\dots X_n.A)}
             {\forall X_1\dots X_n.A \;\sqsubseteq\;
              \forall X_1'\dots X_m'.\,A[B_1/X_1]\cdots[B_n/X_n]}
\end{multline*}
\end{singlespace}
\index{generic instantiation}
The generic instantiation from $\sigma$ to $\sigma'$
in the \rulename{Inst} rule above can be understood as having two phases:
instantiation from $\sigma$ to $A[B_1/X_1]\cdots[B_n/X_n]$ and
generalization from $A[B_1/X_1]\cdots[B_n/X_n]$ to $\sigma'$.
The instantiation phase, from $\forall.X_1\dots X_n.A$ to $A[B_1/X_1]\cdots[B_n/X_n]$,
can be broken down to $n$ small steps of instantiation ---
each step instantiates one of the quantified variables ($X_1\dots X_n$).
The generalization phase, from $A[B_1/X_1]\cdots[B_n/X_n]$ to
$\forall X_1'\dots X_m'.A[B_1/X_1]\cdots[B_n/X_n]$, can be broken down into
$m$ small steps of generalization --- each step universally quantifies one
of the newly introduced variables ($X_1'\dots X_m$) from the instantiation phase.
The Curry-style System \F\ has rules, which correspond exactly to these 
small steps (see Figure \ref{fig:f} in \S\ref{sec:f}).
The \rulename{TyApp} rule captures the small steps in the instantiation phase.
The \rulename{TyAbs} rule captures the small steps in the generalization phase.
Therefore, we can translate the \rulename{Inst} rule in HM into
consecutive applications of the \rulename{TyApp} rule followed by
consecutive applications of the \rulename{TyAbs} rule in System \F, as below:
{\small
\begin{align*}
& \!\!\!\!\!\!\!\!\!\!\!
	\inference[\sc TyApp]
        {\Delta;\Gamma |- t : \forall X_1\dots X_n.A & \Delta |- \forall X_1\dots X_n.A}
        {\Delta;\Gamma |- t : \forall X_2\dots X_n.A[B_1/X_1]
        \qquad\qquad\quad~~} \\
& \!\!\!\!\!\!\!\!\!\!\!
	\inference[\sc TyApp]
        { \qquad\qquad \vdots \qquad\qquad\qquad\qquad\qquad \quad
        & \Delta |- \forall X_n.A[B_1/X_1]\cdots[B_{n-1}/X_{n-1}]
        }
        { \Delta;\Gamma |- t : A[B_1/X_1]\cdots[B_n/X_n] \qquad \qquad
         \qquad\qquad\qquad\qquad\qquad\qquad\qquad\quad
        } \\
& \!\!\!\!\!\!\!\!\!\!\!
	\inference[\sc TyAbs]
        {}
        {\Delta;\Gamma |- t : \forall X_m'.\,A[B_1/X_1]\cdots[B_n/X_n]} ~
        (X_m'\notin\FV(A[B_1/X_1]\cdots[B_n/X_n])) \\
& \!\!\!\!\!\!\!\!\!\!\!
	\qquad\qquad\qquad\qquad\qquad \vdots \\
& \!\!\!\!\!\!\!\!\!\!\!
	\inference[\sc TyAbs]
        {\Delta;\Gamma |- t : \forall X_2'\dots X_m'.\,A[B_1/X_1]\cdots[B_n/X_n]}
        {\Delta;\Gamma |- t : \forall X_1'\dots X_m'.\,A[B_1/X_1]\cdots[B_n/X_n]} ~
        (X_1'\notin\FV\left(\begin{smallmatrix}\forall X_2'\dots X_m'.\\A[B_1/X_1]\cdots[B_n/X_n]\end{smallmatrix}\right))
\end{align*}
} %end \small

\subsection{Syntax-directed typing rules}\label{sec:hm:syntyrule}
\index{typing rules!syntax-directed}
The syntax-directed typing rules \cite{Clement86} deduce a type,
rather than a type scheme, for a given term under a typing context
($\Gamma |- t : A$). These rules are syntax-directed, since 
for each syntactic category of terms, there is only one typing rule that can apply.

The syntax-directed typing rules are based on the observation
that the \rulename{Inst} and \rulename{Gen} in the declarative typing rules
are only necessary at the \rulename{Var} and \rulename{Let} rules, respectively.
That is, we only need to apply the \rulename{Inst} rule to the conclusion of
the \rulename{Var} rule, and, we only need to apply the \rulename{Gen} rule to
the first premise ($\Gamma |- s:\sigma$) of the \rulename{Let} rule.
The \rulename{Var$_s$} rule can be understood as a merging of
\rulename{Var} and \rulename{Inst} into one rule.
The \rulename{Abs$_s$} rule and the \rulename{App$_s$} rule remain the same as
their counterparts in the declarative typing rules.
The \rulename{Let$_s$} rule can be understood as a merging of
the \rulename{Let} and the \rulename{Gen} into one rule.

The notation $\overline{\Gamma}(A)$ appearing in the rule \rulename{Let$_s$} is
the generalization closure of the type $A$ with respect to $\Gamma$. 
That is, $\overline{\Gamma}(A)$
generalizes $A$ over all the free type variables occurring in $A$, except the free types
variables occurring in $\Gamma$. The free type variables of $\Gamma$ are defined as
$\FV(\Gamma) = \bigcup_{x:\sigma\in \Gamma} \FV(\sigma)$.

\paragraph{}
The syntax-directed typing rules are sound (Theorem \ref{thm:sdHMsound})
and complete (Theorem \ref{thm:sdHMcomplete}) with respect to
the declarative typing rules.

We will simply sketch the key ideas for the proof of the soundness of $|-s$
since the soundness is rather obvious. All we need to do is transform
any given derivation for $|-s$ into a derivation for $|-$, which is
straightforward.
\begin{theorem}[$|-s$ is sound with respect to $|-$]
$ \inference{\Gamma |-s t : A}{\Gamma |- t : A} $
\label{thm:sdHMsound}
\end{theorem}
\begin{proof}
Recall that the \rulename{Var$_s$} rule can be understood as a merging of
\rulename{Var} and \rulename{Inst}. Thus, we can transform any derivation
step using the \rulename{Var$_s$} rule into two steps of derivation: using
the \rulename{Var} rule and then applying the \rulename{Inst} rule
to the conclusion of the \rulename{Var} rule.

The \rulename{Abs$_s$} rule and the \rulename{App$_s$} rules are mapped
to the \rulename{Abs} rule and the \rulename{App} rule, respectively.

Recall that the \rulename{Let$_s$} rule can be understood as a merging of
\rulename{Let} and \rulename{Gen}. We can transform any derivation step
using the \rulename{Let$_s$} rule into a series of \rulename{Gen} rules
applied to the first premise of the \rulename{Let} rule, and then applying
the \rulename{Let} rule. Since the definition of the closure
$\overline{\Gamma}(A)$ appearing in the \rulename{Let$_s$} rule
generalizes only the free type variables of $A$, which do not
appear free in $\Gamma$, the condition $X\notin\FV(\Gamma)$ appearing
in the \rulename{Gen} rule holds.
\end{proof}

The completeness of $|-s$ is stated below.
Note that the completeness of $|-s$ must be stated
in terms of generalization closure and of
the type scheme ordering relation ($\overline{\Gamma}(A)\sqsubseteq\sigma$)
since the syntax-directed rules can only deduce types, not type schemes.
The syntax-directed rule $|-s$ is complete in the sense that for any given term
we can always deduce a type $A$ such that the closure of $A$ is more general
than the type scheme $\sigma$ deduced from the declarative typing rules.
\begin{theorem}[$|-s$ is complete with respect to $|-$]
$ \inference
        {\Gamma |- t : \sigma}
        {\exists A.\;\Gamma |-s t : A ~\land~
	 \overline{\Gamma}(A) \sqsubseteq \sigma} $
\label{thm:sdHMcomplete}
\end{theorem}
\begin{proof}
We prove this by induction on the derivation of $\Gamma |- t : \sigma$.
Let us consider the cases by the last rule applied
(\ie, root of the derivation).

When the last rule is \rulename{Var}, we know that $x:\sigma \in \Gamma$.
We choose $A$ in the \rulename{Var$_s$} rule to be an instance of $\sigma$,
instantiating each universally quantified variable with a fresh variable,
which neither appears free in $\sigma$ nor $\Gamma$. We further 
restrict $A$ to satisfy $\overline{\Gamma}(A)\sqsubseteq\sigma$.
For example, when $\sigma=\forall X_1.\forall X_2.X_1 -> X_2 -> X$,
we choose $A = X_1' -> X_2' -> X$ where $X_1',X_2'\notin\FV(\Gamma)$.
If $X\in\FV(\Gamma)$, then
$\overline{\Gamma}(A) = \forall X_1'.\forall X_2'.X_1' -> X_2' -> X$,
which is $\alpha$-equivalent to $\sigma$, therefore,
$\overline{\Gamma}(A)\sqsubseteq\sigma$.
Otherwise, if $X\notin\FV(\Gamma)$, then
$\overline{\Gamma}(A) = \forall X_1'.\forall X_2'.\forall X.X_1' -> X_2' -> X$;
so, $\overline{\Gamma}(A)\sqsubseteq\sigma$ still holds.

When the last rule is \rulename{Abs}, it is straightforward by induction.

When the last rule is \rulename{App}, it is straightforward by induction.

When the last rule is \rulename{Let} ($\<let> x = s \<in> t$), we know by induction that there exists
$A'$ and $B'$ such that
$\Gamma |-s s : A' \land \overline{\Gamma}(A') \sqsubseteq \sigma$ and
$\Gamma,x:\sigma |-s t : B' \land
        \overline{\Gamma,x:\sigma}(B') \sqsubseteq B$.
The case for \rulename{Let} would be complete if we could show that
$\Gamma,x:\overline{\Gamma}(A') |-s t : B' \land
        \overline{\Gamma,x:\overline{\Gamma}(A')}(B') \sqsubseteq B$.
Instead we can only show $\Gamma,x:\sigma |-s t : B' \land
        \overline{\Gamma,x:\sigma}(B') \sqsubseteq B$.
We use Lemma \ref{lem:generalizeGamma} to prove
$\Gamma,x:\overline{\Gamma}(A') |-s t : B'$ from $\Gamma,x:\sigma |-s t : B'$,
and we use Lemma \ref{lem:closureMoreGen} to prove
$\overline{\Gamma,x:\overline{\Gamma}(A')}(B') \sqsubseteq B$
from $\overline{\Gamma,x:\sigma}(B') \sqsubseteq B$ and the transitivity of
$\sqsubseteq$. These two lemmas are introduced directly following the proof of this theorem.

When the last rule is \rulename{Inst}, it is straightforward by induction
and transitivity of $\sqsubseteq$.

When the last rule is \rulename{Gen}, we know by induction that there exists
$A$ such that $\overline{\Gamma}(A) \sqsubseteq \sigma$. We also know
that $X \notin\FV(\overline{\Gamma}(A))$ by the definition of generalization closure.
This step follows from a proof by contradiction argument.
If it were the case that $X\in\overline{\Gamma}(A)$, then it should be the case that
$X\in\FV(\Gamma)$
by the definition of generalization closure. This contradicts the side condition of
the \rulename{Gen} rule: $X\notin\FV(\Gamma)$.
Recall that generic instantiation allows quantifying type variables
that do not appear free in the original type scheme. Thus,
$\overline{\Gamma}(A) \sqsubseteq \forall X.\sigma$ by definition
of $\sqsubseteq$.
\end{proof}

\begin{lemma}[generalizing typing context is safe] \label{lem:genGamma}
$ \inference{\Gamma |-s t : A & \Gamma' \sqsubseteq \Gamma}
            {\Gamma' |-s t : A} $ where
\begin{quote}
$\Gamma' \sqsubseteq \Gamma$ when for any $x:\sigma\in \Gamma$,
there exists $x:\sigma'\in\Gamma'$ such that $\sigma'\sqsubseteq\sigma$.
\end{quote}
\label{lem:generalizeGamma}
\end{lemma}
\begin{proof}
This is an intuitively obvious property since assuming
more general type schemes for variables only makes it possible
to deduce all the judgments of $|-s$ and more, but no less,
by transitivity of $\sqsubseteq$ over type schemes.
We will simply give a proof for the base case, the \rulename{Var} rule,
which illustrates this intuition.
Other cases are straightforward by induction on the derivation of $|-s$.

When $\Gamma |-s x : A$, we know that there exists
$x,\sigma'\in\Gamma'$ such that $\sigma'\sqsubseteq\sigma$.
By the \rulename{Var$_s$} rule, we can deduce any $A'$ for $x$
such that $\Gamma' |-s x : A'$ and $\sigma'\sqsubseteq A'$.
By transitivity of $\sqsubseteq$, $\sigma' \sqsubseteq \sigma \sqsubseteq A$.
Therefore, $\Gamma' |-s x : A$.
\end{proof}

\begin{lemma}[closure of a more general typing context is more general]
\begin{quote}
$ \inference{\Gamma' \sqsubseteq \Gamma}
            {\overline{\Gamma'}(A) \sqsubseteq \overline{\Gamma}(A)} $
for any $A$.
\end{quote}
\label{lem:closureMoreGen}
\end{lemma}
\begin{proof}
It is obvious once we observe that $\FV(\Gamma') \subseteq \FV(\Gamma)$.
To show this, it suffices to show that ($\subseteq$) relation holds
pointwise on the type schemes in the context. That is,
$\FV(\sigma')\sqsubseteq\FV(\sigma)$ when $x:\sigma' \in \Gamma'$
and $x:\sigma \in \Gamma$. Note that $\dom(\Gamma')=\dom(\Gamma)$
by definition of $\sqsubseteq$ over contexts.
From the assumption $\Gamma' \sqsubseteq \Gamma$, we know that
$\sigma'\sqsubseteq\sigma$ when $x:\sigma' \in \Gamma'$
and $x:\sigma \in \Gamma$. Thus, we only need to show that
$\sigma' \sqsubseteq \sigma$ implies $\FV(\sigma') \subseteq \FV(\sigma)$,
which is not difficult to observe from the definition of $\sqsubseteq$ over
type schemes (\rulename{GenInst} in Figure \ref{fig:hm}).
\end{proof}

\subsection{The type inference algorithm $W$}\label{sec:hm:W}
\index{type inference!algorithm W}
\citet{DamMil82} presented the type inference algorithm $W$
(Figure \ref{fig:algW}) and proved its soundness and completeness
with respect to the declarative typing rules. Here, we show the
soundness and completeness of the type inference algorithm, $W$ (Figure \ref{fig:algW}),
with respect to the syntax-directed typing rules.
Each rule of the type inference algorithm $W$
has a similar structure to the corresponding syntax-directed rule.
The type inference algorithm has additional details of explicitly
managing fresh type variable introduction and substitution.

The unification of $A_1$ and $A_2$ succeeds when there exists a substitution $S$
such that $S A_1 = S A_2$.  When the unification succeeds,
we write $\unify(A_1,A_2) ~> S$, where the resulting substitution $S$ is
a unifier of $A_1$ and $A_2$.  Furthermore, $S$ is a most general unifier
\cite{Herbrand30,Robinson65} whose domain is a subset of $\FV(A_1)\cup\FV(A_2)$.
That is, for any unifier $S'$ such that $S' A_1 = S' A_2$ and
$\dom(S') \subseteq \FV(A_1) \cup \FV(A_2)$,
there exists a substitution $R$ such that $S' = R\circ S$ and
$\dom(R)\subseteq\dom(S)$. The composition of two substitutions
is defined as $(S_2\circ S_1)A = S_2(S_1(A))$.

\begin{figure}
\begin{singlespace}
\[ \inference[\sc Var$_W$]
        {x:\forall X_1\dots X_n.A\in\Gamma \\
         X_1',\dots,X_n'~\text{fresh}}
        {W(\Gamma,x) ~> (\emptyset,A[X_1'/X_1]\cdots[X_n'/X_n])}
\]
\[ \inference[\sc Abs$_W$]
        {X~\text{fresh} \\
         W((\Gamma,x:X),t) ~> (S_1,A)}
        {W(\Gamma,\l x.t) ~> (S_1,S_1 X -> A)}
\]
%% double check the composition order
\[ \inference[\sc App$_W$]
        {W(\Gamma,t) ~> (S_1,A_1) \\
         W(S_1 \Gamma,s) ~> (S_2,A_2) \\
         X~\text{fresh} \\
         \unify(S_2 A_1,A_2 -> X) ~> S_3 }
        {W(\Gamma,t\;s) ~> (S_3\circ S_2\circ S_1,S_3 X)}
\]

\[ \inference[\sc Let$_W$]
        {W(\Gamma,s) ~> (S_1,A_1) \\
         W((S_1\Gamma,x:\overline{S_1\Gamma}(A_1)),t) ~> (S_2,A_2) }
        {W(\Gamma,\<let> x=s \<in> t) ~> (S_2\circ S_1,A_2)}
\]
\end{singlespace}
\caption{The type inference algorithm $W$.}
\label{fig:algW}
\end{figure}

\begin{proposition} \label{prop:substvdashs}
        $\inference{\Gamma |-s t : A}{S\Gamma |-s t : SA}$
\end{proposition}
\begin{proposition} \label{prop:substclosure}
        $ S(\overline{\Gamma}(A)) = \overline{S\Gamma}(SA) $
\end{proposition}

\begin{theorem}[Soundness of $W$] \label{thm:soundW}
$ \inference{ W(\Gamma,t) ~> (S,A) }{ S\Gamma |-s t : A } $
\end{theorem}
\begin{proof}
By induction on the syntax of the term $t$.
\begin{itemize}
\item[case]($x$)
        Obvious, by definition of $\sqsubseteq$.

\item[case]($\l x.t$)
        We need to show that
        $S_1\Gamma |-s \l x.t : S_1 X -> A$.

        By induction, we know that $S_1(\Gamma,x:X) |-s t : A$.

        We know that $S_1\Gamma,x:S_1X |-s t : A$,
        since $S_1(\Gamma,x:X) = S_1\Gamma,x:S_1 X$.

        By \rulename{Abs$_s$} rule, we have
        $S_1\Gamma |-s \l x.t : S_1 X -> A$.

\item[case]($t\;s$)
        We need to show that $(S_3\circ S_2 \circ S_1)\Gamma |-s t\,s : S_3 X$.

        By induction, we know that
        $S_1\Gamma |-s t : A_1$ and 
        $S_2(S_1\Gamma) |-s s : A_2$.
        By Proposition~\ref{prop:substvdashs},
        $S_3(S_2(S_1\Gamma)) |-s t : S_3(S_2(A_1))$ and 
        $S_3(S_2(S_1\Gamma)) |-s s : S_3 A_2$.

        Due to the property of unification, $S_2 A_1 =  A_2 -> X$.

        Thus, $S_3(S_2(S_1\Gamma)) |-s t : S_3(A_2 -> X)$ and 
        $S_3(S_2(S_1\Gamma)) |-s s : S_3 A_2$.

        That is, $(S_3\circ S_2 \circ S_1)\Gamma |-s t : S_3 A_2 -> S_3 X$ and 
        $(S_3\circ S_2 \circ S_1)\Gamma |-s s : S_3 A_2$.

        By \rulename{App$_s$} rule, we have 
        $(S_3\circ S_2 \circ S_1)\Gamma |-s t\,s : S_3 X$.

\item[case]($\<let> x=s \<in> t$)
        We need to show that $(S_2\circ S_1)\Gamma |-s \<let> x=s \<in> t:A_2$.

        By induction, we have $S_1\Gamma |-s s:A_1$ and
        $S_2(S_1\Gamma,x:\overline{S_1\Gamma}(A_1)) |-s t:A_2$.

        By Proposition~\ref{prop:substvdashs}, we have $S_2(S_1\Gamma) |-s s:S_2 A_1$.

        By Proposition~\ref{prop:substclosure}, we have
        $S_2(S_1\Gamma),x:\overline{S_2(S_1\Gamma)}(S_2 A_1) |-s t: A_2$.

        By \rulename{Let$_s$}, we have
        $S_2(S_1\Gamma) |-s \<let> x=s \<in> t:A_2$.
        Since $S_2(S_1\Gamma)=(S_2\circ S_1)\Gamma$ by definition,
        we have $(S_2\circ S_1)\Gamma |-s \<let> x=s \<in> t:A_2$.
\vspace*{-2em}
\end{itemize}
\end{proof}


\begin{proposition}[\rulename{Abs$_s$} inverse] \label{prop:Abssrev}
$ \inference{\Gamma |-s \l x   .t : A -> B}{\Gamma,x:A |-s t:B} $
\end{proposition}

\begin{proposition}[\rulename{App$_s$} inverse] \label{prop:Appsrev}
$
  \inference{\Gamma |-s t~s : B}{
        \exists A.\left(\Gamma |-s t : A -> B ~\land~ \Gamma |-s s : A\right) } $
\end{proposition}

\begin{proposition}[\rulename{Let$_s$} inverse] \label{prop:Letsrev}
$ \inference{\Gamma |-s \<let> x=s \<in> t : B}{
        \exists A.\left( \Gamma |-s s : A ~\land~
                        \Gamma,x:\overline{\Gamma}(A) |-s t : B \right)} $
\end{proposition}


\begin{theorem}[Completeness of $W$] \label{thm:completeW}
~\\ $~$ $~$ For any $\Gamma$ and $t$,
there exist $S'$, where $\dom(S')\subseteq\FV(\Gamma)$, and $A'$ such that
\[ \inference{ S'\Gamma |-s t : A' }{
        W(\Gamma,t) ~> (S,A_W) ~\land~
        \exists R . \left(
                S'\Gamma = R(S\Gamma) \,\land\,
                R(\overline{S\Gamma}(A_W))\sqsubseteq A' \right) }
\]
\end{theorem}
\begin{proof}
See Appendix \ref{app:completeWproof}.
\end{proof}

\begin{corollary}[Completeness of $W$ under closed context]
%% ~\\ For any $\Gamma$ and $t$, there exist $S'$ and $A'$ such that
\[
\inference{ \Gamma |-s t : A' & \FV(\Gamma)=\emptyset }{
        W(\Gamma,t) ~> (S,A) ~\land~
        \exists R.R(\overline{\Gamma}(A))\sqsubseteq A' }
\]
\end{corollary}
\begin{proof} By Theorem \ref{thm:completeW} and the fact that
        $S'\Gamma = \Gamma$ for any $S'$
        when $\FV(\Gamma) = \emptyset$.
\end{proof}

