\section{Metatheory of \Fixi} \label{sec:fixi:theory}
Recall that we extended \Fixw\ to \Fixi\ to support primitive recursion and
course-of-values recursion. In this section, we show strong normalization and
logical consistency of \Fixi. We also give a partial proof of
the syntactic conditions necessary for well-behaved course-of-values recursion.

\subsection{Strong normalization and logical consistency}
\label{ssec:fixi:theory:sn}
We can prove strong normalization of \Fixi\ by the following strategy.
\begin{itemize}
\item Define a notion of index erasure that
	projects \Fixi\ types to \Fixw\ types.
\item Show that every well-typed \Fixi-term is a well-typed \Fixw-term
	by index erasure.
\item \Fixi\ inherits strongly normalization from \Fixw\ 
	because \Fixw\ is strongly normalizing \cite{AbeMat04}.
\end{itemize}

\index{index erasure}
The definition of the index erasure operation for System \Fixi\ and
the proofs of the related theorems are virtually the same as their counterparts
in System \Fi\ (see \S\ref{sec:fi:theory}). So, we simply illustrate
the definition and the theorems, but omit their proofs.

We define a meta-operation of index erasure ($\circ$) that
projects $\Fixi$ types to $\Fixw$ types and another meta-operation ($\bullet$)
that selects only the index variable bindings ($i^A$) from
the type-level context.
\begin{definition}[index erasure]\label{def:Fixierase}
\[ \fbox{$\kappa^\circ$}
 ~~~~ ~~
 *^\circ =
 *
 ~~~~ ~~
 (p\kappa_1 -> \kappa_2)^\circ =
 p{\kappa_1}^\circ -> {\kappa_2}^\circ
 ~~~~ ~~
 (A -> \kappa)^\circ =
 \kappa^\circ
\]
\[ \fbox{$F^\circ$}
 ~~~~
 X^\circ =
 X
 ~~~~ ~~~~
 (A -> B)^\circ =
 A^\circ -> B^\circ
 ~~~~ ~~~~
 (\fix F)^\circ =
 \fix F^\circ
\]
\[ \qquad
 (\lambda X^{p\kappa}.F)^\circ =
 \lambda X^{p\kappa^\circ}.F^\circ
 ~~~~ ~~~~
 (\lambda i^A.F)^\circ =
 F^\circ
\]
\[ \qquad
 (F\;G)^\circ =
 F^\circ\;G^\circ
 ~~~~ ~~~~ ~~~~ ~~~~ ~~
 (F\,\{s\})^\circ =
 F^\circ
\]
\[ \qquad
 (\forall X^\kappa . B)^\circ =
 \forall X^{\kappa^\circ} . B^\circ
 ~~~~ ~~~~
 (\forall i^A . B)^\circ =
 B^\circ
\]
\[ \fbox{$\Delta^\circ$}
 ~~~~
 \cdot^\circ = \cdot
 ~~~~ ~~
 (\Delta,X^{p\kappa})^\circ = \Delta^\circ,X^{p\kappa^\circ}
 ~~~~ ~~
 (\Delta,i^A)^\circ = \Delta^\circ
\]
\[ \fbox{$\Gamma^\circ$}
 ~~~~
 \cdot^\circ = \cdot
 ~~~~ ~~~~
 (\Gamma,x:A)^\circ = \Gamma^\circ,x:A^\circ
\]
\end{definition}

\begin{definition}[index variable selection]
\[ \cdot^\bullet = \cdot \qquad
        (\Delta,X^{p\kappa})^\bullet = \Delta^\bullet \qquad
        (\Delta,i^A)^\bullet = \Delta^\bullet,i:A
\]
\end{definition}
These two definitions are exactly the same as the definitions of
$^\circ$ and $^\bullet$ in \Fi\ (defined in \S\ref{ssec:fi:sn}),
except for the new constructs of \Fixi: (1) polarities in kinds and (2)
the equi-recursive type operator $\fix$.

\index{strong normalization!System Fixi@System \Fixi}
Once $^\circ$ and $^\bullet$ are defined, the proof of strong normalization
of \Fixi, by index erasure, is virtually the same as the proof of
strong normalization of \Fi\ (\S\ref{ssec:fi:sn}). Here, we list only
the theorems since their proofs can be trivially reconstructed by consulting
the proofs of the corresponding theorems in \S\ref{ssec:fi:sn}.

\begin{theorem}[index erasure on well-sorted kinds]
\label{thm:Fixierasesorting}
        $\inference{|- \kappa : \square}{|- \kappa^\circ : \square}$
\end{theorem}

\begin{theorem}[index erasure on well-formed type-level contexts]
\label{thm:Fixierasetyctx}
\[ \inference{|- \Delta}{|- \Delta^\circ} \]
\end{theorem}

\begin{theorem}[index erasure on kind equality]\label{thm:Fixierasekindeq}
$ \inference{|- \kappa=\kappa':\square}
        {|- \kappa^\circ=\kappa'^\circ:\square}
$
\end{theorem}

\begin{theorem}[index erasure on well-kinded type constructors]
\label{thm:Fixierasekinding}
\[ \inference{|- \Delta & \Delta |- F : \kappa}
                {\Delta^\circ |- F^\circ : \kappa^\circ}
\]
\end{theorem}
\begin{theorem}[index erasure on type constructor equality]
\[ \inference{\Delta |- F=F':\kappa}
                {\Delta^\circ |- F^\circ=F'^\circ:\kappa^\circ}
\]
\label{thm:Fixierasetyconeq}
\end{theorem}

\begin{theorem}[index erasure on well-formed term-level contexts]
\label{thm:Fixierasetmctx}
\[ \inference{\Delta |- \Gamma}{\Delta^\circ |- \Gamma^\circ} \]
\end{theorem}

\begin{theorem}[index erasure on index-free well-typed terms]
\label{thm:Fixierasetypingifree}
\[ \inference{ \Delta |- \Gamma & \Delta;\Gamma |- t : A}
                {\Delta^\circ;\Gamma^\circ |- t : A^\circ}
                {\enspace(\dom(\Delta)\cap\FV(t) = \emptyset)}
\]
\end{theorem}

\begin{theorem}[index erasure on well-formed term-level contexts
                prepended by index variable selection]
\label{thm:Fixierasetmctxivs}
\[ \inference{\Delta |- \Gamma}{\Delta^\circ |- (\Delta^\bullet,\Gamma)^\circ}
\]
\end{theorem}


\begin{theorem}[index erasure on well-typed terms]
\label{thm:Fixierasetypingall}
\[ \inference{\Delta |- \Gamma & \Delta;\Gamma |- t : A}
                {\Delta^\circ;(\Delta^\bullet,\Gamma)^\circ |- t : A^\circ}
\]
\end{theorem}
As stated in the introduction to this section, strong normalization is a direct
consequence of the erasure theorems above.

\paragraph{Logical consistency:}
We show the logical consistency of \Fixi\ by showing that the void type
$\forall X^{*}.X$ is uninhabited. By index erasure, we know that
that the set of terms that inhabit a \Fixi-type is a subset of
the set of terms that inhabit the corresponding (erased) \Fixw-type.
Thus, if $\forall X^{*}.X$ is uninhabited in \Fixw\ then 
$\forall X^{*}.X$ is uninhabited in \Fixi. \citet{AbeMat04} gives
a saturated set interpretation for the type constructors in \Fixw\ 
that is similar to the interpretation given in \S\ref{sec:fw:srsn} for
\Fw\ type constructors. The void type $\forall X^{*}.X$ is obviously
uninhabited (by any closed term) according to this interpretation. That is, no 
strongly-normalizing, closed term inhabits $\forall X^{*}.X$.
\begin{align*}
        [| \forall X^{*}.X |]_\xi &\in \SAT_{*} -> \SAT_{*} \\
        [| \forall X^{*}.X |]_\xi &
        = \bigcap_{\mathcal{A}\in[| * |]} [| X |]_{\xi[X\to\mathcal{A}]}
        = \bigcap_{\mathcal{A}\in[| * |]} \mathcal{A} = \bot
\end{align*}
The minimal saturated set $\bot$ of $\SAT$, which is saturated from
the empty set, does not have any closed terms. See \S\ref{sec:stlc:srsn}
for the definition of saturated sets.

\subsection{Syntactic conditions for well-behaved course-of-values recursion}
\label{ssec:fixi:theory:cv}
In \S\ref{sec:fixi:cv}, we embedded Mendler-style course-of-values
recursion (\McvPr) in \Fixi\ for the base structures ($F$)
that have maps ($\textit{fmap}_F$), also known as
\emph{monotonicity witnesses} \cite{Mat98,Mat99}. The theoretical development of
the termination of \McvPr\ by assuming the existence of maps is elegant, since
it does not require ad-hoc syntactic restrictions on the formation of types.
However, in a language implementation, it is not desirable to require users to
manually witness $\textit{fmap}_F$ every time they need to convince
the type system that \McvPr\ is well-defined over $F$.

It would be very convenient if we could categorize type constructors of
\Fixi\ that have maps by analyzing their polarized kinds. As a consequence,
for any type constructor $F$ whose kind meets certain criteria, users can
immediately assume the existence of $\textit{fmap}_F$ and that \McvPr\ always
terminates for $F$.

For instance, we conjecture that any $F: +* -> *$ should have a map, as in
Conjecture \ref{conj:fixi:fmap} below. In this thesis, we only show that this
conjecture holds for simple cases (Proposition \ref{prop:fixi:fmapBaseCase})
while still encompassing a broad range of types.
The complete proof is left for future work.
\citet{Mat99} showed that positive inductive types, in the context of System \F,
are monotone (\ie, map exists), but it has not yet been studied
whether we can rely on polarized kinds to derive monotonicity
in the context of System \Fw.

\begin{conjecture}\label{conj:fixi:fmap}
For any $F : +* -> *$, there exists\\ \vspace*{-1ex}
$\phantom{A}\qquad
        \textit{fmap}_F : \forall X^{*}.\forall Y^{*}.(X -> Y) -> F\;X -> F\;Y$
\quad such that
\begin{align*}
\textit{fmap}_F~\textit{id} &~=~ \textit{id} \\
\textit{fmap}_F~\textit{f} \;\circ\; \textit{fmap}_F~\textit{g}
&~=~ \textit{fmap}_F~(f\circ g)
\end{align*}
\end{conjecture}
\noindent
Assuming that the conjecture above is true,
we can show that $\McvPr$ is well-defined for any $F: +* -> *$ as follows.
\begin{conjecture} For any $F : +* -> *$, there exists
$\unIn_F : \mu^{+}{*} F -> F(\mu^{+}{*} F)$ such that
$\unIn_F (\In_F\;t) -->+ t$.
\end{conjecture}
\begin{proof}
Because we know that $\textit{fmap}_F$ exists
by Conjecture~\ref{conj:fixi:fmap}, we can define
\[ \unIn_F = \McvPr_{*}\;
            (\l\_.\l\textit{cast}.\l\_.\l x.\textit{fmap}_F\;\textit{cast}\;x)
\]
Because we know that $\textit{fmap}_F\;\textit{id}\;x -->+ x$,
we can show that the unroller $\unIn_F$ has the desired property
$\quad \unIn_F (\In_F\;t) -->+ \textit{fmap}_F\;\textit{id}\;t -->+ t $.
\end{proof}
We believe that the above conjectures are true, but we prove only
a small fragment (outlined below), which is a special case
of Conjecture \ref{conj:fixi:fmap}.
\begin{proposition}\label{prop:fixi:fmapBaseCase}
There exists
$\textit{fmap}_F : \forall X^{*}.\forall Y^{*}.(X -> Y) -> F\;X -> F\;Y$
for any $F : +* -> *$ such that
\begin{itemize}
        \item $F$ is non-recursive, that is, does not have $\fix$,
        \item $F$ is a value, that is, $F$ is
                in normal form (\ie, has no redex) and
                closed (\ie, $F$ has no free variables),
        \item All the bound variables within $F$ are
                introduced by universal quantification
                and those variables are of kind $*$.
\end{itemize}
\end{proposition}
\begin{proof}
        We can derive $\textit{fmap}_F$ from the structure of $F$.
        Since $F$ has kind $+* -> *$, it must be a lambda abstraction
        ($\l Z^{+*}.B$), some kind of application 
        (a normal application, $F'\,G$, or an index application $F'\{s\}$),
        or a variable of kind $+* -> *$.
        No other way of forming $F$ can have kind $+* -> *$.
        
        We also assume that $F$ is in normal form, so if $F$ were an application
        then $F'$ must be a variable with an arrow kind. However, we have
	assumed that all variables have kind $*$. Hence, $F$ cannot be
	a variable since we have assumed that $F$ is a value with
	no free variables, thus $F$ must have the form $\l Z^{+*}.B$.
	We proceed by analyzing the structure of $B$.
 
\begin{itemize}
\item[case]($Z\notin \FV(B)$, \ie, $F$ is a constant function.)
        $ \textit{fmap}_{(\l Z^{+*}. B)} = \l \_ . \l x. x $

        Since $F\;X = F\;Y = B$, we simply return the identity function on $B$.

\item[case]($F \triangleq \l Z^{+*}. Z$. \ie, $F$ is the identity.)
        $ \textit{fmap}_{(\l Z^{+*}. Z)} = \l z . z $

        Since $F\;X = X$ and $F\;X = Y$,
        we return the function $z:X -> Y$ itself.

\item[case]($F \triangleq \l Z^{+*}. \forall X_1^{*} .B_1$, $B$ is a universal quantification.)
        \vspace*{-2ex}
        \[ \textit{fmap}_{(\l Z^{+*}. \forall X_1^{*} .B_1)}
        = \textit{fmap}_{(\l Z^{+*}.B_1)} \]\vskip-2ex
        Here, we need to find an \textit{fmap}
        that works for any valuation of $X_1$.
        That is, we must find an \textit{fmap} that works for
	$\l Z^{+*}. B_1[V/X_1]$ for an arbitrary value $V$.
	Since values are closed, $v$ cannot contain free variables
	including $Z$. Since $v$ is completely independent of $Z$,
	the value $V$ cannot make any difference to the derived \textit{fmap}.
	Hence, we simply ignore $X_1$.

\item[case]($F \triangleq \l Z^{+*}. A -> B_1$, \ie, $B$ is a function.)

        When $Z\notin\FV(A)$,
        $\textit{fmap}_{(\l Z^{+*}.A -> B_1)}
        = \l z.\l y. \l x. \textit{fmap}_{(\l Z^{+*}.B_1)} \; z \; (y \; x)$

        When $A \triangleq \forall X_1^{*}.A_1$,
        $\textit{fmap}_{(\l Z^{+*}.(\forall X_1^{*}.A_1) -> B_1)}
        = \textit{fmap}_{(\l Z^{+*}.A_1 -> B_1)}$
        
        When $B_1 \triangleq \forall X_1^{*}.B_2$,
        $\textit{fmap}_{(\l Z^{+*}.A -> \forall X_1^{*}.B_2)}
        = \textit{fmap}_{(\l Z^{+*}.A -> B_2)}$        

        \begin{singlespace}
        When $A \triangleq A_1 -> \cdots -> A_n -> \forall X_2^{*}.B_2'$,
        \vspace{-1.5ex}
        \[\textit{fmap}_{(\l Z^{+*}.(A_1 -> \cdots -> A_n -> \forall X_2^{*}.B_2') -> B_1)}
        = \textit{fmap}_{(\l Z^{+*}.(A_1 -> \cdots -> A_n -> B_2') -> B_1)} \]

        When $A \triangleq A_1 -> \cdots -> A_n -> B_2$,
        where $B_2$ is not an arrow type
        \vspace{-1.5ex}
        \begin{align*}
          & \textit{fmap}_{(\l Z^{+*}.(A_1 -> \cdots -> A_n -> B_2) -> B_1)} \\
        =~& \l z.\l y. \l x.\;
        \textit{fmap}_{(\l Z^{+*}.B_1)} \; z \;
                (y \; (\l x_1.\ldots\l x_n. ~
                   x  &\!\!\!\!(\textit{fmap}_{(\l Z^{+*}.A_1)} ~ z ~ x_1)& \\
                     &&\!\!\!\!\vdots \qquad\qquad& \\
                     &&\!\!\!\!(\textit{fmap}_{(\l Z^{+*}.A_n)} ~ z ~ x_n)&
                \,) \,)
        \end{align*}
        \end{singlespace}\vspace*{-4ex}
\end{itemize}
\end{proof}

To illustrate that the \textit{fmap}s derived in the proof above are indeed
type-correct, we provide some examples in Haskell accepted by GHC in
Figure~\ref{fig:deriveFunctor}. In fact, all the \texttt{\small Functor}
instances in Figure~\ref{fig:deriveFunctor} are automatically derivable
using the \texttt{\small DeriveFunctor} extension in GHC version 7.4.
However, GHC does not derive functor instances when there are
type constructor variables other than $*$, since the kind system in GHC
does not keep track of polarity as in \Fixw\ or \Fixi.
\begin{figure}
\begin{singlespace}
\begin{lstlisting}[basicstyle={\ttfamily\small},language=Haskell,mathescape]
{-# LANGUAGE RankNTypes #-}

data F1 x = C1 (Int -> Bool)

instance Functor F1 where
  fmap f (C1 z) = C1 z

data F2 x = C2 x

instance Functor F2 where
  fmap f (C2 z) = C2 (f z)

data F3 x = C3 (([x] -> Bool) -> Maybe x)

instance Functor F3 where
  fmap z (C3 y) = C3 (\x -> fmap z (y (\x1-> x (fmap z x1))))

data F4 x = C4 ((forall y . [x] -> y) -> Maybe x)

instance Functor F4 where
  fmap z (C4 y) = C4 (\x -> fmap z (y (\x1-> x (fmap z x1))))

data F5 x = C5 (forall y . ([x] -> y) -> Maybe x)

instance Functor F5 where
  fmap z (C5 y) = C5 (\x -> fmap z (y (\x1-> x (fmap z x1))))

data F6 x = C6 (([x] -> ([x] -> Bool)) -> Maybe x)

instance Functor F6 where
  fmap z (C6 y) =
    C6 (\x -> fmap z (y (\x1 x2 -> x (fmap z x1) (fmap z x2))))
\end{lstlisting}
\end{singlespace}
\caption{Haskell code example to illustrate well-typedness of \textit{fmap}s
        derived in the proof of Proposition \ref{prop:fixi:fmapBaseCase}.}
\label{fig:deriveFunctor}
\end{figure}

Recall that we are proving a simplified version of the desired conjecture with
several simplifying assumptions: $F$ is a non-recursive closed value and
bound variables introduced in $F$ have kind $*$. Let us now generalize
the latter restriction by allowing type constructor variables of kind $p* -> *$
as well as type variables of kind $*$. There are three possibilities for
$X:p* -> *$. Note that the variable $X$ would be used as the function part of
an application, like $X\,G$, within type $B$.
\begin{itemize}
\item[case]($X : +* -> *$)
        By induction,\footnote{
                We need to make sure that this is a well-founded induction.
                It may be a coinductive proof.}
        there exists a map for any valuation of $X$.
        So, we denote that map as $\textit{fmap}_{X}$.
        The map for $\l Z^{+*}.X\,G$ is
        \[ \textit{fmap}_{(\l Z^{+*}.X\,G)} =
                \textit{fmap}_{X} \circ\, \textit{fmap}_{(\l Z^{+*}.G)} \]
        Note that $\textit{fmap}_{X}$ is not fixed until it is instantiated,
	however, we definitely know that it exists for any valuation.

\item[case]($X : -* -> *$)
        According to the ($@$) rule applied to $X\;G$, the argument $G$ should be well-kinded under $-\Delta$.
        Note that $Z^{-*}\in-\Delta$ since $Z^{+*}\in \Delta$.
        So, $Z$ cannot appear in positive positions in $G$.
	It can only appear in negative positions (\eg, $G \triangleq Z -> G'$).
        Any valuation of $X$ will have the form of $\l X_1^{-*}.B_1$
        where $X_1$ appears in negative positions. As a consequence,
        $(\l X_1^{-*}.B_1)G = B_1[G/X_1]$ by a single step reduction.
        Note that $B_1[G/X_1]$ must be in normal form. Since $G$ has kind $*$
        (it cannot be a lambda), substitution of $X_1$ with $G$ cannot
        introduce a new redex. Since $G$ is substituted only into
        negative positions, any $Z$ occurring in a negative position in $G$
        become a positive position in the substituted type.
        Thus, $\l Z^{+*}.B_1[G/X_1] : +* -> *$.
        Hence, by induction,\footnote{
                We need to make sure that this is a well-founded induction.
                It may be a coinductive proof.}
        there exists a map for any valuation of $X$
        since we can derive $\textit{fmap}_{(\l Z^{+*}.B_1[G/X_1])}$.

\item[case]($X : 0* -> *$)
        Note that $G$ cannot have $Z$ in it because $Z$ has $+$ polarity
        in the context. Recall that, $0\Delta$ ignores the variables
        with either $+$ or $-$ polarity. According to the ($@$) rule
        in Figure \ref{fig:Fixi2}, $G$ should be well-kinded under $0\Delta$.
        Since $Z\notin\FV(X\,G)$, we simply return the identity function
        as the map for $\l Z^{+*}.X\,G$.
\end{itemize}
In addition, having $X : A -> *$ does not make a difference since $X\{s\}$
cannot have $Z$ either. Once we know that we can derive maps in the presence
of type constructor variables with single argument, it is easy to generalize
this to arbitrary rank-1 kinded type constructor variables
(\eg, $+* -> A_1 -> -* -> A_2 -> 0* -> *$).

To complete the proof of Conjecture \ref{conj:fixi:fmap}, we need to
consider the equi-recursive type operator ($\fix$) and type constructor
variables of kind higher than rank 1. Considering $\fix$ makes the proof harder
since it becomes less obvious what ``normal form'' of a type constructor means.
Recall that $(\fix\,F_1)$ expands to $F_1(\fix\,F_1)$.  When $F_1$ is
a lambda abstraction, the expansion of $\fix$ introduces a new redex
at the type level. We hope to complete this proof in the future.

Once we have completed the proof of Conjecture \ref{conj:fixi:fmap},
the next step is prove a similar conjecture for higher kinds
(recall \texttt{fmap1} in \S\ref{sec:fixi:cv:unInExamples}).
For instance, we conjecture that maps exist for type constructors
of kind $+(+* -> *) -> (+* -> *)$.

%% \KYA{TODO introduce a general form of a conjecture??}

\begin{comment}
TODO To generalize this even further allowing type constructor variables
of any kind, we prove by induction on the rank of the kind of
the type constructor variables. That is, assuming that it holds for any
value $F$ with up to rank-$n$ kinded type constructor variables, we should
show that it holds when we also allow rank-$(n+1)$ kinded type constructor
variables. What we have proved so far is the base case $n=0$ and the case $n=1$.

For simplicity let us consider type constructor variables
that expect only one argument to become a type. That is,
$X : p\kappa -> *$ and $\kappa$ itself is in the form of $\kappa' -> *$.
\begin{itemize}
        \item[case]($X : 0\kappa -> * $)
        Since $G$ in $X\,G$ cannot have $Z$,
        we simply return the identity function as the map for $\l Z^{+*}.X\,G$.

        \item[case]($X : +\kappa -> * $)
        Valuation of $X$ must have the form of $\l X_1^{+\kappa}. B_1$.
        The argument $G$ in $X\,G$ must either be lambda or an application.
        When $G$ is an application of the form $X_2 G_2$, after reducing
        $(\l X_1^{+\kappa}. B_1)(X_2 G_2)$ to $B_1[X_2G_2/X_1]$, there are
        no new redex introduced by this reduction. Thus, this is just an
        inductive case.
        When $G$ is a lambda abstraction of the form $\l X_2^{p\kappa_2}.B_2$ it will make new redex ...
When G is a lambda it will make new redex ...
$X_1 G_1$ will be $G\,G_1$ since $G_1$.
We know that $G\,G_1$ will reduce to a normal form.

        \item[case]($X : -\kappa -> * $)
\[ \l X_1^{+\kappa}. B_1 : +\kappa -> * \]
G must either be lambda or an application.
When G is an application we already know how to handle this no problem --
reduction won't make any new redex either so it is still a value.
When G is a lambda it will make new redex ...
$X_1 G_1$ will be $G\,G_1$ since $G_1$.
We know that $G\,G_1$ will reduce to a normal form.
\end{itemize}

\begin{proposition}\label{prop:fixi:fmapFree}
If $\textit{fmap}_F:\forall X^{*}.\forall Y^{*}.(X -> Y) -> F\;X -> F\;Y$
exists, then
\begin{align*}
\textit{fmap}_F~\textit{id} &~=~ \textit{id} \\
\textit{fmap}_F~\textit{f} \;\circ\; \textit{fmap}_F~\textit{g}
&~=~ \textit{fmap}_F~(f\circ g)
\end{align*}
\end{proposition}\noindent
This is a well-known parametricity theorem on maps any instance of the type
$\forall X^{*}.\forall Y^{*}.(X -> Y) -> F\;X -> F\;Y$ satisfies
the two equations above. However, for the purpose of defining $\McvPr_{*}$,
we only need to know that there exists one such $\textit{fmap}_F$. That is,
\begin{proposition}\label{prop:fixi:fmapHom}
For any $F : +* -> *$, there exists

$\textit{fmap}_F:\forall X^{*}.\forall Y^{*}.(X -> Y) -> F\;X -> F\;Y$
such that
\begin{align*}
\textit{fmap}_F~\textit{id} &~=~ \textit{id} \\
\textit{fmap}_F~\textit{f} \;\circ\; \textit{fmap}_F~\textit{g}
&~=~ \textit{fmap}_F~(f\circ g)
\end{align*}
\end{proposition}
\begin{proof}
        You can check that each case %% TODO
        in the proof of Theorem \ref{thm:fixi:fmap} (TODO maybe refer to lemmas)
        satisfies the two equations above.
\end{proof}

\begin{proposition} For any $F : +* -> *$, there exists
$\unIn_F : \mu^{+}{*} F -> F(\mu^{+}{*} F)$ such that
$\unIn_F (\In_F\;t) -->+ t$.
\end{proposition}
\begin{proof}
Since we know that $\textit{fmap}_F$ exists by Theorem~\ref{thm:fixi:fmap},
we can define
\[ \unIn_F = \McvPr_{*}\;
            (\l\_.\l\textit{cast}.\l\_.\l x.\textit{fmap}_F\;\textit{cast}\;x)
\]

From Proposition~\ref{prop:fixi:fmapFree}, we know that
$\textit{fmap}_F\;\textit{id}\;x -->+ x$.
Thus,
\[ \unIn_F (\In_F\;t) -->+ \textit{fmap}_F\;\textit{id}\;t -->+ t \]
\end{proof}

\begin{align*}
A \rrarrow_{*} B &~\triangleq~ A -> B \\
F \rrarrow^{p\kappa -> \kappa'} G &~\triangleq~
        \forall X^\kappa.\forall Y^\kappa.
                (X \rrarrow_\kappa Y) -> F X \rrarrow_\kappa F Y \\
F \rrarrow^{A -> \kappa} G &~\triangleq~
        \forall i^A.\forall f^{A->A}. F\{i\} \rrarrow_\kappa F\{f\;i\}
\end{align*}

\[
\textsf{mon}_\kappa
  = \l X^{0\kappa}.X \rrarrow^\kappa X
\]

$\textsf{mon}_{+* -> *} F$ is the type of $\textit{fmap}_F$
where $F : +* -> *$.

$\textsf{mon}_{+(p* -> *)->(p* -> *)} F$ is the type of $\textit{fmap1}_F$
where $F : +(p* -> *)->(p* -> *)$.

if $\textsf{mon}_{+(p* -> *)->(p* -> *)}$ is inhabited
then $\unIn_F$ for any $F : +(p* -> *)->(p* -> *)$?

what about general case? if $\textsf{mon}_\kappa$ is inhabited
then $\unIn_F$ for any $F : \kappa$?
\end{comment}
