\section{Related Work}\label{sec:relwork}
TODO

\subsection{Mendler-style primitive recursion}\label{sec:relwork:mpr}
TODO

TODO
we should introduce Mendler-style primitive recursion (\mpr{}) here
because we are going to discuss \mprsi{}
TODO

\subsection{Type-based termination and sized types}\label{sec:relwork:sized}
\emph{Type-based termination} (coined by Barthe and others \cite{BartheFGPU04})
stands for approaches that integrate termination into type checking,
as opposed to syntactic approaches (\ie, conventional approach) that
reason about termination over untyped term structures.
The Mendler-style approach is, of course, type-based. In fact, the idea of
type-based termination was inspired by Mendler \cite{Mendler87,Mendler91}.
In the Mendler style, we know that well-typed functions defined using
Mendler-style recursion schemes always terminate.  This guarantee flows
from the design of the recursion scheme, where the use of higher-rank 
polymorphic types in the abstract operations enforce the invariants
necessary for termination.

Abel \cite{abel06phd,Abel12talkFICS} summarizes the advantages of
type-based termination as follows:
\textbf{communication} (programmers can think using types),
\textbf{certification} (types are machine checkable certificates),
\textbf{a simple theoretical justification}
        (no additional complication for termination other than type checking),
\textbf{orthogonality} (only small parts of the language are affected,
        \eg, principled recursion schemes instead of general recursion),
\textbf{robustness} (type system extensions are less likely to
                        disrupt termination checking),
\textbf{compositionality}\footnote{This is not listed in Abel's thesis,
                                but comes from his invited talk in FICS 2012.}
        (one needs only types, not the code, for checking the termination), and
\textbf{higher-order functions and higher-kinded datatypes}
        (works well even for higher-order functions and non-regular datatypes,
        as a consequence of compositionality).
In his dissertation \cite{abel06phd} (Section 4.4) on sized types,
Abel views the Mendler-style approach as enforcing size restrictions
using higher-rank polymorphism as follows:
\begin{itemize}
\item The abstract recursive type $r$ in Mendler style corresponds to
        $\mu^\alpha F$ in his sized-type system (System~\Fwhat),
        where the sized type
        for the value being passed in corresponds to $\mu^{\alpha+1} F$.
\item The concrete recursive type $\mu F$ in Mendler style corresponds to
        $\mu^\infty F$ since there is no size restriction.
\item By subtyping, a type with a smaller size index can be cast to
        the same type with a larger size index.
\end{itemize}
The same intuition holds for the termination behaviors
of Mendler-style recursion schemes over positive datatypes.
Mendler-style recursion schemes terminate, for positive datatypes,
because $r$-values are direct subcomponents of the value being eliminated.
They are always smaller than the value being passed in.
Types enforce that recursive calls are only well typed,
when applied to smaller subcomponents.

Abel's System~\Fwhat\ can express primitive recursion quite naturally
using subtyping. The casting operation $(r \to \mu F)$ in Mendler-style
primitive recursion corresponds to an implicit conversion by subtyping
from $\mu^\alpha F$ to $\mu^\infty F$ because $\alpha \leq \infty$.

System~\Fwhat\ \cite{abel06phd} is closely related to
System~\Fixw\ \cite{AbeMat04}. Both of these systems are base on
equi-recursive fixpoint types over positive base structures.
Both of these systems are able to embed (or simulate) Mendler-style
primitive recursion (which is based on iso-recursive types) via
the encoding \cite{Geu92} of arbitrary base structures into
positive base structures.
%% In \S\ref{sec:fixi:data}, we rely on
%% the same encoding, denoted by $\Phi$, when embedding \MPr\ into System~\Fixw.

Abel's sized-type approach provides good intuitions why 
certain recursion schemes terminate over positive datatypes.
But, it does not give a good intuition of whether or not
those recursion schemes would terminate for negative datatypes,
unless there is an encoding that can translate negative datatypes into
positive datatypes. For primitive recursion, this is possible (as we
mentioned above). However, for our recursion scheme \MsfIt, which is
especially useful over negative datatypes, we do not know of an encoding
that can map the inverse augmented fixpoints into positive fixpoints.
So, it is not clear whether Abel's the sized type approach based on
positive equi-recursive fixpoint types can provide a good intuition
for the termination behavior of \MsfIt. In the following section,
we will discuss another Mendler-style recursion scheme (\mprsi{}), which
is also useful over negative datatypes and has a termination property
(not proved yet) based on the size of the index in the datatype.


