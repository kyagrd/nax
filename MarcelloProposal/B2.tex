\documentclass[11pt,twocolumn]{article}

\usepackage{graphicx}

\usepackage{amsfonts,euler}
\usepackage{MnSymbol}

\usepackage[all]{xy}
\UseComputerModernTips
\xyoption{knot}

\usepackage[a4paper,includehead,includefoot,headheight=13.6pt,
  headsep=4mm,top=1cm,bottom=1cm,left=2cm,right=2cm]{geometry}

\usepackage{float}
\floatstyle{ruled}
\restylefloat{figure}

\usepackage{dashrule}

\usepackage{multirow}

\usepackage{color}
\definecolor{grey}{gray}{.85}

\setcounter{secnumdepth}{4}
\newcommand{\myparagraph}[1]{\paragraph*{\em #1}}
\newcommand{\mynumparagraph}[1]{\paragraph{\em #1}}

%%% BEGIN macros
\newenvironment{myitemize}
  {\begin{list}{$\bullet$}
  {\setlength{\topsep}{2pt}
   \setlength{\partopsep}{2pt}
   \setlength{\itemsep}{2.5pt}
   \setlength{\parsep}{2.5pt}
   \setlength{\leftmargin}{1em}
   \setlength{\labelwidth}{.5em}}}
  {\end{list}}
\newenvironment{myindentitemize}
  {\begin{list}{$-$}
  {\setlength{\topsep}{2pt}
   \setlength{\partopsep}{2pt}
   \setlength{\itemsep}{2.5pt}
   \setlength{\parsep}{2.5pt}
   \setlength{\leftmargin}{2.125em}
   \setlength{\labelwidth}{1.625em}}}
  {\end{list}}
\newenvironment{myquote}
  {\begin{list}{}
  {\setlength{\topsep}{2pt}
   \setlength{\partopsep}{2pt}
   \setlength{\itemsep}{2.5pt}
   \setlength{\parsep}{2.5pt}
   \setlength{\rightmargin}{1em}
   \setlength{\leftmargin}{1em}
   \setlength{\labelwidth}{.5em}}}
  {\end{list}}
\newenvironment{btritemize}
  {\begin{list}{\btr}
  {\setlength{\topsep}{2pt}
   \setlength{\partopsep}{2pt}
   \setlength{\itemsep}{2.5pt}
   \setlength{\parsep}{2.5pt}
   \setlength{\leftmargin}{1em}
   \setlength{\labelwidth}{.5em}}}
  {\end{list}}
\newcounter{CC}
\newenvironment{resenumerate}
  {\begin{list}{[\textbf{\arabic{CC}]}}
  {\usecounter{CC}
   \setlength{\topsep}{2pt}
   \setlength{\partopsep}{2pt}
   \setlength{\itemsep}{2.5pt}
   \setlength{\parsep}{2.5pt}
   \setlength{\leftmargin}{1.65em}
   \setlength{\labelwidth}{1.15em}
 }}
  {\end{list}}

\newcommand{\mysf}{\small\sf}
\newcommand{\mytextsf}[1]{\textsf{\small #1}}
\newcommand{\erc}{{\small\sf MaStrPLan}}
\newcommand{\ERC}{Mathematical Structures for Type Theories,\\[-.5mm] Logical
  Systems, and Programming Languages}

\newcommand{\hide}[1]{}
\newcommand{\hidefootnote}[2]{}
\newcommand{\hidewiring}[1]{}
\newcommand{\note}[1]%{}
{\begin{quote}{\color{blue}$\leadsto$ \bf\em #1}\end{quote}}
\newcommand{\hidenote}{\hide}

\newcommand{\vfigspace}[1]{}%{\vspace*{#1}}

\newcommand{\pref}[1]{\,(\ref{#1})}
\newcommand{\itemref}[1]{\textbf{[\ref{#1}]}}

\newcommand{\eg}{\emph{eg.}}
\newcommand{\Eg}{\emph{Eg.}}
\newcommand{\vs}{\emph{vs.}}
\newcommand{\ie}{\emph{ie.}}
\newcommand{\viz}{\emph{viz.}}
\newcommand{\etal}{\emph{et al.}}

\newcommand{\lcalculus}{\mbox{$\lambda$-calculus}}
\newcommand{\SystemL}{\mbox{System~$L$}}
\newcommand{\SystemF}{\mbox{System~$F$}}
\newcommand{\SystemFi}{\mbox{System~$F_i$}}
\newcommand{\SystemFomega}{\mbox{System~$F_\omega$}}
\newcommand{\LC}{\mbox{$LC$}}

\newcommand{\btr}{$\blacktriangleright$}

\newcommand{\reqpsize}{8.113395cm}%{\columnwidth}

\newcommand{\req}[2]{\begin{center}\colorbox{grey}{\begin{minipage}{\reqpsize} 
  \mytextsf{Research question}\hfill$^{\mbox{\scriptsize see #1 }}$\\[-5.5mm]
  \begin{btritemize}
  \item #2
  \end{btritemize}
\end{minipage}}\end{center}}

\newcommand{\reqs}[2]{\begin{center}\colorbox{grey}{\begin{minipage}{\reqpsize}
  \mytextsf{Research questions}\hfill$^{\mbox{\scriptsize see #1 }}$\\[-5.5mm]
  \begin{btritemize}
  \item #2
  \end{btritemize}
\end{minipage}}\end{center}}
\newcommand{\rep}[2]{\begin{center}\colorbox{grey}{\begin{minipage}{\reqpsize}
  \mytextsf{Research pathway}\hfill$^{\mbox{\scriptsize see #1 }}$\\[-5.5mm]
  \begin{btritemize}
  \item #2
  \end{btritemize}
\end{minipage}}\end{center}}

\newcommand{\Set}{{\boldsymbol{\mathscr S}}}
\newcommand{\scat}[1]{\mathbb{#1}}
\newcommand{\cat}[1]{\mathscr{#1}}
\newcommand{\op}{\circ}
\newcommand{\Id}{\mathrm{Id}}
\newcommand{\Di}{\mathrm{Di}}
%%% END macros

\usepackage{fancyhdr}
\pagestyle{fancy}
\lhead{Marcelo Fiore}
\chead{Part\,B2}
\rhead{\erc}
\lfoot{}
\cfoot{\thepage}
\rfoot{}

\usepackage[compact]{titlesec}

\usepackage{compactbib}

\usepackage{setspace}
\setstretch{.9}

\usepackage{times}

\makeatletter
\renewcommand\@biblabel[1]{#1}
\makeatother

\renewcommand{\thesection}{\alph{section}}
\renewcommand{\thesubsection}{\alph{section}-\arabic{subsection}}
\renewcommand{\theparagraph}{\alph{section}-\arabic{subsection}.\roman{paragraph}}

\def\contentsname{\large Contents\\[-7.5mm]}
\usepackage{minitoc}
\tightmtctrue

\usepackage[draft]{hyperref}
\hypersetup{pdftitle={\ERC},pdfauthor={Marcelo Fiore}}

\begin{document}

\twocolumn[\begin{@twocolumnfalse}
	\hfill{\bfseries\LARGE The Scientific Proposal}\hfill\null\\
\end{@twocolumnfalse}]

%{\small\tableofcontents}

\section{\underline{State of the art and ob}j\underline{ectives}}
\label{StateOfTheArtSection}

We propose research in logical methods for computer science.
Specifically, in formal languages and their mathematical models.

We are interested here in two kinds of formal languages:
\begin{myindentitemize}
  \item
    languages for mathematical proof, and 
  \item
    languages for computer programming.
\end{myindentitemize}
The research spectrum covered by these is by now too broad and eclectic to aim
at encompassing them all.  Our focus is on research driven by the thesis that
\begin{myquote}
\item
Languages for mathematical proof and languages for computer programming
are of the same character.
\end{myquote}
This view %conception 
is of course not original to us.  It started developing in the 1970s, with
proponents from computer science (\eg~Dijkstra, Hoare, Strachey) and from
mathematics (\eg~de Bruijn, Scott, Martin-L\"of); and it is currently central
to research in programming languages and in constructive mathematics.

An aspect of the thesis that is fundamental to us here
%There is an aspect of the thesis that is rooted in the constructive approach
%to mathematical practice and is fundamental to us here.  This 
can be 
impressionistically %roughly 
presented as a correspondence as follows
%\begin{equation}\label{PATproportion}\xymatrix@C5pt{
%  \frac{\txt{program}}{\txt{Type}} 
%  & \mbox{\Large$\approx$} & 
%  \frac{\txt{proof}}{\txt{Proposition}}
%}\end{equation}
\[
  \mbox{program : Type} \enspace \approx \enspace \mbox{proof : Proposition} 
\]
One can intuitively argue for it along the lines below:
\begin{myitemize} 
\item 
  to type a program is to establish a property of it (\eg~that certain
  input/output behaviour or invariant is maintained); while 
\item 
  to give a constructive proof of a proposition (\eg~that every number is
  smaller than some prime number) is to construct a program that 
  certifies %realises
  the statement (\eg~one that on inputing a number, outputs a prime number
  bigger than it).
\end{myitemize} 

Two disciplines are clearly involved in investigating this correspondence:
Mathematical Logic and Programming Theory.  
%Much research, as the one proposed here, is geared towards finding a middle
%ground that can accommodate the concerns of both, and expand their 
%realms. %domains.
%
This is however only half of the picture.  The full strength of the
correspondence involves also the disciplines of Type Theory and Category
Theory.  
%Their role is to enrich~(\ref{PATproportion}) as follows 
%\[\xymatrix@C5pt@R2.5pt{
%  & \frac{\txt{\small construction}}{\txt{\small Type}} 
%    \ar@{}[rd]|-{\rotatebox[origin=c]{-45}{\Large$\approx$}}
%  & 
%    \\
%    \frac{\txt{\small program}}{\txt{\small Type}} 
%    \ar@{}[ru]|-{\rotatebox[origin=c]{-135}{\Large$\approx$}}
%    & & 
%    \frac{\txt{\small proof}}{\txt{\small Proposition}}
%  \\
%  & \frac{\txt{\small morphism}}{\txt{\small Object}} 
%  \ar@{}[ru]|-{\rotatebox[origin=c]{-135}{\Large$\approx$}}
%  & 
%}\]
Figure\,\ref{ResearchAreas} 
\vfigspace{-2mm}\begin{figure}[h]
\caption{Research areas and interactions.}
\vspace*{2mm}
\begin{center}
\hspace*{.5mm}
\xymatrix@R=25pt@C=12.5pt{
& 
\raisebox{7mm}{\fbox{\txt{\small Mathematical\\\small Logic}}}
\ar@/_1em/@{<->}[ddl]<-1em>|-
  {\txt{\scriptsize Propositions\\\scriptsize as Types}}
\ar@/^1em/@{<->}[ddr]<1em>|-
  {\txt{\scriptsize Proofs as\\\scriptsize Programs}} 
& 
\\
& 
\ar@{<->}[dl]|-
  {\txt{\scriptsize Internal\\\scriptsize Languages}}
\ar@{<->}[dr]|-
  {\txt{\scriptsize Denotational\\\scriptsize Semantics}} 
\ar@{<->}[u]|-
{\txt{\scriptsize Model Theory}}
\fbox{\txt{\small Category\\\small Theory}}
& 
\\
  \fbox{\txt{\small Type\\\quad\small Theory\quad\null}}
\ar@/_1em/@{<->}[rr]|-
  {\txt{\scriptsize Programming\\\scriptsize Languages}}
& & 
\fbox{\txt{\small Programming\\\small Theory}}
}
\end{center}
\vspace*{-2mm}
\label{ResearchAreas}
\end{figure}\vfigspace{-2mm}
gives a schematic view of the interactions between these four research
areas in this context.  Each of them is complementary to the others.
Together, as a unified whole, they have shaped various fields of
mathematics.  
\hide{
Indeed, consider for instance that: 
\begin{myitemize}
\item
  the categorical interpretation of quantifiers as
  adjoints~\cite{LawvereAinF} informed the development of the
  type-theoretic dependent sums and dependent products~\cite{ScottCV};
\item
  type theories are the foundation of programming-language typing
  systems~\cite{Pierce};
\item
  the control operators of programming languages are key to the
  constructive interpretation of classical proofs~\cite{Griffin}; and 
\item
  model-theoretic studies of the polymorphic lambda
  calculus~\cite{GirardSystemF,Reynolds} led to remarkable small complete
  categories~\cite{Hyland}.
\end{myitemize}
}

It is within this spirit that we propose cross-cutting research in
Category Theory, Mathematical Logic, Programming Theory, and Type Theory.
We contend that an approach neglecting any one of them is to the detriment
of the others, missing the depth and richness of the subject and,
furthermore, missing opportunities for research and development.  

{\color{red}\ldots FINISH: sketch goals, objectives, team \ldots}

\hide{%%% BEGIN hide
\noindent\hrulefill
We propose cross-cutting research in four areas: category theory,
mathematical logic, programming theory, and type theory.  Each of these 
areas is complementary to the others. Together they form a unified whole.
Figure\,\ref{ResearchAreas}
gives a schematic view of the areas together with their interactions.
\vfigspace{-2mm}\begin{figure}[h]
\caption{Research areas and interactions.}
\vspace*{2mm}
\begin{center}
\hspace*{.5mm}
\xymatrix@R=25pt@C=12.5pt{
& 
\raisebox{7mm}{\fbox{\txt{\small Mathematical\\\small Logic}}}
\ar@/_1em/@{<->}[ddl]<-1em>|-
  {\txt{\scriptsize Propositions\\\scriptsize as Types}}
\ar@/^1em/@{<->}[ddr]<1em>|-
  {\txt{\scriptsize Proofs as\\\scriptsize Programs}} 
& 
\\
& 
\ar@{<->}[dl]|-
  {\txt{\scriptsize Internal\\\scriptsize Languages}}
\ar@{<->}[dr]|-
  {\txt{\scriptsize Denotational\\\scriptsize Semantics}} 
\ar@{<->}[u]|-
{\txt{\scriptsize Model Theory}}
\fbox{\txt{\small Category\\\small Theory}}
& 
\\
  \fbox{\txt{\small Type\\\quad\small Theory\quad\null}}
\ar@/_1em/@{<->}[rr]|-
  {\txt{\scriptsize Programming\\\scriptsize Languages}}
& & 
\fbox{\txt{\small Programming\\\small Theory}}
}
\end{center}
\vspace*{-2mm}
\label{ResearchAreas}
\end{figure}\vfigspace{-2mm}
We contend that a study neglecting any one of them is to the detriment of
the others, missing the depth and richness of the subject, and furthermore
missing opportunities for research and development.  
\noindent\hrulefill
}%%% END hide

\subsection{Origins and influences}
\label{Origins}

This section %Section\pref{Origins} 
outlines the origins and influences that led to the holistic
Figure\,\ref{ResearchAreas}. 
%, roughly expanding through 1935 to 1985.
Section\pref{StateOfTheArtSubsection} describes the current state of the
field and raises questions for research.  %Having set the scene,
Section\pref{ObjectivesSubsection} presents the research objectives we
wish to pursue and the team we have assembled to reach them.  
%Both of these points are further expounded upon in
%Sections\,(\ref{MethodologySection}\,\&\,\ref{ResourcesSection}).

\paragraph{Logic and computation.}

%It is worth recalling that 
Computer science was born as a branch of mathematics, specifically
mathematical logic, even before the first electronic computers were built.
Its inception %The inception of computer science 
was Hilbert's %~\cite{?}
1928 Entscheidungsproblem (decision problem),
asking whether there is an algorithmic procedure for deciding mathematical
statements.  Negative answers were provided independently by
Church~\cite{Church1936} in 1936 and Turing~\cite{Turing} in 1937,
giving birth to the mathematical theory of computation.  Their completely
different approaches gave rise to different branches of theoretical
computer science that still persist today.  
On one hand, the line
of development starting with Church's {\lcalculus} %~\cite{?} 
concerns itself with prototypical computational languages that are
used to study high-level programming languages.  
On the other hand, Turing's machines %~\cite{?} 
are the most widely 
used model for analysing computational complexity.

Church's view is central to this proposal.  The {\lcalculus} is a
deceptively simple formal system.  Its syntax consists of three types of
phrases: variables, applications, and abstraction.  Church's seminal
innovation was in introducing the latter one, 
%the other two being already familiar form the realm of
%algebra~\cite{Birkhoff}, 
which in modern terminology is referred to as a binding operator, a notion
that goes beyond the operators of universal algebra~\cite{Birkhoff}.
Binding operators are an integral part of all high-level  programming
languages.  Their mathematical theory is subtle %, see~\ref{}, 
because they define syntax up to the consistent renaming of bound
variables (technically referred to as \mbox{$\alpha$-equivalence}).
This subtlety is made evident by the anecdotal fact that the first definitions of
substitution were formally flawed.  
%These important topics are core to our proposed research in
%Section\pref{AlgebraicTypeTheoryParagraph}.

Incidentally, the notion of variable binding was initially introduced much
earlier, by Frege~\cite{Frege1879}, in the context of mathematical logic.  
He used binders to define a formal symbolic
system axiomatising not only the propositional connectives of Boolean
logic~\cite{Boole} but also, for the first time, the quantifiers of
predicate logic.

The %In modern notation, the 
{\lcalculus} syntax is given by
  \[\begin{array}{rcll}
    s , t & ::= & & \mbox{($\lambda$-terms)}\\
      & \mid & x & \quad\mbox{(variables)}\\
      & \mid & (t)s & \quad\mbox{(application)}\\
      & \mid & \lambda x.\,t & \quad\mbox{(abstraction)}
  \end{array}\]
The system has only one rule of computation:
  \begin{equation}\label{BetaReduction}\begin{array}{rl}
(\beta) & (\lambda x.\,t)s \,\longrightarrow\, t[s/x]
  \end{array}\end{equation}
by which the result of computing a so-called redex $(\lambda x.\,t)s$ is
the \mbox{$\lambda$-term} $t[s/x]$ denoting the result of substituting the
free occurrences of $x$ in $t$ by $s$.  Non-terminating behaviour arises from
self application.

\paragraph{Type theory and logic.}
\label{SectionATypeTheoryAndLogicParagraph}

The concept of `type' was conceived to solve a foundational
problem.  In~\cite{Frege1903}, building on~\cite{Frege1879}, Frege proposed
a logical system as a foundation for mathematics including
arithmetic.  In his now famous paradox, Russell~\cite{Russell1902}
observed that one of Frege's axioms led to inconsistency.  He was led to
this contradiction by related contradictions he found while developing his account of set
theory.  Russell's introduction of the `doctrine of types'~\cite[Appendix~B]{Russell1903} 
was introduced to overcome such foundational problems -- well-typed systems do
not have these inconsistencies.

Type Theory, as we now know it, arose from the integration by Church
of types into his \lcalculus~\cite{Church1940}.  This was a natural step,
particularly if one held the naive interpretation of \mbox{$\lambda$-abstraction} as
defining functions.  The type theory known as the Simply-Typed Lambda
Calculus has a set of types consisting of basic ones closed under a
function-type constructor:
  \[\begin{array}{rcll}
    \sigma,\tau & ::= & & \mbox{(simple types)}\\
    & \mid & \theta & \quad\mbox{(basic types)}\\
    & \mid & \sigma\to\tau & \quad\mbox{(function types)}
  \end{array}\]
In the Simply-Typed Lambda
Calculus, \mbox{$\lambda$-terms}~$t$ are classified by types~$\tau$, in
contexts $\Gamma$ (assigning types to variables), for which the notation
  \[
  \Gamma\vdash t:\tau
\]
is commonly used.  This is done according to syntax-directed rules that in the
mathematical vernacular are presented as follows\\[-2mm]
  \[\begin{array}{c}
    \\ \hline
    x_1:\tau_1,\ldots,x_n:\tau_n\vdash x_i:\tau_i
  \end{array}
  \enspace(1\leq i\leq n)
  \]
  \[\begin{array}{c}
    \Gamma\vdash t:\sigma\to\tau
    \quad 
    \Gamma\vdash s:\sigma
    \\ \hline
    \Gamma\vdash (t)s:\tau
  \end{array}\]
  \begin{equation}\label{LambdaAbstraction} 
  \begin{array}{c}
    \Gamma,x:\sigma\vdash t:\tau
    \\ \hline
    \Gamma\vdash\lambda x.\,t:\sigma\to\tau
  \end{array}
\end{equation}

A fundamental discovery was made
by Curry~\cite{Curry1934} and Howard~\cite{Howard1969}
in two slightly different, though related, logical
contexts (technically, Hilbert-style deduction %~\cite{Hilbert} 
and Gentzen's natural deduction in sequent form~\cite{Gentzen1935}).
We will refer to this discovery as the 
Propositions-as-Types correspondence. %(or Curry-Howard correspondence).  
Roughly, it establishes a correspondence between terms having types and proofs
proving propositions.  For the Simply-Typed Lambda Calculus, the correspondence
can be illustrated by noting that the erasure of term information in the typing 
rules (given above) yields the deduction rules\\[-7mm]
  \begin{center}
  $\begin{array}{c}
    \\ \hline
    \tau_1,\ldots,\tau_n\vdash \tau_i
  \end{array}
  \enspace(1\leq i\leq n)$
  \\[2mm]
  $\begin{array}{c}
    \Gamma\vdash \sigma\to\tau
    \quad 
    \Gamma\vdash\sigma 
    \\ \hline
    \Gamma\vdash\tau
  \end{array}
  \enspace\quad
  \begin{array}{c}
    \Gamma,\sigma\vdash\tau
    \\ \hline
    \Gamma\vdash\sigma\to\tau
  \end{array}$
\end{center}
\vspace*{-1mm}
of Intuitionistic %Minimal 
Logic. %~\cite{?}.

In another very important direction Church introduced a Simple Theory of
Types~\cite{Church1940} as an axiomatization of Higher-Order Logic.
This axiomatization was formalised within a Simply-Typed Lambda Calculus with basic types
for both individuals and propositions, further enriched with constants for the logical
connectives.  In doing so, he adopted a radically new perspective,
shifting the status of the Simply-Typed Lambda Calculus from that of 
a `language' to a `metalanguage';~\ie, a language in which it is
possible to represent and work with other languages. In this particular case
for simple type theories.  When regarded as a metalanguage, the Simply-Typed
Lambda Calculus is considered as an equational theory, with the
computational \mbox{$\beta$-rule}~(\ref{BetaReduction}) stated as an
equation together with the extensionality equation
\[\begin{array}{rll}
(\eta) & \lambda x.\,(t)x = t 
& \mbox{, where $x$ is not free in $t$}
  \end{array}\]

The processes of abstracting from languages to metalanguages has become a
common activity in computer science, and plays an important role in our
proposed investigations.  

\paragraph{Category theory, logic, and type theory.}

The theory of categories was introduced by Eilenberg and Mac
Lane~\cite{EilenbergMacLane}.  A category is a mathematical structure
consisting of objects and morphisms.  Morphisms are classified
by pairs of objects. The notation $f:A\to B$ stipulates that $A$ and $B$ are objects, and
that $f$ is a morphism with domain $A$ and codomain~$B$.  
Categories come equipped with an associative law that composes pairs of
morphisms 
%as on the left below
%\begin{equation}\label{Category}
%  \begin{array}{c}
%  f:A\to B\quad g:B\to C
%  \\ \hline
%  g\,f:A\to C
%\end{array}
%\quad\qquad
%1_A:A\to A
%\end{equation}
together with, for every object, an identity morphism 
%as on the right above
that is a neutral element for composition.  
%Categories were defined to introduce functors, a notion of morphism between
%categories, which in turn was defined to introduce natural transformations, a
%notion of morphism between functors.

It is helpful, in understanding categories, that they have two kinds of uses: in the
large and in the small.  In the large, categories are seen as mathematical
universes of discourse (within which mathematical constructions take place,
typically by means of universal properties); like the categories of: sets and
functions, algebraic structures and homomorphisms, spaces and continuous
functions.  In the small, categories are seen as mathematical objects
themselves; like a set, a monoid, a preorder, and a graph, all of which can
be suitably regarded as a category.

Category Theory analyses mathematical structure by isolating the principles
for which mathematical theories work.  Because of this attention to
essentials, it has had considerable success in unifying ideas from many areas
of mathematics. 
%Today it an indispensable tool in abstract algebra, algebraic geometry,
%mathematical logic, mathematical physics, topology, and theoretical computer
%science, as well as a growing research area in its own right.
%
The connection between category theory, logic, and type theory was
initiated by Lawvere~\cite{LawvereAinF} and Lambek~\cite{LambekI}, both
of whom recognised logical systems and type theories as categories with
structure.  
%Here we are concerned with the former, postponing the latter to the following
%section.

Lawvere's insight was to understand logical connectives and type
constructors as categorical structures arising from universal properties
in the form of adjoint functors, a notion introduced by Kan~\cite{Kan}
motivated by homology theory.  For instance, the categories with structure
corresponding to the Simply-Typed Lambda Calculus with products are
Cartesian Closed Categories.  These have categorical products and
exponentials, respectively denoted $\times$ and $\Rightarrow$, defined by
adjoint situations establishing natural bijective correspondences between
morphisms as follows:
\[
  \begin{array}{c}
    C\to A \enspace,\enspace C\to B
    \\ \hline\hline
    C\to (A\times B)
  \end{array}
  \quad\qquad
  \begin{array}{c}
    (C\times A)\to B
    \\ \hline\hline
    C\to (A\Rightarrow B)
  \end{array}
\]
The required naturality condition amounts to the computational $\beta$
laws and the extensionality $\eta$ laws.

The similarity between the bijective correspondence on the right above and the
typing rule for \mbox{$\lambda$-abstraction}~(\ref{LambdaAbstraction}) is not
casual, and it is now well-understood that the Simply-Typed Lambda Calculus
with products provides an internal language for Cartesian Closed Categories;
namely, it is the calculus of all such models.  This view further enriches the
Propositions-as-Types correspondence as 
follows %in Figure\,\ref{PAT}.
%\vfigspace{-2mm}\begin{figure}[h]
%\caption{Propositions-as-Types Trinity}
%\vspace*{2mm}
\[\xymatrix@C=10pt@R=12.5pt{
    & \ar@{<->}[dl]_(.5){\null\hspace*{-15mm}\txt{\scriptsize Internal
        Language}} 
    \ar@{<->}[dr]^(.55){\null\hspace*{4mm}\txt{\scriptsize Model Theory}}
    \txt{\small Cartesian Closed\\ \small Categories} & \\
    \txt{\small Simply-Typed\\ \small Lambda Calculus\\ \small with products}
    \ar@{<->}[rr]_-{\txt{\scriptsize Propositions as Types}} & & 
    \txt{Minimal\\ Intuitionistic\\ Logic}
  }\]
%\vspace*{-2mm}
%\label{PAT}
%\end{figure}\vfigspace{-2mm}
and is by now the standard with respect to which analogous developments in the
area are measured.  

\paragraph{Type theory and programming.}

The Simply-Typed Lambda Calculus, regarded as a language rather than as a
metalanguage, is a prototypical functional programming language.  The change
of perspective from Type Theory to Programming Theory is not straightforward
and comes with considerations that enrich both subjects.  

In the context of programming languages, Milner %~\cite{Milner1978}
understood early on that while a programming language should come with a
type discipline to classify programs according to their type invariants,
programmers would be better served if type annotations were inferred
automatically.  

The problem of type inference (by which given a program one wishes to compute
the best possible type for it) became of central practical importance.  For
Combinatory Logic\footnote{Combinatory Logic is an important symbolic
  formalism introduced by Sch\"onfinkel~\cite{Schonfinkel} closely related
  to the {\lcalculus}, but based on algebraic combinators, that enjoys the
  Propositions-as-Types correspondence with respect to the Hilbert-style
  deduction system for Intuitionistic %Minimal 
  Logic.} 
this problem was solved by Hindley~\cite{Hindley1969}.  Independently,
however, Milner~\cite{Milner1978} solved it in a context more relevant to
programming; namely, for the Simply-Typed Lambda Calculus with parametric
polymorphism.  The algorithm is now known as the Hindley-Milner type
inference method.  The approach is very robust, extending broadly to
many related systems.
 
Polymorphism in programming refers to languages that support abstraction
mechanisms by which a program (function or procedure) can be used with a
variety of types.  The concept was introduced by Strachey~\cite{Strachey1967},
who further classifies the phenomena into ad-hoc polymorphism and parametric polymorphism.
The former is also referred to as overloading and accounts for uses of a
program with different types (like integers and reals) by means of different
algorithms.  The latter indicates the use of a uniform program for all types
(like a duplicator program making copies of its input).

Reynolds~\cite{Reynolds} formalised the programming intuition by introducing
the Polymorphic Typed Lambda Calculus; an extension of the Simply-Typed Lambda
Calculus with polymorphic types.  Roughly, these are abstract parametric types
whose programs can be used for all instances of the parameter.  Strikingly,
this system had already been proposed several years earlier by
Girard~\cite{GirardSystemF} under the name of {\SystemF}, as the
type-theoretic counterpart of Second-Order Propositional Logic in the
context of the Propositions-as-Types correspondence.  Further, Girard had
also introduced {\SystemFomega}; the type-theoretic counterpart of
Higher-Order Propositional Logic.  These logical systems widely extend the
expressiveness of the Simply-Typed Lambda Calculus, notably by the
possibility of encoding inductive data types~\cite{BoehmBerarducci}.
{\SystemFomega} lies at the core of the Haskell programming
language. %~\cite{EqProofICFP12}.

The type systems mentioned above aim at providing logical foundations. 
When viewed as programming languages, they can only introduce terminating
computations.  There are various ways in which one can extend them to Turing
complete computational languages.  In this direction, and motivated by
model-theoretic studies of the \lcalculus, Scott~\cite{ScottTCS}
introduced an extension of Typed Combinatory Logic with a fixpoint
combinator for general recursion.  Plotkin~\cite{PlotkinLCF} studied it as a
programming language, shifting from Typed Combinatory Logic to Simply-Typed
Lambda Calculus.  In doing so, he opened up further possible distinctions in
the study of type theories for computation; namely, the consideration of
equational theories for different function call mechanisms: by value or by
name~\cite{PlotkinCBVCBN}, as in ML or Haskell.  

The influence of mathematical models on type theories, logical systems, and
programming languages %as advocated by Scott~\cite{?} 
plays a central role throughout the proposal.

\subsection{State of the art: Questions and pathways for research}
\label{StateOfTheArtSubsection}

Having introduced the scientific framework, we turn our attention to topics of
active research.  In each subsection below we identify one or more
broad questions or pathways for research.
In Section\pref{MethodologySection} we return to these questions, point by point,
and construct a detailed research plan.

\setcounter{paragraph}{0}
\paragraph{Foundations.}

We have already mentioned several type theories, and we will mention a few
more in the sequel.  However, the following fundamental question
remains open:
\req{(\ref{AlgebraicTypeTheoryParagraph})}
{What is a type theory?}

Section\pref{AlgebraicTypeTheoryParagraph} aims at a comprehensive
mathematical answer, that will also serve as a framework for our other
type-theoretic developments.  
\hide{
  We regard this as a step towards the related open question:
\req{}{What is a programming language?}
that will be kept in the background of our investigations.
}

\hidewiring{%%% BEGIN hide
\paragraph{Logical wiring.}
\label{LogicalWiringSubsection}

Lambek~\cite{LambekI} recognised the similarity between the basic
structure of a category, given by identities and
composition~(see\pref{Category}), and the wiring of logical deduction
systems, specifically Gentzen's sequent calculi~\cite{Gentzen1935}, given
by the axiom and cut rules.  Intuitionistic sequents led
Lambek~\cite{LambekII} to axiomatize their algebra under the concept of
multicategory.  This correspondence is roughly as outlined below (where the
vector notation $\vec{\,\cdot\,}$ stands for finite sequences of objects): 
\begin{center}\begin{tabular}{|c|c|}\hline
    \begin{tabular}{c}
      \small Intuitionistic\\[-1mm] \small Sequent Calculus
  \end{tabular}
  &
  \begin{tabular}{c}
    \small
    Multicategories
  \end{tabular}
  \\ \hline\hline
  \begin{tabular}{l}
    \hspace*{-15mm}
    \small 
    (Axiom)
    \\[-3mm]
    $\begin{array}{c}\\ \hline P \vdash P\end{array}$
  \end{tabular}
  & 
  \begin{tabular}{l}
    \small
    \hspace*{-7mm}
    (Identity)
  \\[1mm]
  $(A)\stackrel{1_A}\longrightarrow A$
  \end{tabular}
  \\ \hline
  \begin{tabular}{l} 
    \hspace*{-2mm}(Cut) 
    \\[1mm]
    $\begin{array}{c}
    \Gamma\vdash P\quad\Gamma_1,P,\Gamma_2\vdash Q
    \\ \hline 
    \Gamma_1,\Gamma,\Gamma_2\vdash Q
    \end{array}$
  \end{tabular}
  &
  \begin{tabular}{l}
    \small
    \hspace*{-3mm}
  (Multicomposition)
  \\%[1mm]
    \hspace*{-2mm}
    $\begin{array}{c}
    \vec Y \stackrel f\longrightarrow A
    \\
    \vec X,A,\vec Z \stackrel g\longrightarrow B
    \\ \hline\\[-4mm]
    \vec X, \vec Y,\vec Z
    \xymatrix@C=25pt{\ar[r]^-{g\, _{\vec
          X}\hspace*{-.0125mm}\circ\hspace*{-.25mm}_{\vec Z}f}&} 
    B
    \end{array}$
    \hspace*{-2mm}
  \end{tabular}
  \\ \hline
\end{tabular}\end{center}
The analogous for classical sequents was done by Szabo~\cite{Szabo} with the
introduction of polycategories.

The structure of multicategory appeared independently in a very different
mathematical context, the work of May~\cite{May} on homotopical algebra, under
the name of Operad.  Operads are now central to studies in higher-dimensional
algebra (see~\eg~\cite{Leinster,LodayVallette}).  

\rep{(\ref{WiringStructureParagraph})}
{Investigate interactions between logical sequent calculi and algebraic operad
  theory.}
}%%% END hide

\paragraph{Dependent types.}
\label{DependentTypesParagraph}

Dependent type theory is a formalism introduced by
de~Bruijn~\cite{deBruijn} that extends simple type theories by
allowing types to be indexed (or parameterised by) other types.  Such
objects abound in computer science and mathematics.  For instance, in
combinatorics one is interested in the type of permutations $\mathfrak
S(n)$ on $n$ elements, as the index (or parameter) $n$ ranges over the
natural numbers $\mathbb N$.  In modern notation, this is presented by a
judgement of the form
\[
  n:\mathbb N \vdash \mathfrak S(n) \mbox{ \mysf type}
\]
There are two fundamental constructions on such dependent judgements as
follows
\[
  \begin{array}{c}
  i: I \vdash T(i) \mbox{ \mysf type}
  \\ \hline
  \vdash \Sigma\,{i:\!I}.\, T(i) \mbox{ \mysf type}
  \end{array}
  \qquad
  \begin{array}{c}
  i: I \vdash T(i) \mbox{ \mysf type}
  \\ \hline
  \vdash \Pi\,{i:\!I}.\, T(i) \mbox{ \mysf type}
  \end{array}
\]
respectively known as dependent sums and dependent product types
(see~\eg~\cite{Jacobs}).  These generalise the product and function types of
Simply-Typed Lambda Calculus and, under the Propositions-as-Types
correspondence, amount to existential and universal quantification.  We omit
discussing the syntax of terms for these types.  As for their equational
theory, we only mention that dependent sums may be weak or strong and that
dependent products may be intensional or extensional.
%---see \eg~\cite{?}.

A fundamental problem in the area is to:
\rep{(\ref{IntensionalTypeTheoryParagraph})}
  {Investigate extensionality in dependent type theory.}
This lays at the core of our proposed investigations in
Section\pref{IntensionalTypeTheoryParagraph}.

The passage from sum and product types to their dependent versions
required new type theories.  Categorical models suggest further
generalisations.  These are the subject of
%Sections\,(\ref{GeneralisedTypeTheoryParagraph}\,\&\,\ref{DirectedTypeTheoryParagraph})
Section\pref{CalculiSubsection}
under the following. 
\rep{(\ref{CalculiSubsection})}
  {Develop type theories from mathematical models.}

\paragraph{Mathematical universes.}
\label{MathematicalUniversesParagraph}

The ability to construct new mathematical universes of discourse from old ones
is a fundamental part of the technical toolkit of researchers in semantics, be
it either in logic or computation.

One technique to do so is to enrich the semantic universe with a mode of
variation.  In its basic form, starting from the universe of sets and
functions~$\Set$ one considers the universe~$\Set^{\scat C}$ of so-called
presheaves consisting of the \mbox{$\scat C$-variable} sets for a small
category~$\scat C$.  The importance of this passage is that the kind of
variation embodied in the parameter small category translates to new, often
surprising, internal structure in the universe of presheaves.  

The presheaf construction was introduced by Grothendieck together with a
more sophisticated and important refinement of it known as the sheaf
construction~\cite{SGA4} (in the context of the Weil conjectures in
cohomology theory).  Sheaves are a central object of study in the area of
mathematics known as Topos Theory~\cite{Elephant}; a topos being a
universe of discourse for Higher-Order Intuitionistic
Logic~\cite{LambekScott}.  

In view of the many possible applications, it is natural to ask:
\reqs{(\ref{MethodologyMathematicalUniversesParagraph})}
  {Which mechanisms are there for changing from a type theory to another one
    as universes of discourse? 

    Can this be done while maintaining the relevant computational properties
    and then incorporated into mechanical proof assistants?}

This question should not only be considered from the topos-theoretic viewpoint
mentioned above; but also from other approaches, notably that of the related
forcing technique of Cohen~\cite{Cohen} (introduced by him to prove the
independence of both the axiom of choice and the continuum hypothesis from
Zermelo-Fraenkel Set Theory) and its elaboration by Scott and Solovay as
Boolean-valued models~\cite{ScottSolovay}.

%Recent work in this direction has been done by
%Coquand~\cite{CoquandForcingInTT} and Jaber, Tabareau, and
%Sozeau~\cite{TypeTheoryWithForcing}.\footnote{Note
%  that~\cite{TypeTheoryWithForcing}, despite its title, is about internalising
%  the presheaf construction on partial orders.}

\paragraph{Indexed programming.}
\label{IndexedProgrammingIntro}

The use of presheaf categories in computer science applications has been
prominent;~\eg~in 
programming language theory, %~\cite{?}, 
lambda calculus, %~\cite{?}, 
domain theory, %~\cite{?}, 
concurrency theory, %~\cite{?}, 
and type theory. %~\cite{?}.  
In particular, in their most elementary discrete form, presheaves can be
found in programming languages as indexed datatypes; programming with
which will be generically referred to as indexed programming.  

Indexed programming developed from two main influences (none to do with
presheaves though): the practical needs of supporting data structures able to
maintain strong data invariants, like nested %~\cite{?} 
and generalised algebraic %~\cite{?}
datatypes (GADTs) in functional programming~\cite{Omega,Haskell}; and the
experimentation with dependently-typed programming
languages~\cite{Cayenne,Epigram} as a by-product of dependent type theory.
These two views somehow pull in opposite directions and, as such, lead to
conceptually different languages.  One is lead to investigate the following.
\rep{(\ref{IndexedProgrammingParagraph})} 
  {Develop foundational type theories for indexed datatypes.  
    
   Design and %subsequently 
   implement indexed programming languages from these and
   %programming-language 
   pragmatics.}

\paragraph{Resources, effects, modalities.}
\label{ResourcesEffectsModalitiesParagraph}

In the late 1980s, two important analyses of computation, respectively for
resources and effects, were proposed by Girard~\cite{GirardLinearLogic} and by
Moggi~\cite{MoggiLambdaC}.  The former was in the contexts of logic and proof
theory; the latter in that of denotational semantics and category theory.
Both, however, have had tremendous impact in programming language theory.  The
resource analysis in the form of Linear Logic calculi; the effect analysis
in the form of Computational \mbox{$\lambda$-calculi}.

From the viewpoint of categorical models, the resource and effect management
structures are respectively seen as comonadic and monadic structures.
Comonads and monads being dual categorical concepts that arose in the context
of cohomology theory in the 1960s~\cite{BeckThesis}.

A classical basic result of category theory establishes a strong
correspondence between (co)monads and adjoint functors
(see~\eg~\cite{MacLane}).  One aspect of this is that every adjunction
\begin{equation}\label{ResourceEffectAdjunction}
  \xymatrix@C=40pt{
    \cat D \ar@/^.75em/[r]|(.625){\mbox{$\,G\,$}}\ar@{}[r]|-\swvdash&
    \ar@/^.75em/[l]|(.625){\mbox{$\,F\,$}} \cat C }
\end{equation}
with $F$ and $G$ respectively left and right adjoint to each other, gives
rise (by composition) to a comonad on $\cat D$ and a monad on $\cat C$.  This
viewpoint gave impetus to further analyses based on the more primitive notion
of adjunction.  

Models of Linear Logic are founded on the theory of
monoidal categories~\cite[Chapter~VII.1]{MacLane}.  For them, one requests
that the adjunction be monoidal with respect to linear structure on $\cat D$
and cartesian (or multiplicative) structure on $\cat C$
(see~\eg~\cite{MelliesCMLL}).  On the other hand, models of Computational
\mbox{$\lambda$-calculi} rely on enriched category theory~\cite{KellyBook}.
Here, the structure is roughly given by an enriched adjunction with $\cat C$
cartesian and $\cat D$ with powers (or exponentials) relative to $\cat C$.

In the context of linear logic, examples are the mixed linear/cartesian models
and calculi of Benton and Wadler~\cite{BentonWadler} and of Barber and
Plotkin~\cite{BarberPlotkin}.  In the context of effect calculi, examples are
the Call-By-Push-Value of Levy~\cite{LevyCBPV} and the Enriched Effect
Calculus of Egger, M{\o}gelberg, and Simpson~\cite{EEC}.  Metaphorically
speaking, both these lines of work regard resources and effects as being
opposite sides of the same coin.  However, this is not so in all models;
and the following question remains unanswered.
%
\req{\pref{PolarisationParagraph}}
  {How can resources and effects be reconciled and unified?}
%
To answer it, a broader view of the subject involving the orthogonal
notion of polarisation seems to be needed.  From the programming-language
viewpoint, this further enriches the computational picture with the
ability of distinguishing between eager \vs~lazy modes of computation and
data structure.  We incorporate this into the following.
%
\rep{(\ref{PolarisationParagraph}\,\&\,\ref{ProgrammingEffectsParagraph})}
  {Study and develop the theory of resource management, computational
    effects, and polarisation.  Percolate this down into the design of
    programming languages.}

From the logical point of view, resource comonads and effect monads are
so-called modal operators, and some of the literature has indeed
considered them as such (see \eg~\cite{Kobayashi}).  %BiermanDePaiva
The field of modal logics is broad, with many subfields of specialised
logics motivated by computation, linguistics, and philosophy.  While a
class of modal logics known as temporal logics have played a prominent
role, and been very successful, in the area of computer aided verification;
the impact of modal logics on programming languages has been peripheral.
It is thus natural and important to reconsider them in this context. 
%
\rep{(\ref{ModalLogicsParagraph}\,\&\,\ref{MetaprogrammingParagraph})}
  {Investigate the Propositions-as-Types 
    correspondence %Trinity 
    for modal logics, and apply it to programming languages.}
We stress that we are specially interested in modalities for computation with
reflection.

\paragraph{Sequent calculi.}
\label{SequentCalculiParagraph}

A theme that runs orthogonal to the logics under consideration is whether
they are specified in natural deduction or sequent calculus
style (see \eg~\cite{vonPlato}).

Most of the work on the Propositions-as-Types correspondence has been done for
natural deduction systems; especially in relation to programming language
theory, where the logical syntax matches that of functional languages.  

On the other hand, there is as yet no established syntax for sequent-style
calculi.  A question at the core of this situation is:
%
\req{(\ref{ProgrammingEffectsParagraph})}
  {What is the categorical algebra of classical sequent calculi?}
%
Nevertheless, a syntactic system that is proving robust in applications
seems to be emerging.  This is the {\SystemL} of Curien and
Herbelin~\cite{CurienHerbelin}, applied further
in~\eg~\cite{Wadler,Munch,CurienMunch}.

The main novel features of {\SystemL}, and its philosophy, are an
intrinsic symmetry (reflecting the computation roles of program and
environment) and a close connection with abstract machines (which are
internalised into the calculus).  From this perspective we ask:
%
\req{(\ref{ProgrammingEffectsParagraph})}
  {What can the proof theory of sequent calculi do for programming?}

\subsection{Research objectives}
\label{ObjectivesSubsection}

The general aim of {\erc} is to build a group dedicated to combined
research in category theory, mathematical logic, programming theory, and
type theory to advance current knowledge in their mathematical foundations
and practical applications.  To this end, {\erc} assembles an
international team of world-leading experts in each of the project
research areas.  

The team consists of \emph{Principal Investigator} Marcelo Fiore;
\emph{Senior Visiting Researchers} Pierre-Louis Curien, Peter Dybjer, and
Tim Sheard; and \emph{Research Associates} Ki Yung Ahn, Nicola Gambino,
and Guillaume Munch-Maccagnoni.  Figure\,\ref{ercTeam} gives a schematic
presentation of the team that matches the research areas as rendered in
Figure\,\ref{ResearchAreas}.  
\vfigspace{-2mm}\begin{figure}[h]
\caption{{\erc} team}
\vspace*{2mm}
\begin{center}
\hspace*{.125mm}
\xymatrix@R=10pt@C=32.5pt{
& 
\raisebox{5mm}
  {\fbox{\txt{\small Pierre-Louis\\ \small Curien}}}
\ar@/_1em/@{<->}[dddl]<-1em>
\ar@/^1em/@{<->}[dddr]<1em> 
& 
\\
\\
& 
  {\fbox{\txt{\small Marcelo\\ \small Fiore}}}
\ar@{<->}[dl]|-
  {\txt{\scriptsize Nicola\\ \raisebox{1mm}{\scriptsize Gambino}}}
\ar@{<->}[dr]|-
  {\txt{\scriptsize Ki Yung\\ \raisebox{1mm}{\scriptsize Ahn}}}
\ar@{<->}[uu]|(.475)
  {\txt{\scriptsize Guillaume\\ \raisebox{1mm}{\scriptsize
        Munch-Maccagnoni}}}
& 
\\
\raisebox{0mm}{\fbox{\txt{\small Peter\\ \small\ Dybjer\ \ }}}
\ar@/_1em/@{<->}[rr]
& & 
\raisebox{0mm}{\fbox{\txt{\small Tim\\ \small\ Sheard\ \ }}}
}
\end{center}
\vspace*{-2mm}
\label{ercTeam}
\end{figure}\vfigspace{-2mm}
Further details are deferred to Section\pref{ResourcesSection}. 

The overall research objectives that we are to undertake are classified
under four headings as follows.
\begin{myitemize}
\item[{\bfseries 1\enspace Foundations:}]\mbox{}\enspace\thinspace 
  A comprehensive research programme on the metamathematics of type theories.
  We will provide an algebraic answer to what type theories are.
  \hidewiring{, establishing connections with areas of abstract algebra
    where related structures play a central role.}

\item[{\bfseries 2\enspace Models:}]\mbox{}\enspace\thinspace
  Study %Development and study 
  of mathematical models for type theories and logical systems. 
  We will investigate model theories targeting semantic problems at the
  forefront of current understanding, specifically for dependent type theory
  and polarised logic.
  
\item[{\bfseries 3\enspace Calculi:}]\mbox{}\enspace\thinspace
  Development of formalisms of deduction as internal languages of mathematical
  theories.
  We will research type theories that go beyond current logical
  frameworks, especially motivated by (higher-dimensional) category theory.
  %and complexity theory.
  
\item[{\bfseries 4\enspace Programming:}]\mbox{}\enspace\thinspace
  Design and implementation of, and experimentation with, novel computational
  languages.  
  We will target languages with indexed data structures, programming paradigms
  stemming from sequent calculi embodying computational effects, and explore
  extensions for metaprogramming.
\end{myitemize}

\section{\underline{Methodolo}gy\hspace{-1mm}\underline{\,}}
\label{MethodologySection}

This section expands our research objectives providing a plan together
with the methodology for its completion.

\note{\ldots}

\subsection{Foundations}
\label{Foundations}

\note{\ldots}

\setcounter{paragraph}{0}
\paragraph{Algebraic Type Theory.}
\label{AlgebraicTypeTheoryParagraph}

To understand our approach it is best to start by stripping type theories
down to their essential bare structure.  To do so, let us eliminate type
constructors and binding operators from them.  What one is left with is a
notion of type theory that reduces to that of many-sorted algebraic
theory~\cite{Birkhoff}, a thoroughly studied area of algebra.

The modern understanding of many-sorted algebraic equational theories is
through three interrelated perspectives as in the Algebraic Trinity of
Figure\,\ref{AlgebraicTrinity}.
\vfigspace{-2mm}\begin{figure}[h]
  \caption{Algebraic Trinity}
  \vspace*{2mm}
  \begin{center}$\xymatrix@R=7.5pt@C=35pt{
      & \txt{\small Categorical\\ \small Algebra} 
      \ar@{<->}[dr]^(.55){\null\hspace*{5mm}\txt{\scriptsize Lawvere~\cite{LawvereThesis}}}
      \ar@{<->}[dl]_(.55){\null\hspace*{-10mm}\txt{\scriptsize Linton~\cite{Linton}}}
      & 
      \\
      \txt{\small Equational\\ \small Logic}
      \ar@{<->}[rr]_-{\txt{\scriptsize Birkhoff~\cite{Birkhoff}}}
      &
      & 
      \txt{\small Universal\\ \small Algebra} 
    }$\end{center}
  \vspace*{-2mm}
\label{AlgebraicTrinity}
\end{figure}\vfigspace{-2mm}
Equational Logic is the metalanguage of algebraic theories, while
Universal Algebra is its model theory.  They were both introduced by
Birkhoff~\cite{Birkhoff}, with the former as a by-product of the latter
related by a soundness and completeness theorem.  The categorical
viewpoint of algebraic theories came later, with the work of
Lawvere~\cite{LawvereThesis} and Linton~\cite{Linton}.  Two crucial
categorical structures that play a pivotal role here are: Lawvere
theories~(closely related to Hall's abstract clones) and finitary
monads~(a notion 
%originally introduced in cohomology theory by Beck~\cite{Check???}, 
arising in algebra from free constructions).

Each of the approaches to algebraic theories in the Algebraic Trinity gives a
different viewpoint of the subject.  Thus, it is important to consider them
all.  For instance, Equational Logic provides the deductive system for formal
reasoning about equations from axioms; Universal Algebra provides a general
notion of model from which one derives an abstract notion of syntactic
structure by means of freely generated models; Categorical Algebra provides an
invariant notion of theory together with a notion of translation between them
that lifts to adjoint functors between categories of models.

Having deconstructed type theories down to many-sorted algebraic theories:
Can one reconstruct them back, and in the process enrich them, as a unified
mathematical theory encompassing all aspects of the Algebraic Trinity?  This
is the main general goal of this research track.

To make substantial progress in the area, our proposal is to
systematically explore a broad spectrum of key features present in type
theories. The scale at which we will be attempting this is unprecedented,
and is graphically represented by the research space of investigation in
Figure\,\ref{TypeTheoryResearchSpace}, 
\vfigspace{-2mm}\begin{figure}[t]
\caption{Algebraic Type Theory research space}
\vspace*{1mm}
\begin{center}
$\xymatrix@C=2.5pt@R=25pt{
*+[]{\fbox{\scriptsize\txt{Many-Sorted\\ Univ.\,Alg.
%\\ Eq.\,Logic
}}}
\ar[rrr]|-{\mbox{\scriptsize\txt{\emph{type}\\\emph{dependency}}}}
& 
& 
& 
*+[]{\fbox{\scriptsize\txt{Dependently-Sorted\\ Algebra}}}
\ar[dd]|-{\mbox{\scriptsize\emph{binding}}}
\\
*+[]{\fbox{\scriptsize\txt{Univ.\,Alg.%\\ Eq.\,Logic
}}}
\ar[u]|-{\mbox{\scriptsize\emph{sorting}}}
\ar[rd]|-{\mbox{\scriptsize\emph{binding}}} 
\ar[dd]|-{\mbox{\scriptsize\emph{linearity}}}
& &
*+[]{\fbox{\scriptsize\txt{Simple\\Type Theory}}}
\ar[rd]|-{\mbox{\scriptsize\txt{\emph{type}\\\emph{dependency}}}}
\ar[dd]|-{\mbox{\scriptsize\txt{\emph{explicit}\\\emph{polymorphism}}}}
& 
\\ 
& 
*+[]{\fbox{\scriptsize\txt{Binding\\Algebra}}}
\ar[ur]|-{\mbox{\scriptsize\txt{\emph{simple}\\\emph{types}}}}
&
&
*+[]{\fbox{\scriptsize\txt{Dependent\\Type Theory}}}
%\ar@{.>}[d]
\\
*+[]{\fbox{\scriptsize\shortstack{Operads}}}
& & 
*+[]{\fbox{\scriptsize\txt{Polymorphic\\Type Theory}}}
%\ar@{.>}[r]
& 
%*+[]{\fbox{\scriptsize\txt{?}}}
}$
\end{center}
\vspace*{-3mm}
\label{TypeTheoryResearchSpace}
\end{figure}\vfigspace{-2mm}
where the \emph{linearity} dimension could be transported along or mixed
with all the other ones.

Initial investigations in this direction have been carried out by Fiore
and PhD~students~\cite{FioreHur,FioreMahmoud}.  These extend the Algebraic
Trinity to the realm of Binding Algebra,~\viz~algebraic languages with binding
operators.  The extension to simple type theories should follow along
similar lines further taking into account the algebraic structure of
types.  As for the other points of the research space in
Figure\,\ref{TypeTheoryResearchSpace}, they are yet to be investigated.  For
these we have the following specific goals.
\begin{resenumerate}\setcounter{CC}{0}
\item
  To develop a mathematical algebraic framework for the semantics of
  language phrases, whereby free models universally characterise the
  abstract syntax of the language.

\item
  To synthesise metalanguages for type theories in the form of formal
  systems for equational deduction, that are sound and complete for the
  model theory.

\item
  To extract syntactic notions of translations between type theories as
  suggested by the mathematical models, and to use these to provide
  constructions for the modular combination of type theories.

\item
  To test and apply the above mathematical theories by formalising them in
  Coq or Agda, while at the same time exercising these systems to the
  limit to identify shortcomings triggering new research in the context of
  proof assistants.
\end{resenumerate}
The outcome of this work will be mathematical foundations for the
aforementioned aspects of type theories that are currently treated in an
ad~hoc fashion.  
\hide{
Furthermore, it will provide a principled approach by which
to reconsider the problem of integrating polymorphism and dependent types. 
%%%to tackle the problem of integrating polymorphism and dependent types in 
%%% a satisfactory 
%%%manner---the unknown in Figure\,\ref{TypeTheoryResearchSpace}. 
}

\hide{ polynomial functors (from discrete to groupoidal) --- the meaning of
  typing rules}

\hidewiring{%%% BEGIN hide
\paragraph{Wiring structure.}
\label{WiringStructureParagraph}

A basic concern in the investigation of
Section\pref{AlgebraicTypeTheoryParagraph} is the
understanding of the mathematical structure of sequents and judgements.
This subject has already received attention, but new perspectives are emerging
that call for its reconsideration.

Remitting ourselves back to Section\pref{LogicalWiringSubsection}, the
first thing to notice is that there are two possible axiomatizations for
multicategory composition: the one exhibited there, known as partial
composition, and a total (or simultaneous) one of the form
{\small\begin{equation}\label{TotalComposition}\begin{array}{c}
  g: A_1,\ldots,A_n \to B
  \\[1mm]
  f_i:\vec{X_i}\to A_i\enspace(1\leq i\leq n)
  \\[.75mm] \hline\\[-2.5mm]
    g\circ(f_1,\ldots,f_n):\vec{X_1},\ldots,\vec{X_n}\to B
\end{array}\end{equation}}%
In the presence of identities, both notions are equivalent.  An interesting
observation from the theory of operads is that this is not so
otherwise~\cite{Markl}.

The %Algebraically, the 
total composition structure~(\ref{TotalComposition}) is well understood
(see~\eg~\cite{FioreFossacs}).
%as given by semigroup structure (technically in the category of Joyal Linear
%Species~\cite{JoyalAdvMath} with respect to the substitution tensor
%product~\cite{KellyOperads,JoyalLNM1234}).  
But, what about the concept of partial composition?  That, after all, is the
predominant in logic and type theory.  For it, Fiore has recently uncovered an
unexpected connection with structure stemming from the mathematical theory of
\emph{Lie algebras}. %~\cite{LieRef}. 
%; specifically, the pre-Lie algebra structure of
%Gerstenhaber~\cite{Gerstenhaber}.  Furthermore, and analogous, though more
%interesting situation occurs with polycategories~\cite{Szabo} and the
%Lie-admissible algebraic structure of Albert~\cite{Albert}.
%
%We regard these observations as the starting point for interactions between
%operad theory and logic and type theory for pursuing research as follows.
%
This is the starting point for research on
\begin{resenumerate}\setcounter{CC}{0}
\item
  New interactions between abstract algebra, and proof and type theory.
\end{resenumerate}
Specifically, 
\begin{resenumerate}\setcounter{CC}{1}
\item 
  There is a rich variety of algebraic structures with notions of composition,
  \eg~motivated by studies in combinatorics, geometry, physics.  What do they,
  and their mathematical theory, transfer to in proof and type theory?  
  
\item 
  Conversely, the binding operators of logic and type theory introduce
  algebraic structure of a kind so far unseen in mathematics.  This enriches
  the theory and is worth investigating.
\end{resenumerate}

\hide{%%% BEGIN hide
\noindent
From operad theory to logic and type theory:
\begin{resenumerate}\setcounter{CC}{0}
  \item
    Kontsevitch introduced the notion of a meager PROP (renamed
    as \raisebox{.75mm}{\scriptsize$\frac 1 2$}PROP by
    Markl~\cite{Markl}) sitting in between that of operad and
    polycategory.  Is there any use for these in logic and type theory?

  \item
    What about other generalisations, like the preshuffle algebras of
    Ronco~\cite{RoncoShuffleBialgebras} and the permutads of Ronco and
    Loday~\cite{Permutads}?  In particular, the former embodies the
    algebra needed in computational contexts with effects, where the
    notion of total composition does not make sense, and the notion of
    partial composition is subject to weaker axioms to those of
    multicategory.  This is relevant to our proposed investigations in
Sections\,(\ref{PolarisationParagraph}\,\&\,\ref{ProgrammingEffectsParagraph}).

  \item
    There is a rich variety of notions of operads, and notions of
    composition between them, many of which are motivated by
    combinatorics.  What can they offer to logic and type theory?  This
    question is not as vacuous as it may seem at first sight.  For instance,
    the shuffle and quasi-shuffle composition (see~\eg~\cite{AguiarMahajan})
    seem to be related to a type theory for non-commutative linear logic of
    Polakow and Pfenning~\cite{NatDedForIntNonCommLinLog}.

\hide{
  \item
    Foundations for a TTyacc (Type Theory yacc) \ldots this is related to what
    is a Type Theory? \ldots The feasibility of building a TTyacc will be
    investigated \ldots relevant to reflection in type theory \ldots extract a
    language from the mathematics
  }
\end{resenumerate}

\noindent
From logic and type theory to operad theory:
  \begin{resenumerate}\setcounter{CC}{3}
  \item
    The new Lie-algebraic view of multicategory composition mentioned
    above gives a rational reconstruction of an explicit characterisation of
    the pre-Lie operad by Chapoton and Livernet~\cite{ChapotonLivernet}.  The
    case of polycategory composition leads to a conjectured characterisation
    for the Lie-admissible operad.  A possible approach to proving it leads to
    the introduction of new interesting algebraic structures.

  \item\label{BellantoniCook}
    A combination of the \emph{linearity} dimension of
    Figure\,\ref{TypeTheoryResearchSpace} with the other dimensions (that take
    place in the traditional \emph{cartesian} setting, where all structural
    rules on contexts are allowed) leads to a notion of mixed
    operad~\cite{FioreMFPS}, whose associated composition operation is new
    to the mathematical theory.

    Mixed operads are the wiring of linear/cartesian languages, which since
    the work of Bellantoni and Cook~\cite{BellantoniCook} play a role in
    implicit complexity theory.  It would be interesting to develop this
    connection.  A main aim would be to introduce new mathematical structure
    in implicit complexity theory that would help with its problems.  This is
    relevant to our proposed investigations in
    Section\pref{SubstructuralTypeTheoryParagraph}.

  \item
    The algebraic structure on operads considered in mathematics is in
    the tradition of linear algebra (\ie~first-order).  However, the binding
    operators of logic and type theory transport to operads as a new kind of
    algebraic structure which thereby enriches the theory and is worth
    investigating.
  \end{resenumerate}
}%%% END hide 
}%%% END hide

\subsection{Models}
\label{Models}

\hidenote{\ldots}

\setcounter{paragraph}{0}
\paragraph{Equality and identity in dependent type theory.}
\label{IntensionalTypeTheoryParagraph}

We have briefly mentioned  intensional and extensional type theory in
Section\pref{DependentTypesParagraph}.  This section examines these two
flavours of type theory and presents a research programme geared to
investigate the main open problem in the area: 
\begin{resenumerate}\setcounter{CC}{0}
\item
  Reconcile Intensional and Extensional Type Theory in a computational
  framework.
\end{resenumerate}
Various aspects of what follows arose and are under discussion with Warren
(IAS School of Mathematics, Princeton).

Central to a type theory is its associated equational theory.  Equational
theories of dependent type theories may be specified in different (though
typically equivalent) forms (see~\eg~\cite{Adams}).  For the purpose of our
informal discussion here it will be enough to assume that they determine a
congruence relation
%\[
%  \Gamma \vdash M_1 = M_2: A
%\]
%for valid judgements $\Gamma\vdash M_i:A$ ($i=1,2$), 
henceforth referred to as judgmental equality.

In Pure Intensional Type Theory, judgemental equality is generated by $\beta$
rules of computation (recall
Section\pref{SectionATypeTheoryAndLogicParagraph}).
%
As for Extensional Type Theory, the terminology will be used here to refer to
extensions of Pure Intensional Type Theory with extensionality features.  In
their simplest form, these consist of incorporating $\eta$ rules
of extensionality (recall Section\pref{SectionATypeTheoryAndLogicParagraph})
into judgemental equality.  Such extensions will be referred to here as Weak
Extensional Type Theory.

Already for these basic kind of systems, 
%Pure Intensional and Weak Extensional Type Theory, 
our knowledge of the subject is limited. %poor.
%
For instance, %In particular, 
the concrete mathematical models (as opposed to syntactic ones) of Pure
Intensional Type Theory 
%with dependent sums and products 
that researchers work with (like groupoids, strict $\omega$-categories, and
simplicial sets) interpret dependent sums and dependent products as adjoints
and hence are necessarily models of Weak Extensional Type Theory.  A first
question arises:
\begin{resenumerate}\setcounter{CC}{1}
\item 
  Are there natural mathematical models of Pure Intensional Type Theory?
\end{resenumerate}
In examining this one should also consider:
\begin{resenumerate}\setcounter{CC}{2}
\item 
  What is a model of Pure Intensional Type Theory?
\end{resenumerate}
In this respect, we note that the current categorical formalisms
(see~\eg~\cite{Jacobs}) are not flexible enough (as they are either based on
adjoint interpretations or on ad hoc variations thereof).  On the other hand,
the approach of Dybjer on Internal Type Theory~\cite{DybjerITT} will provide
models within Cartmell's Generalised Algebraic Theories~\cite{Cartmell}.
Their relationship to developments in the Algebraic Type Theory of
Section\pref{AlgebraicTypeTheoryParagraph} will be investigated.

\hide{CwFs \vs~substitution algebras}

Conceptually, one may regard a type theory as a basic deduction system
extended with type constructors.  For instance, Martin-L\"of Type Theory
embodies dependent sums and dependent products, as well as Identity Types,
W~Types, and Universes~\cite{ProgMLTT}.  As such, a type theory is to be
thought not just as a single universe of discourse, but rather as a
variety of universes of discourse corresponding to the various possible
restrictions.  From this perspective, it is thus important to understand the
interaction of modularly extending type theories, central to which is the
notion of conservative extension.  Informally, an extension $T'$ of $T$ is
conservative if the restriction of the judgemental equality of $T'$ to $T$
coincides with the judgemental equality of $T$.

Conservativity has been studied for some simple type 
theories, %, see~\eg~\cite{Lafont,FioreRemarks}, 
but it has been overlooked for dependent type theories.  We propose here to:
\begin{resenumerate}\setcounter{CC}{3}
\item
  Develop a model-theoretic framework for establishing conservative
  extension results for Martin-L\"of Type Theory (and Pure Type Systems).
\end{resenumerate}
This will lead to sophisticated model constructions that will deepen our
understanding of the subject.  In particular because it was already the case
in proving the conservative extension of the Simply-Typed Lambda Calculus
extended with (non-dependent) extensional strong sums~\cite{FioreRemarks}.

The discussion of further notions of extensionality requires the consideration
of Identity Types
\[
  x:T,y:T\vdash\Id_T(x,y)\ \mbox{\small$\mathsf{type}$}
\]
that are meant to provide a notion of equality internal to the type theory.

In Strong Extensional Type Theory, one requires that Identity Types completely
internalise judgemental equality (see~\cite{MartinLofETT}).  Since for rich
type theories this leads to undecidable type checking, Pure Intensional Type
Theory has become the foundational core of proof assistants for constructive
mathematics; like
Automath, %~\cite{deBruijn}, 
Coq, %~\cite{?} 
and 
Agda. %~\cite{?}.  
However, as extensionality lays at the core of mathematics and its practice,
the quest for a computational framework in between Pure Intensional Type
Theory and Strong Extensional Type Theory continues.

In this direction, Martin-L\"of proposed a notion of Identity Type
(see~\eg~\cite{ProgMLTT}) which is given as an inductively generated
family~\cite{DybjerIF} with a reflexivity constructor.  These Identity Types
are currently receiving renewed special attention, stemming from connections
with homotopy theory and higher-dimensional category
theory~\cite{UnivalentProgramme}.  For them some fundamental matters are still
to be understood.  For instance, 
\hide{
whether the introduced intensional identity
%(an internal notion of equality) 
affects 
%(the external) 
judgemental equality.
Precisely,
}
\begin{resenumerate}\setcounter{CC}{4}
\item
  Does extending (Pure Intensional or Weak Extensional) Type Theory for
  dependent sums and dependent products with Identity Types produce a
  conservative extension?
\end{resenumerate}
Also, when seen from the viewpoint of Algebraic Type Theory:
\begin{resenumerate}\setcounter{CC}{5}
\item 
  What is the relationship between the algebraic theory of Identity Types and
  the algebraic theory of Weak Higher-Dimensional Categories?
\end{resenumerate}
This is to be investigated specially noting that the former crucially relies
on binding operators while this is not so for the latter.

A recent interesting approach to extensionality being intensively
investigated is the extension by Voevodsky of Pure Intensional Type Theory
with a so-called Univalence Axiom %~\cite{UnivalenceAxiom} 
motivated by homotopical foundations.  This axiom has been shown consistent
from the outset, but:
\begin{resenumerate}\setcounter{CC}{6}
\item 
  Is the extension of (Pure Intensional or Weak Extensional) Martin-L\"of Type
  Theory with the Univalence Axiom conservative?
\end{resenumerate}

It is an open problem, referred to as Voevodsky's Main Computational
Conjecture, as to how to give a constructive interpretation of the Univalence
Axiom.  The question arises as to whether one could circumvent this problem by
concentrating instead on a notable consequence of the axiom; namely, that it
somehow endows Identity Types with logical character, very roughly in that the
Identity Type constructor logically commutes with all the other constructors
up to equivalence.  Therefore we will rather aim to:
\begin{resenumerate}\setcounter{CC}{7}
\item 
  Design a constructive type theory with built-in Logical Identity Types.
\end{resenumerate}
In this context, we will:
\begin{resenumerate}\setcounter{CC}{8}
\item
  Investigate possible connections to Strachey's notion of parametric
  polymorphism~\cite{Strachey1967} in computer science as formalised by
  Reynolds~\cite{Reynolds} using logical relations.
\end{resenumerate}
Were these connections to materialise, it will open up a new flow of ideas
between hitherto disconnected fields.

\paragraph{Mathematical universes.}
\label{MethodologyMathematicalUniversesParagraph}

We speculate here on directions in relation to
Section\pref{MathematicalUniversesParagraph}, where we are interested in
building new models from old ones.  A main source of inspiration and
guidance for the development comes from Topos Theory~\cite{Elephant}.

Our proposal is to:
\begin{resenumerate}\setcounter{CC}{0}
\item\label{ConstructionsOnTypeTheories}
  Investigate presheaf, orthogonality, sheaf, glueing, and forcing
  constructions on type theories.
\end{resenumerate}
with the general goal to:
\begin{resenumerate}\setcounter{CC}{1}
\item
  Build a mathematical theory of constructions on type theories that
  produce new type theories from old ones, together with an interpretation
  (or compilation) of the latter into the former.
  
\item \label{MMUPAppItem}
  Implement the mathematics in proof assistants, and exercise it in
  applications.  
\end{resenumerate}

Initial work along this line has been done by Jaber, Tabareau, and
Sozeau~\cite{TypeTheoryWithForcing}, who formalised the presheaf
construction on a partial order in the Calculus of Constructions, and then
re-interpreted back the Calculus of Constructions in the internal presheaf
model.  However, not only a wide spectrum of other possible constructions
(as detailed in
item~\textbf{\ref{MethodologyMathematicalUniversesParagraph}}\thinspace\itemref{ConstructionsOnTypeTheories}
above) remains to be explored; but, even in this basic setting, work remains
to be done: notably to incorporate Inductive Types.

What we envisaged is also more general, in that we take the view that the
new type theory need not be of the same character as the old one.  Thus
there are two aspects to our proposed investigations: the mathematical one
of studying model constructions and the type-theoretic one of designing
internal languages.  To fix ideas, consider as an example that the
presheaf construction on a monoidal category endows the new universe of
discourse with monoidal closed structure (formally via Day's convolution
monoidal structure~\cite{Day}), and thereby introduces new linear structure.
This is important in many applications, for which the reader may
consult~\cite{FioreFossacs}, %, and complementary to our proposed
%investigations in Section\pref{SubstructuralTypeTheoryParagraph}.
that will be at the core of
item~\mbox{\textbf{\ref{MethodologyMathematicalUniversesParagraph}}\thinspace\textbf{\itemref{MMUPAppItem}}}
above.

The study of Cohen Forcing for type theory was stated as an open problem
by Beeson~\cite{BeesonBook}, who wrote: ``Forcing has yet to be worked out
directly for Martin-L\"of's system.''.  Work in this direction has only
recently begun.  Specifically, by Coquand and Jaber~\cite{CoquandNote} who
presented an interesting example.  Much remains to be explored both with
respect to the vast literature on set-theoretic forcing and, more
relevantly, with respect to Krivine's Classical
Realizability~\cite{KrivineRA}, an extension of forcing.

\paragraph{Polarised logic.}
\label{PolarisationParagraph}

We propose here a model-theoretic study of a rich variety of logical
systems encompassing aspects of resource management and computational
effects (recall Section\pref{ResourcesEffectsModalitiesParagraph}).  This
research is work under discussion with Curien and
Munch-Maccagnoni (Laboratoire~PPS, Universit\'e Paris Diderot - Paris~7).
A distinguishing novelty of our approach is that it will be pursued as
informed by the logical notion of polarisation.

The notion of explicit polarisation in logical systems was introduced by
Andreoli~\cite{Andreoli} in his study of proof search in Linear Logic, in
particular by focalisation.  According to it, logical connectives are
classified as either being positive or negative; with these two worlds
being dual to each other (categorically by adjunction).  Following
Andreoli's work, Girard understood the relevance of focalisation via
polarisation as a way to tame down the inherent non-determinism in
computation (by cut elimination) in classical logic; and, in this
direction, introduced the first explicitly polarised logic:
\LC~\cite{GirardLC}.  A crucial aspect of this system, is to make eager
(\viz~call-by-value) and lazy~(\viz,~call-by-name) modes of computation
explicit, allowing for their combination in a framework with eager and
lazy data structures.

Polarisation, though not recognised as such, is also present in the
Computational \mbox{$\lambda$-calculi} based on adjunction models as mentioned
in Section\pref{ResourcesEffectsModalitiesParagraph}.  This key
observation leads to a refinement of the
model~(\ref{ResourceEffectAdjunction}) as follows
\begin{equation}\label{LCBPV}
  \begin{minipage}{\columnwidth}
  \xymatrix@R=5pt{
    & & 
    \\
    \ar@{}[u]|-{\text{\scriptsize(negative)}}
    \cat N \ar@/^.75em/[dr] & \ar@{}[rd]|-{\sevdash} \ar@{}[dl]|-{\swvdash} &
    \ar@/^.75em/[dl] \cat M \ar@{}[u]|-{\text{\scriptsize(multiplicative)}}\\
    & \ar@/^.75em/[lu] \cat P \ar@/^.75em/[ur]
    \ar@{}[d]|-{\txt{\scriptsize(positive/linear)}}& 
    \\ & & 
  }
\end{minipage}
\end{equation}
with comonadic resource structure given by the adjunction on the right and
monadic effect structure given by the adjunction on the left.

When the effect structure is trivial, the above restricts to models of Linear
Logic~\cite{MelliesCMLL}; while, when the resource structure is trivial, one
recovers the models of Call-By-Push-Value~\cite{LevyCBPV}.  One is thus led to
the following programme.
\begin{resenumerate}\setcounter{CC}{0}
\item\label{ItemOne}
  Devise a sound and complete logical system for the model
  theory~(\ref{LCBPV}), and relate it to existing polarised
  systems~\cite{Munch,CurienMunch}.
\item
  A first model-theoretic result shows that the effect structure on $\cat
  P$ lifts to one on $\cat M$.  Show that this corresponds to an encoding
  of Call-By-Push-Value in the devised logical system.
\item
  Develop a generic syntactic theory able to incorporate concrete
  computational effects into the devised calculus.  Two research possibilities
  are: (i)~considering algebraic theories of effects following the work of
  Plotkin and Power~\cite{PlotkinPowerAlgOpsAndGenEffs}, and
  (ii)~revisiting and refining Filinski's result on representing
  monads~\cite{Filinski} that reduces general monadic effects to the
  storage and escape effects.  Either of these approaches will necessarily
  have to overcome serious shortcomings.  For instance, on the one hand,
  there is as yet no general operational theory of algebraic effects; and,
  on the other, a logical system for storage is not yet in place.  These
  problems will be investigated.
\item\label{ItemFour}
  The previous development will provide foundational metalanguages.  The
  next step in the programme is to promote them to programming languages,
  where the effects are implicit, by putting them in so-called
  direct-style.  The kind of model theory involved in this development
  will be discussed below.
\end{resenumerate}

The model theory of~(\ref{LCBPV}) can be further specialised to
models with enough internal structure so as to extend the picture as
follows
\begin{equation}\label{LEEC}
  \begin{minipage}{\columnwidth}
  \xymatrix@R=5pt{
    & \ar@/^.75em/[dl] \ar@{}[dl]|-\sevdash \cat N \ar@/^.75em/[dr] &
    \ar@{}[rd]|-{\sevdash} \ar@{}[dl]|-{\swvdash} & \ar@/^.75em/[dl] \cat
    M \\
    \ar@/^.75em/[ur] \cat P^\op & & \ar@/^.75em/[lu] \cat P \ar@/^.75em/[ur] & 
  }\end{minipage}
\end{equation}
where the new adjunction between the negatives and the opposite dual of
the positives enriches the models with control structure of the
linear %~\cite{ThieleckeEtAl} 
and/or delimited continuation %~\cite{PeytonJones?}
kind, though this is to be investigated.

The model theory~(\ref{LEEC}) refines that of two recent developments: the
aforementioned Enriched Effect Calculus, which amounts to the case in
which the linear structure collapses to cartesian one; and the Tensor
Logic of Melli\`es and Tabareau~\cite{TensorLogic}, where the effect
structure is collapsed.  Our programme for~(\ref{LCBPV}),
\ie~items~\mbox{\textbf{\ref{PolarisationParagraph}}\thinspace\textbf{[\ref{ItemOne}--\ref{ItemFour}]}},
will be then also pursued for~(\ref{LEEC}) drawing connections with these
works.

Let us now return to the distinction between metalanguages and programming
languages briefly mentioned in
item~\textbf{\ref{PolarisationParagraph}}\thinspace\itemref{ItemFour} above.
Model theoretically, this can be roughly seen as follows: whereas the
metalanguage is an internal language for co/monadic structure; the programming
language is the internal language for the derived structure of co/free
algebras for the co/monad---typically defined by the categorical co/Kleisli
construction~\cite{MacLane}.  In particular, the Kleisli category of a
computational monad provides a model of call-by-value~\cite{MoggiLambdaC};
while the coKleisli category of a linear exponential comonad provides one
of call-by-name~(\ie,~a cartesian closed category)~\cite{Seely}.

The programming languages corresponding to the metalanguages of the
polarised models~(\ref{LCBPV}) and~(\ref{LEEC}) should be the internal
languages for an analogous, but more intricate, construction relative to
an adjunction rather than a co/monad.  In this context, the following will
be investigated.
\begin{resenumerate}\setcounter{CC}{4}
\item
  What is the mathematical structure of such construction?  
  %
  The question is challenging.  For instance, we envisaged models of {\LC} to
  arise in this manner and, because of the interaction with call-by-value and
  call-by-name modes of computation, the structure should be more general than
  that of a category.

\item
  What would then be the mathematical theory of universal constructions for
  these structures giving rise to logical connectives?  
  %
  Note again from considering {\LC} that this will be subtle, requiring
  constructions of objects with mixed polarities.

\item
  What can the resulting direct-style calculi contribute to programming?
  For discussion on this see Section\pref{ProgrammingEffectsParagraph}.
\end{resenumerate}

\hidenote{polarised dialectica for program extraction motivated by
  polarised dialectica category construction}

\hidenote{Investigate relationship to Chuck Liang \& Dale Miller's work}

\hidenote{Investigate relationship to Oleg Kiselyov and Chung-chieh Shan's
  substructural type system for delimited control}

\paragraph{Modal logics.}
\label{ModalLogicsParagraph}

Recall from Section\pref{ResourcesEffectsModalitiesParagraph} that resource
comonads and effect monads are modalities.  Polarisation in the context of
modal logics has not been considered yet.  We will thus complement the
previous section with the 
\begin{resenumerate}\setcounter{CC}{0}
\item
  Investigation of Polarised Modal Logics.
\end{resenumerate}

Our overall goal here is to
\begin{resenumerate}\setcounter{CC}{1}
\item
  Develop the Propositions-as-Types correspondence for modal logics.
\end{resenumerate}
Here, we are specially interested in grounding it through categorical models
(see Figure\,\ref{ResearchAreas}), that have been scarcely investigated.

Looking ahead in connection to the proposed research in
Section\pref{MetaprogrammingParagraph}, we will be particularly interested
in studying: 
%
Borghuis' Modal Pure Type Systems~\cite{ModalPTS};
%
Artemov's Logic of Proofs~\cite{ArtemovLP} and related systems;
%(\eg~\cite{AltArtemov,ArtemovIemhoff}); 
%Daniyar S Shamkanov: Strong Normalization and Confluence for Reflexive
%  Combinatory Logic 
%
Mendler's Multimodal CK~\cite{MendlerMMCK};
and
%
Park and Im's Calculus $S_\Delta$~\cite{ParkIm}.

\subsection{Calculi}
\label{CalculiSubsection}

\hidenote{\ldots}

\setcounter{paragraph}{0}
\paragraph*{Categorical Type Theory.}

The evolution of term and type structure in type theories from its origin
to the current state can be succinctly represented as follows
\begin{center}\begin{tabular}{|c||c|c|c|c|}\hline
  & \small Term Structure & \small Type Structure
  \\[.25mm] \hline\hline
  \small Equational Logic & \small Algebraic & \small ---
  \\[.25mm] \hline
  \txt{\\ \raisebox{1mm}{\small Simple Type}\\\raisebox{1mm}{\small Theory}} &
  \raisebox{-.75mm}{\small Binding} & \raisebox{-.75mm}{\small Algebraic}
  \\[.25mm] \hline
  \txt{\\ \raisebox{.5mm}{\small Polymorphic}\\ \raisebox{1mm}{\small Type
      Theory}} & 
  \raisebox{-.75mm}{\small Binding} &
  \raisebox{-.75mm}{\small Binding} 
  \\[1.5mm] \hline
  \txt{\\ \raisebox{.5mm}{\small Dependent}\\\raisebox{1mm}{\small Type Theory}}
  & 
  \multicolumn{2}{c|}{\raisebox{-1mm}{\small Binding}}
  \\[.25mm] \hline
\end{tabular}\end{center}
Surprisingly, this has not changed since de~Bruijn's Automath in the late
1960s.  We firmly believe that further evolution will come from mathematical
input and, in this direction, propose two lines of research for type theories
as formal languages of categorical structures.  
%These are presented in the two sections below.

\paragraph{Generalised Type Theory.}
\label{GeneralisedTypeTheoryParagraph}

We present first work discussed with Gambino (DMI, Universit\`a degli Studi di
Palermo) and Hyland (DPMMS, University of Cambridge) with whom the development
will be undertaken.  

Section\pref{MethodologyMathematicalUniversesParagraph} considers presheaf
categories in the large as mathematical universes of discourse (or gros
toposes).  There is however an alternative view of them in the small as
spaces (or petit toposes) with morphisms between them given by linear (or
cocontinuous) functors, resulting in the bicategory of profunctors (or
distributors)~\cite{Benabou}.  Lawvere~\cite{LawvereMetric} considered
these structures in the general context of enriched category theory and
synthesised a Generalised Logical Calculus out of them.  The enrichments
give rise to different interpretations, \eg~in preorders, categories, and
metric spaces.

The general goal of our research here is to:
\begin{resenumerate}\setcounter{CC}{0}
\item\label{GeneralisedTypeTheoryItem}
   Develop a type theory providing a Propositions-as-Types interpretation of
   the Generalised Logical Calculus.
\end{resenumerate}
This is to be investigated in two stages, respectively corresponding to
enrichment over cartesian closed categories and over symmetric monoidal
closed categories.  In the first case, one roughly has the following
translation table between the Generalised Logical Calculus and Predicate
Logic.
\begin{center}\begin{tabular}{|c|c|}\hline
  sum & disjunction\\ 
  coend & existential quantification\\ 
  product & conjunction\\ 
  end & universal quantification\\ 
  exponential & implication\\ 
  hom & equality\\ 
  presheaf application & predicate membership\\
  \hline
\end{tabular}\end{center}
The second stage introduces linearity. 

Success in this endeavour will yield a system where the mathematical
calculations done with profunctors (\eg~in the context of the coherence of
profunctor composition~\cite{Benabou}, the mathematical theory of
substitution~\cite{FioreFossacs}, generalised species of
structures~\cite{Species}, and generalised polynomial
functors~\cite{FioreICALP}) can be established formally.

Intuitively, the coend and end constructions mentioned in the table above are
quotients of dependent sums under a compatibility condition and restrictions
of dependent products under a parametricity condition.  Taking this view
seriously, we will aim to:
\begin{resenumerate}\setcounter{CC}{1}
\item
  Extend
  item~\textbf{\ref{GeneralisedTypeTheoryParagraph}}\thinspace\itemref{GeneralisedTypeTheoryItem}
  to a fully-fledged, possibly higher-dimensional, dependent type theory.  
\end{resenumerate}

Finally, in an intriguing complementary direction,
Fiore~\cite{FioreFossacs} has suggested a graphical reading of the
Generalised Logical Calculus that is surprisingly close to formalisms for
categorical graphical languages~\cite{Selinger}, and
proof~\cite{GirardLinearLogic}/interaction~\cite{Lafont} nets.  This
aspect deserves investigation.

\paragraph{Directed Type Theory.}
\label{DirectedTypeTheoryParagraph}

Homotopy Type Theory (under the acronym HoTT, and the slogan `types are
spaces') is the name being used for the body of work at the boundary between
Homotopy Theory and Dependent Type Theory through Identity Types.

The question arises:
\begin{resenumerate}\setcounter{CC}{0}
\item
  Is there a notion of Directed Type connecting Higher-Dimensional Category
  Theory and Dependent Type Theory on which to establish a body of work on
  Directed Type Theory (DiTT)?
\end{resenumerate}
In this context, while an Identity Type
\[
  \Id_T(x,y)
\]
is intended to establish the intensional equality of elements of a type $T$ so
that $\Id_T(x,y)$ and $\Id_T(y,x)$ are equivalent; the intuitive idea behind a
Directed Type 
\[
  \Di_T(x,y)
\]
would be to classify the possible ways in which an element of a type $T$ may
evolve to another one in a possibly irreversible manner; as it happens, for
instance, in computation.  Thereby leading to not necessarily equivalent types
$\Di_T(x,y)$ and $\Di_T(y,x)$, and to the DiTT slogan `types are directed
spaces'.

An analysis of the elimination rule for Identity Types reveals that there
are at least two sources leading to their inherent reversible character.
We mention them below as possible directions of research for investigating
Directed Types.
\begin{resenumerate}\setcounter{CC}{1}
\item
  The use of contexts that allow the commutativity of two consecutive
  variables of the same type, which suggests moving on to a non-commutative
  setting (as \eg~in non-commutative linear logic).
  %~\cite{NatDedForIntNonCommLinLog}).  
  %(This is related to the considerations in
  %Section\pref{SubstructuralTypeTheoryParagraph} below.)

\item\label{TypeDualityItem}
  The lack of a type duality, a property that seems to be inherent to
  directionality, whereby every type $T$ has an associated dual type $T^\circ$
  for which $\Di_T(x,y)$ and $\Di_{T^\circ}(y,x)$ are equivalent.  
\end{resenumerate}

The previous item is of course related to the considerations of the
previous section.  Pursuing this further, it would be natural to
generalise from presheaves, now regarded as discrete fibrations, to the
general notion of fibration~\cite{SGA1} and aim to:
\begin{resenumerate}\setcounter{CC}{3}
\item
  Develop a type theory modelled on Grothendieck fibrations.
\end{resenumerate}

As a final point, we note that yet another approach would be to:
\begin{resenumerate}\setcounter{CC}{4}
\item
  Investigate a directed-cylinder construction, intuitively adding a new base
  point to a type that canonically evolves to every element.  
\end{resenumerate}
This would generalise the lifting construction of Domain
Theory~\cite{LiftingKZ} and possibly establish connections with the
Complete Cuboidal Sets model of Axiomatic Domain
Theory~\cite{CompleteCuboidalSets} opening new directions.  

\hide{%%% BEGIN hide
\paragraph{Substructural Type Theory.}
\label{SubstructuralTypeTheoryParagraph}

The work of Bellantoni and Cook~\cite{BellantoniCook}
%, after Leivant~\cite{Leivant}, 
on implicit computational complexity gave a characterisation of the
polynomial-time computable functions on finite strings by means of a
safe-recursion scheme that restricted primitive recursion.  A crucial
ingredient of their approach is to divide function arguments into normal and
safe ones.  Recursion is allowed on normal arguments but not on safe ones.  In
the terminology that we have been using, normal arguments are linear while
safe ones are cartesian.  In fact, the notion of safe composition
coincides with that for mixed linear/cartesian operads (see
item~\textbf{\ref{WiringStructureParagraph}}\thinspace\itemref{BellantoniCook}).
%on page~\pageref{BellantoniCook}).  
Since then, the role of resource management in taming complexity has been
prominently recognised, and a big body of work on implicit computational
complexity (\eg~on soft, bounded, light, and linear logics and types) has
followed.
% Baillot, Dal Lago, Hofmann, Lafont

Recently, Beckmann, Buss, and Friedman~\cite{BeckmannBussFriedman} 
%and later on Arai~\cite{Arai} 
have extended the safe-recursion scheme to arbitrary set functions.
Changing viewpoint form the mathematical universe of sets to that of
types, one is lead to:
\begin{resenumerate}\setcounter{CC}{0}
\item
  Extend dependent type theory to the mixed linear/cartesian setting.
\end{resenumerate}
In particular, in what concerns to taming computational complexity:
\begin{resenumerate}\setcounter{CC}{1}
\item
  Investigate a safe version of the Induction-Recursion scheme of
  Dybjer~\cite{DybjerIR}.  
\end{resenumerate}
This development should be also accompanied by a model theory, possibly based
on 
\begin{resenumerate}\setcounter{CC}{2}
\item 
  Realisability models relative to the Linear Combinatory Algebras of
  Abramsky, Haghverdi, and Scott~\cite{AHS}.
\end{resenumerate}
}%%% END hide

\subsection{Programming}
\label{Programming}

This track of the proposal is concerned with the design and implementation
of programming languages from first principles and pragmatics, followed up
by their subsequent test, use, and distribution.

Three directions for this research are presented.  These will be
considered as units in their own right, and also in relation to each other.

\setcounter{paragraph}{0}
\paragraph{Indexed programming.}
\label{IndexedProgrammingParagraph}

This section outlines work under discussion with Ahn and Sheard
(Department of Computer Science, Portland State University) on indexed
programming (recall Section\pref{IndexedProgrammingIntro}).

The main problem to be addressed here is to:
\begin{resenumerate}\setcounter{CC}{0}
\item
  Design a language for programming indexed data structures that will
  inform the design of the next generation of programming languages.
\end{resenumerate}
The task can be approached from two different viewpoints.  In a top-down
fashion, one may turn dependent type theories into programming languages;
conversely, in a bottom-up fashion, one may extend functional programming
languages with indexing structure.  These two approaches pull the
programming language design into two opposite directions; as pictorially
presented in Figure\,\ref{DesignSpaceFigure}, 
\vfigspace{-2mm}\begin{figure}[h]
\caption{Indexed-programming design space}
\vspace*{-2mm}
\[
\null\hspace*{-2mm}
\txt{\small Functional\\ \small Programming\\ \small Language}
\hspace*{3mm}\xy
0;/r.75pc/:+(0,.25)
,{\hloop-\hcross\hcross\hcross\hcross\hcross\hcross\hcross\hloop}
\endxy\null\hspace*{3mm}
\txt{\small Constructive\\ \small Proof\\ \small Assistant}
\vspace*{-2mm}
\]
\hdashrule[1ex]{\columnwidth}{1pt}{2.5pt}
\begin{tabular}{lcl}
%\scriptsize Programming && \scritpsize Proof\\ 
\small
ML, %~\cite{SML}, 
Haskell %~\cite{Haskell} 
&\hspace*{25mm}&
\small
Coq, %~\cite{Coq}, 
Agda %~\cite{Agda}
\\[-.5mm]
\small
\SystemFomega %~\cite{GirardSystemF} 
&& 
\small
MLTT, %~\cite{MLTT}, 
ICC %~\cite{ICC}
\\[-.5mm]
\small
Polymorphism %~\cite{Strachey1967} 
&& 
\small
Type dependency %~\cite{deBruijn} 
\\[-.5mm]
\small
Impredicativity && 
\small
Universes %~\cite{Universes}
\\[-.5mm]
\small
Type inference %~\cite{Milner1978,Wells} 
&&
\small
Type checking
\\[-.5mm]
\small
Recursive types %~\cite{FPC}
&& 
\small
Inductive types~ %\cite{InductiveFamilies}
\\[-.5mm]
\small
Equality types %~\cite{GHC-IFP2012} 
&& 
\small
Identity types %~\cite{IdTypes}
\end{tabular}
\label{DesignSpaceFigure}
\end{figure}\vfigspace{-2mm}
where the main features inherent to each approach are also listed.

Most of the work in this area has concentrated on exploring the top-down
approach (see~\eg~\cite{Cayenne} and~\cite{Epigram}).  Here we will pursue
research on the bottom-up approach, which has not been explored
systematically.  A strong pragmatic reason for this is that our uppermost
interest is in a language targeted to programmers.  This does not mean that we
are not interested in the activity of proving as in constructive proof
assistants.  However, our main goal, under the Propositions-as-Types paradigm,
is: 
\begin{resenumerate}\setcounter{CC}{1}
\item
  To prove properties as a by-product of programming; rather than to extract
  programs from proving properties.
\end{resenumerate}
For this to be meaningful, both in theory and practice, the language will
need to guarantee logical consistency in a setting that naturally allows
programming strong invariants for rich indexing type structure.  For
instance, so that the code of a compiler guarantees its correctness.

In the above direction, we are investigating:
\begin{resenumerate}\setcounter{CC}{2}
\item
  {\SystemFi}: an extension of {\SystemFomega} with static type-indexing
  structure.
\end{resenumerate}
{\SystemFi} is to serve as the mathematical foundation for designing, and
giving operational semantics to, our programming language.  We stress here
that indices are static, \ie~determined at compile time.  This is the main
feature pulling us away from traditional dependently-typed formalisms.

The logical consistency of {\SystemFi} amounts to establishing its strong
normalisation; while as an extension of {\SystemFomega} it will embody
rich type structure.  In this setting, following Ahn and
Sheard~\cite{AhnSheard11}, we will:
\begin{resenumerate}\setcounter{CC}{3}
\item
  Study programming primitives corresponding to a variety of induction
  proof principles.
\end{resenumerate}
These are to be incorporated in the programming language design.  Here, it
is interesting to note that the scheme provided by Mendler's
iterator~\cite{MendlerIter}, that crucially takes advantage of
parametric polymorphism, allows for terminating iteration over recursively
defined datatypes of possibly mixed variance.  Thus, going beyond the
typical inductive types of type theory stemming from Dybjer's Inductive
Families~\cite{DybjerIF}.  It would be interesting to establish a formal
connection between these approaches so that they can inform each other;
in particular in the context of termination checking. %~\cite{AbelFoetus}. 

\hide{
Dually, we will also:
\begin{resenumerate}\setcounter{CC}{4}
\item
  Develop programming primitives corresponding to coinduction proof
  principles.
\end{resenumerate}
These have received less attention.
}

The above design considerations are to be shaped by the following maxim:
\begin{resenumerate}\setcounter{CC}{4}%{5}
\item
  Design language constructs with minimal type annotation supporting
  maximal type inference.
\end{resenumerate}
Indeed, this is the main open problem in dependently-typed programming language
theory.

Finally, let us mention an intriguing possibility suggested by the
model-theoretic considerations of
Section\pref{MethodologyMathematicalUniversesParagraph}.  The indexing
structure in dependent type theory is discrete.  In many applications,
however, one is interested in indexed types that furthermore relate
structure on indices to structure on the associated indexed family of types.
The question arises as to how to:
\begin{resenumerate}\setcounter{CC}{5}%{6}
\item
  Build idioms or language abstractions for defining and then programming
  with indexed types equipped with internal varying structure.
\end{resenumerate}
A specific motivation for this comes from the possibility of directly
programming, as opposed to having to code, presheaf structure (\eg~as is
needed in Normalisation by Evaluation~\cite{FiorePPDP}).

\paragraph{Effects.}
\label{ProgrammingEffectsParagraph}

Our main aim here is to consider the model-theoretic investigations of
Section\pref{PolarisationParagraph} from a proof-theoretic viewpoint and
percolate this down to programming language theory.

A first main novelty in our approach is that the proof theory to be
considered is based on sequent calculus, rather than the traditional line
followed so far in programming language foundations based on sequent-style
natural deduction.

The proof-theoretic formalism that we will be adhering to is {\SystemL},
as recalled in Section\pref{SequentCalculiParagraph}.  A first main
question that arises is:
\begin{resenumerate}\setcounter{CC}{0}
\item 
  What is the programming paradigm stemming from sequent calculi in
  general, and {\SystemL} in particular?
\end{resenumerate}
For instance, 
\begin{resenumerate}\setcounter{CC}{1}
\item
  Does the inherent symmetry of {\SystemL} lead to a new programming
  style?

\item
  How can the close connection between {\SystemL} and abstract machines be
  exploited in programming?
\end{resenumerate}
These investigations will also require mathematical principles to be
developed.  Specifically, we will 
\begin{resenumerate}\setcounter{CC}{3}
\item
  Establish the Propositions-as-Types correspondence for sequent calculi
  through Category Theory (see Figure\,\ref{ResearchAreas}) relating
  {\SystemL} to adjoints.
\end{resenumerate}

As for polarisation, we believe that programmers will be able to intuitively
assimilate the eager {\vs}~lazy modes of computation and data structures
underlying it.  But, pragmatically,
\begin{resenumerate}\setcounter{CC}{4}
\item
  How will a programmer be able to easily code polarisation in a
  programming language?
\end{resenumerate}

Once the above is sorted, one can consider experimenting with programming
languages based on the various systems of
Section\pref{PolarisationParagraph} further enriched with computational
effects.  To this end, there is a large body of theoretical work on
effects that needs to be examined, evaluated, and reconsidered.  An
interesting stepping stone here is the recent work of Bauer and
Pretnar~\cite{BauerPretnar} on the experimental programming language Eff.  

Eff is based on the algebraic theory of effects and handlers of Plotkin
\etal~\cite{PlotkinPowerAlgOpsAndGenEffs,PlotkinPretnar};
but, for the pragmatics of supporting seemingly non-algebraic control
operators, it goes beyond algebraic models.  This is indeed so with
respect to (first-order) algebraic theories.  Interestingly, as it has
transpired in conversation between Fiore and Staton, the consideration of
the richer Second-Order Algebraic Theories of Fiore
\etal~\cite{FioreHur,FioreMahmoud} seems to enrich the class of
specifiable effects to incorporate aspects of control.  Thus, our research
into this topic will also
\begin{resenumerate}\setcounter{CC}{5}
\item
  Investigate Second-Order (and, when they are developed, Polymorphic)
  Algebraic Theories as a mathematical basis for an algebraic theory of
  effects encompassing control.
\end{resenumerate}
Targeted goals here will be program logics for effects and control, and formal
correctness proofs of (continuation and/or abstract machine) implementations
of programming languages with effects.

\hidenote{Dependent CBPV}

\paragraph{Metaprogramming.}
\label{MetaprogrammingParagraph}
{\bf feel free to change any of this -- Tim}

In a metaprogramming system there are two languages of interest. The
object-language which is the object of study and the meta-language which
manipulates object programs as data. The purpose of the meta-language is
to define algorithms for the purpose of constructing or analyzing
object-programs. Metaprograms play a role in many kinds of systems

{\bf Generic programming.}
One style of generic programming defines a datastucture that reflects how
users define new types. For example, a value of type \verb+Def Tree+
describes how a new \verb+Tree+ datatype is defined. A generic program then
analyzes this structure of \verb+Def+ types to write code parametric over
any user defined type. For example a generic equality might have type
\verb+Def t -> t -> t -> Bool+ and a generic marshalling function
might have type \verb+Def t -> t -> List Bit+. 

{\bf Reflection.} One kind of reflection is where the object-language and the meta-language
coincide

\note{TDPE with sums: reflection with delimited control operators}

\note{System $F_\omega^*$: Tillmann Rendel, Klaus Ostermann, Christian
  Hofer (2009),Typed Self-Representation}

\hidenote{\SystemL internalisation of stacks?}

\hidenote{ssreflect?}


\section{\underline{Resources}}
\label{ResourcesSection}

\paragraph*{Team.}

During the 2011-12 academic year, I was on sabbatical leave.  Two important
events for this grant proposal happened then as follows.  

On the one hand, I was awarded a \emph{Research in Paris} grant from the
\emph{programme d'accueil des chercheurs \'etrangers de la Ville de Paris}
to visit Laboratoire PPS at Universit\'e Paris Diderot - Paris~7 for three
months.  There, I mainly interacted with Pierre-Louis Curien and his PhD
student Guillaume Munch-Maccagnoni, establishing the Mathematical Logic
axis of Figure\,\ref{ercTeam}.

On the other hand, Tim Sheard (Department of Computer Science,
Portland State University), who was also on sabbatical leave during the
2011-12 academic year, visited Microsoft Research Cambridge during
October--December 2011 together with his PhD student Ki Yung Ahn who
instead visited the Computer Laboratory (University of Cambridge).  The
Programming Theory axis of Figure\,\ref{ercTeam} was established then.

To complete the team, for the Type Theory axis of Figure\,\ref{ercTeam} it
was most natural to recruit Peter Dybjer (Department of Computer Science
and Engineering, Chalmers University of Technology) and Nicola Gambino
(Dipartimento di Matematica e Informatica, Universit\`a degli Studi di
Palermo).  
%held a \ldots at DPMMS, University of Cambridge, during during
%????--???? and we had then collaborated on~\cite{Species}.  
%Since then, we have also kept in contact; most recently regarding our
%common interest on the topic of dependent polynomial
%functors~\cite{GambinoHylad,GambinoKock,FioreICALP} which is also
%relevant to topics of the proposal.  

\note{Makoto --- Marco --- Martin --- Michael}

\paragraph*{Principal investigator.}

As Principal Investigator, \textbf{\em Marcelo Fiore} would be concerned with
all areas of the project and head the {\erc} team, comprising Senior Visiting
Researchers and Research Associates.  
\begin{myitemize}
\item
For the Principal Investigator, salary is requested for eight months per
year for the five years of the duration of the project.  
\end{myitemize}
The reduction acknowledges that the Principal Investigator will concentrate on the project while continuing with some teaching to attract new PhD students and still have some departmental administration duties.

\paragraph*{Senior Visiting Researchers.}

The Senior Visiting Researchers Pierre-Louis Curien, Peter Dybjer, and Tim
Sheard are renown researchers in their respective field of expertise.  
\begin{myitemize}
\item
For the Senior Visiting Researchers, funding is requested for annual one-month
research visits to the Computer Laboratory (University of Cambridge) for the
five years of the duration of the project.  
\end{myitemize}
Our collaboration will however go beyond each visit; continuing by email,
conference calls, and/or visits to their respective sites. 
%(for which travel funding is also requested).  
Their academic credentials together with the specific areas of the project
to which they will be engaged follow.

\smallskip\noindent
\textbf{\em Pierre-Louis Curien}

\begin{tabular}{l|l}
Section & Keyword
\\ \hline
%(\ref{WiringStructureParagraph}) & operads
\\
(\ref{IntensionalTypeTheoryParagraph}) & dependent types, parametricity
\\
(\ref{PolarisationParagraph}) & polarisation
\\
(\ref{ModalLogicsParagraph}) & Linear Logic
\\
(\ref{ProgrammingEffectsParagraph}) & \SystemL
\\
\end{tabular}

\smallskip\noindent
\textbf{\em Peter Dybjer}

\begin{tabular}{l|l}
Section & Keyword
\\\hline
(\ref{AlgebraicTypeTheoryParagraph}) & Internal Type Theory
\\
(\ref{IntensionalTypeTheoryParagraph}) & Dependent Type Theory
\\
(\ref{MethodologyMathematicalUniversesParagraph}) & NbE, glueing
\\
%(\ref{SubstructuralTypeTheoryParagraph}) & induction-recursion
\\
(\ref{IndexedProgrammingParagraph}) & Agda
\\
\end{tabular}

\smallskip\noindent
\textbf{\em Tim Sheard}

\begin{tabular}{l|l}
Section & Keyword
\\\hline
(\ref{ModalLogicsParagraph}) & MetaML
\\
%(\ref{SubstructuralTypeTheoryParagraph}) & iterators
\\
(\ref{IndexedProgrammingParagraph}) & Nax, Omega
\\
(\ref{ProgrammingEffectsParagraph}) & PL design \& implementation
\\
(\ref{MetaprogrammingParagraph}) & MetaML
\\
\end{tabular}

\paragraph*{Research Associates.}

The Research Associates Ki Yung Ahn, Nicola Gambino, and Guillaume
Munch-Maccagnoni will be based at the Computer Laboratory (University of
Cambridge), working in the project full time.  Ahn and Munch-Maccagnoni
are currently finishing their PhD dissertations.  Nicola Gambino is
already an established researcher.  All would be ready to join the project
as it starts.  
\begin{myitemize}
\item 
  Salary is requested to employ each Research Associate full-time for the five
  years of duration of the project.
\end{myitemize}

\smallskip\noindent
\textbf{\em Ki Yung Ahn}
\marginpar{\color{red}REVISE}
is a PhD candidate at the Department of Computer Science,
Portland State University. He is currently writing up his dissertation on Nax,
a language designed to support indexed programming from a bottom-up fashion 
(Section (\ref{IndexedProgrammingParagraph})) but also aim for logical
consistency. His recent work, supported by U.S. National Science Foundation
grant 0910500, include Mendler-style recursion schemes~\cite{AhnSheard11},
indexed programming in Nax (awaiting review process to appear in IFL'12),
and \SystemFi\ (to be published).
Discussions on \SystemFi, which is a theoretical foundation of Nax,
started during the Cambridge visit, October-December 2011.

His research interest is to investigate how to incorporate resources and effects
into indexed programming (Section (\ref{ProgrammingEffectsParagraph})),
especially in the context of extending the Nax language, hoping to promote
indexed programming into the real-world software development practices.

He has also been actively contributing to the Haskell community.
His contributions include open source Haskell libraries (yices, logic-TPTP),
a publication in the Haskell symposium~\cite{AhnSheard08}, and
a Korean translation of Hutton's Haskell programming textbook \cite{Hutton07}.
The prototype implementations of \SystemFi\ and Nax are also developed
in Haskell by Ahn and Sheard.

%% \begin{tabular}{l|l}
%% Section & Keyword
%% \\\hline
%% %(\ref{SubstructuralTypeTheoryParagraph}) & iterators
%% \\
%% (\ref{IndexedProgrammingParagraph}) & Nax
%% \\
%% (\ref{ProgrammingEffectsParagraph}) & Haskell
%% \\
%% \end{tabular}

\smallskip\noindent
\textbf{\em Nicola Gambino}
%
%\begin{tabular}{l|l}
%Section & Keyword
%\\\hline
%(\ref{WiringStructureParagraph}) & generalised species
%\\
%(\ref{IntensionalTypeTheoryParagraph}) & HoTT
%\\
%(\ref{MethodologyMathematicalUniversesParagraph}) & sheaves, Algebraic Set
%Theory
%\\
%(\ref{GeneralisedTypeTheoryParagraph}) & generalised species
%\\
%(\ref{DirectedTypeTheoryParagraph}) & HoTT
%\\
%\end{tabular}
%
is University Researcher in Mathematical Logic at the University of Palermo
since December 2008.  He has a consistent record of publications in leading
journals in theoretical computer science and mathematics, and is frequently
invited to speak at international conferences.  
In particular, he has been one of the plenary invited speakers at the 2010
Logic Colloquium.  
He has held visiting positions at several research institutions. 
%, including the Institut Mittag-Leffler (Stockholm), the Fields Institute
%(Toronto), and the Centre de Recerca Matem\`atica (Barcelona).  
The most recent one, as %He has also been 
a member of the IAS School of Mathematics (Princeton) in the Autumn of 2011.
There he worked in contact with Voevodsky %and Awodey 
on Homotopy Type Theory, one of the areas of the proposed research (see
Section\pref{IntensionalTypeTheoryParagraph}). 
%He is currently on the editorial board of the journal Mathematical Structures
%in Computer Science.

Dr Gambino has expertise and experience in type theory and category theory,
two of the research areas of the proposal (see Figure\,\ref{ResearchAreas}). 
%, which he gained working under the
%mentorship of some of the leading scholars in these areas, including Aczel
%(The School of Computer Science, University of Manchester), Hyland (DPMMS,
%University of Cambridge), and Joyal (D\'epartement de math\'ematiques,
%UQ\`AM).  
%In joint research with Garner (Department of Computing, Macquarie University)
%he obtained some of the first results relating precisely type theory and
%homotopy theory.  
During his EPSRC Postdoctoral Fellowship at the DPMMS, University of
Cambridge, he collaborated successfully with the Principal Investigator on a
project that forms part of the proposed research in
Section\pref{GeneralisedTypeTheoryParagraph}.

\smallskip\noindent
\textbf{\em Guillaume Munch-Maccagnoni} 
%
%\begin{tabular}{l|l}
%Section & Keyword
%\\\hline
%(\ref{PolarisationParagraph}) & polarisation
%\\
%(\ref{ModalLogicsParagraph}) & Linear Logic
%\\
%(\ref{ProgrammingEffectsParagraph}) & \SystemL, delimited control
%\\
%\end{tabular}
%
is a PhD student at Laboratoire PPS, Universit\'e Paris Diderot - Paris~7.
He is currently writing up his dissertation, where he contributes to the
Propositions-as-Types correspondence of Figure\,\ref{ResearchAreas} in the
context of classical logic and programming languages with effects and rich
type structures.  Some of his results have already been published at
conference proceedings. %\eg~\cite{Munch,CurienMunch}.
For instance, 
%
%At the ``Proof Theory'' vertex of the triangle, he contributed to the
%understanding of polarisation in the framework of Krivine's Classical
%Realisability~\cite{}, extending the latter in particular to a call-by-value
%mode of evaluation (thus answering an open question of Danos
%and Beffara~\cite{}). He also contributed to the understanding of Girard's
%controversial involutive classical negation, by developing a computational
%interpretation in terms of Smalltalk-style control operators.
%%Ref Gir? Ref Parigot/LQTdF for the controversy?
%
in~\cite{Munch}, 
%At the ``Programming Languages'' vertex, 
he gave an original way of representing proofs in linear logic (including the
modalities) in a way that turns out to match closely the model theories
(\ref{LCBPV}) and (\ref{LEEC}).  This is relevant to the proposed
investigations in Section\mbox{\pref{PolarisationParagraph}}, and for further
research on modalities as proposed in Section\pref{ModalLogicsParagraph}.
%
%He also showed that the semantic construction behind polarisation generalises
%to models of computation with effects such as ``delimited control'' calculi,
%including the one of Danvy and Filinski.  We noticed similarities with the
%models mentioned in b-2.iii, and it allows an original approach to effects in
%Section b-4.ii.%(add internal refs)

%The ``Category Theory'' vertex of the triangle aims at a further algebraic
%understanding of both the syntax and the semantics considered; it is currently
%under development in his ongoing collaboration with the Principal
%Investigator. For instance they develop a notion of ``non-associative
%category'' for which there are concrete results in preliminary form.
%Questions from sections b-2.iii, 2.iv and 4.ii stem from this collaboration.

Supported by \emph{Fondation Sciences Math\'ematiques de Paris},
Munch-Maccagnoni visited the Principal Investigator at the Computer
Laboratory, University of Cambridge, during March--May~2011.  Discussions on
topics of Sections\pref{PolarisationParagraph} started then.  These pave the
way for the proposed research in Section\pref{ProgrammingEffectsParagraph}.

%The triangle is held together by his variant of the L system which replaces
%the $\lambda$ calculus at the center of the new triangle --- it remedies the
%defects of the latter notation, which becomes insufficient when it comes to
%the interactive forms of computation considered.

\paragraph*{Additional costs.}

\begin{myitemize}
\item
Travel funding is requested for attending workshops and 
conferences, \hidefootnote{\Eg~CALCO, CIE, CSL, CT, ICALP, ICFP, LICS, MFPS,
  POPL, PPDP, PSSL, RTA, TLCA, TYPES, WOLLIC.}, 
and for visiting and/or inviting researchers. \hidefootnote{\Eg~Aarhus
  (A\,Kock); 
  Bamberg (Mendler);
  Barcelona (J\,Kock);
  Bath (Power); 
  Berkeley (D\,Scott); 
  Birmingham (Escard\'o, Levy); 
  Bologna (Asperti, Dal Lago);
  Buffalo (Lawvere);
  Copenhagen (Birkedal, Filinski); 
  Darmstadt (Streicher); 
  Dublin (Dotsenko);
  Edinburgh (Leinster, Plotkin, A\,Simpson); 
  Genova (Moggi, Rosolini); 
  G\"oteborg (Coquand, Dybjer);
  Gunma (Hamana);
  Ljubljana (Bauer, Pretnar);
  London (Oliva);
  Manchester (Aczel);
  Marseille (Girard, Lafont, Regnier); 
  Montreal (Joyal, Makkai, Panangaden);
  Munich (Abel, Hofmann);
  Nice (C\,Simpson, Vallette);
  Nijmegen (Moerdijk);
  Nottingham (Altenkirch);
  Ottawa (P\,Scott); 
  Oxford (Abramsky, Coecke, Doring); 
  Paris (Burroni, Curien, Herbelin, Krivine, Melli\`es, Metayer);
  Philadelphia (Freyd, Pierce, Weirich);
  Pittsburgh (Avigad, Awodey, Harper, Pfenning, Reynolds);
  Portland (Sheard);
  Princeton (Warren);
  Strathclyde (Ghani, McBride);
  Swansea (Beckmann, Berger);
  Sydney (Garner, Lack, Street);
  Warsaw (Zawadowski).}.
\item
Further funds are requested for notebooks for the Principal Investigator and
the Research Associates, and the cost of books and broadband for the Principal
Investigator.
\end{myitemize}

{\setstretch{0}\footnotesize
\begin{thebibliography}{000}
%\bibitem{AHS}
%S.\,Abramsky, E.\,Haghverdi and P.\,Scott (2002). 
%\newblock Geometry of Interaction and linear combinatory algebras. 
%%\newblock \emph{Mathematical Structures in Computer Science} 12:625--665.
%\newblock \emph{MSCS} 12.

%\bibitem{AguiarMahajan}
%M.\,Aguiar and S.\,Mahajan (2010).
%\newblock \emph{Monoidal functors, species, and Hopf algebras}.
%\newblock CRM Monograph Series, AMS.

\bibitem{Adams}
R\,Adams (2006).
\newblock Pure type systems with judgmental equality.
\newblock \emph{Journal of Functional Programming}, 16(2):219--246.

\bibitem{AhnSheard08}
K.Y.\,Ahn and T.\,Sheard (2008).
\newblock Shared subtypes: subtyping recursive parametrized algebraic data
types.  
\newblock In \emph{Haskell'08}.  

\bibitem{AhnSheard11}
K.Y.\,Ahn and T.\,Sheard (2011).
%\newblock A Hierarchy of Mendler style Recursion Combinators: Taming
%  Inductive Datatypes with Negative Occurrences.  
\newblock A Hierarchy of Mendler style Recursion Combinators. 
\newblock In \emph{ICFP'11}.  


%\bibitem{Albert}
%A.A.\,Albert (1948).
%\newblock Power-associative rings.
%\newblock \emph{Trans.\ Amer.\ Math.\ Soc.}, 64(3):552--593.

%\bibitem{AltArtemov}
%J.\,Alt and S.\,Artemov (2001).
%\newblock Reflective $\lambda$-calculus.
%\newblock In \emph{PTCS'01}, LNCS 2183, \pp\,22--37.

\bibitem{Andreoli}
J.-M.\,Andreoli (1992).
\newblock Logic programming with focusing proof in linear logic.
%\newblock \emph{Journal of Logic and Computation}, 2(3):297--347.
\newblock \emph{J.\ of Logic and Computation} 2.

\bibitem{ArtemovLP}
S.\,Artemov (2001).
\newblock Explicit provability and constructive semantics.
\newblock \emph{Bulletin for Symbolic Logic}, 7(1):1--36.

%\bibitem{ArtemovIemhoff}
%S.\,Artemov and R.\,Iemhoff (2007).
%\newblock The Basic Intuitionistic Logic of Proofs.
%\newblock \emph{Journal of Symbolic Logic}, 72(2):439--451.

\bibitem{SGA4}
M.\,Artin, A.\,Grothendieck, and J.-L.\,Verdier, eds. (1972).
\newblock \emph{SGA~4}, LNM 269. 

\bibitem{UnivalentProgramme}
S.\,Awodey, T.\,Coquand, and V.\,Voevodsky (2012)
\newblock Univalent Foundations of Mathematics.
\newblock IAS School of Mathematics program, Princeton.
%\url{http://www.math.ias.edu/sp/univalent}.  

%\bibitem{AwodeyHofstraWarren}
%S.\,Awodey, P.\,Hofstra and M.\,Warren (2012).
%\newblock Martin-L\"of complexes.
%\newblock In \url{http://arxiv.org/abs/0906.4521}.

\bibitem{Cayenne}
L.\,Augustsson (1998).
\newblock Cayenne --- a language with dependent types.
\newblock In \emph{ICFP'98}, pp\,239--250.

\bibitem{BarberPlotkin}
A.\,Barber (1996).
\newblock Dual Intuitionistic Linear Logic. 
\newblock PhD thesis, University of Edinburgh. 

\bibitem{BauerPretnar}
A,\,Bauer and M.\,Pretnar (2012).
\newblock Programming with Algebraic Effects and Handlers.
%\newblock In \url{http://arxiv.org/abs/1203.1539}.

\bibitem{BeckThesis}
J.\,Beck (1967).
\newblock \emph{Triples, algebras and cohomology}.
\newblock \emph{Reprints in TAC} 2, 2003. 

%\bibitem{BeckmannBussFriedman}
%A.\,Beckmann, S.\,Buss, and S.\,Friedman (2012).
%\newblock Safe Recursive Set Functions.
%%\newblock Available on-line.

\bibitem{BeesonBook}
M.\,Beeson (1984).
\newblock Foundations of constructive mathematics.
%\newblock Springer Verlag.

%\bibitem{BellantoniCook}
%S.\,Bellantoni and S.\,Cook (1992).
%\newblock A new recursion-theoretic characterisation of the polytime
%functions.
%\newblock \emph{Computational Complexity}, 2:97--110.

\bibitem{Benabou}
J.~B{\'e}nabou (1967). 
\newblock \emph{Introduction to bicategories}. 
%\newblock In LNM 47, pp\,1--77. 
\newblock In LNM 47.

\bibitem{BentonWadler}
N.\,Benton and P.\,Wadler (1996). 
\newblock Linear logic, monads, and the lambda calculus.
\newblock In \emph{11th LICS}.

\bibitem{Birkhoff}
G.\,Birkhoff (1935).
\newblock On the structure of abstract algebras.
\newblock {\em P.\ Camb.\ Philos.\ Soc.}, 31:433--454. 
  
\bibitem{BoehmBerarducci}
C.\,Boehm and A.\,Berarducci (1985). 
\newblock Automatic Synthesis of Typed Lambda-Programs on Term Algebras.
%\newblock \emph{Theoretical Computer Science}, 39:135--154.
\newblock \emph{TCS} 39.

\bibitem{Boole}
G.\,Boole (1847).
\newblock \emph{The Mathematical Analysis of Logic}.%, Being an Essay Towards
  %a Calculus of Deductive Reasoning}. 
%\newblock Macmillan, Barclay, \& Macmillan. 

\bibitem{ModalPTS}
T\,Borghuis (1998).
\newblock Modal Pure Type Systems: Type Theory for Knowledge
Representation.  
\newblock \emph{Journal of Logic, Language, and Information}, 7:265--296. 

\bibitem{Cartmell}
J\,Cartmell (1986).
\newblock Generalised algebraic theories and contextual categories.
%\newblock \emph{Annals of Pure and Applied Logic}, 32:209--243.
\newblock \emph{Annals of Pure and Applied Logic} 32. %:209--243.

%\bibitem{ChapotonLivernet}
%F.\,Chapoton and M.\,Livernet (2001).
%\newblock Pre-Lie algebras and the rooted trees operad.
%\newblock \emph{Int.\ Math.\ Res.\ Not.}, 8:395--408.

\bibitem{Church1936}
A.\,Church (1936).
\newblock An unsolvable problem of elementary number theory.
%\newblock \emph{American Journal of Mathematics}, 58:345--363. 
\newblock \emph{American Journal of Mathematics} 58.

\bibitem{Church1940}
A.\,Church (1940).
\newblock A formulation of the simple theory of types.
\newblock {\em J.\,Symbolic Logic}, 5:56--68. 
  
\bibitem{Cohen}
P.\,Cohen (1966).
\newblock \emph{Set theory and the continuum hypothesis}.
%\newblock W A Benjamin.

\bibitem{CoquandNote}
T.\,Coquand and G.\,Jaber (2012).
\newblock A note on forcing and type theory.
\newblock \emph{Fundamenta Informatica} 100:43--52.

%\bibitem{CoquandHuet}
%T.\,Coquand and G.\,Huet (1988). 
%\newblock The Calculus of Constructions. 
%\newblock \emph{Information and Computation}, 76(2--3).

\bibitem{CurienHerbelin}
P.-L.\,Curien and H.\, Herbelin (2000).
\newblock The duality of computation.
\newblock In \emph{ICFP'00}, pp\,233--243.

\bibitem{CurienMunch}
P.-L.\,Curien and G.\,Munch-Maccagnoni (2010).
\newblock The duality of computation under focus.
\newblock In \emph{IFIP TCS}.

\bibitem{Curry1934}
H.\,Curry (1934).
\newblock Functionality in Combinatory Logic.
%\newblock In \emph{Proceedings of the National Academy of Sciences},
%  20:584--590.
\newblock In \emph{Proceedings of the National Academy of Sciences} 20.

\bibitem{Day}
B.\,Day (1970).
\newblock On closed categories of functors.
%\newblock In LNM 137, pp\,1--38.
\newblock In LNM 137. 

\bibitem{deBruijn}
N.G.\,de~Bruijn (1968).
\newblock Automath, a language for mathematics.
%\newblock In \emph{Automation and Reasoning}, vol\,2, Classical papers on
%Computational Logic 1967--1970, pp\,159--200, 1983.
\newblock In \emph{Automation and Reasoning}, pp\,159--200, 1983.

\bibitem{DybjerIF}
P.~Dybjer (1994).
\newblock Inductive families.
\newblock \emph{Formal Aspects of Computing}, 6:440--465.

\bibitem{DybjerITT}
P.~Dybjer (1996).
\newblock Internal Type Theory.
\newblock In \emph{TYPES}, LNCS 1158, pp\,120--134.

%\bibitem{DybjerIR}
%P.\,Dybjer (2000).
%\newblock A General Formulation of Simultaneous Inductive-Recursive
%Definitions in Type Theory. 
%%\newblock \emph{J.\ Symb.\ Log.}, 65(2):525--549.
%\newblock \emph{JSL} 65.

\bibitem{EEC}
J.\,Egger, R.\,M{\o}gelberg, and A.\,Simpson (2009). 
\newblock Enriching an Effect Calculus with Linear Types. 
\newblock In \emph{CSL'09}.

\bibitem{EilenbergMacLane}
S.\,Eilenberg and S.\,Mac Lane (1945).
\newblock General Theory of Natural Equivalences.
\newblock \emph{Trans.\ Amer.\ Math.\ Soc.},
  58(2):231--294. 

\bibitem{Filinski}
A.\,Filinski (1994).
\newblock Representing monads.
%\newblock In \emph{POPL'94}, pp\,446--457.
\newblock In \emph{POPL'94}.

\bibitem{LiftingKZ}
M.\,Fiore (1995).  
\newblock Lifting as a KZ-doctrine.  
%\newblock In \emph{CTCS'96}, LNCS 953, pp.\,146--158.  
\newblock In \emph{CTCS'96}.

\bibitem{FiorePPDP}
M.\,Fiore (2002). 
\newblock Semantic analysis of Normalisation By Evaluation for typed
lambda calculus.  
\newblock In \emph{PPDP'02}.

\bibitem{FioreFossacs}
M.\,Fiore (2005).  
\newblock Mathematical models of computational and combinatorial
structures.  
%\newblock In \emph{FOSSACS'05}, LNCS 3441, pp\,25--46.  
\newblock In \emph{FOSSACS'05}, LNCS 3441. 

%\bibitem{FioreMFPS}
%M.\,Fiore (2006).
%\newblock On the structure of substitution.  
%\newblock MFPS XXII invited address. 

\bibitem{FioreICALP}
M.\,Fiore (2012).   
\newblock Discrete Generalised Polynomial Functors.  
\newblock In \emph{ICALP'12}, LNCS 7392, pp\,214--226.

\bibitem{FioreRemarks}
M.\,Fiore, R.\,Di~Cosmo and V.\,Balat (2002). 
\newblock Remarks on isomorphisms in typed lambda calculi with empty and
sum types.  
\newblock In \emph{LICS'02}, pp\,147-156.  

\bibitem{Species}
M.\,Fiore, N.\,Gambino, M.\,Hyland, and G.\,Winskel (2008).   
\newblock The cartesian closed bicategory of generalised species of
structures.
\newblock \emph{J.\ London Math.\ Soc.}, 77:203-220. 

\bibitem{FioreHur}
M.\,Fiore and C.-K.\,Hur (2010).   
\newblock Second-order equational logic.  
\newblock In \emph{CSL'10}, LNCS 6247, pp\,320--335. 
  
\bibitem{FioreMahmoud}
M.\,Fiore and O.\,Mahmoud (2010).   
\newblock Second-order algebraic theories.  
\newblock In \emph{MFCS'10}, LNCS 6281, pp\,368--380. 

\bibitem{CompleteCuboidalSets}
M.\,Fiore, G.\,Plotkin and A.J.\,Power (1997). 
\newblock Complete cuboidal sets in axiomatic domain theory.  
%\newblock In \emph{LICS'97}, pp.\,268--279.  
\newblock In \emph{LICS'97}.

\bibitem{Frege1879}
G.\,Frege (1879).
\newblock \emph{Begriffsschrift, a formula language, modeled upon that of
  arithmetic, for pure thought}. 
%\newblock In \emph{\cite{vanHeijenoort}}, pp\,1--82.
\newblock In \emph{\cite{vanHeijenoort}}. 

\bibitem{Frege1903}
G.\,Frege (1893/1903).
\newblock 
%\emph{Grundgesetze der Arithmetik}.%\footnote{Basic laws of arithmetic.}
%\newblock Vol.\,I Jenna, 1893.  Vol.\,II 1903.
\emph{Grundgesetze der Arithmetik I/II}.

\bibitem{Gentzen1935}
G.\,Gentzen (1935). 
\newblock Untersuchungen \"uber das logische Schlie{\ss}en. 
\newblock Mathematische Zeitschrift 39:176--210.

%\bibitem{Gerstenhaber}
%M.\,Gerstenhaber (1963).
%\newblock The cohomology structure of an associative ring.
%\newblock \emph{Annals of Mathematics}, 78(2):267--288.

\bibitem{GirardSystemF}
J.-Y.\,Girard (1972).
\newblock \emph{Interpr\'{e}tation Fonctionnelle et \'{E}li\-mi\-na\-tion des
  Coupures de l'Arithm\'{e}tique d'Ordre Sup\'{e}rieur}.
\newblock Th\`{e}se de doctorat d'\'{e}tat, Universit\'{e} Paris VII. 

\bibitem{GirardLinearLogic}
J.-Y.\,Girard (1987). 
\newblock Linear logic.
%\newblock \emph{Theoretical Computer Science}, 50:1--101.
\newblock \emph{TCS}, 50:1--101.

\bibitem{GirardLC}
J.-Y.\,Girard (1991). 
\newblock A new constructive logic: Classical logic.
\newblock \emph{Math.\ Struct.\ Comp.\ Sci.}, 1(3).

\hide{
\bibitem{Griffin}
T.\,Griffin.
\newblock A Formulae-as-Types notion of control.
\newblock In \emph{POPL'90}, pp.\,47--58.
}

\bibitem{SGA1}
A.\,Grothendieck and M.\,Raynaud (1971).
%\newblock Rev\^etements \'etales et groupe fondamental.
%\newblock LNM 224.
\newblock \emph{SGA~1}, LNM 224.

%\bibitem{Hilbert}
%D.\,Hilbert (1927).
%\newblock The foundations of mathematics.
%\newblock In \emph{\cite{vanHeijenoort}}, pp\,464--479.

\bibitem{Hindley1969}
R.\,Hindley (1969).
\newblock The Principal Type-Scheme of an Object in Combinatory Logic.
\newblock \emph{Trans.\ Amer.\ Math.\ Soc.}, 146:29--60.

%\bibitem{Hofmann}
%M.\,Hofmann (1997).
%\newblock Syntax and semantics of dependent types. 
%\newblock In \emph{Semantics and Logics of Computation}, pp\,79--130. 
%%\newblock Cambridge University Press.

%\bibitem{HS}
%M.\,Hofmann and T.\,Streicher (1998).
%\newblock
%The groupoid interpretation of type theory. 
%\newblock In \emph{Twenty-five years of constructive type theory}, pp\,83--111.
%%\newblock Oxford University Press. 

\bibitem{Howard1969}
W.\,Howard (1969).
\newblock The formulae-as-types notion of construction.
\newblock In \emph{\cite{ToHBCurry}}, pp\,479--490. 

\bibitem{Hutton07}
G.\,Hutton (2007).
\newblock Programming in Haskell.
\newblock Cambridge University Press.

\hide{
\bibitem{Hyland}
M.\,Hyland (1988).
\newblock A small complete category.
%\newblock APAL 40:135--165.
\newblock APAL 40.
}

\bibitem{TypeTheoryWithForcing}
G.\,Jaber, N.\,Tabareau and M.\,Sozeau (2012).
\newblock Extending Type Theory with Forcing.
\newblock In \emph{LICS'12}, pp\,395--404.

\bibitem{Jacobs}
B\,Jacobs (1999).
\newblock \emph{Categorical Logic and Type Theory}.
%\newblock Elsevier.

\bibitem{Elephant}
P.\,Johnstone (2002).
\newblock Sketches of an Elephant: A Topos Theory Compendium. 
\newblock Oxford University Press.

%\bibitem{JoyalAdvMath}
%A.\,Joyal (1981). 
%\newblock Une th\'eorie combinatoire des s\'eries formelles. 
%\newblock \emph{Advances in Mathematics}, 42:1--82.

%\bibitem{JoyalLNM1234}
%A.\,Joyal (1986). 
%\newblock Foncteurs analytiques et esp\`eces de structures. 
%\newblock \emph{Combinatoire \'Enum\'erative}, LNM 1234, pp\,126--159.

\bibitem{Kan}
D.\,Kan (1958).
\newblock Adjoint functors.
%\newblock \emph{Trans.\ Amer.\ Math.\ Soc.}, 87:294--329.
\newblock \emph{Trans.\ Amer.\ Math.\ Soc.} 87.

\bibitem{KellyBook}
G.M.\,Kelly (1982).
\newblock \emph{Basic Concepts of Enriched Category Theory}.
\newblock \emph{Reprints in TAC} 10, 2005.

%\bibitem{KellyOperads}
%G.M.\,Kelly (1972).
%\newblock On the operads of J.P.\,May.
%\newblock \emph{Reprints in TAC} 13, 2005.

\bibitem{Kobayashi}
S.\,Kobayashi (1997).
\newblock Monad as modality.
%\newblock In \emph{TCS}, 175:29--74.
\newblock In \emph{TCS} 175.

\bibitem{KrivineRA}
J.-L.\,Krivine (2011).
\newblock Realizability algebras: a program to well order R.
\newblock \emph{LMCS}-7(3:2).

\bibitem{Lafont}
Y.\,Lafont (1990).
\newblock Interaction nets.
\newblock In \emph{POPL'90}.

\bibitem{LambekI}
J.\,Lambek (1968).
\newblock Deductive systems and categories I.
\newblock \emph{J.\ Math.\ Systems Theory}, 2:278--318.

%\bibitem{LambekII}
%J.\,Lambek (1969).
%\newblock Deductive systems and categories II.
%\newblock LNM 86, pp\,76--122.

\bibitem{LambekScott}
J.\,Lambek and P.\,Scott (1986).
\newblock \emph{Introduction to Higher Order Categorical Logic}.
\newblock Cambridge University Press.

\bibitem{LawvereThesis}
F.W.\,Lawvere (1963). %\,\&\,1968).
\newblock Functorial Semantics of Algebraic Theories. %and Some Algebraic
  %Problems in the context of Functorial Semantics of Algebraic Theories.
\newblock \emph{Reprints in TAC} 5, 2004.
  
\bibitem{LawvereAinF}
F.W.\,Lawvere (1969).
\newblock Adjointness in foundations.
\newblock \emph{Reprints in TAC} 16, 2006.

\bibitem{LawvereMetric}
F.W.\,Lawvere (1973).
\newblock Metric spaces, generalized logic and closed categories.
\newblock \emph{Reprints in TAC} 1, 2002.

%\bibitem{Leinster}
%T.\,Leinster (2004). 
%\newblock \emph{Higher Operads, Higher Categories}.
%%\newblock London Mathematical Society Lecture Note Series 298, Cambridge
%%University Press.
%\newblock LMS Lecture Note Series 298, Cambridge
%University Press.

\bibitem{LevyCBPV}
P.\,Levy (2005).
\newblock Adjunction models for Call-By-Push-Value with stacks.
\newblock \emph{TAC}, 14(5):75--110.

\bibitem{Linton}
F.\,Linton (1966).
\newblock Some aspects of equational theories.
\newblock In {\em Proc.\ Conf.\ on Categorical Algebra at La Jolla}, pp\,84--95.

%\bibitem{Permutads}
%J.-L.\,Loday and M.\,Ronco (2012).
%\newblock Permutads.
%%\newblock In \url{http://arxiv.org/abs/1105.5271}.

%\bibitem{LodayVallette}
%J.-L.\,Loday and B.\,Vallette (2012).
%\newblock \emph{Algebraic Operads}.
%%\newblock Grundlehren der mathematischen Wissenschaften, Volume 346. 
%\newblock Grundlehren der mathematischen Wissenschaften.

\bibitem{MacLane}
S.\,Mac Lane (1971).
\newblock Categories for the Working Mathematician.
\newblock Springer Verlag, Second Edition 1998.

%\bibitem{Markl}
%M.\,Markl (2006). 
%\newblock Operads and PROPs.
%%\newblock In \url{http://arxiv.org/abs/math/0601129}.

\bibitem{MartinLofETT}
P.\,Martin-L{\"o}f (1984).
\newblock Intuitionistic Type Theory.
\newblock Bibliopolis.

%\bibitem{May}
%J.P.\,May (1972).
%\newblock \emph{The geometry of iterated loop spaces}.
%\newblock LNM 271.

\bibitem{Epigram}
C.\,McBride.
\newblock Epigram: practical programming with dependent types.
\newblock In \emph{AFP'04}, pp\,130--170.

\bibitem{MelliesCMLL}
P.-A.\,Melli\`es (2009).
\newblock Categorical Semantics of Linear Logic.
\newblock In \emph{Panoramas et synth\`esis}, 27:1--196.

\bibitem{TensorLogic}
P.-A.\,Melli\`es and N.\,Tabareau (2010).
\newblock Resource modalities in tensor logic.
%\newblock \emph{Ann.\ Pure and Appl.\ Logic}, 161(5):632--653.
\newblock \emph{APAL}, 161(5):632--653.

\bibitem{MendlerMMCK}
M.\,Mendler and S.\, Scheele (2011).
\newblock Cut-free Gentzen calculus for multimodal CK.
%\newblock \emph{Information and Computation}, 209(12):1465--1490.
\newblock \emph{Inf.\ \& Comp.}, 209(12):1465--1490.

\bibitem{MendlerIter}
N.\,Mendler (1991). 
\newblock Inductive types and type constraints in the second-order lambda
calculus. 
%\newblock \emph{Ann.\ Pure Appl.\ Logic}, 51(1--2):159--172.
\newblock \emph{APAL}, 51(1--2):159--172.

\bibitem{Milner1978}
R.\,Milner (1978).
\newblock A Theory of Type Polymorphism in Programming.
%\newblock \emph{Journal of Computer and System Sciences}, 17:348--375.
\newblock \emph{J.\ of Computer and System Sciences}, 17:348--375.

\bibitem{MoggiLambdaC}
E.\,Moggi (1991).
\newblock Notions of computation and monads. 
\newblock \emph{Information And Computation}, 93(1).

\bibitem{Munch}
G.\,Munch-Maccagnoni (2009).
\newblock Focalisation and classical realisability.
\newblock In \emph{CSL'09}, LNCS 5771, pp\,409--423.

\bibitem{ProgMLTT}
B.\,Nordstr\"om, K.\,Petersson, and J.\,Smith (1990).
\newblock Programming in Martin-L\"of Type Theory: An Introduction.
%\newblock Oxford University Press.
\newblock OUP.

\bibitem{ParkIm}
S.\,Park and H.\,Im (2011).
\newblock A modal logic internalizing normal proofs.
\newblock \emph{Information and Computation}, 209:1519--1535.

\hide{
\bibitem{Pierce}
B.\,Pierce (2002).
\newblock \emph{Types and programming languages}.
%\newblock The MIT Press.
}

\bibitem{PlotkinCBVCBN}
G.\,Plotkin (1975).
\newblock Call-by-name, call-by-value and the \mbox{$\lambda$-calculus}.
\newblock \emph{Theoretical Computer Science}, 1:125--159.

\bibitem{PlotkinLCF}
G.\,Plotkin (1977).
\newblock LCF considered as a programming language.
\newblock \emph{Theoretical Computer Science}, 5:223--255.

\bibitem{PlotkinPowerAlgOpsAndGenEffs}
G.\,Plotkin and A.J.\,Power (2003).
\newblock Algebraic operations and generic effects.
\newblock \emph{Applied Categorical Structures} 11.

\bibitem{PlotkinPretnar}
G.\,Plotkin and M.\,Pretnar (2009).
\newblock Handlers of algebraic effects.
\newblock \emph{Programming Languages and Systems}, pp\,80--94.

%\bibitem{NatDedForIntNonCommLinLog}
%J.\,Polakow and F.\,Pfenning (1999).  
%\newblock Natural Deduction for Intuitionistic Non-Commutative Linear
%Logic.
%\newblock In \emph{TLCA'99}.

\bibitem{Reynolds}
J.\,Reynolds (1983).
\newblock Types, abstraction and parametric polymorphism.
\newblock \emph{Information Processing} 83, pp\,513--523.

%\bibitem{RoncoShuffleBialgebras}
%M.\,Ronco (2008). 
%\newblock Shuffle bialgebras.
%%\newblock In \url{http://arxiv.org/abs/math/0703437}.

\bibitem{Russell1902}
B.\,Russell (1902).
\newblock Letter to Frege. 
\newblock In \emph{\cite{vanHeijenoort}}, pp\,124--125. 

\bibitem{Russell1903}
B.\,Russell (1903).
\newblock \emph{The principles of mathematics}.
%\newblock Cambridge University Press. 
\newblock CUP.

\bibitem{ScottSolovay}
D.\,Scott (1967).
\newblock A proof of the independence of the continuum hypothesis
\newblock \emph{Theory of Computing Systems}, 1(2):89--111.

\bibitem{ScottTCS}
D.\,Scott (1969).
\newblock A type-theoretical alternative to ISWIM, CUCH, OWHY
%\newblock In \emph{Theoretical Computer Science}, 121(1--2):411--440,
%1993.
\newblock In \emph{TCS}, 121(1--2):411--440, 1993.

\hide{
\bibitem{ScottCV}
D.\,Scott (1970).
\newblock Constructive validity.
%\newblock In LNM 125, pp\,237--275.
\newblock In LNM 125. 
}

\bibitem{Schonfinkel}
M.\,Sch\"{o}nfinkel (1924).
\newblock On the building blocks of mathematical logic.
\newblock In \emph{\cite{vanHeijenoort}}, pp\,355--366.

\bibitem{Seely}
R.\,Seely (1989).
\newblock Linear logic, *-autonomous categories and cofree coalgebras 
\newblock \emph{Contemporary Mathematics} 92. 

\bibitem{Selinger}
P.\,Selinger (2011).
\newblock A survey of graphical languages for monoidal categories.
%\newblock In \emph{New Structures for Physics}, LNP 813, pp\,289--355. 
\newblock In LNP 813, pp\,289--355. 

\bibitem{Omega}
T.\,Sheard (2004).
\newblock Languages of the future.
\newblock In \emph{SIGPLAN Notices}.

\bibitem{Strachey1967}
C.\,Strachey (1967).
\newblock Fundamental Concepts in Programming Languages.  
%\newblock In \emph{Higher-Order and Symbolic Computation}, 13:11--49,
%2000.
\newblock In \emph{HOSC}, 13:11--49, 2000.

%\bibitem{Szabo}
%M.\,Szabo (1975).
%\newblock Polycategories.
%\newblock \emph{Communications in Algebra}, 3(8):663--689.

\bibitem{ToHBCurry}
J.\,Seldin and J.R.\,Hindley (1980).
\newblock \emph{To H.B.\,Curry: Essays on Combinatory Logic, Lambda
  Calculus and Formalism}.
%\newblock Academic Press. 
 
\bibitem{Turing}
A.\,Turing (1937). 
\newblock On computable numbers, with an application to the
  Entscheidungsproblem. 
%\newblock \emph{Proc.\ of the London Mathematical Society}, 42:230--265.
\newblock \emph{Proc.\ LMS}, 42:230--265.

\bibitem{vanHeijenoort}
J.\,van Heijenoort, ed.\ (1967).
\newblock \emph{From Frege to G\"odel: A source book in mathematical
  logic, 1879--1931}.
%\newblock Harvard University Press.

\bibitem{vonPlato}
J.\,von Plato (2012).
\newblock Gentzen's proof systems: byproducts in a work of genius.
\newblock \emph{Bull.\ Symbolic Logic}, 18(3):313--367.

%\bibitem{EqProofICFP12}
%D.\,Vytiniotis, S.\,Peyton~Jones, and J.P.\,Magalhaes (2012). 
%\newblock Equality proof and deferred type errors. 
%\newblock In \emph{ICFP'12}.

\bibitem{Wadler}
P.\,Wadler (2003).
\newblock Call-by-value is dual to call-by-name.
\newblock In \emph{ICFP'03}, pp\,189--201.

\bibitem{Haskell}
The Haskell Programming Language.
\newblock \url{http://www.haskell.org/haskellwiki/Haskell}.
\end{thebibliography}
}

%{\hspace*{-6cm}
%\includegraphics[height=16.5cm,width=11.5cm]
%{costs.pdf}}

\end{document}
