\section{Systems \Fi\ and \Fixi}\label{sec:Fi}

\subsection{Introduction to Systems \Fi\ and \Fixi, and their key properties}
System \Fi\ is an extension of a Curry style \Fw\ by term indexed types.
By curry style, we mean that lambda terms at term level are unannotated.
That is, the term syntax of \Fi\ and \Fw\, in the Curry style, are
the same as the term syntax of the untyped lambda calculus.
The key design principle of \Fi\ is that we extend the kind syntax of \Fw\
by allowing predefined types ($\tau : *$) to appear in domain of arrow kinds
($\tau -> *$), as follows:
\begin{align*}
\text{\Fw\ kinds} ~~~ \kappa ::= ~ & * \mid \kappa -> \kappa \\
\text{\Fi\; kinds}~~~ \kappa ::= ~ & * \mid \kappa -> \kappa \mid \tau -> \kappa
\end{align*}
We allow $\tau$ to appear in only the domain, but not in the range, of
arrow kinds, since we want the kinds to be either $*$ or arrow kinds
that eventually result in $*$ (\ie, $\vec{\kappa} -> *$) -- recall that
type constructors eventually become types when they are fully applied.
The extension to the type syntax follow from the extension to the kind syntax.
However, the term syntax does not change -- \Fi\ and \Fw\ has exactly the
same term.
The extensions we make to \Fi\ enables us to express term indexed types.
In \S\ref{sec:mendler}, we have seen examples of term index types such as
the length indexed list type $\textit{Vec}$, whose kind is $\textit{Nat} -> *$.
Note that \textit{Nat} is a predefined type appearing in the domain of
the arrow kind ($\textit{Nat} -> *$).

I am working with on a paper (to be submitted to an appropriate venue) that
describes the details of System \Fi. I plan to reformat and extend
the contents of this paper into a chapter in my thesis. Here, I summarize
the three key properties of \Fi, which I will provide proofs in my thesis:
\begin{description}
\item[\quad Type safety]
\Fi\ must have the usual type safety property (\ie, progress and preservation).

\item[\quad Index erasure]
Index erasure is a property that well-typed terms in \Fi\ is also 
well-typed in \Fw\, and its type in \Fw\ is given by the index erasure
of the type in \Fi. That is, if $\Gamma |-_{\Fi} t : \tau$ then
$\Gamma^\circ |-_{\Fw} t : \tau^\circ$, where $\circ$ is the notation
for index erasure. The index erasure property implies that the indices are
only relevant for type checking at compile time, but computationally irrelevant
at runtime. For instance, length indexed lists should behave exactly the same as
regular (non-indexed) lists at runtime.

\item[\quad Strong normalization]
The proof of strong normalization almost automatically follows from
index erasure, since we know that \Fw\ is normalizing.
\end{description}

System \Fixi\ is an extension of \Fixw\ by term indexed types.
\Fixw\ is a calculus developed to give a reduction preserving embedding of
the Mendler style primitive recursion family. \Fixw\ extends \Fw\ with
polarized kinds and equi-recursive types. In \Fixi, they track polarities of
kinds so that they may only take fixpoints of positive polarities.
\Fixi\ is an extension of \Fixw\ by term indexed types. The key design principle
of \Fixi\ is pretty much the same as the key design principle of \Fi.
We extend the kind syntax with predefined types in the domain of arrow kinds,
while keeping track of polarities, as follows:
\begin{align*}
\text{\Fixw\ kinds} ~~~ \kappa ::= ~ & * \mid \kappa^p -> \kappa \\
\text{\Fixi\; kinds}~~~ \kappa ::= ~ & * \mid \kappa^p -> \kappa \mid \tau^p -> \kappa
\end{align*}
where the polarity $p$ may be either $+$, $-$, or $\circ$.
Although \Fixi\ is still in its early stage of development, I foresee that
the work on proving the three key properties of \Fi\ would
naturally transfer to \Fixi with minor changes to proof structure
regarding the bookkeeping of polarities.

\subsection{Embeddings of the Mendler style recursion combinators}
In addition to showing the three key properties of Systems \Fi\ and \Fixi,
we also need demonstrate that there exist reduction preserving embeddings
of the Mendler style recursion combinators into either \Fi\ or \Fixi.
Showing that there can be reduction preserving embedding of \MIt in \Fw\
and \MPr\ in \Fixw, was the purpose of introducing \Fw\ and \Fixw\
in the literature on Mendler style recursion schemes.

I have extended these embeddings of \MIt\ and \MPr, taking term indexed
types into consideration, by introducing new calculi \Fi\ and \Fixw,
which are extensions of \Fw\ and \Fixi with term indices. In addition,
I will show that other families of the Mendler style recursion combinators
also have reduction preserving embeddings into either \Fi\ or \Fixi.
In particular, \MsfIt\ is embedded into \Fi and the course of values
recursion combinators (\McvIt\ and \McvPr) are embedded into \Fixi.

The embedding may be different for each family even though some of
them embed into the same target calculi. For instance, the target
calculi for \MIt\ and \MsfIt\ are both \Fi. However, the embedding
of \MIt\ and \MsfIt\ is different. The embedding of a certain family
is a pair of translations -- a translation of the recursive type operator,
and a translation of the Mendler style recursion combinator. In general,
the translation of the recursive type operator $\mu^\kappa$ is different for
each family, even though their target calculi may coincide.

