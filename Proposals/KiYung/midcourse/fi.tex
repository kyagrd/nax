\section{Systems \Fi\ and \Fixi}\label{sec:Fi}

In this document I have proposed that Mendler-style operators have translations
into strongly normalizing $\lambda$-calculi. The strategy is to develop
a sequence of calculi of increasing expressiveness. Several papers by other
researchers have begun this process. Mendler's ordinal work \cite{Mendler87}
extended System \textsf{F}, \citet{AbeMat04} extended System \Fw\ to get \Fixw.
In my thesis, I will follow in these footsteps by introducing System \Fi\
(a more expressive extension to \Fw) and System \Fixi\ (an extension to \Fixw).


\subsection{Introduction to Systems \Fi\ and \Fixi, and their key properties}
System \Fi\ is an extension of a Curry style \Fw\ by term indexed types.
By curry style, we mean that lambda terms at term level are unannotated.
That is, the term syntax of \Fi\ and \Fw\, in the Curry style, are
the same as the term syntax of the untyped lambda calculus.
The key design principle of \Fi\ is that the kind syntax of \Fw\ is extended
by allowing types ($\tau : *$) to appear in the domain of arrow kinds
($\tau -> *$), as follows:
\begin{align*}
\text{\Fw\ kinds} ~~~ \kappa ::= ~ & * \mid \kappa -> \kappa \\
\text{\Fi\; kinds}~~~ \kappa ::= ~ & * \mid \kappa -> \kappa \mid \tau -> \kappa
\end{align*}
Types, $\tau$, can appear only in the domain, but not in the range of
arrow kinds, since all kinds should be either $*$ or arrow kinds
that eventually result in $*$ (\ie, $\vec{\kappa} -> *$) -- recall that
type constructors eventually become types (\ie\ have kind $*$) when they are fully applied.
The extension to the type syntax follows directly from the extension to the kind syntax.
However, the term syntax does not change -- \Fi\ and \Fw\ have exactly the
same terms.
The extensions to \Fi\ enable users to express term indexed types.
In \S\ref{sec:mendler}, we saw several examples of term index types, such as
the length indexed list type $\textit{Vec}$, whose kind is $\textit{Nat} -> *$.
Note that \textit{Nat} is a type appearing in the domain of the arrow kind
($\textit{Nat} -> *$).

I am working with on a paper (to be submitted to an appropriate venue) that
describes the details of System \Fi. I plan to reformat and extend
the contents of this paper in a chapter in my thesis. Here, I summarize
the three key properties of \Fi. I will provide proofs in my thesis of
these properties.
\begin{description}
\item[\quad Type safety.]
\Fi\ must have the usual type safety properties (\ie, progress and preservation).

\item[\quad Index erasure.]
Index erasure is a property that well-typed terms in \Fi\ are also 
well-typed in \Fw\, and their types in \Fw\ are given by the index erasure
of their types in \Fi. That is, if $\Gamma |-_{\Fi} t : \tau$ then
$\Gamma^\circ |-_{\Fw} t : \tau^\circ$, where $\circ$ is the notation
for index erasure. The index erasure property implies that the indices are
only relevant for type checking at compile time, but computationally irrelevant
at runtime. For instance, length indexed lists should behave exactly the same as
regular (non-indexed) lists at runtime.

\item[\quad Strong normalization.]
The proof of strong normalization follows almost automatically from
index erasure, since we know that \Fw\ is normalizing.
\end{description}

System \Fixi\ is an extension of \Fixw\ by term indexed types. \Fixw\ is a
calculus developed to give a reduction preserving embedding of the Mendler
style primitive recursion family. \Fixw\ extends \Fw\ with polarized kinds
and equi-recursive types. In \Fixi, polarities of kinds are tracked so that
only the fixpoints of types with kinds of positive polarity can be taken.
Interesting properties of \Fixw\ include the ability to define constant
time predecessors.

\Fixi\ is an extension of \Fixw\ by term indexed types. The key design principles
of \Fixi\ are pretty much the same as the key design principles of \Fi.
We extend the kind syntax with types in the domain of arrow kinds,
while keeping track of polarities, as follows:
\begin{align*}
\text{\Fixw\ kinds} ~~~ \kappa ::= ~ & * \mid \kappa^p -> \kappa \\
\text{\Fixi\; kinds}~~~ \kappa ::= ~ & * \mid \kappa^p -> \kappa \mid \tau^p -> \kappa
\end{align*}
where the polarity $p$ may be either $+$, $-$, or $\circ$.
Although \Fixi\ is still in the early stages of development, I foresee 
that the work on proving the three key properties of \Fi\ will
naturally transfer to \Fixi\ with only minor changes to proof structure
regarding the bookkeeping of polarities.

\subsection{Embeddings of the Mendler style recursion combinators} In
addition to proving the three key properties of System \Fi\ and System
\Fixi, we also need to demonstrate that there exist reduction preserving
embeddings of the Mendler-style recursion combinators into either \Fi\ or
\Fixi. Showing that there are reduction preserving embeddings of \MIt in
\Fw\ and \MPr\ in \Fixw, was the sole purpose of introducing \Fw\ and \Fixw\ in
the literature on Mendler-style recursion schemes. In my thesis
I will follow this design pattern.

The embedding of a kind-indexed family of Mendler-style operators is
a pair of translations -- a translation of the recursive type operator
($\mu^\kappa$), and a translation of the Mendler style recursion combinator. 
I have extended embeddings of \MIt\ and \MPr, taking term indexed
types into consideration, by introducing new calculi \Fi\ and \Fixw,
which are extensions of \Fw\ and \Fixi with term indices. In addition,
I will show that other families of the Mendler-style recursion combinators
also have reduction preserving embeddings into either \Fi\ or \Fixi.
In particular, \MsfIt\ will be embedded in \Fi, and the course of values
recursion combinators (\McvIt\ and \McvPr) will be embedded in \Fixi.

The details of the embedding may be different for each family even though
some of them embed into the same target calculi. For instance, the target
calculi for \MIt\ and \MsfIt\ are both \Fi. However, the embeddings
of \MIt\ and \MsfIt\ are different. Generally, the translation of
the recursive type operator $\mu^\kappa$ is different for each family,
even though the target calculus of the translation may be the same.
In practice, we may want to use several different families of Mendler style
recursion combinators in one program. Therefore, we need to reconcile
these different encodings into a coherent theory, in order
to support our language Nax, which supports several different families of Mendler style
recursion combinators. I will briefly introduce the Nax language
in the following section.

