\section{Summary}\label{sec:summary}
My dissertation will consist of ten chapters.
Tentative titles of the chapters and important topics I plan to cover in each
chapter are listed below. Much of this (about 35\%) has already been written down
(in this document, in published papers, and in papers being prepared
for publication). 
\begin{quote}
\begin{enumerate}[1.]
\item Introduction
\begin{itemize}
\item My thesis and motivation for my work
\item Discusion of the role of typed systems
   \begin{itemize}
     \item Logics and reasoning systems
     \item Typed programming languages
     \item The Curry-Howard correspondence
   \end{itemize}
\item Previous work on defining recursive types in Logic and Programs
\item The role of induction
\end{itemize}

\item Polymorphic type systems 
 \begin{itemize}
     \item System \textsf{F}
     \item System \Fw
     \item Hindley-Milner
 \end{itemize}
 
\item Mendler style recursion schemes
  \begin{itemize}
     \item Historical progression
     \item Kind indexed families
     \item The zoo of Mendler style operators
     \item Inventing new families
     \item Termination proofs by Embedding in strongly normalizing calculi
  \end{itemize}

\item System \Fi
  \begin{itemize}
     \item Design principles and goals (Type indexed kinds)
     \item Syntax, Types, and Semantics
     \item Proof of strong normalization
     \item Embedding recursive types
     \item Embedding Mendler style operations
  \end{itemize}
  
\item System \Fixi
  \begin{itemize}
     \item Design principles and goals (Constant time predecessors)
     \item Syntax, Types, and Semantics
     \item Proof of strong normalization
     \item Embedding recursive types
     \item Embedding Primitive recursion
  \end{itemize}
  
\item The Nax language
  \begin{itemize}
    \item Goals for a programming language
    \item Structure of Mendler-Operators without Higher order polymorphism
    \item Index transformers
    \item Type Safety
    \item Type inference
    \item Embedding into \Fixw
  \end{itemize}
  
\item Case studies
  \begin{itemize}
    \item Lists and Natural numbers
    \item Term indexed types
    	\begin{itemize}
    	  \item Data structures with invariants (length indexed lists, balanced trees, etc)
    	  \item Languages, substitution, progress and type safety
    	\end{itemize}
    \item Higher order abstract syntax
    \item Normalization by reduction
  \end{itemize}

\item Related work
  \begin{itemize}
    \item Logical systems
    \item Strong normalization
    \item Indexed types
    \item GADTs
    \item Dependently typed systems
  \end{itemize}
  
\item Future work
  \begin{itemize}
    \item Trellys -- combining logic and programming
    \item Extension to a dependently typed framework.
 \end{itemize}
 
\item Conclusion
\end{enumerate}
\end{quote}

In the introduction, I will state my thesis and motivation for my work.
Also, I plan to give a summary on the history of typed programming
languages and formal reasoning. in particular, I will cover the Curry-
Howard correspondence, recursive types, and induction.

Then, I will review System \textsf{F}, System \Fw, and the Hindley-Milner
type system, in order to lead up to the discussions in the following
sections on Mendler style recursion schemes, System \Fi, System \Fixi, and
the Nax language.

In the case studies section, I will introduce many Nax programs to
demonstrate the expressiveness of Nax. Some of these will considerably
larger than the examples in this document. One of the examples I have in
mind is normalization by evaluation.

In the related work section, I will provide further details on what I have
summarized in the introduction section, and also discuss some recent work
related to my approach.

Then, I will discuss future work, which will mainly be discussion on further
extensions to the Nax language, and their ramifications to
the target calculi and the type inference algorithm. I will also
outline how this work relates to the larger Trellys project.

Finally, I will summarize and conclude my thesis.

\paragraph{The role of this document.} This mid-course document
was prepared as a requirement set by my thesis committee after
my thesis proposal paper and talk. I hope that it satisfies my
committee -- providing concrete
information about the goals, focus, and scope of my thesis. I would like
to thank the committee for the time they invested in my research, as I believe this
exercise
has focused my research in a manner that would not have been possible had
I not written it.

I would also like to thank professors Marcello Fiore and Andrew Pitts for
the many hours I spent with them between October 1st and December 31, 2011
while I visited Cambridge University. I would also like to thank my Trellys
team members and the NSF which supported this work (under grant 0613969).

