\section{The Nax language}\label{sec:Nax}

Nax is a language, which supports non-recursive datatypes, recursive type
operators ($\mu^\kappa$), and Mendler style recursion combinators (\MIt, \MPr,
\McvIt, \McvPr, \MsfIt) as language constructs. The calculi (\Fi\ and \Fixi)
we discussed in the previous section are theoretically powerful enough to
capture the semantics of recursive types and Mendler style recursion schemes.
However, they are far from practical programming purposes since even very simple
datatypes, such as boolean type, must be defined via impredicative encodings
(\eg, an encoding for the boolean type is $\forall a . a -> a -> a$). To be
programmer friendly, Nax supports datatypes and Mendler style recursion
combinators as primitive constructs of the language. I will introduce other
programmer friendly features of the Nax language in \S\ref{sec:Nax:tysym}.
Then, in \S\ref{sec:Nax:theory}, I will discuss the desired properties of Nax,
which is to be proven in the thesis.

\subsection{Type synonyms and macro derivations}\label{sec:Nax:tysym}
We are designing Nax to be friendly for functional programmers in several ways:
Nax has a Haskell-like syntax, conservatively extends Hindley-Milner
type inference, and supports type synonyms and derivation macros for defining
recursive types more conveniently. The examples in \S\ref{sec:mendler}
are pretty formatted Nax programs. If you know Haskell, you can tell
that their syntax are similar to Haskell. I mentioned how we do
type inference (or, reconstruction) based on the user provided
index transformers annotations on Mendler style recursion combinators.
I also mentioned that Nax supports type synonyms, but have not shown its
concrete syntax in \S\ref{sec:mendler}. Here, in Figure \ref{fig:naxmacros},
I illustrate the syntax for type synonyms and derivation macros, which make
recursive type definitions more convenient. The two versions of
a recursive type definition, one with type synonym and
the other with a derivation macro, are essentially the same.
The derivation macro version is expanded into the type synonym version
by the Nax implementation.

In the type synonym definition, we abbreviate the fixpoint $\mu^{*}(L\;a)$
as a type synonym $List$, and define constructor functions
$\textit{nil} : a -> \textit{List}\;a$ and
$\textit{cons} : a -> \textit{List}~a -> \textit{List}~a$.
Recall that we define a recursive type in two levels in Nax,
by taking a fixpoint of the base structure (\eg, $L$) for the recursive type.
The type synonym (\textit{List}) and its constructor functions
(\textit{nil} and \textit{cons}) correspond to the Haskell definition of
the recursive datatype (\textit{List}) and its data constructors
(\textit{Nil} and \textit{Cons}).

In the derivation macro definition, the derivation macro
(\textbf{deriving fixpoint} \textit{List}) attached to
the definition of the base datatype ($L$) instructs
the Nax implementation to automatically generate the type synonym 
(\textit{List}) for the recursive type and its constructor functions
(\textit{nil} and \textit{cons}) with the lowercase names.

\begin{figure}
\begin{tabular}{c|c|c}
Haskell  & Nax with synonyms &  Nax with derivation \\
\hline

\begin{minipage}[t]{.28\linewidth}
\small\vspace{.1em}
\textbf{data} \textit{List} $a$\\
$~~~$ $=$ \textit{Nil}\\ 
$~~~$ $\;|\;$ \textit{Cons} $a$ (\textit{List} a)
\vspace{4.9em}\\
$x = \textit{Cons}~3~(\textit{Cons}~2~\textit{Nil}\,)$
\end{minipage} 

& 

\begin{minipage}[t]{.32\linewidth}
\small\vspace{.1em}
\textbf{data} $L : * -> * -> *$ \textbf{where}\\
$~~~$  \textit{Nil}$~~\; :$ $L\;a\;r$\\
$~~~$  \textit{Cons}    $:$ $a -> r -> L\;a\;r$
\vspace{.3em}\\
\textbf{synonym} \textit{List} $a = \mu^{*} (L\;a)$
\vspace{.3em}\\
\textit{nil}$~~~~~~~~~= \In^{*} \textit{Nil}$ \\
\textit{cons}  $x\;xs = \In^{*} (Cons\;x\;xs)$
\vspace{.5em}\\
$x = \textit{cons}~3~(\textit{cons}~2~\textit{nil}\,)$
\end{minipage}

&

\begin{minipage}[t]{.32\linewidth}
\small\vspace{.1em}
\textbf{data} $L : * -> * -> *$ \textbf{where}\\
$~~~$  \textit{Nil}$~~\; :$ $L\;a\;r$\\
$~~~$  \textit{Cons}    $:$ $a -> r -> L\;a\;r$\\
$~~$ \textbf{deriving} \textbf{fixpoint} \textit{List}
\vspace{3.5em}\\
$x = \textit{cons}~3~(\textit{cons}~2~\textit{nil}\,)$
\end{minipage}

\end{tabular}
\caption{Two versions of a recursive type definition of \textit{List} in Nax --
         with a \textit{type synonym} and with a \emph{derivation macro}
         (\cf\ Haskell definition of $List$)}
\label{fig:naxmacros}
\end{figure}

\subsection{Properties of Nax}\label{sec:Nax:theory}
\paragraph{Type saftey:}
Nax, being a typed programming language, should be type safe.
I will prove type safety of Nax in the thesis.

\paragraph{Normalization:}
I will prove normalization of Nax by a reduction preserving embedding into
either \Fi\ or \Fixw, depending on which families of Mendler style recursion
combinators we decide to include in Nax. Since it is likely that we want
primitive recursion most of the time, the target of the embedding would
eventually be \Fixw. But, for simplicity of illustration, I will first
illustrate an embedding of a small version of Nax, with \MIt\ and \MsfIt\
families only, into \Fi. Then, I will illustrate an embedding of Nax,
including other families, into \Fixw, with gradually increasing complexity.

\paragraph{Type inference}
properties about type inference \\
syntax directed type system (conversion is inlined) \\
type inference algorithm is sound and complete \\
most general type w.r.t. index transformer annotation

Design of Nax
    Design Macros - 2 level types plus type (and term) synonyms
    Static choices (Mu *)  (In *->*)   index transformers
    kind and type inference.

