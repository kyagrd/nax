\section{The Nax language}\label{sec:Nax}

To validate the second part of my thesis -- the Mendler-style can support 
a simple and expressive programming language, we have designed the
programming language Nax.

Nax supports non-recursive datatypes, the recursive type
operator ($\mu^\kappa$), and the Mendler-style recursion combinators (\MIt, \MPr,
\McvIt, \McvPr, \MsfIt). The calculi, \Fi\ and \Fixi,
discussed in the previous sections, are theoretically powerful enough to
capture the semantics of both recursive types and Mendler-style recursion schemes.
However, they are far from practical programming languages. Even very simple
datatypes, such as the boolean type, must be defined via impredicative 
Church-style encodings
(\eg, an encoding for the boolean type is $\forall a . a -> a -> a$). 
Fortunately, it is possible to provide a more user friendly layer of abstraction
over these calculi. Nax supports datatypes and Mendler style recursion
combinators as primitive constructs of the language. I will introduce other
programmer friendly features of the Nax language in \S\ref{sec:Nax:tysym}.
Then, in \S\ref{sec:Nax:theory}, I will discuss the desired properties of Nax,
which will be demonstrated in the thesis.

\subsection{Type synonyms and macro derivations}\label{sec:Nax:tysym}

Nax was designed with two goals in mind. First it must be
directly definable in terms of these calculi, and it must exhibit a user friendly
interface to programmers. Towards these goals, 
Nax has a Haskell-like syntax, and conservatively extends Hindley-Milner
type inference. It also supports type synonyms and term macros. and a mechanism to
automatically derive macros to make defining
recursive types more convenient. The examples in \S\ref{sec:mendler}
are really reformatted Nax programs. If you know Haskell, you will see
that the Nax syntax is similar to Haskell. We have already discussed how Nax performs
type inference (or, type reconstruction) based on user provided
index transformers attached to Mendler-style recursion combinators.
I also mentioned that Nax supports type synonyms, but did not
show any details of the concrete syntax in \S\ref{sec:mendler}. In Figure \ref{fig:naxmacros},
I illustrate the syntax for type synonyms and derivation macros, which make
recursive type definitions more convenient. In the figure I give three
versions. The first is a Haskell version for reference. The other
two versions are both Nax programs. The first Nax version makes explicit the type synonyms
and macros. In the the second Nax version we use the derivation facility.
The two versions define equivalent programs..

In the type synonym version, we abbreviate the fixpoint $\mu^{*}(L\;a)$
using the type synonym $List$, and define constructor functions
$\textit{nil} : a -> \textit{List}\;a$ and
$\textit{cons} : a -> \textit{List}~a -> \textit{List}~a$.
Recall, that we define recursive types using two levels in Nax.
We take the fixpoint of the base structure (\eg, $L$) to define a recursive type.
The type synonym (\textit{List}), and its constructor functions
\textit{nil} and \textit{cons}, correspond to the Haskell definition of
the recursive datatype (\textit{List}) and its data constructors
(\textit{Nil} and \textit{Cons}).

In the the third version using derivation, the declaration
(\textbf{deriving fixpoint} \textit{List}) attached to
the definition of the base datatype ($L$) instructs
the Nax implementation to automatically generate the type synonym 
(\textit{List}) for the recursive type and its constructor functions
(\textit{nil} and \textit{cons}) with names derived by dropping the case
of the initial alphabetic character.

\begin{figure}
\begin{tabular}{c|c|c}
Haskell  & Nax with synonyms &  Nax with derivation \\
\hline

\begin{minipage}[t]{.28\linewidth}
\small\vspace{.1em}
\textbf{data} \textit{List} $a$\\
$~~~$ $=$ \textit{Nil}\\ 
$~~~$ $\;|\;$ \textit{Cons} $a$ (\textit{List} a)
\vspace{4.9em}\\
$x = \textit{Cons}~3~(\textit{Cons}~2~\textit{Nil}\,)$
\end{minipage} 

& 

\begin{minipage}[t]{.32\linewidth}
\small\vspace{.1em}
\textbf{data} $L : * -> * -> *$ \textbf{where}\\
$~~~$  \textit{Nil}$~~\; :$ $L\;a\;r$\\
$~~~$  \textit{Cons}    $:$ $a -> r -> L\;a\;r$
\vspace{.3em}\\
\textbf{synonym} \textit{List} $a = \mu^{*} (L\;a)$
\vspace{.3em}\\
\textit{nil}$~~~~~~~~~= \In^{*} \textit{Nil}$ \\
\textit{cons}  $x\;xs = \In^{*} (Cons\;x\;xs)$
\vspace{.5em}\\
$x = \textit{cons}~3~(\textit{cons}~2~\textit{nil}\,)$
\end{minipage}

&

\begin{minipage}[t]{.32\linewidth}
\small\vspace{.1em}
\textbf{data} $L : * -> * -> *$ \textbf{where}\\
$~~~$  \textit{Nil}$~~\; :$ $L\;a\;r$\\
$~~~$  \textit{Cons}    $:$ $a -> r -> L\;a\;r$\\
$~~$ \textbf{deriving} \textbf{fixpoint} \textit{List}
\vspace{3.5em}\\
$x = \textit{cons}~3~(\textit{cons}~2~\textit{nil}\,)$
\end{minipage}

\end{tabular}
\caption{Two versions of a recursive type definition of \textit{List} in Nax --
         with a \textit{type synonym} and with a \emph{derivation macro}
         (\cf\ Haskell definition of $List$)}
\label{fig:naxmacros}
\end{figure}

\subsection{Properties of Nax}\label{sec:Nax:theory}

Nax is designed to demonstrate my thesis. As such, the Nax language must exhibit
a number of properties. I list a few of them here.

\paragraph{Type saftey:}
Nax, being a typed programming language, should be type safe.
I will prove type safety of Nax in the thesis.

\paragraph{Normalization:}
I will prove normalization of Nax by a reduction preserving embedding into
either \Fi\ or \Fixw, depending on which families of Mendler style recursion
combinators we decide to include in Nax. Since it is likely that 
primitive recursion is most useful, the eventual target of the embedding will
be \Fixw. But, for simplicity of illustration, I will initially
illustrate an embedding of a subset of Nax, with \MIt\ and \MsfIt\
families only, into \Fi. Then, I will illustrate an embedding of Nax,
including other families, into \Fixw, with gradually increasing complexity.

\paragraph{Type inference}
properties about type inference \\
syntax directed type system (conversion is inlined) \\
type inference algorithm is sound and complete \\
most general type w.r.t. index transformer annotation

Design of Nax
    Design Macros - 2 level types plus type (and term) synonyms
    Static choices (Mu *)  (In *->*)   index transformers
    kind and type inference.

