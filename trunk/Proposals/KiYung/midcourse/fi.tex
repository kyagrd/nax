\section{System \Fi and \Fixi}\label{sec:Fi}

System \Fi\ is an extension of \Fw\ by term indexed types.
The key design principle of \Fi\ is that we extend the kind syntax of \Fw\
by allowing predefined types ($\tau : *$) to appear in domain of arrow kinds
($\tau -> *$), as follows:
\begin{align*}
\text{\Fw\ kinds} ~~~ \kappa ::= ~ & * \mid \kappa -> \kappa \\
\text{\Fi\; kinds}~~~ \kappa ::= ~ & * \mid \kappa -> \kappa \mid \tau -> \kappa
\end{align*}
We allow $\tau$ to appear in only the domain, but not in the range, of
arrow kinds, since we want the kinds to be either $*$ or arrow kinds
that eventually result in $*$ (\ie, $\vec{\kappa} -> *$) -- recall that
type constructors eventually become types when they are fully applied.
Extensions to the type and term syntax follow from this extension to
the kind syntax. I am working with on a paper (to be submitted to an
appropriate venue) that describes the details of System \Fi. I plan to
reformat and extend the contents of this paper into a chapter in my
thesis.

System \Fixi\ is an extension of \Fixw\ by term indexed types.
\Fixw\ is a calculus developed to give a reduction preserving embedding of
the Mendler style primitive recursion family. \Fixw\ extends \Fw\ with
polarized kinds and equi-recursive types. In \Fixi, they track polarities of
kinds so that they may only take fixpoints of positive polarities.
\Fixi\ is an extension of \Fixw\ by term indexed types. The key design principle
of \Fixi\ is pretty much the same as the key design principle of \Fi.
We extend the kind syntax with predefined types in the domain of arrow kinds,
while keeping track of polarities, as follows:
\begin{align*}
\text{\Fixw\ kinds} ~~~ \kappa ::= ~ & * \mid \kappa^p -> \kappa \\
\text{\Fixi\; kinds}~~~ \kappa ::= ~ & * \mid \kappa^p -> \kappa \mid \tau^p -> \kappa
\end{align*}
where the polarity $p$ may be either $+$, $-$, or $\circ$.
Although \Fixi\ is still in its early stage of development, I foresee that
most of the important properties of \Fi\ would naturally transfer to \Fixi,
with minor changes in the proofs regarding the bookkeeping of polarities.


TODO

The embedding would be different from
the embedding of \MPr\ though. The embedding of some families of
the Mendler-style recursion combinators is a pair of translations --
a translation of the recursive type operator, and a translation of
the Mendler style recursion combinator. In general, the translation of
the recursive type operator $\mu^\kappa$ is different for each family
of the Mendler style recursion combinators, even though their target
calculi may coincide.

