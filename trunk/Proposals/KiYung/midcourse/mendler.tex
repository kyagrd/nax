\section{Mendler style recursion schemes}\label{sec:mendler} I claim to
have organized the Mendler style operators into a natural hierarchy. We
describe this hierarchy here. There are common patterns (or, general
forms) of Mendler style recursion combinators that I use in my
dissertation (\S\ref{sec:mendler:intro}). These patterns have two
dimensions. The first dimension describes what and how the operator
computes, and the second dimension abstracts over the kind of a data
type. We call all Mendler operators that compute the same way, but over
different types with different kinds,  a kind-indexed family of
operators. Mendler operators that compute things using a different
computation pattern belong to different families.

Despite the fact that different families compute things differently, all
families have a common structure. I discuss this common stucture first,
then I introduce several families of Mendler style combinators
(\S\ref{sec:mendler:it}-\ref{sec:mendler:sf}) which I plan to support in
the Nax language design. Lastly, I postulate on other possible
generalizations and extensions (\S\ref{sec:mendler:etc}).

\subsection{Two level types and Mendler style recursion combinators}
\label{sec:mendler:intro}
In this section, I introduce the general pattern behind a Mendler style recursion scheme.
We formulate recursive types in two levels,  by applying the recursive type operator
$\mu^\kappa$ to a non-recursive base structure $T$. 
We illustrated this in Figure \ref{fig:approaches}
in Section \ref{sec:relwork:other}. Figure \ref{fig:approaches}
only illustrates regular datatypes without indices
(i.e. where the $\kappa$ of $\mu^\kappa$ is kind $*$). Here, I will describe
the most general form where $\kappa$ of $\mu^\kappa$ has the form
$\vec{\kappa} -> *$ (\ie, When $\vec{\kappa}$ is an empty sequence,
$(\vec{\kappa} -> *)$ means $(*)$. Otherwise, when $\vec{\kappa}$ is
$\kappa_1 \cdots \kappa_n$, $(\vec{\kappa} -> *)$ means
$(\kappa_1 -> \cdots -> \kappa_n -> *)$).

\paragraph{The pattern of recursive type construction.}
The general form of a non-recursive datatype definition, which is to be used
as the base structure for a recursive types, is the following:
\begin{align*}
\textbf{data}~ T : \vec{\kappa}_p -> (\vec{\kappa}_i -> *) ->
                                      \vec{\kappa}_i -> *
~\textbf{where}~
&~ C_1 : \vec{\tau}_1 -> T\,\vec{p}\;r\;\vec{\tau_i}_{1} \\
&\quad \vdots \\
&~ C_n : \vec{\tau}_n -> T\,\vec{p}\;r\;\vec{\tau_i}_{n}
\end{align*}
The datatype constructor $T$ expects a series of type arguments $\vec{p}$, $r$,
and $\vec{i}$, whose kinds are $\vec{\kappa}_p$, $(\vec{\kappa}_i -> *)$, and
$\vec{\kappa}_i$, respectively. Thus, a value of this datatype would have
type $(T\,\vec{p}\,r\,\vec{i})$ where the type variables $\vec{p},r,\vec{i}$
are instantiated to certain types (or in the case of the $\vec{i}$, they are instantiated to either terms or types).
\begin{itemize}
\item The sequence of type variables $\vec{p}$ stand for the type parameters of the
type. Type parameters always appear regularly 
(\ie, they are always the same type variables $\vec{p}$\
no matter where they occur) in the type siguatures of
the data constructors $C_1,\cdots,C_n$. 

\item The type variable $r$ stands for
recursive ocurrences. Note it can have a rather complicted kind $(\vec{\kappa}_i -> *)$.

\item The sequence of type variables $\vec{i}$ stands for indices
that can vary (\eg, $\vec{\tau_i}_1,\vec{\tau_i}_n$) in the type signatures of
the data constructors $C_1,\cdots,C_n$. 

\item Note that classifying whether an argument is a parameter or an index is
determined soley by its position-- whether it comes before or after $r$.

\end{itemize}
We discuss the pattern as it occurs for some common types
\begin{enumerate}

\item For non-parametrized types (or, monotypes) such as the natural numbers,
there are no parameters and no indices. For instance, the base structure
for natural numbers would have the form $N\;r$. 

\item For parametrized,
but non-indexed, types (or, regular polymorphic types), there are parameters
before $r$ but no indices after $r$. For instance, the base structure for
polymorphic lists would have the form $L\;p\;r$ where $p$ is the type of
the elements in the list. 

\item For indexed types, there are indices following $r$.
For instance, the base structure for length indexed lists
(see Section \ref{sec:mendler:it}) would have the
form $V\;p\;r\;i$ where $p$ is the type of the elements in the list and
$i$ is the index that specifies the length of the list.

\end{enumerate}

To define the recursive types, we apply the recursive type operator ($\mu$) to
base structures. There is  family of recursive type operators $\mu^\kappa$
superscripted by kind $\kappa$, and its term constructor $\In^\kappa$ is also
superscripted by kind $\kappa$. The recursive type operator $\mu$
for regular datatypes in Figure \ref{fig:approaches} corresponds to $\mu^{*}$,
and its constructor is $\In^{*}$. In general the type operator $\mu^\kappa$
has kind $(\kappa -> \kappa) -> \kappa$ and the term constructor
$\In^\kappa$ has type $f (\mu^\kappa\; f) -> \mu^\kappa\; f$ where
$f$ has kind $(\kappa -> \kappa)$.

One must use the recursive type operator of the appropriate kind,
which depends on the indices ($\vec{i}\,$) of the base structure.
For example, $\mu^{*}N$ is the natural number type;
$\mu^{*}(L\,p)$ is the list type; $\mu^{\textit{Nat}-> *}(V\,p)$ is the length indexed vector type.

In general, $\mu^\kappa : (\kappa -> \kappa) -> \kappa$ can be
applied to $T\,\vec{p} : \kappa -> \kappa$ to form a recursive type
(or, recursive type constructor, if there are indices)
$\mu^\kappa(T\,\vec{p}) : \kappa$. Note that datatypes become a type of
kind $*$, only when they are fully applied. For example,
$T\,\vec{p}\,r\,\vec{i} : *$. 

Since we know that the recursive type is
well formed only when $T\,\vec{p} : \kappa -> \kappa$,
the next type argument $r$ must be of kind $\kappa$.
Also, we know that $\kappa$ should always be of the form
$\vec{\kappa}_i -> *$ where each of $\vec{\kappa}_i$ is the kind for
each of the indices $\vec{i}$. Note that in a well formed
type, $(\mu^\kappa\; f)$ and $f$ always have the same
kind $(\kappa -> \kappa)$.

\paragraph{The pattern of a Mendler style operator type.}
In this section we illustrate that even Mendler style operations that
compute different things still have much in common.
Mendler style recursion combinators rely on
higher-rank polymorphism to ensure their normalization proprieties. 
Recall that, in Figure \ref{fig:approaches},
the type of $\varphi$ was required to be polymorphic over a type variable
$X$ (which corresponds to $r$ in this section).
The following is the general form of the type of a Mendler style
recursion combinator in the presence of type indices:
\begin{multline*}
 \textsf{Mxxx} : \forall t . \forall a .
   \big(\, \forall r . \;
      \overbraceset{\begin{smallmatrix}
                        \text{abstract} \\
                        \text{recursive call} \\
                        \text{over $r$-structure}
                       \end{smallmatrix}}{
            (\forall\vec{i} . r\,\vec{i} -> a\,\vec{i}\,) }
   -> \overbraceset{\begin{smallmatrix}
                        \text{other} \\
                        \text{abstract operations} \\
                        \text{over $r$-structure}
                       \end{smallmatrix}}{
            (\forall\vec{i} . \cdots) -> \cdots }
   -> \overbraceset{\begin{smallmatrix}
                        \text{function body desc.:} \\
                        \text{combines a $t$ structure}\\
                        \text{filled with $r$-values} \\
                        \text{into an answer}
                       \end{smallmatrix}}{
            (\forall\vec{i} . t\, r\, \vec{i}  -> a\,\vec{i}\,) }
   \big) \\
 -> \underbraceset{\text{resulting recursive function}}{
       (\forall\vec{i} . (\mu^\kappa t)\,\vec{i} -> a\,\vec{i}\,) }
\end{multline*}


The two dimensions of our hierarchy are both illustrated here.
Different families have different specifications as to the number and
type of the {\it other abstract operations}. Members of the same family
differ over the number and kind of the indices $\vec{i}$.

One can think of a Mendler style recursion combinator as a higher order function.
It returns a recursive function when applied to a set of abstract operations. The
operations are abstract over the (possibly indexed) type $r$. These operations
must never rely on the structure of $r$, and the higher-rank polymorphism enforces
this.

In general one thinks of the {\em abstract recursive call} and the {\em other
abstract operations} as given, and the user must supply the definition of the body
which {\em combines a t-structure filled with r-values into an answer}.


We can think of the above as an encoding of the following where $r$ is an abstract
type, and each of the functions in the product, a different abstract operation over $r$.

\begin{multline*}
 \textsf{Mxxx} : \forall t . \forall a .
 \exists r.
  \big(  (\forall\vec{i} . r\,\vec{i} -> a\,\vec{i}\,)
   \times (\forall\vec{i} . \cdots) \times \cdots
   \times (\forall\vec{i} . t\, r\,\vec{i} -> a\,\vec{i}\,)
   \big) \\
 -> (\forall\vec{i} . (\mu^\kappa t)\,\vec{i} -> a\,\vec{i}\,)
\end{multline*}
The abstract recursive call operation $(\forall\vec{i} . r\,\vec{i} -> a\,\vec{i}\,)$ is available in
all the Mendler style families we study. Each family (\textsf{xxx})
is distinguished by what other additional abstract operations are available.
Note that the abstract operations manipulate values of type $r$ when we describe
the function body, but we can apply the resulting function
to a concrete structure of type $(\mu^\kappa t)$, which is a recursive type.
At runtime $r$-values really are $(\mu^\kappa t)$-values, but
by using an abstract type $r$ while type-checking, we
can restrict the use of the recursive caller (and other abstract operations) to those uses that are guaranteed to normalize. 

Recall that we apply $\mu^\kappa$ to the datatype constructor ($T$) after it is
partially applied to its parameters (all the arguments of $T$ before $r$). So,
the type variable $t$ is always instantiated to $T\,\vec{p}$.

The type of \textsf{Mxxx} involves both higher-rank polymorphism and
indexed types. Both of these features make type inference difficult.
First, type inference for higher-rank polymorphism is, in general, undecidable.
Recall that in the Hindley-Milner type inference, terms have monomorphic types
by default, and only a limited form of polymorphism (type schemes) is allowed.
Only variables introduced by the text {\bf let}-syntax can be
polymorphic, and specialization only occurs on variables. 
Second, type inference for indexed types is, in general, undecidable as well.
This is why programming languages like Haskell cannot infer types of programs
involving pattern matching over GADTs, and proof assistants, like Coq,
supporting dependent types cannot support type inference in general. 

Despite these limitations, we hope to recover type inference for our surface
language Nax. We have designed Nax to exclude certain kinds of problematic
constructions, and to include just the right amount of type information to
make type inference possible.

The exclusions include supporting the use of higher-rank polymorphism only in
Mendler style recursion combinators, and the use of a special syntax to define
them. To support type inference over index types Nax includes a small amount of
type annotation ($\psi$). These annotations are necessary only on language
features that analyze data -- the case statement and Mendler style recursion
combinators. They are required only when the data being analyzed has
type indices. We call these annotations ($\psi$) index transformers,
since they are type level functions that specify how to construct
the resulting type from the indices of the value being analyzed.
There are many uses of the special syntax for Mendler style combinators,
and many uses of index transformers throughout the examples in
\S\ref{sec:mendler:it}-\S\ref{sec:mendler:ix}. 
With these special design parameters, we can do type inference on Nax programs.


\subsection{Mendler style iteration}\label{sec:mendler:it}
The iteration family \MIt\ is the most basic Mendler style recursion scheme.
It supports one abstract operation, abstract recursive call.
The type signature of \MIt\ is:
\[ \MIt : \forall t . \forall a .
   \big(\, \forall\, r\, . \;
         \overbraceset{\begin{smallmatrix}
                        \text{abstract} \\
                        \text{recursive call} \\
                        \text{over $r$-structure}
                       \end{smallmatrix}}{
            (\forall\vec{i} . r\,\vec{i} -> a\,\vec{i}\,) } \;
   -> \; \overbraceset{\begin{smallmatrix}
                        \text{function body desc.:} \\
                        \text{combines a $t$-structure} \\
                        \text{of $r$-values } \\
                        \text{into an answer}
                       \end{smallmatrix}}{
            (\forall\vec{i} . t\, r\,\vec{i} -> a\,\vec{i}\,) }
   \big)
 -> \; \overbraceset{\text{resulting recursive function}}{
          (\forall\vec{i} . (\mu^\kappa t)\,\vec{i} -> a\,\vec{i}\,) } 
\]
\begin{figure}
{\bf Natural Numbers}
\begin{align*}
\text{Base structure:}\qquad\qquad
&\textbf{data}~N : * -> * ~\textbf{where} \\
&\qquad \textit{Zero} : N\;r\\
&\qquad \textit{Succ} : r -> N\;r
\end{align*}
\[\text{Type synonym:}\qquad\qquad \textit{Nat} ~=~ \mu^{*} N\]
\begin{align*}
\text{Constructor functions:}\quad
&zero~~\,~=~ \In^{*} \textit{Zero} \\
&succ\;n ~=~ \In^{*} (\textit{Succ}\;n)
\end{align*}

{\bf Polymorphic Lists}
\begin{align*}
\text{Base structure:}\qquad\qquad
&\textbf{data}~L : * -> * -> * ~\textbf{where} \\
&\qquad \textit{Nil}~~\, : L\;p\;r \\
&\qquad \textit{Cons}    : r -> p -> L\;p\;r
\end{align*}
\[\text{Type synonym:}\qquad\qquad \textit{List} \; a  ~=~ \mu^{*} (L\;a)\]
\begin{align*}
\text{Constructor functions:}\quad
&nil~~~~~~~~\,~=~ \In^{*} \textit{Nil} \\
&cons\;x\;xs  ~=~ \In^{*} (\textit{Cons}\;x\;xs)
\end{align*}

{\bf Length indexed Lists}
\begin{align*}
\text{Base structure:}\qquad\qquad
&\textbf{data}~V : * -> (* -> *) -> * -> * ~\textbf{where} \\
&\qquad \textit{VNil}~~\, : V\;p\;r\;\{`zero\} \\
&\qquad \textit{VCons}    : r\;\{i\} -> p -> V\;p\;r\;\{`succ\;i\}
\end{align*}
\[\text{Type synonym:}\qquad\qquad \textit{Vec}\;a\;i  ~=~ \mu^{* -> *} (L\;a) \;i\]
\begin{align*}
\text{Constructor functions:}\quad
&vnil~~~~~~~~\,~=~ \In^{*} \textit{VNil} \\
&vcons\;x\;xs  ~=~ \In^{*} (\textit{VCons}\;x\;xs)
\end{align*}


\caption{Two level type definition of natural numbers, lists, and
	 length indexed lists as the fixpoints of their base structures, and
         the definition of their constructor functions as \In\ terms.}
\label{fig:natdef}
\end{figure}
A typical example of iteration is the length function for lists which
returns a natural number. Lists and natural numbers are introduced as two-level types
in Figure \ref{fig:natdef}. We define the \textit{length} function using the Mendler style
iteration combinator as follows:
\begin{align*}
\textit{length}~l = \MIt^{\{\}}~l~\textbf{with}~~
&  len ~\textit{Nil}~~~~~~~~~~~ = zero \\
&  len \;(\textit{Cons}\;x\;\textit{xs}) = succ\,(len\;\textit{xs})
\end{align*}
Since the list type does not involve any indices, we give an empty index
transformer annotation ($\{\}$) on \MIt. We use the \textbf{with} clause special
syntax to define the higher-rank body of the combinator. The keyword \MIt\ 
decides the exact type of the abstract recursive caller $len$. Its type is
a type scheme. We need only assume this type scheme within each of the clauses
of the \textbf{with}, rather than check that a function has a higher-rank type.

The type signature for the length function is
$\textit{length}:\mu^{*}(L\;p) -> \mu^{*}N$ where $N$ is the
base structure for the natural number type. Let us assume that we have
type synonyms $\textit{List} = \mu^{*}(L\;p)$ and
$\textit{Nat} = \mu^{*}N $.\footnote{The Nax implementation supports
a rich notion of type synonyms.}
Then, we can say that $\textit{length}:\textit{List}\;p -> \textit{Nat}$.
This type for \textit{length} is inferred by the type system of
the Nax language without any type annotation. Not a completely surprising
result, since \textit{length} has a Hindley-Milner type.

It is worth emphasizing that $len : r -> \textit{Nat}$ is
an abstract recursive call operation that consumes only $r$-values.
Note $\textit{xs}:r$ since the type signature for \textit{Cons} expects
its second argument to be of type $r$. Therefore, Mendler style iteration
does not have direct access to the tail of the list. One cannot define
a constant time tail function by just returning \textit{xs} since its type $r$
is an abstract recursive type ($r$) rather than the concrete list type
($\textit{List}\;p$, a synonym for $\mu^{*}(L\;p)$). 

In the next subsection, we shall see that the primitive recursion
Mendler style family is able to natrually express such functions having
direct acess to the predcessor values.

The examples we discussed so far were all about regular datatypes
(natrual numbers and lists), which do not involve any indices.
Mendler style recursion combinators generalize natrually for term indexed
types (\eg, length indexed lists, see Figure \ref{fig:natdef}).
I will continue the discussion on Mendler style recursion combinators
over indexed types in \S\ref{sec:mendler:ix}.
But, before we move on to indexed types, let us have a tour on other
Mendler style families (\S\ref{sec:mendler:pr}-\S\ref{sec:mendler:sf}),
starting from the primitive recursion family.

\subsection{Mendler style primitive recursion}\label{sec:mendler:pr}
The primitive recursion family \MPr\ is has an additional abstract operation,
which we call cast, which explicitly converts
a value of the abstract recursive type ($r$-value) into
a value of the concrete recursive type $(\mu^\kappa t)$.

The type signature of \MPr\ is:
\begin{multline*}
 \MPr : \forall t . \forall a .
   \big(\; \forall\, r\, . \;
         \overbraceset{\begin{smallmatrix}
                        \text{abstract} \\
                        \text{recursive call}
                       \end{smallmatrix}}{
            (\forall\vec{i} . r\,\vec{i} -> a\,\vec{i}\,) } \;
   -> \; \overbraceset{\begin{smallmatrix}
                        \text{cast from $r$-values} \\
                        \text{to $(\mu^\kappa t)$-values} \\
                       \end{smallmatrix}}{
            (\forall\vec{i} . r\,\vec{i} -> (\mu^\kappa t)\,\vec{i}\,) } \;
   -> \;    (\forall\vec{i} . t\, r\,\vec{i} -> a\,\vec{i}\,) \,
   \big) \\
 -> (\forall\vec{i} . (\mu^\kappa t)\,\vec{i} -> a\,\vec{i}\,)
\end{multline*}
Since \MPr\ has an additional abstract operation when compared
to \MIt, it can express all the functions
that we can express with \MIt, but some can be more efficient because of the
additional casting operation.

A typical example of primitive recursion is the factorial function:
\begin{align*}
\textit{factorial}~x = \MIt^{\{\}}~x~\textbf{with}~~
&  \textit{fac} ~cast ~\textit{Zero}~~~~\; = succ~zero \\
&  \textit{fac} ~cast \;(\textit{Succ}\;n) = mult~(succ\;(cast\;n))~(\textit{fac}\;n)
\end{align*}
where \textit{mult} is the multiplication function for natural numbers.
In addition to the abstract recursive call $\textit{fac} : r -> \textit{Nat}$,
\MPr\ supports the casting operation $cast : r -> \textit{Nat}$.
Therefore, we can convert from an abstract value ($n : r$) to
a concrete value ($cast\;n : \textit{Nat}$).

The casting operation also enables us to define a constant time tail function for
lists as follows:
\begin{align*}
\textit{tail}~x = \MIt^{\{\}}~x~\textbf{with}~~
&  tl ~cast~\textit{Nil}~~~~~~~~~~~ = nil \\
&  tl ~cast\;(\textit{Cons}\;x\;\textit{xs}) = cast\;\textit{xs}
\end{align*}
We simply need to return $cast\;\textit{xs}$ for the result
without recursing on \textit{xs} using $tl$. Nax can infer that
$\textit{tail} ~:~ \textit{List}\;p -> \textit{List}\;p$.
Note that we needed to return some default value ($nil$) for the empty list case
($Nil$) because all functions defined with Mendler style recursion combinators
must be total. 

However, for length index lists (or, vectors), we can imagine
a version of a tail function
$\textit{vtail}\,:\textit{Vec}\;p\;\{`succ\;n\} -> \textit{Vec}\;p\;\{n\}$,
which only expects non-empty lists -- therefore, we need not specify a default
value for the empty list case. In fact, we need not supply an empty-list case
since by type considerations it is unreachable. To express functions like
$vtail$, which have a constraint on the input index (\ie, it must be
$\{`succ\;n\}$), we need additional features in our type system, which advance
beyond what we have been discussing so far. We will discuss such possible
extensions to Mendler style recursion combinators in \S\ref{sec:mendler:etc}.

In order to translate Mendler style primitive recursion into
a normalizing calculus, we need a more complicated calculus than \Fi,
one we call \Fixi. Just as I extended \Fw\ to \Fi\ in order to formalize
the theory of \MIt\ involving term indices, I extend \Fixw\ to \Fixi\
in order to formalize the theory of \MPr\ involving term indices.

\subsection{Mendler style course of values iteration \& primitive recursion}
\label{sec:mendler:cv}
Some computations are naturally expressed by recursing over substructure
of the input value which is not the immediate substructure (\eg, predecessor
of predecessor, or tail of tail). This pattern of recursion is called
\emph{course of values} recursion. Course of values recursion often appears
in number sequences defined by recurrence relations (\eg, Fibonacci's sequence).
We can extend \MIt\ and \MPr\ with a new abstract operation, which enables
us to express course of values recursive functions, as follows:
\begin{align*}
\!\!\!\!\!\!\!\!
 \McvIt\, :&\; \forall t . \forall a .
   \big( \forall r .
         (\forall\vec{i} . r\,\vec{i} -> a\,\vec{i}\,)
   -> \overbraceset{\begin{smallmatrix}
                     \text{open $r$-structure to} \\
                     \text{reveal $t$-structure} \\
                    \end{smallmatrix}}{
         (\forall\vec{i} . r\,\vec{i} -> t\,r\, i) }
   ->    (\forall\vec{i} . t\, r\,\vec{i} -> a\,\vec{i}\,)
   \big) \\
& -> (\forall\vec{i} . (\mu^\kappa t)\,\vec{i} -> a\,\vec{i}\,) \\
\!\!\!\!\!\!\!\!
 \McvPr :&\; \forall t . \forall a .
   \big( \forall r .   
         (\forall\vec{i} . r\,\vec{i} -> a\,\vec{i}\,)
   -> \overbraceset{\begin{smallmatrix}
                     \text{open $r$-structure to} \\
                     \text{reveal $t$-structure} \\
                    \end{smallmatrix}}{
         (\forall\vec{i} . r\,\vec{i} -> t\,r\,i) }
   -> \overbraceset{\begin{smallmatrix}
                     \text{cast from $r$-values} \\
                     \text{to $(\mu^\kappa t)$-values} \\
                    \end{smallmatrix}}{
         (\forall\vec{i} . r\,\vec{i} -> (\mu^\kappa t)\,\vec{i}\,) }
   ->    (\forall\vec{i} . t\, r\,\vec{i} -> a\,\vec{i}\,)
   \big)\\ 
& -> (\forall\vec{i} . (\mu^\kappa t)\,\vec{i} -> a\,\vec{i}\,)
\end{align*}
\McvIt\ stands for Mendler style course of values iteration,
and \McvPr\ stands for Mendler style course of values primitive recursion.

We can define the Fibonacci function using \McvIt\ as follows:
\begin{align*}
\textit{fibonacci}~x = \McvIt^{\{\}}~x~\textbf{with}~~
&  \textit{fib} ~out ~\textit{Zero}~~~~\; = succ~zero \\
&  \textit{fib} ~out \;(\textit{Succ}\;n) =
            \textbf{case}^{\{\}}~out\;n~\textbf{of} \\
&\qquad\qquad\qquad\qquad\quad
              \textit{Zero}~~~\, -> succ~zero \\
&\qquad\qquad\qquad\qquad\quad
              \textit{Succ}\;n'  -> plus\;(\textit{fib}\;n')\;(\textit{fib}\;n)
\end{align*}
Recall that we cannot inspect the structure of the predecessor $n$ since
its type is abstract (\ie, $n:r$). The $out : r -> \textit{N r}\,$ operation
opens up the abstract structure ($r$), and reveals the base structure ($N$).
So, it becomes possible to destruct (or, pattern match against) $out\;n$,
and have access to the predecessor of predecessor $n'$. If needed, we can
traverse down three steps by applying $out$ to $n'$, and even further by
repeatedly applying $out$ to the inner structure. I also have several examples
that traverses down the inner structure a number of steps that depends
upon run-time values. These will appear in my dissertation.

An important aspect of the Mendler style course of values recursion is that
it only guarantees normalization over positive recursive types. One of our
recent contribution is that we have reported a counterexample of \textsf{McvIt}
involving negative recursive type, which does not normalize. So, to be more
precise, the type signatures of \textsf{McvIt} and \textsf{McvPr} need to
be constrained ($\forall t \text{~such that~$t$ is positive}.\cdots$)
instead of being unconstrainedly polymorphic on $t$.

I strongly believe that a reduction preserving embedding of
\McvIt\ and \McvPr\ into \Fixw\, and \Fixi\ are possible.
(I haven't written this down yet but I see it is going to be
quite straightforward.)

\subsection{Mendler style iteration with syntactic inverse}
Although \MIt\ is normalizing for negative recursive types, there exist many
total functions over negative recursive types, which are useful, but not easily
expressed in terms of \MIt. Most of these functions were first studied in
the conventional setting, and have definitions as recursive functions.
My recent contribution is the discovery of a new family of the Mendler style
recursion combinators, which can express these functions. We call this new
family \MsfIt. The generic type (abstracting over the number and kind of
indices) of this family is shown below:
\label{sec:mendler:sf}
\[
 \MsfIt : \forall t . \forall a .
   \big( \forall r.
            (\forall\vec{i} . r\,\vec{i} -> a\,\vec{i}\,)
   -> \overbraceset{\begin{smallmatrix}
                        \text{inverse from} \\
                        \text{answers to $r$-values} \\
                       \end{smallmatrix}}{
            (\forall\vec{i} . a\,\vec{i} -> r\,\vec{i}\,) }
   -> \;    (\forall\vec{i} . t \, r\,\vec{i} -> a\,\vec{i}\,)
   \big)
 -> (\forall\vec{i} . (\mu^\kappa t)\,\vec{i} -> a\,\vec{i}\,)
\]
In addition to the abstract recursive call, \MsfIt\ supports an abstract
inverse from answer values ($a \,\vec{i}\,$) to abstract recursive values
($r\,\vec{i}\,$). Several examples and a detailed discussion about \MsfIt\
can be found in our recent work \cite{AhnShe11} (where we call this family
\textit{msfcata}). We \cite{AhnShe11} proved normalization properties of
\MsfIt\ by embedding \MsfIt\ into \Fw. I plan to extend the proof for \MsfIt\,
to a language with term-indexed types, by embedding \MsfIt\ into \Fi, which is
an extension of \Fw.

I plan to construct more complex examples
(\emph{normalization by evaluation} and \emph{parallel reduction})
in the Nax language using \MsfIt.

I will also discuss why we have not extended the primitive recursion family
and the course of values family with syntactic inverse (such extensions
may lead to possible non-termination).

\subsection{Mender style recursion combinators over indexed types}
\label{sec:mendler:ix}
Mendler style recursion combinators generalize naturally to indexed types.
In this section, I will illustrate the use of Mendler style
recursion combinators over indexed types. Here, the examples will only
involve the iteration family (\MIt), but you will have not difficulty
imagining how to uses of other families over indexed types, since all
Mendler style recursion combinators generalize naturally to indexed types
in the same fashion.

The following is the base structure for vectors, or length indexed lists:
\begin{align*}
\textbf{data}~V : * -> (Nat -> *) -> Nat -> * ~\textbf{where}~~
& \textit{VNil}~~\, : V\;p\;r\;\{`zero\} \\
& \textit{VCons}    : r\;\{n\} -> p -> V\;p\;r\;\{`succ\;n\}
\end{align*}
We use curly braces ($\{\dots\}$) to emphasize that $\{`zero\}$ and
$\{`succ\;n\}$ are \emph{term indices} (\ie, term used as indices) in order to
distinguish them from ordinary type indices (\ie, types used as indices).
The backquotes ($`$) prefixed on $zero$ and $succ$ emphasizes that
they are predefined values, bound in the current term environment
mapping term-variables to values. Other variables (\eg, $p$,$r$,and $n$)
are free type (or index) variables that will be generalized.

We define the length function for vectors using the Mendler style
iteration combinator as follows:
\begin{align*}
\textit{vlength}~l = \MIt^{\{\{n\} . \textit{Nat}\}}~l~\textbf{with}~~
&  vlen ~\textit{VNil}~~~~~~~~~~~ = zero \\
&  vlen \;(\textit{VCons}\;x\;\textit{xs}) = succ\,(vlen\;\textit{xs})
\end{align*}
Note, the Mendler style iteration combinator $\MIt^\psi$ is annotated by
the index transformer $\psi = \{\{n\} . \textit{Nat}\}$. This is
a type-level constant function that ignores the index $n$ and always returns
the type $\textit{Nat}$. In general, type inference for programs involving
type indices is undecidable without some annotation describing how the indices
vary over each of the \textsf{with} clauses. However, with the
index transformer annotation, Nax can infer that
$vlengh : \textit{Vec}\;p\;\{n\} -> \textit{Nat}$,
where \textit{Vec} is a type synonym defined as
$\textit{Vec} = \mu^{\textit{Nat} -> *}(V\;p)$.
Index transformers describe the minimal amount of type information required,
and also describe where such information us necessary (on constructs that
analyze data -- \textbf{case} and Mendler style combinators).
They are required only on data that has indices, and are conservative over
Hindley-Milner type inference. That is, every program typable using
Hindley-Milner has a Nax counterpart which also requires no type annotation
whatsoever.


The append function for vectors illustrates a more interesting use of
the index transformer annotation:
\begin{align*}
\textit{vappend}~\,l_1 =~ &
 \MIt^{\{\{n\} . \textit{Vec}\;p\;\{m\} \, -> \, \textit{Vec}\;p\;\{`plus\;n\;m\}\}}~l_1
 ~\textbf{with} \\
&\qquad\qquad\qquad\qquad  vapp ~\textit{VNil}~~~~~~~~~~~ ~ l_2 = l_2 \\
&\qquad\qquad\qquad\qquad  vapp \;(\textit{VCons}\;x\;\textit{xs}) ~ l_2 = vcons\;x\;(vapp\;\textit{xs}\;l_2)
\end{align*}
where $vcons\;x\;\textit{xs} = \In^{\textit{Nat} -> *}(\textit{VCons}\;x\;\textit{xs})$.
The index transformer
$\{\{n\} . \textit{Vec}\;p\;\{m\} \, -> \, \textit{Vec}\;p\;\{`plus\;n\;m\}\}$
specifies that the append function applied to a list of length $n$,
shall return a function, which expects a vector with length index $m$
and returns a vector with length index $(plus\;n\;m)$, where $plus$ is
the addition function for natural numbers.

We can also use indexed types to encode mutually recursive types.
For instance, consider the following definitions:
\begin{align*}
&\textbf{data}~ \textit{Parity} = E \mid O \\
&\textbf{data}~ \textit{Pf}\, : (\textit{Nat} -> \textit{Parity} -> *) ->
                                    \textit{Nat} -> \textit{Parity} -> *
  ~\textbf{where}\\
&\quad \textit{BaseE}\, : \textit{Pf}~\,r\;\{`zero\}\;\{E\} \\
&\quad \textit{StepE}\; : r\;\{n\}\;\{O\} -> \textit{Pf}~\,r\;\{`succ\;n\}\;\{O\} \\
&\quad \textit{StepO}\; : r\;\{n\}\;\{E\} -> \textit{Pf}~\,r\;\{`succ\;n\}\;\{O\}
\end{align*}
Let $\textit{Proof} = \mu^{\textit{Nat} -> \textit{Parity} -> *} \textit{Pf}$,
$\textit{Even}\;\{n\} = \textit{Proof}~\{n\}\;\{E\}$, and
$\textit{Odd}\;\{n\} = \textit{Proof}~\{n\}\;\{O\}$ be type synonyms.
Then, \textit{Even} and \textit{Odd} are encodings of the following
mutually recursive datatypes:
\begin{align*}
&\textbf{data}~\textit{Even} : \textit{Nat} -> \textit{Parity} -> *
 ~\textbf{where} \\
&\quad \textit{BaseE}\, : \textit{Even}\;\{`zero\}\\
&\quad \textit{StepE}\; : \textit{Odd}\;\{n\} -> \textit{Even}\;\{`succ\;n\} \\
&\textbf{data}~\textit{Odd} : \textit{Nat} -> \textit{Parity} -> *
 ~\textbf{where} \\
&\quad \textit{StepO}\; : \textit{Even}\;\{n\} -> \textit{Odd}\;\{`succ\;n\}
\end{align*}
In the thesis, I will include other examples involving indexing and
mutually recursive datatypes.


\paragraph{The theory of term-indexed types.}
It is known that \MIt\ for regular datatypes and indexed types with ordinary
type indices has a reduction preserving embedding into System \Fw\ (\ie, \MIt\
is definable in \Fw\ with a constant reduction step difference).
In my dissertation, I will show that \MIt\ for term-indexed types has a
reduction preserving embedding into System \Fi, which is an extension of \Fw.
Also, I will describe and prove properties of the inference algorithm of Nax,
and also discuss why the index transformer is a sweet spot for the design of
type systems suppering type inference in the presence of indexed types.



\subsection{Other possible extensions}\label{sec:mendler:etc}

In section \S\ref{sec:mendler:pr}, I mentioned that we 
need to extend the type system to express the tail function which can only 
be applied to non-empty vectors (or, length indexed lists).
In particular, we need \emph{equality constraints} and
the \emph{unreachable} term (evidence of a logical contradiction).
\begin{align*}
\textit{vtail}~x =~& \MIt^{\{\{n\} \,|\, n=`succ\;m \; . \, \textit{Vec}\;p\;\{n\} -> \textit{Vec}\;p\;\{m\}\}}~x~\textbf{with}\\
&\qquad  tl ~cast~\textit{VNil}~~~~~~~~~~~ = \textit{unreachable} \\
&\qquad  tl ~cast\;(\textit{VCons}\;x\;\textit{xs}) = cast\;\textit{xs}
\end{align*}
Under the equality constraint ($n=`succ\;m$), the length index of
the input list ($n$) must be non-zero. When the type checker types
the term \textit{unreachable}, it should try to show that the current
constraints are unsatisfiable. In first equation, we know that $n$ is zero
because \textit{VNil}'s type is indexed by zero. Since $zero=succ\;m$ is not
satisfiable, \textit{unreachable} should type check vacuously, since it marks
a path that will never be taken in a well-typed program. It might be possible
to use some syntactic sugar that inlines the equality constraint and
checks that every omitted equation is unreachable. Then, we could write
the $vtail$ function as follows:
\begin{align*}
\textit{vtail}~x =~& \MIt^{\{ \{`succ\;m\}. \textit{Vec}\;p\;\{`succ\;m\} -> \textit{Vec}\;p\;\{m\}\}}~x~\textbf{with}\\
&\qquad  tl ~cast\;(\textit{VCons}\;x\;\textit{xs}) = cast\;\textit{xs}
\end{align*}
In my dissertation, I plan to study such extensional features that enable
programmers to write useful functions like $vtail$.

The $vtail$ example above involves converting the type of terms using
definitional equality (\ie, $\beta$-conversion). As we reason about more
complicated functions, with more interesting properties, we will need to
represent provable equality. An example of provable equality is
the associativity of natural numbers over addition. Using the standard
definition of addition (call it $plus$) for unary natural numbers, the two terms
$(plus\;(plus\;n_1\;n_2)\;n_3)$ and $(plus\;n_1\;(plus\;n_2\;n_3))$,
are not convertible. But, we can prove by induction that they are always equal
for any three natural numbers $n_1$, $n_2$, and $n_3$. The addition of
provable equality will extend the programs we can type. Of course,
to construct such proofs, we will need an induction principle. Fortunately,
every Mendler style operator is an induction principle. This is one reason
why a language built on Mendler style recursion combinators make a perfect
substrate for a sound logical reasoning system. Which, in case you don't recall,
is the first part of my thesis statement!

%%%% TODO Mention Marcelo's idea on richer calculi????

%% \subsection{summary???}
%% G) Theory?
%%    formalization, typing, dynamic semantics, type inference
%%    statements (and perhaps proofs) of properties that should hold.
%%    Strong normalization, soundness with respect to dynamic semantics,
%%    logical soundness (F-omega argument)

