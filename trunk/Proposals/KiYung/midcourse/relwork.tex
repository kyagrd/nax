\section{Background and Related Work}\label{sec:relwork}

In this section, I will first introduce the Curry-Howard correspondence 
(\S\ref{sec:relwork:prelim}), a preliminary concept needed for designing
a system that unifies logical reasoning and programming. Second, I will
introduce a major challenge (normalization) to realizing such a unified system.
Third, I will position my approach by comparing it with other approaches with
similar goals (\S\ref{sec:relwork:other}). Finally, I will give a short tour of
the theory and previous work related to my approach (\S\ref{sec:relwork:bg}).

\subsection{The Curry-Howard correspondence and Normalization}
\label{sec:relwork:prelim}
One promising approach to designing a system that unifies
logical reasoning and programming is \emph{the Curry-Howard correspondence}.
Howard \cite{Howard69} observed that a typed model of computation
(\ie, a typed lambda calculus) gives an interpretation to a (natural deduction)
proof system (for an intuitionistic logic). More specifically, one can interpret
a type (in the lambda calculus) as a formula (in the logic) and
a term of that type, as a proof for that formula. For instance,
the typing rule for function applications (APP) in a typed lambda calculus
corresponds to Modus Ponens (MP) in a logic:
\[ \inference[(APP)]
      {\Gamma |- t_1 : A -> B & \Gamma |- t_2 : A}
      {\Gamma |- t_1~t_2 : B}
 ~~~~~~~~
   \inference[(MP)]
      {A -> B & A}
      {B}
\]
As you can see above, combining terms ($t_1$ and $t_2$) to build a new term
($t_1~t_2$) can be interpreted as combining proofs for formulae
($A -> B$ and $A$), to construct a proof for a new formula ($B$).
More generally, we may expect that programming (\ie, building larger terms)
corresponds to constructing larger proofs only when the typed lambda calculi
meets certain standards -- \emph{type soundness} and \emph{normalization}.

The Curry-Howard correspondence is a promising approach to designing a
unified system for both logical reasoning and programming. Only one language
system is needed for both the logic and the programming language. An
alternate approach is to use an external logical language to talk about
programs as the objects that the logic reasons about. In this approach, one
has the obligation to argue that the soundness of the logic, with respect to
the programming language semantics, holds.

Under the Curry-Howard correspondence, the logic is internally related to the
semantics of program -- there is no need to argue for the soundness of the
logic,  externally outside of the programming language system. The soundness
of the logic follows directly from the type soundness of the language under
the Curry-Howard correspondence.

Let us consider a proposition to be true
(or, valid) when it has a canonical (\ie, cut-free) proof.
That is, there is a program, whose type is the proposition under
consideration, and that program has a normal form. 
By type soundness, any term,
of that type, will preserve its type during the reduction steps. Thus
reduction preserves truthfulness. If we assume
that the language is normalizing (\ie, every well-typed term reduces to
a normal form), then any term of that type which is a non-canonical proof,
implies the existence of a canonical proof, which in turn implies that
the proposition specified by the type is indeed true. That is, all provable
propositions are valid (\ie, the logic is sound) when the language is
\emph{type sound} and \emph{normalizing}.

\emph{Normalization} is also essential for the consistency of the logic.
For the lambda calculus to be interpreted as a \emph{consistent} logic,
there must be no diverging terms. A diverging term (\ie, a term that does
not have a normal form) may inhabit any arbitrary type. Thus, a diverging term
can be a proof for any proposition under the Curry-Howard correspondence.
General purpose functional programming languages (\eg, Haskell and ML), that
support unrestricted general recursion, cannot be interpreted as a consistent
logic, since they allow diverging terms (non-terminating programs).
For example, a diverging Haskell definition $\textit{loop} = \textit{loop}$
can be given an arbitrary type such as
$\textit{loop}\mathrel{::}\textit{Bool}$,
$\textit{loop}\mathrel{::}\textit{Int} -> \textit{Bool}$,
and even $\textit{loop}\mathrel{::}\forall a. a$, which is a proof of false.


Therefore, useful logical reasoning systems based on the Curry-Howard
correspondence (\eg, Coq, Agda) never support language features that can
lead to diverging terms. For example, in both Coq and Agda,
unrestricted general recursion (at term level) is not supported. 
Instead, these logical reasoning systems
often provide principled recursion schemes over recursive types that are
guaranteed to normalize. 

Recursive types (\ie, recursion at type level)
can also lead to diverging terms when they are not restricted carefully.
Many of the conventional logical reasoning systems, based on
Curry-Howard correspondence, restrict recursive types in a way,
which is not an ideal design choice, if one's goal is a unified system for
logic and programming. My approach explores another design space not yet
completely explored. I will introduce both approaches to restricting recursive
types to ensure normalization in the following two subsections.

\subsection{Restriction on recursive types for normalization}
\label{sec:relwork:other}
We have argued that normalization is essential for logical reasoning systems
based on the Curry-Howard correspondence. One challenge to the successful
design of such systems is how to restrict recursion at the type level
so that normalization of terms is preserved. 
There are two different
design choices illustrated in Figure \ref{fig:approaches}. 
The conventional approach restricts the formation
of recursive types (\ie, the restriction is in datatype definition), and
the Mendler style approach restricts the elimination
of the values of recursive types (\ie, the restriction is in pattern matching).

\begin{figure}
{\centering
\begin{tabular}{p{3cm}|p{12.5cm}}
\parbox{3cm}{~~Functional\\programming\\$~~~~$language} &
\parbox{12.5cm}{
 kinding:~
  \inference[($\mu$-form)]{\Gamma |- F : * -> *}{\Gamma |- \mu F : *} \\
 \\
 typing:\quad
  \inference[($\mu$-intro)]{\Gamma |- t : F (\mu F)}{\Gamma |- \In~t:\mu F} ~~~~
  \inference[($\mu$-elim)]{\Gamma |- t : \mu F}{\Gamma |- \unIn~t : F (\mu F)}\\
 \\
 reduction:
  \inference[(\unIn-\In)]{}{\unIn~(\In~t) \rightsquigarrow t}
} \\
\\ \hline\hline
\parbox{3cm}{$~$Conventional\\$~~~$approach for\\consistent logic} &
\parbox{12.5cm}{$\phantom{a}$\\
 kinding:~
  \inference[($\mu$-form$^{+}$)]{ \Gamma |- F : * -> * 
                           & \mathop{\mathsf{positive}}(F)}
                           {\Gamma |- \mu F : *} \\
 \\
 typing:~
  \text{{\small($\mu$-intro)} and {\small($\mu$-elim)}
                same as functional language} \\
  \[\inference[(\It)]{\Gamma |- t : \mu F & \Gamma |- \varphi : F A -> A}
                     {\Gamma |- \It~\varphi~t : A}\]
 reduction:~ \text{{\small(\unIn-\In)} same as functional language}
  \[\inference[(\It-\In)]{}{\It~\varphi~(\In~t) \rightsquigarrow
                            \varphi~(\textsf{map}_F~(\It~\varphi)~t)}\]
}
\\ \hline
\parbox{3cm}{Mendler style\\$~~$approach for\\consistent logic} &
\parbox{12.5cm}{$\phantom{a}$\\
 kinding:~ \text{{\small($\mu$-form)} same as functional language} \\
 \\
 typing:~
  \text{{\small($\mu$-intro)} same as functional language}
  \[\inference[(\MIt)]
     { \Gamma |- t : \mu F &
       \Gamma |- \varphi : \forall X . (X -> A) -> F X -> A}
     {\Gamma |- \MIt~\varphi~t : A} \]
 reduction:~
  \inference[(\MIt-\In)]
     {}
     {\MIt~\varphi~(\In~t) \rightsquigarrow \varphi~(\MIt~\varphi)~t}
}
\end{tabular} }
\caption{Two different approaches to designing a logic
         (in contrast to functional languages)}
\label{fig:approaches}
\end{figure}

\paragraph{Recursive types in functional programming languages.}
Let us start with a review of the theory of recursive types used
in functional programming languages. Here, the term
language is not expected to be normalizing, so restrictions are few.

Just as we can capture the essence of unrestricted general recursion at term
level, by a fix point operator (usually denoted by \textsf{Y} or \textsf{fix}),
we can capture the essence of recursive types by the
use of fix point operator, $\mu$, at type level. 
The rules for the formation {\small($\mu$-form)},
introduction {\small($\mu$-intro)}, and elimination {\small($\mu$-elim)} of
the recursive type operator $\mu$ are described in Figure \ref{fig:approaches}.
We also need a reduction rule {\small(\unIn-\In)}, which relates \In,
the data constructor for recursive types, and \unIn, the destructor for
recursive types, at the term level.

Surprisingly (perhaps), the recursive {\em type} operator,\ $\mu$, as described in
Figure \ref{fig:approaches}, is already powerful enough to express 
non-terminating programs, even without introducing the recursive {\em term}
operator, \textsf{fix}, to the language. We illustrate this below. First a 
short reminder of how a fix point at the term level operates. The typing rule
and the reduction rule for \textsf{fix} can be given as follows:
\[ \text{typing:}~ \inference{\Gamma |- f : A -> A}{\textsf{fix}\,f : A}
 \qquad\qquad
   \text{reduction}:~ \textsf{fix}\,f \rightsquigarrow f(\textsf{fix}\,f)
\]
We can actually implement \textsf{fix}, using $\mu$, as follows
(using some Haskell-like syntax):
\begin{align*}
& \textbf{data}~T\;a\;r = C\;(r -> a) \quad
          \texttt{-}\texttt{-}~\text{\small a non-recursive datatype} \\
& w \,:\, \mu(T\;a) -> a ~~ \qquad \qquad
          \texttt{-}\texttt{-}~\text{\small an encoding of the untyped
                                     $(\lambda x.x\;x)$
                                     in the typed $\lambda$-calculus}~ \\
& w = \lambda x . \,\textbf{case}~\unIn~x~\textbf{of}~C\;f -> f\;x \\
& \textsf{fix} \,:\, (a -> a) -> a \qquad \qquad
          \texttt{-}\texttt{-}~\text{\small an encoding of 
                                     $(\lambda f.(\lambda x.f(x\;x))\,
                                                 (\lambda x.f(x\;x)))$} \\
& \textsf{fix} = \lambda f. (\lambda x. f (w\;x))\,(\In(C(\lambda x. f (w\;x))))
\end{align*}

Thus, to avoid the loss of termination guarantees, we need to alter the rules
for $\mu$, in someway, to ensure a consistent logic. One way, is to restrict
the rule {\small $\mu$-form}; the other way, is to restrict the rule
{\small $\mu$-elim}. Once we decide which of these two alterations of the
rules we will use, the design of principled recursion combinators (\eg, \It\
for the former and \MIt\ for the latter) follows from that choice.

\paragraph{Recursive types in the conventional approach to consistent logic.}
In the conventional approach, the formation (\ie, datatype definition) of
recursive types is restricted, but arbitrary elimination (\ie, pattern matching)
over the values of recursive types is allowed. In particular, the formation of
negative recursive types is restricted. Only positive recursive types are
supported. Thus, in Figure \ref{fig:approaches}, we have a restricted version of
the formation rule {\small($\mu$-form$^{+}$)} has an additional condition that
require $F$ to be positive. The other rules {\small($\mu$-intro)},
{\small($\mu$-elim)}, and {\small(\unIn-\In)} remain the same as for
functional languages. Since we have restricted the recursive types
at the type level and we do not have general recursion at the term level,
the language is indeed normalizing. However, we can neither write
interesting (\ie, recursive) programs that involves recursive types nor
inductively reason about those programs, unless we have principled recursion
schemes that are guaranteed to normalize. One such recursion scheme is called
iteration (\aka\ catamorphism). The typing rules for the conventional iteration
\It\ are illustrated in Figure \ref{fig:approaches}. Note, we have the typing
rule {\small(\It)} and the reduction rule {\small(\It-\In)} for \It\,
in addition to the rules for the recursive type operator $\mu$.

\paragraph{Recursive types in the Mendler style approach to consistent logic.}
In the Mendler style approach, we allow arbitrary formation
(\ie, datatype definition) of recursive types, but we restrict
the elimination (\ie, pattern matching) over the values of recursive types. 
The formation rule {\small($\mu$-form)} remains the same as
for functional languages. That is, we can define arbitrary recursive types,
both positive and negative. However, we no longer have the elimination
rule {\small($\mu$-elim)}. That is, we are not allowed to pattern match over
the values of recursive types in the normal fashion. We can only pattern match
over the values of recursive types through the Mendler style recursion
combinators. The rules for the Mendler style iteration combinator \MIt\
are illustrated in Figure \ref{fig:approaches}.
Note, there are no rules for \unIn\ in the Mendler style approach.
The typing rule {\small($\mu$-elim)} is replaced by {\small(\MIt)} and
the reduction rule {\small(\unIn-\In)} is replaced by {\small(\MIt-\In)}.
More precisely, the typing rule {\small \MIt} is both an elimination rule
for recursive types and a typing rule for the Mendler style iterator.
You can think of the rule {\small(\MIt)} as replacing both the elimination rule
{\small($\mu$-elim)} and the typing rule for conventional iteration
{\small(\It)}, but in a safe way that guarantees normalization.

\paragraph{Justification of the Mendler style as a design choice.}
I choose to base my approach to the design of a seamless synthesis of both
logic and programming on the Mendler style. It restricts the elimination (\ie,
pattern matching) over values of recursive types, rather restricting the
formation (\ie, datatype definition) of recursive types (a more conventional
approach). The impact of this design choice is that it enables the logic to
include all datatype definitions that are used in functional programming
languages.

Functional programming promotes ``functions as first class values''.
It is natural to pass functions as arguments and embed functions into
(recursive) datatypes. If embedding functions in datatypes is allowed,
we can embed a function whose domain is the very type we are defining.
For example, the recursive datatype definition
$\mathbf{data}~T = C\;(T -> \textit{A})$ in Haskell is such a recursive
datatype definition. Such datatypes are called negative recursive datatypes
since the recursive occurrence $T$ appears in a negative position.
We say that $T$ is in a negative position, since $(T -> A)$ is analogous to
$(\neg T \land A)$ when we think of $->$ as a logical implication. There exist
both interesting and useful examples in functional programs. I will
introduce several such examples, and discuss them, in my dissertation.

Recall that the goal of my dissertation (quoting again from \S\ref{sec:intro})
is to contribute to answering the question of {\em how does one build a
seamless system where programmers can both write (functional) programs and
formally reason about those programs}. Under the Curry-Howard correspondence,
to formally reason about a program, the logic needs to refer to the type of
the program, since the type, interpreted as a proposition, describes its
properties. Since the Mendler style approach does not restrict recursive
datatype definitions, we can directly refer to the types of programs that use
negative recursive types.

The Mendler style is a promising approach to building a unified system because
all the recursive types, both positive and negative, are definable in the
logic. Although the conventional approach is widely followed in the design of
formal reasoning systems based on the Curry-Howard correspondence (\eg, Coq,
Agda), it cannot directly refer to programs that use non-positive recursive
types. One may object that it is possible to indirectly model negative
recursive types in the conventional style, via alternative equivalent
encodings which map negative recursive types into positive ones. But, such
encodings do not meet our goal of designing a seamless unified system for both
programming and reasoning -- programmers should not have to pre-process their
programs in order to reason about them.

Another benefit of the Mendler style, is that Mendler recursion combinators
are both useful and well-behaved (\ie, guaranteed to normalize) for arbitrary
recursive types. A Mendler combinator is an induction principle
(principled recursion scheme). It would not be meaningful to have
arbitrary recursive types in the logic unless we have useful and well-behaved
induction principles to reason about the programs using those recursive types.
This is especially true for the negative recursive types.

The Mendler style iteration combinator family, \MIt, is a well-behaved
combinator for both positive and negative recursive types.
For positive recursive types, \MIt\ is as expressive as the conventional style
iteration combinator \It\ (\ie, both can be defined in terms of each other). 
Furthermore, \MIt\ behaves well (\ie, always normalizes), even for negative
recursive types.

There exist other families of Mendler style recursion combinators,
which are even more useful than \MIt\ for negative recursive types,
one of which I have discovered in my thesis research. Throughout
my dissertation, I will show that the Mendler style recursion combinators are
indeed useful and well-behaved induction principles. You can find brief
summary of the Mendler style recursive schemes in this mid-course document
in \S\ref{sec:mendler}.


\subsection{Background and previous work related to my approach}
\label{sec:relwork:bg}
\subsubsection{A summary on the studies of the Mendler style}
\citet{Mendler87} discovered an interesting way of formalizing
primitive recursion, which was later dubbed ``the Mendler style'',
while he was formalizing a logic that extended System \textsf{F} with
primitive recursion. Interestingly, Mendler did not seem to notice
(or maybe, just did not bother to mention) that his style of formalizing
primitive recursion also guaranteed normalization for non-positive recursive
types -- Mendler required recursive types to be positive in his extension of
System \textsf{F}. A decade later, \citet{matthes98phd} and \citet{uustalu98phd}
noticed that Mendler never used the positivity condition in his proof of
strong normalization.

\citet{AbeMat04} generalized Mendler's primitive recursion combinator
\cite{Mendler87} into a family of combinators that are uniformly defined for
type constructors of arbitrary kinds. This was necessary for
handling nested datatypes. Their system extends System \Fw\ 
(\citet{Mendler87} extends System \textsf{F}). The notion
of a kind indexed family of Mendler combinators has now become the norm.

\citet{AbeMat04} prove strong normalization of their language \textsf{MRec},
which extends System \Fw by adding a family of kind-indexed Mendler style
primitive recursion combinators. They show that \textsf{MRec} has
a reduction preserving embedding into a calculus they call \Fixw.
Then, they show that \Fixw\ is strongly normalizing.

Abel, Matthes, and Uustalu \cite{AbeMatUus03,AbeMatUus05} studied
a kind-indexed family of iteration combinators, along with examples
involving nested datatypes that make use of those combinators.
Iteration (\aka\ catamorphism) is a recursion scheme, which has the same
computational power as primitive recursion (\ie, both can be defined
in terms of each other), but has different algorithmic complexity. 

It is strongly believed that primitive recursion is more efficient than
iteration. For instance, it is trivial to define a constant time predecessor
for natural numbers with primitive recursion, but it is believed impossible
to define the constant time predecessor with iteration. The Mendler style
iteration family can be embedded into \Fw\ in a reduction preserving manner.
That is, we can encode the family of Mendler style iteration combinators
into \Fw\ in such a way that the number of reduction steps of the original
and the embedding differ only by a constant number of steps. The primitive
recursion family, in contrast, is not believed to have a reduction preserving
embedding into \Fw. \citet{AbeMat04} needed a more involved embedding of
\textsf{MRec} into \Fixw, which has a richer structure than \Fw.

Although Matthes, Uustalu, and others, were well aware of the fact that
the Mendler style iteration family and the primitive-recursion family both
normalize for negative recursive, they did not explore or document actual
examples. They postponed ``the search for exciting examples of negative
recursive types" until another time. They stated that the normalization
of negative types ``may have a theoretical value
only''\cite{UusVen99}. So, until recently, the study on Mendler style recursion
combinators focused on examples of positive recursive types with type (rather than term) based indexing.

\subsubsection{Recent contributions of my thesis work}
I have developed several new contributions to the study of the Mendler style
in my thesis work. These contributions fall into 3 broad categories.
\begin{itemize}
\item Some Mendler operators terminate for negative datatypes, and are also quite expressive.
\item Some Mendler operators are restricted to positive datatypes.
\item Mendler operators can be extended to term indexed types (GADTs).
\end{itemize}
Each topic will be thoroughly described in the final thesis, and is outlined here.


\paragraph{A new family of Mendler style recursion combinators for negative datatypes.}

While it is known that iteration and primitive recursion terminate for all types
\cite{AbeMatUus05,AbeMat04}, they are not particularly expressive over negative
recursive types. Identifying additional Mendler style operators that work
naturally, and are more expressive than iteration, has been an important result
of my thesis work.

Interesting examples of Mendler style operators over negative recursive types
have been neglected in the literature. One of the reasons, I think, is because
it is possible to encode negative recursive types into positive recursive ones. Because conventional iteration and primitive recursion normalize for
positive recursive types we can use standard techniques on these encodings
of translating negative recursive types into positive recursive types.
What we gain by using such encodings must be traded against the loss in
transparency that such encodings force upon the implementation. The natural
structure, which were evident in the negative data type, become obscured by
the encoding.

A series of papers \cite{Pat93,MeiHut95,FegShe96,DesPfeSch97,bgb} studied 
techniques that define recursion schemes over negative recursive types in
the conventional setting. In our recent work \cite{AhnShe11}, we discovered
that this work can be naturally captured as a kind-indexed family of Mendler
tyle combinator. The \MsfIt\ combinator (\aka\ \textit{msfcata}) corresponds to
the conventional recursion combinator discovered by \citet{FegShe96} and
later refined by \citet{bgb}. With this new family, we were able to write many
interesting programs, involving negative recursive types, that are impossible,
or very unnatural, to write with just Mendler style iteration.

\paragraph{Some Mendler style recursion combinators need
           a positivity restriction.}
One of the results of my thesis research has been to identify that several
Mendler style operators require a positivity requirement. 
Recent work \cite{AhnShe11} shows that course-of-values iteration
\cite{vene00phd,UusVen02} actually needs a positivity condition on
recursive types to be well-defined. That is, Mender style course-of-values
iteration only normalizes for positive recursive types. In the special case
that a datatype uses only positive occurrences (the normal case for first-order
data) more expressive terminating Mendler style operators are possible.

\paragraph{Examples involving indexed datatypes.} 
We \cite{AhnShe11} have illustrated that Mendler style recursion combinators
can be defined not only for nested datatypes but also for arbitrary indexed
datatypes, whose can index may vary in the result types of the
data constructors as well as in the argument types of the data constructors.
For example, many examples using Generalized Algebraic DataTypes (GADTs),
a popular extension to the type system of Haskell, involves indexed datatypes
beyound the scope of nested datatypes.

Previous work, that promoted the expressiveness and generality of
the Mendler style \cite{AbeMatUus03,AbeMatUus05,AbeMat04} used examples
involving nested datatypes, a particularly simple kind of indexing
where the variation of indices are restricted to the argument types of
the data constructors. In order to handle nested datatypes, the translation of
the Mendler style combinators into System \textsf{F}, had to be extended to
translation into \Fw. To translate Mendler style combinators over arbitrary
indexed types including term indices we have to extend \Fw, to a new calculus
we call \Fi, and prove that this new calculus is also strongly normalizing.
I have identified what the extensions must look like and have been developing
the theoretical background necessary to accomplish this task.
This theory is an important part of the thesis and is elaborated in
\S\ref{Fi} and \S\ref{sec:Fi}.

Despite, not having a completed theory to support the translation of GADTs,
we have identified many examples \cite{AhnShe11}. Including examples
that use term indices to exploit the Curry-Howard isomorphism, and
examples of mutual recursion (between types) encoded in terms of GADTs.

In general, we have pushed the boundaries of what is known about
the Mendler style operators in many ways. We have identified new domains
in which they are applicable, identified new operators, adapted old ones,
and organized them all into a natural hierarchy that explains how all are
instances of a common pattern. However, we know of no guarantee that every
principled recursion scheme has a Mendler style operator, so there is
much work still to be done.


\subsubsection{Work in progress} \label{Fi}

The thesis will also contain two chapters on work in progress.

\paragraph{System \Fi.}
I am developing a typed lambda calculus called System \Fi\, which is an
extension of System \Fw\ with term-indexed types. Term-indexed types
naturally support the Curry-Howard isomorphism, as a type constructor can be
seen as a relation between one or more terms.  This kind of expressiveness is
available in dependently typed languages (where there is no distinction between
types and terms), and is also encodable using GADTs in functional languages
such as Haskell.  GADTs in Haskell are often indexed by uninhabited types
which are isomorphic to terms. The type indices used in GADTs in Haskell are
often really intended to be values at term level, but are just simulations of
such terms at type level.  For example one might introduce uninhabited type
constructors \verb+Zero+ and \verb+Succ+ in order to simulate natural numbers
when defining length indexed lists, as follows:
\begin{verbatim}
data Zero
data Succ x

data Vector a n where
  Nil:: Vector a Zero
  Cons:: a -> Vector a n -> Vector a (Succ n)
\end{verbatim}  

Within System \Fi, we can rigorously define types indexed by terms,
while keeping the term level and the type level cleanly separated.
We can also argue about the properties of the Mendler style combinators
over indexed types.  You can find a summary of \Fi\ in \S\ref{sec:Fi}.

\paragraph{The programing language Nax.}
I am also designing a surface language called Nax that supports
non-recursive datatypes, a recursive type operator $\mu$, and
Mendler style recursion combinators as language constructs.
Nax supports Hindley-Milner style type inference, with a few type annotations
for indexed types. Nax is designed to be embeddable into System \Fi.
You can find a summary of our progress in defining Nax in \S\ref{sec:Nax}.

