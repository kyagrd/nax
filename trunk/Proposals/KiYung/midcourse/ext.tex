\section{Possibilities for further work}\label{sec:ext}
There are many extensions that would make Nax more attractive
as a programming language. Here, I list some of them.




% \subsection{Other possible extensions}\label{sec:mendler:etc}

\paragraph{Equality types.}
In section \S\ref{sec:mendler:pr}, I mentioned that we 
need to extend the type system to express the tail function which can only 
be applied to non-empty vectors (or, length indexed lists).
In particular, we need \emph{equality constraints} and
the \emph{unreachable} term (evidence of a logical contradiction).
\begin{align*}
\textit{vtail}~x =~& \MIt^{\{\{n\} \,|\, n=`succ\;m \; . \, \textit{Vec}\;p\;\{n\} -> \textit{Vec}\;p\;\{m\}\}}~x~\textbf{with}\\
&\qquad  tl ~cast~\textit{VNil}~~~~~~~~~~~ = \textit{unreachable} \\
&\qquad  tl ~cast\;(\textit{VCons}\;x\;\textit{xs}) = cast\;\textit{xs}
\end{align*}
Under the equality constraint ($n=`succ\;m$), the length index of
the input list ($n$) must be non-zero. When the type checker types
the term \textit{unreachable}, it should try to show that the current
constraints are unsatisfiable. In first equation, we know that $n$ is zero
because \textit{VNil}'s type is indexed by zero. Since $zero=succ\;m$ is not
satisfiable, \textit{unreachable} should type check vacuously, since it marks
a path that will never be taken in a well-typed program. It might be possible
to use some syntactic sugar that in lines the equality constraint and
checks that every omitted equation is unreachable. Then, we could write
the $vtail$ function as follows:
\begin{align*}
\textit{vtail}~x =~& \MIt^{\{ \{`succ\;m\}. \textit{Vec}\;p\;\{`succ\;m\} -> \textit{Vec}\;p\;\{m\}\}}~x~\textbf{with}\\
&\qquad  tl ~cast\;(\textit{VCons}\;x\;\textit{xs}) = cast\;\textit{xs}
\end{align*}
In my dissertation, I plan to study such extensional features that enable
programmers to write useful functions like $vtail$.


\paragraph{Provable Equality.}
The $vtail$ example above involves converting the type of terms using
definitional equality (\ie, $\beta$-conversion). As we reason about more
complicated functions, with more interesting properties, we will need to
represent provable equality. An example of a provable equality is
the associativity of natural numbers over addition. Using the standard
definition of addition (call it $plus$) for unary natural numbers, the two terms
$(plus\;(plus\;n_1\;n_2)\;n_3)$ and $(plus\;n_1\;(plus\;n_2\;n_3))$,
are not convertible. But, we can prove by induction that they are always equal
for any three natural numbers $n_1$, $n_2$, and $n_3$. The addition of
provable equality will extend the programs we can type. Of course,
to construct such proofs, we will need an induction principle. Fortunately,
every Mendler-style operator is an induction principle. This is one reason
why a language built on Mendler-style recursion combinators make a perfect
substrate for a sound logical reasoning system. Which, in case you don't recall,
is the first part of my thesis statement!





\paragraph{Computational index transformers:}
Earlier in this section we saw good reasons to extend
the index transformer syntax to support \emph{equality constraints}.
Another possible extension to index transformers is to allow
\emph{type level computations}, which eventually reduced to a type,
to appear in the body of an index transformer. The current index
transformer syntax only allows the body of an index transformers to be a type.
The only possible variation of the return type is variation over
the term indices used as inputs in the index transformer
when the return type is indexed by some of the inputs. However, we can imagine
(and justify in our formal systems) 
a function that returns totally unrelated types (\eg, \textit{Bool} and Nat)
depending on the input indices. To support functions of such type, we need
to allow case expressions in the body of an index transformer, which examines
a term and returns a type. For example, consider the following definition:
\begin{align*}
\textit{foo}~x =
 \MIt^{\left\{\{n\} . \begin{smallmatrix}\textbf{case}^{\{\}}\;n\;\textbf{of}
                                        &\textit{Zero}~~~  -> \textit{Bool} \\
                                        &\textit{Succ}\;m\;-> \textit{Nat}\;
                                        \end{smallmatrix}
       \right\}}~\;x~\;\textbf{with}
&\quad  f ~\textit{VNil}~~~~~~~~~~~ = \textit{True} \\
&\quad  f \;(\textit{VCons}\;x\;xs) = zero
\end{align*}
%% (\eg, \textbf{case} n \textbf{of} \textit{Zero} )

The function $foo$ takes a length indexed list as an input, and returns
$\textit{True} : \textit{Bool}$ when the list is empty, but returns $zero
: \textit{Nat}$ otherwise.  This ability to write functions whose types
vary computationally (rather than just parametrically) over their inputs
is often called a large elimination. Such an extension is not a problem
to either the type safety or logical consistency of Nax, since the target
calculi \Fi\ and \Fixi\ already have the ability to express type level
computation. However, we are unsure of its ramifications on the type
inference algorithm of Nax. In my thesis, I will elaborate on the
argument that such extensions do not interfere with type safety and
logical consistency of Nax, and also investigate the ramifications 
of such an extension the type inference algorithm of Nax.

\paragraph{Allowing non-termination in the language:}
Recall that the motivation behind my thesis is to contribute to 
building a seamless system where programmers can both write
(functional) programs and formally reason about those programs.
Those programs include non-terminating ones.

I don't propose to come up with a full solution integrating logic and
programming in my thesis work. But I will certainly summarize the recent work
\cite{SjoCasAhnColEadFuKimSheStuWei12} of the Trellys project group, trying to
bridge the two worlds of logic and programming by a modal type system.
This will give a better context for the Nax language and its contributions.

\paragraph{Kind polymorphism:}
The calculi \Fi\ and \Fixi\ are polymorphic at the type level but monomorphic at
the kind level, just like \Fw. Support for polymorphism at the kind level is another
dimension to the extension of Nax. With kind polymorphism, one can
have a single recursive type operator $\mu$, which has a polymorphic kind, instead
of family recursive type operators super scripted by kinds ($\mu^\kappa$).
Recently, \citet{YorWeiCrePeyVytMag12} proposed $F_C^{\uparrow}$, which is
an extension to $F_C$ (the core language of the Glasgow Haskell Compiler)
with kind polymorphism and a limited form of term indexed types.
Although the kind polymorphism extension enables further generalizations and
abstractions over the type structure, we will need extra caution when we consider
logical consistency -- too much (impredicative) polymorphism at the kind level may
lead to inconsistency. Dependently typed proof assistants, such as Coq and Agda,
need not consider the issue of kind polymorphism (or, more generally,
universe polymorphism) affecting consistency, since they simply limit
impredicative polymorphism, because they are based on predicative type systems.
However, I choose to base the theory Nax on an impredicative type system,
extending \Fw, since I believe impredicative polymorphism is
more natural for typing programs, which might not necessarily be proofs.

Although I don't plan to include kind polymorphism in Nax design in
my thesis work, I will certainly summarize the related work, and try to
contemplate on what considerations are needed, if we were to adopt
extensions like kind polymorphism in Nax.

\paragraph{Other possible extensions:}

Some extensions to the Hindley-Milner type system, such as
first class polymorphism, existential types, and qualified types,
would also be possible for Nax. I won't go into much detail on these issues
in the thesis, but I will give pointers to related work on these subjects.

Mendler-style recursion schemes for full dependent types are known to be well
defined for positive datatypes\footnote{Not in published work, but a well-known
folklore among researchers who studied the Mendler-style.}. I will briefly
introduce Mendler-style recursion combinators in a dependently typed setting
in my thesis.

