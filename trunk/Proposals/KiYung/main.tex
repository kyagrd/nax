\documentclass[letterpaper,12pt]{article}
\usepackage{fullpage}
\usepackage{amsmath}
\usepackage{amssymb}
\usepackage{semantic}
\usepackage[backref]{hyperref}
\usepackage[all]{xy}
\usepackage[sort&compress,square,comma,numbers,longnamesfirst]{natbib}
\usepackage{xcolor}
\usepackage{url}

\newcommand{\Fw}[0]{\ensuremath{\mathsf{F}_{\!\omega}}}
% \newcommand{\SystemF}[0]{System \ensuremath{\mathsf{F}}}
\newcommand{\F}[0]{\ensuremath{\mathsf{F}}}
\newcommand{\eg}[0]{e.g., }
\newcommand{\ie}[0]{i.e., }
\newcommand{\aka}[0]{a.k.a. }


\newcommand{\ts}[1]{\textcolor{purple}{\textbf{[#1 ---TS]}}}

\newcommand{\defeq}[0]{\ensuremath{\stackrel{\text{def}}{=}}}

\begin{document}

\title{Inductiveness of types and Normalization of terms}
\author{Ki Yung Ahn}
\maketitle

\chapter{Introduction}\label{ch:intro}

\section{Programming and Formal Reasoning}\label{sec:intro:motiv}
In this dissertation, we contribute to answering the question:
``how does one build a seamless system where programmers can both
write (functional) programs and formally reason about those programs''.
In late 1960s, \citet{Howard69} observed that natural deduction, which is
a proof system of a formal logic, and a typed lambda calculus, which is
a model of computation, are directly related --
a proof of a proposition corresponds to a program and its type.
Since this observation, known as the Curry--Howard correspondence,
logicians and programming language researchers 
have  dreamed of
building a system in which one can both write programs
(\ie, model computation) and formally reason about (\ie, construct proofs of)
the properties (\ie, types) of those programs.

However, building a practical system that unifies programming and
formal reasoning, based on the Curry--Howard correspondence, is still
an open research problem. The gap between the conflicting
design goals of typed functional programming languages, such as ML and Haskell,
and formal reasoning systems, such as Coq and Agda, is still wide.

\begin{itemize}

\item
Programming languages are typically designed to achieve
computational expressiveness. They often sacrifice logical consistency
to achieve this goal. Programmers should be able to
conveniently express all possible computations, regardless of whether those
computations have a logical interpretation or not.

\item
Formal reasoning systems are typically designed to achieve logical consistency.
They often sacrifice computational expressiveness to achieve that goal.
Users expect that it is only possible to prove true propositions,
and it is impossible to prove falsity. They are willing
live with the difficultly (or even impossibility) to
express certain computations within the reasoning system,
to achieve logical consistency.

\end{itemize}

As a result, the recursion schemes of programming languages and
formal reasoning systems differ considerably.
Programming languages provide unrestricted general recursion,
to conveniently express computations
that may or may not terminate.
Formal reasoning systems provide induction principles for sound reasoning,
or, in the computational view, principled recursion schemes
that can only express terminating computation.

The two different design goals also lead to significant differences
in their type system as well.
Programming languages are based on \emph{recursive types},
which which place only syntactic restrictions on the definition of new types.
Programmers can express computations over a wide variety types.
In addition, most (statically typed) functional programing languages have
clear distinction between terms and types (\ie, terms do not appear in types).
Reasoning systems are usually based on \emph{inductive types},
which place semantic restrictions, accepting only type definitions that support
conventional induction principles.
In addition, most reasoning systems, based on the Curry--Howard correspondence,
allow types to depend on terms (\ie, terms can appear in types) to specify
fine grained properties.

This dissertation explores a sweet spot where one can benefit from
the advantages of both programming languages and formal reasoning systems.
That is, we design a unified language system, called Nax, which is
logically consistent while being able to conveniently express
many useful computations. We do this by placing few restriction on type definitions,
as is done in programming languages, but also provide a rich set of
non-conventional recursion schemes (or, induction principles) that
always terminate. These non-conventional recursion schemes are known as
\emph{the Mendler style}. Another major design choice in Nax is
supporting \emph{term indices} in types, a middle ground, which sits between
polymorphic types and dependent types.

In the following section, we explain what we mean by the sweet spot between programming languages
and reasoning systems. Our thesis is that the design choices we explain below
are reasonable for achieving the goal of combining programming and resoning systems.

\section{Thesis}\label{sec:intro:thesis}
Whatever design choices we make, the sweet spot should have the following features.

\begin{enumerate}[(1)]
 \item \textbf{A convenient programming} style
         supported by the major constructs of
         modern functional programming languages: 
         parametric polymorphism, recursive datatypes,
         recursive functions, and type inference,
 \item \textbf{An expressive logic}
         that can specify fine-grained program properties using types, and terms that
         witness proofs of these properties 
         (the Curry--Howard correspondence),
 \item \textbf{A small theory} based upon a minimal foundational calculus that is
         expressive enough to support the programming features, expressive
         enough to embed propostions and proofs about
         programs, and logically consistent
         to avoid paradoxical proofs in the logic, and
 \item \textbf{A simple implementation} that keeps the trusted base small.
\end{enumerate}
We claim that a language design based on \emph{Mendler-style recursion schemes}
and \emph{term-indexed types} can lead to a system that supports these four
features.

\paragraph{}
From a bird's-eye view, the following chapters back up our claim as follows:
Mendler-style recursion schemes support (1) because they are based on
parametric polymorphism and well-defined over a wide range of datatypes.
Term-indexed types support (2), because they can statically track program
properties. For instance the size of data structures can be tracked by using
a natural number term in their types.
To support (3), we design several foundational calculi, each which extends
a well known polymorphic lambda calculus with term-indexed types.
Mendler-style recursion schemes also also support (4) because their
termination is type-based -- no need for an auxiliary termination checker.

In next section, we summarize important ideas mentioned in our thesis above.

\section{Preliminary concepts}\label{sec:intro:concepts}
We give summaries of the following preliminary concepts:
Curry--Howard correspondence (\S\ref{sec:intro:concepts:CH}),
Mendler-style recursion schemes
(\S\ref{sec:intro:concepts:CH}, \S\ref{sec:intro:concepts:mendler}),
and term-indexed types (\S\ref{sec:intro:concepts:indexed}).
Further details and historical backgrounds on each of these concepts
will appear in the following chapters (see \S\ref{sec:intro:overview}
for the overview of chapter organization).

\subsection{The Curry--Howard correspondence and Normalization}
\label{sec:intro:concepts:CH}
One promising approach to designing a system that unifies
logical reasoning and programming is \emph{the Curry--Howard correspondence}.
Howard \cite{Howard69} observed that a typed model of computation
(\ie, a typed lambda calculus) gives an interpretation to a (natural deduction)
proof system (for an intuitionistic logic). More specifically, one can interpret
a type (in the lambda calculus) as a formula (in the logic) and
a term of that type, as a proof for that formula. For instance,
the typing rule for function applications (APP) in a typed lambda calculus
corresponds to Modus Ponens (MP) in a logic:
\[ \inference[(APP)]{\Gamma |- t_1 : A -> B & \Gamma |- t_2 : A}{
        \Gamma |- t_1~t_2 : B}
 ~~~~~~~~
   \inference[(MP)]{A -> B & A}{B}
\]
As you can see above, combining terms ($t_1$ and $t_2$) to build a new term
($t_1~t_2$) can be interpreted as combining proofs for formulae
($A -> B$ and $A$), to construct a proof for a new formula ($B$).
More generally, we may expect that programming (\ie, building larger terms)
corresponds to constructing larger proofs, but only when the typed lambda calculi
meets certain standards -- \emph{type soundness} and \emph{normalization}.

The Curry--Howard correspondence is a promising approach to designing a
unified system for both logical reasoning and programming. Only one language
system is needed for both the logic and the programming language. An
alternate approach is to use an external logical language to talk about
programs as the objects that the logic reasons about. In this approach, one
has the obligation to argue that the soundness of the logic, with respect to
the programming language semantics, holds.

Under the Curry--Howard correspondence, the logic is internally related to the
semantics of program -- there is no need to argue for the soundness of the
logic,  externally outside of the programming language system. The soundness
of the logic follows directly from the type soundness of the language under
the Curry--Howard correspondence.

Let us consider a proposition to be true
(or, valid) when it has a canonical (\ie, cut-free) proof.
That is, there is a program, whose type is the proposition under
consideration, and that program has a normal form. 
By type soundness, any term,
of that type, will preserve its type during the reduction steps. Thus
reduction preserves truthfulness. If we assume
that the language is normalizing (\ie, every well-typed term reduces to
a normal form), then any term of that type which is a non-canonical proof,
implies the existence of a canonical proof, which in turn implies that
the proposition specified by the type is indeed true. That is, all provable
propositions are valid (\ie, the logic is sound) when the language is
\emph{type sound} and \emph{normalizing}.

\emph{Normalization} is also essential for the consistency of the logic.
For the lambda calculus to be interpreted as a \emph{consistent} logic,
there must be no diverging terms. A diverging term (\ie, a term that does
not have a normal form) may inhabit any arbitrary type. Thus, a diverging term
can be a proof for any proposition under the Curry--Howard correspondence.
General purpose functional programming languages (\eg, Haskell and ML), that
support unrestricted general recursion, cannot be interpreted as a consistent
logic, since they allow diverging terms (\ie, non-terminating programs).
For example, a diverging Haskell definition $\textit{loop} = \textit{loop}$
can be given an arbitrary type such as
$\textit{loop}\mathrel{::}\textit{Bool}$,
$\textit{loop}\mathrel{::}\textit{Int} -> \textit{Bool}$,
and even $\textit{loop}\mathrel{::}\forall a. a$, which is a proof of false.


Therefore, useful logical reasoning systems based on the Curry--Howard
correspondence (\eg, Coq, Agda) never support language features that can
lead to diverging terms. For example, in both Coq and Agda,
unrestricted general recursion (at term level) is not supported. 
Instead, these logical reasoning systems
often provide principled recursion schemes over recursive types that are
guaranteed to normalize. 

Recursive types (\ie, recursion at type level)
can also lead to diverging terms when they are not restricted carefully.
Many of the conventional logical reasoning systems, based on
Curry--Howard correspondence, restrict recursive types in a way,
which is not an ideal design choice, if one's goal is a unified system for
logic and programming. Our approach explores another design space not yet
completely explored. We introduce both approaches to restricting recursive
types to ensure normalization in the following two subsections.


\subsection{Restriction on recursive types for normalization}
\label{sec:intro:concpets:recursive}
We have argued that normalization is essential for logical reasoning systems
based on the Curry--Howard correspondence. One challenge to the successful
design of such systems is how to restrict recursion at the type level
so that normalization of terms is preserved. 
There are two different
design choices illustrated in Figure~\ref{fig:approaches}. 
The conventional approach restricts the formation
of recursive types (\ie, the restriction is in datatype definition), and
the Mendler-style approach restricts the elimination
of the values of recursive types (\ie, the restriction is in pattern matching).

\begin{figure}
{\centering
\begin{tabular}{p{3cm}|p{12.5cm}}
\parbox{3cm}{~~Functional\\programming\\$~~~~$language} &
\parbox{12.5cm}{
 kinding:~
  \inference[($\mu$-form)]{\Gamma |- F : * -> *}{\Gamma |- \mu F : *} \\
 \\
 typing:\quad
  \inference[($\mu$-intro)]{\Gamma |- t : F (\mu F)}{\Gamma |- \In~t:\mu F} ~~~~
  \inference[($\mu$-elim)]{\Gamma |- t : \mu F}{\Gamma |- \unIn~t : F (\mu F)}\\
 \\
 reduction:
  \inference[(\unIn-\In)]{}{\unIn~(\In~t) \rightsquigarrow t}
} \\
\\ \hline\hline
\parbox{3cm}{$~$Conventional\\$~~~$approach for\\consistent logic} &
\parbox{12.5cm}{$\phantom{a}$\\
 kinding:~
  \inference[($\mu$-form$^{+}$)]{ \Gamma |- F : * -> * 
                           & \mathop{\mathsf{positive}}(F)}
                           {\Gamma |- \mu F : *} \\
 \\
 typing:~
  \text{{\small($\mu$-intro)} and {\small($\mu$-elim)}
                same as functional language} \\
  \[\inference[(\It)]{\Gamma |- t : \mu F & \Gamma |- \varphi : F A -> A}
                     {\Gamma |- \It~\varphi~t : A}\]
 reduction:~ \text{{\small(\unIn-\In)} same as functional language}
  \[\inference[(\It-\In)]{}{\It~\varphi~(\In~t) \rightsquigarrow
                            \varphi~(\textsf{map}_F~(\It~\varphi)~t)}\]
}
\\ \hline
\parbox{3cm}{Mendler-style\\$~~$approach for\\consistent logic} &
\parbox{12.5cm}{$\phantom{a}$\\
 kinding:~ \text{{\small($\mu$-form)} same as functional language} \\
 \\
 typing:~
  \text{{\small($\mu$-intro)} same as functional language}
  \[\inference[(\MIt)]
     { \Gamma |- t : \mu F &
       \Gamma |- \varphi : \forall X . (X -> A) -> F X -> A}
     {\Gamma |- \MIt~\varphi~t : A} \]
 reduction:~
  \inference[(\MIt-\In)]
     {}
     {\MIt~\varphi~(\In~t) \rightsquigarrow \varphi~(\MIt~\varphi)~t}
}
\end{tabular} }
\caption{Two different approaches to designing a logic
         (in contrast to functional languages)}
\label{fig:approaches}
\end{figure}

\paragraph{Recursive types in functional programming languages.}
Let us start with a review of the theory of recursive types used
in functional programming languages. Here, the term
language is not expected to be normalizing, so restrictions are few.

Just as we can capture the essence of unrestricted general recursion at term
level, by a fix point operator (usually denoted by \textsf{Y} or \textsf{fix}),
we can capture the essence of recursive types by the
use of fix point operator, $\mu$, at type level. 
The rules for the formation {\small($\mu$-form)},
introduction {\small($\mu$-intro)}, and elimination {\small($\mu$-elim)} of
the recursive type operator $\mu$ are described in Figure \ref{fig:approaches}.
We also need a reduction rule {\small(\unIn-\In)}, which relates \In,
the data constructor for recursive types, and \unIn, the destructor for
recursive types, at the term level.

Surprisingly (if you hadn't known), the recursive {\em type} operator, $\mu$,
as described in Figure \ref{fig:approaches}, is already powerful enough to
express non-terminating programs, even without introducing the general recursive
{\em term} operator, \textsf{fix}, to the language. We illustrate this below.
First a short reminder of how a fixpoint at the term level operates.
The typing rule and the reduction rule for \textsf{fix} can be given as follows:
\[ \text{typing:}~ \inference{\Gamma |- f : A -> A}{\textsf{fix}\,f : A}
 \qquad\qquad
   \text{reduction}:~ \textsf{fix}\,f \rightsquigarrow f(\textsf{fix}\,f)
\]
We can actually implement \textsf{fix}, using $\mu$, as follows
(using some Haskell-like syntax):
\begin{align*}
& \textbf{data}~T\;a\;r = C\;(r -> a) \quad
          \texttt{-}\texttt{-}~\text{\small a non-recursive datatype} \\
& w \,:\, \mu(T\;a) -> a ~~ \quad
          \texttt{-}\texttt{-}~\text{\small an encoding of the untyped
                                     $(\lambda x.x\;x)$
                                     in a typed language}~ \\
& w = \lambda x . \,\textbf{case}~\unIn~x~\textbf{of}~C\;f -> f\;x \\
& \textsf{fix} \,:\, (a -> a) -> a \quad
          \texttt{-}\texttt{-}~\text{\small an encoding of 
                                     $(\lambda f.(\lambda x.f(x\;x))\,
                                                 (\lambda x.f(x\;x)))$} \\
& \textsf{fix} = \lambda f. (\lambda x. f (w\;x))\,(\In(C(\lambda x. f (w\;x))))
\end{align*}

Thus, to avoid the loss of termination guarantees, we need to alter the rules
for $\mu$, in someways, to ensure a consistent logic. One way, is to restrict
the rule {\small $\mu$-form}; the other way, is to restrict the rule
{\small $\mu$-elim}. Once we decide which of these two alterations of the
rules we will use, the design of principled recursion combinators (\eg, \It\
for the former and \MIt\ for the latter) follows from that choice.

\paragraph{Recursive types in the conventional approach to consistent logic.}
In the conventional approach, the formation (\ie, datatype definition) of
recursive types is restricted, but arbitrary elimination (\ie, pattern matching)
over the values of recursive types is allowed. In particular, the formation of
negative recursive types is restricted. Only positive recursive types are
supported. Thus, in Figure \ref{fig:approaches}, we have a restricted version of
the formation rule {\small($\mu$-form$^{+}$)} has an additional condition that
require $F$ to be positive. The other rules {\small($\mu$-intro)},
{\small($\mu$-elim)}, and {\small(\unIn-\In)} remain the same as for
functional languages. Since we have restricted the recursive types
at the type level and we do not have general recursion at the term level,
the language is indeed normalizing. However, we can neither write
interesting (\ie, recursive) programs that involves recursive types nor
inductively reason about those programs, unless we have principled recursion
schemes that are guaranteed to normalize. One such recursion scheme is called
iteration (\aka\ catamorphism). The typing rules for the conventional iteration
\It\ are illustrated in Figure \ref{fig:approaches}. Note, we have the typing
rule {\small(\It)} and the reduction rule {\small(\It-\In)} for \It\,
in addition to the rules for the recursive type operator $\mu$.

\paragraph{Recursive types in the Mendler-style approach to consistent logic.}
In the Mendler-style approach, we allow arbitrary formation
(\ie, datatype definition) of recursive types, but we restrict
the elimination (\ie, pattern matching) over the values of recursive types. 
The formation rule {\small($\mu$-form)} remains the same as
for functional languages. That is, we can define arbitrary recursive types,
both positive and negative. However, we no longer have the elimination
rule {\small($\mu$-elim)}. That is, we are not allowed to pattern match over
the values of recursive types in the normal fashion. We can only pattern match
over the values of recursive types through the Mendler-style recursion
combinators. The rules for the Mendler-style iteration combinator \MIt\
are illustrated in Figure \ref{fig:approaches}.
Note, there are no rules for \unIn\ in the Mendler-style approach.
The typing rule {\small($\mu$-elim)} is replaced by {\small(\MIt)} and
the reduction rule {\small(\unIn-\In)} is replaced by {\small(\MIt-\In)}.
More precisely, the typing rule {\small \MIt} is both an elimination rule
for recursive types and a typing rule for the Mendler-style iterator.
You can think of the rule {\small(\MIt)} as replacing both the elimination rule
{\small($\mu$-elim)} and the typing rule for conventional iteration
{\small(\It)}, but in a safe way that guarantees normalization.

\subsection{Justification of the Mendler-style as a design choice.}
\label{sec:intro:concepts:mendler}
We choose to base our approach to the design of a seamless synthesis of both
logic and programming on the Mendler-style. It restricts the elimination (\ie,
pattern matching) over values of recursive types, rather restricting the
formation (\ie, datatype definition) of recursive types (a more conventional
approach). The impact of this design choice is that it enables the logic to
include all datatype definitions that are used in functional programming
languages.

Functional programming promotes ``functions as first class values''.
It is natural to pass functions as arguments and embed functions into
(recursive) datatypes. If embedding functions in datatypes is allowed,
we can embed a function whose domain is the very type we are defining.
For example, the recursive datatype definition
$\mathbf{data}~T = C\;(T -> \textit{A})$ in Haskell is such a recursive
datatype definition. Such datatypes are called negative recursive datatypes
since the recursive occurrence $T$ appears in a negative position.
We say that $T$ is in a negative position, since $(T -> A)$ is analogous to
$(\neg T \land A)$ when we think of $->$ as a logical implication. There exist
both interesting and useful examples in functional programming involving
negative datatypes. In \S\ref{sec:msf}, we illustrate that
the Mendler-style recursion scheme we discovered can express
interesting examples involving negative datatypes.

Recall that the motivation of this dissertation research
(quoting again from \S\ref{sec:intro:motiv})
is to contribute to answering the question of {\em how does one build a
seamless system where programmers can both write (functional) programs and
formally reason about those programs}. Under the Curry--Howard correspondence,
to formally reason about a program, the logic needs to refer to the type of
the program, since the type, interpreted as a proposition, describes its
properties. Since the Mendler-style approach does not restrict recursive
datatype definitions, we can directly refer to the types of programs that use
negative recursive types.

The Mendler style is a promising approach to building a unified system because
all the recursive types (both positive and negative) are definable and
the recursion schemes over those types are normalizing.
%% As we mentioned previously, the Mendler-style iteration
%% (\MIt) always terminate for both positive and negative recursive types.
%% There exist other families of Mendler-style recursion combinators,
%% which also guarantee for negative recursive types, and more useful
%% than \MIt\ over negative datatypes.
Although the conventional approach is widely followed
in the design of formal reasoning systems (\eg, Coq, Agda), it cannot directly
refer to programs that use non-positive recursive types.One may object that
it is possible to indirectly model negative recursive types
in the conventional style, via alternative equivalent encodings
which map negative recursive types into positive ones. But, such
encodings do not align with our motivation towards a seamless unified
system for both programming and reasoning. It is undesirable to require
programmers to significantly change their programs just to reason about them.
If the change is unavoidable, it should be kept small. That is,
the changed program should syntactically resemble the original program,
which the programmer would usually write in a functional programming language.
In Chapter 3, we show a number of examples of programs written in
the Mendler style that look more close to the programs written using
general recursion than the programs written in the conventional style.

%% Throughout this dissertation,
%% we show that the Mendler-style recursion schemes are
%% indeed useful and well-behaved induction principles.

\subsection{Term-indexed types, type inference, and datatypes}
\label{sec:intro:concepts:indexed}
One of the most frequently asked questions about our design choices for Nax,
regarding term-indexed types, is ``why not dependent types?". Our answer
is that a moderate extension to the polymorphic calculus is a better candidate
than a dependently typed calculus as the basis for a practical programming
system. Recall, that we hope to design a unified system for programming
as well as reasoning. Language designs based on indexed types can
benefit from existing compiler technology and type inference algorithms
for functional programming languages. In addition, theories for
term-indexd datatypes are simpler than theories for full-fledged
dependent datatypes, because term-indexd datatypes can be encoded as
functions (using Church-like encodings).

The implementation technology for functional programming languages based on
polymorphic calculi is quite mature. There exist industrial
strength implementations, such as the Glasgow Haskell Compiler (GHC),
whose intermediate core language is an extension of \Fw.
Our term-indexed calculi described in Part \ref{part:Calculi} are closely
related to \Fw\ by an index-erasure property. The hope is that
our implementation can benefit from these technologies.

Type inference algorithms for functional programming languages are often
based on certain restrictions of the Curry-style polymorphic lambda calculi.
These restrictions are designed to avoid higher-order unification during
type inference.
We develop a conservative extension of Hindley--Milner type inference for
Nax (Chapter \ref{ch:naxTyInfer}). This is possuble because Nax is based on our
term-indexed calculi (Part \ref{part:Calculi}). Dependently typed languages,
on the other hand, are often based on bidirectional type checking, which
requires annotations on top level definitions, rather than
Hindley--Milner-style type inference.

In dependent type theories, datatypes are usually supported as primitive
constructs with axioms, rather than as functional encodings
(\eg, Church encodings). One can give functional encodings for datatypes
in a dependent type theory, but one soon realizes that the induction principles
(or, dependent eliminators) for those datatypes cannot be derived within
the pure dependent calculi \cite{Geuvers01}.
So, dependently typed reasoning systems support datatypes as primitives.
For instance, Coq is based on Calculus of Inductive Constructions, which
extends Calculus of Constructions \cite{CoqHue86} with dependent datatypes
and their induction principles.

In contrast, in polymorphic type theories, all imaginable datatypes
within the calculi have functional encodings (\eg, Church encodings).
For instance, \Fw\ need not introduce datatypes as primitive constructs,
since \Fw\ can embed all imaginable datatypes, including non-regular
recursive datatypes with type indices. 

Another reason to use term-indexed calculi, rather than dependent type theories,
is to extend the application of Mendler-style recursion schemes,
which are well-understood in the context of \Fw.
Researchers have thought about (though not published)\footnote{
     Tarmo Uustalu described this on a whiteboard
     when we met with him at the University of Cambridge in 2011.
     We discuss this in Chapter \ref{ch:relwork}.}
Mendler-style primitive recursion over dependently-typed functions
over positive datatypes (\ie, datatypes that have a map), but not for
negative (or, mixed-variant) datatypes. In our term-indexed calculi,
we can embed Mendler-style recursion schemes (just as we embedded them in \Fw)
that are also well-defined for negative datatypes.

\section{Contributions}\label{sec:intro:contrib}
This dissertation makes contributions in several areas.
\begin{itemize}
\item[1.]
It organizes and expands the realm of \emph{Mendler-style recursion schemes}
(Part~\ref{part:Mendler}, \ie, Chapter \ref{ch:mendler})

\item[2.] It establishes a meta-theories for \emph{term-indexed types}
        (Part~\ref{part:Calculi}),

\item[3.] It designs a practical language (with an implementation)
        \emph{in the sweet spot} between programming and logical reasoning
        (Part~\ref{part:Nax}), and

\item[4.] It identifies several interesting open problems related to above.
\end{itemize}

\subsection{Contributions related to the Mendler style}
We organize a hierarchy of Mendler-style recursion schemes in two dimensions.
The first dimension is the abstract operations they support. For instance,
the Mendler-style iteration (\MIt) supports a single abstract operation
the recursive call. All the other Mendler-style recursion schemes
support the recursive call and an additional set of abstract operations. 
The second dimension is over the kind of the datatypes they operate over.
For example, \texttt{Nat} has kind $*$, while \texttt{Vec}
has kind $* -> \mathtt{Nat} -> *$. Each recursion scheme is actually a
family of recursion combinators sharing the same term definition
(\ie, uniformly defined) but with different type signatures at each kind.

We expand the realm of Mendler-style recursion schemes in several ways.
First, we report on a new recursion scheme $\MsfIt$, which is useful
for negative datatypes.  Second, we study the termination behaviors
of Mendler-style recursion schemes. Some recursion schemes (\eg, \MIt, \MsfIt)
always terminate for any recursive type, while others (\eg, \McvPr) only
terminate for certain classes of recursive types. Third, we extend
all Mendler-style recursion schemes to be expressive over term-indexed types.
The Mendler style has been studied in the context of \Fw\ (and several
extensions) which can express {\bf type}-indexed types. To extend Mendler-style
recursion schemes to be expressive over {\bf term}-indexed types, we report on
several theories for calculi (\Fi\ and \Fixi) that support term indices.
This is another important area of our contribution.

We provide examples that illustrate when each recursion scheme is useful
in Chapter \ref{ch:mendler}. The most interesting example among them is
the type-preserving evaluator for a simply-typed HOAS (\S\ref{sec:evalHOAS}),
which involves negative datatypes with indices.
This example is our novel discovery, which implies that
a type-preserving evaluator for a simply-typed HOAS
can be expressed within \Fw.

In addition, we develop a better understanding of some existing
Mendler-style recursion schemes. For instance, the existence of
Mendler-style course-of-values recursion (\McvPr) is reported
in the literature, but the calculus that can embed \McvPr\ was unknown.
We embed Mendler-style course-of-values recursion into \Fixi\ 
(or into \Fixw\ \cite{AbeMat04}, when we do not consider term-indices).

\subsection{Contributions to the theory of Term-Indexed Types}
Mendler-style recursion schemes have been studies in the context of
polymorphic lambda calculi. For instance, \citet{AbeMatUus03} embedded 
Mendler-style iteration (\MIt) into \Fw\ and \citet{AbeMat04} embedded
Mendler-style primitive recursion (\MPr) into \Fixw. These calculi
support type-indexed types.

To extend the realm of Mendler-style recursion schemes to include
term-indexed types, we extended \Fw\ and \Fixw\ to support term indices.
In Part \ref{part:Calculi}, we present our new calculi
\Fi\ (Chapter \ref{ch:fi}), which extends \Fw\ with term indices, and
\Fixi\ (Chapter \ref{ch:fixi}), which extends \Fixw\ with term indices.
These calculi have an erasure property that states that well-typed terms
in each calculus are also well typed terms (when erased) in the 
underlying calculus. For instance, any well typed term in \Fi\ is also
a well-typed term in \Fw, and there are no additional well-typed terms
in \Fi\ that are not well-typed in \Fw.

Our new calculi, \Fi\ and \Fixi, are strongly normalizing and
logically consistent. We show strong normalization and logical consistency
using the erasure properties. That is, strong normalization and
logical consistency of \Fi\ and \Fixi\ are inherited from \Fw\ and \Fixw.
Since \Fi\ and \Fixi\ are strong normalizing and logically  consistent,
the Mendler-style recursion schemes that can be embedded into these calculi
are adequate for logical reasoning as well as programming.

\subsection{Contributions in the design of the Nax language}
We design and implement a prototypical language Nax that explores
the sweet spot between programming oriented systems and logic oriented systems.
The language features supported by Nax provide the advantages
of both programming oriented systems and logic oriented systems.
Nax supports both term- and type-indexed datatypes,
rich families of Mendler-style recursion combinators,
and a conservative extension of Hindley--Milner type inference.
We designed Nax so that its foundational theory and
implementation framework could be kept simple.

Term- and type-indexed datatypes can express fine grained program properties
via the Curry--Howard correspondence, as in logic oriented systems. Although
not as flexible as full-fledged dependent types, indexed datatypes can
still express program invariants, such as type preserving compilation
(\S\ref{sec:example}), and size invariants on data structures.
Index types can simulate much of what
dependent types can do using singleton types. Since Nax has only erasable
indices, the foundational theory can be kept simple, and it supports
features that have the advantages of programming oriented systems 
(\eg, type inference, arbitrary recursive datatypes).

Adopting Mendler style provides merits of both programming oriented systems
and logic oriented systems. Since Mendler style is elimination based, one can
define all recursive datatypes usually supported in functional programming
languages. In addition, the programs written using Mendler-style recursion
combinators look more similar to the programs written using general recursion
than programs written in Squiggol style.
Since Nax supports only the well-behaved (\ie, strongly normalizing)
Mendler-style recursion combinators, it is safe to construct proofs using them.
In addition, Mendler-style recursion combinators are naturally well-defined
over indexed datatypes, which are essential to express fine-grained program
properties. Mendler style provides type based termination, that is, termination
is a by-product of type checking. Thus, it makes the implementation framework
simple since we do not need extra termination checking theories or algorithm.

Hindley--Milner-style type inference is familiar 
to functional programmers.
Nax can infer types for all programs that involve only regular datatypes,
which are already inferable in Hindley--Milner, without any type annotation.
Nax requires programs involving indexed datatypes to annotate their eliminators
by index transformers, which specify the relation between the input type index
and the result type. Eliminators of non-recursive datatypes are case expressions
and eliminators of recursive datatypes are Mendler-style recursion combinators.

\subsection{Contributions identifying open problems}
We identify several open problems alongside the contributions mentioned
in previews subsections. We will discuss the details of these open problems
in the future work chapter (Chapter \ref{ch:futwork}).
Here, we briefly introduce two of them.

\paragraph{Handling different interpretations of $\mu$ in one language system:}
Nax provides multiple recursion schemes (or, induction principles) used
to describe different kinds of recursive computations over recursive datatypes.
These recursion schemes are all motivated by concrete examples, which explains
the need for multiple schemes. It is more convenient to express various kinds of
recursive computations in Nax, by choosing a recursion scheme that fits
the structure of the computation, than in those systems that provide
only one induction scheme. However, there is theoretical difficulty
handling multiple interpretations of the recursive type operator $\mu$
in one language system.

Recall that we can embed datatypes as functional encodings in
our indexed type theory. Recursive datatypes and their recursion schemes in Nax
are embedded using Mendler-style encodings.
In Mendler style, one encodes the recursive type operator $\mu$
and its eliminator (the recursion scheme) as a pair.
So, there are several different encodings of $\mu$,
one for each recursion scheme. Some recursion schemes subsume others
(\ie, the more expressive one can simulate the other).

It would have been easy to describe the theory for Nax if we had
one most powerful recursion scheme that subsumes all the others,
which leads to a single interpretation of $\mu$. Unfortunately, we know of
no Mendler-style recursion scheme that subsumes all the other recursion schemes
in Nax. For instance, iteration (\MIt) can be subsumed by either 
iteration with a syntactic inverse (\MsfIt) or primitive recursion (\MPr).
But, there is no known recursion scheme that can subsume both \MsfIt\ and \MPr.

However, we strongly believe that it is okay to apply \MsfIt\ to
the result of \MPr\ (when \MPr\ produces a recursive value) and vice versa.
Intuitively, the different interpretations of $\mu$ only matter during
the internal computation of the recursion scheme. That is, one may consider
that (recursive) values resulting from different recursion schemes
share a common abstract representation of $\mu$.
The theoretical justification for this is still ongoing work.

\paragraph{Deriving positivity (or monotonicity) from polarized kinds:}
One can extend the kind syntax of arrow kinds in \Fw\ with polarities
($p\kappa_1 -> \kappa_2$ where the polarity $p$ is either $+$, $-$, or $0$)
to track whether a type constructor argument is used in
covariant (positive), contra-variant (negative), or
mixed-variant (both positive and negative) positions.
It is still an open problem whether it is possible to derive monotonicity
(\ie, the  existence of a map) for a type constructor from its polarized kind,
without examining the type constructor definition.

We identified a useful application for a solution to this open problem.
We discovered an embedding of Mendler-style course-of-values recursion in
a polarized system for positive (or monotone) type constructors.
That is, once you can show the existence of a map for a datatype,
course-of-values recursion always terminates.
However, in a practical language system, it is not desirable to burden users
with the manual derivation for every datatype on which they might want to
perform course-of-values recursion. If the type system can automatically
categorize datatypes that have maps from their polarized kinds,
this burden can be alleviated.


\section{Methodology and Overview}\label{sec:intro:overview}
This dissertation consist of five parts:
Part \ref{part:Prelude} (Prelude),
Part \ref{part:Mendler} (The Mendler style),
Part \ref{part:Calculi} (Term-indexed lambda calculi),
Part \ref{part:Nax} (The Nax language), and
Part \ref{part:Postlude} (Postlude).
The three parts in the middle, excluding the prelude and postlude parts,
describes the three steps of our approach. First, we experiment new ideas on
Mendler-style recursion schemes driven from concrete examples
using Haskell (with some GHC extensions), which is a functional language
based on an extension of \Fw\ (Part \ref{part:Mendler}). Second, we develop
theories (\ie, lambda calculi) for term-indexed datatypes to prove that
the Mendler-style recursion schemes are well-defined over indexed datatypes
and have the expected termination behavior. Lastly, we design a language system
with practical features that embodies our new ideas in and is based on the
theory we developed. Figure~\ref{fig:overview} summarizes the organization of
key concepts throughout the dissertation.

\begin{figure}
TODO \\

STLC\\
\F\
\Fw\    \Fi\ \\
\Fixw\  \Fixi\ \\
Hindley--Milner Nax type inference \\

TODO \\
make arrow diagrams
TODO \\
TODO \\
TODO \\
TODO \\
TODO \\
\caption{Summary of key concepts}
\label{fig:overview}
\end{figure}

\paragraph{Part \ref{part:Prelude} (Prelude)}\hspace{-1em} opens
the dissertation by giving an introduction (Chapter \ref{ch:intro}),
which you are currently reading, followed by
reviews on several well-known typed lambda calculi (Chapter \ref{ch:poly}).
In Chapter \ref{ch:poly}, we review
the simply-typed lambda calculus (STLC) (\S\ref{sec:stlc}),
System \F\ (\S\ref{sec:f}),
System \Fw\ (\S\ref{sec:fw}), and
the Hindley--Milner type system (\S\ref{sec:hm}).

From \S\ref{sec:stlc} to \ref{sec:fw}, we prove strong normalization using
saturated sets for each of the three calculi:
STLC (no polymorphism), System \F\ (polymorphism over types), and
System \Fw\ (polymorphism over type constructors).
The normalization proof on later sections extends upon
the normalization proof of the previous section,
as the calculus extends its feature of polymorphism.
We use the strong normalization of System \Fw\ to show that
our term-indexed lambda calculi in Part \ref{part:Calculi} are
strongly normalizing. Readers familiar with strong normalization proofs
on these calculi may skip or quickly skim over these sections.
It is worth noticing two stylistic choices in our formalization of
System \F\ and \Fw: (1) terms are in Curry style and
(2) typing contexts in System \F\ and \Fw\ are divided into two parts
    (one for type variables and the other for term variables).
This prepares readers for our formalization of the term-indexed calculi
in Part \ref{part:Calculi}, which have Curry-style terms and
typing contexts divided into two parts.

In \S\ref{sec:hm}, we review the type inference algorithm for
the Hindley--Milner type system (\S\ref{sec:hm}).
The Hindley--Milner type system (HM) is a restriction of System~\F,
which makes it possible to infer types without any type annotation on terms.
Later in Part~\ref{part:Nax} Chapter \ref{ch:naxTyInfer},
we formulate a conservative extension of HM, which restricts
the term-indexed calculus System \Fi\ (Chapter \ref{ch:fi}) in a similar manner.

\paragraph{Part \ref{part:Mendler} (the Mendler style)}\hspace{-1em} introduces
the concept of Mendler-style recursion schemes (Chapter \ref{ch:mendler})
using examples written in Haskell (with some GHC extensions). So, the readers
of Chapter \ref{ch:mendler} need no background knowledge on typed lambda calculi
but only some familiarity to functional programming. We explain the concepts of
a number of Mendler-style recursion schemes, their termination properties, and
how one recursion scheme is related to another, in an intuitive manner by using
examples written in Haskell. We also provide semi-formal proofs of termination
for some of the recursion schemes (\MIt\ and \MsfIt) by embedding the those
recursion schemes into \Fw\ fragment of Haskell. More formal and general
proof by embedding into our term-indexed lambda calculi comes later in
Part \ref{part:Calculi}.

The Mendler-style recursion schemes discussed in Chapter \ref{ch:mendler}
include iteration (\MIt), iteration with syntactic inverse (\MsfIt),
primitive recursion (\MPr), course-of-values iteration (\McvIt),
and course-of-values recursion (\McvPr). Among them, \MsfIt\ is a
Mendler-style recursion scheme we discovered.
There are even more Mendler-style recursion schemes, which are not
discussed in Chapter \ref{ch:mendler} -- we give pointers to them later in
the related work chapter (Chapter \ref{ch:relwork} in Part \ref{part:Postlude}).

\paragraph{Part \ref{part:Calculi} (term-indexed lambda calculi)}\hspace{-1em}
establishes theories for term-indexed types.
We formalized two term-indexed lambda calculi,
System \Fi\ (Chapter \ref{ch:fi}) and System \Fixi\ (Chapter \ref{ch:fixi}),
which are extensions of polymorphic calculi with term indices.
System \Fi\ extends System \Fw\ with term indices and
System \Fixi\ extends System \Fixw\ \cite{AbeMat04} with term indices.

We prove both strong normalization and logical consistency of
these term-indexed calculi using their index erasure property.
The index erasure property of a term-indexed calculus
projects a typing in the term-index calculi into
the polymorphic calculus it was extended from.
That is, all well-typed terms in \Fi\ and \Fixi\ are
also well-typed typed terms in \Fw\ and \Fixw.
That is, our term-indexed calculi, \Fi\ and \Fixi,
inherits strong normalization and logical consistency
from the polymorphic calculi, \Fw\ and \Fixw.

By embedding those recursion schemes into our term-indexed lambda calculi,
we prove that Mendler-style recursion schemes are well-defined and
terminates over term-indexed datatypes  For instance,
\MIt\ and \MsfIt\ can be embedded into System \Fi,
and, \MPr\ and \McvPr\ can be embedded into System \Fixi.

\paragraph{Part \ref{part:Nax} (the Nax language)}\hspace{-1em} consist of
three chapters.
First, we introduce the features of Nax (Chapter \ref{ch:naxFeatures})
in a tutorial format using small Nax code snippet examples.
Next, we discuss the design principles of the type system (Chapter \ref{ch:nax})
in comparison to two other systems: Haskell's datatype promotion and Agda.
We develop the discussion In Chapter \ref{ch:nax} along
a larger and more practical example Nax programs:
a type preserving interpreter and a stack safe compiler.
Lastly, we discuss type inference in Nax (Chapter \ref{ch:naxTyInfer}),
which is a conservative extension of Hindley--Milner type system (HM).
That is, any program, whose type can be inferred in HM, can also be
inferred its type in Nax without any annotation. Programs involving
term- or type-indexed datatypes, which are not supported in HM, needs
some annotation to infer their types in Nax. Annotations are only
required on three syntactic entities (datatype declarations, case expressions,
and Mendler-style recursion combinators) and nowhere else.

\paragraph{Part \ref{part:Postlude} (Postlude)}\hspace{-1em} closes
the dissertation by summarizing
  related work (Chapter~\ref{ch:relwork}),
  future work (Chapter~\ref{ch:futwork}), and
  conclusions (Chapter~\ref{ch:concl}).



\section{Background: Recursive types, Inductive types, and Normalization}
\label{sec:bg}
The literature which studies recursive types and inductive types can be
categorized into two different paradigms: the recursive type paradigm
(types in programs), and the inductive type paradigm (types in logic).

The recursive type paradigm is a syntactic approach. It considers any type
definition, constructed using a well formed syntax, as defining a valid type.
The recursive type paradigm originated in the programming language community,
where the primary interest was type safety. From this perspective, types are
viewed as safety properties to be preserved throughout the execution of
programs, which are possibly non-terminating. We discuss this perspective
on recursive types in \S\ref{ssec:rectype}.

The inductive type paradigm is a semantic approach. It admits only the types
that have simple and well-behaved interpretations (\eg sets). The types are
built up from basic types, whose interpretations are trivial (\eg finite sets),
using well-understood connectives and induction principles, so that
all definable types have well-behaved interpretations by construction.
The inductive type paradigm originated in the constructive mathematics community, where the primary interest was logical consistency.
In this perspective, types are viewed as propositions inhabited by proofs.
A proof is a normal form which has the proposition as its type.
Thus, only the normalizing programs, which produce normal forms interpreted
as sound proofs, can be well typed. We discuss this perspective of
inductive types in \S\ref{ssec:indtype}.

There exist certain classes of recursive types, which cannot be admitted as
valid types in the inductive type paradigm. One would naturally question
``when do recursive types coincide with inductive types?'' And, ``how should
recursive types be treated when they do not correspond to inductive types?''

The literature attempting to answer these questions is rich, and with many more
questions than answers, remains an area of active research. This document
discusses the relationship between recursive types and inductive types
in \S\ref{ssec:recVSind}, and exhibits some examples of recursive types
that are not inductive types in \S\ref{ssec:nonindrec}.
The general scope of the proposed thesis is to answer questions about how
and when can programs involving recursive types in a programming language,
can also be interpreted as proofs in a logic.

\subsection{Recursive types}\label{ssec:rectype}

\newcommand{\List}{ \mu\,\alpha . \mathsf{Unit} + (A \times \alpha) }
\newcommand{\ListHole}[1]{ \mathsf{Unit} + (A \times (#1)) }

In the recursive type paradigm, a recursive type operator $\mu$ is provided
for defining recursive types. The typing rules for $\mu$ allow a recursive type
to be unrolled (\ie the full type can be substituted for recursive occurrences
in its body). This substitution can be used as many times as necessary.
For example, a recursive type definition for lists containing elements of
type $A$ is $\List$, which represents the solution of $X$ for
the recursive type equation $X = \ListHole{X}$. That is, a list is either
an empty list, represented by the value of type $\mathsf{Unit}$,
or a non empty list, represented by a pair value of type $A$ (head) and
type $\alpha$ (tail). Note, the type variable $\alpha$, bound by $\mu$, occurs
where the recursive list type is expected (\ie tail of the list). We are
allowed to unroll the list type as many times as we need (going downwards):
\begin{quote}
$\List$\\
$\ListHole{\List}$\\
$\ListHole{\ListHole{\List}}$\\
$\ListHole{\ListHole{\ListHole{\List}}}$\\
$\vdots$
\end{quote}
Conversely, we can roll the list type the other way (going upwards).
The typing rules for unrolling and rolling recursive types are based purely on
syntactic substitution (see Figure\;\ref{fig:isoVSequi}), regardless of whether
a recursive type definition actually has a well behaved interpretation.
In particular, it is well known that we can construct non-terminating terms
with unrestricted use of recursive types, even without introducing unrestricted
general recursion into the programming language. When our interest is
not strong normalization but only type safety, having non-termination
in the programming language is not a problem, as long as we can show
type safety of the language. General purpose programming languages,
which are Turing complete, should be able to express non-terminating
computations anyway. Therefore, typed lambda calculi extended with
recursive types have been studied as theoretic models for understanding
general purpose programming languages. These studies include
functional languages (e.g. ML, Haskell), object-oriented languages (e.g. Java),
and imperative languages (e.g. Algol 68 \cite{ALGOL68}).

There are two styles of typing rules for recursive types
(Figure\;\ref{fig:isoVSequi}):
\emph{equi-recursive types} and \emph{iso-recursive types}.
The typing rules for equi-recursive types allow implicit rolling unrolling
of recursive types. That is, equi-recursive types are not syntax directed,
and the implementation of the type system should infer where to apply the
rolling and unrolling of recursive types.
The typing rules for iso-recursive types are syntax directed. The explicit
term syntax, \textsf{roll} and \textsf{unroll}, guides exactly where to apply
the rolling and unrolling rules. Thus, type checking does not become any more
complicated by adding iso-recursive types into the language. However, we need
an additional reduction rule, such as $\mathsf{unroll}\,(\mathsf{roll}~e)\to e$,
for the extra term syntax introduced by adding iso-recursive types.

\begin{figure}\centering
\begin{minipage}{.45\textwidth}\centering
equi-recursive types
\[
\inference[equi-roll]{\Gamma |- e:T[\mu\alpha.T/\alpha]}
                     {\Gamma |- e:\mu\alpha.T} \]
\[
\inference[equi-unroll]{\Gamma |- e:\mu\alpha.T}
                       {\Gamma |- e:T[\mu\alpha.T/\alpha]} \]
\end{minipage}
\begin{minipage}{.45\textwidth}\centering
iso-recursive types
\[
\inference[iso-roll]{\Gamma |- e:T[\mu\alpha.T/\alpha]}
                    {\Gamma |- \mathsf{roll}~e:\mu\alpha.T} \]
\[
\inference[iso-unroll]{\Gamma |- e:\mu\alpha.T}
                      {\Gamma |- \mathsf{unroll}~e:T[\mu\alpha.T/\alpha]} \]
\end{minipage}

\caption{Typing rules for equi-recursive types and iso-recursive types}
\label{fig:isoVSequi}
\end{figure}

System \F\ \cite{Gir71,Rey74} and its related extensions, such as \Fw, %% cite??
are particularly interesting when studying the recursive type paradigm, since
it is possible to embed all recursive type definitions inside these systems.
Such embeddings are, of course, less expressive than having the recursive
type operator $\mu$ as a primitive language construct, since languages like
\F\ and \Fw\ are strongly normalizing. Embeddings of recursive types in
such languages restrict certain \emph{uses} of these recursive types,
while being able to \emph{define} all of them. More specifically,
such embeddings amount to supporting arbitrary use of rolling,
but restricting the use of unrolling. For functional programmers,
this would amount to the free use of data constructors to construct values,
but restricted use of pattern matching to destruct existing values.
Such a pattern of terminating computation over recursive types is called
\emph{iteration} in contrast to primitive recursion. We will discuss further
on interation and recursion over non-inductive types in \S\ref{sec:prelim}.

Another way of retaining strong normalization in typed lambda calculi,
in the presence of recursive types, is to limit the recursive type definitions
to only the well-behaved ones \cite{Mat98,Mat99,Mat01}.
This alternative approach is closely related to the paradigm of inductive types.
We discuss further the relationship between the recursive types and
inductive types in \S\ref{ssec:recVSind}.

\subsection{Inductive types}\label{ssec:indtype}

In the inductive type paradigm, types must have well-behaved interpretations
(\ie sets). Thus, types should be built from well-understood types using
well-understood connectives, so that all types have well-behaved
interpretations by construction. Martin-L\"of's Intuitionistic Type Theory
\cite{Mar84itt} is the representative system using this paradigm. So, we
often refer to the inductive type paradigm as the Martin-L\"of paradigm.
In Martin-L\"of's type theory, finite sets (or, finite types) are given,
and we can build other types by a dependent function connective ($\Pi$-type),
a dependent pair connective ($\Sigma$-type), and a well-founded induction
principle ($W$-type). This paradigm is suitable for designing type systems
of formal proof assistants, which support logical reasoning by interpreting
types as propositions and programs of those types as proofs (\aka Curry-Howard
correspondence).

Note, Martin-L\"of's type theory \cite{Mar84itt} is a very powerful system
supporting transfinite induction ($W$-type) and dependent types ($\Pi$-type,
$\Sigma$-type). The inductive type paradigm naturally incorporates
dependent types by interpreting them as indexed families of sets.

\subsection{Approaches to relating recursive types and inductive types}
\label{ssec:recVSind}

Not all recursive types are inductive types. That is, there are recursive types,
which cannot be interpreted as types in the inductive type paradigm.
For example, The recursive type $\mu \alpha . \alpha \to \alpha$ is
a classic example of a recursive type without a well defined meaning as a set.
Types whose definitions involve the function space over the type being defined
often have this problem. Such types are often called \emph{reflexive types}.
More specifically, recursive type definitions involving function spaces, which
mention the recursive type being defined in the domain of a function space
(\ie left-hand side of $\to$), do not correspond to inductive types. We will see
more examples of such non-inductive recursive types in \S\ref{ssec:nonindrec}.

Knowing that non-inductive recursive types exist, one would naturally question
\begin{itemize}
\item[]Question (1): when do recursive types coincide with inductive types, and
\item[]Question (2): how should recursive types be treated when they don't.
\end{itemize}
There have been many studies around these questions, and the proposed
thesis will hopefully answer some aspects of these questions.

\subsubsection*{Question (1): When do recursive types coincide with inductive types?}

A widely accepted answer to this question is that when the recursive type
is \emph{strictly positive}. A recursive type is strictly positive when
the recursive type variable does not appear free on the left-hand side
of the $\to$ (\ie in the domain of the function space), but only on
the right-hand side of the $\to$ (\ie in the range of the function space).
For sum types and product types, both of their components should be
strictly positive. Non-recursive types are strictly positive by default.
\citet{Dyb97} showed that any strictly positive type definitions using
single recursive variable (\ie only one $\mu$ appears in a type definition)
can be represented using $W$-types.
\citet{AbbAltGha05,AbbAltGha04} generalized Dyber's work \cite{Dyb97} to
strictly positive type definitions using arbitrary number of recursive type
variables (\ie many $\mu$ can appear in a type definition).
\citet{GamHyl03} have generalized these results
\cite{AbbAltGha05,AbbAltGha04,Dyb97} to dependently typed setting.
\citet{CoqPau90} developed a similar construction based on
Calculus of Constructions (CC, or CoC) \cite{CoqHue86}, but without relating
the inductive definitions with $W$-types.

However, the notion of strict positivity does not generalize well to richer
families of datatypes in languages more expressive than System \F.
Strict positivity is a syntactic condition, which makes good sense for
recursive types in the context of System \F\ (and, of course, in more simple
languages like the simply typed lambda calculus) where recursive type variables
bound by $\mu$ have kind $\star$ (\ie range over types rather than
type constructors). In more expressive languages like \Fw, it becomes unclear
how we should generalize the syntactic condition of strict positivity, since
we can also have recursive type operators for type constructors (of arbitrary
kinds) as well as for types (of kind $\star$). That is, we have a family of
recursive type operators $\mu_\kappa$ indexed by kind $\kappa$, which can
express richer families of recursive types. For example, the $\mu$ operator for
types corresponds to $\mu_{\star}$. In order to have $\mu_{\star}\alpha_0.T$
be well kinded, the bound variable should be kinded: $\alpha_0:\star$,
the body of the $\mu$ should be kinded: $T:\star$, and as a result 
the complete $\mu$-expression is kined as $(\mu_{\star}\alpha_0.T):\star$.

For type constructors of kind $\star\to\star$, which take one type argument to
produce a type, we have the $\mu_{\star\to\star}$ operator to define recursive
type constructors, or families of recursive types indexed by a type. In order to
have $\mu_{\star\to\star}\alpha_1.F$ be well kinded, the bound variable should
be kinded: $\alpha:\star\to\star$, the body of the $\mu$ should be kinded:
$F:\star\to\star$, and as a result the complete $\mu$-expressions should be
kinded: $(\mu_{\star\to\star}\alpha_1.F):\star\to\star$.

Note, it is not enough to restrict $\alpha_1:\star\to\star$ to be used
in strictly positive positions since it represents the type constructor
defined by the recursive definition, which takes a type and produces a type.
In the case of $\mu_\star$, where the type variable $\alpha_0$ ranges over
simple types, it is okay to consider that $\alpha_0$ corresponds to well-behaved
inductive types, provided that all the ground types in the language 
(\eg $\mathsf{Unit}$) correspond to well-behaved inductive types.
For the type variable $\alpha_1$ which is bound by $\mu_{\star\to\star}$
(more generally, by $\mu_\kappa$ where $\kappa$ is not $\star$), which ranges
over type constructors, we cannot make the same argument as we did for
$\alpha_0$, since we don't really have ground type constructors to serve
as a base case argument.

%% TODO cite also Matthes' PhD thesis????
\citet{Mat99} gives a remarkably good answer by relating extensions of
System \F\ with inductive types and fixed-point types (\ie recursive types)
using a notion of \emph{monotonicity}, and later generalizes the notion of
monotonicity to an extension of System \Fw\ for type constructors of rank 2
\cite{Mat01}. Instead of requiring syntactic constraints on $\mu\alpha.T$,
only a \emph{monotonicity witness}, which is a term of type
$\forall\alpha.\forall\beta(\alpha\to\beta)\to T\to T[\beta/\alpha]$,
is required \cite{Mat99}. The primitive recursion over any recursive type
(of rank 1, or kind $\star$) is guaranteed to terminate, once we can provide
a monotonicity witness for that type. \citet{Mat01} generalizes the
notion of monotonicity up to rank 2 type constructors, and poses the open
question whether the notion of monotonicity could generalize further to
type constructors of rank higher than 2.
%% another open questions can monotonicity can work for course of values rec?

Although monotone types are very good penalization of
strictly positive datatypes and positive datatypes,
sharing the same computational behavior in regards to primitive recursion,
it is not the definitive answer for types being inductive.
Being a monotone type makes it a good candidate for being an inductive type,
but it is not a sufficient condition. The inductive type paradigm requires
set theoretic interpretation of types to view them as propositions.
Some positive, but not strictly positive, datatypes do not have set theoretic
interpretations, although all positive datatypes are monotone. For instance,
The recursive type $\mu \alpha.(\alpha\to\textsf{Bool})\to\textsf{Bool}$ is
positive, but not strictly positive. This type, when interpreted as
a set theoretic proposition, asserts an isomorphism between
the powerset of powerset of $\alpha$ and $\alpha$ itself,
which is a set theoretic nonsense.
Whether and how monotone data constructors of higher-rank can be understood
inside the inductive type paradigm is (to my knowledge) an open question.

\subsubsection*{Question (2): How should non-inductive recursive types be treated?}

Some recursive types do not correspond to inductive types. That is, there exist
recursive types, which cannot be interpreted as sets. \citet{Mendler87}
showed that reflexive types can introduce non-terminating computation
even without having unrestricted general recursion in the language.
\citet{ConMen85} suggested an approach of interpreting
non-inductive recursive types as Scott domains, and the function space over
such recursive types as partial functions over the Scott domains.
Scott domains are mathematical models, developed by \citet{Sco76},
for the languages capable of expressing non-terminating computations such as
the untyped lambda calculus and languages with unrestricted general recursion.
For example, in the type for infinitely branching trees
$\mu \alpha. T + (\mathbb{N}\to\alpha)$, where the leaves contain values of
type $T$, they use the usual arrow ($\to$) for total functions, and in
the reflexive type $\mu \alpha. \alpha \rightsquigarrow \alpha$,
which represents the semantics of the untyped lambda calculus,
they use a different arrow ($\rightsquigarrow$) for partial functions.

Although interpreting non-inductive recursive types as Scott domains
is indeed a sound interpretation, it is, in a way, an over-approximation.
Non-termination is not a characteristics of non-inductive types in general.
Recall that the inductive type paradigm is all about insisting that types have
simple interpretations as sets (more generally, indexed families of sets).
When we are faithful to the design principles of inductive types,
we get strong normalization of the language as a consequence. However,
not all strongly normalizing languages belong to the inductive type paradigm.
There exist well known strongly normalizing languages, which do not have
set theoretic interpretations. \citet{ReyPlo93} proved
the non-existence of a set theoretic interpretation for System \F\ using
the encodings of non-strictly positive datatypes. Sets cannot be interpretations
for some System \F\ types, but neither are Scott domains satisfactory, since
all functions are total in System \F. Therefore, it is an over-approximation
to categorize all the function spaces over non-inductive recursive types as
partial functions of Scott's domain theory.

Therefore, I strongly believe that a better approach is to separate concerns
about termination of programs from concerns about the inductiveness of types
definitions (recall Figure\;\ref{fig:probspace} discussed in
\S\ref{sec:intro}). These two properties need not always be linked
by design: inductiveness does require termination, but non-inductiveness
does not imply non-termination. Let us first observe how certain desired
properties are guaranteed for inductive types, and recognize that the same
strategy can apply to recursive types.

%% would "formation" be a better word than "definition" below ???
%% would "elimination" be a better word than "use" below ???
We should recognize that the type \emph{definition} itself does not guarantee
types to be inductive (\ie interpreted as sets). We also need some restrictions
on how we \emph{use} the terms of those types.  When we allow unrestricted
general recursion in the language, types cannot be interpreted as sets,
regardless of whether they are inductive or not. In the presence of unrestricted
general recursion, all types need to be interpreted as Scott domains,
in order to handle non-termination. Note, the type definitions,
which look like inductive types, (or possibly inductive definitions)
become truly inductive type definitions only when we stick to
principled recursion schemes, which are known to guarantee termination
(\eg primitive recursion, structural recursion). In other words, we must rely on
principled \emph{use} of inductive types to guarantee desired properties such as
termination and the Curry-Howard correspondence (\ie types are propositions and
programs are proofs), as well as the inductiveness of their \emph{definitions}.

Similarly, principled \emph{uses} of recursive types can guarantee certain
desired properties such as termination. Although the termination property (or
strong normalization) of recursive types has been observed in various contexts
\cite{Mendler87, Mendler91, Geu92, Mat98, Mat05, Bla03, MeiHut95, FegShe96,
DesPfeSch97, DesLel99, bgb, AbeMatUus03, AbeMatUus05, AhnShe11}, % TODO Harper
it has not been put to use in a systematic way of handling termination
separate from inductiveness in any language design we are aware of.
There have been largely two approaches to handling non-termination
in formal reasoning systems: using coinductive types and modeling types
as domains. Neither of these two approaches handle all possible combinations of
termination and inductiveness in a systematic way. The coinductive types
approach are limited to strictly positive types, just as inductive types are.
The types as domains approach has the problem of over-approximating
non-termination as we discussed earlier.

Many proof assistants (e.g. Cog, Agda, Epigram) support coinductive types
as well as inductive types. Coinductive types (or, greatest fixpoint types),
which are dual constructions of inductive types (or, least fixpoint types),
can have values that are possibly infinite (\eg infinite lists). We can model
certain class of non-terminating computations by principled corecursion schemes
over such infinite structures (or, codinductive values).
Nevertheless, coinductive types are limited to strictly positive datatypes,
just as inductive types are. That is, even with coinductive types, we cannot
directly express and reason about all recursive types in general, especially
the non-inductive types.

The types as domains approach originates from Scott's observation that types
used in programs are different from types used in logic \cite{Sco69,Sco93}.
The systems \cite{Mil72,GorMilWad79,MulNipOheSlo99} supporting Scott's
Logic for Computable Functions (LCF) model types as domains to reason about
possibly non-terminating programs. The functions in LCF are partial functions
defined over Scott domains. However, LCF is not designed for reasoning about
types in programs coincide with types in logic, or totality of functions
defined by principled recursion schemes. In other words,
using the four problem spaces we defined in \S\ref{sec:intro}, we can say
that Scott's domain theory, or LCF, does not distinguish \REC\ from \RECbot.

We will discuss some more details on the termination property of
non-inductive recursive types in \S\ref{sec:prelim}.

\subsection{Examples of non-inductive recursive types}
\label{ssec:nonindrec}
In this section, I introduce three example programs which make use of
non-inductive recursive types:
Scott numerals (\S\ref{sssec:ScottNum}),
Higher-Order Abstract Syntax (\S\ref{sssec:HOAS}), and
Normalization by Evaluation (\S\ref{sssec:NbE}).
Here, and in the other parts of this document, we will use the
{\em strictly positive types} as an example of inductive recursive types for simplicity,
rather than examining the more general concept of monotonicity.
We can categorize non-inductive recursive types, that is
non-strictly positive datatypes, into two categories:
(1) positive (but not strictly positive) datatypes, and (2) negative datatypes.
The type for the Scott numerals is an example of a (not strictly) positive datatype
(\aka covariant recursive types), and the others are examples of
negative datatypes (\aka contravariant types).

\subsubsection{Scott numerals} \label{sssec:ScottNum}

 There are multiple ways of encoding the natural numbers
in lambda calculi.
The Church numerals are the most well known of those encodings.
An alternate encoding in the untyped lamba calculus is one proposed by Scott.
%% \cite{?} in the year ?.
In 1993, \citet{AbaCarPlo93} wrote a note on assigning a type
to the constructors of the Scott numerals in an extended System F 
(extended with covariant recursive types, \ie positive datatypes). 

You can observe
the different encodings of zero ($\mathsf{0}$), successor ($\mathsf{succ}$),
and the primitive conditional operations: $\mathsf{case}$ (for the Scott numerals),  
and $\textsf{zero?}$ (for the Church numerals) in Figure\;\ref{fig:ScottNum}.
The expression $\textsf{zero?}\;n$ reduces to
$\textsf{T}$ when $n$ reduces to the Church numeral $\mathsf{0}$, and
$\textsf{F}$ otherwise, where $\mathsf{T}\defeq\lambda x.\lambda y.x$
and $\mathsf{F}\defeq\lambda x.\lambda y.y$ are the Church encodings of
the boolean values, True and False.
The expression $\textsf{case}\;n\;a\;f$, reduces to
$a$, when $n$ reduces to the Scott numeral $\mathsf{0}$, and
$f\,n'$ otherwise, where $n'$ is the predecessor of $n$.

\paragraph{}
Let us start the discussion in the context of untyped lambda calculus
(see the upper row of Figure\;\ref{fig:ScottNum}). 

The normal form for the Church numeral of value $n$ is
$\lambda x.\lambda f.f^n\;x$, where $f^n\;x$ is an abbreviation for
$n$ applications of $f$ to $x$. For instance, the normal form for
the Church numeral $3$ is $\lambda x.\lambda f.f(f(f\;x))$.
The normal form for the Scott numeral of value $n>0$ is
$\lambda x.\lambda y.y\;n'$, where $n'$ is the normal form for
the Scott numeral of value $n-1$ (\ie predecessor of $n$). For instance,
the normal form for the Scott numeral $3$ is
$\lambda x.\lambda y.
 y(\lambda x.\lambda y.
  y(\lambda x.\lambda y.
   y(\lambda x.\lambda y.x)))$.

One advantage of
the Scott numerals over the Church numerals is the existence of
the predecessor function $\lambda n.n\,\mathsf{0}\,\mathsf{id}$,
where $\mathsf{id}\defeq\lambda x.x$, which reduces in a constant number of steps
when applied to a Scott numeral in normal form. This is not the case for
the predecessor function for Church numerals, which needs a number of reduction steps
linearly proportional to the value of the applied argument.

\begin{figure}\small\centering
\begin{minipage}{.5\textwidth}
Church numerals in untyped $\lambda$ calculus
\begin{align*}
\textsf{0}     &\defeq \lambda f.\lambda x. x \\
\textsf{succ}  &\defeq \lambda n.\lambda f.\lambda x. f\,(n\,f\,x)\\
\textsf{zero?} &\defeq \lambda n. n\,(\lambda x.\mathsf{F})\,\mathsf{T}
\end{align*}
\end{minipage}
\begin{minipage}{.45\textwidth}
Scott numerals in untyped $\lambda$ calculus
\begin{align*}
\mathsf{0}    &\defeq \lambda x.\lambda y. x \\
\mathsf{succ} &\defeq \lambda n.\lambda x.\lambda y. y\,n \\
\mathsf{case} &\defeq \lambda n.\lambda a.\lambda f. n\,a\,f
\end{align*}
\end{minipage}
\\~\\~\\
\begin{minipage}{.5\textwidth}
Church numerals in System $\F^{\phantom{+\mu\text{(equi)}}}$
\begin{align*}
        N &\defeq \forall\beta.(\beta\to\beta)\to\beta\to\beta\\
\textsf{0}    &\defeq \Lambda\beta.
                      \lambda f:(\beta\to\beta).\lambda x:\beta.x\\
              &\;:\;N\\
\textsf{succ} &\defeq \lambda n:N.
                      \Lambda\beta.
                      \lambda f:(\beta\to\beta).\lambda x:\beta.
                      f\,(n\,\beta\,f\,x)\\
              &\;:\;N\to N\\
\textsf{zero?}&\defeq \lambda n:N.\Lambda\beta.n\,\beta\,(\mathsf{T}\,\beta)\,(\lambda x:\beta.\mathsf{F}\,\beta)\\
              &\;:\;N\to B
\end{align*}
\end{minipage}
\begin{minipage}{.45\textwidth}
Scott numerals in System $\F^{+\mu\text{(equi)}}$
\begin{align*}
        S    &\defeq \mu\alpha.\forall\beta.\beta\to(\alpha\to\beta)\to\beta\\
\mathsf{0}   &\defeq \Lambda\beta.\lambda x:\beta.\lambda y:(S\to\beta). x \\
             &\;:\; S\\
\mathsf{succ}&\defeq \lambda n:S.\Lambda\beta.\lambda x:\beta.\lambda y:(S\to\beta). y\,n \\
             &\;:\; S\to S\\
\mathsf{case}&\defeq \lambda n.\lambda a.\lambda f. n\,\alpha\,a\,f\\
             &\;:\; S\to \forall \beta.\beta\to(S\to \beta)\to \beta
\end{align*}
\end{minipage}
\caption{The Church numerals and the Scott numerals
         in the untyped and typed $\lambda$ calculi}
\label{fig:ScottNum}
\vspace*{.5em}\hrule
\end{figure}

Now, let us shift our discussion to similar encodings in a typed calculi
that are powerful enough to assign types to each of the different encodings of
natural numbers (see the bottom row of Figure\;\ref{fig:ScottNum}). 

On one hand, we can
assign types to the Church numerals in System \F\ (without any extensions).
The type $N$ for the Church numerals is
$\forall\beta.(\beta\to\beta)\to\beta\to\beta$, which is
an impredicative encoding (\ie $\beta$ can be instantiated with $N$ itself)
of the natural number type. The boolean type ($B$) and
its values (\textsf{T} and \textsf{F}) appearing in the definition of
\textsf{zero?} are defined as: $B\defeq \forall\beta.\beta\to\beta\to\beta$,
$\mathsf{T}\defeq \Lambda\beta.\lambda x:\beta.\lambda y:\beta.x$, and
$\mathsf{F}\defeq \Lambda\beta.\lambda x:\beta.\lambda y:\beta.y$.

On the other hand, we cannot assign proper types to the Scott numerals
in System \F. We can assign types to the Scott numerals only when we have
an extended System \F\ with positive datatypes. This extra power required
for the type system, in order to type the Scott numerals, is due to
the ability to define a constant time predecessor.\footnote{
More generally, predecessor-like functions (\eg a tail function for lists) of
constant reduction steps are not known to be definable in System \F. This is a
characteristic that distinguishes iteration from (primitive) recursion.
I will discuss this further in \S\ref{sec:prelim}.}

The type $S$ for Scott Numerals is defined to be
$\mu\alpha.\forall\beta.\beta\to(\alpha\to\beta)\to\beta$,
which is a positive, but not strictly positive, datatype.
Note, $\alpha$ appears in a double negative position, thus, positive.
To make this clear, let us explicitly parenthesize the $\to$,
which is right associative, as follows:
$\mu\alpha.\forall\beta.\beta\to((\alpha\to\beta)\to\beta)$.
Then, we can observe that $((\alpha\to\beta)\to\beta)$ is in a positive position
since it appears to the right of $\to$. The subterm $(\alpha\to\beta)$ is in
a negative position, since it is to the left of $\to$, and the variable $\alpha$
is also in negative position inside the negative subterm $(\alpha\to\beta)$.
Thus, the variable $\alpha$ is in a positive (or, covariant) position.

\begin{figure}
\begin{verbatim}
{-# LANGUAGE RankNTypes #-}

module Scott where

data Scott = Scott 
    (forall b . b                   -- return this if its zero
                -> (Scott -> b)     -- how to preceed given the predecessor?
                -> b)

z   = Scott n  where n z s = z
  
s x = Scott n  where n :: b -> (Scott -> b) -> b
                     n z s = s x

scottCase:: Scott -> a -> (Scott -> a) -> a
scottCase (Scott n) a f = n a f

pred n = scottCase n z id

pred2 (Scott n) = n z id
\end{verbatim}
\caption{Scott numerals in Haskell}\vspace*{.5em}\hrule
\label{fig:ScottNum}
\end{figure}

Figure\;\ref{fig:ScottNum} may be helpful for understanding Scott numerals,
if you are familliar with Haskell or any other similar functional language.

\subsubsection{Higher-Order Abstract Syntax} \label{sssec:HOAS}
An important example that uses a negative datatype is
Higher-Order Abstract Syntax (HOAS) \cite{Church40,MilNad87,PfeEll88}.
HOAS is a technique used to model an object-language with a syntactic
form which is binding construct (the syntactic form binds a variable
in the scope of another term). The technique uses a data structure in
the host-language (or, meta-language) which employs a meta-language function
embedded in the data structure to encode the binding construct.

For instance, the recursive type definition of the HOAS for
the untyped lambda calculus, which is the most simple HOAS, can be defined as
$\mu\alpha.(\alpha\to\alpha)+(\alpha\times\alpha)$. Note, $\alpha$ appears
in a negative position, left of $\to$ in $(\alpha\to\alpha)$. In this example
the left summand $(\alpha\to\alpha)$ is used to encode the binding lambda, and
the right summand $(\alpha\times\alpha)$ is used to encode
the binary application. In Haskell we might encode this negative datatype
as follows:
\begin{verbatim}
  data Term  =  Lam (Term -> Term)  |  App (Term, Term)
\end{verbatim}

When we use HOAS, we get capture avoiding substitution over
the object-language terms, as a simple homomorphism. That is, HOAS lifts the burden of writing
substitution functions in programming language implementations, and
the burden of proving substitution lemmas in the mechanized metatheories of
programming languages. For this reason, HOAS is used in implementations of
interpreters and partial evaluators \cite{SumKob99,DanRhi00,CarKisSha09},
and in mechanized theories \cite{Des95,HonMicSca01,Abe08,Chl08}.
%% maybe TODO find more references and HOAS itself is a huge subject
%% bgb paper and tech report have many good pointers

\subsubsection{Normalization by Evaluation}\label{sssec:NbE}
Another well-known use of contravariant recursive types (or, negative datatypes)
appear in the work on Normalization by Evaluation (NbE) \cite{BerSch91}.
The idea of NbE is to use an evaluator, or an interpreter,
(\ie denotational semantics) for normalization. More specifically,
NbE works by evaluating a syntactic term into a value in the semantic domain,
and then reifying the resulting value of the evaluation back into
a normalized syntactic term. The recursive type representing
any interesting semantic domain involving functions is always
a negative datatype, since the semantic domain would contain its function space
(\ie $D \supseteq [D\to D]$). Recently, NbE has been popularized as
an implementation technique for dependently typed languages
\cite{LohMcbSwi07,AbeAehDyb07,AbeCoqDyb07}, some of which are
specifically studied in the context of Martin-L\"of's type theory.
Note, such implementations using NbE rely on recursive type definitions
for semantic domains, which is not admissible in Martin-L\"of's type theory. 
An amusing irony: to believe in the soundness of this kind of implementations
for the Martin-L\"of's type theory, we also have to believe in the soundness of
certain use of recursive types, which are outside the scope of
the Martin-L\"of paradigm.




%% =====================================\\
%% TODO study about Bove-Capretta technique and Strong Computability used in
%% the thesis of James Chapman http://www.ioc.ee/~james/Publications.html
%% 
%% ====================================\\
%% \paragraph{}
%% some recent article on delimited continuation.
%% I think Sabry's paper contains an Haskell datatype definition for it.\\
%% http://www.pps.jussieu.fr/~saurin/tpdc2011/\\
%% I wonder whether this is an example of a non-monotonic rank-2 tycon or not.
%% 
%% \paragraph{}
%% 
%% Discussion from the types mailing list\\
%% http://www.cis.upenn.edu/~bcpierce/types/archives/1993/msg00027.html\\
%% http://www.cis.upenn.edu/~bcpierce/types/archives/1993/msg00032.html
%% 
%% \paragraph{}
%% %%% TODO
%% Inductive Datatype System\\
%% http://www-rocq.inria.fr/~blanqui/papers/tcs02.pdf\\
%% have a short introduction to the pointers of important papers
%% 
%% \paragraph{}
%% ====================================\\
%% TODO
%% Computation is mainly centered around the semantic domain,
%% which is an exotic recursive type. But, once the main computation
%% (\ie evaluation) is done, it comes back to the syntactic terms,
%% which is an inductive type. This pattern can be useful not only in NbE,
%% but in many other situations. I think this is what we want to be able
%% to express in Trellys. also related to extensional type theory

%% TODO??? there is a more fundamental/philosophical issue why type theory
%% should be based on set theory, but I'm not going into this unless I have
%% excessive time

%% TODO do we need discussions on extensional and intensional type theory???

%% TODO find citations for applications of HOAS

%% TODO find citations for applications of HOAS


%% Types are not sets
%% http://portal.acm.org/citation.cfm?doid=512927.512938

%% Logical Framework and HOAS

%% Delphin language and HOAS
%% Programming with Higher-Order Abstract Syntax by Sch\"urmann

%% inductive datatypes

%% maybe sketch of type soundness proof of STLC + mu
%% then System \F\ encoding of natural numbers and lists, and maybe mu???
%% then \Fw\ encoding of mu and mcata
%% induction is not derivable form etc
%% cf. Mendler-style and conventional style later on in different section

%% \subsection{TODO summary and discussions}
%% summary and questions/discussions leading the following sections



\section{Preliminary work: Mendler style Iteration over Negative Datatypes}
\label{sec:prelim}
Our preliminary work \cite{AhnShe11} is about a Mendler style iteration over
negative datatypes using syntactic inverses. In this section, I will
introduce the concept of iteration and recursion (\S\ref{ssec:iter}),
and iteration in Mendler style (\S\ref{ssec:Mendler}). Then, I will give
a summary of the preliminary results (\S\ref{ssec:msfcata}),
related work (\S\ref{ssec:MendlerRW}), and
future work (\S\ref{ssec:MendlerFW}) I plan to work on.

\subsection{Iteration, (primitive) recursion, and induction} \label{ssec:iter}
In this subsection, we discuss datatypes in regards to iteration, recursion,
and induction.
I will first overview the inclusion relation between the types related to
each three concepts (\S\ref{sssec:inclusionIterRecInd}), and contrast
the difference between iteration and primitive recursion on natural numbers
(\S\ref{sssec:recVSiter}). Then, I will introduce folds (\S\ref{sssec:folds}),
which are implementations of iteration in functional language,
over inductive datatypes other than natural numbers.

\subsubsection{Inclusion relations of the types relevant to each concept}
\label{sssec:inclusionIterRecInd}

\paragraph{Induction and recursion:}
All types having well-funded induction are also normalizing under
primitive recursion. However, there exists types normalizing under
primitive recursion, which do not have well-founded induction principle
with set theoretic interpretations.

A proof by induction can be realized by primitive recursion \cite{PfePau90}.
That is, the computational content of a proof term using an induction principle
is no more than a primitive recursive function. \citet{PfePau90} showed that
extending the Calculus of Constructions by inductive types and their induction
principles does not alter the set of functions in its computational fragment,
\Fw. In short, inductive types are normalizing under primitive recursion.
However, not all primitive recursive functions have their logical counterparts
of inductive predicates. As we mentioned earlier in \S\ref{ssec:recVSind},
not all types normalizing under primitive recursion have set theoretic
interpretations.

\paragraph{Recursion and iteration:}
All types normalizing under primitive recursion are also normalizing under
iteration. However, there exists types normalizing under iteration but
not under primitive recursion. Negative datatypes are not normalizing
under primitive recursion. In fact, Mendler observed that negative datatypes
are not normalizing even without any recursion at the term level.
The following is Mendler's observation transcribed into Haskell:
\begin{verbatim}
data T = C (T -> ())
p (C f) = f
w t = (p t) t

selfapp = w (C w)  -- corresponds to (\x.xx) (\x.xx) in untyped lambda calc
\end{verbatim}
The inductive datatype \verb|T| is negative, and \verb|p| and \verb|w| are
well typed non-recursive functions. Note, we did not use any recursion above,
yet \verb|selfapp| diverges:
\verb|w (C w)| $\rightsquigarrow$ 
\verb|(p (C w)) (C w)|  $\rightsquigarrow$
\verb|w (C w)|  $\rightsquigarrow \cdots$.
Another more interesting example, which shows that negative datatypes can
encode non-termination, is the HOAS for untyped lambda calculus:
\begin{verbatim}
data Exp = Lam (Exp -> Exp) | App (Exp, Exp)
\end{verbatim}
This datatype can model arbitrary terms in the untyped lambda calculus,
some of which are diverging.

\paragraph{}
I have mentioned several times in this document that iteration can ensure
normalization for negative datatypes as well as positive datatypes, but
have not yet introduced what iteration is. In the remainder of this subsection,
I will introduce the concept of iteration in comparison to primitive recursion.
I will first start with the most simple case of natural numbers, and then
other inductive types such as lists and trees.
The normalization property of iteration over recursive types in general,
including negative datatypes, will be discussed in the next subsection
(\S\ref{ssec:Mendler}) when we compare two different styles of forming
catamorphsim, which is another name for iteration in the context of
functional languages.

\subsubsection{Primitive recursion and iteration on natural numbers}
\label{sssec:recVSiter}
\paragraph{Primitive recursion:} The primitive recursion on natural numbers,
in the tradition of G\"oedel's System \textsf{T} \cite{God58},
can be defined by the three reduction rules as follows:
\[\inference[\textsf{Pr-0}]{}{\mathsf{Pr}\;\mathsf{0}\;e_0\;e_2 \to e_0}\]
\[\inference[\textsf{Pr-s}]{}{\mathsf{Pr}\;(\mathsf{S}\,n)\;e_0\;e_2 \to
                              e_2\;n\;(\mathsf{Pr}\;n\;e_0\;e_2)}\]
\[\inference[\textsf{Pr-}ctx]{e\to e'}
       {\mathsf{Pr}\;e\;e_0\;e_2 \to \mathsf{Pr}\;e'\;e_0\;e_2} \]
The primitive recursion operator, or recursor, \textsf{Pr} have three arguments:
the first argument is the natural number to recurse on;
the second argument is the resulting expression
when the value of the first argument is zero; and
the third argument is an expression expecting two arguments,
which we use when the value of the first argument is non-zero.

The first rule \textsf{Pr-0} defines the reduction when the first argument
is zero (\textsf{0}), simply reducing to $e_0$.

The second rule \textsf{Pr-s} defines the reduction when the first argument
is in the successor form ($\mathsf{S}\,n$). The result of the reduction is
$e_2$ applied to the predecessor $n$ and the result of the primitive recursion
over the predecessor $(\mathsf{Pr}\;n\;e_0\;e_2)$.

The third rule \textsf{Pr-}ctx defines the reduction when the first argument
is not in canonical form (\ie either $\mathsf{0}$ or $\mathsf{S}\;n$).
It is a self explanatory context rule.

\paragraph{Iteration:} We can formulate the iteration on natural numbers
in a similar fashion to the primitive recursion as follows:
\[\inference[\textsf{It-0}]{}{\mathsf{It}\;\mathsf{0}\;e_0\;e_1 \to e_0}\]
\[\inference[\textsf{It-s}]{}{\mathsf{It}\;(\mathsf{S}\,n)\;e_0\;e_1 \to
                              e_1\;(\mathsf{It}\;n\;e_0\;e_1)}\]
\[\inference[\textsf{It-}ctx]{e\to e'}
                   {\mathsf{It}\;e\;e_0\;e_1 \to \mathsf{It}\;e'\;e_0\;e_1} \]
The three reduction rules for the iteration operator, or iterator, \textsf{It}
are very similar to the definition of the recursor \textsf{Pr}.
The only difference from primitive recursion is that iteration does not have
direct access to the predecessor. Note, $e_1$ only expects one argument
$(\mathsf{It}\;n\;e_0\;e_1)$, which is the result of the iteration over
the predecessor, in the second rule $\textsf{it-s}$.

\paragraph{Comparison of recursion and iteration:}
We can calculate the predecessor of $n$ using the recursor
by $\mathsf{Pr}\;n\;\mathsf{0}\;(\lambda x.\lambda y.x)$ in constant time,
provided that $n$ is in canonical form.

Calculating the predecessor using the iterator is not as simple as using
the recursor, since we can no longer directly refer to the predecessor
in iteration. It is known that the iterator \textsf{It} has
the same computational power as the recursor \textsf{Pr},
provided that we have pairs in the language \cite{AlvFerFloMac10}.
We can define \textsf{Pr} and \textsf{It} in terms of each other.
Defining \textsf{It} in terms of \textsf{Pr} is trivial:
\[ \mathsf{It}\;n\;e_0\;e_1 \defeq \mathsf{Pr}\;n\;e_0\;(\lambda x.e_1)
   ~~~~~\text{where $x$ does not appear free in $e_1$} \]
Conversely, we can define \textsf{Pr} in terms of \textsf{It} using pairs,
storing the predecessor in the first element and
the result of the iteration in the second element, as follows:
\[ \mathsf{Pr}\;n\;e_0\;e_2 \defeq
 \pi_2(\mathsf{It}\;n\;\langle e_0,\mathsf{0}\rangle\;
  (\lambda y.\langle e_2\,(\pi_1\,y)\,(\pi_2\,y),\,\textsf{S}(\pi_1\,y)\rangle))
\]
However, the number of reduction steps required for calculating the predecessor
is not constant when we use this encoding of the recursor, which is defined
in terms of the iterator. Using this encoding of the recursor, the number of
reduction steps for calculating the predecessor is linear to the value of
the given natural number $n$.

The observation that computational power of primitive recursion and iteration
is the same but efficiency differs holds for inductive datatypes more generally.
I will shortly introduce the iteration for other datatypes, which is also called
folds in the context of function programming, in \S\ref{sssec:folds}.
However, for non-inductive datatypes, especially for negative datatypes, this
observation no longer holds. In fact, iteration needs to be formulated in a
different style since popular style of formulating folds only generalize to
limited class of inductive types (\ie not even all inductive datatypes, not to
mention of non-inductive datatypes such as negative datatypes). I will introduce
two different styles of formulating iteration in \S\ref{ssec:Mendler}.

\subsubsection{Iterators, or folds, for other inductive datatypes}
\label{sssec:folds}
We have discussed iteration over natural numbers so far.
Similarly, we can define iteration for other inductive datatypes.
In functional languages, the functions implementing iteration are called folds.
The following Haskell code is the definitions of folds for several datatypes:
{\small
\begin{verbatim}
data Nat    = Zero | Succ Nat
data List p = Nil | Cons p (List p)
data Tree p = Leaf p | Node (Tree p) (Tree p)
data Blah p = Con1 p | Con2 Int p | Con3 p (Blah p) | Con4 (Blah p) p

foldNat  :: a -> (a->a) -> Nat -> a
foldNat v f Zero     = v
foldNat v f (Succ n) = f (foldNat v f n)

foldList :: a -> (p->a->a) -> List p -> a
foldList v f Nil         = v
foldList v f (Cons x xs) = f x (foldList v f xs)

foldTree :: (p->a) -> (a->a->a) -> Tree p -> a
foldTree fL fN (Leaf x)     = fL x
foldTree fL fN (Node t1 t2) = fN (foldTree fL fN t1) (foldTree fL fN t2)

foldBlah :: (p->a) -> (Int->p->a) -> (p->a->a) -> (a->p->a) -> Blah p -> a
foldBlah f1 f2 f3 f4 (Con1 x)   = f1 x
foldBlah f1 f2 f3 f4 (Con2 n x) = f2 n x
foldBlah f1 f2 f3 f4 (Con3 x b) = f3 x (foldBlah f1 f2 f3 f4 b)
foldBlah f1 f2 f3 f4 (Con3 b x) = f4 (foldBlah f1 f2 f3 f4 b) x
\end{verbatim}
}
The function \texttt{foldNat} for the natural number type \texttt{Nat}
is basically the same definition to the iterator \textsf{It} we discussed
earlier, except that the natural number we recurse on passed into
the last argument rather than the first argument. The first argument \texttt{v}
is the answer when the last argument is zero (\texttt{Zero}), and
the second argument \texttt{f} is the function to be applied to answer of
the fold over the predecessor (\texttt{foldNat v f n}) when the last argument
is non-zero (\texttt{Succ n}).

The function \texttt{foldList} is the fold for the list type \texttt{List}.
Since \texttt{List} have two constructors, like \texttt{Nat} does,
\texttt{foldList} also have two arguments before the list argument to
fold over. The difference from \texttt{foldNat} is that the second argument
\texttt{f} is a binary function, rather than unary function,
for the non-empty list case, which combines the head element (\texttt{x})
with the answer of the fold over the tail (\texttt{foldList v f xs}).

The function \texttt{foldTree} is the fold for the binary tree type
\texttt{Tree}. Since there are two constructors \texttt{Leaf} and \texttt{Node},
we have two arguments \texttt{fL} and \texttt{fN} to handle each constructor.
The \texttt{fL} is a unary function applied to the value inside \texttt{Leaf}.
The \texttt{fN} is a binary function combines the answers of the folds
over the left and right children of the \texttt{Node}.

You can see that the type signature of the fold functions become larger as
the number and arity of the data constructors become larger. The fold for
\texttt{Blah} type (\texttt{foldBlah}) have four arguments (\texttt{f1},
\texttt{f2}, \texttt{f3}, and \texttt{f4} before it takes the \texttt{Blah}
argument to fold over. Each of those argument has the appropriate type
to handle the values inside each of the constructor.

All of these folds are normalizing, provided that supplied arguments are
normalizing. Why? It is because they are structurally recursive on its
last argument. The recursive call is always on the recursive subcomponents.
For \texttt{Nat}, the recursive call is on the predecessor;
for \texttt{List}, the recursive call is on the tail;
for \texttt{Tree}, the recursive calls are on the children
in the definition of their folds. When datatypes are well founded,
which is the case for inductive datatypes, they is guaranteed to terminate
by structural recursion.

Let us see some functions we can define in terms of folds.
For example, we can define a sum function for lists (\texttt{sumList})
and a length function for lists \texttt{lenList} as follows:
\begin{verbatim}
sumList = foldList 0 (\x a->x+a)
lenList = foldList 0 (\x a->1+a)
\end{verbatim}
The following illustrates some of the reduction steps calculating the sum of
all the elements in an integer list using \texttt{sumList}:\\
$\phantom{\rightsquigarrow^{*}}$ \verb|sumList (Cons 1 (Cons 2 (Cons 3 Nil)))|\\
$\rightsquigarrow^{*}$ \verb|foldList 0 (\x a->x+a) (Cons 1 (Cons 2 (Cons 3 Nil)))|\\
$\rightsquigarrow^{*}$ \verb|(\x a->x+a) 1 (foldList 0 (\x a->x+a) (Cons 2 (Cons 3 Nil)))|\\
$\rightsquigarrow^{*}$ \verb|(\x a->x+a) 1 5|\\
$\rightsquigarrow^{*}$ \verb|1+5|\\
$\rightsquigarrow^{*}$ \verb|6|

We will revisit this example when we discuss catamorphism.

\subsection{Conventional (Squiggol style) vs. Mendler style}
\label{ssec:Mendler}
Another name for iteration, in the context of functional programming,
is catamorphism. Catamorphism is a generalization of folds. So far, we have
seen how to formulate folds individually for each inductive datatype.
Although we can informally observe that those folds are based on the same
concept and have similar properties, we have not yet formally related them,
or unified into a common interface. To formally describe the general concept
of iteration (\aka catamorhpsim), or unify the interface of folds, we first
tear down an inductive type into two levels, a datatype fixpoint, which ties
the knot of the recursion in a datatype, and a base datatype, which describes
the shape of the datatype including where the recursion occurs. Such a two
level type definitions correspond to iso-recursive types, whereas
plain recursive definitions in the previous subsection correspond to
equi-recursive types. The two level types are commonly used in
both conventional and Mendler style catamorphism.

The conventional (or, Squiggol style) catamorphism (\S\ref{sssec:cata})
is a generalization of folds (\S\ref{sssec:folds}).
The Mendler style catamorphism (\S\ref{sssec:mcata}) is another way of
formulating catamorphism. The Mendler style catamorphism is considerably
more expressive than the conventional catamorphism.

\subsubsection{Two level types are iso-recursive types}
Let us see how we can define the datatypes in the example of folds
(\S\ref{sssec:folds}).

The natural number dataype \texttt{Nat} can be split into two levels as follows:
\begin{verbatim}
newtype Mu f = Roll (f (Mu f))  -- datatype fixpoint
unRoll (Roll e) = e          -- reduction rule for unRoll . Roll composition

data N r = Z | S r           -- base datatype for Nat

type Nat = Mu N        -- Nat defined in terms of Mu and N

zero   = Roll Z        -- data constructor correspoinding to Zero
succ n = Roll (S n)    -- data constructor correspoinding to Succ
\end{verbatim}
The datatype fixpoint \texttt{Mu} correspond to the $\mu$ binder of
the recursive type we discussed earlier in \S\ref{ssec:rectype}.
The data constuctor \texttt{Roll :: f (Mu f) -> Mu f} and
the destructor function \texttt{unRoll :: Mu f -> f (Mu f)}
correspond to \textsf{roll} and \textsf{unroll} appearing in
the typing rules iso-roll and iso-unroll in \S\ref{ssec:rectype}.
We can understand plain version of \texttt{Nat} in \S\ref{sssec:folds}
as equi-recursive types, and the two level type version using \texttt{Mu} as
iso-recursive types needing explicit use of \texttt{Roll} and \texttt{unRoll}.
The data constructors for \texttt{Nat} should now be encoded using \texttt{Roll}
and the data constructors of the base datatype. When we need to destruct
(\ie case match over) the values of \texttt{Nat}, we should use
the deconstructor function \texttt{unRoll}.
For instance, the case match in the plain version of \texttt{Nat}:
\begin{verbatim}
  case n          of { Zero -> e0; Succ n -> e1 }
\end{verbatim}
should be written as follows in the two level version:
\begin{verbatim}
  case (unRoll n) of { Z    -> e0; S n    -> e1 }
\end{verbatim}
In the two level type version, pattern matching over \texttt{Mu}
(\ie unrolling) is considered special since it is the only place
where we tie the recursive knot. All the other pattern matches are
over the non-recursive base datatypes.

We will shortly see that, this change of representation to iso-recursive types
helps us understand the essence of iteration (in contrast to recursion) more
clearly and intuitively in regards to programming language constructs.
Before we continue the discussion on iteration, or catamparhism, let us see
couple more examples of the two level type definitions for other inductive
datatypes.

We can define two level definition of the \texttt{List} datatype,
using the same datatype fixpoint \texttt{Mu}, as follows:
\begin{verbatim}
data L p r = N | C p r -- base datatype for List

type List p = Mu (L p)     -- List defined in terms of Mu and L

nil       = Roll N         -- data constructor corresponding to Nil
cons x xs = Roll (C x xs)  -- data constructor corresponding to Cons
\end{verbatim}
We can define two level definition of the \texttt{Tree} datatype similarly:
\begin{verbatim}
data T p r = L p | C p r -- base datatype for List

type Tree p = Mu (T p)

leaf x     = Roll (L x)     -- data constructor corresponding to Leaf
node t1 t2 = Roll (N t1 t2) -- data constructor corresponding to Node
\end{verbatim}
More complex inductive datatypes can be defined in two level
in a similar fashion.

\subsubsection{Conventional (or, Squiggol style) catamorphism}
\label{sssec:cata}
To generalize from folds to conventional catamorphism, we factor out
a mapping function (\texttt{fmap}),
which handles recursive calls to recursive subcomponents, and
a combining function (\texttt{phi::f a->a}), which combines
the non-recursive components with the answers of processing
the recursive subcomponents. The definition of the conventional catamorphism
\texttt{cata} is as follows:
\begin{verbatim}
cata :: Functor f => (f a->a) -> Mu f -> a 
cata phi (Roll x) = phi (fmap (cata phi) x)
\end{verbatim}
Alternatively, the same definition in point-free style is:
\begin{verbatim}
cata :: Functor f => (f a->a) -> Mu f -> a 
cata phi = phi . fmap (cata phi) . unRoll
\end{verbatim}
The definition of \texttt{cata} captures the essence of structural recursion
on inductive datatypes. Each time \texttt{cata} deepens is its recursive call
one \texttt{Roll} is discharged by pattern matching (or, \texttt{unRoll}).
Since the values of inductive datatypes consists of finite number of
data constructors, each of which is encoded by one \texttt{Roll} and
a data constructor of the base, \texttt{cata} is guaranteed to normalize,
provided that \texttt{fmap} and \texttt{phi} function is non-recursive and
does add more \texttt{Roll} constructor in its result.

The overloaded function \texttt{fmap :: Functor f => (a -> b) -> f a -> f b}
should naturally define where to apply the recursive call in the base datatype.
We should think that the definition for the \texttt{fmap} function is
naturally derived from the datatype definition, rather than a user defined
function. That is, \texttt{fmap} describes an inherent property of the
inductive datatype. For instance, the definition of \texttt{fmap} for base
\texttt{N} for natural numbers is defined as follows:
\begin{verbatim}
instance Functor N where
  fmap h Z     = Z
  fmap h (S x) = S (h x)
\end{verbatim}
Note, we do nothing for non-recursive case \texttt{Z}, and 
the function \texttt{h} is applied to the predecessor position \texttt{x}
for the recursive case (\texttt{S x}).

Similarly, the \texttt{fmap} for bases \texttt{L} and \texttt{T}
for lists and trees is defined as follows:
\begin{verbatim}
instance Functor (L p) where
  fmap h N        = N
  fmap h (C x xs) = C x (h xs)

instance Functor (T p) where
  fmap h (L x)     = L x
  fmap h (N t1 t2) = N (h t1) (h t2)
\end{verbatim}
Note, the function \texttt{h} is applied to the recursive subcomponents,
that is, to the tail position for lists and to the children position for trees.

Once we have seen how \texttt{fmap} is defined for each datatype,
we can have better understanding of \texttt{cata}.
Let us focus our attention back to the definition of \texttt{cata}.
\begin{verbatim}
cata :: Functor f => (f a->a) -> Mu f -> a 
cata phi (Roll x) = phi (fmap (cata phi) x)
\end{verbatim}
In the definition of \texttt{cata}, the first argument to \texttt{fmap}
is (\texttt{cata phi}), which is passed into \texttt{h} in the body of
\texttt{fmap}. Thus, \texttt{fmap (cata phi) :: f(Mu f) -> f a} is a function
that maps recursive subcomponents in a into answers of applying
the catamorphism to those recursive subcomponents. After processing
the subcomponents into answers, the combining function
\texttt{phi :: f a -> a} combines the base structure containing
the answers of processing the recursive subcomponents into the finial answer.

We can define specific functions by supplying the user defined combining
function into \texttt{phi}. For example, we can define the function
\texttt{sumList}, which sums up all the elements in an integer list,
as follows:
\begin{verbatim}
sumList = cata phi
  where
    phi N         = 0
    phi (C x ans) = x + ans
\end{verbatim}
The following illustrates some of the reduction steps calculating the sum of
all the elements in an integer list using \texttt{sumList}:\\
$\phantom{\rightsquigarrow^{*}}$
\verb|sumList (Roll(C 1 (Roll(C 2 (Roll(C 3 (Roll N)))))))|\\
$\rightsquigarrow^{*}$ \verb|cata phi (Roll(C 1 (Roll(C 2 (Roll(C 3 (Roll N)))))))|\\
$\rightsquigarrow^{*}$ \verb|phi (fmap (cata phi) (C 1 (Roll(C 2 (Roll(C 3 (Roll N)))))))|\\
$\rightsquigarrow^{*}$ \verb|phi (C 1 5)|\\
$\rightsquigarrow^{*}$ \verb|1+5|\\
$\rightsquigarrow^{*}$ \verb|6|

\paragraph{The limitation of conventional (or, Squiggol style) catamorhpsim}
is that it only works for regular inductive datatypes. That is, the limitation
of \texttt{cata} comes in two dimensions:
The definition of \texttt{cata} generalize neither
for non-regular datatypes (including nested datatypes)
nor for non-inductive datatypes (including negative datatypes).

Firstly, the conventional catamorphism does not generalize well to
non-regular datatypes. The datatype we have seen so far, while introducing
folds and \texttt{cata} are all regular datatypes. There exist many examples of
non-regular dataypes in functional programming including nested datatypesa
and GADTs. One example of a nested datatype is the powerlist datatype defined
as follows:
\begin{verbatim}
data Powl a = Nil | Cons a (Powl (a,a))
\end{verbatim}
Note, the recursion is non-regular in the sense that the recursion
(\texttt{Powl (a,a)}) is different from (\texttt{Powl a}).
The definition of \texttt{cata} won't generalize to nested datatypes
in a trivial way.
There has been several approaches \cite{BirPat99,MarGibBay04,Hin00}
to extend folds or conventional catamorphisms for nested datatypes.

Secondly, the conventional catamorphism does not generalize well to
non-inductive datatypes, especially for negative datatypes.

\subsubsection{Mendler style catamorphism}
The functional programming community has traditionally focused on families of
combinators that work well in Hindley-Milner languages, characterized by folds,
or more generally (conventional) catamorphism, which we have been discussed
so far. On the other hand, the Mendler style combinators were originally
developed in the context of the Nuprl \cite{Con86} type system. Nuprl made
extensive use of g polymorphism and dependent types. General type checking
in Nuprl was done by interactive theorem proving -- not by type inference.
The conventional catamorphism is widely known, especially on the list type
(\eg \texttt{foldr} in Haskell standard library). The conventional catamorphism
has been more often used in functional programming than the Mendler style
catamorphism, but it does not generalize easily to non-regular datatypes
such as GADTs, or nested datatypes.  The Mendler style catamorphism,
being free from the two limitations of the conventional style combinators,
is considerably more expressive than the conventional
(or, Squiggol school \cite{AoP} style) catamorhism.

Here, we briefly introduce Mendler style catamorphism and focus on its
characteristics on non-inductive datatypes, in particular, negative datatyps.
For further details, including its characteristics on non-regular
inductive datatypes, you may refer to our conference paper \cite{AhnShe11}
on Mendler style iteration and recursion combinators.

\subsubsection{Definition of a Mendler style catamorphism}
\label{sssec:mcata}
The definition of a Mendler style catamorphism is the following:
\begin{verbatim}
mcata :: (forall r . (r -> a) ->  (f r -> a)) -> Mu f ->  a
mcata phi (Roll x) = phi (mcata phi) x
\end{verbatim}
Although we defined \verb|Mu| as a newtype and \verb|mcata|
as a function in Haskell, you should consider them as an
information hiding abstraction.
The rules of the game are that you are only allowed to construct values
using the \verb|Roll| constructor (as in \verb|zero|, \verb|succ|,
\verb|nil| and \verb|cons|),
but you are not allowed to deconstruct those values by pattern matching
against \verb|Roll| (or, by using the selector function \verb|unRoll|).

Note, \verb|mcata| does not require \verb|Functor| class in its type signature.
The Mendler catamorphism \verb|mcata| lifts the restriction that the
base type be a functor, but still maintains the strict termination
behavior of \verb|cata|. This restriction is lifted by using two devices.
\begin{itemize}
  \item The combining function \verb|phi| becomes a function of 2 arguments
        rather than 1. The first argument is a function that represents an
        abstract recursive caller, the second the conventional base structure
        that must be combined into an answer. The abstract recursive caller
        allows the programmer to direct where recursive calls must be made.
        The \verb|Functor| class requirement is lifted,
        because no call to \verb|fmap|
        is required in the definition of \verb|mcata|:
\begin{verbatim}
mcata phi (Roll x) = phi (mcata phi) x
\end{verbatim}
  \item The second device uses higher-rank polymorphism to insist that
        the abstract caller, with type (\verb|r -> a|), and
        the base structure, with type (\verb|f r|),
        work over an abstract type, denoted by (\verb|r|). 
\begin{verbatim}
mcata :: (forall r. (r -> a) -> (f r -> a)) -> Mu f -> a
\end{verbatim}
\end{itemize}

\subsubsection{Mendler style catamorphism over inductive datatypes}

The intuitive reasoning behind the termination property of \verb|mcata| for
all inductive datatypes is that (1) \verb|mcata| strips off one \verb|Roll|
constructor each time it is called, and (2) \verb|mcata| only recurses on the
direct subcomponents (e.g., tail of a list) of its argument (because the type
of the abstract recursive caller won't allow it to be applied to anything else).
Once we observe these two properties, it is obvious that \verb|mcata| always
terminates since those properties imply that every recursive call to \verb|mcata|
decreases the number of \verb|Roll| constructors in its argument.\footnote{We assume
that the values of inductive types are always finite. We can construct infinite
values (or, co-inductive values) in Haskell exploiting lazyness, but we exclude
such infinite values from our discussion in this work.}

Writing the list length example in Mendler style will give clearer intuition
explained above. The following is a side-by-side definition of
the list length function in general recursion style (left)
and in Mendler style (right).
\begin{center}
\begin{minipage}{.49\linewidth}
\begin{verbatim}
 
data List p
   =  N
   |  C p (List p)


len :: List p -> Int
len N         = 0
len (C x xs)  = 1 + len xs
\end{verbatim}
\end{minipage}
\begin{minipage}{.49\linewidth}
\begin{verbatim}
data L p r = N | C p r
type List p = Mu (L p)
nil        = Roll N
cons x xs  = Roll (C x xs)

lenm :: List p -> Int
lenm = mcata phi where
  phi len N         = 0
  phi len (C x xs)  = 1 + len xs
\end{verbatim}
\end{minipage}
\end{center}
In the definition of \verb|lenm|,
we name the first argument of \verb|phi|, the abstract recursive caller,
to be \verb|len|.  We use this \verb|len| exactly where we would recursively
call the recursive function in the general recursion style
(\verb|len| on the left).

However, unlike the general recursion style, it is not possible to call
\verb|len::r->Int| on anything other than the tail \verb|xs::r|.
Using general recursion, we could easily err (by mistake or by design)
causing length to diverge, if we wrote its second equation as follows:
\verb|len (C x xs) = 1 + len (C x xs)|.

We cannot encode such diverging recursion in Mendler style because
\verb|len::r->Int| requires its argument to have the parametric type \verb|r|,
while \verb|(C x xs) :: L p r| has more specific type than \verb|r|.

\subsubsection{Mendler style catamorphism over negative datatypes}
\label{sssec:mcataNegative}
Let us revisit the negative inductive datatype \verb|T|
(from \S\ref{sssec:inclusionIterRecInd})
from which we constructed a diverging computation.
We can define a two level version of \verb|T|, let us name it \verb|T_m|,
as follows:
\begin{verbatim}
data TBase r = C_m (r -> ())
type T_m = Mu TBase
\end{verbatim}
If we can write two functions \verb|p_m :: T_m -> (T_m -> ())|,
and \verb|w_m :: T_m -> ()|, corresponding to \verb|p| and \verb|w|
from \S\ref{sssec:inclusionIterRecInd}, we would be able to reconstruct
the same diverging computation.
The function
\begin{verbatim}
w_m x = (p_m x) x
\end{verbatim}
is easy since it is just a non-recursive function. The function
\verb|p_m| is problematic. By the rules of the game,
we cannot pattern match on \verb|Roll|
(or use \verb|unRoll|) so we must resort to using one of the
combinators, such as \verb|mcata|.
However, it is not possible to write \verb|p_m|
in terms of \verb|mcata|.
Here is an attempt (seemingly the only one possible) that fails:
\begin{verbatim}
p_m :: T_m -> (T_m -> ())
p_m =  mcata phi
  where
    phi :: (r -> (T_m -> ())) -> TBase r -> (T_m -> ())
    phi _ (C_m f) = f
\end{verbatim}
We write the explicit type signature for the combining function \verb|phi|
(even though the type can be inferred from the type of \verb|mcata|),
to make it clear why this attempt fails to type check. The combining
function \verb|phi| take two arguments. The recursive caller (for which we
have used the pattern \verb|_|, since we don't intend to call it) and the
base structure \verb|(Cm f)|, from which we can extract
the function \verb|f :: r -> ()|. Note that \verb|r| is an abstract
(universally quantified) type, and the result type of \verb|phi| requires
\verb|f :: T_m -> ()|. The types \verb|t| and \verb|T_m| can never match, if \verb|r|
is to remain abstract. Thus, \verb|p_m| fails to type check.

There is a function, with the right type, that you can define:
\begin{verbatim}
pconst :: T_m -> (T_m -> ())
pconst = mcata phi
  where
    phi g (C f) = const ()
\end{verbatim}
Not surprisingly, given the abstract pieces composed of
the recursive caller \verb|g :: r -> ()|, the base structure \verb|(C f) :: TBase r|,
and the function we can extract from the base structure \verb|f :: r -> ()|,
the only function that returns a unit value (modulo extensional
equivalence) is, in fact, the constant function returning the unit value.

This illustrates the essence of how Mendler catamorphism guarantees
normalization even in the presence of negative occurrences in the
inductive type definition. By quantifying over the recursive type
parameter of the base datatype (e.g. \verb|r| in \verb|TBase r|), it prevents an
embedded function with a negative occurrence from flowing into any
outside terms (especially terms embedding that function).

\begin{figure}
\begin{verbatim}
data FooF r = Noo | Coo (r -> r) r
type Foo = Mu FooF
noo       = Roll Noo
coo f xs  = Roll (Coo f xs)

lenFoo :: Foo -> Int
lenFoo = mcata phi where
  phi len Noo         =  0
  phi len (Coo f xs)  =  1 + len (f xs)
\end{verbatim}
%% 
%% loopFoo :: Foo -> Int
%% loopFoo = mhist0 phi where
%%   phi out len Noo         =  0
%%   phi out len (Coo f xs)  =  case out xs of
%%                                Noo       -> 1 + len (f   xs)
%%                                Coo f' _  -> 1 + len (f'  xs)
%% 
%% foo :: Foo -- loops for loopFoo
%% foo = coo0 (coo1 noo) where   coo0 = coo id
%%                               coo1 = coo coo0
%% 

\caption{An example of a total function \texttt{lenFoo} using \texttt{mcata}
         over a negative datatype \texttt{Foo}}\vspace*{.5em}\hrule
%%%,
%%%     and a counterexample \texttt{loopFoo} illustrating that \texttt{mhist0}
%%%         can diverge for negative datatypes.}
\label{fig:LoopHisto}
\end{figure}

Given these restrictions, the astute reader may ask, are types with
embedded function with negative occurrences good for anything at all?
Can we ever call such functions?  A simple example which uses an
embedded function inside a negative inductive datatype is illustrated
in Figure \ref{fig:LoopHisto}.  The datatype \verb|Foo| (defined as a fixpoint
of \verb|FooF|) is a list-like data structure with two data constructors \verb|Noo|
and \verb|Coo|.  The data constructor \verb|Noo| is like the nil and \verb|Coo| is like
the cons.  Interestingly, the element (with type \verb|Foo->Foo|) contained \verb|Coo|
is a function that transforms a \verb|Foo| value into another \verb|Foo| value.
The function \verb|lenFoo|, defined with \verb|mcata|, is a length like function,
but it recurses on the transformed tail, \verb|(f xs)|,
instead of the original tail \verb|xs|.
The intuition behind the termination of \verb|mcata| for this negative datatype
\verb|Foo| is similar to the intuition for positive dataypes.  The embedded function
\verb|f::r->r| can only apply to the direct subcomponent of its parent, or to its
sibling, \verb|xs| and its transformed values (e.g. \verb|f xs|, \verb|f (f xs)|, $\ldots$),
but no larger values that contains \verb|f| itself.  We illustrate a general proof
on termination property of \verb|mcata| in Figure \ref{fig:proof}.

\begin{figure}
\begin{verbatim}
type Mu f = forall a . (forall r . (r -> a) -> f r -> a) -> a

mcata :: (forall r . (r -> a) -> f r -> a) -> Mu f -> a
mcata phi r = r phi

roll :: f (Mu f) -> Mu f
roll r phi = phi (mcata phi) r
\end{verbatim}
\caption{$F_{\omega}$ encoding of \texttt{Mu} and \texttt{mcata} in Haskell}
\label{fig:proof}
\vspace*{.5em}\hrule
\end{figure}

\subsection{Iteration over negative datatypes using syntactic inverses}
\label{ssec:msfcata}
Although we can define some simple functions such as \texttt{pconst} and
\texttt{lenFoo} with \texttt{mcata}, the functions we can define with
\texttt{mcata} are rather limited. In the functional programming community,
variations of catamorphism over datatypes with embedded functions, including
negative datatyps, has been studied to write more useful total functions.
Our contribution is that we have shown it is also possible to formulated
such a variation of catamorphism in Mendler style as well, and proved its
termination property. We will introduce our development of \texttt{msfcata},
namely the Mendler style Sheard-Fegaras catamorphism, using a case
study on formatting Higher-Order Abstract Syntax (HOAS).

\subsubsection{Formatting HOAS} \label{sec:bg:showHOAS}

To lead up to the Mendler style solution to formatting HOAS,
we first review some earlier work on turning expressions, expressed in 
Higher-Order Abstract Syntax (HOAS) \cite{Church40,PfeEll88}, into strings.
The most simple HOAS datatype definition (of the untyped lambda calculus)
in Haskell is:
\verb/data Exp = Lam (Exp -> Exp) | App Exp Exp/.
We want to format a term of \texttt{Exp} into a string.
For example, \verb|Lam(\x->(Lam(\y->App x y)))| can be
formatted into \verb|(\x->(\y->(x y)))|.
A solution to this problem was suggested by \citet{FegShe96}.
They were studying yet another abstract recursion scheme described by
Paterson \cite{Pat93} and Meijer and Hutton \cite{MeiHut95} that could only be
used if the combining function \texttt{phi} had a true inverse.
This seemed a bit limiting, so Fegaras and Sheard introduced the idea of
a syntactic inverse (or, a placeholder). The syntactic inverse was realized by
augmenting the \verb|Mu| type with a second constructor, which we call here
\verb|Rec|. This augmented datatype fixpoint \verb|Rec| has the same structure
as \verb|Mu|, but with an additional data constructor as follows:
\begin{verbatim}
data Rec f a = Roll' (f (Rec f)) | Inverse a
unRoll' (Roll' e) = e
\end{verbatim}
The algorithm worked, but, the augmentation introduces junk (\ie the values
constructied by \texttt{Inverse} is only an intermediate placeholder to
calculate the answer later on, but can never be a valid input value).
Washburn and Weirich\cite{bgb} eliminated the junk by exploiting parametricity.
It is a coincidence that Mendler style iteration/recursion schemes also use
the same technique, parametricity, for a different purpose, to guarantee
termination. Fortunately, these two approaches work together without getting in
each other's way.  

\begin{figure}
\begin{verbatim}
data Exp_g = Lam_g (Exp_g -> Exp_g) | App_g Exp_g Exp_g | Var_g String

showExp_g :: Exp_g -> String
showExp_g e = show' e vars where
  show' (App_g x y)  = \vs      -> "("++ show' x vs ++" "
                                      ++ show' y vs ++")"
  show' (Lam_g z)    = \(v:vs)  -> "(\\"++v++"->"
                                      ++ show' (z (Var_g v)) vs ++")"
  show' (Var_g v)    = \vs      -> v
\end{verbatim}
\begin{verbatim}
data ExpF r = Lam (r -> r) | App r r
type Exp' a = Rec ExpF a
type Exp = forall a . Exp' a
lam e    = Roll' (Lam e)
app f e  = Roll' (App f e)

showExp :: Exp -> String
showExp e = msfcata  phi e vars where
  phi :: (([String] -> String) -> r) -> (r -> ([String] -> String))
      -> ExpF r -> ([String] -> String)
  phi inv show' (App x y) = \vs     -> "("++ show' x vs ++" "
                                          ++ show' y vs ++")"
  phi inv show' (Lam z)   = \(v:vs) -> "(\\"++v++"->"
                                 ++ show' (z (inv (const v))) vs ++")"
\end{verbatim}
\begin{verbatim}
vars :: [String]
vars = [ [i] | i <- ['a'..'z'] ] ++ [ i:show j | j<-[1..], i<-['a'..'z'] ]
\end{verbatim}
\caption{\texttt{msfcata} example: String formatting function for Higher-Order Abstract Syntax (HOAS)}
\label{fig:HOASshow}
\end{figure}

\subsubsection{A general recursive implementation for open HOAS}
\label{sec:bg:showHOAS:recursive}

The recursive datatype \verb|Exp_g| in Figure \ref{fig:HOASshow}
is an open HOAS. By \emph{open}, we express that \verb|Exp_g| has
a data constructor \verb|Var_g|, which enables us to introduce free variables.
The constructor \verb|Lam_g| holds an embedded function of type
\verb|(Exp_g -> Exp_g)|.
This is called a shallow embedding, since we use functions in the host language,
Haskell, to represent lambda abstractions in the object language \verb|Exp_g|.
For example, using the Haskell lambda expressions,
we can construct some \verb|Exp_g| representing lambda expressions as follows:
\begin{verbatim}
k_g   = Lam_g (\x -> Lam_g (\y -> x))
s_g   = Lam_g (\x -> Lam_g (\y -> Lam_g (\z -> App_g x z `App_g` App_g y z)))
skk_g = App_g s_g k_g `App_g` k_g
w_g   = Lam_g (\x -> x `App_g` x)
\end{verbatim}
Since we can build any untyped lambda expressions with \verb|Exp_g|, 
even the problematic self application expression \verb|w_g|,
it is not possible to write a terminating evaluation function for \verb|Exp_g|.
However, there are many  functions that recurse over the structure of
\verb|Exp_g|, and when they terminate produce something useful.
One of them is the string formatting function \verb|showExp_g| defined in
Figure \ref{fig:HOASshow}.

Given an expression (\verb|Exp_g|) and a list of fresh variable names
(\verb|[String]|), the function \verb|show'| (defined in the \verb|where|
clause of \verb|showExp_g|) returns a string (\verb|String|) that represents
the given expression.  To format an application expression \verb|(App_g x y)|,
we simply recuse over each of the subexpressions \verb|x| and \verb|y|.
To format a lambda expression, we take a fresh name \verb|v| to represent
the binder and we recurse over \verb|(z (Var_g v))|, which is the application of
the embedded function \verb|(z :: Exp_g -> Exp_g)| to a variable expression
\verb|(Var_g v :: Exp_g)| constucted from the fresh name.
Note, we had to create a new variable expression to format the function body
since we cannot look inside the function values of Haskell.
To format a variable expression \verb|(Var_g v)|,
we only need to return its name \verb|v|.  The local function is \verb|show'|
(and hence also \verb|showExp_g|), is total as long as
the function values embedded in the \verb|Lam_g| constructors are total.

With \verb|showExp_g| we can format and print out the terms
\verb|k_g|, \verb|s_g|, \verb|skk_g| and \verb|w_g| as follows:
\begin{quote}\noindent
$>$ \verb|putStrLn (showExp_g k_g)|\\
\verb|(\a->(\b->a))|
\end{quote}\vspace*{-1em}
\begin{quote}\noindent
$>$ \verb|putStrLn (showExp_g s_g)|\\
\verb|(\a->(\b->(\c->((a c) (b c)))))|
\end{quote}\vspace*{-1em}
\begin{quote}\noindent
$>$ \verb|putStrLn (showExp_g skk_g)|\\
\verb|(((\a->(\b->(\c->((a c) (b c))))) (\a->(\b->a)))|\\
\verb|(\a->(\b->a)))|
\end{quote}\vspace*{-1em}
\begin{quote}\noindent
$>$ \verb|putStrLn (showExp_g w_g)|\\
\verb|(\a->(a a))|
\end{quote}\vspace*{-.5em}

Note that \verb|show'| is not structurally inductive in the \verb|Lam_g| case.
The recursive argument (\verb|z (Var_g v)|), in particular \verb|Var_g v|,
is not a subexpression of (\verb|Lam_g z|).  Thus the recursive call to
\verb|show'| may not terminate. This function terminated only because
the embedded function \verb|z| was well behaved, and the argument we passed
to \verb|z|, (\verb|Var_g v|), was well behaved. If we had applied \verb|z|
to the expression (\verb|Lam_g (\x->x)|) in place of \verb|Var_g v|,
or \verb|z| itself had been divergent, the recursive call would have diverged.
If \verb|z| is divergent, then obviously \verb|show' (z x)| diverges for
all \verb|x|. More interestingly, suppose \verb|z| is not divergent
(perhaps something as simple as the identity function) and \verb|show'|
was written to recurse on (\verb|Lam_g (\x->x)|), then what happens?
\begin{verbatim}
show' (Lam_g z) (v:  vs) = "(\\"++v++"->"
                              ++ show' (z (Lam_g (\x->x)) vs ++")"
\end{verbatim}
The function is no longer total.  To format (\verb|z (Lam_g (\x->x))|)
in the recursive call, it loops back to the \verb|Lam_g| case again,
unless \verb|z| is a function that ignores its argument.
This will form an infinite recursion, since this altered \verb|show'| forms
yet another new \verb|Lam_g (\x->x)| expression and keeps on recursing.


\subsubsection{A Mendler style solution for closed HOAS}
\label{sec:bg:showHOAS:msfcata}

Our exploration of the code in Figure \ref{fig:HOASshow} illustrates
three potential problems with the general recursive approach.
\begin{itemize}
\item The embedded functions may not terminate.
\item In a recursive call, the arguments to an embedded function
may introduce a constructor with another embedded function, leading to
a non terminating cycle.
\item We got lucky, in that the answer we required was a \verb|String|, and
we happened to have a constructor \verb|Var_g :: String -> Exp_g|.
In general we may not be so lucky.
\end{itemize}

In Figure \ref{fig:HOASshow}, we defined \verb|Exp_g| in anticipation of
our need to write a function \verb|showExp_g| \verb|::| \verb|Exp_g -> String|,
by including a constructor \verb|Var_g :: String -> Exp_g|.
Had we anticipated another function \verb|f:: Exp_g -> Int|
we would have needed another  constructor \verb|C :: Int -> Exp_g|.
Clearly we need a better solution.  The solution is to generalize the kind of
the datatype from \verb|Exp_g :: *| to \verb|Exp :: * -> *|, and add
a universal inverse.
\begin{verbatim}
data Exp a   =  App (Exp a) (Exp a)
             |  Lam (Exp a -> Exp a)
             |  Inv a

countLam:: Exp Int -> Int   
countLam (Inv n) = n
countLam (App x y) = countLam x + countLam y
countLam (Lam f) = countLam(f (Inv 1))
\end{verbatim}
Generalizing from \verb|countLam| we can define a function from \verb|Exp|
to any type. How do we lift this kind of solution to the Mendler style?
Fegaras and Sheard\cite{FegShe96} proposed moving the general inverse from
the base type to the datatype fixpoint. Later this approach was refined by
Washburn and Weirich\cite{bgb} to remove the junk introduced by
that augmentation (i.e. things like \verb|App (Inv 1) (Inv 1)|).

We use the same inverse augmented datatype fixpoint appearing in
Washburn and Weirich\cite{bgb}.  Here, we call it \verb|Rec|.
The inverse augmented datatype fixpoint \verb|Rec| is similar to
the standard datatype fixpoint \verb|Mu|.
The difference is that \verb|Rec| has an additional type index \verb|a|
and an additional data constructor \verb|Inverse :: a -> Rec a i|,
corresponding to the universal inverse.
The data constructor \verb|Roll'| and the projection function \verb|unRoll'|
correspond to \verb|Roll| and \verb|unRoll| of the normal fixpoint \verb|Mu|.
As usual we restrict the use of \verb|unRoll'|, or pattern matching against
\verb|Roll'|.

We illustrate this in the second part of Figure \ref{fig:HOASshow}.
As usual, we define \verb|Exp' a| as a fixpoint of the base datatype \verb|ExpF|
and define shorthand constructors \verb|lam| and \verb|app|.
Using the shorthand constructor functions,
we can define some lambda expressions: %% as follows:
\begin{verbatim}
k    = lam  (\x -> lam (\y -> x))
s    = lam  (\x -> lam (\y -> lam (\z -> app x z `app` app y z)))
skk  = app s k `app` k
w    = lam  (\x -> x `app` x)
\end{verbatim}
However, there is another way to construct \verb|Exp'| values that is
problematic. Using the constructor \verb|Inverse|, we can turn values of
arbitrary type \verb|t| into values of \verb|Exp' t|.  For example, 
\verb|Inverse True :: Exp' Bool|. This value is junk, since it does 
not correspond to any lambda term. By design, we wish to hide \verb|Inverse|
behind an abstraction boundary. We should never allow the user to construct
expressions such as \verb|Inverse True|, except for using them as placeholders
for intermediate results during computation.


We can distinguish pure expressions that are inverse-free
from expressions that contain inverse values by exploiting parametricity.
The expressions that do not contain inverses have a fully polymorphic type.
For instance, \verb|k|, \verb|s|, \verb|skk| and \verb|w| are of type (\verb|Exp' a|).
The expressions that contain \verb|Inverse| have more specific type
(e.g., \verb|(Inverse True) :: (Exp' Bool)|).
Therefore, we define the type of \verb|Exp| to be \verb|forall a . Exp' a|.
Then, expressions of type \verb|Exp| are guaranteed to be be inverse-free.
Using parametricity to sort out junk introduced by the inverse is the key idea
of Washburn and Weirich\cite{bgb}, and the inverse augmented fixpoint
\verb|Rec| is the key idea of Fegaras and Sheard\cite{FegShe96}.
The contribution we make in this work is putting together these ideas
in Mender-style setting.  By doing so, we are able define recursion combinators
over types with negative occurrences, which have well understood
termination properties enforced by parametricity.
%%  We define 4 such combinators:
%% \verb|msfcata|, \verb|msfhist0|, \verb|msfcata1|, and \verb|msfhist1|. 
%% The combinator \verb|msfcata| is the simplest, to define it
%% we generalize over \verb|mcata| by using the same device we used earlier,
%% we abstract the combining function over an additional argument,
%% this time an abstract inverse.

\begin{itemize}
  \item The combining function \verb|phi| becomes a function of 3 arguments.
        An abstract inverse, an 
        abstract recursive caller, and a base structure.
\begin{verbatim}
  msfcata phi (Roll' x)    = phi Inverse (msfcata phi)  x
  msfcata phi (Inverse z) = z
\end{verbatim}
  \item For inverse values, return the value inside \verb|Inverse| as it is.

  \item We use higher-rank polymorphism to insist that 
        the abstract inverse function, with type (\verb|a -> r a|),
        the abstract recursive caller function, with type (\verb|r a -> a|), and
        the base structure, with type (\verb|f (r a)|), only work
        over an abstract type constructor, denoted by (\verb|r|).
\begin{verbatim}
msfcata :: (forall r. (a -> r a) ->
                       (r a -> a) ->
                       f (r a)    -> a) -> (forall a. Rec f a) -> a
\end{verbatim}
  \item Note, the abstract recursive type \verb|r| is parameterized by
        the answer type \verb|a| because the augmented datatype fixpoint \verb|Rec|
        is parameterized by the answer type \verb|a|.

        Also, note, the second argument of \verb|msfcata|, the object being
        operated on, has the higher-rank polymorphic type
        \verb|(forall a . Rec f a)|, insisting the input value to be inverse-free
        by enforcing \verb|a| to be abstract.
\end{itemize}

In Figure \ref{fig:HOASshow}, using \verb|msfcata|, it is easy to define \verb|showExp|,
the string formatting function for \verb|Exp|, as in Figure \ref{fig:HOASshow}.
The \verb|App| case is similar to the general recursive implementation.
The body of \verb|phi| is almost textually identical to the body of \verb|show'|
in the general recursive solution, except we use the inverse expression
\verb|inv (const v)| to create an abstract \verb|r| value to pass to
the embedded function \verb|z|.  Note, \verb|const v| plays exactly
the same roll as \verb|(Var_g v)| in \verb|show'|.

Does \verb|msfcata| really guarantee termination?  To prove this we need to
address the first two of the three potential problems described at
the beginning of \S\ref{sec:bg:showHOAS:msfcata}.  The first problem
(embedded functions may be partial) is addressed using logicality analysis.
The second problem (cyclic use of constructors as arguments to
embedded functions) is addressed by the same argument we used
in \S\ref{sssec:mcataNegative}.  The abstract type of the inverse
doesn't allow it to be applied to constructors, they're not abstract enough. 
Just as we couldn't define \verb|p_m| (in \S\ref{sssec:mcataNegative})
we can't apply \verb|z| to things like {\small (\verb|Lam(\x->x)|)}.

We provide an embedding of \verb|msfcata|
into the strongly normalizing language $F_\omega$.
This constitutes a proof that \verb|msfcata| terminates for all
inductive datatypes, even those with negative occurrences.

Figure \ref{fig:proofsf} is the $F_\omega$ encoding of the inverse augmented datatype
\verb|Rec| and its catamorphism \verb|msfcata|.  We use the sum type to encode \verb|Rec|
since it consi]sts of two constructors, one for the inverse and the other for
the recursion.  The newtype \verb|Id| wraps answer values inside the inverse.
The catamorphism combinator \verb|msfcata| unwraps
the result (\verb|unId|) when \verb|x| is an inverse.
Otherwise, \texttt{msfcata} runs the combining function \verb|phi| over
the recursive structure \verb|(\f->f(phi Id))|.
The utility function \verb|lift| abstracts a common pattern, useful
when we define the shorthand constructors (\verb|lam| and \verb|app|).

Figure \ref{fig:sumdef} is the $F_\omega$ encoding of the sum type \verb|(:+:)|
and its constructors (or injection functions) \verb|inL| and \verb|inR|.
The case expression \verb|caseSum| for the sum type is just binary function
application. In the $F_\omega$ encoding, they could be omitted
(i.e., \verb|caseSum x f g| simplifies to \verb|x f g|).
But, we choose to write in terms of \verb|caseSum| to make
the definitions easier to read.

In Figure \ref{fig:HOASshowFw}, we define both an inductive datatype for HOAS (\verb|Exp|), and the string formatting function
(\verb|showExp|),
with these $F_\omega$ encodings
We can define simple expressions using the shorthand constructors and print out
those expressions using \verb|showExp|.  For example,
\begin{quote}\noindent
$>$ \verb|putStrLn (showExp (lam(\x->lam(\y->x))))|\\
\verb|(\a->(\b->a))|
\end{quote}

\begin{figure}
\begin{verbatim}
type Rec f r a = (r a) :+: (((r a -> a) -> f (r a) -> a) -> a)

newtype Id x = Id { unId :: x }

msfcata  ::  (forall r . (a -> r a) -> (r a -> a) -> f (r a) -> a)
         ->  (forall a . Rec f Id a) -> a
msfcata phi x = caseSum x unId (\ f -> f (phi Id))

lift :: ((Id a -> a) -> f (Id a) -> a) -> Rec f Id a -> Id a
lift h x = caseSum x id (\ x -> Id(x h))
\end{verbatim}
\caption{$F_\omega$ encoding of \texttt{Rec} and \texttt{msfcata}}
\label{fig:proofsf}
\end{figure}

\begin{figure}
\begin{verbatim}
type a :+: b =  forall c . (a ->  c) ->  (b -> c) ->  c
inL :: a -> (a:+:b)
inL a = \ f g -> f a
inR :: b -> (a:+:b)
inR b = \ f g -> g b
caseSum :: (a:+:b) -> (a -> c) -> (b -> c) -> c
caseSum x f g = x f g
\end{verbatim}
\caption{$F_\omega$ encoding of the sum type}
\label{fig:sumdef}
\end{figure}

\begin{figure}
\begin{verbatim}
data ExpF x = App x x | Lam (x -> x)
type Exp' a = Rec ExpF Id a
type Exp = forall a . Exp' a
app :: Exp' a -> Exp' a -> Exp' a
app x y = inR (\h -> h unId (App (lift h x) (lift h y)))
lam :: (Exp' a -> Exp' a) -> Exp' a
lam f = inR (\h -> h unId (Lam (\x -> lift h(f(inL x))) ))

showExp:: Exp -> String
showExp e = msfcata phi e vars where
  phi inv show' (App x y)  = \vs      ->
                "("++ show' x vs ++" "++ show' y vs ++")"
  phi inv show' (Lam z)    = \(v:vs)  ->
                "(\\"++v++"->"++ show'(z (inv (const v))) vs ++")"
\end{verbatim}

\caption{HOAS string formatting example in $F_\omega$.}
\label{fig:HOASshowFw}
\end{figure}

\subsection{Related Work} \label{ssec:MendlerRW}

%% \citet{Mat11} formalized a version of Mendler style calculi in Coq
%% to prove properties on map fusion for nested datatypes.
%% On the conventional side, \citet{Hin10} tries to unify various morphisms 
%% using the concept he calls \emph{adjoint folds}. But, it remains to be seen
%% whether more exotic species of combinators, such as histomorphisms,
%% can be subsumed by this framework.

Since formal proof systems based on inductive type paradigm such as Coq
do not support negative datatypes, the application of HOAS in such systems
(e.g. \cite{Chl08}) are often based on Weak HOAS \cite{Des95,HonMicSca01}
to avoid the use of negative datatypes. However there have been some work
on iteration, recursion, and induction over HOAS.

Our preliminary work is about iteration over HOAS, and more generally over
negative datatypes. \citet{DesPfeSch97} introduced a primitive recursion
on HOAS in a modal $\lambda$-calculus. Since their work can analyze inside
functions (or, inside the $\lambda$ binder), their system can express
total functions that analyze the subcomponents of the application
(\eg parallel reduction). Such functions are not expressible in
our preliminary work \cite{AhnShe11} and in the work by \citet{FegShe96} and
\citet{MeiHut95}, which our preliminary work is based on.
However, their study of primitive recursion over HOAS is in the context of
simple types, but not parameterized datatypes nor indexed datatypes.
The Mendler style catamorphism work well with parameterized and
indexed datatypes. Later, \citet{DesLel99} tried to extended
primitive recursion over HOAS in the presence of dependent types.
\citet{DesLel99} were able to prove type safety of their system,
but have not proved normalization yet.

Induction principles over HOAS seem to been studied in various contexts,
but I need to further background literature search to list and categorize
the related work on induction over HOAS.

\subsection{Future Work} \label{ssec:MendlerFW}
I am thinking of two follow-up work on our preliminary work.
First is extending the Mendler style catamorphism to dependent types
(\S\ref{sec:plan:depty}).
Second is searching for a Mendler style recursion combinator that
guarantee termination for negative datatypes (\S\ref{sec:plan:recneg}).

\subsubsection{Extending the Mendler style catamorphism to dependent types}
\label{sec:plan:depty}
Consider the following dependently typed program
which shows that every natural number is either even or odd:
\begin{verbatim}
data Nat where              -- inductive definition of natrual numbers
  Zero : Nat
  Succ : Nat -> Nat

data Either (a:Type) (b:Type) where   -- the sum type
  Left  : a -> Either a b
  Right : b -> Either a b

data Even (n:Nat) where               -- inductive definition of
  EvenO : Even Zero                   -- the evenness property,
  EvenS : Odd n -> Even (Succ n)      -- mutually recursive with Odd

data Odd (n:Nat) where                -- inductive definition of
  OddS : Even n -> Odd (Succ n)       -- the oddness property

evenOrOdd : (n:Nat) -> Either (Even n) (Odd n)
evenOrOdd Zero     = Left CZero
evenOrOdd (Succ n) = case evenOrOdd n of
                       Left p  -> Right (OddS p)
                       Right p -> Left (EvenS p)
\end{verbatim}
The function \verb|evenOrOdd| takes a natural number \verb|n| and
returns either a proof that \verb|n| is even or a proof that \verb|n| is odd.
Except for the dependency using the value \verb|n| in the return type,
the recursion pattern has the form of a catamorphism.
It is an open question whether the Mendler style catamorphism naturally extends
to dependent types as it does to indexed types.
Assuming that we were able write a dependent version of the Mendler style
catamorphism, say \verb|mcataD|, then we would be able to write
\verb|evenOrOdd| in terms of \verb|mcataD| as follows:
\begin{verbatim}
data N r = Z | S r
type Nat = Mu N

evenOrOdd = mcataD phi where
  phi eoo Z     = Left CZero
  phi eoo (S n) = case eoo n of
                    Left p  -> Right (OddS p)
                    Right p -> Left (EvenS p)
\end{verbatim}
Recall that, in Mendler style, we encode a datatypes (\eg \verb|Nat|)
as a fixpoint (\eg \verb|Mu|) of base functor (\eg \verb|N|).

The main problem here is that the Mendler style recursion combinators
use parametricity to abstract the type of the argument value, but
the return type of the function depends on the argument value.
In the second equation of the \verb|phi| function above,
\verb|(S n) :: N r|, and therefore \verb|n :: r|, where \verb|r|
is abstract. But, what should be the type of the abstract recursive caller
\verb|eoo|? It would look something like \texttt{
eoo :: (n:r) -> Either (Even n) (Odd n)}.
We can already see that this does not type check since
\verb|Even :: Nat -> Type| and \verb|Odd :: Nat -> Type|
but \verb|n :: r|.  Recall, we cannot unify \verb|r| with
any specific type.  Thus \verb|(Even n)| and \verb|(Odd n)|
are ill-typed (or ill-kinded).
We see that it is hard to use a value without unveiling the details of its type.
This problem is analogous to the problem of using abstract types for
parameterized modules. We want to encode the type of modules to be
abstract types, but we also want to know certain instances of
the parameterized modules have share same parameter since we want
a reasonable separate compilation scheme.
Translucent sums \cite{LillibridgeThesis}
were suggested to solve this problem when type checking modules.
Here, we also need a translucency in the sense that we want the type of
the arguemnt to be abstract when implementing the runtime behavior
in the function definition to limit the dangerous recursion, but
we want to know the type of the argument in the type signature of
the function due to the true value dependency on the argument.

My proposed attempt here tries to implement translucency using
the features of the Trellys language. We do not know yet whether
this is possible, or we need to add new feature to support
translucency.  My preliminary thoughts on the type signature
for the dependent Mendler style catamorphism is the following:
\begin{verbatim}
mcataD : (forall (r:Type) . [tr: r->Mu f] -> [tfr: f r->Mu f]
                         -> [pr: tr = id ] -> [pfr: tfr = Roll ]
                         -> ((z:r) -> a (tr z)) -> (y:f r) -> a (tfr y))
        -> (x:Mu f) -> a x
\end{verbatim}
This dependent version of the Mendler style catamorphism combinator
has four additional arguments for the \verb|phi| function
(\verb|tr|, \verb|tfr|, \verb|pr|, and \verb|pfr|),
when compared to \verb|mcata|.
Note, \verb|pr| has an equality proof involving \verb|tr| and
\verb|prf| has an equality proof involving \verb|tfr|.

Here, we use an interesting feature found in the Trellys language:
we require those four arguments to be \emph{erasable arguments}
by annotating them with square brackets. Erasable arguments can only be
used for type checking purposes, but have no effect on the runtime behavior
of the function.  The first two arguments,
\verb|tr| and \verb|tfr|, are type casting functions that turn the abstract
types \verb|r| and \verb|f r| into a concrete inductive datatype \verb|Mu f|.
Since these type casting functions (\verb|tr| and \verb|tfr|) and
their equality property proofs (\verb|pr| and \verb|pfr|) break parametricity
of \verb|r|, we should limit their use in the runtime definition of
the \verb|phi| function, but only allow their use in the type signatures
and type casting purposes.

Another interesting feature of Trellys we require in this proposed approach
is the \emph{heterogeneous equality} in the equality types of \verb|pr| and
\verb|pfr|.
Note, the left- and right-hand sides of \verb|tr = id| and \verb|tfr = Roll|
have different types (e.g., \texttt{tfr:f r->Mu f} and
\texttt{Roll:f (Mu f)->Mu f}). 
These heterogeneous equality makes it possible to type check the
\verb|evenOrOdd| example.  Consider the case branch
\verb|Left p -> Right (OddS p)|.  Since \verb|eoo n : a (tr n)|,
\texttt{Left p : a (tr n)} and \verb|Right (OddS p) : a (Roll (S (tr n)))|.
Since the final return type of \verb|phi| must be must be \verb|a (trf (S n)))|,
we should show that \texttt{a (Roll (S (tr n))) = a (trf (S n)))}.
Since \verb|tr = id|, the left-hand side is equivalent to \verb|a (Roll (S n))|.
Since \verb|trf = Roll|, the right-hand side is also equivalent to
\verb|a (Roll (S n))|.

Note, the tentative approach discussed above is only a preliminary thought
(which may or may not work) and we might end up with a better approach.
I started with this tentative approach to understand the problem better,
but not expecting this approach leads to complete success. The advantage
of using existing language features, in contrast to inventing new features
or language constructs, is that we do not need to worry about breaking the
soundness and consistency of the type system, provided that the properties
of the language features we rely on are well-studied.

\subsubsection{Mendler style recursion combinator for negative datatypes}
\label{sec:plan:recneg}
In \S\ref{ssec:MendlerRW}, I mentioned related work on primitive recursion
for HOAS using modal types by \citet{DesPfeSch97,DesLel99}. Their work was
not in Mendler style. I suspect that it may be possible to refine
the Mendler style histomorhpism, which is a Mendler style recursion combinator
capable of encoding course of values, to grantee termination for
negative datatypes. To discover such a recursion combinator, I will try to
encode the ideas of \citet{DesPfeSch97} in a Mendler style setting.

The Mendler style histormorphism is proven to gurantee termination for regular
inductive datatypes \cite{vene00phd}.
For non-regular inductive datatypes (\eg \texttt{Powl} in \S\ref{sssec:cata}),
the termination property of the Mendler style histomorhpism is left out as
a conjecture strongly believed to be true \cite{Mat01}.\footnote{
\citet{Mat01} left the proof on course of values recursion as an open question
in his work on monotonicity.
It may have already been proven, but I haven't yet encountered one yet.}
For negative datatypes, I recently found a counterexample that
the Mendler style histomorhpism is not normalizing for negative datatypes.
You look up the counterexample in our conference paper \cite{AhnShe11}.

I have omitted the discussion on histomorphism in this document
in order to avoid too much digression into details on course of values
recursion combinators, and, instead, focus on iteration on negative datatypes
such as HOAS. So, I am closing the discussion of future work on this subject.
You can find further details on the Mendler style histomorhpism
in our conference paper \cite{AhnShe11} and Vene's thesis \cite{vene00phd}.



\section{Plans for dissertation work}\label{sec:plan}

My dissertation will contain four parts.
The first two parts will review and summarize the literature
organized from the viewpoint of my thesis, and the later two parts will
contain mostly original contributions.

Part I is the introduction and background. This part will consist of chapters
extending \S\ref{sec:intro} (introduction) and \S\ref{sec:bg} (background)
of this document. That is, first, further details on what is already in this document,
second, additional background material from an extended literature search, and third, introduction to the issues to be
discussed in later parts.

Part II discusses current ideas from the literature on normalization.
Strong  normalization is an important component of logical consistency, so
understanding normalization is a necessary component of the thesis.
The current literature deals mostly with positive datatypes,
thus this part will consist of chapters discussing normalization of
both strictly positive datatypes and non-strictly positive datatypes. 

Part III is on normalization of negative datatypes. This part will consist of
chapters extending \S\ref{sec:prelim} of this document, where we discussed our
preliminary results using Mendler style iteration. Further details will be added.
In addition, this part will have chapters addressing some of the open questions and
issues related to normalization of negative datatypes
(\eg interaction with dependent types, course of values recursion).

Part IV explores the design of languages that can shift gears between
different fragments of the problem spaces (\IND, \INDbot, \REC, \RECbot).
I will extend lambda calculi with recursive types, and then extend them with
facilities that track which fragment the terms belong to, and what conditions
can make the terms shiftable from one fragment to another.

\subsection{Outline and Timeline}

\paragraph{Part 1 (introduction and background):}
This part will be an extended literature search, adding to the material from
the introduction (\S\ref{sec:intro}) and the background (\S\ref{sec:bg})
sections of this document. A tentative list of additional subjects I am
planning to survey include:
\begin{itemize}
\item Examples of formal reasoning system designs that relate
      \IND\ and \INDbot (\eg the bar type \cite{ConSmi93} in Nuprl).
\item More generally, how \emph{extensional type theory}
      (in contrast to \emph{intensional type theory}), can be more flexible
      when tracking non-terminating computations using the type system.
      Extensional type theory does not distinguish computational equality
      from propositional equality. Type checking become undecidable when
      non-terminating computation is allowed while type checking.
\item \emph{Observational type theory}, which claims to take advantage of
      the benefits from both intensional and extensional type theories.
\item Libraries that track partiality as effects or monads
      in formal reasoning systems (\eg Agda, Epigram), which lie in \IND
%%% TODO refer to work like http://www.cs.nott.ac.uk/~txa/talks/bctcs06.pdf ???
\item Various termination analysis methods, and whether those
      methods are known to reduce to more primitive ways of
      ensuring termination (\eg primitive recursion or structural recursion)
\item More details and examples on the use of monotonicity, especially for
      non-strictly positive datatypes
      (\eg Matthes' thesis mentions that there exists monotone but not positive
      datatypes due to Ulrich Berger).
%%      \ts{Brian Huffmans thesis discusses monotonicty}
\end{itemize}

I plan to finish the writing of Part I of the thesis in November 2011,
this fall.

%% Some strange functions like\\
%% \url{http://en.wikipedia.org/wiki/McCarthy_91_function}

%%% maybe a good resource
%%% http://www.scholarpedia.org/article/Computational_type_theory

\paragraph{Part II (positive datatypes)}
Although this part is also a background literature search, I decided to
separate this material from Part I and add more detail on positive datatypes
for several reasons.

Firstly, strong normalization is necessary for logical consistency.
So understanding current approaches is necessary for part IV.

Secondly, a detailed understanding of how normalization is ensured for
positive datatypes, supports comparing and contrasting with how normalization
is ensured for negative datatypes later in Part III.

Thirdly, it is one of my theses that normalization and inductiveness
(or, logical interpretation of types) are indeed separate concerns
of logical consistency. By reviewing the current literature
on positive dataypes (where the two ideas are conflated), we can 
begin to unravel the knot that currently binds them together.

Computationally, all positive datatypes behave the same
regardless of whether they are strictly positive or not.
Primitive recursion supports normalization of both strictly positive datatypes
and non-strictly positive datatypes (more generally, monotone datatypes).
However, logically, not all positive datatypes can be accepted as propositions,
depending on what is considered an acceptable logic. In the inductive paradigm,
where types must have set theoretic interpretations, strictly positive datatypes
are valid types, but not all positive datatypes are considered to be valid types
in general. For instance, when we interpret types as sets and $\to$ as
function space over sets, the positive, but not strictly positive, datatype
$\mu \alpha.(\alpha\to\textsf{Bool})\to\textsf{Bool}$ asserts an isomorphism
between the powerset of powerset of $\alpha$ and $\alpha$ itself, which is
a set theoretic nonsense.

Although monotonicity is a more general principle than the syntactic condition
of positivity, positivity is nice approximation since monotonicity witness for
positive datatypes can be simply derived automatically. For non-positive
monotone datatypes, I have not seen work on automatic derivation of their
monotonicity witnesses. So, I think a practical system would use the syntactic
conditions like positivity by default, and to be more flexible, it can have
a mechanism to check and accept monotonicity witness manually provided by
the user.

I plan to finish the first pass of writing Part II in December 2011,
this winter.

\paragraph{Part III (negative datatypes)}
This part will report the preliminary results I have already finished.
Additional writing will be necessary.
 
I will add more detailed descriptions to the material reported in this document
(\S\ref{sec:prelim}) on Mendler style iteration for negative datatypes,
and a more detailed summary of the literature on primitive recursion and
induction principles proposed for HOAS. I plan to write these chapters
in January 2012. Considerable text in this area exists from published papers,
which needs to be recast in the style of a thesis.

As my research progresses, I will address some of the open questions and
issues related to normalization of negative datatypes
(\eg interaction with dependent types, course of values recursion).
I will write additional chapters on those results.
Since this is speculative work to be done, I cannot predict a confident timeline
for this, but my initial plan is to finish writing this in March 2012.

\paragraph{Part IV (language design)}
This part is on the high level design of logical languages
(or, formal reasoning systems) that can shift gears between
the four fragments of \IND, \INDbot, \REC, and \RECbot.
Although the ideal design would cover all of the four fragments,
I will only focus on four pairs of fragments to avoid accidental complexity.
That is, I will illustrate four typed lambda calculi that cover
\IND-\INDbot, \IND-\REC, \REC-\RECbot, and \INDbot-\RECbot.

The highlight of the design of the type system for these 4 different calculi
is going to be how to represent and track the side conditions that determines
when a term in a larger fragment can be seen as a term in a smaller fragment
(\ie a term in \INDbot\ as a term in \IND, a term in \REC\ as a term in \IND,
a term in \RECbot\ as a term in \REC, a term in \RECbot\ as a term in \INDbot).

In addition to developing these calculi, there will be a chapter of case studies
that demonstrates the usefulness of the developed calculi. In the case study
chapter, I will formulate the examples that work over more than one fragment
(\eg Normalization by Evaluation) inside one of these calculi.

Since this part is again speculative work to be done, I cannot make a confident
timeline for this, but my initial plan is to complete the writing for Part III
in May 2012.

\subsection{Publication goals and methods for research}
Since Part I and II are survey on existing work by literature literature search,
results for publication will come out from working on Part III and IV.

\paragraph{}
I am planning to work on a journal version of our preliminary work.
I have two improvements to make in mind from our conference version.
First, I will have better context discussions and clear use of vocabularies
in the journal version. I was less clear on distinguishing iteration from
recursion at the point when we were writing the conference version.
Secondly, I will discuss further on related work. Due to the space limitations,
the conference version lacks the detailed discussion of the related work
in comparison to our results.

\paragraph{}
As mentioned in \S\ref{ssec:MendlerFW}, I am searching for more general and
more expressive Mendler style iteration and recursion for negative datatypes.
The result this work will be another material for publication.
In particluar, there are two generalization in consideration.

First is to generalize Mendler style iteration to dependently typed setting.
Menlder style iteration rely on parametriclty, but dependent types makes it
hard to rely on parametricity. The method I am currently trying is to use
erasable arguments and heterogeneous equality, in order to make
the dependent type indices be observable only at the type level,
but remain abstract at the value level (see \S\ref{ssec:MendlerFW}).
I will try applyng this method on examples other than even/odd datatype
to see if this method makes sense for other examples too. After trying out
several examples and become more confident that this may be a general enough
method, I will formally describe a calcui with dependent types, erasable
argument, and heterogeneous equality, and start proving the termination
behavior of the Mendler style iteration in that calculus,

Second is to formulate Mendler style recursion, rather than just iteration,
which guarantee totality. As we have seen in \S\ref{ssec:MendlerRW}, primitive
recursion for HOAS has been studied using modal types. However, those work are
for simple types and not in Mendler style. In our preliminary work, we have
been successful in formulating Fegaras-Sheard catamorphism in Mendler style,
which has only been studied in conventional style previously. Similarly,
I will try to formulate primitive recursion for HOAS in Mendler style.

\paragraph{}
Lastly, the calculi to be developed in Part IV is yet another topic for
publication. I will start from four calculi that is known to be type safe
in each of the four fragment (\IND, \INDbot, \REC, \RECbot), where the two
calculi for \IND\ and \REC\ should be normalizing calculi. Among the four
calculi, the calculus for \RECbot\ is the most flexible calculi allowing
all the recursive types, which will be the basis for the other three calculi.
The other three calculi would have additional restrictions that restrict certain
formation and use of recursive types and recursive control structures. Then, I
will try to design a type system that bridge between the four pairs we focus on,
which preserves the original property of the individual calculi and yet
possible to shift gears between the two calculi of the different fragments.
I would also need to show that the logic described by the calculus for \IND\,
or any pair involving \IND\, is consistent. To prove the consistency of our
developed calculus, I will study the literature on proving logical consistency
and try to apply the proof strategies to the calculi to be developed.



\section*{Acknowledgements}
Bob Harper's and Hugo Herbelin's OPLSS 2011 lectures\footnote{
\url{http://www.cs.uoregon.edu/Activities/summerschool/summer11/curriculum.html}
} and
Alexandre Miquel's Types summer school 2005 lecture notes\footnote{
\url{http://www.cse.chalmers.se/research/group/logic/Types/tutorials.html}
} has been a great guide in writing background section.

\bibliographystyle{plainnat}
\bibliography{main}

\end{document}
