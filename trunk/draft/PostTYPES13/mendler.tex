\section{Mendler-style recursion schemes}
\label{sec:mendler}
In this section, we introduce basic concepts of two Mendler-style recursion
schemes: iteration (\lstinline{mcata}\,) and iteration with syntactic inverses
(\lstinline{msfcata}\,). Further details on Mendler-style recursion schemes,
including these two and more,
can be found in \cite{AhnShe11,AbeMatUus05,UusVen00,AbeMat04}.

\begin{figure}
\begin{lstlisting}[
	caption={Mendler-style iteration (\MIt{}) and
		Mendler-style iteration with syntactic inverses (\msfit{})
		at kind $*$ and $*\to *$ transcribed in Haskell
		\label{lst:reccomb}}]
data Mu0   (f::(* -> *))               = In0   (f (Mu0   f)  )
data Mu1 (f::(* -> *) -> (* -> *)) i = In1 (f (Mu1 f) i)

type Phi0   f a = forall r . (    r   -> a  )  -- %\textcomment{abstract recursive call%}
                    -> (f r   -> a  )
type Phi1 f a = forall r i.(forall i.r i -> a i)  -- %\textcomment{abstract recursive call%}
                    -> (f r i -> a i)

mcata0   :: Phi0   f a ->  Mu0  f    -> a
mcata1 :: Phi1 f a ->  Mu1 f i -> a i

mcata0   phi (In0   x) = phi (mcata0   phi) x
mcata1 phi (In1 x) = phi (mcata1 phi) x


data Rec0   f a   = Roll0   (f (Rec0 f a)    ) | Inverse0   a
data Rec1 f a i = Roll1 (f (Rec1 f a) i) | Inverse1 (a i)

type Phi0'   f a = forall r . (     a  -> r a  )  -- %\textcomment{abstract recursive call%}
                    -> (     r a  -> a  )  -- %\textcomment{abstract inverse%}
                    -> (f (r a)   -> a  )
type Phi1' f a = forall r i.(forall i.a i -> r a i)  -- %\textcomment{abstract recursive call%}
                    -> (forall i.r a i -> a i)  -- %\textcomment{abstract inverse%}
                    -> (f (r a) i -> a i)

msfcata0   :: Phi0'   f a -> (forall a. Rec0   f a  ) -> a
msfcata1 :: Phi1' f a -> (forall a. Rec1 f a i) -> a i

msfcata0  phi r = msfcata phi r where
  msfcata phi (Roll0   x)      = phi Inverse0  (msfcata phi)  x
  msfcata phi (Inverse0 z)    = z

msfcata1  phi r = msfcata phi r  where
  msfcata  :: Phi1' -> Rec1 f a i -> a i
  msfcata phi (Roll1 x)     = phi Inverse1 (msfcata phi)  x
  msfcata phi (Inverse1 z) = z
\end{lstlisting}
\vspace*{-3ex}
\end{figure}

Each Mendler-style recursion scheme is described by a pair:
a fixpoint and its constructors, and, the recursion scheme itself.
In Listing~\ref{lst:reccomb}, we illustrate the two recursion schemes,
\lstinline{mcata} and \lstinline{msfcata}, using Haskell. We follow
a certain style of Haskell, where we restrict the use of certain langauge
features and some of the definitions we introduce. We will explain such
stylistic restrictions as we discuss the details of Listing~\ref{lst:reccomb}.

A Mendler-style recursion scheme is characterized by
the set of abstract operations it supports. The types of
these abstract operations are evident in the type sigature
of the recursion scheme. In Listing~\ref{lst:reccomb},
we made this more evident by factoring out the first agument types
as the type synonyms prefixed by \lstinline{Phi}.
Mendler-style recursion schemes take two arguments:
the first is a function\footnote{By convention,
	we denote the function as $\varphi$. Guess why
	the type synonyms are prefixed by \lstinline{Phi}.}
that describes the computation, and,
the second is a recursive value to compute over.
One describes the computation by defining the function, which is to be passed
as the first argument, in terms of the abstract operations supported by
the recursion scheme.

\subsection{Mendler-style iteration}
Mendler-style iteration (\MIt{}) operates on recursive types constructed by
the fixpoint $\mu$. The fixpoint $\mu$ is indexed by a kind. We describe
$\mu$ at kind $*$ and $*\to*$ using Haskell in Listing~\ref{lst:reccomb}.
There are two stylistic restrictions on the Haskell code
following Mendler style:
\begin{itemize}
\item Recursion is allowed only by the fixpoint at type-level and
	by the recursion scheme at term-level.
\lstinline{Mu0} expects a non-recursive base structure \lstinline{f : * -> *}
to construct a recursive type \lstinline{(Mu0 f : *)}. \lstinline{Mu1} expects
a non-recursive base structure \lstinline{f : (* -> *) -> (* -> *)}
to construct a recursive type constructor \lstinline{(Mu1 f  : * -> *)},
which expects one type index (\lstinline{i :: *}). We do not rely on
recursive datatype definition natively supported by Haskell, except for
defining $\mu$. We do not rely on function definitions using general recursion
either, except for defining \msfit{}.

\item Elimination of recursive values is only allowed via the recursion scheme.
One is allowed to freely introduce recursive values using \In{}-constructors,
but not allowed to freely eliminate (\ie, pattern match against \In{})
those recursive values. One can only eliminate recursive values via
the recursion scheme. Note that \lstinline{mcata0} and \lstinline{mcata1}
are defined using pattern matching against \In{*} and \In{*\to*}.
Pattern matching against them elsewhere is prohibited.
\end{itemize}

The type synonyms \lstinline{Phi0} and \lstinline{Phi1} stands for
the first argument types of \lstinline{mcata0} and \lstinline{mcata1}.
These type synonyms indicate that Mendler-style iteration supports
one abstract operation: abstract recursive call.
The type (constructor) variable $r$ stands for abstract recursive values.
Since $r$ universally quantified within \lstinline{Phi0} and \lstinline{Phi1},
functions of either \lstinline{Phi0}-type or \lstinline{Phi1}-type must be
parametric over $r$ (\ie, must not rely on examining specifics of $r$-values).
In \lstinline{Phi0}, \lstinline{(r -> a)} is the type for
an abstract recursive call, which computes an answer of type \lstinline{a}
from the abstract recursive type \lstinline{r}. One uses an handle to
this abstract recursive call when implementing a function of type
\lstinline{f r -> a}, which computes en answer (\lstinline{a})) from
\lstinline{f}-structures filled with abstract recursive values (\lstinline{r}).
Similarly, \lstinline{(forall i.r i -> a i)} in \lstinline{Phi1} is the type
for an abstract recursive call, which is an index preserving function that
computes an indexed answer \lstinline{(a i)} from an indexed recursive value
\lstinline{(r i)}. In the Haskell definitions of \lstinline{mcata0} and
\lstinline{mcata1}, these abstract operations are implemented by Haskell's
natively supported recursive function call. Note that the first arguments to
\lstinline{phi} in the definitions of \lstinline{mcata0} and \lstinline{mcata1}
are \lstinline{(mcata0 phi)} and \lstinline{(mcata1 phi)}.

Uses of Mendler-style recursion schemes are best explained by examples.
Listing~\ref{lst:Len} is a well-known example of a list length function
defined in terms of \lstinline{mcata0}. The recursive type for lists
\lstinline{(List a)} is defined as a fixpoint of \lstinline{(L a)},
where \lstinline{L} is the base structure for lists. The data constructors
of \lstinline{List}, \lstinline{nil} and \lstinline{cons}, are defined
in terms of \lstinline{In0} and the data constructors of \lstinline{L}.
We define \lstinline{length} by applying \lstinline{mcata0} to
the \lstinline{phi} function, The function \lstinline{phi} is defined
by two equations, one for the \lstinline{N}-case and the other for
the \lstinline{C}-case. When the list is empty (\lstinline{N}-case),
the \lstinline{phi} function simply returns 0. When the list has an
element (\lstinline{N}-case), we first compute the length of the tail
(\ie, the list excluding the head, that is, the first element) by
applying the abstract recursive call \lstinline{(len :: r -> Int)}\footnote{
	Here, the answer type is \lstinline{Int}. }
to the (abstract) tail \lstinline{(xs :: r)},\footnote{
	Note that \lstinline{C x xs :: L p r} since \lstinline{xs :: r}.}
and, then, we add 1 to the length of the tail \lstinline{(len xs)}.

\begin{figure}
\lstinputlisting[
	caption={List length example using \lstinline{mcata0} \label{lst:Len}},
	firstline=3]{Len.hs}
\vspace*{-3ex}
\end{figure}


\subsection{Mendler-style iteration with syntactic inverses}
Mendler-style iteration with syntactic inverses (\msfit{}) operates on
recursive types constructed by the fixpoint $\mu'$. The fixpoint $\mu'$
is a non-standard fixpoint additionally parameterized by the answer type ($a$)
and has two constructors $\In{}'$ and \textit{Inverse}. $\In{}'$-constructors
are analogous to \In{}-constructors of $\mu$. \textit{Inverse}-constructors
hold answers to be computed by \msfit{}. For example,\footnote{
	In fact, this example is ill typed, because \msfit{} expects
	its second argument type to be parametric over
	(\ie, does not rely on specifics of) the answer type.
	This example is just to illustrate the intuitive idea.}
the result of computing \lstinline{ msfcata0  phi (Inverse0 5) }
is \lstinline{5} regardless of \lstinline{phi}.
The stylistic restrictions on the Haskell code involving \msfit{} are:
\begin{itemize}
\item Recursion is only allowed by the fixpoint at type-level ($\mu'$)
	and by the recursion scheme (\msfit{}) at term-level.
We do not rely on recursive datatype definitions and function definitions
general recursion
natively supported in Haskell.
\item Elimination of recursive values are only allowed via the recursion scheme.
One is allowed to freely construct recursive values using $\In{}'$-constructors,
but not allowed to freely eliminate (\ie, pattern match against $\In{}'$) them.
Pattern matching against \textit{Inverse} is also restricted.
\end{itemize}
These restrictions are similar to the stylistic restrictions involving \MIt{}.

The abstract operations supported by \msfit{} are evident
in the first argument type -- \lstinline{Phi0'} and \lstinline{Phi1'}
are the type synonyms for the first argument types of \lstinline{msfcata0}
and \lstinline{msfcata1}. Note that the abstract recursive type $r$ is also
additionally parameterized by the answer type $a$ in the type sigatures
of \lstinline{msfcata0} and \lstinline{msfcata1}, since $\mu'$ is additionaly
parameterized by $a$. In addition to the abstract recursive call, \msfit{}
also supports the abstract inverse operation. Note that the types for
abstract inverse (\lstinline{(a -> r a)} and \lstinline{(a i -> r a i)})
are indeed the types for inverse functions of abstract recursive call
(\lstinline{(r a -> a)} and \lstinline{(r a i -> a i)}). Instead of using
actual inverse functions to compute inverse images from answer values
during computation, one can hold intermediate answer values, whose inverse
images are irrelevant, inside \textit{Inverse}-constructors during
the computation using \msfit{}.

The type signature of \msfit{} expects the second argument to be
parametric over the answer type. Note the second argument types
\lstinline{(forall a. Rec0 f a)} and \lstinline{(forall a. Rec1 f a i)}
in the type signatures of \lstinline{msfcata0} and \lstinline{msfcata1}. 
Using \textit{Inverse} to construct recursive values elsewhere is, in a way,
prohibited due to the second argument type of \msfit{}. Using \textit{Inverse}
to construct concrete recursive values makes the answer type specific.
For example, \lstinline{(Inverse0 5) :: Rec0 f Int}, whose answer type
made specific to \lstinline{Int}, cannot be passed to \lstinline{msfcata0}
its second argument. The constructor \textit{Inverse} is only intended to
define \msfit{} and its first argument (\lstinline{phi}). One can indirectly
access \textit{Inverse} via the abstract inverse operation supported by
\msfit{}. Note, in the Haskell definitions of \lstinline{msfcata0} and
\lstinline{msfcata1}, the second arguments to \lstinline{phi} are
\lstinline{Inverse0} and \lstinline{Inverse0}. That is, the abstract inverse
operation is implemented by the \textit{Inverse}-constructor.

We adopt the HOAS formatting
\begin{figure}
\lstinputlisting[
	caption={Formatting an untyped HOAS expression into a \lstinline{String}
		\label{lst:HOASshow} (adopted from \cite{AhnShe11}) },
	firstline=5]{HOASshow.hs}
\vspace*{-3ex}
\end{figure}
TODO Write text about the showExp example TODO

%% Using general recursion, one would have defined
%% the datatype |Exp_g :: * -> *| that corresponds to |Exp|
%% as follows, using Haskell's native recursive datatype definition.
%% \begin{code}
%% data Exp_g t where
%%   Lam_g :: (Exp_g a -> Exp_g b) -> Exp_g (a -> b)
%%   App_g :: Exp_g (a -> b) -> Exp_g a -> Exp_g b
%% \end{code}


