%\documentclass[preliminary,copyright,creativecommons]{eptcs}
%\documentclass[submission,copyright,creativecommons]{eptcs}
\documentclass[adraft,copyright,creativecommons]{eptcs}
\providecommand{\event}{FICS or MSFP 2012} % Name of the event to submit
\usepackage{breakurl}              % Not needed if you use pdflatex only.
\usepackage{amssymb}
\usepackage[fleqn]{amsmath}
\usepackage{semantic}

\title{Nax theory}

\author{Ki Yung Ahn \qquad Tim Sheard
\institute{Portland State University\thanks{Thanks here if needed}\\
           Portland, Oergon, USA}
\email{kya@cs.pdx.edu \qquad sheard@cs.pdx.edu}
\and
Marcelo P. Fiore \quad\qquad Andrew M. Pitts
\institute{University of Cambridge\thanks{Another thanks here if needed}\\
           Cambridge, UK}
\email{\{Marcelo.Fiore,Andrew.Pitts\}@cl.cam.ac.uk}
}
\def\titlerunning{title running}
\def\authorrunning{KY. Ahn, T. Sheard \& M. P. Fiore, A. M. Pitts}

\newcommand{\eg}{{e.g.}}
\newcommand{\ie}{{i.e.}}

\newcommand{\Fi}{\ensuremath{\mathsf{F}_i}}
\newcommand{\Fw}{\ensuremath{\mathsf{F}_\omega}}
\newcommand{\fix}{\mathsf{fix}}
\newcommand{\Fix}{\mathsf{Fix}}
\newcommand{\Fixw}{\ensuremath{\Fix_{\omega}}}
\newcommand{\Fixi}{\ensuremath{\Fix_{i}}}

\newcommand{\Nat}{\ensuremath{\mathsf{Nat}}}
\newcommand{\Bool}{\ensuremath{\mathsf{Bool}}}
\newcommand{\sfList}{\ensuremath{\mathsf{List}}}
\newcommand{\sfVec}{\ensuremath{\mathsf{Vec}}}
\newcommand{\SAT}{\ensuremath{\mathsf{SAT}}}

\newcommand{\oz}{\oldstylenums{0}}
\newcommand{\ka}{{\check\kappa}}

\newcommand{\calS}{\mathcal{S}}
\newcommand{\calA}{\mathcal{A}}
\newcommand{\calR}{\mathcal{R}}
\newcommand{\dom}{\mathop{\mathsf{dom}}}

\begin{document}
\maketitle

\begin{abstract}
This is a sentence in the abstract.
This is another sentence in the abstract.
This is yet another sentence in the abstract.
This is the final sentence in the abstract.
\end{abstract}

\chapter{Introduction}\label{ch:intro}

\section{Programming and Formal Reasoning}\label{sec:intro:motiv}
In this dissertation, we contribute to answering the question:
``how does one build a seamless system where programmers can both
write (functional) programs and formally reason about those programs''.
In late 1960s, \citet{Howard69} observed that natural deduction, which is
a proof system of a formal logic, and a typed lambda calculus, which is
a model of computation, are directly related --
a proof of a proposition corresponds to a program and its type.
Since this observation, known as the Curry--Howard correspondence,
logicians and programming language researchers 
have  dreamed of
building a system in which one can both write programs
(\ie, model computation) and formally reason about (\ie, construct proofs of)
the properties (\ie, types) of those programs.

However, building a practical system that unifies programming and
formal reasoning, based on the Curry--Howard correspondence, is still
an open research problem. The gap between the conflicting
design goals of typed functional programming languages, such as ML and Haskell,
and formal reasoning systems, such as Coq and Agda, is still wide.

\begin{itemize}

\item
Programming languages are typically designed to achieve
computational expressiveness. They often sacrifice logical consistency
to achieve this goal. Programmers should be able to
conveniently express all possible computations, regardless of whether those
computations have a logical interpretation or not.

\item
Formal reasoning systems are typically designed to achieve logical consistency.
They often sacrifice computational expressiveness to achieve that goal.
Users expect that it is only possible to prove true propositions,
and it is impossible to prove falsity. They are willing
live with the difficultly (or even impossibility) to
express certain computations within the reasoning system,
to achieve logical consistency.

\end{itemize}

As a result, the recursion schemes of programming languages and
formal reasoning systems differ considerably.
Programming languages provide unrestricted general recursion,
to conveniently express computations
that may or may not terminate.
Formal reasoning systems provide induction principles for sound reasoning,
or, in the computational view, principled recursion schemes
that can only express terminating computation.

The two different design goals also lead to significant differences
in their type system as well.
Programming languages are based on \emph{recursive types},
which which place only syntactic restrictions on the definition of new types.
Programmers can express computations over a wide variety types.
In addition, most (statically typed) functional programing languages have
clear distinction between terms and types (\ie, terms do not appear in types).
Reasoning systems are usually based on \emph{inductive types},
which place semantic restrictions, accepting only type definitions that support
conventional induction principles.
In addition, most reasoning systems, based on the Curry--Howard correspondence,
allow types to depend on terms (\ie, terms can appear in types) to specify
fine grained properties.

This dissertation explores a sweet spot where one can benefit from
the advantages of both programming languages and formal reasoning systems.
That is, we design a unified language system, called Nax, which is
logically consistent while being able to conveniently express
many useful computations. We do this by placing few restriction on type definitions,
as is done in programming languages, but also provide a rich set of
non-conventional recursion schemes (or, induction principles) that
always terminate. These non-conventional recursion schemes are known as
\emph{the Mendler style}. Another major design choice in Nax is
supporting \emph{term indices} in types, a middle ground, which sits between
polymorphic types and dependent types.

In the following section, we explain what we mean by the sweet spot between programming languages
and reasoning systems. Our thesis is that the design choices we explain below
are reasonable for achieving the goal of combining programming and resoning systems.

\section{Thesis}\label{sec:intro:thesis}
Whatever design choices we make, the sweet spot should have the following features.

\begin{enumerate}[(1)]
 \item \textbf{A convenient programming} style
         supported by the major constructs of
         modern functional programming languages: 
         parametric polymorphism, recursive datatypes,
         recursive functions, and type inference,
 \item \textbf{An expressive logic}
         that can specify fine-grained program properties using types, and terms that
         witness proofs of these properties 
         (the Curry--Howard correspondence),
 \item \textbf{A small theory} based upon a minimal foundational calculus that is
         expressive enough to support the programming features, expressive
         enough to embed propostions and proofs about
         programs, and logically consistent
         to avoid paradoxical proofs in the logic, and
 \item \textbf{A simple implementation} that keeps the trusted base small.
\end{enumerate}
We claim that a language design based on \emph{Mendler-style recursion schemes}
and \emph{term-indexed types} can lead to a system that supports these four
features.

\paragraph{}
From a bird's-eye view, the following chapters back up our claim as follows:
Mendler-style recursion schemes support (1) because they are based on
parametric polymorphism and well-defined over a wide range of datatypes.
Term-indexed types support (2), because they can statically track program
properties. For instance the size of data structures can be tracked by using
a natural number term in their types.
To support (3), we design several foundational calculi, each which extends
a well known polymorphic lambda calculus with term-indexed types.
Mendler-style recursion schemes also also support (4) because their
termination is type-based -- no need for an auxiliary termination checker.

In next section, we summarize important ideas mentioned in our thesis above.

\section{Preliminary concepts}\label{sec:intro:concepts}
We give summaries of the following preliminary concepts:
Curry--Howard correspondence (\S\ref{sec:intro:concepts:CH}),
Mendler-style recursion schemes
(\S\ref{sec:intro:concepts:CH}, \S\ref{sec:intro:concepts:mendler}),
and term-indexed types (\S\ref{sec:intro:concepts:indexed}).
Further details and historical backgrounds on each of these concepts
will appear in the following chapters (see \S\ref{sec:intro:overview}
for the overview of chapter organization).

\subsection{The Curry--Howard correspondence and Normalization}
\label{sec:intro:concepts:CH}
One promising approach to designing a system that unifies
logical reasoning and programming is \emph{the Curry--Howard correspondence}.
Howard \cite{Howard69} observed that a typed model of computation
(\ie, a typed lambda calculus) gives an interpretation to a (natural deduction)
proof system (for an intuitionistic logic). More specifically, one can interpret
a type (in the lambda calculus) as a formula (in the logic) and
a term of that type, as a proof for that formula. For instance,
the typing rule for function applications (APP) in a typed lambda calculus
corresponds to Modus Ponens (MP) in a logic:
\[ \inference[(APP)]{\Gamma |- t_1 : A -> B & \Gamma |- t_2 : A}{
        \Gamma |- t_1~t_2 : B}
 ~~~~~~~~
   \inference[(MP)]{A -> B & A}{B}
\]
As you can see above, combining terms ($t_1$ and $t_2$) to build a new term
($t_1~t_2$) can be interpreted as combining proofs for formulae
($A -> B$ and $A$), to construct a proof for a new formula ($B$).
More generally, we may expect that programming (\ie, building larger terms)
corresponds to constructing larger proofs, but only when the typed lambda calculi
meets certain standards -- \emph{type soundness} and \emph{normalization}.

The Curry--Howard correspondence is a promising approach to designing a
unified system for both logical reasoning and programming. Only one language
system is needed for both the logic and the programming language. An
alternate approach is to use an external logical language to talk about
programs as the objects that the logic reasons about. In this approach, one
has the obligation to argue that the soundness of the logic, with respect to
the programming language semantics, holds.

Under the Curry--Howard correspondence, the logic is internally related to the
semantics of program -- there is no need to argue for the soundness of the
logic,  externally outside of the programming language system. The soundness
of the logic follows directly from the type soundness of the language under
the Curry--Howard correspondence.

Let us consider a proposition to be true
(or, valid) when it has a canonical (\ie, cut-free) proof.
That is, there is a program, whose type is the proposition under
consideration, and that program has a normal form. 
By type soundness, any term,
of that type, will preserve its type during the reduction steps. Thus
reduction preserves truthfulness. If we assume
that the language is normalizing (\ie, every well-typed term reduces to
a normal form), then any term of that type which is a non-canonical proof,
implies the existence of a canonical proof, which in turn implies that
the proposition specified by the type is indeed true. That is, all provable
propositions are valid (\ie, the logic is sound) when the language is
\emph{type sound} and \emph{normalizing}.

\emph{Normalization} is also essential for the consistency of the logic.
For the lambda calculus to be interpreted as a \emph{consistent} logic,
there must be no diverging terms. A diverging term (\ie, a term that does
not have a normal form) may inhabit any arbitrary type. Thus, a diverging term
can be a proof for any proposition under the Curry--Howard correspondence.
General purpose functional programming languages (\eg, Haskell and ML), that
support unrestricted general recursion, cannot be interpreted as a consistent
logic, since they allow diverging terms (\ie, non-terminating programs).
For example, a diverging Haskell definition $\textit{loop} = \textit{loop}$
can be given an arbitrary type such as
$\textit{loop}\mathrel{::}\textit{Bool}$,
$\textit{loop}\mathrel{::}\textit{Int} -> \textit{Bool}$,
and even $\textit{loop}\mathrel{::}\forall a. a$, which is a proof of false.


Therefore, useful logical reasoning systems based on the Curry--Howard
correspondence (\eg, Coq, Agda) never support language features that can
lead to diverging terms. For example, in both Coq and Agda,
unrestricted general recursion (at term level) is not supported. 
Instead, these logical reasoning systems
often provide principled recursion schemes over recursive types that are
guaranteed to normalize. 

Recursive types (\ie, recursion at type level)
can also lead to diverging terms when they are not restricted carefully.
Many of the conventional logical reasoning systems, based on
Curry--Howard correspondence, restrict recursive types in a way,
which is not an ideal design choice, if one's goal is a unified system for
logic and programming. Our approach explores another design space not yet
completely explored. We introduce both approaches to restricting recursive
types to ensure normalization in the following two subsections.


\subsection{Restriction on recursive types for normalization}
\label{sec:intro:concpets:recursive}
We have argued that normalization is essential for logical reasoning systems
based on the Curry--Howard correspondence. One challenge to the successful
design of such systems is how to restrict recursion at the type level
so that normalization of terms is preserved. 
There are two different
design choices illustrated in Figure~\ref{fig:approaches}. 
The conventional approach restricts the formation
of recursive types (\ie, the restriction is in datatype definition), and
the Mendler-style approach restricts the elimination
of the values of recursive types (\ie, the restriction is in pattern matching).

\begin{figure}
{\centering
\begin{tabular}{p{3cm}|p{12.5cm}}
\parbox{3cm}{~~Functional\\programming\\$~~~~$language} &
\parbox{12.5cm}{
 kinding:~
  \inference[($\mu$-form)]{\Gamma |- F : * -> *}{\Gamma |- \mu F : *} \\
 \\
 typing:\quad
  \inference[($\mu$-intro)]{\Gamma |- t : F (\mu F)}{\Gamma |- \In~t:\mu F} ~~~~
  \inference[($\mu$-elim)]{\Gamma |- t : \mu F}{\Gamma |- \unIn~t : F (\mu F)}\\
 \\
 reduction:
  \inference[(\unIn-\In)]{}{\unIn~(\In~t) \rightsquigarrow t}
} \\
\\ \hline\hline
\parbox{3cm}{$~$Conventional\\$~~~$approach for\\consistent logic} &
\parbox{12.5cm}{$\phantom{a}$\\
 kinding:~
  \inference[($\mu$-form$^{+}$)]{ \Gamma |- F : * -> * 
                           & \mathop{\mathsf{positive}}(F)}
                           {\Gamma |- \mu F : *} \\
 \\
 typing:~
  \text{{\small($\mu$-intro)} and {\small($\mu$-elim)}
                same as functional language} \\
  \[\inference[(\It)]{\Gamma |- t : \mu F & \Gamma |- \varphi : F A -> A}
                     {\Gamma |- \It~\varphi~t : A}\]
 reduction:~ \text{{\small(\unIn-\In)} same as functional language}
  \[\inference[(\It-\In)]{}{\It~\varphi~(\In~t) \rightsquigarrow
                            \varphi~(\textsf{map}_F~(\It~\varphi)~t)}\]
}
\\ \hline
\parbox{3cm}{Mendler-style\\$~~$approach for\\consistent logic} &
\parbox{12.5cm}{$\phantom{a}$\\
 kinding:~ \text{{\small($\mu$-form)} same as functional language} \\
 \\
 typing:~
  \text{{\small($\mu$-intro)} same as functional language}
  \[\inference[(\MIt)]
     { \Gamma |- t : \mu F &
       \Gamma |- \varphi : \forall X . (X -> A) -> F X -> A}
     {\Gamma |- \MIt~\varphi~t : A} \]
 reduction:~
  \inference[(\MIt-\In)]
     {}
     {\MIt~\varphi~(\In~t) \rightsquigarrow \varphi~(\MIt~\varphi)~t}
}
\end{tabular} }
\caption{Two different approaches to designing a logic
         (in contrast to functional languages)}
\label{fig:approaches}
\end{figure}

\paragraph{Recursive types in functional programming languages.}
Let us start with a review of the theory of recursive types used
in functional programming languages. Here, the term
language is not expected to be normalizing, so restrictions are few.

Just as we can capture the essence of unrestricted general recursion at term
level, by a fix point operator (usually denoted by \textsf{Y} or \textsf{fix}),
we can capture the essence of recursive types by the
use of fix point operator, $\mu$, at type level. 
The rules for the formation {\small($\mu$-form)},
introduction {\small($\mu$-intro)}, and elimination {\small($\mu$-elim)} of
the recursive type operator $\mu$ are described in Figure \ref{fig:approaches}.
We also need a reduction rule {\small(\unIn-\In)}, which relates \In,
the data constructor for recursive types, and \unIn, the destructor for
recursive types, at the term level.

Surprisingly (if you hadn't known), the recursive {\em type} operator, $\mu$,
as described in Figure \ref{fig:approaches}, is already powerful enough to
express non-terminating programs, even without introducing the general recursive
{\em term} operator, \textsf{fix}, to the language. We illustrate this below.
First a short reminder of how a fixpoint at the term level operates.
The typing rule and the reduction rule for \textsf{fix} can be given as follows:
\[ \text{typing:}~ \inference{\Gamma |- f : A -> A}{\textsf{fix}\,f : A}
 \qquad\qquad
   \text{reduction}:~ \textsf{fix}\,f \rightsquigarrow f(\textsf{fix}\,f)
\]
We can actually implement \textsf{fix}, using $\mu$, as follows
(using some Haskell-like syntax):
\begin{align*}
& \textbf{data}~T\;a\;r = C\;(r -> a) \quad
          \texttt{-}\texttt{-}~\text{\small a non-recursive datatype} \\
& w \,:\, \mu(T\;a) -> a ~~ \quad
          \texttt{-}\texttt{-}~\text{\small an encoding of the untyped
                                     $(\lambda x.x\;x)$
                                     in a typed language}~ \\
& w = \lambda x . \,\textbf{case}~\unIn~x~\textbf{of}~C\;f -> f\;x \\
& \textsf{fix} \,:\, (a -> a) -> a \quad
          \texttt{-}\texttt{-}~\text{\small an encoding of 
                                     $(\lambda f.(\lambda x.f(x\;x))\,
                                                 (\lambda x.f(x\;x)))$} \\
& \textsf{fix} = \lambda f. (\lambda x. f (w\;x))\,(\In(C(\lambda x. f (w\;x))))
\end{align*}

Thus, to avoid the loss of termination guarantees, we need to alter the rules
for $\mu$, in someways, to ensure a consistent logic. One way, is to restrict
the rule {\small $\mu$-form}; the other way, is to restrict the rule
{\small $\mu$-elim}. Once we decide which of these two alterations of the
rules we will use, the design of principled recursion combinators (\eg, \It\
for the former and \MIt\ for the latter) follows from that choice.

\paragraph{Recursive types in the conventional approach to consistent logic.}
In the conventional approach, the formation (\ie, datatype definition) of
recursive types is restricted, but arbitrary elimination (\ie, pattern matching)
over the values of recursive types is allowed. In particular, the formation of
negative recursive types is restricted. Only positive recursive types are
supported. Thus, in Figure \ref{fig:approaches}, we have a restricted version of
the formation rule {\small($\mu$-form$^{+}$)} has an additional condition that
require $F$ to be positive. The other rules {\small($\mu$-intro)},
{\small($\mu$-elim)}, and {\small(\unIn-\In)} remain the same as for
functional languages. Since we have restricted the recursive types
at the type level and we do not have general recursion at the term level,
the language is indeed normalizing. However, we can neither write
interesting (\ie, recursive) programs that involves recursive types nor
inductively reason about those programs, unless we have principled recursion
schemes that are guaranteed to normalize. One such recursion scheme is called
iteration (\aka\ catamorphism). The typing rules for the conventional iteration
\It\ are illustrated in Figure \ref{fig:approaches}. Note, we have the typing
rule {\small(\It)} and the reduction rule {\small(\It-\In)} for \It\,
in addition to the rules for the recursive type operator $\mu$.

\paragraph{Recursive types in the Mendler-style approach to consistent logic.}
In the Mendler-style approach, we allow arbitrary formation
(\ie, datatype definition) of recursive types, but we restrict
the elimination (\ie, pattern matching) over the values of recursive types. 
The formation rule {\small($\mu$-form)} remains the same as
for functional languages. That is, we can define arbitrary recursive types,
both positive and negative. However, we no longer have the elimination
rule {\small($\mu$-elim)}. That is, we are not allowed to pattern match over
the values of recursive types in the normal fashion. We can only pattern match
over the values of recursive types through the Mendler-style recursion
combinators. The rules for the Mendler-style iteration combinator \MIt\
are illustrated in Figure \ref{fig:approaches}.
Note, there are no rules for \unIn\ in the Mendler-style approach.
The typing rule {\small($\mu$-elim)} is replaced by {\small(\MIt)} and
the reduction rule {\small(\unIn-\In)} is replaced by {\small(\MIt-\In)}.
More precisely, the typing rule {\small \MIt} is both an elimination rule
for recursive types and a typing rule for the Mendler-style iterator.
You can think of the rule {\small(\MIt)} as replacing both the elimination rule
{\small($\mu$-elim)} and the typing rule for conventional iteration
{\small(\It)}, but in a safe way that guarantees normalization.

\subsection{Justification of the Mendler-style as a design choice.}
\label{sec:intro:concepts:mendler}
We choose to base our approach to the design of a seamless synthesis of both
logic and programming on the Mendler-style. It restricts the elimination (\ie,
pattern matching) over values of recursive types, rather restricting the
formation (\ie, datatype definition) of recursive types (a more conventional
approach). The impact of this design choice is that it enables the logic to
include all datatype definitions that are used in functional programming
languages.

Functional programming promotes ``functions as first class values''.
It is natural to pass functions as arguments and embed functions into
(recursive) datatypes. If embedding functions in datatypes is allowed,
we can embed a function whose domain is the very type we are defining.
For example, the recursive datatype definition
$\mathbf{data}~T = C\;(T -> \textit{A})$ in Haskell is such a recursive
datatype definition. Such datatypes are called negative recursive datatypes
since the recursive occurrence $T$ appears in a negative position.
We say that $T$ is in a negative position, since $(T -> A)$ is analogous to
$(\neg T \land A)$ when we think of $->$ as a logical implication. There exist
both interesting and useful examples in functional programming involving
negative datatypes. In \S\ref{sec:msf}, we illustrate that
the Mendler-style recursion scheme we discovered can express
interesting examples involving negative datatypes.

Recall that the motivation of this dissertation research
(quoting again from \S\ref{sec:intro:motiv})
is to contribute to answering the question of {\em how does one build a
seamless system where programmers can both write (functional) programs and
formally reason about those programs}. Under the Curry--Howard correspondence,
to formally reason about a program, the logic needs to refer to the type of
the program, since the type, interpreted as a proposition, describes its
properties. Since the Mendler-style approach does not restrict recursive
datatype definitions, we can directly refer to the types of programs that use
negative recursive types.

The Mendler style is a promising approach to building a unified system because
all the recursive types (both positive and negative) are definable and
the recursion schemes over those types are normalizing.
%% As we mentioned previously, the Mendler-style iteration
%% (\MIt) always terminate for both positive and negative recursive types.
%% There exist other families of Mendler-style recursion combinators,
%% which also guarantee for negative recursive types, and more useful
%% than \MIt\ over negative datatypes.
Although the conventional approach is widely followed
in the design of formal reasoning systems (\eg, Coq, Agda), it cannot directly
refer to programs that use non-positive recursive types.One may object that
it is possible to indirectly model negative recursive types
in the conventional style, via alternative equivalent encodings
which map negative recursive types into positive ones. But, such
encodings do not align with our motivation towards a seamless unified
system for both programming and reasoning. It is undesirable to require
programmers to significantly change their programs just to reason about them.
If the change is unavoidable, it should be kept small. That is,
the changed program should syntactically resemble the original program,
which the programmer would usually write in a functional programming language.
In Chapter 3, we show a number of examples of programs written in
the Mendler style that look more close to the programs written using
general recursion than the programs written in the conventional style.

%% Throughout this dissertation,
%% we show that the Mendler-style recursion schemes are
%% indeed useful and well-behaved induction principles.

\subsection{Term-indexed types, type inference, and datatypes}
\label{sec:intro:concepts:indexed}
One of the most frequently asked questions about our design choices for Nax,
regarding term-indexed types, is ``why not dependent types?". Our answer
is that a moderate extension to the polymorphic calculus is a better candidate
than a dependently typed calculus as the basis for a practical programming
system. Recall, that we hope to design a unified system for programming
as well as reasoning. Language designs based on indexed types can
benefit from existing compiler technology and type inference algorithms
for functional programming languages. In addition, theories for
term-indexd datatypes are simpler than theories for full-fledged
dependent datatypes, because term-indexd datatypes can be encoded as
functions (using Church-like encodings).

The implementation technology for functional programming languages based on
polymorphic calculi is quite mature. There exist industrial
strength implementations, such as the Glasgow Haskell Compiler (GHC),
whose intermediate core language is an extension of \Fw.
Our term-indexed calculi described in Part \ref{part:Calculi} are closely
related to \Fw\ by an index-erasure property. The hope is that
our implementation can benefit from these technologies.

Type inference algorithms for functional programming languages are often
based on certain restrictions of the Curry-style polymorphic lambda calculi.
These restrictions are designed to avoid higher-order unification during
type inference.
We develop a conservative extension of Hindley--Milner type inference for
Nax (Chapter \ref{ch:naxTyInfer}). This is possuble because Nax is based on our
term-indexed calculi (Part \ref{part:Calculi}). Dependently typed languages,
on the other hand, are often based on bidirectional type checking, which
requires annotations on top level definitions, rather than
Hindley--Milner-style type inference.

In dependent type theories, datatypes are usually supported as primitive
constructs with axioms, rather than as functional encodings
(\eg, Church encodings). One can give functional encodings for datatypes
in a dependent type theory, but one soon realizes that the induction principles
(or, dependent eliminators) for those datatypes cannot be derived within
the pure dependent calculi \cite{Geuvers01}.
So, dependently typed reasoning systems support datatypes as primitives.
For instance, Coq is based on Calculus of Inductive Constructions, which
extends Calculus of Constructions \cite{CoqHue86} with dependent datatypes
and their induction principles.

In contrast, in polymorphic type theories, all imaginable datatypes
within the calculi have functional encodings (\eg, Church encodings).
For instance, \Fw\ need not introduce datatypes as primitive constructs,
since \Fw\ can embed all imaginable datatypes, including non-regular
recursive datatypes with type indices. 

Another reason to use term-indexed calculi, rather than dependent type theories,
is to extend the application of Mendler-style recursion schemes,
which are well-understood in the context of \Fw.
Researchers have thought about (though not published)\footnote{
     Tarmo Uustalu described this on a whiteboard
     when we met with him at the University of Cambridge in 2011.
     We discuss this in Chapter \ref{ch:relwork}.}
Mendler-style primitive recursion over dependently-typed functions
over positive datatypes (\ie, datatypes that have a map), but not for
negative (or, mixed-variant) datatypes. In our term-indexed calculi,
we can embed Mendler-style recursion schemes (just as we embedded them in \Fw)
that are also well-defined for negative datatypes.

\section{Contributions}\label{sec:intro:contrib}
This dissertation makes contributions in several areas.
\begin{itemize}
\item[1.]
It organizes and expands the realm of \emph{Mendler-style recursion schemes}
(Part~\ref{part:Mendler}, \ie, Chapter \ref{ch:mendler})

\item[2.] It establishes a meta-theories for \emph{term-indexed types}
        (Part~\ref{part:Calculi}),

\item[3.] It designs a practical language (with an implementation)
        \emph{in the sweet spot} between programming and logical reasoning
        (Part~\ref{part:Nax}), and

\item[4.] It identifies several interesting open problems related to above.
\end{itemize}

\subsection{Contributions related to the Mendler style}
We organize a hierarchy of Mendler-style recursion schemes in two dimensions.
The first dimension is the abstract operations they support. For instance,
the Mendler-style iteration (\MIt) supports a single abstract operation
the recursive call. All the other Mendler-style recursion schemes
support the recursive call and an additional set of abstract operations. 
The second dimension is over the kind of the datatypes they operate over.
For example, \texttt{Nat} has kind $*$, while \texttt{Vec}
has kind $* -> \mathtt{Nat} -> *$. Each recursion scheme is actually a
family of recursion combinators sharing the same term definition
(\ie, uniformly defined) but with different type signatures at each kind.

We expand the realm of Mendler-style recursion schemes in several ways.
First, we report on a new recursion scheme $\MsfIt$, which is useful
for negative datatypes.  Second, we study the termination behaviors
of Mendler-style recursion schemes. Some recursion schemes (\eg, \MIt, \MsfIt)
always terminate for any recursive type, while others (\eg, \McvPr) only
terminate for certain classes of recursive types. Third, we extend
all Mendler-style recursion schemes to be expressive over term-indexed types.
The Mendler style has been studied in the context of \Fw\ (and several
extensions) which can express {\bf type}-indexed types. To extend Mendler-style
recursion schemes to be expressive over {\bf term}-indexed types, we report on
several theories for calculi (\Fi\ and \Fixi) that support term indices.
This is another important area of our contribution.

We provide examples that illustrate when each recursion scheme is useful
in Chapter \ref{ch:mendler}. The most interesting example among them is
the type-preserving evaluator for a simply-typed HOAS (\S\ref{sec:evalHOAS}),
which involves negative datatypes with indices.
This example is our novel discovery, which implies that
a type-preserving evaluator for a simply-typed HOAS
can be expressed within \Fw.

In addition, we develop a better understanding of some existing
Mendler-style recursion schemes. For instance, the existence of
Mendler-style course-of-values recursion (\McvPr) is reported
in the literature, but the calculus that can embed \McvPr\ was unknown.
We embed Mendler-style course-of-values recursion into \Fixi\ 
(or into \Fixw\ \cite{AbeMat04}, when we do not consider term-indices).

\subsection{Contributions to the theory of Term-Indexed Types}
Mendler-style recursion schemes have been studies in the context of
polymorphic lambda calculi. For instance, \citet{AbeMatUus03} embedded 
Mendler-style iteration (\MIt) into \Fw\ and \citet{AbeMat04} embedded
Mendler-style primitive recursion (\MPr) into \Fixw. These calculi
support type-indexed types.

To extend the realm of Mendler-style recursion schemes to include
term-indexed types, we extended \Fw\ and \Fixw\ to support term indices.
In Part \ref{part:Calculi}, we present our new calculi
\Fi\ (Chapter \ref{ch:fi}), which extends \Fw\ with term indices, and
\Fixi\ (Chapter \ref{ch:fixi}), which extends \Fixw\ with term indices.
These calculi have an erasure property that states that well-typed terms
in each calculus are also well typed terms (when erased) in the 
underlying calculus. For instance, any well typed term in \Fi\ is also
a well-typed term in \Fw, and there are no additional well-typed terms
in \Fi\ that are not well-typed in \Fw.

Our new calculi, \Fi\ and \Fixi, are strongly normalizing and
logically consistent. We show strong normalization and logical consistency
using the erasure properties. That is, strong normalization and
logical consistency of \Fi\ and \Fixi\ are inherited from \Fw\ and \Fixw.
Since \Fi\ and \Fixi\ are strong normalizing and logically  consistent,
the Mendler-style recursion schemes that can be embedded into these calculi
are adequate for logical reasoning as well as programming.

\subsection{Contributions in the design of the Nax language}
We design and implement a prototypical language Nax that explores
the sweet spot between programming oriented systems and logic oriented systems.
The language features supported by Nax provide the advantages
of both programming oriented systems and logic oriented systems.
Nax supports both term- and type-indexed datatypes,
rich families of Mendler-style recursion combinators,
and a conservative extension of Hindley--Milner type inference.
We designed Nax so that its foundational theory and
implementation framework could be kept simple.

Term- and type-indexed datatypes can express fine grained program properties
via the Curry--Howard correspondence, as in logic oriented systems. Although
not as flexible as full-fledged dependent types, indexed datatypes can
still express program invariants, such as type preserving compilation
(\S\ref{sec:example}), and size invariants on data structures.
Index types can simulate much of what
dependent types can do using singleton types. Since Nax has only erasable
indices, the foundational theory can be kept simple, and it supports
features that have the advantages of programming oriented systems 
(\eg, type inference, arbitrary recursive datatypes).

Adopting Mendler style provides merits of both programming oriented systems
and logic oriented systems. Since Mendler style is elimination based, one can
define all recursive datatypes usually supported in functional programming
languages. In addition, the programs written using Mendler-style recursion
combinators look more similar to the programs written using general recursion
than programs written in Squiggol style.
Since Nax supports only the well-behaved (\ie, strongly normalizing)
Mendler-style recursion combinators, it is safe to construct proofs using them.
In addition, Mendler-style recursion combinators are naturally well-defined
over indexed datatypes, which are essential to express fine-grained program
properties. Mendler style provides type based termination, that is, termination
is a by-product of type checking. Thus, it makes the implementation framework
simple since we do not need extra termination checking theories or algorithm.

Hindley--Milner-style type inference is familiar 
to functional programmers.
Nax can infer types for all programs that involve only regular datatypes,
which are already inferable in Hindley--Milner, without any type annotation.
Nax requires programs involving indexed datatypes to annotate their eliminators
by index transformers, which specify the relation between the input type index
and the result type. Eliminators of non-recursive datatypes are case expressions
and eliminators of recursive datatypes are Mendler-style recursion combinators.

\subsection{Contributions identifying open problems}
We identify several open problems alongside the contributions mentioned
in previews subsections. We will discuss the details of these open problems
in the future work chapter (Chapter \ref{ch:futwork}).
Here, we briefly introduce two of them.

\paragraph{Handling different interpretations of $\mu$ in one language system:}
Nax provides multiple recursion schemes (or, induction principles) used
to describe different kinds of recursive computations over recursive datatypes.
These recursion schemes are all motivated by concrete examples, which explains
the need for multiple schemes. It is more convenient to express various kinds of
recursive computations in Nax, by choosing a recursion scheme that fits
the structure of the computation, than in those systems that provide
only one induction scheme. However, there is theoretical difficulty
handling multiple interpretations of the recursive type operator $\mu$
in one language system.

Recall that we can embed datatypes as functional encodings in
our indexed type theory. Recursive datatypes and their recursion schemes in Nax
are embedded using Mendler-style encodings.
In Mendler style, one encodes the recursive type operator $\mu$
and its eliminator (the recursion scheme) as a pair.
So, there are several different encodings of $\mu$,
one for each recursion scheme. Some recursion schemes subsume others
(\ie, the more expressive one can simulate the other).

It would have been easy to describe the theory for Nax if we had
one most powerful recursion scheme that subsumes all the others,
which leads to a single interpretation of $\mu$. Unfortunately, we know of
no Mendler-style recursion scheme that subsumes all the other recursion schemes
in Nax. For instance, iteration (\MIt) can be subsumed by either 
iteration with a syntactic inverse (\MsfIt) or primitive recursion (\MPr).
But, there is no known recursion scheme that can subsume both \MsfIt\ and \MPr.

However, we strongly believe that it is okay to apply \MsfIt\ to
the result of \MPr\ (when \MPr\ produces a recursive value) and vice versa.
Intuitively, the different interpretations of $\mu$ only matter during
the internal computation of the recursion scheme. That is, one may consider
that (recursive) values resulting from different recursion schemes
share a common abstract representation of $\mu$.
The theoretical justification for this is still ongoing work.

\paragraph{Deriving positivity (or monotonicity) from polarized kinds:}
One can extend the kind syntax of arrow kinds in \Fw\ with polarities
($p\kappa_1 -> \kappa_2$ where the polarity $p$ is either $+$, $-$, or $0$)
to track whether a type constructor argument is used in
covariant (positive), contra-variant (negative), or
mixed-variant (both positive and negative) positions.
It is still an open problem whether it is possible to derive monotonicity
(\ie, the  existence of a map) for a type constructor from its polarized kind,
without examining the type constructor definition.

We identified a useful application for a solution to this open problem.
We discovered an embedding of Mendler-style course-of-values recursion in
a polarized system for positive (or monotone) type constructors.
That is, once you can show the existence of a map for a datatype,
course-of-values recursion always terminates.
However, in a practical language system, it is not desirable to burden users
with the manual derivation for every datatype on which they might want to
perform course-of-values recursion. If the type system can automatically
categorize datatypes that have maps from their polarized kinds,
this burden can be alleviated.


\section{Methodology and Overview}\label{sec:intro:overview}
This dissertation consist of five parts:
Part \ref{part:Prelude} (Prelude),
Part \ref{part:Mendler} (The Mendler style),
Part \ref{part:Calculi} (Term-indexed lambda calculi),
Part \ref{part:Nax} (The Nax language), and
Part \ref{part:Postlude} (Postlude).
The three parts in the middle, excluding the prelude and postlude parts,
describes the three steps of our approach. First, we experiment new ideas on
Mendler-style recursion schemes driven from concrete examples
using Haskell (with some GHC extensions), which is a functional language
based on an extension of \Fw\ (Part \ref{part:Mendler}). Second, we develop
theories (\ie, lambda calculi) for term-indexed datatypes to prove that
the Mendler-style recursion schemes are well-defined over indexed datatypes
and have the expected termination behavior. Lastly, we design a language system
with practical features that embodies our new ideas in and is based on the
theory we developed. Figure~\ref{fig:overview} summarizes the organization of
key concepts throughout the dissertation.

\begin{figure}
TODO \\

STLC\\
\F\
\Fw\    \Fi\ \\
\Fixw\  \Fixi\ \\
Hindley--Milner Nax type inference \\

TODO \\
make arrow diagrams
TODO \\
TODO \\
TODO \\
TODO \\
TODO \\
\caption{Summary of key concepts}
\label{fig:overview}
\end{figure}

\paragraph{Part \ref{part:Prelude} (Prelude)}\hspace{-1em} opens
the dissertation by giving an introduction (Chapter \ref{ch:intro}),
which you are currently reading, followed by
reviews on several well-known typed lambda calculi (Chapter \ref{ch:poly}).
In Chapter \ref{ch:poly}, we review
the simply-typed lambda calculus (STLC) (\S\ref{sec:stlc}),
System \F\ (\S\ref{sec:f}),
System \Fw\ (\S\ref{sec:fw}), and
the Hindley--Milner type system (\S\ref{sec:hm}).

From \S\ref{sec:stlc} to \ref{sec:fw}, we prove strong normalization using
saturated sets for each of the three calculi:
STLC (no polymorphism), System \F\ (polymorphism over types), and
System \Fw\ (polymorphism over type constructors).
The normalization proof on later sections extends upon
the normalization proof of the previous section,
as the calculus extends its feature of polymorphism.
We use the strong normalization of System \Fw\ to show that
our term-indexed lambda calculi in Part \ref{part:Calculi} are
strongly normalizing. Readers familiar with strong normalization proofs
on these calculi may skip or quickly skim over these sections.
It is worth noticing two stylistic choices in our formalization of
System \F\ and \Fw: (1) terms are in Curry style and
(2) typing contexts in System \F\ and \Fw\ are divided into two parts
    (one for type variables and the other for term variables).
This prepares readers for our formalization of the term-indexed calculi
in Part \ref{part:Calculi}, which have Curry-style terms and
typing contexts divided into two parts.

In \S\ref{sec:hm}, we review the type inference algorithm for
the Hindley--Milner type system (\S\ref{sec:hm}).
The Hindley--Milner type system (HM) is a restriction of System~\F,
which makes it possible to infer types without any type annotation on terms.
Later in Part~\ref{part:Nax} Chapter \ref{ch:naxTyInfer},
we formulate a conservative extension of HM, which restricts
the term-indexed calculus System \Fi\ (Chapter \ref{ch:fi}) in a similar manner.

\paragraph{Part \ref{part:Mendler} (the Mendler style)}\hspace{-1em} introduces
the concept of Mendler-style recursion schemes (Chapter \ref{ch:mendler})
using examples written in Haskell (with some GHC extensions). So, the readers
of Chapter \ref{ch:mendler} need no background knowledge on typed lambda calculi
but only some familiarity to functional programming. We explain the concepts of
a number of Mendler-style recursion schemes, their termination properties, and
how one recursion scheme is related to another, in an intuitive manner by using
examples written in Haskell. We also provide semi-formal proofs of termination
for some of the recursion schemes (\MIt\ and \MsfIt) by embedding the those
recursion schemes into \Fw\ fragment of Haskell. More formal and general
proof by embedding into our term-indexed lambda calculi comes later in
Part \ref{part:Calculi}.

The Mendler-style recursion schemes discussed in Chapter \ref{ch:mendler}
include iteration (\MIt), iteration with syntactic inverse (\MsfIt),
primitive recursion (\MPr), course-of-values iteration (\McvIt),
and course-of-values recursion (\McvPr). Among them, \MsfIt\ is a
Mendler-style recursion scheme we discovered.
There are even more Mendler-style recursion schemes, which are not
discussed in Chapter \ref{ch:mendler} -- we give pointers to them later in
the related work chapter (Chapter \ref{ch:relwork} in Part \ref{part:Postlude}).

\paragraph{Part \ref{part:Calculi} (term-indexed lambda calculi)}\hspace{-1em}
establishes theories for term-indexed types.
We formalized two term-indexed lambda calculi,
System \Fi\ (Chapter \ref{ch:fi}) and System \Fixi\ (Chapter \ref{ch:fixi}),
which are extensions of polymorphic calculi with term indices.
System \Fi\ extends System \Fw\ with term indices and
System \Fixi\ extends System \Fixw\ \cite{AbeMat04} with term indices.

We prove both strong normalization and logical consistency of
these term-indexed calculi using their index erasure property.
The index erasure property of a term-indexed calculus
projects a typing in the term-index calculi into
the polymorphic calculus it was extended from.
That is, all well-typed terms in \Fi\ and \Fixi\ are
also well-typed typed terms in \Fw\ and \Fixw.
That is, our term-indexed calculi, \Fi\ and \Fixi,
inherits strong normalization and logical consistency
from the polymorphic calculi, \Fw\ and \Fixw.

By embedding those recursion schemes into our term-indexed lambda calculi,
we prove that Mendler-style recursion schemes are well-defined and
terminates over term-indexed datatypes  For instance,
\MIt\ and \MsfIt\ can be embedded into System \Fi,
and, \MPr\ and \McvPr\ can be embedded into System \Fixi.

\paragraph{Part \ref{part:Nax} (the Nax language)}\hspace{-1em} consist of
three chapters.
First, we introduce the features of Nax (Chapter \ref{ch:naxFeatures})
in a tutorial format using small Nax code snippet examples.
Next, we discuss the design principles of the type system (Chapter \ref{ch:nax})
in comparison to two other systems: Haskell's datatype promotion and Agda.
We develop the discussion In Chapter \ref{ch:nax} along
a larger and more practical example Nax programs:
a type preserving interpreter and a stack safe compiler.
Lastly, we discuss type inference in Nax (Chapter \ref{ch:naxTyInfer}),
which is a conservative extension of Hindley--Milner type system (HM).
That is, any program, whose type can be inferred in HM, can also be
inferred its type in Nax without any annotation. Programs involving
term- or type-indexed datatypes, which are not supported in HM, needs
some annotation to infer their types in Nax. Annotations are only
required on three syntactic entities (datatype declarations, case expressions,
and Mendler-style recursion combinators) and nowhere else.

\paragraph{Part \ref{part:Postlude} (Postlude)}\hspace{-1em} closes
the dissertation by summarizing
  related work (Chapter~\ref{ch:relwork}),
  future work (Chapter~\ref{ch:futwork}), and
  conclusions (Chapter~\ref{ch:concl}).



\section{Background}\label{sec:bg}
\subsection{Indexed types with static indices}\label{sec:bg:ix}

 A representative example of a term
indexed type is the length indexed list type (often called the vector type).
A regular polymorphic list type ($\sfList\;a$) is parametrized by a type
parameter ($a$), which can be instantiated to a specific type, while the
vector type ($\sfVec\;a\;\{n\}$) has an additional term index ($n$), which
can be instantiated to a specific natural number value. The curly brackets
($\{\cdots\}$) around $n$ is to syntactically distinguish term indices from
other type arguments. 
\subsection{recursive types}\label{sec:bg:rec}
\subsection{Mendler style iterators}

%% \section{TODO}

\section{Systems \Fi\ and \Fixi}\label{sec:Fi}

In this document I have proposed that Mendler-style operators have translations
into strongly normalizing $\lambda$-calculi. The strategy is to develop
a sequence of calculi of increasing expressiveness. Several papers by other
researchers have begun this process. Mendler's ordinal work \cite{Mendler87}
extended System \textsf{F}, \citet{AbeMat04} extended System \Fw\ to get \Fixw.
In my thesis, I will follow in these footsteps by introducing System \Fi\
(a more expressive extension to \Fw) and System \Fixi\ (an extension to \Fixw).


\subsection{Introduction to Systems \Fi\ and \Fixi, and their key properties}
System \Fi\ is an extension of a Curry style \Fw\ by term indexed types.
By curry style, we mean that lambda terms at term level are unannotated.
That is, the term syntax of \Fi\ and \Fw\, in the Curry style, are
the same as the term syntax of the untyped lambda calculus.
The key design principle of \Fi\ is that the kind syntax of \Fw\ is extended
by allowing types ($\tau : *$) to appear in the domain of arrow kinds
($\tau -> *$), as follows:
\begin{align*}
\text{\Fw\ kinds} ~~~ \kappa ::= ~ & * \mid \kappa -> \kappa \\
\text{\Fi\; kinds}~~~ \kappa ::= ~ & * \mid \kappa -> \kappa \mid \tau -> \kappa
\end{align*}
Types, $\tau$, can appear only in the domain, but not in the range of
arrow kinds, since all kinds should be either $*$ or arrow kinds
that eventually result in $*$ (\ie, $\vec{\kappa} -> *$) -- recall that
type constructors eventually become types (\ie\ have kind $*$) when they are fully applied.
The extension to the type syntax follows directly from the extension to the kind syntax.
However, the term syntax does not change -- \Fi\ and \Fw\ have exactly the
same terms.
The extensions to \Fi\ enable users to express term indexed types.
In \S\ref{sec:mendler}, we saw several examples of term index types, such as
the length indexed list type $\textit{Vec}$, whose kind is $\textit{Nat} -> *$.
Note that \textit{Nat} is a type appearing in the domain of the arrow kind
($\textit{Nat} -> *$).

I am working with on a paper (to be submitted to an appropriate venue) that
describes the details of System \Fi. I plan to reformat and extend
the contents of this paper in a chapter in my thesis. Here, I summarize
the three key properties of \Fi. I will provide proofs in my thesis of
these properties.
\begin{description}
\item[\quad Type safety.]
\Fi\ must have the usual type safety properties (\ie, progress and preservation).

\item[\quad Index erasure.]
Index erasure is a property that well-typed terms in \Fi\ are also 
well-typed in \Fw\, and their types in \Fw\ are given by the index erasure
of their types in \Fi. That is, if $\Gamma |-_{\Fi} t : \tau$ then
$\Gamma^\circ |-_{\Fw} t : \tau^\circ$, where $\circ$ is the notation
for index erasure. The index erasure property implies that the indices are
only relevant for type checking at compile time, but computationally irrelevant
at runtime. For instance, length indexed lists should behave exactly the same as
regular (non-indexed) lists at runtime.

\item[\quad Strong normalization.]
The proof of strong normalization follows almost automatically from
index erasure, since we know that \Fw\ is normalizing.
\end{description}

System \Fixi\ is an extension of \Fixw\ by term indexed types. \Fixw\ is a
calculus developed to give a reduction preserving embedding of the Mendler
style primitive recursion family. \Fixw\ extends \Fw\ with polarized kinds
and equi-recursive types. In \Fixi, polarities of kinds are tracked so that
only the fixpoints of types with kinds of positive polarity can be taken.
Interesting properties of \Fixw\ include the ability to define constant
time predecessors.

\Fixi\ is an extension of \Fixw\ by term indexed types. The key design principles
of \Fixi\ are pretty much the same as the key design principles of \Fi.
We extend the kind syntax with types in the domain of arrow kinds,
while keeping track of polarities, as follows:
\begin{align*}
\text{\Fixw\ kinds} ~~~ \kappa ::= ~ & * \mid \kappa^p -> \kappa \\
\text{\Fixi\; kinds}~~~ \kappa ::= ~ & * \mid \kappa^p -> \kappa \mid \tau^p -> \kappa
\end{align*}
where the polarity $p$ may be either $+$, $-$, or $\circ$.
Although \Fixi\ is still in the early stages of development, I foresee 
that the work on proving the three key properties of \Fi\ will
naturally transfer to \Fixi\ with only minor changes to proof structure
regarding the bookkeeping of polarities.

\subsection{Embeddings of the Mendler style recursion combinators} In
addition to proving the three key properties of System \Fi\ and System
\Fixi, we also need to demonstrate that there exist reduction preserving
embeddings of the Mendler-style recursion combinators into either \Fi\ or
\Fixi. Showing that there are reduction preserving embeddings of \MIt in
\Fw\ and \MPr\ in \Fixw, was the sole purpose of introducing \Fw\ and \Fixw\ in
the literature on Mendler-style recursion schemes. In my thesis
I will follow this design pattern.

The embedding of a kind-indexed family of Mendler-style operators is
a pair of translations -- a translation of the recursive type operator
($\mu^\kappa$), and a translation of the Mendler style recursion combinator. 
I have extended embeddings of \MIt\ and \MPr, taking term indexed
types into consideration, by introducing new calculi \Fi\ and \Fixw,
which are extensions of \Fw\ and \Fixi with term indices. In addition,
I will show that other families of the Mendler-style recursion combinators
also have reduction preserving embeddings into either \Fi\ or \Fixi.
In particular, \MsfIt\ will be embedded in \Fi, and the course of values
recursion combinators (\McvIt\ and \McvPr) will be embedded in \Fixi.

The details of the embedding may be different for each family even though
some of them embed into the same target calculi. For instance, the target
calculi for \MIt\ and \MsfIt\ are both \Fi. However, the embeddings
of \MIt\ and \MsfIt\ are different. Generally, the translation of
the recursive type operator $\mu^\kappa$ is different for each family,
even though the target calculus of the translation may be the same.
In practice, we may want to use several different families of Mendler style
recursion combinators in one program. Therefore, we need to reconcile
these different encodings into a coherent theory to build a usful language,
we call Nax, which supports several different families of Mendler style
redcursion combinators. I will berifely introduce the Nax language
in the following section.



%% \chapter{System \Fixi}\label{ch:fixi}

In this chapter we investigate how the framework we have developed needs to be extended
to handle Mendler style operators that that support primitive recursion. Recall that
a primitive recursive operator supplies access to immediate subterms as well as the
value of recursive calls over those subterms. The factorial function is the
classic example of a primitive recursive function.



It is fairly well known that there cannot be a reduction-preserving embedding of
primitive recursion\footnote{
	Although we can define primitive recursion for positive datatypes
	in terms of iteration, which is embeddable in System \F, such an
	embedding would not reduction-preserving. That is, it will take
	more reduction steps than the usual definition of primitive recursion.
	}
in System \F. A proof of this is outlined in the paper 
{\it Induction is not derivable in second-order dependent type theory}
\cite{Geuvers01}. For similar reasons, researchers strongly believed that
there is no reduction-preserving embedding of primitive recursion 
in System \Fw. Fortunately, all hope is not lost
for finding a reduction preserving embedding in a relatively simple calculus. 
\citet{AbeMat04} have designed \Fixw, which is a simple extension of \Fw,
which embeds primitive recursion with the desired reduction behavior.
Their embedding relys on a novel use of the following extensions to \Fw,
polarized kinds and an equi-recursive fixpoint type operator,
in order to define primitive recursion within \Fixw.

As a natural extension of these ideas, in this chapter we present \Fixi,
an extension of \Fixw\ with erasable term-indices,
that embeds primitive recursion over term-indexed datatypes
as well as type-indexed datatypes and regular datatypes.

The organization of this chapter is analogous to Chapter \ref{ch:fi}, 
where we added term-indices to \Fw\ to obtain \Fi. Here
we add term indices to \Fixw\ to obtain \Fixi. We describe \Fixi\ focusing on
its differences from \Fi. Readers may refer back to Chapter \ref{ch:fi} for
those details that remain unchanged from System~\Fi.

We describe syntax and typing rules (\S\ref{sec:fixi:def}),
we illustrate embeddings of primitive recursion (\S\ref{sec:fixi:data})
and we discuss embeddings of course-of-values primitive recursion
(\S\ref{sec:fixi:cv}). Lastly, we
discuss the metatheory of \Fixi\ (\S\ref{sec:fixi:theory}).

\section{System \Fixi} \label{sec:fixi:def}
The syntax and rules of System~\Fi\ are described in
Figures \ref{fig:Fixi}, \ref{fig:Fixi2} and~\ref{fig:eqFixi}.
The extensions new to System~\Fixi, which are not original part of
either System~\Fw\ or System~\Fixw\ are highlighted by either
\dbox{dashed boxes} or \newFi{\text{grey boxes}}.

The extensions, which are not originally part of System~\Fixw, are highlighted
by \newFi{\text{grey boxes}}. Those extensions in grey boxes are to support
term indexing.  Eliding all the grey boxes from Figures~\ref{fig:Fixi},
\ref{fig:Fixi2} and~\ref{fig:eqFixi},
one obtains a version of System~\Fixw\ with typing contexts separated into
two parts.\footnote{The original description of \Fixw \cite{AbeMat04} has
one combined typing context.}

The extensions, which are not originally part of System \Fw\ but
present in System~\Fixw, are highlighted by \dbox{dashed boxes}.
Those extensions in dashed boxes are to support equi-recursive types. 
Eliding all the dashed boxes, as well as all the grey boxes,
from Figures~\ref{fig:Fixi}, \ref{fig:Fixi2} and~\ref{fig:eqFixi}, one obtains
the Curry-style System \Fw\ with typing contexts separated into two parts.


\begin{figure}\begin{singlespace}
	\small
\paragraph{Syntax:}
\begin{align*}
\!\!\!\!\!\!\!\!&\text{Sort}
 	& \square
	\\
\!\!\!\!\!\!\!\!&\text{Term Variables}
 	& x,i
\\
\!\!\!\!\!\!\!\!&\text{Type Constructor Variables}
 	& X
\\
\!\!\!\!\!\!\!\!&\text{\dbox{Polarities}}
	& \dbox{$p$} &~ ::= + \mid - \mid 0
\\
\!\!\!\!\!\!\!\!&\text{Kinds}
 	& \kappa		&~ ::= ~ *
				\mid \dbox{$p\kappa$} -> \kappa
				\mid \newFi{A -> \kappa}
\\
\!\!\!\!\!\!\!\!&\text{Type Constructors}
	& A,B,F,G		&~ ::= ~ X
				\mid A -> B
				\mid \dbox{$\fix F$} \\ &&& ~\quad
				\mid \lambda \dbox{$X^{p\kappa}$}.F
				\mid F\,G
				\mid \forall X^\kappa . B \\ &&& ~\quad
				\mid \newFi{\lambda i^A.F
				\mid F\,\{s\}
				\mid \forall i^A . B}
\\
\!\!\!\!\!\!\!\!&\text{Terms}
	& r,s,t			&~ ::= ~ x \mid \lambda x.t \mid r\;s
\\
\!\!\!\!\!\!\!\!&\text{Typing Contexts}
	& \Delta		&~ ::= ~ \cdot
				\mid \Delta,\dbox{$X^{p\kappa}$}
				\mid \newFi{\Delta, i^A} \\
&	& \Gamma		&~ ::= ~ \cdot
				\mid \Gamma, x : A
\end{align*}
\paragraph{Reduction:} \fbox{$t \rightsquigarrow t'$}
\[ 
   \inference{}{(\lambda x.t)\,s \rightsquigarrow t[s/x]}
 ~~~~
   \inference{t \rightsquigarrow t'}{\lambda x.t \rightsquigarrow \lambda x.t'}
 ~~~~
   \inference{r \rightsquigarrow r'}{r\;s \rightsquigarrow r'\;s}
 ~~~~
   \inference{s \rightsquigarrow s'}{r\;s \rightsquigarrow r\;s'}
\]
~\\
\end{singlespace}
\caption{Syntax and Reduction rules of \Fixi}
\label{fig:Fixi}
\end{figure}

\begin{figure}\begin{singlespace}\small
\paragraph{Well-formed typing contexts:}
\[ \fbox{$|- \Delta$}
 ~~~~
   \inference{}{|- \cdot}
 ~~~
   \inference{|- \Delta & |- \kappa:\square}{|- \Delta,\dbox{$X^{p\kappa}$}}
      \big( X\notin\dom(\Delta) \big)
 ~~~ \newFi{
   \inference{|- \Delta & \cdot |- A:*}{|- \Delta,i^A}
      \big( i\notin\dom(\Delta) \big) }
\]
$ \fbox{$\Delta |- \Gamma$}
 ~~~~
   \inference{|- \Delta}{\dbox{$0\Delta$} |- \cdot}
 ~~~~
   \inference{\Delta |- \Gamma & \Delta |- A:*}{
              \Delta |- \Gamma,x:A}
      \big( x\notin\dom(\Gamma) \big)
$ \vskip1ex ~
\paragraph{Sorting:} \fbox{$|- \kappa : \square$}
$ \quad
  \inference[($A$)]{}{|- *:\square}
 ~~~
   \inference[($R$)]{|- \kappa:\square & |- \kappa':\square}{
		     |- \dbox{$p\kappa$} -> \kappa' : \square}
 ~~~
   \newFi{
   \inference[($Ri$)]{\cdot |- A:* & |- \kappa:\square}{
                      |- A -> \kappa : \square} }
$
~\\
\paragraph{Kinding:} \fbox{$\Delta |- F : \kappa$}
$ \qquad
   \inference[($Var$)]{\dbox{$X^{p\kappa}$}\in\Delta & |- \Delta}{
 		       \Delta |- X : \kappa}
		\, \dbox{$(p\in \{+,0\})$} $
\[
   \inference[($->$)]{\dbox{$-\Delta$} |- A : * & \Delta |- B : *}{
                      \Delta |- A -> B : * }
 \qquad \qquad \dbox{
   \inference[($\fix$)]{\Delta |- F : +\kappa -> \kappa}{
		      \Delta |- \fix F : \kappa } }
\]
\[
  \inference[($\lambda$)]{ |- \kappa:\square
                         & \Delta,\dbox{$X^{p\kappa}$} |- F:\kappa'}{
  	\Delta |- \lambda \dbox{$X^{p\kappa}$}.F : \dbox{$p\kappa$} -> \kappa'}
 ~~~~ \qquad
 \newFi{
  \inference[($\lambda i$)]{\cdot |- A:* & \Delta,i^A |- F : \kappa}{
			    \Delta |- \lambda i^A.F : A->\kappa} }
\]
\[
  \inference[($@$)]{ \Delta |- F : \dbox{$p\kappa$} -> \kappa'
		    & \dbox{$p\Delta$} |- G : \kappa }{
                     \Delta |- F\,G : \kappa'}
 ~~~~
 \newFi{
   \inference[($@i$)]{ \Delta |- F : A -> \kappa
   		     & \dbox{$0\Delta$\!};\cdot |- s : A }{
		      \Delta |- F\,\{s\} : \kappa} }
\]
\[
   \inference[($\forall$)]{ |- \kappa:\square
   			  & \Delta,\dbox{$X^{0\kappa}$} |- B : *}{
                           \Delta |- \forall X^\kappa . B : *}
 ~~~~ \qquad ~\,
	\newFi{
   \inference[($\forall i$)]{\cdot |- A:* & \Delta, i^A |- B : *}{
                             \Delta |- \forall i^A . B : *} }
\]
\[ \newFi{
   \inference[($Conv$)]{ \Delta |- A : \kappa
                       & \Delta |- \kappa = \kappa' : \square }{
                        \Delta |- A : \kappa'} }
\]
~\\
\paragraph{Typing:} \fbox{$\Delta;\Gamma |- t : A$}
$ \qquad
 ~~~~
 \inference[($:$)]{(x:A) \in \Gamma & \Delta |- \Gamma}{
                   \Delta;\Gamma |- x:A}
 ~~~~ \newFi{
   \inference[($:i$)]{i^A \in \Delta & \Delta |- \Gamma}{
                      \Delta;\Gamma |- i:A} }
$
\[
   \inference[($->$$I$)]{\Delta |- A:* & \Delta;\Gamma,x:A |- t : B}{
                         \Delta;\Gamma |- \lambda x.t : A -> B}
 ~~~~ ~~~~
   \inference[($->$$E$)]{\Delta;\Gamma |- r : A -> B & \Delta;\Gamma |- s : A}{
                         \Delta;\Gamma |- r\;s : B}
\]
\[ \inference[($\forall I$)]{|- \kappa:\square
			    & \Delta,\dbox{$X^{0\kappa}$\!};\Gamma |- t : B}{
                             \Delta;\Gamma |- t : \forall X^\kappa.B}
			    (X\notin\FV(\Gamma))
 ~~~~ ~~~~
   \inference[($\forall E$)]{ \Delta;\Gamma |- t : \forall X^\kappa.B
                            & \Delta |- G:\kappa }{
                             \Delta;\Gamma |- t : B[G/X]}
\]
\[ \newFi{
   \inference[($\forall I i$)]{\cdot |- A:* & \Delta, i^A;\Gamma |- t : B}{
                               \Delta;\Gamma |- t : \forall i^A.B}
   \left(\begin{matrix}
		i\notin\FV(t),\\
		i\notin\FV(\Gamma)\end{matrix}\right)
 ~~~~
   \inference[($\forall E i$)]{ \Delta;\Gamma |- t : \forall i^A.B
                              & \Delta;\cdot |- s:A}{
                               \Delta;\Gamma |- t : B[s/i]} }
\]
\[ \inference[($=$)]{\Delta;\Gamma |- t : A & \Delta |- A = B : *}{
                     \Delta;\Gamma |- t : B}
\]
\end{singlespace}
\caption{Sorting, Kinding, and Typing rules of \Fixi}
\label{fig:Fixi2}
\end{figure}

\begin{figure}\begin{singlespace}\small
\paragraph{Kind equality:} \fbox{$|- \kappa=\kappa' : \square$}
$ \quad
 ~~~~
   \inference{}{|- * = *:\square} $
\[
   \inference{ |- \kappa_1 = \kappa_1' : \square
             & |- \kappa_2 = \kappa_2' : \square }{
   |- \dbox{$p\kappa_1$}-> \kappa_2 = \dbox{$p\kappa_1'$}-> \kappa_2' : \square}
 ~~~~ \newFi{
   \inference{\cdot |- A=A':* & |- \kappa=\kappa':\square}{
              |- A -> \kappa = A' -> \kappa' : \square} }
\]
\[ \inference{|- \kappa=\kappa':\square}{
              |- \kappa'=\kappa:\square}
 ~~~~
   \inference{ |- \kappa =\kappa' :\square
             & |- \kappa'=\kappa'':\square}{
              |- \kappa=\kappa'':\square}
\]
~
\paragraph{Type constructor equality:} \fbox{$\Delta |- F = F' : \kappa$}
$\qquad \dbox{$
  \inference{\Delta|- F:+\kappa-> \kappa}{\Delta|- \fix F=F(\fix F):\kappa}$ } $
\[
   \inference{ \Delta,\dbox{$X^{p\kappa}$} |- F:\kappa'
   	     & \dbox{$p\Delta$} |- G:\kappa}{
	      \Delta |- (\lambda X^{p\kappa}.F)\,G = F[G/X]:\kappa'}
 ~~~~ \newFi{
   \inference{ \Delta,i^A |- F:\kappa
             & \dbox{$0\Delta$\!};\cdot |- s:A}{
              \Delta |- (\lambda i^A.F)\,\{s\} = F[s/i]:\kappa} }
\]
\[ \inference{\Delta |- X:\kappa }{\Delta |- X=X:\kappa}
 ~~~~
   \inference{-\Delta |- A=A':* & \Delta |- B=B':*}{\Delta |- A-> B=A'-> B':*}
 ~~~~
   \inference{\Delta |- F=F' : +\kappa -> \kappa}{
	      \Delta |- \fix F = \fix F' : \kappa}
\]
\[
   \inference{ |- \kappa:\square
   	     & \Delta,\dbox{$X^{p\kappa}$} |- F=F' : \kappa'}{
   	\Delta |- \lambda \dbox{$X^{p\kappa}$}.F
		= \lambda \dbox{$X^{p\kappa}$}.F': \dbox{$\kappa$} -> \kappa'}
 ~~~~ ~
 \newFi{
   \inference{\cdot |- A:* & \Delta,i^A |- F=F' : \kappa}{
	      \Delta |- \lambda i^A.F=\lambda i^A.F' : A -> \kappa} }
\]
\[\!\!\!\!\!\!\!\!\!\!\!\!
   \inference{ \Delta |- F=F' : \dbox{$p\kappa$} -> \kappa'
   	     & \dbox{$p\Delta$} |- G=G':\kappa}{
              \Delta |- F\,G = F'\,G' : \kappa'}
 ~~~~
 \newFi{
   \inference{ \Delta |- F=F': A -> \kappa
             & \dbox{$0\Delta$\!};\cdot |- s=s':A}{
	      \Delta |- F\,\{s\} = F'\,\{s'\} : \kappa} }
\]
\[
   \inference{ |- \kappa:\square
   	     & \Delta,\dbox{$X^{0\kappa}$} |- B=B':*}{
              \Delta |- \forall X^\kappa.B=\forall X^\kappa.B':*}
 ~~~~ \quad
 \newFi{
   \inference{\cdot |- A:* & \Delta,i^A |- B=B':*}{
              \Delta |- \forall i^A.B=\forall i^A.B':*} }
\]
\[ \inference{\Delta |- F = F' : \kappa}{\Delta |- F' = F : \kappa}
 ~~~~
   \inference{\Delta |- F = F' : \kappa & \Delta |- F' = F'' : \kappa}{
              \Delta |- F = F'' : \kappa}
\]
~
\paragraph{Term equality:} \fbox{$\Delta;\Gamma |- t = t' : A$}
\[
   \inference{\Delta;\Gamma,x:A |- t:B & \Delta;\Gamma |- s:A}{
              \Delta;\Gamma |- (\lambda x.t)\,s=t[s/x] : B}
 ~~~~
   \inference{\Delta;\Gamma |- x:A}{\Delta;\Gamma |- x=x:A}
\]
\[ \inference{\Delta |- A:* & \Delta;\Gamma,x:A |- t=t':B}{
              \Delta;\Gamma |- \lambda x.t = \lambda x.t':B}
 ~~~~
   \inference{\Delta;\Gamma |- r=r':A-> B & \Delta;\Gamma |- s=s':A}{
              \Delta;\Gamma |- r\;s=r'\;s':B}
\]
\[ \inference{ |- \kappa:\square
	     & \Delta, \dbox{$X^{0\kappa}$\!};\Gamma |- t=t' : B}{
              \Delta;\Gamma |- t=t' : \forall X^\kappa.B}
	     (X\notin\FV(\Gamma))
 ~~~~ ~~~~
   \inference{ \Delta;\Gamma |- t=t' : \forall X^\kappa.B
             & \Delta |- G:\kappa }{
              \Delta;\Gamma |- t=t' : B[G/X]}
\]
\[ \newFi{
   \inference{\cdot |- A:* & \Delta, i^A;\Gamma |- t=t' : B}{
              \Delta;\Gamma |- t=t' : \forall i^A.B}
   \left(\begin{smallmatrix}
		i\notin\FV(t),\\
		i\notin\FV(t'),\\
		i\notin\FV(\Gamma)\end{smallmatrix}\right)
 ~~~~
   \inference{ \Delta;\Gamma |- t=t' : \forall i^A.B
             & \Delta;\cdot |- s:A}{
              \Delta;\Gamma |- t=t' : B[s/i]} }
\]
\[ \inference{\Delta;\Gamma |- t=t':A}{\Delta;\Gamma |- t'=t:A}
 ~~~~
   \inference{\Delta;\Gamma |- t=t':A & \Delta;\Gamma |- t'=t'':A}{
              \Delta;\Gamma |- t=t'':A}
\]
\end{singlespace}
\caption{Equality rules of \Fixi}
\label{fig:eqFixi}
\end{figure}

The grey-boxed extensions for term-indexing are essentially the same as
those grey-boxed extensions in System \Fi\ (\S\ref{sec:fi:fi}). So, we will
only focus our description on the dashed-box extensions regarding polarities
(\S\ref{ssec:fixi:def:polarity}) and equi-recursive types
(\S\ref{ssec:fixi:def:equirec}).

\subsection{Polarities} \label{ssec:fixi:def:polarity}
Polarities track how type constructor variables are used.
A polarity ($p$) is either covariant ($+$), contravariant ($-$), or
avariant ($0$). When a type variable is bound, its polarity is 
made explicit both at its binding site, and in the context.
The avariant polarity ($0$) means a variable can be used both
covariantly and contravariantly\footnote{the notation ``invariant'' is sometimes used
	(see \cite{AbeMat04}), but we think this notation is quite misleading,
	since $0$ means that the system \emph{does not care} about that
	variable's polarity,
	rather than maintaining some unchanging set of properties about the variable's polarities.}
We prefix a kind  by a polarity (\ie, $p\kappa$) to specify the variable's
kind and polarity. For example,
\[
\begin{array}{rlrl}
X_1^{+*},X_2^{-*} |- \!\!\! & X_2 -> X_1 : *
	& ~~~\text{justifies} & \l X_1^{+*}.\l X_2^{-*}.X_2 -> X_1 \\
X_1^{0*},X_2^{0*} |- \!\!\! & X_2 -> X_1 : *
	& ~~~\text{also justifies} & \l X_1^{0*}.\l X_2^{0*}.X_2 -> X_1 \\
X^{0*} |- \!\!\! & X\phantom{_.} -> X\phantom{_2} : *
	& ~~~\text{justifies} & \l X^{0*}.X -> X
\end{array}
\]
Note that we can replace $+$ and $-$ in the first example with $0$
as in the second example, since the variables of avariant polarity can be used
in any position, that is both in covariant position and contravariant position.
In the third example, the polarity of $X$ can be neither $+$ nor $-$, but must
must be $0$, since $X$ appears in both covariant and contravariant positions.

\paragraph{Syntax using polarized kinds:}
The kind syntax are polarized. That is, the domain kind ($\kappa$) of
an arrow kind ($p\kappa -> \kappa'$) must be prefixed by its polarity ($p$).
A type abstractions ($\l X^{p\kappa}.F$) in the type syntax are annotated by
polarity-prefixed kinds ($p\kappa$). Type constructor variables ($X$) bound
in the type-level contexts ($\Delta$) are annotated by polarity-prefixed kinds
($p\kappa$), likewise. Note the syntax for extending the type-level context
$\Delta,X^{p\kappa}$ in Figure~\ref{fig:Fixi}. The kinding rule ($\lambda$)
make use of all these three uses of polarized kinds -- in type abstractions,
in kind arrows, and in type-level contexts.

\paragraph{Polarity operation on type-level context ($p\Delta$):}
The kinding judgment $\Delta |- F : \kappa$ assumes that $F$ is in covariant
positions. This is why the ($Var$) rule requires the polarity of $X$ to be
either $+$ or $0$, but not $-$. To judge well-kindedness of type constructors
in contravariant positions (\eg, $A$ in $A -> B$), we should invert
the polarities of all the type constructor variables in the context.
This idea of inverting polarities in the context is captured by $-\Delta$
operation in the kinding rule ($->$). More generally, the well-kindedness $F$
expected to be used as $p$-polarity can be determined by the judgement
$p\Delta |- F : \kappa$, where $p\Delta$ operation is defined as:
\[
\begin{array}{lcl}
p~\cdot &=& \cdot \\
p(\Delta,X^{p'\kappa}) &=& p\Delta,X^{(pp')\kappa} \\
p(\Delta,i^A) &=& p\Delta,i^A
\end{array}
\quad\text{where $pp'$ is the usual sign product}~~
\begin{smallmatrix}
+ p' & = & p' \\
- +  & = & -  \\
- -  & = & +  \\
- 0  & = & 0  \\
0 p' & = & 0
\end{smallmatrix}
\]
Note the use of $p\Delta$ operation in the kinding rule ($@$) in order to
determine the well-kindedness of $G$ expected to be used as $p$-polarity
by the type constructor $F : p\kappa -> \kappa'$ being applied to $G$.

\paragraph{Where polarities are irrelevant (or, avariant):}
Polarities are irrelevant (or, avariant) for universally quantified variables,
indices, and in the typing rules. This is because the sole purpose of tracking 
polarities in \Fixi\ is to make sure that we only take the equi-recursive
fixpoint over covariant type constructors, as in the kinding rule ($\mu$).
Note that we can only take fixpoints over type constructors of covariant
arrow kinds whose domain and codomain coincides ($+\kappa -> \kappa$).
We can never take fixpoints over forall types (or, universal quantification)
and type constructor expecting an index because they are not of arrow kinds.
Forall types are always of kind $*$ and type constructors expecting an index
are of arrow kinds ($A -> \kappa$). So, we give universally quantified variables
avariant polarity ($X^{0\kappa}$ in the ($\forall$) rule) and nullify polarities
when type checking indices ($0\Delta$ in the ($@i$) rule). For similar reasons,
we assume that type-level contexts are nullified in the typing rules;
note $0\Delta$ in the well-formedness condition for $\Delta |- \Gamma$
in Figure~\ref{fig:Fixi2}. That is, we always type check under nullified
type-level context for all terms in general, as well as indices appearing
in type applications. As a result, the typing rules of \Fixi\ have no dashed-box
extensions except for $X^{0\kappa}$ in the generalization rule ($\forall$$I$)
where we introduce a universally quantified type constructor variable.

\subsection{The Equi-recursive type operator $\fix$}
\label{ssec:fixi:def:equirec}
\Fixi\ provides the equi-recursive type operator $\fix$.
The kinding rule ($\fix$) in Figure~\ref{fig:Fixi2} is similar to
the ($\mu$) rule of System \Fi\ (see Figure~\ref{fig:Fi2} in \S\ref{sec:fi:fi}),
but requires base structure $F$ to be covariant (or, positive), \ie,
$F : +\kappa -> \kappa$. This restriction on the polarity of $F$ is due to
the equi-recursive nature of $\fix$, \ie, $\fix F=F(\fix F)$, which is
described by the first type constructor equality rule inside a dashed box
in Figure~\ref{fig:eqFixi}. Recall that adding equi-recursive types
without restricting the polarity of base structure breaks strong normalization.
\KYA{TODO should reference intro chapter section which is not written yet}

Note that there are no explicit term syntax that guides the conversion between
$\fix F$ and $F(\fix F)$, unlike in iso-recursive systems where $\In$ and
$\mathsf{unIn}$ are term syntax that explicitly guide rolling
(from $\mu F$ to $F(\mu F)$) and unrolling (from $F(\mu F)$ to $\mu F$).
Since $\fix F=F(\fix F)$ is given definitionally (\ie, by the equality rule
definition), the type constructor conversion rule ($Conv$) can silently
roll (from $F(\fix F)$ to $\fix F$) and unroll (from $F(\fix F)$ to $\fix F$)
the recursive types, just as it can silently $\beta$-convert type constructors.

In the following section, we will review how iso-recursive type operator
$\mu_\kappa$ and its $\In_\kappa$ constructor, which is well-behaved
(\ie, strongly normalize) for base structures of arbitrary polarity,
can be embedded into \Fixi\ being defined in terms of the equi-recursive
type operator $\fix$, which is only well-behaved for covariant base structures.

 %% \label{sec:fixi:def}

\section{Embedding datatypes and primitive recursion}
\label{sec:fixi:data}
Embedding for primitive recursion over datatypes of arbitrary polarities into
\Fixi\ was discovered by \citet{AbeMat04}. We review these embeddings
in the context of \Fixi.

The embeddings of non-recursive datatypes in Figure~\ref{fig:fixiNonRecData}
are exactly the same as in \Fi\ (see \S\ref{sec:fi:data}), other than tracking
polarities of the type constructor variables. That is, we use the usual
impredicative encodings for non-recursive datatypes such as void, unit, pairs,
sums, and existential types. The examples in Figure \ref{fig:fixiNonRecData}
are mostly from \citet{AbeMat04}, except for the last example of $\exists_A$,
an existential type over term-indices of type $A$.
\begin{figure}
\begin{singlespace}
\begin{align*}
\bot &~\triangleq~ \forall X^{*}.X
	&:\;& *\\
\textsf{Unit} &~\triangleq~ \forall X.\lambda X^{0*}. X
	&:\;& * \\
\times &~\triangleq~
	\l X_1^{+*}.\l X_2^{+*}.\forall X^{*}.(X_1 -> X_2 -> X) -> X
	&:\;& +* -> +* -> * \\
+ &~\triangleq~
	\lambda X_1^{+*}.\l X_2^{+*}.\forall X^{*}.(X_1 -> X) -> (X_2 -> X) -> X
	&:\;& +* -> +* -> * \\
\exists_\kappa &~\triangleq~
	\l X_{\!F}^{0\kappa -> *}.\forall X^{*}.
		(\forall X_1^\kappa.X_{\!F}\,X_1 -> X) -> X
	&:\;& +(0\kappa -> *) -> * \\
\exists_A &~\triangleq~
	\l X_{\!F}^{A -> *}.\forall X^{*}.
	(\forall i^A.X_{\!F}\{i\} -> X) -> X
	&:\;& +(A -> *) -> *
\end{align*}
\caption{Embeddings of some well-known non-recursive datatypes in \Fixi}
\label{fig:fixiNonRecData}
\end{singlespace}
\end{figure}

Embedding recursive datatypes and their Mendler-style primitive recursion
amounts to embedding $\mu_\kappa$, $\In_\kappa$, and $\MPr_\kappa$ described
in \S\ref{sec:mpr}. Figure~\ref{fig:embedMPr} illustrates the embeddings
discovered by \citet{AbeMat04}, reformatted using our conventions (see
Figure~\ref{fig:mu} in \S\ref{ssec:embedTwoLevel}) and taking term-indices
into consideration. To confirm the correctness of these embeddings,
we should check that (1) the embeddings are well-kinded and well-typed
and that (2) the primitive recursion behaves well
(\ie, $\MPr_\kappa\;s\;(\In_\kappa) -->+ s\;\textit{id}\;(\MPr_\kappa\;s)\;t$).
From the term encodings of $\MPr_\kappa$ and $\In_\kappa$, it is obvious that
the reduction of primitive recursion behaves well. Thus, we only need to check
that $\mu_\kappa$ is well-kinded and $\MPr_\kappa$ and $\In_\kappa$ are
well-typed.
\afterpage{ %%%%%%%%%%%%%%%%%%%%%%% begin afterpage
\begin{landscape}
\begin{figure}
\begin{singlespace}
\begin{multline*} \text{notation:}\quad
   \boldsymbol{\l}\mathbb{I}^\kappa.F =
	\lambda I_1^{K_1}.\cdots.\lambda I_n^{K_n}.F \qquad
   \boldsymbol{\forall}\mathbb{I}^\kappa.B =
	\forall I_1^{K_1}.\cdots.\forall I_n^{K_n}.B \qquad
   F\mathbb{I} = F I_1 \cdots I_n \qquad
   F \stackrel{\kappa}{\pmb{\pmb{->}}} G =
	\boldsymbol{\forall}\mathbb{I}^\kappa.F\mathbb{I} -> G\mathbb{I} \\
\begin{array}{lll}
\text{where}
 	& \kappa = K_1 -> \cdots -> K_n -> * & \text{and} ~~~
 	\text{$I_i$ is an index variable ($i_i$) when $K_i$ is a type,}
 		\\
 	& \mathbb{I}\,=I_1,\;\dots\;\dots\;,\;I_n& \qquad~\qquad
		\text{a type constructor variable ($X_i$) otherwise
			(\ie, $K_n=p_i\kappa_i$).}
\end{array}
\end{multline*} ~ \vspace*{-5pt}
\hrule  \vspace*{-2pt}
\begin{align*}
\mu_\kappa &\;:~ 0(0\kappa -> \kappa) -> \kappa \\
\mu_\kappa &\triangleq
\l X_{\!F}^{0(0\kappa -> \kappa)}.\fix(\Phi_\kappa\,X_{\!F})\\
\Phi_\kappa &\;:~ 0(0\kappa -> \kappa) -> +\kappa -> \kappa \\
\Phi_\kappa &\triangleq \l X_{\!F}^{0(0\kappa -> \kappa)}.
\l X_c^{+\kappa}.\boldsymbol{\l}\mathbb{I}^\kappa.
\forall X^\kappa.
(\forall X_r^\kappa. (X_r \karrow{\kappa} X_c)
		-> (X_r \karrow{\kappa} X)
		-> (X_{\!F}\,X_r \karrow{\kappa} X) ) -> X\,\mathbb{I}\\
~\\
\MPr_\kappa &\;:~
	\forall X_{\!F}^{0(0\kappa-> \kappa)}.\forall X^\kappa.
	(\forall {X_r}^{\!\!\kappa}.
	 (X_r \karrow{\kappa} \mu_\kappa X_{\!F}) ->
	 (X_r \karrow{\kappa} X) ->
	 (X_{\!F}\,X_r \karrow{\kappa} X) ) ->
	 (\mu_\kappa X_{\!F} \karrow{\kappa} X) \\
\MPr_\kappa &\triangleq \l s.\l r.r\;s \\
~\\
\In_\kappa &\;:~ \forall X_{\!F}^{0(0\kappa-> \kappa)}.
		X_{\!F}(\mu_\kappa X_{\!F}) \karrow{\kappa} \mu_\kappa X_{\!F}\\
\In_\kappa &\triangleq \l t.\l s.s\;\textit{id}\;(\MPr_\kappa\;s)\;t \\
\textit{id} &\triangleq \l x.x
\end{align*}
\end{singlespace}
\caption{Embedding of the recursive type operators ($\mu_\kappa$),
	their data constructors ($\In_\kappa$),
	and the Mendler-style primitive recursors ($\MPr_\kappa$) in \Fixi.}
\label{fig:embedMPr}
\end{figure}

\begin{figure}
\begin{singlespace}
\[
\text{The type of $r$ can be expanded by the defintion of $\fix$
	and the equi-recursive equality rule on $\fix$ as follows:}\]
	\vskip-6ex
\begin{multline*}\,
\mu_\kappa X_{\!F}\,\mathbb{I} = \fix(\Phi_\kappa X_{\!F})\mathbb{I}
= \Phi_\kappa X_{\!F}(\fix(\Phi_\kappa X_{\!F}))\mathbb{I}
= \Phi_\kappa X_{\!F}(\mu_\kappa X_{\!F}) \mathbb{I} \\
= \forall X^\kappa.\underbrace{
	(\forall X_r^\kappa.
		(X_r \karrow{\kappa} \mu_\kappa X_{\!F}) ->
		(X_r \karrow{\kappa} X) ->
		(X_{\!F}\,X_r \karrow{\kappa} X) )}_\text{exactly matches with
						the type of {\,\small$s$}}
	-> X\,\mathbb{I} \,
\end{multline*}
\[
\inference{
	\inference{
	X_{\!F}^{0(0\kappa-> \kappa)}, X^{0\kappa},
	\mathbb{I}^\kappa
	; \;
	s: (\forall {X_r}^{\!\!\kappa}.
	 (X_r \karrow{\kappa} \mu_\kappa X_{\!F}) ->
	 (X_r \karrow{\kappa} X) ->
	 (X_{\!F}\,X_r \karrow{\kappa} X) ),
	r: \mu_\kappa X_{\!F}\,\mathbb{I}
	|- r\;s : X\,\mathbb{I}
	}{
	X_{\!F}^{0(0\kappa-> \kappa)}, X^{0\kappa} ; \;
	s: (\forall {X_r}^{\!\!\kappa}.
	 (X_r \karrow{\kappa} \mu_\kappa X_{\!F}) ->
	 (X_r \karrow{\kappa} X) ->
	 (X_{\!F}\,X_r \karrow{\kappa} X) )
	|- \l s.r\;s : \mu_\kappa X_{\!F} \karrow{\kappa} X
	}
}{
	\cdot;\cdot |- \l s.\l r.r\;s :
	\forall X_{\!F}^{0\kappa-> \kappa}.\forall X^\kappa.
	(\forall {X_r}^{\!\!\kappa}.
	 (X_r \karrow{\kappa} \mu_\kappa X_{\!F}) ->
	 (X_r \karrow{\kappa} X) ->
	 (X_{\!F}\,X_r \karrow{\kappa} X) ) ->
	 (\mu_\kappa X_{\!F} \karrow{\kappa} X) 
}
\]
\end{singlespace} \vskip-1.5ex
\caption{Well-typedness of the $\MPr$ embedding in \Fixi}
\label{fig:embedMPrJustify}
\end{figure}

\begin{figure}
\begin{singlespace}~\qquad
Let ~ $\Delta =
	X_{\!F}^{0(0\kappa-> \kappa)}, \mathbb{I}^\kappa,
	X^{0\kappa}$ ~ and ~
$\Gamma =
	t: X_{\!F}(\mu_\kappa X_{\!F})\mathbb{I},
	s: (\forall X_r^\kappa.
		(X_r \karrow{\kappa} \mu_\kappa X_{\!F}) ->
		(X_r \karrow{\kappa} X) ->
		(X_{\!F}\,X_r \karrow{\kappa} X) )$.
\[
\inference{
	\inference{
		\inference{
			\inference{
				{\begin{array}{llll}
			\Delta;\Gamma |- s \;:
			(\mu_\kappa X_{\!F} \karrow{\kappa} \mu_\kappa X_{\!F})
			~ -> &
			(\mu_\kappa X_{\!F} \karrow{\kappa} X)
			~ -> &
			(X_{\!F}(\mu_\kappa X_{\!F}) \karrow{\kappa} X)
			& \qquad \text{\small(by instantiating $X_r$
						with $\mu_\kappa X_{\!F})$}
				\\
			\Delta;\Gamma |- \textit{id} :
			(\mu_\kappa X_{\!F} \karrow{\kappa} \mu_\kappa X_{\!F})
				\\
			\Delta;\Gamma |- (\MPr_\kappa\; s) ~~ : &
			(\mu_\kappa X_{\!F} \karrow{\kappa} X)
				\\
			\Delta;\Gamma |- t \qquad\qquad\, : & &
			X_{\!F}(\mu_\kappa X_{\!F})\mathbb{I}
				\end{array}}
			}{
	X_{\!F}^{0(0\kappa-> \kappa)}, \mathbb{I}^\kappa,
	X^{0\kappa}
	; \;
	t: X_{\!F}(\mu_\kappa X_{\!F})\mathbb{I},
	s: (\forall X_r^\kappa.
		(X_r \karrow{\kappa} \mu_\kappa X_{\!F}) ->
		(X_r \karrow{\kappa} X) ->
		(X_{\!F}\,X_r \karrow{\kappa} X) )
	|- s\;\textit{id}\;(\MPr_\kappa\;s)\;t : X\,\mathbb{I}
			}
		}{
	X_{\!F}^{0(0\kappa-> \kappa)}, \mathbb{I}^\kappa ; \;
	t: X_{\!F}(\mu_\kappa X_{\!F})\mathbb{I}
	|- \l s.s\;\textit{id}\;(\MPr_\kappa\;s)\;t :
	\forall X^\kappa.
	(\forall X_r^\kappa.
		(X_r \karrow{\kappa} \mu_\kappa X_{\!F}) ->
		(X_r \karrow{\kappa} X) ->
		(X_{\!F}\,X_r \karrow{\kappa} X) ) -> X\,\mathbb{I}
		}
	}{
	X_{\!F}^{0(0\kappa-> \kappa)}, \mathbb{I}^\kappa ; \;
	t: X_{\!F}(\mu_\kappa X_{\!F})\mathbb{I}
	|- \l s.s\;\textit{id}\;(\MPr_\kappa\;s)\;t :
		\mu_\kappa X_{\!F}\,\mathbb{I} \quad
		\qquad \text{\small(We can expand the type into above
				as in Figure \ref{fig:embedMPrJustify})}
	}
}{
	\cdot;\cdot |- \l t.\l s.s\;\textit{id}\;(\MPr_\kappa\;s)\;t :
	\forall X_{\!F}^{\kappa-> \kappa}.
		X_{\!F}(\mu_\kappa X_{\!F}) \karrow{\kappa} \mu_\kappa X_{\!F}
}
\]
\end{singlespace} \vskip-2.5ex
\caption{Well-typedness of the $\In$ embedding in \Fixi}
\label{fig:embedInJustify}
\end{figure}

\end{landscape}
} %%%%%%%%%%%%%%%%%%%%%%% end of afterpage

Note that the polarities appearing in the embedding of $\mu_\kappa$ are all
$0$. The embedding from a non-polarize kind $\kappa$ into
a polarize kinds $\ulcorner\kappa\urcorner$ can be defined as:
\[ \ulcorner * \urcorner = * \qquad
\ulcorner \kappa_1 -> \kappa_2 \urcorner =
0\ulcorner\kappa_1\urcorner -> \ulcorner\kappa_2\urcorner \qquad
\ulcorner A -> \kappa \urcorner = A -> \ulcorner \kappa \urcorner.
\]

It is easy to see that the embedding of the non-polarized recursive
type operator $\mu_\kappa : 0(0\kappa -> \kappa) -> \kappa$
is well-kinded, provided that
$\Phi_\kappa : 0(0\kappa -> \kappa) -> +\kappa -> \kappa$
is well kinded. Note that $\Phi_\kappa$ turns an avariant type constructor
($0\kappa -> \kappa$) into a positive type constructor
($+\kappa -> \kappa$). From the definition of $\Phi_\kappa$, we only need 
to check that $(X_r \karrow{\kappa} X_c)$, $(X_r \karrow{\kappa} X)$,
$X_F\,X_r \karrow{\kappa} X$ and $X\;\mathbb{I}$ are of kind $*$
under the context $ X_{\!F}^{0(0\kappa -> \kappa)},
		X_c^{+\kappa}, \mathbb{I}^\kappa, X^{0\kappa}$,
which is not difficult to see.

Well-typedness of $\MPr_\kappa$ and $\In_\kappa$ are justified in
Figures~\ref{fig:embedMPrJustify} and \ref{fig:embedInJustify}


 %% \label{sec:fixi:data}

\section{Embedding course-of-values primitive recursion}
\label{sec:fixi:cv}

\lstset{language=Haskell,
	basicstyle=\ttfamily\small,
%	keywordstyle=\color{ta4chameleon},
%	emph={List,Int,Bool},
	commentstyle=\color{grey},
	literate =
		{forall}{{$\forall$}}1
%		{|}{{$\mid\;\,$}}1
%		{=}{{\textcolor{ta3chocolate}{$=\,\;$}}}1
		{::}{{$:\!\,:$}}1
		{->}{{$\to$}}1
		{=>}{{$\Rightarrow$}}1
		{\\}{{$\lambda$}}1
	}

\afterpage{ %%%%%%%%%%%%%%%%%%%%%%% begin afterpage
\begin{landscape}
\begin{figure}
\begin{singlespace}
\begin{align*}
\mu^{+}_\kappa &\;:~ 0(+\kappa -> \kappa) -> \kappa \\
\mu^{+}_\kappa &\triangleq
\l X_{\!F}^{0(+\kappa -> \kappa)}.\fix(\Phi^{+}_\kappa\,X_{\!F})\\
\Phi^{+}_\kappa &\;:~ 0(+\kappa -> \kappa) -> +\kappa -> \kappa \\
\Phi^{+}_\kappa &\triangleq \l X_{\!F}^{0(+\kappa -> \kappa)}.
\l X_c^{+\kappa}.\boldsymbol{\l}\mathbb{I}^\kappa.
\forall X^\kappa.
(\forall X_r^\kappa. (X_r \karrow{\kappa} X_{\!F}\,X_r)
		-> (X_r \karrow{\kappa} X_c)
		-> (X_r \karrow{\kappa} X)
		-> (X_{\!F}\,X_r \karrow{\kappa} X) ) -> X\,\mathbb{I}\\
~\\
\McvPr_\kappa &\;:~
	\forall X_{\!F}^{+\kappa-> \kappa}.\forall X^\kappa.
	(\forall {X_r}^{\!\!\kappa}.
	 (X_r \karrow{\kappa} X_{\!F}\,X_r) ->
	 (X_r \karrow{\kappa} \mu^{+}_\kappa X_{\!F}) ->
	 (X_r \karrow{\kappa} X) ->
	 (X_{\!F}\,X_r \karrow{\kappa} X) ) ->
	 (\mu^{+}_\kappa X_{\!F} \karrow{\kappa} X) \\
\McvPr_\kappa &\triangleq \l s.\l r.r\;s\\
~\\
\In_F &\;:~ F(\mu^{+}_\kappa F) \karrow{\kappa} \mu^{+}_\kappa F\\
\In_F &\triangleq
	\l t. \l s. s~\unIn_F\;\textit{id}\;\,(\McvPr_\kappa\;s)\;\,t \\
\end{align*}\vskip -2.5ex
Provided that there exists
$\unIn_{F} : \mu^{+}_\kappa F \karrow{\kappa} F(\mu^{+}_\kappa F)$
for the base structure $F:+\kappa -> \kappa$, 
such that $\unIn_F(\In_F\;t) -->+ t$
where the reduction is constant regardless of $t$
(steps may vary between each base structure $F$ though).
\end{singlespace} \vskip-3.5ex
\[\text{See Figure \ref{fig:unInExamples} for
embeddings of unrollers ($\unIn_F$) for
some well-known positive base structures ($F$).}
\]
\caption{Embedding of the recursive type operators ($\mu^{+}_\kappa$),
	the Mendler-style course-of-values primitive recursors
	($\McvPr_\kappa$), and the roller ($\In_F$) in \Fixi,
        provided that the embedding of $\unIn_F$ exists}
\label{fig:embedMcvPr}
\end{figure}

\begin{figure}
\[\!\!\!\!\!\!\!
\begin{array}{llcll}
	& \text{\textbf{Regular datatypes}} \\
N &\!\!\!\triangleq \l X^{+*}.X + \textsf{Unit} &\qquad&
\unIn_N &\!\!\!\triangleq \McvPr_{*} (\l\_.\l\textit{cast}.\l\_.
	\l x. x\;(\texttt{InL}\circ\textit{cast})\;\texttt{InR})
	\\
L &\!\!\!\triangleq \l X_a^{+*}.\l X^{+*}.(X_a\times X) + \textsf{Unit} &&
\unIn_{(LA)} &\!\!\!\triangleq \McvPr_{*} (\l\_.\l\textit{cast}.\l\_.
	\l x. x\;(\texttt{InL}\circ(\textit{id}\times cast))\;\texttt{InR})
	\\
R &\!\!\!\triangleq \l X_a^{+*}.\l X^{+*}.(X_a\times \texttt{List}\,X) -> X &&
\unIn_{(RA)} &\!\!\!\triangleq \McvPr_{*} (\l\_.\l\textit{cast}.\l\_.
	\l x. x\;(\textit{id}\times \textit{fmap}_\texttt{List}\;cast) )
	\\
& \text{\textbf{Type-indexed datatypes}} \phantom{G^{G^{G^{G^{G^G}}}}}\\
P &\!\!\!\triangleq \l X^{+* -> *}.\l X_a^{+*}.
	X_a \times X(X_a \times X_a) + \textsf{Unit} &&
\unIn_P &\!\!\!\triangleq \McvPr_{+* -> *} (\l\_.\l\textit{cast}.\l\_.
	\l x. x \;(\texttt{InL}\circ(\textit{id}\times\textit{cast}))
		\;\texttt{InR})
	\\
B &\!\!\!\triangleq \l X^{+* -> *}.\l X_a^{+*}.
	X_a \times X(X\,X_a) + \textsf{Unit} &&
\unIn_B &\!\!\!\triangleq \McvPr_{+* -> *} (\l\_.\l\textit{cast}.\l\_.
 	\l x. x \;(\texttt{InL}\circ
 		(\textit{id}\times
 			(\textit{cast}\circ\textit{fmap}\;\textit{cast})))
 		\;\texttt{InR})
	\\
	& \text{\textbf{Term-indexed datatypes}} \phantom{G^{G^{G^{G^{G^G}}}}}
\end{array}
\]\vskip-4.5ex
\[
V \triangleq \l X_a^{*}.\l X^{\texttt{Nat} -> *}.\l i^{\texttt{Nat}}.
(\exists j^\texttt{Nat}.((i=\texttt{succ}\,j) \times X_a \times X\{j\})) +
(i=\texttt{zero})
\]
\[
\begin{array}{lll}
\texttt{VCons} &\!\!\!\triangleq \l x_a.\l x.
	\texttt{InL}(\mathtt{Ex_{Nat}}(\mathtt{Eq_{\,Nat}},x_a,x))
& : \;
\forall X_a^{*}. \forall X^{\texttt{Nat} -> *}. \forall i^\texttt{Nat}.
	X_a -> X\,\{i\} -> V\,X_a\,X\,\{\texttt{succ}\,i\}
	\\
\texttt{VNil} &\!\!\!\triangleq \texttt{InR}~\mathtt{Eq_{\,Nat}}
& : \;
\forall X_a^{*}. \forall X^{\texttt{Nat} -> *}. V\,X_a\,X\,\{\texttt{zero}\}
\end{array}
\]
\[
\unIn_{(V\,A)} \triangleq \McvPr_{\texttt{Nat} -> *}(\l\_.\l\textit{cast}.\l\_.
\l x. x \;(\texttt{InL}\circ
		(\textit{id}\times\textit{id}\times\textit{cast}))
	\;\texttt{InR})
\]
The notation $\exists j^A.B$ is a shorthand for $\exists_A(\l j^A.B)$
where $\exists_A$ is defined in Figure~\ref{fig:fixiNonRecData}.\\
$\mathtt{Ex_{A}} : \forall F^{A -> *}.\exists_A F$ and
$\mathtt{Eq_{A}} : \forall i^A.\forall j^A.(i=j)$ are
the data constructors of the existential type and the equality type.
\[\text{
See Figure \ref{fig:embedMcvPr} for the embedding of the Mendler-style
course-of-values primitive recursor ($\McvPr_\kappa$)}
\]
\caption{Embeddings of unroller ($\unIn_F$)
	for some well-known positive base structures ($F$).}
\label{fig:unInExamples}
\end{figure}

\end{landscape}
} %%%%%%%%%%%%%%%%%%%%%%% end of afterpage

%%%%%%%%%%%% TODO move this to somewhere else
To add a new Mendler-style combinator, we need to address several issues:
\begin{itemize}
\item First, we need to add an appropriate type-level fixpoint operator
	(\eg, $\mu_\kappa$ for primitive recursion)
that is used to build recursive types. This type-level operation needs to
capture not only the tying of the recursive knot, but also the compatible
structure needed to encode the new Mendler operation.
\item Second, we need to specify the behavior of the new Mendler operation.
	This means discovering the characteristic equations
	that the operation should obey
	(\eg, Haskell definition of Mendler-style combinators
	in Chapter~\ref{ch:mendler}).
\item Third, we need to find an embedding that preserves
	the characteristic equations in the host calculus
	(\eg, embedding of $In_\kappa$ and $\MPr_\kappa$).
\item In practice, the second and third steps are intimately entwined as
the equations and embedding are carefully designed
(using a Church-style encoding) to achieve the desired result
(\eg, $In_\kappa$ is defined in terms of
	the Mendler-style combinator $\MPr_\kappa$).
\end{itemize}

To embed Mendler-style course of values recursion, we can follow the steps above
just ad we did for Mendler-style primitive recursion. In addition, we need
to embed an unroller $\unIn_F$, which is the key operation to support
course-of-values recursion. Recall the use of \textit{out} in the Fibonacci
number example (Figure~\ref{fig:fib}) and the Lucas number example
(Figure~\ref{fig:lucas}) in Section~\ref{ssec:tourHist0}.

\subsection{The general form for the embedding of course-of-values recursion}
Figure~\ref{fig:embedMcvPr} and illustrates the embedding of
a new iso-recursive type operator ($\mu^{+}_\kappa$),
the Mendler-style course-of-values primitive recursor ($\McvPr_\kappa$),
and the roller $\In_F$ in \Fixi. Since the embedding of $\In_F$ uses $\unIn_F$,
we also need embed unroller $\unIn_F$ in order to complete the embedding of
the roller $\In_F$. Embeddings of $\unIn_F$ are possible for a fairly large
class of positive base structures. Figure~\ref{fig:unInExamples} illustrates
some of those unrollers. But, it may not be possible to give an embedding
of the unroller for some base structures. Recall that not all base structures
can have well-defined course-of-values recursion that grantee termination
(see Figure~\ref{fig:LoopHisto} in Section~\ref{ssec:tourHist0}).

The embeddings of $\mu^{+}_\kappa$, $\McvPr_\kappa$, and $\In_F$
for course-of-values recursion (see Figure~\ref{fig:embedMcvPr}) are
very similar to the embeddings of $\mu_\kappa$, $\MPr_\kappa$, and $\In_\kappa$
for primitive recursion (see Figure~\ref{fig:embedMPr}), which we discussed
earlier in the previous section. The definition of
$\McvPr_\kappa \triangleq \l s.\l r.r\,s$ is exactly the same as
the definition of $\MPr_\kappa$, but only differs in its type signature.
The definition of $\In_F$ is similar to $\In_\kappa$ but uses the additional
$\unIn_F$, which implements the abstract unroller. So, the last piece
of the puzzle for embedding Mendler-style course-of-values recursion is
the embedding of $\unIn_F$.

In the following subsections, we will elaborate on how to embed unrollers
($\unIn_F$) through examples (\S\ref{sec:fixi:cv:unInExamples}), derive
uniform embeddings of the unrollers generalizing from those examples, and
discuss whether the embeddings of unrollers satisfy their desired properties.

\subsection{Embedding unrollers}
\label{sec:fixi:cv:unInExamples}
Embeddings of unrollers for some well-known positive datatypes
are illustrated in Figure~\ref{fig:unInExamples}. The general idea is to
use $\McvPr_\kappa$ to define $\unIn_F$ for the base structure
$F:+\kappa -> \kappa$ without using the abstract recursive call operation
but only using the abstract cast operation in order be define constant
time unrollers. To define an unroller, we map non-recursive
components ($X_a$) as they are using \textit{id} and map abstract recursive
components ($X_r$) to concrete recursive components
($\mu^{+}_\kappa F$) using the abstract \textit{cast} operation provided
by the $\McvPr_\kappa$ combinator. We can embed unrollers
for regular datatypes such as natural numbers (the base $N$) and lists
(the base $L$), type-indexed datatypes such as powerlists (the base $P$),
and term-indexed datatypes such as vectors (the base $V$) in this way.
The notation for combining functions for tuples are defined as
$(f \times g) \triangleq \l x.(f\,x,g\,x)$, and $(f\times g\times h)$
are defined similarly for triples.

\paragraph{Embeding unrollers for regular datatypes:}
The embeddings of $\unIn_N$ and $\unIn_{(L A)}$ are self explanatory.
The embeddings of $\unIn_{(R A)}$ relies on the map function for lists,
since the rose tree is indirect datatype where recursive subcomponents
appear inside the list ($\texttt{List}\,X_r$).
The $\textit{fmap}_\texttt{List}$ function applies \textit{cast} to each of
the the abstract recursive subcomponents of type $X_r$ inside $F\;X_r$
values into a concrete recursive type $\mu_{*}^{+}F$ in order to obtain
$F(\mu_{*}^{+}F)$ values.

\paragraph{Embeding unrollers for nested datatypes} are no more complicated
than embedding unrollers for regular datatypes. For instance, the embedding
of $\unIn_P$ for powerlists is almost identical to the embedding of $\unIn_L$
for regular lists except the use of $\McvPr_{* -> *}$ instead of $\McvPr_{*}$.
This is because the cast operation provided by $\McvPr_{* -> *}$ is polymorphic
over the type index:
$\textit{cast}:\forall X_i. X_r X_i -> \mu^{+}_{* -> *} X_F X_i$.
Since unrollers preserve indices, there is no extra work to be done
other than applying the \textit{cast}. In the embeding of $\unIn_P$,
the $\texttt{cast}$ function performs an index preserving cast from
an abstract recursive type $X_r (X_a\times X_a)$ to the concrete powerlist type 
$\mu^{+}P (X_a\times X_a)$.

Embedding unrollers for \emph{truly nested datatypes} \cite{AbeMatUus05},
such as bushes, are similar to embedding unrollers for indirect regular
recursive types. Truly nested datatypes are recursive datatypes whose
indices may involve themselves. Truly nested datatypes are similar to
indirect recursive types in the sense that a bunch of recursive components
are contained in certain data structures -- in case of truly nested datatypes
those data structures are exactly the nested datatypes themselves.
Assuming that the nested datatype has a notion of monotone map, we can
use \textit{fmap} to push down the \textit{cast} to the inner structure
and then \textit{cast} the outer structure. Note the use of
$(\textit{cast} \circ \textit{fmap}\;\textit{cast})$ in the embedding of
the unroller $\unIn_R$ for bushes.

\paragraph{Embeding unrollers for indexed datatypes} are no more complicated
than embedding unrollers for regular datatypes, as well. To embed unrollers
for term-indexed datatypes, we would often need existential types
(Figure~\ref{fig:fixiNonRecData}) and equality types. We can encode
equality types in \Fixi\ as a Leibniz equality over indices,
\ie, $(i=j) \triangleq \forall F^{A -> *}.F\{i\} -> F\{j\}$,
as discussed in \S\ref{Leibniz}. These extra encodings for maintaining
term-indices does not terribly complicate the embeddings of unrollers,
since unrollers are index preserving. The embedding of $\unIn_{(V A)}$ for
length indexed lists are almost identical to the embedding of $\unIn_{(L A)}$
for regular lists, except that it has one more \textit{id}. The first
\textit{id} appearing in $(\textit{id}\times\textit{id}\times\textit{cast})$ is
for the index equality, to leave the index equality unchanged.

\paragraph{}
Giving different definitions of $\unIn_F$ for each different $F$ as illustrated
in Figure~\ref{fig:unInExamples} looks too ad-hoc. So, we will discuss how to
generalize the embeddings of the $\unIn$ operations, assuming that $F$ has
a notion of a monotone map (\eg, \textit{fmap} for $F:+* ->*$) in the following
subsection. Later on, in Section~\ref{ssec:fixi:theory:cv}, we will
reason about what are the conditions for $F$ to have a notion of a monotone map.

\begin{figure}
\begin{singlespace}
\begin{lstlisting}
newtype Mu0 f = In0 { unIn0 :: f(Mu0 f) }

mcvpr0 :: Functor f => (forall r. (r -> f r) ->
                           (r -> Mu0 f) ->
                           (r -> a) ->
                           (f r -> a) )
       -> Mu0 f -> a
mcvpr0 phi = phi unIn0 id (mcvpr0 phi) . unIn0

newtype Mu1 f i = In1 { unIn1 :: f(Mu1 f)i }

mcvpr1 :: Functor1 f =>
         (forall r i'. Functor r => (forall i. r i -> f r i) ->
                              (forall i. r i -> Mu1 f i) ->
                              (forall i. r i -> a i) ->
                              (f r i' -> a i') )
       -> Mu1 f i -> a i
mcvpr1 phi = phi unIn1 id (mcvpr1 phi) . unIn1

class Functor1 h where
  fmap1'  :: Functor f => (forall i j. (i -> j) -> f i -> g j)
                     -> (a -> b) -> h f a -> h g b
  -- fmap1' h = fmap1 (h id)

  fmap1 :: Functor f => (forall i. f i -> g i)
                     -> (a -> b) -> h f a -> h g b
  fmap1 h = fmap1' (\f -> h . fmap f)

instance (Functor1 h, Functor f) => Functor (h f) where
  fmap f = fmap1 id
        -- fmap1' (\f -> id . fmap f)

instance Functor (f (Mu1 f)) => Functor (Mu1 f) where
  fmap f = In1 . fmap f . unIn1
\end{lstlisting}
\end{singlespace}
\caption{$\mu_{*}$, $\McvPr_{*}$, and $\mu_{* -> *}$, $\McvPr_{* -> *}$
	transcribed into Haskell}
\label{fig:HaskellFunctor1}
\end{figure}

\begin{figure}
\begin{lstlisting}
data N r   = S r   | Z  deriving Functor
type Nat = Mu0 N

unInN :: Mu0 N -> N(Mu0 N)
unInN = mcvpr0 (\ _ cast _ ->  fmap cast)

data L a r = C a r | N  deriving Functor
type List a = Mu0 (L a)

unInL :: Mu0(L a) -> (L a) (Mu0(L a))
unInL = mcvpr0 (\ _ cast _ ->  fmap cast)

data R a r = F a [r]    deriving Functor -- relies on (Functor [])
type Rose a = Mu0 (R a)

unInR :: Mu0(R a) -> (R a) (Mu0(R a))
unInR = mcvpr0 (\ _ cast _ ->  fmap cast)
\end{lstlisting}
\caption{Embeddings of $\unIn_N$, $\unIn_{(L A)}$, $\unIn_{(R A)}$
	transcribed into Haskell}
\label{fig:HaskellunInRegular}
\end{figure}

\begin{figure}
\begin{singlespace}
\begin{lstlisting}
data P r i = PC i (r (i,i)) | PN
type Powl i = Mu1 P i

instance Functor1 P where
  fmap1' h f (PC x a) = PC (f x) (h (\(i,j) -> (f i,f j)) a)
  fmap1' _ _ PN = PN

unInP :: Mu1 P i -> P(Mu1 P) i
unInP = mcvpr1 (\ _ cast _ -> fmap1 cast id)
  -- mcvpr1 phi where
  --   phi _ cast _ (PC x xs) = PC x (cast xs)
  --   phi _ cast _ PN = PN

data B r i = BC i (r (r i)) | BN
type Bush i = Mu1 B i

instance Functor1 B where
  fmap1' h f (BC x a) = BC (f x) (h (h f) a)
  fmap1' _ _ BN = BN

unInB :: Mu1 B i -> B (Mu1 B) i
unInB = mcvpr1 (\ _ cast _ -> fmap1 cast id)
  -- mcvpr1 phi where
  --   phi _ cast _ (BC x xs) = BC x (cast (fmap cast xs))
  --   phi _ cast _ BN = BN
\end{lstlisting}
\end{singlespace}
\caption{Embedding of $\unIn_P$ and $\unIn_B$ transcribed into Haskell}
\label{fig:HaskellunInNested}
\end{figure}

\begin{figure}
\begin{singlespace}
\begin{lstlisting}
class FunctorI1 (h :: (* -> *) -> * -> *) where
  fmapI1 :: (forall i . f i -> g i) -> h f j -> h g j

mcvprI1 :: FunctorI1 f =>
          (forall r j. (forall i. r i -> f r i) ->
                  (forall i. r i -> Mu1 f i) ->
                  (forall i. r i -> a i) ->
                  (f r j -> a j) )
       -> Mu1 f i' -> a i'
mcvprI1 phi = phi unIn1 id (mcvprI1 phi) . unIn1


data Succ n
data Zero

data V a r i where
  VC :: a -> r n -> V a r (Succ n)
  VN :: V a r Zero

instance FunctorI1 (V a) where
  fmapI1 h (VC x a) = VC x (h a)
  fmapI1 _ VN = VN

unInV :: Mu1 (V a) i -> (V a) (Mu1 (V a)) i
unInV = mcvprI1 (\_ cast _ -> fmapI1 cast)


instance FunctorI1 P where
  fmapI1 h (PC x a) = PC x (h a)
  fmapI1 _ PN = PN

unInP' :: Mu1 P a -> P (Mu1 P) a
unInP' = mcvprI1 (\_ cast _ -> fmapI1 cast)
\end{lstlisting}
\end{singlespace}
\caption{Embedding of $\unIn_{(V A)}$ and
	yet another embedding of $\unIn_P$\\ transcribed into Haskell}
\label{fig:HaskellunInIndexed}
\end{figure}

\subsection{Deriving uniform embeddings of the unrollers}
To derive uniform embeddings of the unrollers, we transcribed the embeddings
of the unrollers appearing in Figure~\ref{fig:unInExamples} into Haskell
and observed common patterns among them. These results are summarized
in Figures~\ref{fig:HaskellFunctor1}, \ref{fig:HaskellunInRegular},
\ref{fig:HaskellunInNested}, and \ref{fig:HaskellunInIndexed}.
This excercise of Haskell transcription not only helped us derive
uniform embeddings of the unrollers, but also helped us recognize
the conditions that the base structures should satisfy in order to
have an embedding of its unroller in \Fixi.

Haskell transcriptions of the unroller embeddings for regular datatypes are
given in Figure~\ref{fig:HaskellunInRegular} (Haskell definitions of
\texttt{Mu0} and \texttt{mcvpr0} are given in Figure~\ref{fig:HaskellFunctor1}).
Note that the definitions of these unrollers are uniform:
\lstinline$mvcvp0 (\ _ cast _ ->  fmap cast)$.
This uniform definition relies on the existence of \textit{fmap} over
the base structures -- note the \lstinline$deriving Functor$
in the data declarations. In Section~\ref{ssec:fixi:theory:cv},
we will show that \textit{fmap} exists for any $F:+* -> *$ in \Fixi.
So, we can derive a uniform embedding for the unroller $\unIn_{*}$
for any base $F:+* -> *$ as follows:
\begin{align*}
\unIn_{*} &: \forall X_F^{+* -> *}.\mu_{*}^{+}X_F -> X_F(\mu_{*}^{+}X_F) \\
\unIn_{*} &\triangleq \McvPr_{*} (\lambda\_.\lambda \textit{cast}.\lambda\_.
					\textit{fmap}_{X_F} ~\textit{cast})
\end{align*}

Haskell transcriptions of the unroller embeddings for nested datatypes are
given in Figure~\ref{fig:HaskellunInNested}. (Haskell definitions of
\texttt{Mu1}, \texttt{mcvpr1}, \texttt{fmap1'} are given
in Figure~\ref{fig:HaskellFunctor1}). Note that the definitions of
the unrollers \texttt{unInP} for powerlists and \texttt{unInB} for bushes
are uniform: \lstinline$mvcvp1 (\ _ cast _ ->  fmap1 cast id)$.
The function \texttt{fmap1} is the rank 2 monotone map for type constructors
of kind $+(+* -> *) -> (+* -> *)$, which analogous to rank 1 monotone map
\texttt{fmap} for type constructors of kind $+* -> *$. Or, you can think of 
the \texttt{Functor1} class as a bifunctor over type constructors of kind
$+(+* -> *) -> +* -> *$, whose first argument
($+\underline{(+* -> *)} -> +* -> *$) is a type constructor
and second argument ($+(+* -> *) -> +\underline{*} -> *$) is a type.

Lastly, a Haskell transcription of the unroller embedding for
the length indexed list datatype, which is a term-indexed datatype,
is given in Figure~\ref{fig:HaskellunInIndexed}. To give an embedding of
$\unIn_{(V A)}$, we defined another version of Mendler-style course-of-values
recursion combinator \texttt{mcvprI1}, which is similar to \texttt{mcvpr1}
in Figure~\ref{fig:HaskellFunctor1} but requires the base structure
to be an instance of the \texttt{FunctorI1} class rather than an instance of
the \texttt{Functor1} class. The \texttt{FunctorI1} class is more simple than
the \texttt{Functor1} class since \texttt{FunctorI1} requires only the first
argument to be monotone while \texttt{Functor1} requires both the first and
the second argument to be monotone -- this is evident when you compare
the types of thier member functions \texttt{fmap1} and \texttt{fmapI1}
side-by-side:
\begin{lstlisting}
    fmap1  :: (Functor f, Functor1 h) =>
              (forall i. f i -> g i) -> (a -> b) -> h f a -> h g b
    fmapI1 :: FunctorI1 h =>
              (forall i. f i -> g i)           -> h f a -> h g a
\end{lstlisting}
Note that there fmapI1 does not have the extra \texttt{Functor}
requirement since it does not require the second argument type
to be monotone. The function \texttt{fmapI1} only transforms the first type
constructor argument and preserves the second argument (which may be
a type index or a term index), while \texttt{fmap1} is able to transform
both the first and the second arugment. For term-indexed datatypes,
\texttt{fmapI1} is enough to construct embeddings of the unrollers,
since there is no need to transform indices -- recall that term-indices
do not appear at value level, so no values to transform in the first place.
And even if we had such values, we do not need to transform them anyway
since unrollers are index preserving by definition.
Thus, in the type signature of \texttt{mcvprI1}, we need not reqire
the abstract recursive structure \texttt{r} to be a \texttt{Functor},
unlike in the type signature of \texttt{mcvpr1} where we did require
\texttt{Functor r}. The embedding of $\unIn_{(V A)}$ has the shape
you would expect: \lstinline$mcvprI1 (\_ cast _ ->  fmapI1 cast)$.

An interesting observation is that we can give yet another
Haskell transcription of $\unIn_P$ via in terms of \texttt{mcvprI1}
(see \texttt{unInP'} in Figure~\ref{fig:HaskellunInIndexed}),
rather than in terms of \texttt{mcvpr1}. This is because we did not
really need the ability to transform type indices to embed unrollers
for powerlists. However, for truly nested datatypes such as bushes,
this alternative is not possible. Recall that we do needed to
transform indices from abstract recursive type to concrete recursive type
to embed $\unIn_B$, because bushes were indexed by its own structure.
In summary, unroller embeddings for indexed datatypes, regardless of
term-indexed or type-indexed, without requiring indices to be monotone
unless the datatype is truley nested. For truly nested datatype,
we do have to require indices to be monotone as well as the recursive
type constructor itself, in order to embed their unrollers. We can have
a good approximation of this idea by the type system due to the polirzed kinds
in \Fixi. We conjecture that all base structures of kind $+(+* -> *) -> +* -> *$
are instances of \texttt{Functor1}, while all base structures of kind
$+(0* -> *) -> 0* -> *$ are instances of \texttt{Functor1}. We will
discuss this more formally in Section~\ref{ssec:fixi:theory:cv}.

\subsection{Properties of the unrollers}
We expect two properties to hold for the embeddings of the unrollers.
First, $\unIn_F$ must be a left identity of $\In_\kappa$.
That is, $\unIn_F(\In_\kappa t) -->+ t$ for any term $t$.
Second, $\unIn_F$ should be a constant time operation regardless of its
argument being supplied. That is, $\unIn_F(\In_\kappa t) -->+ t$ takes
constant steps independent of $t$ (but may vary between differrent $F$s).
One bad news is that some embeddings of the unrollers illustrated in
Figure~\ref{fig:unInExamples} are not constant time. However, the good news is
that we can safely optimized them into constant time functions due to
the metatheoretic property of $\McvPr_\kappa$ and $\textit{fmap}$:
\begin{itemize}
\item The \textit{cast} operation of $\McvPr_\kappa$
	is implemened as the identity function \textit{id}.
\item $(\textit{fmap}_F\;\textit{id})\;t -->+ t\;$ for any $\;t : F A$.
	This property generalizes to monotone maps of higher kinds.
	For instance,
      $(\textit{fmap1}_H\;\textit{id}\;\textit{id})\;t -->+ t$
      for any $t : H\,F\,A$ (see Figure~\ref{fig:HaskellFunctor1}
      for the definition of \textit{fmap1}).
\end{itemize}

For instance, $\unIn_{(R A)}$ for the rose tree datatype, which is an
\emph{indirect recursive datatype}, are not constant time. The map function
for lists $\textit{fmap}_\textit{List}$ appearing in the definition
of $\unIn_{(R A)}$ is obviously not a constant time function. That is,
we traverse the list inside a rose tree to to \textit{cast} each element of
the list. Thus, $\unIn_{(R A)}$ is linear to the length of the list apparing
in the rose tree. We can safely optimize $\unIn_{(R A)}$ into a constant time
operation by optimizing $(\textit{fmap}_\texttt{List}\;\textit{cast})$
into the identity function $\textit{id}$. This optimization is safe due to
the property of \textit{cast} being implemented by \textit{id}
and the property of $(\textit{fmap}\;\textit{id})$ being
equivalent to $\textit{id}$.
However, this does not mean we have a constant time embedding of $\unIn_{(R A)}$
within \Fixi, since the optimized term is not type correct.
The identity function $\textit{id}\triangleq \l x.x$ cannot be given
the same type as $(\textit{fmap}_\texttt{List}\;\textit{id}) :
	\texttt{List}(X_r\,X_a) -> \texttt{List}(\mu^{+}_\kappa R\,X_a)$.

For similar reasons, $\unIn_B$ for the bush datatype, which is a
\emph{truly nested datatype}, is not constant time either
due to the use of $\textit{fmap}\;\textit{cast} :
			(X_r(X_r\,X_a)) -> (X_r(\mu^{+}_{* -> *}B\,X_a))$,
which traverses the outer $X_r$ structure (an abstract bush) to cast each
elelemt from $(X_r\,X_a)$ to $(\mu^{+}_{* -> *}B\,X_a)$. But this time,
there is yet another subtlety before worrying about the embedding of
$\unIn_B$ not being constant time. Note, we boldly assumed that
the abstract recursive type $X_r$ has an \textit{fmap} operation
(specified by \texttt{{\bf\ttfamily Functor} r} in the Haskell transcription).
Previously, in the embedding of $\unIn_R$ for the rose tree, we relied on
a property of a specific instance of \textit{fmap} for \texttt{List},
which is a well known type to have \textit{fmap} and indeed equipped with
it desired property. In case of $\unIn_B$, we cannot assume anything but
the kind of $X_r : +* -> *$ since it is abstract. So, we should rely on
a more general property that \textit{fmap} is well-defined for any
type constructors of kind $+* -> *$ in \Fixi. We will discuss this
general property of \textit{fmap} in Section~\ref{ssec:fixi:theory:cv}.

Lastly, we even further optimize the unrollers since we observed
in the previous subsection that all unroller embeddings have uniform shape of
\[
\McvPr_\kappa (\l _.\l cast.\l _.
		\textit{fmap??}~\textit{cast}~\textit{id}\dots\textit{id}))
\]
% Embeddings of unrollers for term-indexed datatypes (e.g. vectors)
% are intuitively more simple than type-indexed datatypes (e.g. powerlists,
% bushes). Due to the erasure property, existence of unroll
% We will discuss the conditions when we can further NO, it's the other way
% around

%% \KYA{ TODO cite Monotone Inductive and Coinductive Constructors of Rank 2  - Matthes CSL paper }
%% 
%% \KYA{ TODO state a theorem about this in metatheory and
%% 	cite some previous work on this maybe? }
%% \KYA{ TODO P do not need Functor requirement but $+* -> *$ kind already
%%         implies that so it will be true anyway}
%% \KYA{ TODO someone must have proved free theorems for higher-kinded cases? }


%% Apart from the limitations of constant-time undefinability of $\unIn_F$
%% discussed above, the embeddings illustrated in Figure~\ref{fig:unInExamples}
%% are not in spirit of Mendler-style. Note that the embeddings of $\unIn_F$ are
%% polytypic (different term encodings for each different $F$) rather than
%% polymorphic (one uniform term encoding whose type is polymorphic over $F$).
%% Recall that the key advantage of Mendler-style comes from being polymorhpic.
%% 
%% Fortunately, there does exists more proper Mendler-style embeddings
%% of the course-of-values combinators over arbitrary positive datatypes
%% using both iteration and coiteration schemes \cite{TODO}. Since coiteration
%% is embeddable in \Fi\ and co(-primitive-)recursion is embeddable in \Fixi,
%% the result directly applies without extending our calculi. However,
%% to our knowledge, course-of-values combinators over higher-kinded
%% type constructors (\ie, type constructors other than kind $*$) has not been
%% well studied enough, even in that setting of using both iteration/recursion
%% and coiteration/corecursion. That is, course-of-values combinators for
%% regular indirect recursive datatypes are very likely to be embeddable in
%% a calculus like \Fi\ or \Fixi\ directly applying the known results, but
%% we may need further investigation to assure ourselves for the behavior of
%% course-of-values combinators over higher-kinded datatypes.
%% 
%% We leave the search for embeddings for arbitrary positive datatypes,
%% including indirectly recursive datatypes and truly nested datatypes,
%% as future work, since coiteration and corecursion are out of the scope of
%% this dissertation.



 %% \label{sec:fixi:cv}

\section{Metatheory} \label{sec:fixi:theory}

\subsection{TODO Strong normalization} \label{ssec:fixi:theory:sn}
We can prove strong normalization of \Fixi\ by erasing term-indices in \Fixi\ 
types into \Fixw\ types. Since \Fixw\ is strongly normalizing \cite{AbeMat04},
the existence of a index erasure that maps a valid typing judgment on a term
in \Fixi\ to a valid typing judgment on the same term in \Fixw\ implies
strong normalization of \Fixi.

The definition of the index erasure operation and the proofs for
the related theorems are almost exactly the same as their counterparts
in System \Fi\ (see \S\ref{sec:fi:theory}). So, we simply illustrate
the definition and just give a very brief sketch of the proofs for the
theorems.

We define a meta-operation of index erasure that projects $\Fixi$ types
to $\Fixw$ types.
\begin{definition}[index erasure]\label{def:Fixierase}
\[ \fbox{$\kappa^\circ$}
 ~~~~ ~~
 *^\circ =
 *
 ~~~~ ~~
 (p\kappa_1 -> \kappa_2)^\circ =
 p{\kappa_1}^\circ -> {\kappa_2}^\circ
 ~~~~ ~~
 (A -> \kappa)^\circ =
 \kappa^\circ
\]
\[ \fbox{$F^\circ$}
 ~~~~
 X^\circ =
 X
 ~~~~ ~~~~
 (A -> B)^\circ =
 A^\circ -> B^\circ
 ~~~~ ~~~~
 (\mu F)^\circ =
 \mu F^\circ
\]
\[ \qquad
 (\lambda X^{p\kappa}.F)^\circ =
 \lambda X^{p\kappa^\circ}.F^\circ
 ~~~~ ~~~~
 (\lambda i^A.F)^\circ =
 F^\circ
\]
\[ \qquad
 (F\;G)^\circ =
 F^\circ\;G^\circ
 ~~~~ ~~~~ ~~~~ ~~~~ ~~
 (F\,\{s\})^\circ =
 F^\circ
\]
\[ \qquad
 (\forall X^\kappa . B)^\circ =
 \forall X^{\kappa^\circ} . B^\circ
 ~~~~ ~~~~
 (\forall i^A . B)^\circ =
 B^\circ
\]
\[ \fbox{$\Delta^\circ$}
 ~~~~
 \cdot^\circ = \cdot
 ~~~~ ~~
 (\Delta,X^{p\kappa})^\circ = \Delta^\circ,X^{p\kappa^\circ}
 ~~~~ ~~
 (\Delta,i^A)^\circ = \Delta^\circ
\]
\[ \fbox{$\Gamma^\circ$}
 ~~~~
 \cdot^\circ = \cdot
 ~~~~ ~~~~
 (\Gamma,x:A)^\circ = \Gamma^\circ,x:A^\circ
\]
\end{definition}

\begin{theorem}[index erasure on well-sorted kinds]
\label{thm:Fixierasesorting}
	$\inference{|- \kappa : \square}{|- \kappa^\circ : \square}$
\end{theorem}

\begin{theorem}[index erasure on well-formed type level contexts]
\label{thm:Fixierasetyctx}
\[ \inference{|- \Delta}{|- \Delta^\circ} \]
\end{theorem}

\begin{theorem}[index erasure on kind equality]\label{thm:Fixierasekindeq}
$ \inference{|- \kappa=\kappa':\square}
	{|- \kappa^\circ=\kappa'^\circ:\square}
$
\end{theorem}

\begin{theorem}[index erasure on well-kinded type constructors]
\label{thm:Fixierasekinding}
\[ \inference{|- \Delta & \Delta |- F : \kappa}
		{\Delta^\circ |- F^\circ : \kappa^\circ}
\]
\end{theorem}
\begin{theorem}[index erasure on type constructor equality]
\[ \inference{\Delta |- F=F':\kappa}
		{\Delta^\circ |- F^\circ=F'^\circ:\kappa^\circ}
\]
\label{thm:Fixierasetyconeq}
\end{theorem}

\begin{theorem}[index erasure on well-formed term level contexts]
\label{thm:Fixierasetmctx}
\[ \inference{\Delta |- \Gamma}{\Delta^\circ |- \Gamma^\circ} \]
\end{theorem}

\begin{theorem}[index erasure on index-free well-typed terms]
\label{thm:Fixierasetypingifree}
\[ \inference{ \Delta |- \Gamma & \Delta;\Gamma |- t : A}
		{\Delta^\circ;\Gamma^\circ |- t : A^\circ}
		{\enspace(\dom(\Delta)\cap\FV(t) = \emptyset)}
\]
\end{theorem}


We introduce an index variable selection meta-operation that selects all
the index variable bindings from the type level context.
\begin{definition}[index variable selection]
\[ \cdot^\bullet = \cdot \qquad
	(\Delta,X^{p\kappa})^\bullet = \Delta^\bullet \qquad
	(\Delta,i^A)^\bullet = \Delta^\bullet,i:A
\]
\end{definition}

\begin{theorem}[index erasure on well-formed term level contexts
		prepended by index variable selection]
\label{thm:Fixierasetmctxivs}
\[ \inference{\Delta |- \Gamma}{\Delta^\circ |- (\Delta^\bullet,\Gamma)^\circ}
\]
\end{theorem}

\begin{theorem}[index erasure on well-typed terms]
\label{thm:Fixierasetypingall}
\[ \inference{\Delta |- \Gamma & \Delta;\Gamma |- t : A}
		{\Delta^\circ;(\Delta^\bullet,\Gamma)^\circ |- t : A^\circ}
\]
\end{theorem}


\KYA{TODO has there been any studies on the logical consistencies of
	implicit calculus + equi-recursive type?}



\subsection{Conditions for well-behaved course-of-values recursion}
\label{ssec:fixi:theory:cv}

\begin{proposition}\label{prop:fixi:fmap}
For any $F : +* -> *$, there exists
\[ \textit{fmap}_F : \forall X^{*}.\forall Y^{*}.(X -> Y) -> F\;X -> F\;Y \]
\end{proposition}

start with a more easy one
\begin{proposition}
There exists
$\textit{fmap}_F : \forall X^{*}.\forall Y^{*}.(X -> Y) -> F\;X -> F\;Y$
for any $F : +* -> *$ such that
\begin{itemize}
	\item all free variables of $F$, including $X$ and $Y$,
		are introduced by universal quantification
		and those variabels are of kind $*$,
	\item all the bound variabels within $F$ are
		introduced by universal quantification
		and those variables are of kind $*$, and
	\item $F : +* -> *$ is in normal form, that is, $F$ can neither be
		an application ($F'\,G$) nor index application ($F'\{s\}$).
\end{itemize}
\end{proposition}
\begin{proof}
	We can derive $\textit{fmap}_F$ from the structure of $F$.
	Due to the kind of $F$ and the fact that $F$ is in normal form
	it must be a in the form of $\l Z^{+*}.B$.
\begin{itemize}
\item[case]($Z\notin \FV(B)$)
	$ \textit{fmap}_{(\l Z^{+*}. B)} = \l \_ . \l x. x $

	Since $F\;X = F\;Y = B$, we just return the identity function on $B$.

\item[case]($F \triangleq \l Z^{+*}. Z$)
	$ \textit{fmap}_{(\l Z^{+*}. Z)} = \l z . z $

	Since $F\;X = X$ and $F\;X = Y$,
	we return the function $z:X -> Y$ itself.

\item[case]($F \triangleq \l Z^{+*}. \forall X_1^{*} .B_1$)
	$\textit{fmap}_{(\l Z^{+*}. \forall X_1^{*} .B_1)}
	= \textit{fmap}_{(\l Z^{+*}.B_1)}$

\item[case]($F \triangleq \l Z^{+*}. A -> B_1$)

	When $Z\notin\FV(A)$,
	$\textit{fmap}_{(\l Z^{+*}.A -> B_1)}
	= \l z.\l y. \l x. \textit{fmap}_{(\l Z^{+*}.B_1)} \; z \; (y \; x)$

	When $A \triangleq \forall X_1^{*}.A_1$,
	$\textit{fmap}_{(\l Z^{+*}.(\forall X_1^{*}.A_1) -> B_1)}
	= \textit{fmap}_{(\l Z^{+*}.A_1 -> B_1)}$

	\begin{singlespace}
	When $A \triangleq A_1 -> \cdots -> A_n -> \forall X_2^{*}.B_2'$,
	\vspace{-1.5ex}
	\[\textit{fmap}_{(\l Z^{+*}.(A_1 -> \cdots -> A_n -> \forall X_2^{*}.B_2') -> B_1)}
	= \textit{fmap}_{(\l Z^{+*}.(A_1 -> \cdots -> A_n -> B_2') -> B_1)} \]

	When $A \triangleq A_1 -> \cdots -> A_n -> B_2$,
	where $B_2$ is not an arrow type
	\vspace{-1.5ex}
	\begin{align*}
	  & \textit{fmap}_{(\l Z^{+*}.(A_1 -> \cdots -> A_n -> B_2) -> B_1)} \\
	=~& \l z.\l y. \l x.\;
	\textit{fmap}_{(\l Z^{+*}.B_1)} \; z \;
		(y \; (\l x_1.\ldots\l x_n. ~
		   x  &\!\!\!\!(\textit{fmap}_{(\l Z^{+*}.A_1)} ~ z ~ x_1)& \\
		     &&\!\!\!\!\vdots \qquad\qquad& \\
		     &&\!\!\!\!(\textit{fmap}_{(\l Z^{+*}.A_n)} ~ z ~ x_n)&
		\,) \,)
	\end{align*}
	\end{singlespace}

\end{itemize}
\end{proof}

TODO to give an idea that the derived fmaps are type correct in the above
proof give a example in Haskell accepted by GHC for each case
(already have it almost done put it in the repository)

What if there are type constructor variables?
Should still work, let's write down the rules

\begin{proposition}\label{prop:fixi:fmapFree}
If $\textit{fmap}_F:\forall X^{*}.\forall Y^{*}.(X -> Y) -> F\;X -> F\;Y$
exists, then
\begin{align*}
\textit{fmap}_F~\textit{id} &~=~ \textit{id} \\
\textit{fmap}_F~\textit{f} \;\circ\; \textit{fmap}_F~\textit{g}
&~=~ \textit{fmap}_F~(f\circ g)
\end{align*}
\end{proposition}\noindent
This is a well-known parametricity theorem on maps any instance of the type
$\forall X^{*}.\forall Y^{*}.(X -> Y) -> F\;X -> F\;Y$ satisfies
the two equations above. However, for the purpose of defining $\McvPr_{*}$,
we only need to know that there exists one such $\textit{fmap}_F$. That is,
\begin{proposition}\label{prop:fixi:fmapHom}
For any $F : +* -> *$, there exists

$\textit{fmap}_F:\forall X^{*}.\forall Y^{*}.(X -> Y) -> F\;X -> F\;Y$
such that
\begin{align*}
\textit{fmap}_F~\textit{id} &~=~ \textit{id} \\
\textit{fmap}_F~\textit{f} \;\circ\; \textit{fmap}_F~\textit{g}
&~=~ \textit{fmap}_F~(f\circ g)
\end{align*}
\end{proposition}
\begin{proof}
	You can check that each case
	in the proof of Proposition \ref{prop:fixi:fmap}
	satisfies the two equations above.
\end{proof}

\begin{proposition} For any $F : +* -> *$, there exists
$\unIn_F : \mu^{+}{*} F -> F(\mu^{+}{*} F)$ such that
$\unIn_F (\In_F\;t) -->+ t$.
\end{proposition}
\begin{proof}
Since we know that $\textit{fmap}_F$ exists by Proposition~\ref{prop:fixi:fmap},
we can define
\[ \unIn_F = \McvPr_{*}\;
            (\l\_.\l\textit{cast}.\l\_.\l x.\textit{fmap}_F\;\textit{cast}\;x)
\]

From Proposition~\ref{prop:fixi:fmapFree}, we know that
$\textit{fmap}_F\;\textit{id}\;x -->+ x$.
Thus,
\[ \unIn_F (\In_F\;t) -->+ \textit{fmap}_F\;\textit{id}\;t -->+ t \]
\end{proof}

\begin{align*}
A \rrarrow_{*} B &~\triangleq~ A -> B \\
F \rrarrow^{p\kappa -> \kappa'} G &~\triangleq~
	\forall X^\kappa.\forall Y^\kappa.
		(X \rrarrow_\kappa Y) -> F X \rrarrow_\kappa F Y \\
F \rrarrow^{A -> \kappa} G &~\triangleq~
	\forall i^A.\forall f^{A->A}. F\{i\} \rrarrow_\kappa F\{f\;i\}
\end{align*}

\[
\textsf{mon}_\kappa
  = \l X^{0\kappa}.X \rrarrow^\kappa X
\]

$\textsf{mon}_{+* -> *} F$ is the type of $\textit{fmap}_F$
where $F : +* -> *$.

$\textsf{mon}_{+(p* -> *)->(p* -> *)} F$ is the type of $\textit{fmap1}_F$
where $F : +(p* -> *)->(p* -> *)$.

if $\textsf{mon}_{+(p* -> *)->(p* -> *)}$ is inhabited
then $\unIn_F$ for any $F : +(p* -> *)->(p* -> *)$?

what about general case? if $\textsf{mon}_\kappa$ is inhabited
then $\unIn_F$ for any $F : \kappa$?

 %% \label{sec:fixi:theory}



%% \section{TODO}

\section{TODO}
let's write a paper for maybe one of the following venues?
\begin{itemize}
\item FICS \url{http://www.inf.u-szeged.hu/fics2012/}\\
Abstract submission: 	4 Dec 2011\\
Paper submission: 	11 Dec 2011\\
8 pages using eptcs style
\item MSFP \url{http://cs.ioc.ee/msfp/msfp2012/}\\
Submission of papers 16 December\\
There is no specific page limit but authors should strive for brevity.
\end{itemize}

~\\~\\
\cite{article,bookA} just dummy citation %% \nocite{*}
\bibliographystyle{eptcs}
\bibliography{main}

\end{document}
