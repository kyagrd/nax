\section{Nax}\label{sec:nax}
\begin{figure}
\begin{framed}
\paragraph{Syntax:}
\begin{align*}
&\!\!\!\!\!\!\!\!\!\!\!\!\!\!\!\!\!\!\!
 \text{Kinds}
 \!\!\!\!\!\!\!\!\!\!\!\!\!\!\!\!\!\!\!
	& \kappa	&~ ::= ~ * \mid \kappa -> \kappa \mid A -> \kappa
 \!\!\!\! \!\!\!\! \\
&\!\!\!\!\!\!\!\!\!\!\!\!\!\!\!\!\!\!\!
 \text{Type Constructors}
 \!\!\!\!\!\!\!\!\!\!\!\!\!\!\!\!\!\!\!
	& F,G,A,B	&~ ::= ~ X \mid T
			   \mid \mu^\kappa(T\,\overline\tau)
			   \mid F\;G \mid F\,\{s\}
			   \mid A -> B
 \!\!\!\! \!\!\!\! \\
&\!\!\!\!\!\!\!\!\!\!\!\!\!\!\!\!\!\!\!
 \text{Type Schemes}
 \!\!\!\!\!\!\!\!\!\!\!\!\!\!\!\!\!\!\!
	& \sigma	&~ ::= A
			   \mid \forall X.\sigma
			   \mid \forall i.\sigma
 \!\!\!\! \!\!\!\! \\
&\!\!\!\!\!\!\!\!\!\!\!\!\!\!\!\!\!\!\!
 \text{Terms}
 \!\!\!\!\!\!\!\!\!\!\!\!\!\!\!\!\!\!\!
	& r,s,t		&~ ::= ~ x \mid `x \mid i
			   \mid \lambda x.t \mid r\;s
			   \mid \<let> x=s \<in> t %% \\
%% &	&		&~~~~~
			   \mid \varphi^\psi
			   \mid \<MIt> x.\varphi^\psi
			   \mid \In^\kappa
 \!\!\!\! \!\!\!\! \\
&\!\!\!\!\!\!\!\!\!\!\!\!\!\!\!\!\!\!\!
 \text{Program}
 \!\!\!\!\!\!\!\!\!\!\!\!\!\!\!\!\!\!\!
	& Prog		&~ ::= ~ \overline{D};t
 \!\!\!\! \!\!\!\! \\
&\!\!\!\!\!\!\!\!\!\!\!\!\!\!\!\!\!\!\!
 \text{Declarations}
 \!\!\!\!\!\!\!\!\!\!\!\!\!\!\!\!\!\!\!
	& D		&~ ::= \<data> T : \overline{K} -> * \<where>
                               \overline{C : \overline{A} -> T\,\overline{\tau}}
			 ~~\mid~ `x = t
 \!\!\!\! \!\!\!\! \\
&\!\!\!\!\!\!\!\!\!\!\!\!\!\!\!\!\!\!\!
 \text{List of Declarations}
 \!\!\!\!\!\!\!\!\!\!\!\!\!\!\!\!\!\!\!
	& \overline{D}	&~ ::= \cdot \mid D,\overline{D}
 \!\!\!\! \!\!\!\! \\
&\!\!\!\!\!\!\!\!\!\!\!\!\!\!\!\!\!\!\!
 \text{Kind Arguments}
 \!\!\!\!\!\!\!\!\!\!\!\!\!\!\!\!\!\!\!
	& K		&~ ::= ~ \kappa \mid A
 \!\!\!\! \!\!\!\! \\
&\!\!\!\!\!\!\!\!\!\!\!\!\!\!\!\!\!\!\!
 \text{Type Arguments}
 \!\!\!\!\!\!\!\!\!\!\!\!\!\!\!\!\!\!\!
	& \tau		&~ ::= ~ G \mid \{s\}
 \!\!\!\! \!\!\!\! \\
&\!\!\!\!\!\!\!\!\!\!\!\!\!\!\!\!\!\!\!
 \text{Type Argument Variables}
 \!\!\!\!\!\!\!\!\!\!\!\!\!\!\!\!\!\!\!
	& \iota		&~ ::= ~ X \mid i
 \!\!\!\! \!\!\!\! \\
&\!\!\!\!\!\!\!\!\!\!\!\!\!\!\!\!\!\!\!
 \text{Index Transformers}
 \!\!\!\!\!\!\!\!\!\!\!\!\!\!\!\!\!\!\!
	& \psi		&~ ::= \cdot \mid \overline\iota . B
 \!\!\!\! \!\!\!\! \\
&\!\!\!\!\!\!\!\!\!\!\!\!\!\!\!\!\!\!\!
 \text{Case Branches}
 \!\!\!\!\!\!\!\!\!\!\!\!\!\!\!\!\!\!\!
	& \varphi	&~ ::= \overline{C\,\overline{x} -> t}
 \!\!\!\! \!\!\!\! \\
&\!\!\!\!\!\!\!\!\!\!\!\!\!\!\!\!\!\!\!
 \text{Contexts}
 \!\!\!\!\!\!\!\!\!\!\!\!\!\!\!\!\!\!\!
	& \Sigma	&~ ::= ~ \cdot
			   \mid \Sigma, T: \kappa
			   \mid \Sigma, C: \sigma
			   \mid \Sigma, `x: \sigma
 \!\!\!\! \!\!\!\! \\
&\!\!\!\!\!\!\!\!\!\!\!\!\!\!\!\!\!\!\!
 \!\!\!\!\!\!\!\!\!\!\!\!\!\!\!\!\!\!\!
	& \Delta	&~ ::= ~ \cdot
			   \mid \Delta, X^\kappa
			   \mid \Delta, i^\sigma
 \!\!\!\! \!\!\!\! \\
&\!\!\!\!\!\!\!\!\!\!\!\!\!\!\!\!\!\!\!
 \!\!\!\!\!\!\!\!\!\!\!\!\!\!\!\!\!\!\!
	& \Gamma	&~ ::= ~ \cdot \mid \Gamma,x:\sigma
 \!\!\!\! \!\!\!\!
\end{align*}

\paragraph{Well-formed contexts:}
\[ \fbox{$|- \Sigma$}
 ~~~~
   \inference{}{|- \cdot}
 ~~~~
   \inference{|- \Sigma & \Sigma |- \kappa:\square}
             {|- \Sigma,T:\kappa}
      \big( T\notin\dom(\Sigma) \big)
\]
\[ \inference{|- \Sigma & \Sigma;\cdot |- \sigma:*}
             {|- \Sigma,C:\sigma}
      \big( C\notin\dom(\Sigma) \big)
 ~~~~
   \inference{|- \Sigma & \Sigma;\cdot |- \sigma:*}
             {|- \Sigma,`x:\sigma}
      \big( `x\notin\dom(\Sigma) \big)
\]
\[ \fbox{$\Sigma|- \Delta$}
 ~~~~
   \inference{|- \Sigma}{\Sigma |- \cdot}
 ~~~~
   \inference{\Sigma |- \Delta & \Sigma |- \kappa:\square}
             {\Sigma |- \Delta,X^\kappa}
      \big( X\notin\dom(\Delta) \big)
 ~~~~ 
   \inference{\Sigma |- \Delta & \Sigma;\cdot |- \sigma:*}
             {\Sigma |- \Delta,i^\sigma}
      \big( i\notin\dom(\Delta) \big)
\]
\[ \fbox{$\Sigma;\Delta |- \Gamma$}
 ~~~~
   \inference{\Sigma |- \Delta}{\Sigma;\Delta |- \cdot}
 ~~~~
   \inference{\Sigma;\Delta |- \Gamma & \Sigma;\Delta |- A:*}
             {\Sigma;\Delta |- \Gamma,x:A}
      \big( x\notin\dom(\Gamma) \big)
\]
~\\

\paragraph{Reduction:}
$ \fbox{$t \rightsquigarrow t'$}
 ~~~~
   \inference{}{(\lambda x.t)\,s \rightsquigarrow [s/x]t}
 ~~~~
   \inference{}{\<let> x=s \<in> t \rightsquigarrow [s/x]t}
$
\[ \inference{ C\,\overline{x} -> t \in \varphi }
             { \varphi^\psi (C\,\overline{t})  \rightsquigarrow
               [\overline{t}/\overline{x}]t }
 ~~~~
   \inference{}
             { \<MIt> x.\varphi^\psi\;(\In^\kappa t) \rightsquigarrow
               [\<MIt> x.\varphi^\psi/x]\varphi^\psi ~ t}
 ~~~~
   \inference{ `x=t \in \overline{D} }{ `x \rightsquigarrow t }
\]
\[  \inference{t \rightsquigarrow t'}{\lambda x.t \rightsquigarrow \lambda x.t'}
 ~~~~
   \inference{r \rightsquigarrow r'}{r\;s \rightsquigarrow r'\;s}
 ~~~~
   \inference{s \rightsquigarrow s'}{r\;s \rightsquigarrow r\;s'}
 ~~~~
   \inference{t_i \rightsquigarrow t_i'}
             {C~t_1\cdots t_i \cdots t_n \rightsquigarrow
              C~t_1\cdots t_i'\cdots t_n }
\]
\[ \inference{s\rightsquigarrow s'}
             {\<let> x=s \<in> t \rightsquigarrow \<let> x=s' \<in> t}
 ~~~~
 \inference{t\rightsquigarrow t'}
             {\<let> x=s \<in> t \rightsquigarrow \<let> x=s \<in> t'}
 ~~~~
   \inference{\varphi^\psi \rightsquigarrow \varphi'^\psi}
             {\<MIt> x.\varphi^\psi \rightsquigarrow \<MIt> x.\varphi'^\psi}
\]
\[ \inference{t_i \rightsquigarrow t_i'}
      {(C_1\,\overline{x_1} -> t_1;\cdots
       ;C_i\,\overline{x_i} -> t_i;\cdots
       ;C_n\,\overline{x_n} -> t_n)^\psi
      ~\rightsquigarrow~
       (C_1\,\overline{x_1} -> t_1;\cdots
       ;C_i\,\overline{x_i} -> t_i';\cdots
       ;C_n\,\overline{x_n} -> t_n)^\psi
      }
\]
\end{framed}
\caption{Syntax and Reduction rules of Nax}
\label{fig:NaxSyntax}
\end{figure}

\begin{figure}
\begin{framed}
\paragraph{Sorting:}
\[ \fbox{$\Sigma |- \kappa : \square$}
 ~~~~
  \inference[($A$)]{}{\Sigma |- * : \square}
 ~~
   \inference[($R$)]{ \Sigma |- \kappa  : \square
                    & \Sigma |- \kappa' : \square }
                    { \Sigma |- \kappa -> \kappa' : \square }
 ~~
   \inference[($Ri$)]{\Sigma;\cdot |- A : * & \Sigma |- \kappa : \square}
                     {\Sigma |- A -> \kappa : \square}
\]

\paragraph{Kinding:}
$ \fbox{$\Sigma;\Delta |- \sigma : \kappa$ }
 ~~~~
   \inference[($\forall$)]{\Sigma;\Delta, X^\kappa |- \sigma : *}
                          {\Sigma;\Delta |- \forall X . \sigma : *}
 ~~~~
   \inference[($\forall i$)]{\Sigma;\Delta, i^A |- \sigma : *}
                            {\Sigma;\Delta |- \forall i . \sigma : *}
$
\[ \fbox{$\Sigma;\Delta |- F : \kappa$}
 ~~~~
   \inference[($Var$)]{X^\kappa\in\Delta & \Sigma |- \Delta}
                      {\Sigma;\Delta |- X : \kappa}
 ~~~~
   \inference[($TCon$)]{T:\kappa\in\Sigma & \Sigma |- \Delta}
                       {\Sigma;\Delta |- T : \kappa}
 ~~~~
   \inference[($\mu$)]{\Sigma;\Delta |- T\,\overline\tau : \kappa -> \kappa}
                      {\Sigma;\Delta |- \mu^\kappa(T\,\overline\tau) : \kappa}
\]
\[ \inference[($@$)]{ \Sigma;\Delta |- F : \kappa -> \kappa'
                    & \Sigma;\Delta |- G : \kappa }
                    {\Sigma;\Delta |- F\,G : \kappa'}
 ~~~~
   \inference[($@i$)]{ \Sigma;\Delta |- F : A -> \kappa
                     & \Sigma;\Delta;\cdot |- s : A }
                     {\Sigma;\Delta |- F\,\{s\} : \kappa}
\]
\[ \inference[($->$)]{\Sigma;\Delta |- A : * & \Sigma;\Delta |- B : *}
                     {\Sigma;\Delta |- A -> B : * }
 ~~~~
   \inference[($Conv$)]{ \Sigma;\Delta |- A : \kappa
                       & \Sigma |- \kappa = \kappa' : \square }
                       {\Sigma;\Delta |- A : \kappa'}
\]


\paragraph{Typing:}
\[ \fbox{$\Sigma |- Prog : A$}
 ~~~~
   \inference[($\cdot;t$)]{\Sigma;\cdot;\cdot |- t:A}
                          {\Sigma |- \cdot;t : A}
 ~~~~
   \inference[($D$)]{ \Sigma |- D \Rrightarrow \Sigma'
                    & \Sigma' |- \overline{D};\; t : A}
                    {\Sigma |- D,\overline{D};\; t : A}
\]
\[ \fbox{$\Sigma;\Delta;\Gamma |- t : A$}
 ~~~~
   \inference[($=$)]{ \Sigma;\Delta;\Gamma |- t : A
                    & \Sigma;\Delta |- A = B : *}
                    {\Sigma;\Delta;\Gamma |- t : B}
\]
\[ \inference[($:$)]
      { x:\sigma \in \Gamma & \Sigma;\Delta |- A\prec\sigma
      & \Sigma;\Delta |- \Gamma }
      {\Sigma;\Delta;\Gamma |- x:A}
 ~~~~
   \inference[($:i$)]
      { i^\sigma \in \Delta & \Sigma;\Delta |- A\prec\sigma
      & \Sigma;\Delta |- \Gamma }
      {\Sigma;\Delta;\Gamma |- i:A}
\]
\[ \inference[($:C$)]
      { C:\sigma \in \Sigma & \Sigma;\Delta |- A\prec\sigma
      & \Sigma;\Delta |- \Gamma }
      {\Sigma;\Delta;\Gamma |- C:A}
 ~~~~
 \inference[($:`$)]
      { `x:\sigma \in \Sigma & \Sigma;\Delta |- A\prec\sigma
      & \Sigma;\Delta |- \Gamma }
      {\Sigma;\Delta;\Gamma |- `x:A}
\]
\[ \inference[($->$$I$)]{\Sigma;\Delta;\Gamma,x:A |- t : B}
                        {\Sigma;\Delta;\Gamma |- \lambda x.t : A -> B}
 ~~~~ ~~~~ ~~~~ ~~~~
   \inference[($->$$E$)]{ \Sigma;\Delta;\Gamma |- r : A -> B
                        & \Sigma;\Delta;\Gamma |- s : A}
                        {\Sigma;\Delta;\Gamma |- r\;s : B}
\]
\[ \inference[(let)]
      { \Sigma;\Delta,\overline{\iota^K};\Gamma |- s : A \\
        \Sigma;\Delta;\Gamma,x:\forall\,\overline{\iota}.A |- t : B }
      {\Sigma;\Delta;\Gamma |- \<let> x=s \<in> t : B }
      \begin{pmatrix} \overline{\iota} \cap \FV(s) = \emptyset \\
                      \overline{\iota} \cap \FV(\Gamma) = \emptyset
      \end{pmatrix}
 ~~~~
   \inference[(case)]
      { \Sigma;\Delta;\Gamma |-^\psi \varphi
                             : \forall\,\overline\iota . F\,\overline\iota
                                                    -> \psi(\overline\iota) }
      {\Sigma;\Delta;\Gamma |- \varphi^\psi
                             : F\;\overline\tau -> \psi(\overline\tau) }
\]
\[ \inference[(MIt)]
      { \Sigma;\Delta,X^\kappa;
               \Gamma,x:\forall\,\overline{\iota'} . X\overline{\iota'}
                                              -> \psi(\overline{\iota'})
           |-^\psi \varphi
           : \forall\,\overline\iota . F\,X\,\overline\iota
                                     -> \psi(\overline\iota) }
      {\Sigma;\Delta;\Gamma
           |- \<MIt> x.\varphi^\psi
           : \mu^{\kappa}\,F\;\overline\tau -> \psi(\overline\tau) }
      \big( X\notin\FV(\Gamma) \big)
\]
\[ \inference[(In)] %% TODO well kindedness
      {}
      {\Sigma;\Delta;\Gamma |- \In^\kappa
        : F(\mu^\kappa F)\,\overline{\tau} -> \mu^\kappa F\;\overline{\tau} }
\]
\[ \fbox{$\Sigma;\Delta;\Gamma |-^\psi \varphi : \sigma$}
 ~~
   \inference
      { \Sigma|_T = \overline{C_k:\sigma_k}^{\;k=1..n} &
        \overline{
        \begin{matrix}
         \Sigma;\Delta
           |- \overline{A} -> T\,\overline{\tau}\,\overline{\tau}_k
              \prec
              \sigma_k
        &
         \Sigma;\Delta;\Gamma,\overline{x:A} |- t : \psi(\overline{\tau}_k) 
        \end{matrix} }^{\;k=1..n} }
      { \Sigma;\Delta;\Gamma
           |-^\psi \overline{C_k\;\overline{x} -> t}^{\;k=1..n}
           : \forall\,\overline\iota . T\,\overline{\tau}\,\overline\iota
                                    -> \psi(\overline\iota) }
\]

\paragraph{Extending the Global Context:}
\fbox{$\Sigma |- D \Rrightarrow \Sigma'$}
\[ \inference[($\Sigma,T$)]
      { \overline{ \Sigma,T:\kappa;\;\overline{\iota^K}
                      |- \overline{A} -> T\,\overline\tau :*} }
      { \Sigma |- \<data> T : \kappa \<where>
                  \overline{C : \overline{A} -> T\,\overline\tau }
               ~\Rrightarrow~
                  \Sigma,T:\kappa
                        ,\overline{C:\forall\,\overline{\iota}.\overline{A}
                                                       -> T\,\overline\tau} }
\]
\[ \inference[($\Sigma,`x$)]
       {\Sigma;\overline{\iota^K};\cdot |- t : A}
       {\Sigma |- `x = t \Rrightarrow \Sigma,`x:\forall\,\overline{\iota}.A}
      \big( \overline{\iota} \cap \FV(t) = \emptyset \big)
\]
\end{framed}
\caption{Typing rules of Nax}
\label{fig:NaxTyping}
\end{figure}

\subsection{Language definition}
The Nax language definition is described in \Fig{NaxSyntax} and \Fig{NaxTyping}.
\Fig{NaxSyntax} illustrates syntax, reduction rules, and well-formedness
conditions for typing contexts.
\Fig{NaxTyping} illustrates sorting, kinding, and typing rules.

\paragraph{Typing contexts}
The typing context of Nax is spearated into three zones. In addition to
the two zones of \Fi (the type level context $\Delta$ and
the term level context $\Gamma$), we have top level contexts ($\Sigma$).
The top level contexts can contain three kinds of bindings:
type constructor bindings ($T:\kappa$), data constructor bindings ($C:\sigma$),
and top level variable bindings ($`x:\sigma$). These bindings are introduced
from delcarations ($D$). Type constructor bindings ($T:\kappa$) and
data constructor bindings ($C:\sigma$) are introduced from datatype declarations
($\<data> T:\dots \<where> \dots$). Top level variable bindings ($`x:\sigma$)
are introduced from top level definitions ($`x = t$).
The rules for well-formed contexts in Nax are similar to those rules in \Fi.

\paragraph{Kinds and their sorting rules}
The kind syntax of Nax is exactly the same as the kind syntax of \Fi.
The sorting rules are the same as \Fi\ except we judge the sorts of kinds
under the top level context ($\Sigma$).

\paragraph{Type constructors and their kinding rules}
The syntax for type constructors of Nax is similar to \Fi, but
different from \Fi\ in two aspects.

Firstly, polymorphic types are separate out as type schemes ($\sigma$) in Nax
since the type system of Nax is in flavour of Hindley-Milner to support
type inference (or, reconstruction).

Secondly, there are no type level abstractions and index abstractions in Nax.
Instead of defining type constructors expecting type arguments by abstraction
and index abstraction at type level, Nax supports datatype declarations
($\<data> T:\kappa \<where> \dots$) and recursive type operators
($\mu^\kappa$) as language constructs.

Intuitively, the kinding rule for the recursive type operator should be
$\Sigma |- \mu^\kappa : (\kappa -> \kappa) -> \kappa$. However, we restrict
the recursive type operator ($\mu^\kappa$) only to be applied to datatypes
($T\,\overline{\tau}$). This restriction is evident in both
the type constructor syntax in \Fig{NaxSyntax}
and the kinding rule ($\mu$) in \Fig{NaxTyping}. What this restriction
really excludes are nested applications of recursive type operators.
For instance, $\mu^\kappa(\mu^{\kappa -> \kappa} F)$ where
$F:(\kappa -> \kappa) -> \kappa -> \kappa$ is not allowed
although it would be well-kinded under the less restrictive kinding rule
($\Sigma |- \mu^\kappa : (\kappa -> \kappa) -> \kappa$).
Motivation behind this restriction is type inference.
In order to infer a type for a Mendler style iterator, we need to restrict
the form of its body since the body must be polymorphic over the indices
(see (MIt) rule). In general, we do not want polymorphic types to be first
class since we want type inference. One simple design choice is to allow
case branches ($\varphi$) to have polymorphic types, or type schemes, and
annotate case branches with index transformers ($\varphi^\psi$). For the
exact same reason (\ie, type inference), we restrict the body of the
Mendler style iterators be case terms (\ie, ($\<MIt> x.\varphi^\psi$)
instead of ($\<MIt> x.t$)).

\paragraph{Terms and their typing rules}
The term syntax of Nax has six additional term constructs than \Fi:
data constructors ($C$), top level variables ($`x$),
polymorphic let bindings ($\<let> x=s \<in> t$),
eliminators for datatypes ($\varphi^\psi$),
Mendler style iterators ($\<MIt>x.\varphi^\psi$), and
constructors for recursive types ($\In^\kappa$).
Typing rules for them are provided in \Fig{NaxTyping}.

The typing rules ($:C$) and ($:`$) are for data constructors ($C$)
and top level variables ($`x$) bound in the top level context ($\Sigma$).
Data constructors ($C$) are introduced from datatype declarations
($\<data> T:\kappa \<where> \dots$) by the rule ($\Sigma,T$), and
top level variables ($`x$) are introduced from top level definitions
($`x=t$) by the rule ($\Sigma,`x$).
The typing rules ($:C$) and ($:`$) behave similar to the rule ($:$)
for the variables and the rule ($:i$) for index variables.
All these four rules ($:$), ($:i$), ($:C$), and ($:`$) for
identifiers look up a certain context (one of the three zones $\Sigma$,
$\Delta$, and $\Gamma$). Since the Nax type system is in flavour of
Hindley-Milner, identifiers are bound to type schems ($\sigma$) and
the typing rules for the identifiers instantiate type schemes to types ($A$).
Note, a type instantiation ($\Sigma;\Delta |- A\prec\sigma$) is a judgement
under the top level context ($\Sigma$) and the type level context ($\Delta$),
since the instantiated type needs to be well-kinded under
$\Sigma$ and $\Delta$.

Polymorphic let bindings in Nax are just the usual polymorphic bindings of
Hindley-Milner type system for generalizing types of local definitions into
type schemes. In Nax, we generalize over term indices as well as types.
The typing rule for let bindings is the (let) rule.

Eliminators for datatypes ($\varphi^\psi$), or case-terms, are
case branches ($\varphi$) annotated by index transformers ($\psi$).
For non-indexed types, case-terms are the usual single level patterm matching
expressions in functional languages. For example, a Nax case-term applied to
non-indexed typed term ($\varphi^\cdot~s$) corresponds to a Haskell
case-expression over that term ($\mathbf{case}~s~\mathbf{of}~\{\varphi\}$).
Note, we give trivial index transformer annotation (\ie, $\psi=\cdot$) for
non-indexed types (\eg, booleans, natural numbers) since there are no indices
to worry about. For indexed types, the indexed transformer annotations
privide useful information for type reconstruction. For example, consider
the following datatype declaration:
\begin{align*}
\<data> \mathtt{Judgement : Bool -> *} \<where>
&~\mathtt{TJ : Formula -> Judgement~\{True\}};\\
&~\mathtt{FJ : Formula -> Judgement~\{False\}}
\end{align*}
The datatype $\mathtt{Judgement}$ is index by boolean terms
(\eg, $\mathtt{True}$ and $\mathtt{False}$ of type $\mathtt{Bool}$).
The data constructor $\mathtt{TJ}$ contains a formula expected to be true and
the data constructor $\mathtt{FJ}$ contains a formula expected to be false.
We can define a function, which produces an inverted judgement by negating
the formula contained in a given judgement, as follows\footnote{case branches
are laid out in multiple lines for better readability}:
\[
\begin{pmatrix}
 \mathtt{TJ}~x -> \mathtt{FJ} (\mathop{neg}~x);\\
 \mathtt{FJ}~x -> \mathtt{TJ} (\mathop{neg}~x)\phantom{;}
\end{pmatrix}^{i\,.\,\mathtt{Judgement}~\{`not~i\}}
~~~
\begin{matrix}
 \text{where $neg$ is a function that produces negated formula}\\
~~~\text{and $`not$ is a top level function that negates booleans.}
\end{matrix}
\]
Note that the index transformer ($i\,.\,\mathtt{Judgement}~\{`not~i\}$)
captures the idea that the resulting inverted judgement has opposite
expectations from the given judgement. The types of such case-terms
involving indexed types can also be inferred when we annotate
the case-terms with appropriate index transformers.
Reduction rules for case-terms (\Fig{NaxSyntax}) are standard.

Mendler style iterators ($\<MIt> x.\varphi^\psi$) are eliminators for
recursive types. A case-term expects a datatype argument
(of type $T\,\overline\tau$).
A Mendler style iterator expects a recursive type argument
(of type $\mu^\kappa(T\,\overline\tau)$).
Intuitively, Mendler style iterators open up the recursive type
($\mu^\kappa(T\,\overline\tau)$) and case branch over
its base datatype structure (of type $T\,\overline\tau$).
This intuition is captured by the reduction rule for
Mendler style iterators (\Fig{NaxSyntax}):
$\<MIt> x.\varphi^\psi\;(\In^\kappa t) \rightsquigarrow
 [\<MIt> x.\varphi^\psi/x]\varphi^\psi\;t$.
Note that a Mendler style iterator ($\<MIt> x.\varphi^\psi$) applied to
a term of recursive type ($\In^\kappa t)$ constructed by the $\In^\kappa$
constructor reduces to a case-term ($[\<MIt> x.\varphi^\psi/x]\varphi^\psi$)
applied to the base structure ($t$) contained in the $\In^\kappa$ constructor.
The variable ($x$) bound by $\<MIt>$ is a label for the recursive call.
Note that the case-term ($[\<MIt> x.\varphi^\psi/x]\varphi^\psi$) appearing in
the reduction rule substitutes $x$ with the Mendler style iterator itself.
However, unlike the fixpoint operator for unrestricted general recursion,
Mendler style iterators are guaranteed to normalize because of their carefully
designed typing rule (MIt) due to Mendler.

The constructors for recursive types ($\In^\kappa$) are standard
(see rule (In) in \Fig{NaxTyping}). The kind annotation $\kappa$ on
the $\In^\kappa$ constructor aids kind inference. If we were to simulate
the recursive type operator $\mu^\kappa$ and its constructor $\In^\kappa$ in
a functional language like Haskell (with GADT and kind annotation extensions),
we would simulate them by the following recursive datatype:
\[ \<data> \mu^\kappa : (\kappa -> \kappa) -> \kappa \<where>
      \In^\kappa : X(\mu^\kappa X)\overline\iota -> \mu^\kappa X\, \overline\iota \]
However, such a simulation of $\mu^\kappa$ by a recursive datatype
cannot guarantee normalization of the language, since unlimited elimination
of $\In^\kappa$ via case branches is already powerful enough to encode
non-terminating computation even without using any recursion at term level.
Thus, Nax provides $\mu^\kappa$ and $\In^\kappa$ as primitive language
constructs, and only allow elimination of $\In^\kappa$ via
Mendler style iteration. %\footnote{More recursion schemes}
%%% TODO example?
% Nax does not provide eliminators, or inverse functions of $\In^\kappa$,
% which are often denoted as $\In^{-1}$, \textsf{unIn}, or \textsf{out},
% since we want to support arbitrary recursive types in a normalizing language.

\paragraph{Nax programs and their typing rules}
A Nax program ($\overline{D};t$) is a list of declarations ($\overline{D}$)
followed by a term ($t$). A declaration can be either a datatype declaration
($\<data> T : \kappa \<where> \dots$) or a top level definition ($`x=t$).
The list of declarations ($\overline{D}$) are processed by the rules
($\Sigma,T$) and ($\Sigma,`x$) before type checking the term ($t$).
The kinding and typing information from the datatype declarations and
the top level definitions preceding the term are captured into
the top level context ($\Sigma$) according to the rules ($\Sigma,T$)
and ($\Sigma,`x$). The top level context extended by these rules are
used for type checking the term following the list of declarations.
Therefore, the sorting, kinding, and typing rules of Nax (\Fig{NaxTyping})
involves $\Sigma$ in addtion to $\Delta$ and $\Gamma$, while
the corresponding rule of \Fi\ (\Fig{Fi}) involves $\Delta$ and $\Gamma$
only.
%% TODO talk about embedding (fwd ref)???

\paragraph{Reduction rules}
Reduction rules are defined in \Fig{NaxSyntax}. First five rules are
the redex rules that makes an actual reduction step on redexes.
A redex is be one of the following: a lambda term applied to an argument,
a let binding, a case term applied to a constructor term,
a Mendler style iterator applied to an $\In^\kappa$-constructed term,
and a top level variable.

Note, the reduction rule for $`x$ mentions $\overline{D}$).
Although we illustrate the reduction as a relation on terms
($t\rightsquigarrow t'$), we implicitly assume that there exists
some fixed list of declarations ($\overline{D}$)
for the reduction relation ($\rightsquigarrow$).
In order to make a reduction step for top level variables,
we need to know the top level definition for $`x$, which
should be contained in $\overline{D}$. Since the list of declarations
are given by the input program ($\overline{D};t$) to type check,
it is non-ambiguous which $\overline{D}$ to use for reducing $`x$.
In case when it is ambiguous, we could use a notation like
$t\stackrel{\overline{D}}\rightsquigarrow t'$ to make it more precise.

The other rules, following the top level varaibel reduction rule
($`x\rightsquigarrow t$), are context rules to make a reduction step
for the terms whose redexes appear inside their subterms.

\paragraph{Equality over types and kinds}
TODO

\subsection{Syntax-directed type system}
The kinding and typing rules of Nax illustrated in \Fig{NaxTyping} is not
syntax directed since the conversion rules ($Conv$) and ($=$) are not syntax
directed. These conversion rules can apply to anywhere regardless of the
syntactic category of terms and types.

We can easly adapt the system to be syntax directed by embedding
the conversion rules into application-like rules (\eg, ($@i$), ($->$$E$)).
\KYA{We need to prove that the syntax directed system are equivalent to
the standard system modulo conversion}
Among the kinding rules, the only place where conversion is truly necessary
is in the index application rule ($@i$).
We can define the syntax directed application rule ($@i$)$_s$ as follows:
\[ \inference[($@i$)$_s$]{ \Sigma;\Delta |-_s F : A -> \kappa
                     & \Sigma;\Delta;\cdot |-_s s : A'
                     & \Sigma;\cdot |-_s A=A' : * }
                     {\Sigma;\Delta |-_s F\,\{s\} : \kappa}
\]
Among the typing rules, we need to embed conversion into the ($->$$E$) rule
and probably the rules (let) and (MIt), and the branch checking rule as well???
TODO

\subsection{Type inference (or, reconstruction)}

\newcommand{\return}[0]{\mathrel{\rhd}}
\newcommand{\fresh}[0]{\textit{fresh}}
\newcommand{\frvar}[0]{\textit{frvar}}
\newcommand{\unify}[0]{\textit{unify}}

\paragraph{Kinding:}
$ \fbox{$\Sigma;\Delta |- \sigma \return \kappa$ }
 ~~~~
   \inference[($\forall$)]{ X <- \frvar \\
                            \Sigma;\Delta, X^\kappa |- \sigma \return *}
                          {\Sigma;\Delta |- \forall X . \sigma \return *}
 ~~~~
   \inference[($\forall i$)]{ i <- \frvar \\
                              \Sigma;\Delta, i^A |- \sigma \return *}
                            {\Sigma;\Delta |- \forall i . \sigma \return *}
$
\[ \fbox{$\Sigma;\Delta |- F \return \kappa$}
 ~~~~
   \inference[($Var$)]{X^\kappa\in\Delta & \Sigma |- \Delta}
                      {\Sigma;\Delta |- X \return \kappa}
 ~~~~
   \inference[($TCon$)]{T:\kappa\in\Sigma & \Sigma |- \Delta}
                       {\Sigma;\Delta |- T \return \kappa}
 ~~~~
   \inference[($\mu$)]{\Sigma;\Delta |- T\,\overline\tau : \kappa -> \kappa}
                      {\Sigma;\Delta |- \mu^\kappa(T\,\overline\tau) : \kappa}
\]
\[ \inference[($@$)]{ \Sigma;\Delta |- F : \kappa -> \kappa'
                    & \Sigma;\Delta |- G : \kappa }
                    {\Sigma;\Delta |- F\,G : \kappa'}
 ~~~~
   \inference[($@i$)]{ \Sigma;\Delta |- F : A -> \kappa
                     & \Sigma;\Delta;\cdot |- s : A }
                     {\Sigma;\Delta |- F\,\{s\} : \kappa}
\]
\[ \inference[($->$)]{\Sigma;\Delta |- A : * & \Sigma;\Delta |- B : *}
                     {\Sigma;\Delta |- A -> B : * }
\]


\paragraph{Typing:}
\fbox{$\Sigma;\Delta;\Gamma |- t \rhd A$}
\[ \inference[($:$)]
      { x:\sigma \in \Gamma \\ A <- inst(\sigma) }
      {\Sigma;\Delta;\Gamma |- x \return A}
 ~~~
   \inference[($:i$)]
      { i^\sigma \in \Delta \\ A <- inst(\sigma) }
      {\Sigma;\Delta;\Gamma |- i \return A}
 ~~~
   \inference[($:C$)]
      { C:\sigma \in \Sigma \\ A <- inst(\sigma) }
      {\Sigma;\Delta;\Gamma |- C \return A}
 ~~~
 \inference[($:`$)]
      { `x:\sigma \in \Sigma \\ A <- inst(\sigma) }
      {\Sigma;\Delta;\Gamma |- `x \return A}
\]
\[ \inference[($->$$I$)]
      { A <- \fresh \\ \Sigma;\Delta;\Gamma,x:A |- t \return B }
      {\Sigma;\Delta;\Gamma |- \lambda x.t \return A -> B}
 ~~~~
   \inference[($->$$E$)]{ \Sigma;\Delta;\Gamma |- r \return A'
                        & \Sigma;\Delta;\Gamma |- s \return A \\
                          B <- \fresh & \unify(A',\;A -> B) }
                        {\Sigma;\Delta;\Gamma |- r\;s \return B}
\]
\[ \inference[(let)]
      { \Sigma;\Delta;\Gamma |- s \return A & \sigma <- gen(A) \\
        \Sigma;\Delta;\Gamma,x:\sigma |- t \return B }
      {\Sigma;\Delta;\Gamma |- \<let> x=s \<in> t \return B }
 ~~~~  
   \inference[(In)]
      { F <- \fresh & \overline\tau <- \fresh^{|\kappa|} }
      {\Sigma;\Delta;\Gamma |- \In^\kappa \return
         F(\mu^\kappa F)\,\overline{\tau} -> \mu^\kappa F\;\overline{\tau} }
\]
\[ \inference[(case)]
      { \Sigma;\Delta;\Gamma |-^\psi \varphi \return \sigma \\
         A <- inst(\sigma) }
      {\Sigma;\Delta;\Gamma |- \varphi^\psi \return A }
 ~~~
   \inference[(MIt)]
      { \overline{K} <- \fresh^{|\psi|} & X <- \frvar
      & \kappa = \overline{K} -> * & \overline{i'} <- \frvar^{|\psi|} \\
        \Sigma;\Delta,X^\kappa;
               \Gamma,x:\forall\,\overline{\iota'} . X\overline{\iota'}
                                              -> \psi(\overline{\iota'})
           |-^\psi \varphi \return \sigma
      & A <- inst(\sigma) \\
        \overline\tau <- \fresh^{|\psi|} & F <- \fresh
      & \unify( A,\; F\,X\,\overline\tau -> \psi(\overline\tau) ) }
      {\Sigma;\Delta;\Gamma
           |- \<MIt> x.\varphi^\psi
           : \mu^{\kappa}\,F\;\overline\tau -> \psi(\overline\tau) }
\]

\fbox{$\Sigma;\Delta;\Gamma |-^\psi \varphi \return \sigma$}
\[ \inference
      { \Sigma|_T = \overline{C_k:\sigma_k}^{\;k=1..n}
      & \Sigma;\cdot |- T \return \kappa
      & \overline\tau <- \fresh^{|\kappa| - |\psi|}
      & \overline{\textit{InferBranch}}^{\;k=1..n}
      & \overline\iota <-\fresh^{|\psi|} }
      { \Sigma;\Delta;\Gamma
           |-^\psi \overline{C_k\;\overline{x} -> t}^{\;k=1..n} \return
             \forall\,\overline\iota . T\,\overline{\tau}\,\overline\iota ->
                                       \psi(\overline\iota) }
\]
\begin{align*}
\text{where} ~~~~
\textit{InferBranch} ~\defeq~
&~ \overline{\tau}_k <- \fresh^{|\psi|} \qquad
   B <- inst(\sigma_k) \qquad 
   \overline{A} <- \fresh^{|B|} \\
&~ \unify(B,\; \overline{A} -> T\,\overline{\tau}\,\overline{\tau}_k) \quad
   \Sigma;\Delta;\Gamma,\overline{x:A} |- t \return B' \quad
   \unify(B',\; \psi(\overline{\tau}_k))
\end{align*}

