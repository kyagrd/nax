\documentclass{llncs}
\usepackage[numbers]{natbib}

\title{The Nax programming language \\
	{\small\rm(extended abstract)}}
\titlerunning{The Nax programming language}
\author{Ki Yung Ahn\inst{1} \and Tim Sheard\inst{1} \and
	Author2 SName2\inst{2} % \and Author2 SName2\inst{2}
	}
\institute{
	Portland State University, Portland, Oregon, USA \thanks{supported by NSF grant 0910500.}
	\\ \email{kya@cs.pdx.edu} \qquad \email{sheard@cs.pdx.edu}
	\and
	University of Cambridge, Cambridge, UK
	\\ \email{todo@cl.cam.ac.uk} % \and \email{todo@cl.cam.ac.uk}
	}

\newcommand{\Wi}{\ensuremath{W_i}}

\newcommand{\Fi}{\ensuremath{\mathsf{F}_i}}
\newcommand{\Fw}{\ensuremath{\mathsf{F}_\omega}}

\begin{document}
\maketitle
%% \begin{abstract}
%% 	Write abstract here. Write abstract here.
%% 	Write abstract here. Write abstract here.
%% 	Write abstract here. Write abstract here.
%% 	Write abstract here. Write abstract here.
%% 	Write abstract here. Write abstract here.
%% 	Write abstract here. Write abstract here.
%% 	Write abstract here. Write abstract here.
%% 	Write abstract here. Write abstract here.
%% \keywords{asdf, asdf, asdf, asdf}
%% \end{abstract}

\section{Introduction}

proof assistants mechanized theorem proving
Curry-Howard correspondence.

A light-weight verification
indexed datatypes
Programming with indexed datatypes has become popular in functional programming
due to the support of the Generalized Algebraic Data Type (GADT) extension to
Haskell in Glasgow Haskell Compiler (GHC).

\subsection{Indexed datatypes}

Type-indexed datatypes

Term-indexed datatypes

\subsection{Mendler-style recursion combinators}

TODO

\subsection{Contributions}

TODO

\section{Syntax}
TODO

\section{Typing rules}
We define two set of typing rules

declarative typing rules
syntax-directed typing rules

\cite{Girard72} dummy citation
\cite{AhnShe11}

\citet{Girard72} citet test
\citet{AhnShe11} citet test

\citeauthor{Girard72} citeauthor test
\citeauthor{AhnShe11} citeauthor test

\section{Embedding Nax into System \Fi}

\section{The type inference algorithm \Wi}

\section{Implementation}

\section{Ongoing and future work}

\subsection{Correctness of the type inference algorithm}

\subsection{Large eliminations}

\subsection{Generalized arrow types in the recursion combinators}

\subsection{Non-logical language fragments}

%% \bibliographystyle{splncs03}
\bibliographystyle{splncsnat}
\bibliography{main}

\end{document}
