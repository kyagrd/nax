\documentclass[final]{beamer}
%% beamer 3.10: do NOT use option hyperref={pdfpagelabels=false} !

\mode<presentation> {
%% examples http://www-i6.informatik.rwth-aachen.de/~dreuw/latexbeamerposter.php
% \usetheme{Berlin}
%% you should define your own theme e.g. for big headlines using your own logos 
  \usetheme{I6pd2}
}
\usepackage[english]{babel}
\usepackage[latin1]{inputenc}
\usepackage{listings}
\usepackage{amsmath,amsthm,amssymb,latexsym,textcomp,setspace}
%\usepackage{times}\usefonttheme{professionalfonts}  % times is obsolete
\usefonttheme[onlymath]{serif}
\boldmath
\usepackage[orientation=landscape,size=a0,scale=1.4,debug]{beamerposter}

\newcommand{\Fi}{\ensuremath{\mathsf{F}_{\!i}}}

\title[Fancy Posters]{{\VERYHuge System \Fi} {\Huge :}
		{\huge a Higher-Order Polymorphic $\lambda$-calculus} \\
		{\huge with Erasable Term Indices}
  }
\author[Ki Yung Ahn]{ Ki Yung $\,$Ahn $\,$ {\Large\texttt{kya@cs.pdx.edu}} }
\institute[Portland State University]{
	Department of Computer Science, Portland State University }
\date{2012-09}
\begin{document}
\begin{frame}[fragile]
%%   \vfill
%%   \begin{block}{Fontsizes}
%%     \centering
%%     {\tiny tiny}\par
%%     {\scriptsize scriptsize}\par
%%     {\footnotesize footnotesize}\par
%%     {\normalsize normalsize}\par
%%     {\large large}\par
%%     {\Large Large}\par
%%     {\LARGE LARGE}\par
%%     {\Huge Huge}\par
%%     {\veryHuge veryHuge}\par
%%     {\VeryHuge VeryHuge}\par
%%     {\VERYHuge VERYHuge}\par
%%   \end{block}
%% \vfill
\begin{columns}[t]

\begin{column}{.48\linewidth}

\begin{block}{Indexed Datatypes (a.k.a. Lightweight Dependent Types)}
\begin{itemize}
\item Indexed datatyeps are non-regular datatypes
	with \emph{static (or, compile-time) dependencies}.\\
        (c.f. full-fledged dependent datatypes can have
        both static and dynamic dependencies) \\
	Also known as higer-kinded datatypes, higher-rank datatypes,
	or lightweight dependent types
\item Use of indexed datatypes, or the lightweight approach,
	has become popular over the past decade
	in even in real-world functional programming
	due to the support of the GADT extension in Glasgow Haskell Compiler.
\item Type-indexed datatypes
	\begin{itemize}
		\item Nested datatypes $\qquad$
			\lstinline[language=Haskell,
				basicstyle=\ttfamily\small,
				keywordstyle=\color{green}]!data Powl a = PCons Powl (a,a) | PNil!
		\item Representation types in datatype geneirc programming
		\begin{lstlisting}[language=Haskell,
				basicstyle=\ttfamily\small,
				keywordstyle=\color{green},
				literate={::}{$:\!\,:$}1{->}{$\to$}1]
		data Rep t where
		  RInt  :: Rep Int
		  RBool :: Rep Bool
		  RPair :: Rep a -> Rep b -> Rep (a,b)
		  RFun  :: Rep a -> Rep b -> Rep (a -> b)
		\end{lstlisting}
	\end{itemize}
\item Term-indexed datatypes
	\begin{itemize}
        	\item length-indexed list
		\begin{lstlisting}[language=Haskell,
			basicstyle=\ttfamily\small,
			keywordstyle=\color{green},
			literate={::}{$:\!\,:$}1{->}{$\to$}1]
		data Vec (a :: *) {n :: Nat} where
		  VCons :: a -> Vec a {i} -> Vec a {Succ i}
		  VNil  :: Vec {Zero}
		\end{lstlisting}
		\item de Bruijn terms indexed by the size of their context
		\begin{lstlisting}[language=Haskell,
			basicstyle=\ttfamily\small,
			keywordstyle=\color{green},
			literate={::}{$:\!\,:$}1{->}{$\to$}1]
		data BTerm (c :: Nat -> *) n where
		  BVar :: c {i} -> BTerm c {i}
		  BApp :: BTerm c {i} -> BTerm {i} -> BTerm c {i}
		  BAbs :: BTerm c {Succ i} -> BTerm c {i}
		\end{lstlisting}
	\end{itemize}
\end{itemize} %% $$\alpha=\gamma, \sum_{i}$$
\end{block}

\begin{block}{Introduction}
\begin{itemize}
\item some items
\item some items
\item some items
\item some items
\end{itemize}
\end{block}

\end{column}
%%%%%%%%%%%%%%%%%%%%%%%%% column sep %%%%%%%%%%%%%%%%%%%%%%%%%%%%%%%%%%%%%
\begin{column}{.48\linewidth}
\begin{block}{Introduction}
\begin{itemize}
\item some items
\item some items
\item some items
\item some items
\end{itemize}
\end{block}

\begin{block}{Introduction}
\begin{itemize}
\item some items and $\alpha=\gamma, \sum_{i}$
\item some items
\item some items
\item some items
\end{itemize}
$$\alpha=\gamma, \sum_{i}$$
\end{block}

\end{column}
\end{columns}

\end{frame}
\end{document}
