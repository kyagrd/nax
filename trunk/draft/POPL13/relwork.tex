\section{Related work}
\label{sec:relwork}
Among theoretical calculi, \Fi\ is most closely related to
Curry-style System \Fw \cite{AbeMatUus03,AbeMatUus05,GHR93}
and Implicit Calculus of Constructions (ICC) \cite{Miquel01}.
All terms typable in Curry-style System \Fw\ are typable in System \Fi\ 
with the same type, and all terms typable in \Fi\ are typable in ICC
with the same type.\footnote{The $*$ kind in \Fw\ and \Fi\ corresponds
	to \textsf{Set} in ICC.}
So, the subject reduction and strong normalization of \Fi\ 
can be automatically derived from ICC. ICC is more than just an extension of
\Fi, as described in our work, with dependent types and stratified universes,
since ICC includes $\eta$-reduction and the extensionality typing rule.
We do not foresee any problem of adding $\eta$-reduction and
the extensionality typing rule to \Fi. Although System \Fi\ accepts
less terms than ICC, \Fi\ enjoys a stronger erasure property
(Theorem \ref{thm:ierasetyping}), which ICC cannot not provide
due to its support for full dependent types. In System \Fi, index terms
appearing in types (\eg, $s$ in $F\{s\}$) are always erasable.
\citet{LingerS08} formalized a generic framework which describes the erasure on
arbitrary Church-style calculi (EPTS) and Curry-style calculi (IPTS).

In \S\ref{ssec:rationale}, we have mentioned that Curry-style calculi enjoys
better reduction properties (\eg,$\beta\eta$-reduction is Church-Rosser)
than Church-style calculi. \citet{Nederpelt73} showed a counterexample to
the Church-Rosser property for $\beta\eta$-reduction of Church-style terms.
\citet{Geuvers92} proved that $\beta\eta$-reduction is Church-Rosser
in functional PTSs, which is a certain class of Church-style calculi.
\citet{Seldin08} discusses the relation between the Church-style typing
and the Curry-style typing.

In a more practical setting for language implementation,

\citet{YorgeyWCJVM12}
"Giving Haskell a Promotion" 
most closely related work would be this

\citet{Swamy11}
value dependent types in F-star  from MSR


what others to discuss?

