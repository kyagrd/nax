\section{Encodings of Term-Indexed Datatypes} \label{sec:data}
Recall that our motivation was a foundational calculus
that can encode term-indexed datatypes. In \S\ref{sec:motiv},
we showed that there exist Church encodings of
{\small\tt List} (a regular datatype),
{\small\tt Powl} (a type-indexed datatype), and
{\small\tt Vec} (a term-indexed datatype).
In this section, we discuss briefly about a more complicated example
of a datatype with nested term-indices, and, encoding schemes that
better relates to practical implementations --
encoding indexed datatypes using equality constraints
and the Mendler-style encoding.

\paragraph{Nested term-indices\,$:$} One great merit of System \Fi\ is
the ability to express datatypes with nested term-indices. That is,
types of term-indices in indexed datatypes themselves can be yet
another indexed datatype. For instance, consider the resource-state
tracking environment \cite{BraHam10} below (in Nax-like syntax):
\vspace*{-3pt}
\begin{lstlisting}[basicstyle={\ttfamily\small},language=Haskell]
   data Env : ({st} -> *) -> {Vec st {n}} -> * where
     Extend : res {x} -> Env res {xs} -> Env res {VCons x xs}
     Empty  : Env res {VNil}
\end{lstlisting}
\vskip-.7ex\noindent
Note that {\small\tt Env} has a term-index of type {\small\tt Vec},
which is again indexed by {\small\tt n}.
For simplicity,\footnote{Nax \cite{AhnSheFioPit12} supports
	rank-1 kind-level polymorphism. %% over kinds, types, and term-indices.
	It would be virtually useless if nested term-indices % like {\texttt{n}}
	were only limited to constants rather than polymorphic variables.}
assume that {\small\tt n} is some fixed constant (\eg, {\small\tt S(S(S Z))})
and that {\small\tt st} is some fixed resource-state type (\eg, read--write
access modes on files). Provided that {\small\tt st} is expressible in \Fi,
we can encode {\small\tt Env} as follows:
\vspace*{-5pt}
\begin{multline*}
\texttt{Env} \triangleq
\l Y^{\,\texttt{st} -> *}.\,\l i^{(\texttt{Vec}\;\texttt{st}\;\texttt{n})}.\,
\forall X^{(\texttt{Vec}\;\texttt{st}\,\{\texttt{n}\}) -> *}.\, \\
( \forall j^\texttt{st}\!.\,
  \forall k^{(\texttt{Vec}\;\texttt{st}\;\texttt{n})}\!.\,
  Y\!\{j\} -> X\!\{k\} ->
				X\!\{\!\texttt{\small VCons}\;j\,k\!\} )
-> X \{\texttt{\!\small VNil\!}\} -> X\!\{i\}
\end{multline*}
\vskip-1ex\noindent
The term encodings for {\small\tt Extend} and {\small\tt Empty} are
exactly the same as the term encodings for {\small\tt Cons} and {\small\tt Nil}
of the {\small\tt List} datatype in \S\ref{sec:motiv}.

\paragraph{Encoding indexed datatypes using equality constraints\,$:$}
Systematic encodings of GADTs \cite{Sheard04equality,Crary98}, which are
used in practical implementations, typically involve equality constraints
and existential quantification. Here, we want to emphasize that such encoding
schemes are expressible within System \Fi, since it is possible to define
equalities and existentials over both types and term-indices in \Fi.

It is well known that Leibniz equality over type constructors
can be defined within System \Fw\ as 
$
(\stackrel{\kappa}{=}) \triangleq \l X_1^\kappa.\,\l X_2^\kappa.\,
     \forall X^{\kappa\to*}.\, X X_1 -> X X_2
$.
Similarly, Leibniz equality over term-indices is defined as
$
(\stackrel{A}{=}) \triangleq
        \l i^A.\, \l j^A.\, \forall X^{A\to*}.\, X\{i\}\to X\{j\}
$
in System \Fi.
Then, we can encode {\small\tt Vec} as sum of its two data constructor types:
\vspace*{-5pt}
\[ \mathtt{Vec} \triangleq \l A^{*}.\,\l i^\texttt{Nat}.\,\forall X^{\texttt{Nat} -> *}.\,
	(\exists j^\texttt{Nat}.\,(\texttt{S}\;j\stackrel{\texttt{Nat}}{=}i)\times A \times X\{j\})
        +
	(\texttt{Z}\stackrel{\texttt{Nat}}{=}i)
\] \vskip-.7ex\noindent
where $+$ and $\times$ are the usual impredicative encoding of sums and
products. We can encode the existental quantification over indices 
($\exists$ used in the encoding of {\small\tt Vec} above) as
$ \exists i^A.B \triangleq \forall X^{*}. (\forall i^A.B -> X) -> X $,
which is similar to the usual encoding of existential quantification
over types in System $\mathsf{F}$ or \Fw.

Compared to the simple Church encoded versions in \S\ref{sec:motiv},
the encodings using equality constraints work particularly well when
defining functions that constrain their domain types by restricting
their indices, such as the function $\texttt{safeTail} :
	\texttt{Vec}\;a\;\{\texttt{S}\;n\} -> \texttt{Vec}\;a\;\{n\}$,
which can only be applied to non-empty length indexed lists due
the index of the domain type ($\texttt{S}\;n$).

\paragraph{The Mendler-style encoding\,$:$}
TODO things to say

can encode 

recursion at type-level is factorized by recursive type operator,
and encodings of eliminators for recursive types become uniform 
-- in a logical view, inductive reasoning become better mondularized

strong normalization while supporting negative datatypes
(\eg, higher-order abstract syntax, semantic domains of lambda calculi)

\noindent
CHECKING SPACE 1\\
CHECKING SPACE 2\\
CHECKING SPACE 3

