\section{Encodings of Term-Indexed Datatypes} \label{sec:data}
Recall that our motivation was a foundational calculus
that can encode term-indexed datatypes. In \S\ref{sec:motiv},
we gave Church encodings of
{\small\tt List} (a regular datatype),
{\small\tt Powl} (a type-indexed datatype), and
{\small\tt Vec} (a term-indexed datatype).
In this section, we discuss a more complicated datatype \cite{BraHam10}
involving nested term-indices, and several encoding schemes
that we have seen used in practice --
first, encoding indexed datatypes using equality constraints
\cite{Sheard04equality,Crary98} and second, encoding datatypes
in the Mendler-style \cite{AbeMatUus05,AhnShe11}.
\vspace*{-1ex}
\paragraph{Nested term-indices\,$:$} System \Fi\ is able to express
datatypes with nested term-indices. That is, term-indices, which are
themselves term-indexed datatypes. Consider the resource-state
tracking environment \cite{BraHam10} in Nax-like syntax below:
\vspace*{-3pt}
\begin{lstlisting}[basicstyle={\ttfamily\small},language=Haskell]
   data Env : ({st} -> *) -> {Vec st {n}} -> * where
     Extend : res {x} -> Env res {xs} -> Env res {VCons x xs}
     Empty  : Env res {VNil}
\end{lstlisting}
\vskip-.7ex\noindent
Note that {\small\tt Env} has a term-index of type {\small\tt Vec},
which is again indexed by {\small\tt Nat}.
For simplicity,\footnote{Nax supports
        rank-1 kind-level polymorphism. %% over kinds, types, and term-indices.
        It would be virtually useless if nested term-indices % like {\texttt{n}}
        were only limited to constants rather than polymorphic variables.}
assume that {\small\tt n} is some fixed constant (\eg, {\small\tt S(S(S Z))},
\ie, 3). Then, an {\small\tt Env} tracks 3 independent resources
({\small\tt res}), each which could be in a different state ({\small\tt st}).
For example, 3 files in different states -- one open for reading,
the next open for writing, and the third closed.
%% Where {\small\tt st}
%% is the type of these resources states (\eg, OpenRead, OpenWrite, Closed).
We can encode {\small\tt Env} in \Fi\ as follows:
\vspace*{-5pt}
\begin{multline*}
\texttt{Env} \triangleq
\l Y^{\,\texttt{st} -> *}.\,\l i^{(\texttt{Vec}\;\texttt{st}\;\texttt{n})}.\,
\forall X^{(\texttt{Vec}\;\texttt{st}\,\{\texttt{n}\}) -> *}.\, \\
( \forall j^\texttt{st}\!.\,
  \forall k^{(\texttt{Vec}\;\texttt{st}\;\texttt{n})}\!.\,
  Y\!\{j\} -> X\!\{k\} ->
                                X\!\{\!\texttt{\small VCons}\;j\,k\!\} )
-> X \{\texttt{\!\small VNil\!}\} -> X\!\{i\}
\end{multline*}
\vskip-1ex\noindent
The term encodings for {\small\tt Extend} and {\small\tt Empty} are
exactly the same as the term encodings for {\small\tt Cons} and {\small\tt Nil}
of the {\small\tt List} datatype in \S\ref{sec:motiv}.
\vspace*{-1ex}
\paragraph{Encoding indexed datatypes using equality constraints\,$:$}
Systematic encodings of GADTs \cite{Sheard04equality,Crary98}, which are
used in practical implementations, typically involve equality constraints
and existential quantification. Here, we want to emphasize that such encoding
schemes are expressible within System \Fi, since it is possible to define
equalities and existentials over both types and term-indices in \Fi.

It is well known that Leibniz equality over type constructors
can be defined within System \Fw\ as 
$
(\stackrel{\kappa}{=}) \triangleq \l X_1^\kappa.\,\l X_2^\kappa.\,
     \forall X^{\kappa\to*}.\, X X_1 -> X X_2
$.
Similarly, Leibniz equality over term-indices is defined as
$
(\stackrel{A}{=}) \triangleq
        \l i^A.\, \l j^A.\, \forall X^{A\to*}.\, X\{i\}\to X\{j\}
$
in System \Fi.
Then, we can encode {\small\tt Vec} as sum of its two data constructor types:
\vspace*{-5pt}
\[ \mathtt{Vec} \triangleq \l A^{*}.\,\l i^\texttt{Nat}.\,\forall X^{\texttt{Nat} -> *}.\,
        (\exists j^\texttt{Nat}.\,(\texttt{S}\;j\stackrel{\texttt{Nat}}{=}i)\times A \times X\{j\})
        +
        (\texttt{Z}\stackrel{\texttt{Nat}}{=}i)
\] \vskip-.7ex\noindent
where $+$ and $\times$ are the usual impredicative encoding of sums and
products. We can encode the existential quantification over indices 
($\exists$ used in the encoding of {\small\tt Vec} above) as
$ \exists i^A.B \triangleq \forall X^{*}. (\forall i^A.B -> X) -> X $,
which is similar to the usual encoding of the existential quantification
over types in System $\mathsf{F}$ or \Fw.

Compared to the simple Church encoded versions in \S\ref{sec:motiv},
the encodings using equality constraints work particularly well with
encodings of functions that constrain their domain types by restricting
their indices. For instance, the function $\texttt{safeTail} :
        \texttt{Vec}\;a\;\{\texttt{S}\;n\} -> \texttt{Vec}\;a\;\{n\}$,
which can only be applied to non-empty length indexed lists due
the index of the domain type ($\texttt{S}\;n$).
\vspace*{-1ex}
\paragraph{The Mendler-style encoding\,$:$}
Recursive type theories that extend higher-order polymorphic lambda calculi
typically comes with a built-in recursive type operator
$\mu_\kappa : (\kappa -> \kappa) -> \kappa$ for each kind $\kappa$,
which yields recursive types ($\mu_\kappa\,F : \kappa$) when applied to
type constructors of appropriate kind ($F : \kappa -> \kappa$).
For instance,
$\texttt{List} \triangleq \l Y^{*}\!.\,\mu_{*}(\l X^{*}.Y\times X+ \mathbbm{1})$
where $\mathbbm{1}$ is the unit type. One benefit of factoring out
the recursion at type-level (\eg, $\mu_{*}$) from
the base structure (\eg, $\l X^{*}.Y\times X+ \mathbbm{1}$) of
recursive types is that such factorized (or, two-level) representations
are more amenable to express generic recursion schemes (\eg, catamorphism)
that work over different recursive datatypes.
Interestingly, there exists an encoding scheme, namely the Mendler style,
which can embed $\mu_{\kappa}$ within Systems like \Fw\ or \Fi.
The Mendler-style encoding can keep the theoretical basis small,
while enjoying the benefits of factoring out the recursion at type-level.

% exploiting impredicative polymorphism.
%% Thus, in the Mendler style encoding,
%% $\mu_\kappa$ no more needs to be primitive,
%% and encoding scheme of eliminators for recursive types become uniform 
%% -- in a logical view, inductive reasoning become better modularized
%% 
%% strong normalization while supporting negative datatypes
%% (\eg, higher-order abstract syntax, semantic domains of lambda calculi)

