\section{Related Work} \label{sec:relwork}
System~\Fi\ is most closely related to
Curry-style System~\Fw \cite{AbeMatUus05,GHR93}
and the Implicit Calculus of Constructions (ICC) \cite{Miquel01}.
All terms typable in a Curry-style System \Fw\ are typable (with the same type) in System \Fi\ 
and all terms typable in \Fi\ are typable (with the same type\footnote{The $*$ kind in \Fw\ and \Fi\ corresponds
	to \textsf{Set} in ICC}) in ICC.

As mentioned in \label{ssec:sn}, we can derive strong normalization of \Fi\ 
from System~\Fw, and derive logical consistency of \Fi\ from certain
restrictions of ICC \cite{Miquel00,BarrasB08}.
In fact, ICC is more than just an extension of System~\Fi\ 
with dependent types and stratified universes, since ICC includes
$\eta$-reduction and $\eta$-equivalence.
We do not foresee any problems adding
$\eta$-reduction and $\eta$-equivalence to System~\Fi.
Although System~\Fi\ accepts fewer terms than ICC, it enjoys simpler
erasure properties (Theorem \ref{thm:ierasetypingall} and
Corollary \ref{cor:ierasetypingifree}) just by looking at the syntax
of kinds and type type, which ICC cannot enjoy due to its support for
full dependent types.  In System \Fi, term indices appearing in types
(\eg,~$s$ in $F\{s\}$) are always erasable.  \citet{LingerS08} generalized the ICC framework to one which describes
erasure on arbitrary Church-style calculi~(EPTS) and Curry-style
calculi~(IPTS), but only consider $\beta$-equivalence for type conversion.


In the practical setting of programming language implementation,
\citet{YorgeyWCJVM12}, inspired by \citet{SHE}, recently designed an extension
to Haskell's GADTs by allowing datatypes to be used as kinds. For instance,
\texttt{Bool} is promoted to a kind (\ie, $\texttt{Bool}:\square$) and its
data constructors \texttt{True} and \texttt{False} are promoted to types.
They extended System $F_{\!C}$ (the Glasgow Haskell Compiler's
intermediate core language) to support \emph{datatype promotion}
and named it System~$F_{\!C}^\uparrow$. The key difference between
$F_{\!C}^\uparrow$ and \Fi\ is in their kind syntax: %%, as illustrated below:
\vspace*{-2pt}
\[\qquad\quad
\begin{array}{ll}
F_{\!C}^\uparrow\,\textbf{kinds}\quad &
\kappa ::= * \mid \kappa -> \kappa \mid F \vec{\kappa} \mid \mathcal{X} \mid \forall \mathcal{X}.\kappa \mid \cdots \\
\,\Fi\,\,\textbf{kinds}\quad &
\kappa ::= * \mid \kappa -> \kappa \mid A -> \kappa \phantom{A^{A^A}}
\end{array}  
\] ~\vspace*{-6pt}\\
In $F_{\!C}^\uparrow$, all type constructors ($F$) are promotable to the 
kind level and become kinds when fully applied to other kinds
($F\vec\kappa$). On the other hand, in \Fi,  a type can only appear
as the domain of an index arrow kind ($A-> \kappa$).
The ramifications of this difference is that $F_{\!C}^\uparrow$ can
express type level data structues but not nested term indices,
while \Fi\ is the other way around. The promotion of
a type constructor, for instance, $\texttt{List}:* -> *$ to a kind constructor
$\texttt{List}:\square-> \square$ enables type-level data structures
such as $\mathtt{[Nat,Bool,Nat-> Bool]:List\,*}$. Type-level
data structures motivate type-level computations over promoted data.
This is made possible by type families\footnote{
	A GHC extension to define type-level functions in Haskell.}.
The promotion of polymorphic types naturally motivates
kind polymorphism ($\forall \mathcal{X}.\kappa$), which is known to
break strong normalization and cause logical inconsistency \cite{Girard72}.
In a functional {\em programming language}, inconsistency is not an issue.
However, when studying logically consistent systems, we need
a more conservative approach, as in System \Fi.

