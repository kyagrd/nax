\section{Motivation: from System~\Fw\ to System~\Fi, and back}
\label{sec:motiv}
It is well known that datatypes can be embedded into polymorphic lambda
calculi by means of functional encodings
%% such as the Church and Boehm-Berarducci encodings~
\cite{BoehmBerarducci}.

In System~\textsf{F}, one can embed \emph{regular datatypes},
like homogeneous lists:\vspace*{-3pt}
\[
\begin{array}{ll}
\text{Haskell:} & \texttt{data List a = Cons a (List a) | Nil} \\
\text{System \textsf{F}:}~& 
\texttt{{List}}\:\: A\:\:\triangleq\:\:
\forall X.
(A\to X\to X) \to X \to X ~~\; \\
&
\quad~~
\texttt{Cons} \triangleq \l w.\l x.\l y.\l z.\,y\;w\,(x\;y\;z),~
\texttt{Nil} \triangleq \l y.\l z.z
\end{array}
\]\vskip-1ex\noindent
In such regular datatypes, constructors have algebraic structure that
directly translates into polymorphic operations on abstract types as
encapsulated by universal quantification over types (of kind $*$).

In the more expressive System \Fw\ (where one can abstract over
type constructors of any kind), one can encode more general
\emph{type-indexed datatypes} that go beyond the regular datatypes.
For example, one can embed powerlists with heterogeneous elements
in which an element of type \texttt{a} is followed by
an element of the product type \texttt{(a,a)}:\vspace*{-3pt}
\[
\begin{array}{ll}
\text{Haskell:} & \texttt{data Powl a = 
        PCons a (Powl(a,a))
        | 
        PNil 
} \\
& \!\!\!\!\!\!\!\!\!
  \textcolor{gray}{\small\texttt{-- PCons 1 (PCons (2,3) (PCons ((3,4),(1,2)) PNil)) :: Powl Int}}\\
\text{System \Fw:}~& \texttt{{Powl}}\:\triangleq\:
\lambda A^{*}.\forall X^{*\to*}. (A\to X(A\times A)\to X A) \to X A \to XA
\end{array}
\]\vskip-1ex\noindent
Note the non-regular occurrence (\texttt{Powl(a,a)}) in the definition of
(\texttt{Powl a}), and the use of universal quantification over
higher-order kinds ($\forall X^{*\to*}$).
The term encodings for {\small\tt PCons} and {\small\tt PNil} are exactly
the same as the term encodings for {\small\tt Cons} and {\small\tt Nil},
but have different types.

What about term-indexed datatypes?  What extensions to System~\Fw\ are
needed to embed term-indices as well as type-indices?  Our answer is
System~\Fi.

In a functional language supporting term-indexed datatypes, we envisage
that the classic example of homogeneous length-indexed lists would be
defined along the following lines (in Nax\footnote{We are developing
a language called Nax whose theory is based on System \Fi.}-like syntax):
\vspace{-3pt}
\begin{lstlisting}[basicstyle={\ttfamily\small},language=Haskell]
   data Nat = S Nat | Z 
   data Vec : * -> Nat -> * where
     VCons : a -> Vec a {i} -> Vec a {S i}
     VNil  : Vec a {Z}
\end{lstlisting}\vskip-1ex\noindent
Here the type constructor~{\tt Vec} is defined to admit parameterisation
by both type and term-indices.  For instance, the type 
(\verb|Vec (List Nat) {S (S Z)}|) is that of two-dimensional
vectors of natural numbers.  By design, our syntax directly
reflects the difference between type and term-indexing by enclosing
the latter in curly braces.  We also make this distinction in 
System~\Fi, where it is useful within the type system
to guarantee the static nature of term-indexing.

The encoding of the vector datatype in System~\Fi\ is as follows:\vspace*{-3pt}
\begin{equation*}\label{FiVecType}
\texttt{{Vec}}
\triangleq
\begin{array}[t]{l}
\lambda A^{*}.\lambda
i^{\texttt{{Nat}}}.  \forall X^{\texttt{{Nat}}\to *}.
  (\forall j^{\texttt{{Nat}}}.A\to X\{j\}\to X\{\texttt{{S}}\; j\})
  \to X\{\texttt{{Z}}\}
    \to X\{i\}
\end{array}
\end{equation*}\vskip-1ex\noindent
where $\texttt{{Nat}}$, $\mathtt Z$, and $\mathtt S$ respectively encode
the natural number type and its two constructors,  zero and successor.
Again, the term encodings for {\small\tt VCons} and {\small\tt VNil} are exactly
the same as the encodings for {\small\tt Cons} and {\small\tt Nil},
but have different types.

Without going into the details of the formalism, which are given in the
next section, one sees that such a calculus incorporating term-indexing
structure needs four additional constructs (see \Fig{FiSyntax} for the
highlighted extended syntax).
\begin{enumerate}
\item 
  Type-indexed kinding~($A\to\kappa$), as in $(\texttt{{Nat}\ensuremath{\to}*})$
  in the example above, where the compile-time nature of term-indexing
  will be reflected in the typing rules, enforcing that $A$ be a closed
  type~(rule~$(Ri)$ in \Fig{FiTyping}).

\item 
  Term-index abstraction~$\lambda i^A.F$~(as $\lambda
  i^{\texttt{{Nat}}}.\cdots$ in the example above) for constructing (or
  introducing) term-indexed kinds (rule~$(\lambda i)$ in
  \Fig{FiTyping}).  

\item 
  Term-index application~$F\{s\}$ (as $X\{{\tt Z}\}$, $X\{j\}$, and
  $X\{\texttt{S}\;j\}$ in the example above) for destructing (or
  eliminating) term-indexed kinds, where the compile-time nature of
  indexing will be reflected in the typing rules, enforceing that the index be
  statically typed% in that it does not depend on run-time parameters
~(rule~$(@i)$ in \Fig{FiTyping}) .

\item 
  Term-index polymorphism~$\forall i^A.B$ (as $\forall
  j^{\texttt{{Nat}}}.\cdots$ in the example above) where the compile-time
  nature of polymorphic term-indexing will be reflected in the typing rules enforcing
  that the variable~$i$ be static of closed type~$A$~(rule~$(\forall
  Ii)$ in \Fig{FiTyping}).
\end{enumerate}

As described above, System~\Fi\ maintains a clear-cut separation between
type-indexing and term-indexing.  This adds a level of abstraction
to System~\Fw\ and yields types that in addition to parametric polymorphism
also keep track of inductive invariants using term-indices. All term-index
information can be erased, since it is only used at compile-time.  
It is possible to project any well-typed System~\Fi\ term into a well-typed System~\Fw\ term.
For instance, the erasure of the \Fi-type~\texttt{Vec}
is the \Fw-type~\texttt{List}.  This is established in
\S\ref{sec:theory} and used to deduce the strong normalization of
System~\Fi.


\section{Why Term-Indexed Calculi? {\small(rather than dependent types)}}
We claim that a moderate extension to the polymorphic calculus (\Fw)
is a better candidate than a dependently typed calculus for the basis of
a practical programming system. We hope to design
a unified system for programming as well as reasoning.
Language designs based on indexed types can benefit from
existing compiler technology and type inference algorithms for
functional programming languages. In addition, theories for
term-indexd datatypes are simpler than theories for full-fledged
dependent datatypes, because term-indexd datatypes can be encoded
as functions (using Church-like encodings).

The implementation technology for functional programming languages based on
polymorphic calculi is quite mature. The industrial strength
Glasgow Haskell Compiler (GHC),
whose intermediate core language is an extension of \Fw, is used by thousands
every day.
Our term-indexed calculus \Fi\ is closely
related to \Fw\ by an index-erasure property. The hope is that
a language implementation based on \Fi\ can benefit from these technologies.
We have built a language implementation of these ideas, which we call Nax.

Type inference algorithms for functional programming languages are often
based on certain restrictions of the Curry-style polymorphic lambda calculi.
These restrictions are designed to avoid higher-order unification during type
inference. We have developed a conservative extension of Hindley--Milner
type inference for Nax. This was possible because Nax is based on a
restricted \Fi. Dependently typed languages, on the other hand, are
often based on bidirectional type checking, which requires annotations on
top level definitions, rather than Hindley--Milner-style type inference.

In dependent type theories, datatypes are usually introduced as
primitive constructs (with axioms), rather than as
functional encodings  (\eg, Church encodings). 
One can give functional encodings for datatypes in a dependent type theory,
but one soon realizes that the induction principles (or, dependent eliminators)
for those datatypes cannot be derived within the pure dependent calculi
\cite{Geuvers01}.
So, dependently typed reasoning systems support datatypes as primitives.
For instance, Coq is based on Calculus of Inductive Constructions, which
extends Calculus of Constructions \cite{CoqHue86} with dependent datatypes
and their induction principles.

In contrast, in polymorphic type theories, all imaginable datatypes
within the calculi have functional encodings (\eg, Church encodings).
For instance, \Fw\ need not introduce datatypes as primitive constructs,
since \Fw\ can embed all these datatypes, including non-regular
recursive datatypes with type indices.

Another reason to use \Fi\ is to extend the application of 
Mendler-style recursion schemes, which are well-understood
in the context of polymorphic lambda calculi like \Fw.
Researchers have thought about (though not published)\footnote{
     Tarmo Uustalu described this on a whiteboard
     when we met with him at the University of Cambridge in 2011.}
Mendler-style primitive recursion over dependently-typed functions
over positive datatypes (\ie, datatypes that have a map), but not for
negative (or, mixed-variant) datatypes.
In System \Fi, we can embed Mendler-style recursion schemes, (just as we
embedded them in \Fw) that are also well-defined for negative datatypes.

