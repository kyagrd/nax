\section{Motivation: from System~\Fw\ to System~\Fi, and back}
\label{sec:motiv}
It is well known that datatypes can be embedded into polymorphic lambda
calculi by means of functional encodings~\cite{AbeMatUus03}, such
as the Church and Boehm-Berarducci encodings~\cite{BoehmBerarducci}.

In System~\textsf{F}, for instance, one can embed \emph{regular
datatypes}, like homogeneous lists:
\[
\begin{array}{ll}
\text{Haskell:} & \texttt{data List a = Cons a (List a) | Nil} \\
\text{System \textsf{F}:}~& 
\texttt{{List}}\:\: A\:\:\triangleq\:\:
\forall X.
(A\to X\to X) \to X \to X ~~\; 
\end{array}
\]
In such regular datatypes, constructors have algebraic structure that
directly translates into polymorphic operations on abstract types as
encapsulated by universal quantification over types (of kind $*$).

In the more expressive System \Fw\ (where one can abstract over
type constructors of any kind), one can encode more general
\emph{type-indexed datatypes} that go beyond the regular datatypes.
For example, one can embed powerlists with heterogeneous elements
in which an element of type \texttt{a} is followed by
an element of the product type \texttt{(a,a)}:
\[
\begin{array}{ll}
\text{Haskell:} & \texttt{data Powl a = 
        PCons a (Powl(a,a))
        | 
        PNil 
} \\
& \texttt{Cons 1 (Cons (2,3) (Cons ((3,4),(1,2)) Nil)) :: Powl Int}\\
\text{System \Fw:}~& \texttt{{Powl}}\:\triangleq\:
\lambda A^{*}.\forall X^{*\to*}. (A\to X(A\times A)\to X A) \to X A \to XA
\end{array}
\]
Note the non-regular occurrence (\texttt{Powl(a,a)}) in the definition of
(\texttt{Powl a}), and the use of universal quantification over
higher-order kinds.

What about term-indexed datatypes?  What extensions to System~\Fw\ are
needed to embed term indices as well as type indices?  Our answer is
System~\Fi.

In a functional language supporting term-indexed datatypes, we envisage
that the classic example of homogeneous length-indexed lists would be
defined along the following lines (in Nax [TODO cite] syntax):\vspace{-5pt}
\begin{lstlisting}[basicstyle={\ttfamily},language=Haskell]
 data Nat = S Nat | Z 
 data Vec (a:*) {i:Nat} where
   VCons : a -> Vec a {i} -> Vec a {S i}
   VNil  : Vec a {Z}
\end{lstlisting}~\vspace{-15pt}\\ \noindent
Here the type constructor~{\tt Vec} is defined to admit parameterisation
by both type and term indices.  For instance, the type 
(\verb|Vec (List Nat) {S (S Z)}|) is that of two-dimensional
vectors of lists of natural numbers.  By design, our syntax directly
reflects the difference between type and term indexing by enclosing the latter in
curly braces.  We also make this distinction in 
System~\Fi, where it is useful within the types system
to guarantee the static nature of term indexing.

The encoding of the vector datatype in System~\Fi\ is as follows: 
\begin{equation*}\label{FiVecType}
\texttt{{Vec}}
\triangleq
\begin{array}[t]{l}
\lambda A^\mathtt{*}.\lambda
i^{\texttt{{Nat}}}.  \forall X^{\texttt{{Nat}}\to\mathtt{*}}.
  (\forall j^{\texttt{{Nat}}}.A\to X\{j\}\to X\{\texttt{{S}}\; j\})
  \to X\{\texttt{{Z}}\}
    \to X\{i\}
\end{array}
\end{equation*}
where $\texttt{{Nat}}$, $\mathtt Z$, and $\mathtt S$ respectively encode
the natural number type and its two constructors,  zero and successor.
Without going into the details of the formalism, which are given in the
next section, one sees that such a calculus incorporating term-indexing
structure needs four additional constructs (see \Fig{FiSyntax} for the
highlighted extended syntax).
\begin{enumerate}
\item 
  Type-indexed kinding~($A\to\kappa$), as in $(\texttt{{Nat}\ensuremath{\to}*})$
  in the example above, where the compile-time nature of term-indexing
  will be reflected in the typing rules, enforcing that $A$ be a closed
  type~(rule~$(Ri)$ in \Fig{FiTyping}).

\item 
  Term-index abstraction~$\lambda i^A.F$~(as $\lambda
  i^{\texttt{{Nat}}}.\cdots$ in the example above) for constructing (or
  introducing) type-indexed kinds (rule~$(\lambda i)$ in
  \Fig{FiTyping}).  

\item 
  Term-index application~$F\{s\}$ (as $X\{{\tt Z}\}$, $X\{j\}$, and
  $X\{\texttt{S}\;j\}$ in the example above) for destructing (or
  eliminating) type-indexed kinds, where the compile-time nature of
  indexing will be reflected in the typing rules, enforceing that the index be
  statically typed% in that it does not depend on run-time parameters
~(rule~$(@i)$ in \Fig{FiTyping}) .

\item 
  Term-index polymorphism~$\forall i^A.B$ (as $\forall
  j^{\texttt{{Nat}}}.\cdots$ in the example above) where the compile-time
  nature of polymorphic term-indexing will be reflected in the typing rules enforcing
  that the variable~$i$ be static of closed type~$A$~(rule~$(\forall
  Ii)$ in \Fig{FiTyping}).
\end{enumerate}

As described above, System~\Fi\ maintains a clear-cut separation between
type indexing and term indexing.  This adds a level of abstraction
to System~\Fw\ and yields types that in addition to parametric polymorphism
also keep track of inductive invariants using term indices. All term-index
information can be erased, since it is only used at compile-time.  
It is possible to project any well-typed System~\Fi\ term into a well-typed System~\Fw\ term.
For instance, the erasure of the \Fi-type~\texttt{Vec}
is the \Fw-type~\texttt{List}.  This is established in
\S\ref{sec:theory} and used to deduce the strong normalization of
System~\Fi.


