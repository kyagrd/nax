\documentclass[a4paper]{easychair} % A4 is needed for the abstract book

%\documentclass[a4paper, debug]{easychair} 
% can be used to better see overfull boxes

\usepackage{enumerate}

\bibliographystyle{plain}

%\newtheorem{thm}{Theorem}   % no such environments are predefined

\title{Type-Based Termination of\\ Term-Indexed Types}
\titlerunning{Type-Based Termination of Term-Indexed Types}
\author{
Ki Yung Ahn\inst{1}
\and
Tim Sheard\inst{1}
 \and
TODO\inst{2}
 \and
TODO\inst{2}
}
\institute{
  Portland State University,\thanks{Funded by XXX project.} \\
  Portland, OR, USA
\and
  University of Cambridge, \\
  Cambridge, UK
}
\authorrunning{Author1 and Author2}

\newcommand{\cf}[0]{{cf.}}
\newcommand{\eg}[0]{{e.g.}}
\newcommand{\ie}[0]{{i.e.}}
\newcommand{\aka}[0]{{a.k.a.}}

\newcommand{\F}[0]{{\ensuremath{\mathsf{\uppercase{F}}}}}
\newcommand{\Fw}[0]{{\ensuremath{\mathsf{\uppercase{F}}_{\!\omega}}}}

\begin{document}
\maketitle

%\begin{abstract}
%\end{abstract}
% abstracts of abstracts are not compulsory

The context of our work is the Nax programming language project.
We are developing a unified programming and reasoning system,
called Nax, with the following design goals:\vspace*{-1ex}
\begin{enumerate}[(1)]
 \item Nax supports the major constructs of modern functional programming languages,
 such as parametric polymorphism, recursive datatypes, and type inference,
 \vspace*{-1.2ex}
 \item Nax can specify fine-grained program properties as types and
 witness proofs of such properties by writing a program (The Curry--Howard correspondence),
 \vspace*{-1.2ex}
 \item Nax is based on a minimal foundational calculus
 that is expressive enough to embed all the language constructs in (1)
 and is also logically consistent to avoid paradoxical proofs in (2),
 \vspace*{-1.2ex}
 \item Nax has a simple implementation infrastructure that keeps the trusted base small.
\end{enumerate}
Our approach towards these goals is to 
design an appropriate foundational calculus supporting
\emph{Mendler-style recursion schemes}
and \emph{term-indexed datatypes}.

Term-indexed datatypes support (2), for instance,
statically specifying the size of a list using a natural number index in the list type.
Mendler-style recursion schemes support (1) since they are based
on parametric polymorphism and well-defined over wide range of datatypes.
They also support (4) since their termination is type-based --
no other termination checking infrastructure is necessary.

In this abstract, we outline a paper
that will discuss the advantages of adopting the Mendler-style,
and then introduce a new Mendler-style operator.

\vspace*{-.5ex}
\paragraph{Advantages of adopting the Mendler-style\!\!\!}
include the definition of
any recursive datatype, while still providing rich set of
principled eliminators that ensure their well-behaved use.
Certain concepts, such as Higher-Order Abstract Syntax
(HOAS), are most succinctly defined as mixed-variant datatypes.
However, most existing reasoning systems, based on the
Curry--Howard correspondence, unfortunately, restrict
definition of mixed-variant recursive datatypes. As a result, one is forced to devise
tricks to encode concepts like HOAS within the restrictive datatypes allowed in such systems.


We believe it is worthwhile to allow definition of all recursive datatypes
(\eg, non-strictly positive, mixed-variant, nested)
usually available in functional languages, but outlawed in reasoning systems. For instance,
in Haskell, we can define a HOAS for the untyped lambda-calculus
as follows.
\begin{verbatim}
    data Exp = Abs (Exp -> Exp) | App Exp Exp
\end{verbatim}
Even if we assume all functions embedded in \texttt{Abs} are non-recursive,
evaluating HOAS expressions may still cause problems when reasoning logically,
since the untyped lambda calculus has diverging terms. However, there are
many well-behaved computations on \texttt{Exp}. For instance, examining whether
an HOAS expression is \texttt{Abs} or \texttt{App}, and, converting an HOAS expression
to first-order syntax are examples of terminating computation on \texttt{Exp}.
Ahn and Sheard \cite{AhnShe11} formalized a Mendler-style recursion scheme
that captures these well-behaved computations.

If the datatype \texttt{Exp} used term-indexes to express invariants of well
formed expressions, we could rely on these invariants to write even more expressive
programs, such as a well-typed evaluator. Discussion of this idea
will constitute the second part of the paper.

\paragraph{Term-indices that govern termination behavior over datatypes\!\!\!}
motivate a new Mendler-style recursion scheme.
Consider yet another HOAS datatype below for the Simply-Typed Lambda-Calculus (STLC)
defined in Nax-like syntax,\footnote{curly braces emphasize
  term-indices used in types (\texttt{Exp\{t1\}}),
  and types used in kinds (\texttt{\{Ty\}\;->\;*}).}
where HOAS expressions (\texttt{Expr}) are
statically indexed by the terms that represent STLC-types (\texttt{Ty}).
\begin{verbatim}
    data Ty : * where    Iota : Ty
                         Arr  : Ty -> Ty -> Ty

    data Exp : {Ty} -> * where   Abs : (Exp{t1} -> Exp{t2}) -> Exp{Arr t1 t2}
                                 App : Exp{Arr t1 t2} -> Exp{t1} -> Exp{t2}
\end{verbatim}
Unlike the HOAS datatype for untyped lambda calculus, evaluating
these term-indexed HOAS expressions will always terminate,
since the STLC is strongly normalizing.
The intuitive principle behind the termination behavior of \texttt{Exp\,:\,\{Ty\}\,->\,*}
comes from the paradigmatic use of term-indices at negative recursive occurrences.  
In the type of \texttt{Abs}, the term index \texttt{t1}
at negative recursive occurrence is ``smaller'' than the index
(\texttt{Arr t1 t2}) of the result type. In the type of \texttt{App},
the two term-indices (\texttt{Arr t1 t2}) and \texttt{t1} at positive recursive
occurrences are larger than and unrelated to the index \texttt{t2} of the result type.
Thus, we conjecture that our new Mendler-style recursion scheme, namely \textsf{MPrIx},
which enables us to write an evaluator for  \texttt{Exp\,:\,\{Ty\}\,->\,*}, 
is well-defined when every index at a negative recursive occurrence is
smaller than the index of the result type.
In the paper, we will \vspace*{-1ex}
\begin{itemize}
 \item Formulate the type signature and the reduction rule of \textsf{MPrIx},
 \vspace*{-1ex}
 \item Clearly specify what ``smaller'' means, and
 \vspace*{-1ex}
 \item Prove that \textsf{MPrIx} terminates whenever it is well defined.
\end{itemize}

\paragraph{The impact\!\!} of our work is not limited to
systems like Nax that adopt Mendler-style or type-based termination.
We hope to inspire other existing Curry--Howard based reasoning systems
to support datatypes like \texttt{Exp\,:\,\{Ty\}\,->\,*}.
For instance, perhaps the positivity checker in Agda may be extended to accept
not only positive datatypes but also certain term-indexed negative datatypes
whose negative occurrences are guarded by our ``smaller'' condition.

% create the bibliography
\bibliography{main}   % refers to main.bib
\end{document}
