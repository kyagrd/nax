\documentclass[a4paper]{easychair} % A4 is needed for the abstract book

%\documentclass[a4paper, debug]{easychair} 
% can be used to better see overfull boxes

\usepackage{enumerate}
\usepackage{comment}

\bibliographystyle{plain}

%\newtheorem{thm}{Theorem}   % no such environments are predefined

\title{Mendler-style Recursion Schemes for Mixed-Variant Datatypes}
\titlerunning{Mendler-style Recursion Schemes for Mixed-Variant Datatypes}
\author{
Ki Yung Ahn\inst{1}
\and
Tim Sheard\inst{1}
 \and
Marcelo Fiore\inst{2}
}
\institute{
  Portland State University~\thanks{supported by NSF grant 0910500.}
%%, \\ Portland, OR, USA
\and
  University of Cambridge
%%, \\ Cambridge, UK
}
\authorrunning{K.Y. Ahn, T. Sheard and M. Fiore}

\newcommand{\cf}[0]{{cf.}}
\newcommand{\eg}[0]{{e.g.}}
\newcommand{\ie}[0]{{i.e.}}
\newcommand{\aka}[0]{{a.k.a.}}

\newcommand{\F}[0]{{\ensuremath{\mathsf{\uppercase{F}}}}}
\newcommand{\Fw}[0]{{\ensuremath{\mathsf{\uppercase{F}}_{\!\omega}}}}
\newcommand{\Fixw}[0]{{\ensuremath{\mathsf{\uppercase{F}\lowercase{ix}}_{\!\omega}}}}

\newcommand{\msfit}[0]{\texttt{msfit}}
\newcommand{\mprsi}[0]{\texttt{mprsi}}

\begin{document}
\maketitle

%\begin{abstract}
%\end{abstract}
% abstracts of abstracts are not compulsory

The context of our work is the Nax programming language project.
We are developing a unified programming and reasoning system,
called Nax, with the following design goals:\vspace*{-1ex}
\begin{enumerate}[(1)]
 \item Nax supports the major constructs of modern functional programming languages,
 such as parametric polymorphism, recursive datatypes, and type inference,
 \vspace*{-1.2ex}
 \item Nax can specify fine-grained program properties as types and
 witness proofs of such properties by writing a program (The Curry--Howard correspondence),
 \vspace*{-1.2ex}
 \item Nax is based on a minimal foundational calculus
 that is expressive enough to embed all language constructs in (1)
 and is also logically consistent to avoid paradoxical proofs in (2),\vspace*{-1.2ex}
 \item Nax has a simple implementation infrastructure that keeps the trusted base small.
\end{enumerate}
\vspace*{-.5ex}
Our approach towards these goals is to 
design an appropriate foundational calculus
that extends \Fw \cite{Girard72} or \Fixw \cite{AbeMat04} to support
\emph{Mendler-style recursion schemes} \cite{Mendler87} with \emph{term-indexed datatypes}.
Mendler-style recursion schemes support (1) since they are based
on parametric polymorphism and well-defined over wide range of datatypes.
They also support (4) since their termination is type-based --
no other termination checking is necessary.
Term-indexed datatypes support (2), for instance,
statically specifying the size of a list using a natural number index in the list type.

In this abstract, we outline a paper
that will
\textcircled{1} discuss the advantages of adopting the Mendler style,
\textcircled{2} report that we can define an evaluator for the simply-typed
Higher-Order Abstract Syntax (HOAS) using Mendler-style iteration
with syntactic inverses (\msfit), and
\textcircled{3}~propose a new recursion scheme (work in progress) whose
termination rely on the invariants specified by size measures on indices.
\vspace*{-1ex}
\paragraph{Advantages of adopting the Mendler style\!\!\!}
include allowing arbitrary definition of recursive datatypes, while still 
ensuring well-behaved use by providing rich set of principled eliminators.
Certain concepts, such as HOAS, are most succinctly defined as
mixed-variant datatypes. However, most existing reasoning systems
(\eg, Coq, Agda), based on the Curry--Howard correspondence, unfortunately,
restrict definitions of
mixed-variant recursive datatypes.
To use concepts like HOAS within such systems,
one is forced to devise clever encodings \cite{PHOAS}.

We believe it is worthwhile to allow defining all recursive datatypes
available in functional \emph{programming} languages, including those outlawed
in many reasoning systems. For instance, a HOAS datatype for
the untyped $\lambda$-calculus
{\small\texttt{\,data Exp~=~Abs\;(Exp\;->\;Exp)~|~App\;Exp\;Exp\,}} in Haskell.
Even if we assume all functions embedded in \texttt{Abs} are non-recursive,
evaluating HOAS may still cause problems for logical reasoning,
since the untyped $\lambda$-calculus has diverging terms. However, there are
many well-behaved computations on \texttt{Exp}: examining whether
an HOAS expression is \texttt{Abs} or \texttt{App}, and, converting an HOAS expression
to first-order syntax are examples of terminating computation on \texttt{Exp}.
Ahn and Sheard \cite{AhnShe11} formalized a Mendler-style recursion scheme
(\msfit, \aka\ \textit{msfcata}) that captures these well-behaved computations.

If the datatype \texttt{Exp} had indexes to assert invariants of
well-formed expressions, we could rely on these invariants to write
even more expressive programs, such as a terminating well-typed evaluator.
Discussion around this idea will constitute the latter parts of the paper.
\vspace*{-1ex}
\paragraph{A simply-typed HOAS evaluator\!\!\!} can be defined
using \msfit\ at kind \texttt{*\,->\,*}. We recently discovered this
while preparing to write this abstract. Since \msfit\ terminates
for any datatype, we are also proving that the evaluation of
the simply-typed $\lambda$-calculus always terminates
just by defining \texttt{eval\,:\,Exp\;t\;->\;Id\;t\,} in Nax.
The entire Nax code is laid out below.
\vspace*{-.5ex}
{\small
\begin{verbatim}
  data E : (* -> *) -> (* -> *) where      -- the "deriving fixpoint Exp" defines
    Abs : (r a -> r b) -> E r (a -> b)     --   abs f   = In[* -> *] (Abs f)
    App : E r (a -> b) -> E r a -> E r b   --   app f e = In[* -> *] (App f e)
      deriving fixpoint Exp                --   synonym Exp t = Mu[* -> *] E t
\end{verbatim}
\vspace*{-2ex}
\begin{verbatim}
  data Id a = MkId a -- the identity type
  unId (MkId x) = x  -- destructor of Id
\end{verbatim}
\vspace*{-2ex}
\begin{verbatim}
  eval e = msfit { t . Id t } e with
             call inv (App f x) = MkId (unId(call f) (unId(call x)))
             call inv (Abs f) = MkId (\v -> unId(call (f (inv (MkId v)))))
\end{verbatim} }
\vspace*{-.5ex}
\noindent
The type of \texttt{eval\,:\,Exp\;t\;->\;Id\;t\,} is inferred from
\texttt{\{\,t\,.\;Id t\,\}}, which specifies the answer type in relation
to the input type's index.
Conceptually, \texttt{msfit} at kind \texttt{*\,->\,*} has the following type.
Note the types of \texttt{msfit}'s two abstract operations
\texttt{call} and \texttt{inv}.
{\small
\begin{verbatim}
  msfit : (   (forall r i . r i -> ans i) -- call
           -> (forall r i . ans i -> r i) -- inv
           -> (forall r i . f r i -> ans i)      ) -> Mu[* -> *] f j -> ans j
\end{verbatim} }
\vspace*{-3ex}
\paragraph{Evaluation via user-defined value domain\!\!\!\!\!}, instead of
the native value space of Nax, motivates a new Mendler-style recursion scheme,
\mprsi, standing for Mendler-style primitive recursion with sized index.
A user-defined value domain is particularly useful for evaluating expressions
of first-order syntax. Here, we stick to HOAS to avoid introducing a new datatype.
Consider writing an evaluator \texttt{\,veval\;:\;Exp\;t\,->\,Val\;t\,}
via the value domain \texttt{Val\;:\;*\,->\,*\,} below.
{\small
\begin{verbatim}
  data V : (* -> *) -> * -> * where      -- the "deriving fixpoint Val" defines
    Fun : (r a -> r b) -> V r (a -> b)   -- fun f = In [* -> *] (Fun f)
      deriving fixpoint Val              -- synonym Val t = Mu[* -> *] V t

  veval e = msfit { t . V t } e with
              call inv (App f x) = unfun (call f) (call x)  -- how do we deinfe unfun?
              call inv (Abs f) = fun (\v -> (call (f (inv v))))
\end{verbatim} }
\noindent
Only if we were able to define
\texttt{unfun\;:\;Val\,(a\,->\,b)\;->\;Val\;a\,->\,Val\;b},
this would be admitted in Nax.
However, it is not likely that \texttt{unfun} can be defined using
recursion schemes currently available in Nax. Thereby, we propose
a new recursion scheme \mprsi, which extends the Mendler-style primitive recursion
(\texttt{mpr}) with the uncasting operation. 
{\small
\begin{verbatim}
  mprsi : (   (forall r i. r i -> ans i)                      -- call
           -> (forall r i. (i < j) => r i -> Mu[* -> *] f i)  -- cast   
           -> (forall r i. (i < j) => Mu[* -> *] f i -> r i)  -- uncast 
           -> (forall r i. f r i -> ans i) ) -> Mu[* -> *] f j -> ans j

  unfun v = mprsi { (a -> b) . V a -> V b } v with
              call cast (Fun f) = cast . f . uncast   -- dot (.) is function composition
\end{verbatim} }
\noindent
Note the size constraints \texttt{(i\;<\;j)} on both \texttt{cast} and \texttt{uncast} operations
(FYI, \texttt{mpr}'s \texttt{cast} does not have size constraints).
These constraints prevent writing evaluator-like functions on strange expressions
that have constructors like below, which may cause non-termination.
{\small
\begin{verbatim}
 app1 : (Exp1 (a->b) -> Exp1 b) -> Exp1 (a->b)  -- prevented by constraint on uncast
 app2 : (Exp2 a -> Exp2 (a->b)) -> Exp2 (a->b)  -- prevented by constraint on cast
\end{verbatim} }
\noindent
This is still a fresh proposal awaiting proof of termination.
We hope to discuss this in the TYPES 2013 workshop and welcome
enlightening feedback from the TYPES community.

\begin{comment}
\paragraph{Term-indices that govern termination behavior over datatypes\!\!\!}
motivate a new Mendler-style recursion scheme.
Consider yet another HOAS datatype below for the Simply-Typed Lambda-Calculus (STLC)
defined in Nax-like syntax,\footnote{curly braces emphasize
  term-indices used in types (\texttt{Exp\{t1\}}),
  and types used in kinds (\texttt{\{Ty\}\;->\;*}).}
where HOAS expressions (\texttt{Expr}) are
statically indexed by the terms that represent STLC-types (\texttt{Ty}).
\begin{verbatim}
    data Ty : * where    Iota : Ty
                         Arr  : Ty -> Ty -> Ty
\end{verbatim}\vspace*{-2ex}
\begin{verbatim}
    data Exp : {Ty} -> * where   Abs : (Exp{t1} -> Exp{t2}) -> Exp{Arr t1 t2}
                                 App : Exp{Arr t1 t2} -> Exp{t1} -> Exp{t2}
\end{verbatim}
Unlike the HOAS datatype for untyped lambda calculus, evaluating
these term-indexed HOAS expressions will always terminate,
since the STLC is strongly normalizing.
The intuitive principle behind the termination behavior of \texttt{Exp\,:\,\{Ty\}\,->\,*}
comes from the paradigmatic use of term-indices at negative recursive occurrences.  
In the type of \texttt{Abs}, the term index \texttt{t1}
at negative recursive occurrence is ``smaller'' than the index
(\texttt{Arr t1 t2}) of the result type. In the type of \texttt{App},
the two term-indices (\texttt{Arr t1 t2}) and \texttt{t1} at positive recursive
occurrences are larger than and unrelated to the index \texttt{t2} of the result type.
Thus, we conjecture that our new Mendler-style recursion scheme, namely \textsf{MPrIx},
which enables us to write an evaluator for  \texttt{Exp\,:\,\{Ty\}\,->\,*}, 
is well-defined when every index at a negative recursive occurrence is
smaller than the index of the result type.
In the paper, we will \vspace*{-.5ex}
\begin{itemize}
 \item Formulate the type signature and the reduction rule of \textsf{MPrIx},
 \vspace*{-1ex}
 \item Clearly specify what ``smaller'' means, and
 \vspace*{-1ex}
 \item Prove that \textsf{MPrIx} terminates whenever it is well defined.
 \vspace*{-1ex}
\end{itemize}
\vspace{-1ex}
% \paragraph{The impact\!\!} of our work is not limited to
% systems like Nax that adopt Mendler-style or type-based termination.
% We hope to inspire other existing Curry--Howard based reasoning systems
% to support datatypes like \texttt{Exp\,:\,\{Ty\}\,->\,*}.
% For instance, perhaps the positivity checker in Agda may be extended to accept
% not only positive datatypes but also certain term-indexed negative datatypes
% whose negative occurrences are guarded by our ``smaller'' condition.
% \vspace{-1ex}
\end{comment}
\bibliography{main}   % refers to main.bib
\end{document}
