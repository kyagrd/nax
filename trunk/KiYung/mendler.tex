\chapter{Mendler-style recursion schemes}\label{ch:mendler}
In this chapter, we explore a family of terminating recursion combinators
of Mendler style. These combinators behave well over a wide class of
recursive datatypes. This chapter is a revised and extended version of
the conference paper by \citet{AhnShe11}. Here, the names of Mendler-style
recursion combinators are different from the paper to make the names
consistent throughout the dissertation: \MIt, \MsfIt, \McvIt, and
\MsfcvIt\ correspond to \textit{mcata}, \textit{msfcata}, \textit{mhist},
and \textit{msfhist} in the paper. In addition, we introduce several
more families of recursion combinators (\MPr, \McvIt, \McvPr)
and a new example using \MsfIt\ over an indexed datatype,\footnote{
	The type signature of $\MsfIt_{* -> *}$ is different
	from the type signature of $\textit{msfcata1}$ in the paper. 
	\textit{msfcata1} does type check but rather useless.
	Here, we gave a more flexible type signature
	so that we can write useful examples.}
which are not present in the conference paper.

This chapter gives the reader an intuitive understanding of the Mendler-style
recursion combinators, rather than providing a rigorous formulation of
the theories behind the Mendler style. The discussions in this chapter
are guided by a series of examples (and some semi-formal proofs) written
in a certain style of Haskell, which assumes certain conventions
(see \S\ref{sec:intro}).  Haskell is a \emph{real world}
functional programming language \cite{OSullivan08}, which admits
a certain level of formality since it is a pure functional language.
More rigorous formulations of the background theory used to formalize
the Mendler style will come in the following chapters
(Chapter \ref{ch:fi} and Chapter \ref{ch:fixi}).

\input{mendler_intro}

\input{mendler_it}

%% We \cite{AhnShe11} have illustrated that Mendler-style recursion combinators
%% can be defined not only for nested datatypes but also for arbitrary indexed
%% datatypes, whose can index may vary in the result types of the
%% data constructors as well as in the argument types of the data constructors.
%% For example, many examples using Generalized Algebraic DataTypes (GADTs),
%% a popular extension to the type system of Haskell, involves indexed datatypes
%% beyond the scope of nested datatypes.
%% 
%% Previous work, that promoted the expressiveness and generality of
%% the Mendler-style \cite{AbeMatUus03,AbeMatUus05,AbeMat04} used examples
%% involving nested datatypes, a particularly simple kind of indexing
%% where the variation of indices are restricted to the argument types of
%% the data constructors. In order to handle nested datatypes, the translation of
%% the Mendler-style combinators into System \textsf{F}, had to be extended to
%% translation into \Fw. To translate Mendler-style combinators over arbitrary
%% indexed types including term indices we have to extend \Fw, to a new calculus
%% we call \Fi, and prove that this new calculus is also strongly normalizing.
%% I have identified what the extensions must look like and have been developing
%% the theoretical background necessary to accomplish this task.
%% This theory is an important part of the thesis and is elaborated in
%% Chapter \ref{ch:fi}.
%% 
%% Despite, not having a completed theory to support the translation of GADTs,
%% we have identified many examples \cite{AhnShe11}. Including examples
%% that use term indices to exploit the Curry-Howard isomorphism, and
%% examples of mutual recursion (between types) encoded in terms of GADTs.
%% 
%% In general, we have pushed the boundaries of what is known about
%% the Mendler-style operators in many ways. We have identified new domains
%% in which they are applicable, identified new operators, adapted old ones,
%% and organized them all into a natural hierarchy that explains how all are
%% instances of a common pattern. However, we know of no guarantee that every
%% principled recursion scheme has a Mendler-style operator, so there is
%% much work still to be done.

\input{mendler_cv} %%%% ssec:tourHist0

\input{mendler_nonreg} %%%% mendler_nonreg
	%%% three files below are subsections of nonreg
	\input{mendler_nest}   %%%% ssec:tourNested
	\input{mendler_gadt}   %%%% ssec:tourIndexed
	\input{mendler_mutrec} %%%% ssec:tourMutRec

\input{mendler_prim}

\input{mendler_sf}

\input{mendler_proof}
