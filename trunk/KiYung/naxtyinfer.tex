\chapter{Type Inference in Nax} \label{ch:naxTyInfer}
Type inference for a language supporting indexed datatypes are
known to be difficult, and it becomes more difficult when the language
supports wider range of indexed datatypes. Here, in this chapter, we
illustrate the key idea that enables a conservative extension of
Hindley-Miler type inference.  The \emph{index transformers},
which are type annotations on the pattern matching constructs,
plays a key role in inferring types for Nax programs involving
indexed datatypes. We discuss how index transformers help infer types
in the context of a small subset of Nax, only considering nested datatypes
(\ie, no GADT-style datatype definitions, no term indices) but omitting other
details of Nax. There are two reasons why we discuss the type inference
over a small subset of Nax. Firstly, describing type inference over
the full Nax language increases accidental complexity, making it more
difficult to understand the key idea. Secondly, type inference over
other omitted features (\eg, term indices, GADT-style datatype definitions)
are rather orthogonal to the role of index transformers -- they could
be considered separately.

In the following sections, we discuss the role of index transformers
(\S\ref{sec:naxTyInfer:psi}), type inference for Mendler-style recursion
combinators (\S\ref{sec:naxTyInfer:rec}), and how we treat other features
of Nax in the type system (\S\ref{sec:naxTyInfer:etc}).

\section{Index transformers in a small subset of Nax}
\label{sec:naxTyInfer:psi}
We describe the role of index transformers in the type system of
a small language (SmallNax) that supports recursive datatypes.
The syntax of SmallNax is illustrated in Definition\;\ref{def:SmallNax},
its kinding and typing rules are illustrated in Figure\;\ref{fig:SmallNax}.
\begin{definition}[Syntax of SmallNax]
\label{def:SmallNax}
\begin{singlespace}
\begin{align*}
&\textbf{Term}&
t,s&~::= ~ x
    ~  | ~ \l x    . t 
    ~  | ~ t ~ s       
    ~  | ~ \<let> x=s \<in> t
    ~  | ~ C
    ~  | ~ \varphi^\psi
%%    ~  | ~ \<case>_{\!\!\psi}\; s \<of> \overline{C \overline{x} -> t}
\\
&\textbf{Type constructor}&
F,G,A,B&~::= ~ X
        ~  | ~ A -> B            
        ~  | ~ T
        ~  | ~ F ~ G
\\
&\textbf{Type scheme}&
\sigma&~::= ~ \forall X^\kappa.\sigma
       ~  | ~ A
\end{align*}
\end{singlespace}
\end{definition}
\begin{definition}[Type scheme ordering (or, generic instantiation)]
\label{def:SmallNaxGInst}
\framebox{$\sigma \sqsubseteq_\Delta \sigma'$}
\[
 \inference[\sc GInst]{
    X_1',\dots,X_m'\notin\FV(\forall X_1^{\kappa_1}\dots X_n^{\kappa_n}.\sigma)
    \\
    \Delta |- \forall X_1^{\kappa_1}\dots X_n^{\kappa_n}.\sigma : *
    &
    \Delta |- \forall X_1'^{\kappa_1'}\dots X_m'^{\kappa_m'}.\,A[F_1/X_1]\cdots[F_n/X_n] : *
  }{\forall X_1^{\kappa_1}\dots X_n^{\kappa_n}.\sigma \;\sqsubseteq_\Delta\;
    \forall X_1'^{\kappa_1'}\dots X_m'^{\kappa_m'}.\,A[F_1/X_1]\cdots[F_n/X_n]} \]
~
\end{definition}

The syntax of SmallNax is similar to the syntax of 
the Hindley--Milner type system (HM) in \S\ref{sec:hm}.
SmallNax has data constructors ($C$) and case functions ($\varphi^\psi$)
in addition to the terms of HM. A case function $\varphi^\psi$ is
a list of alternatives ($\varphi ::= \overline{C \overline{x} -> t}$)
annotated with an index transformer $\psi$. The case expression
$\textbf{case}_\psi\;s\<of> \varphi$ in Nax correspond to $\varphi^\psi\;s$,
an application of the case function ($\varphi^\psi$) to the scrutinee ($s$).
Considering case expressions as applications simplifies the typing rules
because we do not need a sparate typing rule for case expressions.
In addition to the types of HM, SmallNax type constructure include
type constructor name ($T$) and type constructor application ($F\;G$).
Type scyemes in SmallNax is similar to the type schemes in HM,
but the universally quantified type variable is annotated with its kind.

\begin{figure}
\begin{singlespace}
\[ \textbf{Kinding rules} \quad \framebox{$ \Delta |- F:\kappa$}\]\vspace*{-2em}
\begin{align*}
& \inference[\sc TVar]{X^\kappa \in \Delta}{\Delta |- X:\kappa} &
& \inference[\sc TArr]{\Delta |- A:* & \Delta |- B:*}{\Delta |- A -> B:*} \\
& \inference[\sc TCon]{T^\kappa \in \Delta}{\Delta |- T:\kappa} &
& \inference[\sc TApp]{\Delta |- F : \kappa -> \kappa' & \Delta |- G : \kappa}
                      {\Delta |- F ~ G : \kappa'}
\end{align*}
\begin{align*}
&\textbf{Declarative typing rules}&\quad
&\textbf{Syntax-directed typing rules}
        \\
& \qquad\framebox{$\Delta;\Gamma |- t : \sigma$}
&
&~\qquad\framebox{$\Delta;\Gamma |-s t : A$}
        \\
& \inference[\sc Var]{x:\sigma \in \Gamma}{\Delta;\Gamma |- x:\sigma} &
& \inference[\sc Var$_s$]{x:\sigma \in \Gamma & \sigma \sqsubseteq_\Delta A}
                         {\Delta;\Gamma |-s x:A} \\
& \inference[\sc Abs]{\Delta |- A:* \\ \Delta;\Gamma,x:A |- t : B}
                     {\Delta;\Gamma |- \l x   .t : A -> B} &
& \inference[\sc Abs$_s$]{\Delta |- A:* \\ \Delta;\Gamma,x:A |-s t:B}
                         {\Delta;\Gamma |-s \l x   .t : A -> B} \\
& \inference[\sc App]{\Delta;\Gamma |- t : A -> B \\ \Delta;\Gamma |- s : A}
                     {\Delta;\Gamma |- t~s : B} &
& \inference[\sc App$_s$]{\Gamma |-s t : A -> B \\ \Gamma |-s s : A}
                         {\Gamma |-s t~s : B} \\
& \inference[\sc Let]{ \Delta;\Gamma |- s : \sigma \\
                       \Delta;\Gamma,x:\sigma |- t : B}
                     {\Delta;\Gamma |- \<let> x=s \<in> t : B} &
& \inference[\sc Let$_s$]
            { \Delta;\Gamma |-s s : A \\
              \Delta;\Gamma,x:\overline{\Delta;\Gamma}(A) |-s t : B}
            {\Gamma |-s \<let> x=s \<in> t : B} \\
& \inference[\sc Inst]{ \Delta;\Gamma |- t : \sigma
                      & \sigma \sqsubseteq_\Delta \sigma'}
                      {\Delta;\Gamma |- t : \sigma'} &
&\quad \begin{smallmatrix}\overline{\Delta;\Gamma}(A)=\forall\vec{X}.A
                         ~\text{where}~\vec{X}=\FV(A)\setminus\dom(\Delta)\setminus\FV(\Gamma)
                 \end{smallmatrix}
                 \\
& \inference[\sc Gen]{\Delta,X^\kappa;\Gamma |- t : \sigma}
                     {\Delta;\Gamma |- t : \forall X^\kappa.\sigma}
                     ~ \text{\small$(X \notin\FV(\Gamma))$} &
& \\
& \inference[\sc Con]{C:\sigma \in \Gamma}{\Delta;\Gamma |- C:\sigma} &
& \inference[\sc Con$_s$]{C:\sigma \in \Gamma & \sigma \sqsubseteq_\Delta A}
                         {\Delta;\Gamma |-s C:A} \\
& \inference[\sc Case]{
              \overline{\Delta;\Gamma |-^\psi C\overline{x} ->t : \sigma}
          }{\Delta;\Gamma |- (\overline{C\overline{x} -> t})^\psi : \sigma } &
& \inference[\sc Case$_s$]{
	      \overline{\Delta;\Gamma |-s^\psi C\overline{x} ->t : \sigma} &
              \sigma \sqsubseteq_\Delta A
          }{\Delta;\Gamma |- (\overline{C\overline{x} -> t})^\psi : A } \\
& \qquad\framebox{$\Delta;\Gamma |-^\psi C\overline{x} -> t : \sigma$} &
& \qquad\framebox{$\Delta;\Gamma |-s^\psi C\overline{x} -> t : \sigma$} \\
& \inference*[\sc Alt]{
              \Delta;\Gamma |- C:\overline{A} -> T\overline{B}\,\overline{A'} \\
              \Delta;\Gamma,\overline{x:A} |- t : \psi(\overline{A'})
            }{ \begin{array}{ll}\Delta;\Gamma |-^\psi C\overline{x} -> t \\
                \qquad : \forall\overline{X^\kappa}.
                          T\overline{B}\,\overline{X} -> \psi(\overline{X})
               \end{array} } &
& \inference*[\sc Alt$_s$]{
             \Delta;\Gamma |-s C:\overline{A} -> T\overline{B}\,\overline{A'} \\
             \Delta;\Gamma,\overline{x:A} |-s t : \psi(\overline{A'})
           }{ \begin{array}{ll}\Delta;\Gamma |-s^\psi C\overline{x} -> t \\
               \qquad : \forall\overline{X^\kappa}.
                         T\overline{B}\,\overline{X} -> \psi(\overline{X})
              \end{array} }
\end{align*}
\caption{Kinding and typing rules of SmallNax}
\label{fig:SmallNax}
\end{singlespace}
\end{figure}



\section{Inferring types for Mendler-style recursion combinators}
\label{sec:naxTyInfer:rec}
we first review and go on

\paragraph{A review of monomorphic recursion and polymorphic recursion}
The Hindley--Milner type system (HM) \cite{DamMil82} supports
monomorphic (general) recursion by assigning a monomorphic type ($A$)
to the recursive variable ($x$), as described in the rule \rulename{Fix-m}
below right.\footnote{
	In \S\ref{sec:hm}, we excluded the general recursion in
	our formalization of HM, althouth its original presentation
	has general recursion, because our language does not support
	general recursion.}
The Milner--Mycroft type systems (MM) \cite{Myc84} supports
polymorphic recursion by assigning a polymorphic type ($\sigma$)
to the recursive variable ($x$), as described by in the rule \rulename{Fix-p}
below left.
\[
\qquad
\inference[\sc Fix-m]{\Gamma,x:A |- t : B}{
	\Gamma |- \textbf{fix}\;x.t : B}
\qquad
\inference[\sc Fix-p]{\Gamma,x:\sigma |- t : \sigma}{
	\Gamma |- \textbf{fix}\;x.t : B}
\]
Polymorphic recursion is necessary for writing recursive programs involving
nested datatypes. However, type inference for MM, which supports polymorphic
recursion, is known to be undecidable \cite{Hen93}.

TODO
the result types of 
result types of data constructors are always the same as the left-hand side
of the equational definition, that is, type constructor followed by
type variable arguments

\section{TODO etc}
\label{sec:naxTyInfer:etc}
TODO
existential type variables
weak existential type variables
TODO

