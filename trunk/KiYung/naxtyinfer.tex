\chapter{Type Inference in Nax} \label{ch:naxTyInfer}
Type inference for a language that supports indexed datatypes is known
to be difficult. In this chapter, we illustrate the key idea that
enables a conservative extension of Hindley-Milner type inference (HM).
We will not be as formal and detailed on proofs as we did for HM
in \S\ref{sec:hm}. We extrapolate from the properties of HM that
the same property (soundness of type inference) should hold for
a subset of Nax, which is structurally similar to HM. Then, we will
argue that some key new features in Nax, which are not present in HM
preserve those properties.
\index{Hindley--Milner}
\index{type inference}

\index{index transformer}
\index{pattern matching}
\index{indexed datatype}
\index{datatype!indexed}
\emph{Index transformers}, which are type annotations on pattern matching
constructs, play a key role in inferring types for Nax programs involving
indexed datatypes. We introduce a subset of Nax, SmallNax, only considering
non-recursive datatypes defined by equational declarations, but omitting other
details of Nax (\S\ref{sec:naxTyInfer:small}). Next, we extend SmallNax with
recursive types and Mendler-style iteration, describe their kinding and
typing rules, and discuss the role of index transformers for type inference
(\S\ref{sec:naxTyInfer:rec}). Then, we discuss how we treat other Nax features
such as GADT-style definitions and term indices in our implementation
(\S\ref{sec:naxTyInfer:gadt}).

\section{SmallNax}
\label{sec:naxTyInfer:small}
The syntax of SmallNax is illustrated in Definition\;\ref{def:SmallNax},
its kinding and typing rules are illustrated in Figure\;\ref{fig:SmallNax}.
\begin{definition}[Syntax of SmallNax]
\label{def:SmallNax}
\begin{singlespace}
\begin{align*}
&\textbf{Term}&
t,s&~::= ~ x
   \mid    \l x    . t 
   \mid    t ~ s       
   \mid    \<let> x=s \<in> t
   \mid    C
   \mid    \varphi^\psi
%%    ~  | ~ \<case>_{\!\!\psi}\; s \<of> \overline{C \overline{x} -> t}
\\
&\textbf{Type constructor}&
F,G,A,B&~::= ~ X
      ~ \mid ~ A -> B
      ~ \mid ~ T
      ~ \mid ~ F ~ G
\\
&\textbf{Type scheme}&
\sigma&~::= ~ \forall X^\kappa.\sigma
       ~  | ~ A
\end{align*}
\end{singlespace}
\end{definition}
\begin{definition}[Type scheme ordering (or, generic instantiation)]
\label{def:SmallNaxGInst}
\framebox{$\sigma \sqsubseteq_\Delta \sigma'$}
\[
 \inference[\sc GInst]{
    X_1',\dots,X_m'\notin\FV(\forall X_1^{\kappa_1}\dots X_n^{\kappa_n}.\sigma)
    \\
    \Delta |- \forall X_1^{\kappa_1}\dots X_n^{\kappa_n}.\sigma : *
    &
    \Delta |- \forall X_1'^{\kappa_1'}\dots X_m'^{\kappa_m'}.\,A[F_1/X_1]\cdots[F_n/X_n] : *
  }{\forall X_1^{\kappa_1}\dots X_n^{\kappa_n}.\sigma \;\sqsubseteq_\Delta\;
    \forall X_1'^{\kappa_1'}\dots X_m'^{\kappa_m'}.\,A[F_1/X_1]\cdots[F_n/X_n]} \]
~
\end{definition}

The syntax of SmallNax is similar to the syntax of HM in \S\ref{sec:hm}.
SmallNax has data constructors ($C$) and case functions ($\varphi^\psi$)
in addition to the terms of HM. A case function $\varphi^\psi$ is
a list of alternatives ($\varphi ::= \overline{C \overline{x} -> t}$)
annotated with an index transformer $\psi$.\footnote{Our Nax implementation
	supports nested patterns (\eg, $(C_1\,x_1\,(C_2\,x_2)\,x_3$), but
	SmallNax only allows simple patterns (\ie, data constructor
	followed by variables) in alternatives.}
The case expression $\textbf{case}_\psi\;s\<of> \varphi$ in Nax
corresponds to $\varphi^\psi\;s$, an application of the case function
($\varphi^\psi$) to the scrutinee ($s$). Considering case expressions as
applications simplifies the typing rules because we do not need
a separate typing rule for case expressions. In addition to the types of HM,
the type constructor syntax in SmallNax includes type constructor names ($T$)
and type constructor applications ($F\;G$). The type schemes in SmallNax 
($\forall X^\kappa.\sigma$) is similar to the type schemes ($\forall X.\sigma$)
in HM, but the universally quantified type variable ($X$) is annotated with
its kind ($\kappa$).

We assume that type constructor names and their associated data constructors
are introduced into the context by preprocessing non-recursive equational
datatype definitions. For example,
$\textbf{data}\;\texttt{Maybe}\;a = \texttt{Just}\;a \mid \texttt{Nil}$
introduces a type constructor name \texttt{Maybe} and its associated
data constructors \texttt{Just} and \texttt{Nil} into the context
($\Delta$ and $\Gamma$ in Figure\;\ref{fig:SmallNax}). That is,
$\texttt{Maybe}^{* -> *} \in \Delta$ and
$\texttt{Just}: \forall X_a^{*}.X_a -> \texttt{Maybe}\,X_a,~
 \texttt{Nil} : \forall X_a^{*}.\texttt{Maybe}\,X_a \in \Gamma$.
Data constructors introduced from an equational datatype definition have
uniform return types ($T\;\overline{X}$) and no existential variables
in their types. For instance, return types of both \texttt{Just} and
\texttt{Nil} have the form of $\texttt{Maybe}\,X_a$. For such non-recursive
equational datatype definitions, index transformer annotations are not needed.
So, we either omit the annotation on the case function ($\varphi$) or
write a dot ($\varphi^\cdot$). We will need index transformers to infer types
involving recursive datatypes (\S\ref{sec:naxTyInfer:rec})
and GADTs (\S\ref{sec:naxTyInfer:gadt}).

\begin{figure}
\begin{singlespace}
\[ \textbf{Kinding rules} \quad \framebox{$ \Delta |- F:\kappa$}\]\vspace*{-2em}
\begin{align*}
& \inference[\sc TVar]{X^\kappa \in \Delta}{\Delta |- X:\kappa} &
& \inference[\sc TArr]{\Delta |- A:* & \Delta |- B:*}{\Delta |- A -> B:*} \\
& \inference[\sc TCon]{T^\kappa \in \Delta}{\Delta |- T:\kappa} &
& \inference[\sc TApp]{\Delta |- F : \kappa -> \kappa' & \Delta |- G : \kappa}
                      {\Delta |- F ~ G : \kappa'}
\end{align*}
\begin{align*}
&\textbf{Declarative typing rules}&\quad
&\textbf{Syntax-directed typing rules}
        \\
& \qquad\framebox{$\Delta;\Gamma |- t : \sigma$}
&
&~\qquad\framebox{$\Delta;\Gamma |-s t : A$}
        \\
& \inference[\sc Var]{x:\sigma \in \Gamma}{\Delta;\Gamma |- x:\sigma} &
& \inference[\sc Var$_s$]{x:\sigma \in \Gamma & \sigma \sqsubseteq_\Delta A}
                         {\Delta;\Gamma |-s x:A} \\
& \inference[\sc Abs]{\Delta |- A:* \\ \Delta;\Gamma,x:A |- t : B}
                     {\Delta;\Gamma |- \l x   .t : A -> B} &
& \inference[\sc Abs$_s$]{\Delta |- A:* \\ \Delta;\Gamma,x:A |-s t:B}
                         {\Delta;\Gamma |-s \l x   .t : A -> B} \\
& \inference[\sc App]{\Delta;\Gamma |- t : A -> B \\ \Delta;\Gamma |- s : A}
                     {\Delta;\Gamma |- t~s : B} &
& \inference[\sc App$_s$]{\Gamma |-s t : A -> B \\ \Gamma |-s s : A}
                         {\Gamma |-s t~s : B} \\
& \inference[\sc Let]{ \Delta;\Gamma |- s : \sigma \\
                       \Delta;\Gamma,x:\sigma |- t : B}
                     {\Delta;\Gamma |- \<let> x=s \<in> t : B} &
& \inference[\sc Let$_s$]
            { \Delta;\Gamma |-s s : A \\
              \Delta;\Gamma,x:\overline{\Delta;\Gamma}(A) |-s t : B}
            {\Gamma |-s \<let> x=s \<in> t : B} \\
& \inference[\sc Inst]{ \Delta;\Gamma |- t : \sigma
                      & \sigma \sqsubseteq_\Delta \sigma'}
                      {\Delta;\Gamma |- t : \sigma'} &
&\quad \begin{smallmatrix}\overline{\Delta;\Gamma}(A)=\forall\vec{X}.A
                         ~\text{where}~\vec{X}=\FV(A)\setminus\dom(\Delta)\setminus\FV(\Gamma)
                 \end{smallmatrix}
                 \\
& \inference[\sc Gen]{\Delta,X^\kappa;\Gamma |- t : \sigma}
                     {\Delta;\Gamma |- t : \forall X^\kappa.\sigma}
                     ~ \text{\small$(X \notin\FV(\Gamma))$} &
& \\
& \inference[\sc Con]{C:\sigma \in \Gamma}{\Delta;\Gamma |- C:\sigma} &
& \inference[\sc Con$_s$]{C:\sigma \in \Gamma & \sigma \sqsubseteq_\Delta A}
                         {\Delta;\Gamma |-s C:A} \\
& \inference[\sc Case]{
              \overline{\Delta;\Gamma |-^\psi C\overline{x} ->t : \sigma}
          }{\Delta;\Gamma |- (\overline{C\overline{x} -> t})^\psi : \sigma } &
& \inference[\sc Case$_s$]{
	      \overline{\Delta;\Gamma |-s^\psi C\overline{x} ->t : \sigma} &
              \sigma \sqsubseteq_\Delta A
          }{\Delta;\Gamma |- (\overline{C\overline{x} -> t})^\psi : A } \\
& \qquad\framebox{$\Delta;\Gamma |-^\psi C\overline{x} -> t : \sigma$} &
& \qquad\framebox{$\Delta;\Gamma |-s^\psi C\overline{x} -> t : \sigma$} \\
& \inference*[\sc Alt]{
              \Delta;\Gamma |- C:\overline{A} -> T\overline{B}\,\overline{A'} \\
              \Delta;\Gamma,\overline{x:A} |- t : \psi(\overline{A'})
            }{ \begin{array}{ll}\Delta;\Gamma |-^\psi C\overline{x} -> t \\
                \qquad : \forall\overline{X^\kappa}.
                          T\overline{B}\,\overline{X} -> \psi(\overline{X})
               \end{array} } &
& \inference*[\sc Alt$_s$]{
             \Delta;\Gamma |-s C:\overline{A} -> T\overline{B}\,\overline{A'} \\
             \Delta;\Gamma,\overline{x:A} |-s t : \psi(\overline{A'})
           }{ \begin{array}{ll}\Delta;\Gamma |-s^\psi C\overline{x} -> t \\
               \qquad : \forall\overline{X^\kappa}.
                         T\overline{B}\,\overline{X} -> \psi(\overline{X})
              \end{array} }
\end{align*}
\caption{Kinding and typing rules of SmallNax}
\label{fig:SmallNax}
\end{singlespace}
\end{figure}

\paragraph{Declarative typing rules and syntax-directed typing rules.}
\index{typing rules!declarative}
\index{typing rules!syntax-directed}
The typing rules of SmallNax (Figure\;\ref{fig:SmallNax}),
excluding the rules for datatypes (\rulename{Con}, \rulename{Case},
\rulename{Alt} in the declarative rules and their corresponding
syntax-directed rules), are structurally similar to the typing rules of HM
(Figure\;\ref{fig:hm}). Each of those typing rules in SmallNax has
its corresponding typing rule with the same name in HM.
The differences from HM are the existence of kinding rules to ensure
well-kindedness of type constructors (which can have kinds other than $*$)
and the additional context $\Delta$ in the typing rules to keep track of
whether type constructor variables are in scope with correctly assigned kinds.
The generic instantiation rule (\rulename{GInst}) also takes $\Delta$ into
consideration so that both sides of $\sqsubseteq$ are well-kinded. We can view
HM as a restriction of SmallNax (excluding the features for datatypes) where
kinds are always $*$. So, we know that the syntax-directed typing rules
(excluding \rulename{Con$_s$}, \rulename{Case$_s$}, \rulename{Alt$_s$})
are sound (Theorem\;\ref{thm:sdSmallNaxSound}) and
complete (Theorem\;\ref{thm:sdSmallNaxComplete})
with respect to the declarative typing rules
(excluding \rulename{Con}, \rulename{Case}, \rulename{Alt}) in SmallNax.
\begin{theorem}[$|-s$ is sound with respect to $|-$]
$ \inference{\Delta;\Gamma |-s t : A}{\Delta;\Gamma |- t : A} $
\label{thm:sdSmallNaxSound}
\end{theorem}
\begin{theorem}[$|-s$ is complete with respect to $|-$] ~
\begin{center}
$ \inference{\Delta;\Gamma |- t : \sigma}{
	\exists A.\;\Delta;\Gamma |-s t : A ~\land~
	\overline{\Delta;\Gamma}(A) \sqsubseteq_\Delta \sigma} $
\end{center}
\label{thm:sdSmallNaxComplete}
\end{theorem}

We have argued that Theorem\;\ref{thm:sdSmallNaxSound} and
Theorem\;\ref{thm:sdSmallNaxComplete} hold for all the typing rules
in SmallNax excluding the rules for datatypes. So, we only need to check
whether these two theorems hold for the rules for datatypes, that is,
for the declarative rules \rulename{Con}, \rulename{Case}, and \rulename{Alt},
and their syntax-directed counterparts \rulename{Con$_s$}, \rulename{Case$_s$},
and \rulename{Alt$_s$}. They obviously hold for the rules \rulename{Con} and
\rulename{Con$_s$} because these rules have exactly the same structure as 
\rulename{Var} and \rulename{Var$_s$}. Once we know that \rulename{Alt} is
sound and complete with respect to \rulename{Alt$_s$}, it is quite
straightforward to show that \rulename{Case} is sound and complete
with respect to \rulename{Case$_s$} because the typing one can get
from \rulename{Case$_s$} is a generic instantiation of the typing
one can get from \rulename{Case}. It is indeed the case that
\rulename{Alt} is sound and complete with respect to \rulename{Alt$_s$}
because they have exactly the same structure. Unlike other syntax-directed
typing rules of the form $\Delta;\Gamma |-s t : A$, which assign
a type to a monomorphic type ($A$), the rule \rulename{Alt$_s$}
of the form $\Delta;\Gamma |-s^\psi C\overline{x} -> t : \sigma$
assigns a polymorphic type scheme ($\sigma$). Since \rulename{Alt} and
\rulename{Alt$_s$} have exactly the same structure, one calling on $|-$
and the other calling on $|-s$ in their premises, they must be sound
and complete with respect to each other.

\paragraph{SmallNax is strongly normalizing and logically consistent.}
The type system of SmallNax is sound with respect to System \Fw.
That is, when $\Delta;\Gamma |- t:\sigma$ in SmallNax, then 
$\Delta;\Gamma |- t:\sigma$ in System \Fw.
Considering the let-term ($\<let> x=s \<in> t$) as a syntactic sugar of
a lambda term applied to the scrutinee ($(\l x.t)\,s$), the terms of SmallNax,
except data constructors and case expressions, are exactly the same as
the term of System \Fw, which we discussed in \S\ref{sec:fw}.
For SmallNax terms involving data constructors and case expressions, we use
the Church encoding to translate them into System \Fw\ terms.
We show the soundness of typing with respect to System \Fw\ by reasoning about
the declarative typing rules. Recall that we discussed
the soundness of typing for HM with respect to System \F\ by reasoning about
the declarative typing rules of HM in \S\ref{sec:hm}.

The kinding and typing rules, except those rules for datatypes, are admissible
in System \Fw. The kinding rules except \rulename{TCon} are exactly the same as
the kinding rules with the same name in System \Fw\ (see Figure\;\ref{fig:fw}
on page \pageref{fig:fw}). The declarative typing rules except \rulename{Con},
\rulename{Case}, and \rulename{Alt} are admissible in System~\Fw. The rules
\rulename{Var}, \rulename{Abs}, \rulename{App}, and \rulename{Gen}\footnote{
	The \rulename{Gen} rule in SmallNax corresponds to
	the \rulename{TyAbs} rule in System~\Fw (Figure\;\ref{fig:fw}).
	The other rules, \rulename{Var}, \rulename{Abs}, \rulename{App}
	corresponds to the rules with the same name in System~\Fw.}
are exactly the same as the typing rules of System~\Fw.
We can show that the rules \rulename{Let} and \rulename{Inst} in SmallNax are
admissible in System~\Fw\ by following virtually the same argument that 
we used to show that the rules \rulename{Let} and \rulename{Inst} in HM are
admissible in System~\F\ (see p.\pageref{hm:LetAdmissibleFw} in \S\ref{sec:hm}).
The \rulename{Let} rule in SmallNax corresponds to a consecutive use of
\rulename{App} and \rulename{Abs} in System \Fw. A single derivation step of
\rulename{Inst} in SmallNax corresponds to multiple uses of the rules
\rulename{TyAbs} and \rulename{TyApp} in System \Fw.

The kinding and typing rules involving datatypes (\rulename{TCon},
\rulename{Con}, \rulename{Case}, and \rulename{Alt}) can be understood as
being admissible in System \Fw\ via the Church encodings of datatypes.
In \S\ref{sec:f:data} and \S\ref{sec:fw:data}, we discussed
how non-recursive datatypes (\eg, unit, void, boolean, sums, products)
can be encoded as functions. The rules \rulename{TCon}, \rulename{Con},
\rulename{Case}, and \rulename{Alt} are compatible with those encodings.
Thus, the type system of SmallNax, described in Figure\;\ref{fig:SmallNax},
is sound with respect to System \Fw.
Therefore, SmallNax is strongly normalizing and logically consistent.

\paragraph{From System \Fw\ to SmallNax.}
We will discuss informal and high-level design concepts of what restrictions
from System \Fw\ make SmallNax feasible for type inference, rather than
formally discussing concrete type inference algorithms. There are two
restrictions from System \Fw, rank-1 polymorphism and type constructor names,
which make type inference decidable in SmallNax. In addition, we discuss
the role of index transformers in type inference. Although index transformers
are not essential for pattern matching of datatypes defined by non-recursive
equations, they do play essential roles in inferring types of recursive
datatype definitions and GADT-style datatype definitions.

Let us review what restriction from System \F\ makes HM (without recursion)
feasible for type inference. Type inference is undecidable in System \F\ 
due to its arbitrary rank polymorphism (\ie, polymorphic types can appear
arbitrary deep inside type constructor arguments, in particular, inside
the left-hand side of $->$). Type inference becomes decidable in HM
by restricting the polymorphism to be rank-1 (\ie, universal quantification
can only appear at the top level). Similarly, we restrict the polymorphism
to be rank-1 in SmallNax (see Definition\;\ref{def:SmallNax}).

In addition to arbitrary rank polymorphism, type abstractions
($\l X^\kappa.F$) in System \Fw\ are another feature that makes
type inference undecidable. Type inference algorithms involving
type abstractions would require higher-order unification (\ie, unification
involving reconstruction of function implementation), which is known to be
undecidable \cite{Gol81}. In SmallNax, we can avoid higher-order unification
because there are no type abstractions. Datatypes in SmallNax are introduced
into the context as primitives, that is, type constructor names into $\Delta$ and
their associated data constructors into $\Gamma$. So, we only need
first-order unification for inferring types in SmallNax.

\section{SmallNax with Mendler-style recursion}
\label{sec:naxTyInfer:rec}
In this section, we extend SmallNax with Mendler-style recursion combinators.
We first review type inference and recursion in \S\ref{sec:naxTyInfer:rec:MM}.
Then, we introduce the typing rules for Mendler-style iteration in SmallNax
and discuss the role of index transformers in type inference
in \S\ref{sec:naxTyInfer:rec:rules}.

\subsection{A review of monomorphic recursion and polymorphic recursion}
\label{sec:naxTyInfer:rec:MM}
The Hindley--Milner type system (HM) \cite{DamMil82} supports
monomorphic (general) recursion by assigning a monomorphic type ($A$)
to the recursive variable ($x$), as described in the rule \rulename{Fix-m}
below right.\footnote{
	In \S\ref{sec:hm}, we excluded the general recursion in
	our formalization of HM, although its original presentation
	has general recursion, because our language does not support
	general recursion.}
The Milner--Mycroft type system (MM) \cite{Myc84} supports
polymorphic recursion by assigning a polymorphic type ($\sigma$)
to the recursive variable ($x$), as described in the rule \rulename{Fix-p}
below left.
\[
\qquad
\inference[\sc Fix-m]{\Gamma,x:A |- t : A}{
	\Gamma |- \textbf{fix}\;x.t : A}
\qquad
\inference[\sc Fix-p]{\Gamma,x:\sigma |- t : \sigma}{
	\Gamma |- \textbf{fix}\;x.t : \sigma}
\]
Polymorphic recursion is necessary for writing recursive programs
involving nested datatypes. However, type inference for MM is
known to be undecidable \cite{Hen93}. That is, we cannot generally
decide whether a suitable $\sigma$ exists for $\textbf{fix}\;x.t$
in the rule \rulename{Fix-p}.

What makes HM (including \rulename{Fix-m}) particularly suitable for
type inference while type inference for MM is undecidable?
\citet{Hen93} summarized the peculiarity of HM is that occurrences of
a recursive definition ``\emph{inside} the body of its definition can
only be used \emph{monomorphically}'', whereas occurrences ``\emph{outside}
its body can be used \emph{polymorphically}''.

\subsection{Typing rules for Mendler-style recursion combinators}
\label{sec:naxTyInfer:rec:rules}
We design the typing rules for recursion combinators in SmallNax
based on a similar idea to what makes HM suitable for type inference.
We first discuss a simplified version (with less polymorphism) of
the typing rule for \MIt\ in Figure\;\ref{fig:SmallNaxRec}. Then,
we illustrate a more polymorphic version, which is actually used
in the Nax implementation, in Figure\;\ref{fig:SmallNaxRecGen}.

What makes SmallNax, including the \rulename{mit$'$} rule
in Figure\;\ref{fig:SmallNaxRec}, suitable for type inference is
that the \emph{type parameters} of the recursive function argument are
\emph{monomorphic}, whereas the \emph{type indices} are \emph{polymorphic}
inside the body of the recursive function definition.
Outside the body, the recursive function can be used polymorphically over
both type parameters and type indices.

\begin{figure}
\begin{singlespace}
\textbf{Syntax}\vspace*{-.5ex}
\begin{align*}
&\textbf{Term}&
t,s&~::= ~ \dots \; \mid \; \MIt_\kappa\;x\;\varphi^\psi
\\
&\textbf{Type constructor}&
F,G,A,B&~::= ~ \dots \; \mid \; \mu_\kappa (T\,\overline{G})
\end{align*}
~ \vspace*{-1.3ex} \\
\textbf{Kinding rules} \qquad
$ \inference[\sc mu]{\Delta |- T\,\overline{G} : \kappa -> \kappa}{
	\Delta |- \mu_\kappa (T\,\overline{G}) : \kappa}
$ \\ ~ \\
\textbf{Typing rules}
\[ \inference[\sc mit$'$]{
	\Delta,X_r^\kappa;
	\Gamma,x:\forall\overline{X^{\kappa'}}.X_r\,\overline{X} -> \psi(\overline{X})
	|- \varphi^\psi :
	\forall\overline{X^{\kappa'}}.F\,X_r\,\overline{X} -> \psi(\overline{X})
	}{ \Delta;\Gamma |- \MIt_\kappa\;x\;\varphi^\psi:
	\forall\overline{X^{\kappa'}}.\mu_\kappa F\,\overline{X} -> \psi(\overline{X}) }
\]
\end{singlespace}
\caption{SmallNax extended with $\mu_\kappa$ and $\MIt_\kappa$,
	using a simplified version of the inference rule for $\MIt_\kappa$}
\label{fig:SmallNaxRec}
\end{figure}

Figure\;\ref{fig:SmallNaxRec} highlights the extended parts of the syntax,
kinding rules, and typing rules of SmallNax from Figure\;\ref{fig:SmallNax}.
The term syntax is extended with the Mendler-style iteration combinator.
An application of the Mendler-style iteration combinator to a term
$(\MIt_\kappa\;x\;\overline{C\overline{x} -> t}^\psi)\;s$ in SmallNax
corresponds to $\MIt_\psi\;s\;\overline{x~(C\overline{x}) -> t}$ in Nax,
where $x$ is the name for the abstract recursive function call used in
each case branch $t$. We relaxed the syntax of the Mendler-style iteration
combinator to be used as a first-class function without its recursive argument
$s$ in SmallNax: for the same reason we relaxed the syntax of case expressions
($\textbf{case}_\psi\;s\<of>\varphi$) in Nax into case functions
($\varphi^\psi$) in SmallNax so that it could be used without
the scrutinee ($s$) -- we do not need a separate rule for
the application of the Mendler-style iteration combinator.
The kinding rule \rulename{mu} and the typing rule \rulename{mit$'$} are
admissible in System \Fw\ by the embedding of $\mu_\kappa$ and $\mit_\kappa$
into System \Fw\ as discussed in \S\ref{ssec:embedTwoLevel}.

The rule \rulename{mit$'$} is suitable for type inference.
Unlike the rule \rulename{Fix-p} in MM, which cannot always determine
the type scheme ($\sigma$) of the general recursion ($\textbf{fix}\;x.t$),
the rule \rulename{mit$'$} unambiguously determines the type scheme
($\forall\overline{X^{\kappa'}}.\mu_\kappa F\,\overline{X} -> \psi(\overline{X})$)
of the Mendler-style iteration ($\MIt_\kappa\;x\;\varphi^\psi$) from
the index transformer ($\psi$) -- the number and kinds of universally
quantified variables ($\forall\overline{X^{\kappa'}}.\cdots$) in the type scheme
must match the number and kinds of the arguments to the index transformer.
The rule \rulename{mit$'$} uniquely determines the type scheme of
the Mendler-style iteration, except $F$. The type constructor $F$
in the rule \rulename{mit$'$} is determined by the rules \rulename{Con},
\rulename{Case}, and \rulename{Alt} in Figure\;\ref{fig:SmallNax}.

We can formulate the syntax-directed counterpart of the rule \rulename{mit$'$}
as follows:
\[ \inference[\sc mit$'_s$]{
	\Delta,X_r^\kappa;
	\Gamma,x:\forall\overline{X^{\kappa'}}.X_r\,\overline{X} -> \psi(\overline{X})
	|-s \varphi^\psi :
	\forall\overline{X^{\kappa'}}.F\,X_r\,\overline{X} -> \psi(\overline{X})
	}{ \Delta;\Gamma |-s \MIt_\kappa\;x\;\varphi^\psi:
	\mu_\kappa F\,\overline{G} -> \psi(\overline{G}) }
\]
Note that the type $\mu_\kappa F\,\overline{G} -> \psi(\overline{G})$
in \rulename{mit$'_s$} is a generic instance of the type scheme
$\forall\overline{X^{\kappa'}}.\mu_\kappa F\,\overline{X} -> \psi(\overline{X})$
in \rulename{mit$'$}.
So, the relation between \rulename{mit$'$} and \rulename{mit$'_s$} are
similar to the relation between \rulename{Case} and \rulename{Case$_s$}.
Therefore, the declarative typing rules including \rulename{mit$'_s$}
are sound and complete with respect to the syntax-directed rules including
\rulename{mit$'_s$}, for similar reasons as to why \rulename{Case$_s$}
is sound and complete with respect to \rulename{Case}.

We need additional polymorphism in the typing rule for \MIt\ when we have free
variables in the index transformer ($\psi(\overline{X})$), other than
the indices ($\overline{X}$) of the argument type. For example, consider
the index transformer $\{\{t\}\,.\,
   \texttt{Code}\,\{\textit{ts}\}\,\{\texttt{`cons}\;t\;\textit{ts}\}\}$
appearing in the definition of the stack-safe compiler (\textit{compile})
in \S\ref{ssec:compile}. The free variable (\textit{ts}) should be generalized
as well as the index ($t$) to infer the type of the \textit{compile} function.
The type scheme for the recursive caller ($x$ in the typing rule for \MIt)
should be fully generalized, except for the part of the input type
up to its parameters. That is, we should generalize both the indices
and the free variables of the index transformer. The idea described
in this paragraph is summarized as the rule \rulename{mit}
in Figure\;\ref{fig:SmallNaxRecGen}.

\begin{figure}
\begin{singlespace}
\centering
\[ \inference[\sc mit]{
	\Delta,X_r^\kappa;
	\Gamma,x:\forall\overline{X'^{\kappa'}}.X_r\,\overline{X} -> \psi(\overline{X})
	|- \varphi^\psi :
	\forall\overline{X'^{\kappa'}}.F\,X_r\,\overline{X} -> \psi(\overline{X})
	}{ \Delta;\Gamma |- \MIt_\kappa\;x\;\varphi^\psi:
	\forall\overline{X'^{\kappa'}}.\mu_\kappa F\,\overline{X} -> \psi(\overline{X}) }
\]
$\begin{array}{rl}
\text{where}\!& \overline{X'} = \overline{X} \cup \FV(\psi(\overline{X}))
				\setminus \dom(\Delta) \setminus \FV(\Gamma) \\
\text{and}\!& \text{each $\kappa'$ is an appropriate kind for each $X'$}
\end{array}$
\end{singlespace}
\caption{A more polymorphic version of the inference rule for $\MIt_\kappa$}
\label{fig:SmallNaxRecGen}
\end{figure}

We only discussed Mendler-style iteration, but it is straightforward to
formulate similar typing rules for other Mendler-style recursion schemes.

\section{SmallNax with GADTs}
\label{sec:naxTyInfer:gadt}
So far in this chapter, we have only considered those datatypes defined
by equations.\footnote{A recursive datatype defined using $\mu_\kappa$
over non-recursive equational datatypes can also be described by
a recursive equation.} Such equational datatypes are either regular
(\eg, homogeneous lists) or nested (\eg, powerlists, bushes).
As discussed in \S\ref{sec:naxTyInfer:small}, data constructors
introduced from an equational definition
(\ie, $\<data> T\,\overline{X} = \overline{C \overline{G}}$) have
uniform return types ($T\;\overline{X}$) and no existential variables
in their types. GADT definitions can introduce a wider range of datatypes,
including data constructors with non-uniform return types and data constructors
with existential type variables in their types.

\subsection{Existential type variables}
\label{sec:naxTyInfer:gadt:ex}
GADT definitions can introduce \emph{existential type variables}
in the types of data constructors. Existential type variables are
type variables that do not appear in the result type. In fact,
we have already seen an example of a GADT definition that contains
existential type variables in earlier chapters. Consider
the simply-typed HOAS datatype, which we discussed in \S\ref{sec:evalHOAS},
defined as a GADT in Haskell:\footnote{
	In Nax, we should define \texttt{Exp} in two levels
	by taking the fixpoint $\mu_{* -> *}$ over a non-recursive GADT.
	For simplicity, we illustrate the type-indexed expression datatype
	using Haskell since Haskell GADTs allow recursive definitions.}
\vspace*{-5.2ex}
\begin{singlespace}
\begin{verbatim}
  data Exp t where
    Lam :: (Exp a -> Exp b) -> Exp (a -> b)
    App :: Exp a -> Exp (a -> b) -> Exp b
\end{verbatim}
\end{singlespace}
~\vspace*{-5.3ex}\\
Note that the return types of the two data constructors are not uniform.
The return type of \texttt{Lam} is \texttt{Exp (a -> b)} and
the return type of \texttt{App} is \texttt{Exp b}. Also note that
the type variable \texttt{a} in the type of \texttt{App} does not appear in
the result type \texttt{Exp b}. So, \texttt{a} is an existential type variable
by definition. When we have an application expression
\texttt{\;(App e1 e2)~::~Exp b}, we know that there exists some \texttt{a}
such that \texttt{\;e1~::~a\;} and \texttt{\;e2~::~a -> b}, but there is no way
to statically know what \texttt{a} is even when we have more information about
\texttt{b}. Existential type variables must remain abstract in pattern matching.
That is, they can never be instantiated inside alternatives.

To handle existential variables, we adjust the rule \rulename{Alt}
in Figure\;\ref{fig:SmallNax} as follows:
\begin{align}
 \inference[\sc Alt]{
	 \Delta,\exists_\Gamma(C);\exists_C(\Gamma) |- C:\overline{A} -> T\overline{B}\,\overline{A'} \\
	 \Delta,\exists_\Gamma(C);\Gamma,\overline{x:A} |- t : \psi(\overline{A'}) \qquad~
            }{ \Delta;\Gamma |-^\psi C\overline{x} -> t
               : \forall\overline{X^\kappa}.
                          T\overline{B}\,\overline{X} -> \psi(\overline{X}) }
\label{AltEx}
\end{align}
where $\exists_\Gamma(C)$ is the list of existential variable bindings of $C$
and $\exists_C(\Gamma)$ drops the universal quantification of
the existential variables in the type scheme of $C$
(\ie, makes them into free variables) so that they become abstract.
That is, when $C:\forall\overline{X^\kappa}.A \in \Gamma$ and
$\overline{X'^{\kappa'}} = \overline{X^\kappa} \setminus \exists(C)$,
then $C:\forall\overline{X'^{\kappa'}}.A \in \exists_C(\Gamma)$ and
all other bindings in $\exists_C(\Gamma)$ remain the same as in $\Gamma$.
We only need to make existential variables abstract when assigning types
for pattern variables ($\overline{x:A}$). So, we use $\exists_C(\Gamma)$
only in the first premise. In the second premise, where we type check
the body ($t$) of the alternative, we use the original $\Gamma$ so that
all quantified variables in the type scheme of $C$ can be instantiated.
For example, we should be able to apply
\texttt{\;App~::~Exp\;a -> Exp\;(a -> b) -> Exp\;b\;}
in the example, which we discussed above, to an expression of any type
(\eg, \texttt{Exp\;Int}, \texttt{Exp\;(Int -> Bool)}).

\subsection{Generalized existential type variables and index transformers}
\label{sec:naxTyInfer:gadt:gex}
\citet{Lin10thesis} generalized the notion of existential type variables,
calling them \emph{generalized existential type variables}, while developing
a practical type inference algorithm for GADTs. Intuitively,
generalized existential type variables are ``type variables introduced by
a pattern that receive no parametric instantiation'' \cite{Lin10thesis}.
Generalized existential type variables are a conservative extension of
existential type variables. That is, all existential type variables
are generalized existential type variables. Here, we focus on
the generalized type variables that are not existential type variables.

Consider the type representation (whose value describes the structure of
a type. \aka\ type universe) defined as a GADT in Haskell:
\vspace*{-5ex}
\begin{singlespace}
\begin{verbatim}
  data Rep t where
    INT :: Rep Int
    PAIR :: Rep a -> Rep b -> Rep (a, b)
    FUN :: Rep a -> Rep b -> Rep (a -> b)
\end{verbatim}
\end{singlespace}
~\vspace*{-5ex}\\
There are no existential type variables in the definition of \texttt{Rep}.
The first type constructor \texttt{INT} does not have any type variables
in its type. In the other two type constructors \texttt{PAIR} and \texttt{FUN},
both the type variables \texttt{a} and \texttt{b} appear in their result types,
\texttt{\;Rep~(a,b)\;} and \texttt{\;Rep~(a -> b)\;}. These type variables
could be \emph{generalized existential variables}, which should not be
instantiated to other types, just like existential type variables.
For example, when we pattern match against a value of \texttt{\;Rep\;t},
\texttt{a} and \texttt{b} must remain polymorphic inside the alternatives.
For instance, in the alternative for \texttt{PAIR}, we know that
\texttt{t} = \texttt{(a,b)}, but we should instantiate neither \texttt{a}
nor \texttt{b} because \texttt{t} is polymorphic.

Generalized existential type variables are
extrinsic properties of data constructors, unlike existential type variables,
which are intrinsic properties of data constructors \cite{Lin10thesis}.
Existential type variables always remain abstract (or polymorphic) inside
alternatives regardless of the scrutinee type. On the other hand,
generalized existential type variables depend on the scrutinee type. That is,
we could have a different set of generalized existential type variables
for the very same data constructor, depending on the type of the scrutinee
we pattern match against. For example, when we pattern match against
a value of \texttt{Exp\;(Int,Bool)}, there is no generalized existential
type variable inside the alternative for \texttt{PAIR},\footnote{
	Other alternatives, for \texttt{INT} and \texttt{FUN} are unreachable.
	So, for the scrutinee with type \texttt{Exp\;(Int,Bool)}, all cases
	are covered by a single alternative for \texttt{PAIR}.
	To make coverage checking for such scrutinee types aware of
	unreachable cases, we need an advanced coverage checking algorithm
	rather than just making sure that there exists an alternative for
	every data constructor. Coverage checking for our Nax implementation
	remains future work.}
because
\texttt{a} and \texttt{b} are instantiated to \texttt{Int} and \texttt{Bool}.
%% \citet{Lin10thesis} develops a type inference algorithm for Haskell-like
%% languages with GADTs using the notion of generalized existential type variables.
%% He claims that the notion of generalized existential type variables clarifies
%% when the GADT type inference algorithm should propagate type information
%% from the body of an alternative to the scrutinee type.

We further adjust the rule \rulename{Alt}, adopting Lin's notion of
generalized existential variables, as follows:
\begin{align}
 \inference[\sc Alt]{
	 \Delta,\exists^\psi_\Gamma(C);\exists^\psi_C(\Gamma) |- C:\overline{A} -> T\overline{B}\,\overline{A'} \\
	 \Delta,\exists^\psi_\Gamma(C);\Gamma,\overline{x:A} |- t : \psi(\overline{A'}) \qquad~
            }{ \Delta;\Gamma |-^\psi C\overline{x} -> t
               : \forall\overline{X^\kappa}.
                          T\overline{B}\,\overline{X} -> \psi(\overline{X}) }
\label{AltGEx}
\end{align}
The change from the previous \texttt{Alt} rule (\ref{AltEx})
on page \pageref{AltEx} is using $\exists^\psi_\Gamma$ and $\exists^\psi_C$
for generalized existential type variables instead of
$\exists^\psi_\Gamma$ and $\exists^\psi_C$ for existential type variables.
For example, $\exists^\psi_\Gamma(\texttt{PAIR})$ is the list of two variables
consisting of \texttt{a} and \texttt{b}, while $\exists_\Gamma(\texttt{PAIR})$
is an empty list. Accordingly, the type scheme binding for \texttt{PAIR}
in $\exists^\psi_\texttt{PAIR}(\Gamma)$ is a monomorphic type
\texttt{\;Rep\;a -> Rep\;b -> Rep\;(a,\,b)\;} where \texttt{a} and \texttt{b}
are free,\footnote{
	They are bound in $\exists^\psi_\Gamma(\texttt{PAIR})$, of course.}
while the type scheme binding for \texttt{PAIR} in 
$\exists_\texttt{PAIR}(\Gamma)$ is a polymorphic type scheme
$\forall\texttt{a}^{*}.\forall\texttt{b}^{*}.\,
	\texttt{Rep\;a -> Rep\;b -> Rep\;(a,\,b)}$.

The \rulename{Alt} rule above (\ref{AltGEx}) still lacks consideration for
case functions that only expect scrutinee types with more specific indices
than fully polymorphic variables (\eg, pattern matching only against
values of type \texttt{Exp\;(Int,Bool)}, rather than \texttt{Exp\;t} for
any type \texttt{t}). In our Nax implementation, we do support the syntax for
such pattern matching by allowing index transformers of the form
such as \texttt{\{\,(Int,Bool)\;.\;$A$\,\}}, whose argument list
on the left-hand side of the dot could contain more specific forms
than just type variables. Formulation of typing rules, which further adjust
from the \rulename{Alt} rule (\ref{AltGEx}), and implementation of
coverage checking considering such index transformers with
constrained input index are left for future work.

%% $\exists^\psi_\Gamma(C)$, the list of generalized existential type variables
%% for the data constructor $C$. That is, all the existential type variables
%% ($\exists_\Gamma(C)$), and, the type variables additionally introduced from
%% the indices that are more specific than variable forms
%% in the result type of $C$ 


%% Nax simply avoids the concern of propagating type information between
%% the scrutinee and the alternatives by requiring explicit annotations
%% on the pattern matching constructs (\ie, case expressions
%% and Mendler-style recursion combinators) over indexed datatypes.
%% The index transformer $\psi = \{ X_t\;.\;A \}$ annotated on
%% the case function $\phi^\psi$ can be applied to scrutinee type
%% of arbitrary index. Thus, in the alternatives of $\phi^\psi$,
%% we expect indices to be fully polymorphic. For instance
%% When we pattern matching over a fully polymorphic index
%% $\texttt{Exp}\;X_t$
%% 
%% 
%% $\texttt{Exp}\;(\texttt{Int},\texttt{Bool})$
%% 
%% $\{ (\texttt{Int},\texttt{Bool})\;.\;A \}$
%% 
%% Since we only need to consider existential type variables,
%% but not generalized existential type variables, in Nax,
%% the adjusted \texttt{Alt} rule we discussed in the previous subsection
%% is sufficient.

