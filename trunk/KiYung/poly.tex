\chapter{Polymorphic type systems}\label{ch:poly}
We first review the simply-typed lambda calculus (\S\ref{sec:stlc}),
which is a non-polymorphic type system, to motivate the discussions
on a series of polymorphic type systems -- System \F\ (\S\ref{sec:f}),
System \Fw\ (\S\ref{sec:fw}), and the Hindley-Milner type system
(\S\ref{sec:hm}). We assume the reader to have some familiarity with
lambda calculi, at least with the untyped lambda calculus.

\section{The simply-typed lambda calculus}\label{sec:stlc}
\begin{figure}
\begin{singlespace}
\begin{minipage}{.46\textwidth}
	\begin{center}Church-style\end{center}
\def\baselinestretch{0}
\small
\begin{align*}
\textbf{term syntax} \\
t,s ::= &~ x           & \text{variable}    \\
      | &~ \l(x:A) . t & \text{abstraction} \\
      | &~ t ~ s       & \text{application} \\
\\
\textbf{type syntax} \\
A,B ::= &~ A -> B  & \text{arrow type} \\
      | &~ \iota   & \text{ground type}   \\
\end{align*}
\[ \textbf{typing context} \]\vspace*{-1em}
\begin{align*}\quad
\Gamma ::= &~ \cdot \\
	 | &~ \Gamma, x:A \quad (x\notin \dom(\Gamma)) \\
\end{align*}
\[ \textbf{typing rules}
	\qquad \framebox{$\Gamma |- t : A$} \]
\vspace*{-1em}
\begin{align*}
& \inference[\sc Var]{x:A \in \Gamma}{\Gamma |- x:A} \\
& \inference[\sc Abs]{\Gamma,x:A |- t : B}
		     {\Gamma |- \l(x:A).t : A -> B} \\
& \inference[\sc App]{\Gamma |- t : A -> B & \Gamma |- s : A}
		     {\Gamma |- t~s : B} \\
\end{align*}
\[ \textbf{reduction rules}
	\quad \framebox{$t --> t'$} \]
\vspace*{-1em}
\begin{align*}
& \inference[\sc RedBeta]{}{(\l(x:A).t)~s --> t[s/x]} \\
& \inference[\sc RedAbs]{t --> t'}{\l(x:A).t --> \l(x:A).t'} \\
& \inference[\sc RedApp1]{t --> t'}{t~s --> t'~s} \\
& \inference[\sc RedApp2]{s --> s'}{t~s --> t~s'} \\
\end{align*}
\end{minipage}
\begin{minipage}{.46\textwidth}
	\begin{center}Curry-style\end{center}
\def\baselinestretch{0}
\small
\begin{align*}
\textbf{term syntax} \\
t,s ::= &~ x           \\
      | &~ \l x    . t \\
      | &~ t ~ s       \\
\\
\textbf{type syntax} \\
A,B ::= &~ A -> B \\
      | &~ \iota  \\
\end{align*}
\[ \textbf{typing context} \]\vspace*{-1em}
\begin{align*}\quad
\Gamma ::= &~ \cdot \\
	 | &~ \Gamma, x:A \quad (x\notin \dom(\Gamma)) \\
\end{align*}
\[ \textbf{typing rules}
	\qquad \framebox{$\Gamma |- t : A$} \]
\vspace*{-1em}
\begin{align*}
& \inference[\sc Var]{x:A \in \Gamma}{\Gamma |- x:A} \\
& \inference[\sc Abs]{\Gamma,x:A |- t : B}
		     {\Gamma |- \l x   .t : A -> B} \\
& \inference[\sc App]{\Gamma |- t : A -> B & \Gamma |- s : A}
		     {\Gamma |- t~s : B} \\
\end{align*}
\[ \textbf{reduction rules}
	\quad \framebox{$t --> t'$} \]
\vspace*{-1em}
\begin{align*}
& \inference[\sc RedBeta]{}{(\l x   .t)~s --> t[s/x]} \\
& \inference[\sc RedAbs]{t --> t'}{\l x   .t --> \l x   .t'} \\
& \inference[\sc RedApp1]{t --> t'}{t~s --> t'~s} \\
& \inference[\sc RedApp2]{s --> s'}{t~s --> t~s'} \\
\end{align*}
\end{minipage}
~\\
\caption{Simply-typed lambda calculus in Church-style and Curry-style}
\label{fig:stlc}
\end{singlespace}
\end{figure}
We illustrate two styles of the simply-typed lambda calculus (STLC)
in Figure \ref{fig:stlc}. In both styles, a term can either be a variable,
an abstraction (\aka\ lambda term), or an application; and, a type can
either be an arrow type or a ground type. The distinction between
the two style comes from whether the abstraction has type annotation
in the term syntax. The difference in typing rules and reduction rules
follows from this difference in the term syntax. The type syntax are
exactly the same for both styles.

\paragraph{Remark on the ground type:}
Before we elaborate on each style, I'd like to make a remarks on 
the type syntax, in particular, on the ground type ($\iota$).
There would be no question for arrow types, which are types for functions,
since abstractions represent (certain implementations of) functions.
We need some ground types to populate types. Otherwise, if there weren't
any ground types, we won't have any types\footnote{If we allow infinite
types, then it is possible to do away with ground types. There exist
exotic lambda calculi with infinite types, but rather uncommon.}
-- then, such version of STLC will be very uninteresting since there cannot
be any well-typed terms satisfying the typing judgement according to
the typing rules. Here, I provide the most simple ground type ($\iota$),
which does not inhabit any terms (\aka\ the void type). Note that there
exists no term of type $\iota$. We can have functions over $\iota$,
such as ($\l(x:\iota).x$), the identity on $\iota$ .\footnote{Here,
	I present examples in Church-style since it is more succinct than
	writing typing judgments (\eg, $(\l x.t) : \iota -> \iota$)
	in the Curry-style.  But, the remark on the ground type $\iota$
	holds the same for both styles.}
However, there exist no term, to which we cannot apply ($\l x:\iota.x$).
What we can do is to apply higher-order functions
(\eg, $\l(x_{f}:\iota -> \iota).\l(x_a:\iota).x_f~x_a$, which expects
an argument of type $\iota -> \iota$) to the functions over $\iota$.

When we use STLC as a model of a programming language (with simple types),
we usually need to provide richer set of ground types other than
the void type (\eg, unit, boolean, natural numbers). In such versions of
STLC with further extension to the type syntax, we also need to extend
the term syntax by providing normal terms for the ground types
(\eg, \textsf{true} and \textsf{false} for the boolean type)
and eliminators (\eg, if-then-else expression for booleans)
that can examine the normal terms. Here, having just the void type is enough
for my purpose of leading up the discussion for the polymorphic type systems,
without complicating the term syntax.

\paragraph{Well-known properties of STLC (in both styles):}
type preservation (\aka\ subject reduction)
reduction preserves type
\begin{align*}
\inference{\Gamma |- t : A  & t --> t'}{\Gamma |- t' : A}
 &\qquad \text{type preservation, or, subject reduction} \\
\end{align*}

Strong normalization

\paragraph{In Church style,} the variable ($x$) in an abstraction
($\l(x:A).t$) has a type annotation ($A$). Intuitively, we may think of
the abstraction ($\l(x:A).t$) as a function that expects an argument of
the type ($A$) specified by the type annotation.

There are some interesting properties that holds in Church style,
but not in Curry style -- \emph{Uniqueness of typing} and
\emph{type preservation over well-typed $\beta$-equivalence}.

\emph{Uniqueness of typing}, described below, holds in the Church-style STLC.
\[ \inference{\Gamma |- t : A & \Gamma |- t : A'}{A = A'} \]
More specifically, given a well-formed typing context $\Gamma$ and
a term $t$, if the term is well-typed, that is, $\Gamma |- t : A$,
then $A$ is the unique such type. We can prove this by induction
on the derivation of the typing judgment. For variables, it trivially holds
since the variables appearing in the typing contexts are unique.
For abstractions, we use induction on the derivation. To use the induction
hypothesis we should make sure that the typing context ($\Gamma,x:A$)
and the term $t$ of the premise is uniquely determined. It is easy to see
that they are uniquely determined since all the peaces appearing in the
input (\ie, the typing context and the term) $\Gamma$, $A$, and $t$ are
part of the input, in particular the term, of the conclusion. Therefore,
by induction hypothesis, $B$ is uniquely determined, and, as a consequence,
$A -> B$ is uniquely determined. For application, it is easy to show by
induction for each of the premise. This proof describes the essence of
the type reconstruction algorithm for the Church-style STLC.

Another property (the latter below)
\begin{align*}
\inference{\Gamma |- t : A  & t --> t'}{\Gamma |- t' : A}
 &\qquad \text{type preservation, or, subject reduction} \\
\inference{\Gamma |- t' : A' & t --> t' & \Gamma |- t : A}{A = A'}
 &\qquad \text{Well-typed expansion preserve types}
\end{align*}
\KYA{ In general, well-typeness is not preserved under $\beta$-expansion
	-- there is a well-known $KI\Omega$ example.  Therefore,
	type preservation for $\beta$-expansion, or, subject expansion
	does not hold even in Church style. However, if both sides of $-->$
	are well-typed (or, we can just say Well-typed expansion
	since for redeuction it is just consequence of subject reduction
	and uniqueness of typing. In fact it is just a consequence of them.)
	types are preserved in Church style
	but definitely not in Curry style.
     }

\paragraph{In Curry style,} there is no annotation on the variable in
an abstraction. Since the variable binding in the abstraction is no longer
fixed to a specific type, \emph{uniqueness of typing} does not hold,
unlike in Church style. For instance, the identity function ($\l x.x$)
could have one of any type that has of the form $A -> A$, such as:
\begin{quote}\vspace*{-1em}
\begin{singlespace}
$\iota -> \iota$ \\
$(\iota -> \iota) -> (\iota -> \iota)$ \\
$(\iota -> (\iota -> \iota)) -> (\iota -> (\iota -> \iota))$ \\
$((\iota -> \iota) -> \iota) -> ((\iota -> \iota) -> \iota)$ \\
$((\iota -> \iota) -> (\iota -> \iota)) -> ((\iota -> \iota) -> (\iota -> \iota))$ \\
$~~~~ \vdots $
\end{singlespace}
\end{quote}
So, we read the typing judgment $\Gamma |- t : A$ in Curry style as
\begin{quote}
$t$ \emph{can have type} $A$ under the typing context $\Gamma$,
\end{quote}
unlike in Church style where we read the typing judgment as
\begin{quote}
$t$ \emph{has the type} $A$ under the typing context $\Gamma$.
\end{quote}
However, we don't consider the Curry-style STLC to be
a polymorphic type system since the typing for a variable
under a well-formed context is still unique. That is,
\[ \inference{\Gamma |- x : A & \Gamma |- x : A'}{A = A'} \]

%%%% need to rewrite this
%%%%%%%%%%%% Well-typed expansion does not preserve types in the Curry style.
%%%%%%%%%%%%%%%%%%%%%%%%%%%%%
%% Type preservation for $\beta$-expansion does not to hold either,
%% unlike in Church style. More fundamentally, well-typedness is not preserved
%% for inverse $\beta$-reduction. The reason for this is exactly because
%% uniqueness of typing does not hold in Curry style. Consider the following
%% $\beta$-reduction:
%% \[(\l x_I. (x_I~s_1)~(x_I~s_2)) (\l x.x) --> ((\l x.x)~s_1)~((\l x.x)~s_2)\]
%% This is a perfectly fine reduction according to the reduction rules for the
%% Curry-style STLC in Figure \ref{fig:stlc}. However, there exist $s_1$ and
%% $s_2$ such that the right-hand side is well-typed but the left-hand side
%% is not.
%% 
%% The left-hand side cannot be well-typed, regardless of $s_1$ and $s_2$.
%% The variable $x_I$ appearing in the abstraction must have a type of the form
%% $A -> A$, since the abstraction is applied to the identity term $(\l x.x)$.
%% In the body of the abstraction, which is an application
%% $(x_I~s_1)~(x_I~s_2)$, we have a problem. Since $x_I$ has an arrow type
%% that expects an agrument of type $A$ and returns a result of type $A$,
%% both $(x_I~s_1)$ and $(x_I~s_2)$ have the same type $A$.  Thus,
%% the application $(x_I~s_1)~(x_I~s_2)$ cannot be well-typed, since
%% the function $(x_I~s_1)$ applied to the argument $(x_I~s_2)$ of type $A$
%% must have type $A -> A$. But, we know that $(x_I~s_1)$ has type $A$.
%% The fact that uniqeness of typing does not hold won't help us resolve
%% the contraticting typing requrements of $(x_I~s_1)$ here, because
%% the typing of a variable is still unique under a well-formed context.
%% There is no way to satisfy both $\Gamma |- x_I : A -> A$ and 
%% $\Gamma |- x_I : (A -> A) -> (A -> A)$ even in the Curry-style STLC.
%% 
%% There exist $s_1$ and $s_2$ such that the right-hand side is well-typed.
%% Actually, there are many such $s_1$ and $s_2$. I will just give two
%% instances of them. Try to justify them yourself. They are easy exercises.
%% One instance is to let $s_1 = (\l x_I . \l x . x_I~x)$, which is
%% a higher-order function, and $s_2 = (\l x. x)$. Since uniqueness typing
%% does not hold, we are able to find approprite types for each occcurence of
%% ($\l x.x$) appearing in the right-hand side. Another instance is having
%% both of them to be the identity term, that is, let $s_1 = (\l x . x)$ and
%% $s_2 = (\l x . x)$ as well.

type inference in Curry-style


\section{System \F} \label{sec:f}
\begin{figure}
\begin{singlespace}
\begin{minipage}{.46\textwidth}
	\begin{center}Church-style\end{center}
\def\baselinestretch{0}
\small
\begin{align*}
\textbf{term syntax} \\
t,s ::= &~ x           & \text{variable}    \\
      | &~ \l(x:A) . t & \text{abstraction} \\
      | &~ t ~ s       & \text{application} \\
      | &~ \L X    . t & \text{type abstraction} \\
      | &~ t [A]       & \text{type application} \\
\\
\textbf{type syntax} \\
A,B ::= &~ X           & \text{variable type}   \\
      | &~ A -> B      & \text{arrow type} \\
      | &~ \forall X.B & \text{forall type}   \\
\end{align*}
\[ \textbf{kinding \& typing contexts} \]\vspace*{-1em}
\begin{align*}\quad
\Delta ::= &~ \cdot \\
	 | &~ \Delta, x:A & (X\notin \dom(\Delta)) \\
\Delta;\Gamma ::= &~ \Delta;\cdot \\
	        | &~ \Delta,X;\Gamma    & (X\notin \ran(\Gamma)) \\
	        | &~ \Delta;\Gamma, x:A & (x\notin \dom(\Gamma)) \\
\end{align*}
\[ \textbf{kinding rules} \quad \framebox{$ \Delta |- A $} \]\vspace*{-1em}
\begin{align*}
& \inference[\sc TVar]{X \in \Delta}{\Delta |- X} \\
& \inference[\sc TArr]{\Delta |- A & \Delta |- B}{\Delta |- A -> B} \\
& \inference[\sc TAll]{\Delta,X |- B}{\Delta |- \forall X.B} \\
\end{align*}
\[ \textbf{typing rules} \quad \framebox{$ \Delta;\Gamma |- t : A $ } \]
\vspace*{-1em}
\begin{align*}
& \inference[\sc Var]{x:A \in \Gamma}{\Delta;\Gamma |- x:A} \\
& \inference[\sc Abs]{\Delta |- A & \Delta;\Gamma,x:A |- t : B}
	             {\Delta;\Gamma |- \l(x:A).t : A -> B} \\
& \inference[\sc App]{\Delta;\Gamma |- t : A -> B & \Delta;\Gamma |- s : A}
		     {\Delta;\Gamma |- t~s : B} \\
& \inference[\sc TyAbs]{\Delta,X;\Gamma |- t : B}
		       {\Delta;\Gamma |- \L X.t : \forall X.B} \\
& \inference[\sc TyApp]{\Delta;\Gamma |- t : \forall X.B & \Delta |- A}
		       {\Delta;\Gamma |- t[A] : B[A/X]}
\end{align*}
\end{minipage}
\begin{minipage}{.46\textwidth}
	\begin{center}Curry-style\end{center}
\def\baselinestretch{0}
\small
\begin{align*}
\textbf{term syntax} \\
t,s ::= &~ x           \\
      | &~ \l x    . t \\
      | &~ t ~ s       \\
      \phantom{|} &~ \\
      \phantom{|} &~ \\
\\
\textbf{type syntax} \\
A,B ::= &~ X \\
      | &~ A -> B \\
      | &~ \forall X . B \\
\end{align*}
\[ \textbf{kinding \& typing contexts} \]\vspace*{-1em}
\begin{align*}\quad
\Delta ::= &~ \cdot \\
	 | &~ \Delta, x:A & (X\notin \dom(\Delta)) \\
\Delta;\Gamma ::= &~ \Delta;\cdot \\
	        | &~ \Delta,X;\Gamma    & (X\notin \ran(\Gamma)) \\
	        | &~ \Delta;\Gamma, x:A & (x\notin \dom(\Gamma)) \\
\end{align*}
\[ \textbf{kinding rules} \quad \framebox{$ \Delta |- A $}\]\vspace*{-1em}
\begin{align*}
& \inference[\sc TVar]{X \in \Delta}{\Delta |- X} \\
& \inference[\sc TArr]{\Delta |- A & \Delta |- B}{\Delta |- A -> B} \\
& \inference[\sc TAll]{\Delta,X |- B}{\Delta |- \forall X.B} \\
\end{align*}
\[ \textbf{typing rules} \quad \framebox{$ \Delta;\Gamma |- t : A $ } \]
\vspace*{-1em}
\begin{align*}
& \inference[\sc Var]{x:A \in \Gamma}{\Delta;\Gamma |- x:A} \\
& \inference[\sc Abs]{\Delta |- A & \Delta;\Gamma,x:A |- t : B}
		     {\Delta;\Gamma |- \l x   .t : A -> B} \\
& \inference[\sc App]{\Delta;\Gamma |- t : A -> B & \Delta;\Gamma |- s : A}
		     {\Delta;\Gamma |- t~s : B} \\
& \inference[\sc TyAbs]{\Delta,X;\Gamma |- t : B}
		       {\Delta;\Gamma |- t : \forall X.B} \\
& \inference[\sc TyApp]{\Delta;\Gamma |- t : \forall X.B & \Delta |- A}
		       {\Delta;\Gamma |- t : B[A/X]}
\end{align*}
\end{minipage}
~\\
\caption{System \F\ in Church-style and Curry-style}
\label{fig:f}
\end{singlespace}
\end{figure}



\section{System \Fw} \label{sec:fw}
\begin{figure}
\begin{singlespace}
\begin{minipage}{.46\textwidth}
	\begin{center}Church-Church-style\end{center}\vspace*{-1em}
\def\baselinestretch{0}
\small
\begin{align*}
\textbf{term syntax} \\
t,s ::= &~ x               & \text{variable}    \\
      | &~ \l(x:A) . t     & \text{abstraction} \\
      | &~ t ~ s           & \text{application} \\
      | &~ \L X^\kappa . t & \text{type abstraction} \\
      | &~ t [G]           & \text{type application} \\
\textbf{type syntax} \\
F,G,A,B ::= &~ X                  & \text{variable type} \\
          | &~ A -> B             & \text{arrow type} \\
          | &~ \forall X^\kappa.B & \text{forall type}   \\
          | &~ \l X^\kappa.B      & \text{lambda type}   \\
          | &~ F ~ G              & \text{application type}   \\
\textbf{kind syntax} \\
\kappa ::= &~ \kappa -> \kappa' & \text{arrow kind} \\
         | &~ *                 & \text{star kind}   \\
\end{align*}
\[ \textbf{kinding rules} \quad \framebox{$ \Delta |- F:\kappa $} \]\vspace*{-1em}
\begin{align*}
& \inference[\sc TVar]{X^\kappa \in \Delta}{\Delta |- X:\kappa} \\
& \inference[\sc TArr]{\Delta |- A:* & \Delta |- B:*}{\Delta |- A -> B:*} \\
& \inference[\sc TAll]{\Delta,X^\kappa |- B:*}
		      {\Delta |- \forall X^\kappa.B:*} \\
& \inference[\sc TLam]{\Delta,X^\kappa |- B:*}
		      {\Delta |- \l X^\kappa.B:\kappa -> *} \\
& \inference[\sc TApp]{\Delta |- F : \kappa -> \kappa' & \Delta |- G : \kappa}
		      {\Delta |- F ~ G : \kappa'} \\
\end{align*}
\[ \textbf{typing rules} \quad \framebox{$ \Delta;\Gamma |- t : A $ } \]
\vspace*{-1em}
\begin{align*}
& \inference[\sc Var]{x:A \in \Gamma}{\Delta;\Gamma |- x:A} \\
& \inference[\sc Abs]{\Delta |- A:* & \Delta;\Gamma,x:A |- t : B}
		     {\Delta;\Gamma |- \l(x:A).t : A -> B} \\
& \inference[\sc App]{\Delta;\Gamma |- t : A -> B & \Delta;\Gamma |- s : A}
		     {\Delta;\Gamma |- t~s : B} \\
& \inference[\sc TyAbs]{\Delta,X^\kappa;\Gamma |- t : B}
		       {\Delta;\Gamma |- \L X^\kappa.t : \forall X^\kappa.B} \\
& \inference[\sc TyApp]{\Delta;\Gamma |- t : \forall X^\kappa.B & \Delta |- G:\kappa}
		       {\Delta;\Gamma |- t[G] : B[G/X]} \\
& \inference[\sc Conv]{\Delta;\Gamma |- t : A & \Delta |- A = A' : *}
		      {\Delta;\Gamma |- t : A'}
\end{align*}
\end{minipage}
\begin{minipage}{.46\textwidth}
	\begin{center}Curry-Church-style\end{center}\vspace*{-1em}
\def\baselinestretch{0}
\small
\begin{align*}
\textbf{term syntax} \\
t,s ::= &~ x           \\
      | &~ \l x    . t \\
      | &~ t ~ s       \\
      \phantom{|} &~ \\
      \phantom{|} &~ \\
\textbf{type syntax} \\
F,G,A,B ::= &~ X                  \\
          | &~ A -> B             \\
          | &~ \forall X^\kappa.B \\
          | &~ \l X^\kappa.B      \\
          | &~ F ~ G              \\
\textbf{kind syntax} \\
\kappa ::= &~ \kappa -> \kappa' \\
         | &~ *                 \\
\end{align*}
\[ \textbf{kinding rules} \quad \framebox{$ \Delta |- F:\kappa$}\]\vspace*{-1em}
\begin{align*}
& \inference[\sc TVar]{X^\kappa \in \Delta}{\Delta |- X:\kappa} \\
& \inference[\sc TArr]{\Delta |- A:* & \Delta |- B:*}{\Delta |- A -> B:*} \\
& \inference[\sc TAll]{\Delta,X^\kappa |- B:*}{\Delta |- \forall X^\kappa.B:*} \\
& \inference[\sc TLam]{\Delta,X^\kappa |- B:*}
		      {\Delta |- \l X^\kappa.B:\kappa -> *} \\
& \inference[\sc TApp]{\Delta |- F : \kappa -> \kappa' & \Delta |- G : \kappa}
		      {\Delta |- F ~ G : \kappa'} \\
\end{align*}
\[ \textbf{typing rules} \quad \framebox{$ \Delta;\Gamma |- t : A $ } \]
\vspace*{-1em}
\begin{align*}
& \inference[\sc Var]{x:A \in \Gamma}{\Delta;\Gamma |- x:A} \\
& \inference[\sc Abs]{\Delta |- A:* & \Delta;\Gamma,x:A |- t : B}
		     {\Delta;\Gamma |- \l x   .t : A -> B} \\
& \inference[\sc App]{\Delta;\Gamma |- t : A -> B & \Delta;\Gamma |- s : A}
		     {\Delta;\Gamma |- t~s : B} \\
& \inference[\sc TyAbs]{\Delta,X^\kappa;\Gamma |- t : B}
		       {\Delta;\Gamma |- t : \forall X^\kappa.B} \\
& \inference[\sc TyApp]{\Delta;\Gamma |- t : \forall X^\kappa.B & \Delta |- G:\kappa}
		       {\Delta;\Gamma |- t : B[G/X]} \\
& \inference[\sc Conv]{\Delta;\Gamma |- t : A & \Delta |- A = A' : *}
		      {\Delta;\Gamma |- t : A'}
\end{align*}
\end{minipage}
~\\
\caption{System \Fw\ in Church-Church-style and Curry-Church-style}
\label{fig:fw}
\end{singlespace}
\end{figure}

\begin{figure}
\begin{singlespace}
\begin{minipage}{.46\textwidth}
	\begin{center}Curry-Church-style\end{center}
\def\baselinestretch{0}
\small
\begin{align*}
\textbf{term syntax} \\
t,s ::= &~ x           & \text{variable}    \\
      | &~ \l x    . t & \text{abstraction} \\
      | &~ t ~ s       & \text{application} \\
\textbf{type syntax} \\
F,G,A,B ::= &~ X                  & \text{variable type}    \\
          | &~ A -> B             & \text{arrow type}       \\   
          | &~ \forall X^\kappa.B & \text{forall type}      \\
          | &~ \l X^\kappa.B      & \text{lambda type}      \\
          | &~ F ~ G              & \text{application type} \\
\textbf{kind syntax} \\
\kappa ::= &~ \kappa -> \kappa' & \text{arrow kind} \\
         | &~ *                 & \text{star kind}   \\
\end{align*}
\[ \textbf{kinding rules} \quad \framebox{$ \Delta |- F:\kappa $} \]\vspace*{-1em}
\begin{align*}
& \inference[\sc TVar]{X^\kappa \in \Delta}{\Delta |- X:\kappa} \\
& \inference[\sc TArr]{\Delta |- A:* & \Delta |- B:*}{\Delta |- A -> B:*} \\
& \inference[\sc TAll]{\Delta,X^\kappa |- B:*}{\Delta |- \forall X^\kappa.B:*} \\
& \inference[\sc TLam]{\Delta,X^\kappa |- B:*}
		      {\Delta |- \l X^\kappa.B:\kappa -> *} \\
& \inference[\sc TApp]{\Delta |- F : \kappa -> \kappa' & |- G : \kappa}
		      {\Delta |- F ~ G : \kappa'} \\
\end{align*}
\[ \textbf{typing rules} \quad \framebox{$ \Delta;\Gamma |- t : A $ } \]
\vspace*{-1em}
\begin{align*}
& \inference[\sc Var]{x:A \in \Gamma}{\Delta;\Gamma |- x:A} \\
& \inference[\sc Abs]{\Delta |- A:* & \Delta;\Gamma,x:A |- t : B}
		     {\Delta;\Gamma |- \l x   .t : A -> B} \\
& \inference[\sc App]{\Delta;\Gamma |- t : A -> B & \Delta;\Gamma |- s : A}
		     {\Delta;\Gamma |- t~s : B} \\
& \inference[\sc TyAbs]{\Delta,X^\kappa;\Gamma |- t : B}
		       {\Delta;\Gamma |- t : \forall X^\kappa.B} \\
& \inference[\sc TyApp]{\Delta;\Gamma |- t : \forall X^\kappa.B & \Delta |- G:\kappa}
		       {\Delta;\Gamma |- t : B[G/X]} \\
& \inference[\sc Conv]{\Delta;\Gamma |- t : A & \Delta |- A = A' : *}
		      {\Delta;\Gamma |- t : A'}
\end{align*}
\end{minipage}
\begin{minipage}{.46\textwidth}
	\begin{center}Curry-Curry-style\end{center}
\def\baselinestretch{0}
\small
\begin{align*}
\textbf{term syntax} \\
t,s ::= &~ x           \\
      | &~ \l x    . t \\
      | &~ t ~ s       \\
\textbf{type syntax} \\
F,G,A,B ::= &~ X                  \\
          | &~ A -> B             \\   
          | &~ \forall X.B \\
          | &~ \l X.B      \\
          | &~ F ~ G              \\
\textbf{kind syntax} \\
\kappa ::= &~ \kappa -> \kappa'  \\
         | &~ *                  \\
\end{align*}
\[ \textbf{kinding rules} \quad \framebox{$ \Delta |- F:\kappa $} \]\vspace*{-1em}
\begin{align*}
& \inference[\sc TVar]{X^\kappa \in \Delta}{\Delta |- X:\kappa} \\
& \inference[\sc TArr]{\Delta |- A:* & \Delta |- B:*}{\Delta |- A -> B:*} \\
& \inference[\sc TAll]{\Delta,X^\kappa |- B:*}{\Delta |- \forall X.B:*} \\
& \inference[\sc TLam]{\Delta,X^\kappa |- B:*}
		      {\Delta |- \l X.B:\kappa -> *} \\
& \inference[\sc TApp]{\Delta |- F : \kappa -> \kappa' & |- G : \kappa}
		      {\Delta |- F ~ G : \kappa'} \\
\end{align*}
\[ \textbf{typing rules} \quad \framebox{$ \Delta;\Gamma |- t : A $ } \]
\vspace*{-1em}
\begin{align*}
& \inference[\sc Var]{x:A \in \Gamma}{\Delta;\Gamma |- x:A} \\
& \inference[\sc Abs]{\Delta |- A:* & \Delta;\Gamma,x:A |- t : B}
		     {\Delta;\Gamma |- \l x   .t : A -> B} \\
& \inference[\sc App]{\Delta;\Gamma |- t : A -> B & \Delta;\Gamma |- s : A}
		     {\Delta;\Gamma |- t~s : B} \\
& \inference[\sc TyAbs]{\Delta,X^\kappa;\Gamma |- t : B}
		       {\Delta;\Gamma |- t : \forall X.B} \\
& \inference[\sc TyApp]{\Delta,X^\kappa;\Gamma |- t : B & \Delta |- G:\kappa}
		       {\Delta;\Gamma |- t : B[G/X]} \\
& \inference[\sc Conv]{\Delta;\Gamma |- t : A & \Delta |- A = A' : *}
		      {\Delta;\Gamma |- t : A'}
\end{align*}
\end{minipage}
~\\
\caption{System \Fw\ in Curry-Church-style and Curry-Curry-style}
\label{fig:fw2}
\end{singlespace}
\end{figure}


\paragraph{From Curry-Church-style to Curry-Curry-style}
We can also have a version of \Fw\ where type syntax is also in Curry style.
That is, the forall type ($\forall X.B$) and the lambda type ($\l X.B$)
do not have kind annotations. We call this version of \Fw, where both terms
and types are unannotated, the Curry-Curry-style \Fw.
In Figure \ref{fig:fw2}, the Curry-Curry-style \Fw\ (right) is laid out
side-by-side to the Curry-Church-style \Fw\ (left).

Since we changed the type syntax of forall types and lambda types in
the Curry-Curry-style \Fw, we need to adjust the we need to adjust the rules
involving forall types and lambda types. The rules we need to adjust are
the kinding rules \rulename{TAll} and \rulename{TLam}, and
the typing rules \rulename{TyAbs} and \rulename{TyApp}.

For the kinding rules \rulename{TAll} and \rulename{TLam}, all we need to do
is simply dropping the kind annotations appearing in the forall type and
the lambda type in each rule. In Figure \ref{fig:fw2}, you can see that
\rulename{TAll} and \rulename{TLam} in left (Curry-Church-style) and
right (Curry-Curry-style) are identical except for the kind annotations on
the forall type and the lambda type.

How should we adjust the typing rules \rulename{TyAbs} and \rulename{TyApp}
in the Curry-Curry-style \Fw? Our first attempt may be just dropping
the kind annotations (just as we did for the kinding rules) to adjust to
the changes to the type syntax as below:
\begin{align*}
& \inference[\sc TyAbs]{\Delta,X^\kappa;\Gamma |- t : B}
		       {\Delta;\Gamma |- t : \forall X.B}
& \inference[\sc TyApp]{\Delta;\Gamma |- t : \forall X.B & \Delta |- G:\kappa}
		       {\Delta;\Gamma |- t : B[G/X]}
\end{align*}
The \rulename{TyAbs} rule above is fine. However, the \rulename{TyApp} rule
above is problematic because it fails to require that $X$ must be of kind
$\kappa$ as well as $G$. To ensure that $X$ is of kind $\kappa$, we need to
adjust the \rulename{TyApp} rule as follows:
\begin{align*}
& \phantom{ \inference[\sc TyAbs]{\Delta,X^\kappa;\Gamma |- t : B}
                                 {\Delta;\Gamma |- t : \forall X.B} }
& \inference[\sc TyApp]{\Delta,X^\kappa;\Gamma |- t : B & \Delta |- G:\kappa}
			{\Delta;\Gamma |- t : B[G/X]}
\end{align*}
Note, the first premise ($\Delta,X^\kappa;\Gamma |- t : B$) of
the \rulename{TyApp} rule is exactly the same as the premise of
the \rulename{TyAbs} rule.

\paragraph{Church-Curry-style \Fw}
Although not illustrated in Figures \ref{fig:fw} and \ref{fig:fw2},
we can imagine yet another version of \Fw\ with annotated terms
and unannotated types -- namely, the Church-Curry-style \Fw.
I am not sure when one would need this style, though.

\section{The Hindley-Milner type system} \label{sec:hm}

