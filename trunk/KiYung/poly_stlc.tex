\section{The simply-typed lambda calculus}\label{sec:stlc}
\begin{figure}
\begin{singlespace}
\begin{minipage}{.46\textwidth}
        \begin{center}Church-style\end{center}
\def\baselinestretch{0}
\small
\begin{align*}
\textbf{term syntax} \\
t,s ::= &~ x           & \text{variable}    \\
      | &~ \l(x:A) . t & \text{abstraction} \\
      | &~ t ~ s       & \text{application} \\
\\
\textbf{type syntax} \\
A,B ::= &~ A -> B  & \text{arrow type} \\
      | &~ \iota   & \text{ground type}   \\
\end{align*}
\[ \textbf{typing context} \]\vspace*{-1em}
\begin{align*}\quad
\Gamma ::= &~ \cdot \\
         | &~ \Gamma, x:A \quad (x\notin \dom(\Gamma)) \\
\end{align*}
\[ \textbf{typing rules}
        \qquad \framebox{$\Gamma |- t : A$} \]
\vspace*{-1em}
\begin{align*}
& \inference[\sc Var]{x:A \in \Gamma}{\Gamma |- x:A} \\
& \inference[\sc Abs]{\Gamma,x:A |- t : B}
                     {\Gamma |- \l(x:A).t : A -> B} \\
& \inference[\sc App]{\Gamma |- t : A -> B & \Gamma |- s : A}
                     {\Gamma |- t~s : B} \\
\end{align*}
\[ \textbf{reduction rules}
        \quad \framebox{$t --> t'$} \]
\vspace*{-1em}
\begin{align*}
& \inference[\sc RedBeta]{}{(\l(x:A).t)~s --> t[s/x]} \\
& \inference[\sc RedAbs]{t --> t'}{\l(x:A).t --> \l(x:A).t'} \\
& \inference[\sc RedApp1]{t --> t'}{t~s --> t'~s} \\
& \inference[\sc RedApp2]{s --> s'}{t~s --> t~s'} \\
\end{align*}
\end{minipage}
\begin{minipage}{.46\textwidth}
        \begin{center}Curry-style\end{center}
\def\baselinestretch{0}
\small
\begin{align*}
\textbf{term syntax} \\
t,s ::= &~ x           \\
      | &~ \l x    . t \\
      | &~ t ~ s       \\
\\
\textbf{type syntax} \\
A,B ::= &~ A -> B \\
      | &~ \iota  \\
\end{align*}
\[ \textbf{typing context} \]\vspace*{-1em}
\begin{align*}\quad
\Gamma ::= &~ \cdot \\
         | &~ \Gamma, x:A \quad (x\notin \dom(\Gamma)) \\
\end{align*}
\[ \textbf{typing rules}
        \qquad \framebox{$\Gamma |- t : A$} \]
\vspace*{-1em}
\begin{align*}
& \inference[\sc Var]{x:A \in \Gamma}{\Gamma |- x:A} \\
& \inference[\sc Abs]{\Gamma,x:A |- t : B}
                     {\Gamma |- \l x   .t : A -> B} \\
& \inference[\sc App]{\Gamma |- t : A -> B & \Gamma |- s : A}
                     {\Gamma |- t~s : B} \\
\end{align*}
\[ \textbf{reduction rules}
        \quad \framebox{$t --> t'$} \]
\vspace*{-1em}
\begin{align*}
& \inference[\sc RedBeta]{}{(\l x   .t)~s --> t[s/x]} \\
& \inference[\sc RedAbs]{t --> t'}{\l x   .t --> \l x   .t'} \\
& \inference[\sc RedApp1]{t --> t'}{t~s --> t'~s} \\
& \inference[\sc RedApp2]{s --> s'}{t~s --> t~s'} \\
\end{align*}
\end{minipage}
~\\
\caption{Simply-typed lambda calculus in Church style and Curry style}
\label{fig:stlc}
\end{singlespace}
\end{figure}
\index{lambda calculus!simply-typed}
\index{simply-typed lambda calculus|see{STLC}}
\index{STLC}
We illustrate two styles of the simply-typed lambda calculus (STLC)
in Figure~\ref{fig:stlc}. The left column of the figure describes
the Church style and the right column describes the Curry style.
\index{Curry style}
\index{Church style}

A term can be either a variable, an abstraction (\aka\ lambda term), or
an application. The distinction between the two styles comes from whether
the abstraction has a type annotation in the term syntax. Abstractions in
Church style have the form $\lambda(x:A).t$ with a type annotation $A$ on
the variable $x$ bound in $t$. Abstractions in Curry style have the form
$\lambda x.t$ without any type annotation. The differences in typing rules
and reduction rules between the two styles follow from this distinction.

A type can be either an arrow type or a ground type.
The type syntax is exactly the same in both styles.
Arrow types are types for functions. For instance,
abstractions have arrow types. We need ground types as a base case for
the inductive definition of types. Otherwise, if there were no ground types,
we would not be able to populate types.\footnote{
	If we allow infinite types,
	then it is possible to populate types without ground types.
	There exist exotic lambda calculi with infinite types, but
	these are rather uncommon.}
Here, we choose to include only the simplest ground type, $\iota$, which is
also known as the void type. Note that there does not exist any closed term
of type $\iota$. It is only possible to construct terms of type $\iota$ when
we have a bound variable, whose type is either $\iota$ or an arrow type that
eventually returns $\iota$, in the typing context.

When people use the STLC to model a programming language (with simple types),
they usually provide a richer set of ground types than just the void type alone
(\eg, unit, boolean, natural numbers). In such versions of the STLC, 
one must extend the term syntax by providing normal terms (or, constants) for
those ground types (\eg, \textsf{true} and \textsf{false} for booleans) and
eliminators (\eg, if-then-else expression for booleans) that can examine
the normal terms. Later on, we shall see that polymorphic type systems
such as System \F\ (\S\ref{sec:f}) and System \Fw\ (\S\ref{sec:fw}) are
expressive enough to encode those ground types without introducing them
as primitive constructs of the calculi. Having just the void type as
a ground type is enough to motivate polymorphic type systems, without
complicating the term syntax of the STLC.

Typing rules are the rules to derive (or prove) typing judgments.
A typing judgment $\Gamma |- t : A$ means that the term $t$ has type $A$
under the typing context $\Gamma$. We say $t$ is well-typed
(or, $t$ is a well-typed term) under $\Gamma$ when we can derive (or prove)
a typing judgment $\Gamma |- t : A$ for some $A$ according to the typing rules.
There are just three typing rules --
one typing rule for each item of the term syntax.
Therefore, the typing rules of the STLC are syntax directed in both styles.
That is, there is exactly one rule to choose for the typing derivation
by examining the shape of the term.
 
The reduction rules in Figure \ref{fig:stlc} describes $\beta$-reduction
for the STLC. The \rulename{RedBeta} rule describes the key concept
of $\beta$-reduction, the $\beta$-redex. A $\beta$-redex is an application
of an abstraction to another term. The other three rules describe the idea
that a redex may appear in subterms even though the term itself is not a redex.
The reduction rules of the STLC are virtually the same as the reduction rules
of the untyped lambda calculus. Note that reduction rules
are not deterministic. There is no preferred order when there is more than one
redex in a term. For instance, when there are redexes in both $t$ and $s$
in the application $(t\;\;s)$, one may apply either of the two rules
\rulename{RedApp1} and \rulename{RedApp2}.

\paragraph{}
We first discuss two important properties of the STLC
(subject reduction and strong normalization) which hold
in both Curry style and Church style (\S\ref{sec:stlc:srsn}).
Then, we motivate the discussion of polymorphic type systems
by reviewing the limitations of the STLC (\S\ref{sec:stlc:topoly}).

\subsection{Strong normalization}\label{sec:stlc:srsn}
\index{strong normalization}
We discuss two important properties of the STLC, which hold in both
the Church style and the Curry style -- \emph{subject reduction} (\aka\
type preservation) and \emph{strong normalization}.
\index{subject reduction}
\index{type preservation}
Since we focus on strong normalization, we will be rather brief on
the proof of subject reduction and elaborate in more detail on
the proof of strong normalization.
 
\paragraph{Subject reduction}\hspace{-1em}
states that reduction preserves types.
\begin{theorem}[subject reduction]
        $\inference{\Gamma |- t : A  & t --> t'}{\Gamma |- t' : A}$
\end{theorem}
That is, when a well-typed term takes a reduction step, then the reduced term
has the same type as the well-typed term. We can prove subject reduction
by induction on the derivation of the reduction rules.
The only interesting case is the \rulename{RedBeta} rule. Proving all
the other rules is simply done by applying the induction hypothesis.
Proving the \rulename{RedBeta} rule amounts to proving the substitution lemma:
\begin{lemma}[substitution]
$ \inference{\Gamma,x:A |- t : B  & \Gamma |- s : A}{\Gamma |- t[s/x] : B} $
\end{lemma}
Proof of the substitution lemma is a straightforward induction on
the derivation of the typing judgement.

As an aside, when people use the STLC to model a programming language,
they usually consider another property called \emph{progress},
which states that well-typed terms are either {\em values} or
can take an evaluation step. 
Values are terms that meet certain syntactic criteria, \ie, those terms
that are meant to represent ``final answers", \ie, terms that
are done evaluating. We do not further discuss progress in this dissertation.

\index{evaluation}
\index{normalization}
\index{progress}
\index{type safety}
An {\em evaluation} is a reduction strategy
(\ie, a certain subset of the reduction relation which computes a value --
hence the name {\em evaluation}), which is often deterministic.
In such a setting, type safety is usually defined to be subject reduction
together with progress -- all well-typed terms are either fully evaluated
(\ie, they are values), or they can take a step to another well-typed term.
However, in a calculus considering reductions of terms to normal forms,
rather than evaluations to values, type safety is just subject reduction
since normal terms are irreducible by definition.

\paragraph{Strong normalization.} 
\index{strong normalization!STLC}
When we consider terms of the STLC as proofs in a propositional logic,
using the Curry-Howard correspondence, strong normalization
is another important property of the STLC. Strong normalization states
that every well-typed term reduces to a normal form, no matter what
reduction strategy is followed.

To prove strong normalization of the STLC, we use the following proof strategy.
We first define the set of strongly normalizing terms, which may or may not be
well-typed, and show that all well-typed terms belong to this set.
For each type, we define a distinct set of terms called the interpretation
of that type. We show that the interpretation of every type
belongs to the set of normalizing terms.

The discussion below on strong normalization uses the
Curry-style term syntax, but this proof strategy also works well
for the Church-style STLC\footnote{This proof strategy generalizes well
        to more complicated systems such as System~\F, System~\Fw, and
        even for dependently typed calculi such as
        the Calculus of Constructions\cite{Geuvers94}.}.
In fact, this strategy originates from Girard's strong normalization proof
for System \F\ using reducibility candidates \cite{Gir71}, and later rephrased
using Tait's saturated sets \cite{Tait75}. In particular, I adopt
the notation used in the paper by \citet{AbeMat04}, which includes
a strong normalization proof for an extension of \Fw\ using saturated sets.

The strong normalization proofs for System \F\ (\S\ref{sec:f}) and
System \Fw\ (\S\ref{sec:fw}) in this chapter 
are also based on this strategy using saturated sets. As the languages
increase in complexity, we gradually
increase the complexity of the interpretation of types in those systems.
\index{interpretation!type}

\index{normalizing terms}
The set of strongly normalizing terms ($\SN$) can be defined
using a straight forward inductive definition:
\[
\inference{s_1,\ldots,s_n\in\SN}{x~s_1~\cdots~s_n \in \SN}
\qquad
\inference{t \in \SN}{\l x.t \in \SN}
\qquad
\inference{t' \in \SN & t --> t'}{t \in \SN}
\]
That is, variables and applications of a variable to strongly normalizing terms
are in $\SN$, abstractions are in $\SN$ when their bodies are in $\SN$,
and terms that reduce to $\SN$ are also in $\SN$. Relying on the fact that
normal order reduction (\ie, reduce the outermost leftmost redex first) always
leads to a normal form if a normal form exists, we can alter the last rule of
the inductive definition above to be more syntactic, which defines the same set
$\SN$, as follows:
\[
\inference{s_1,\ldots,s_n\in\SN}{x~s_1~\cdots~s_n \in \SN}
\quad
\inference{t \in \SN}{\l x.t \in \SN}
\quad
\inference{t[s/x]~s_1~\cdots~s_n\in \SN & s\in\SN}
        {(\lambda x.t)~s~s_1~\cdots~s_n \in \SN}
\]

\index{weak head expansion}
A set $\mathcal{A}$ is saturated when it is closed under adding
strongly normalizing neutral terms, and when it is closed under strongly normalizing weak head expansion:
\[
\inference{s_1,\ldots,s_n\in\SN}{x~s_1~\cdots~s_n \in \mathcal{A}}
\qquad\qquad\qquad\qquad\quad
\inference{t[s/x]~s_1~\cdots~s_n\in \mathcal{A} & s\in\SN}
          {(\l x.t)~s~s_1~\cdots~s_n \in \mathcal{A}}
\]
\index{saturated}
\index{saturated set}
\index{neutral terms}
There is a sort of cleverness in this definition of saturated.
A set is saturated when the terms it contains are either variables,
or ``come from" other terms in the saturated set using these two rules
(neutral terms and weak head expansion). We can easily observe that $\SN$ is
a saturated set by definition.  We can get the first and last part of
the inductive definition for $\SN$ when $\mathcal{A}=\SN$. We can define
an arrow operation ($->$), which given two saturated sets, defines
a third saturated set as follows:
\[ \mathcal{A} -> \mathcal{B} =
  \{ t \in \SN \mid t~s \in \mathcal{B} ~\text{for all}~ s \in \mathcal{A} \} \]
It is known that $\mathcal{A} -> \mathcal{B}$ is saturated
when both $\mathcal{A}$ and $\mathcal{B}$ are saturated \cite{Tait75}.

\begin{figure}
\begin{singlespace}
\begin{description}
\item[Interpretation of types] as saturated sets of normalizing terms:
\begin{align*}
[| \iota  |] &= \bot \qquad\qquad (\text{the minimal saturated set}) \\
[| A -> B |] &= [| A |] -> [| B |]
\end{align*}

\item[Interpretation of typing contexts] as sets of valuations ($\rho$):
\[ [| \Gamma |] =
        \{\; \rho\,\in\,\dom(\Gamma) -> \SN ~\mid~
             \rho(x) \in [|\Gamma(x)|] ~\text{for all}~x\in\dom(\Gamma) \;\}
\]

\item[Interpretation of terms] as terms themselves whose free variables are
        substituted according to the given valuation ($\rho$):
\begin{align*}
[|x|]_\rho      &= \rho(x) \\
[|\l x.t|]_\rho &= \l x.[|t|]_\rho  \qquad (x\notin\dom(\rho)) \\
[|t~s|]_\rho    &= [|t|]_\rho~[|s|]_\rho
\end{align*}
\end{description}
\caption[Interpretation of the STLC for proving strong normalization]
        {Interpretation of types, typing contexts, and terms of the STLC
         for the proof of strong normalization}
\label{fig:interpSTLC}
\end{singlespace}
\vspace*{.5em}\hrule
\end{figure}

\index{interpretation!STLC}
\index{saturated subset}
We interpret types as saturated subsets of $\SN$ (\ie, subsets of $\SN$ that
are saturated) as in Figure \ref{fig:interpSTLC}. We interpret the void type as
the minimal saturated set ($\bot$), which is saturated from the empty set.
We choose the symbol $\bot$ since saturated sets form a complete lattice
under the subset relation as the partial order. We may denote $\SN$ as $\top$
since it is the maximal element of the lattice. Note that $\bot$,
or $[|\iota|]$, does not include any abstraction ($\lambda$-terms)
since $\iota$ is not the type of a function. Arrow types ($A -> B$) are
interpreted as the saturated-set arrow over the interpretations of
the domain type and the range type ($[|A|] -> [|B|]$).

We interpret a typing context ($\Gamma$) as a set of valuations ($\rho$).
For every variable binding in the typing context ($x:A \in \Gamma$),
a valuation should map the variable ($x$) to a term that belongs to
the interpretation of its desired type ($[|A|]$). That is,
if $x : A \in \Gamma$ then any $\rho \in [|\Gamma|]$ should
satisfy the proposition that $\rho(x) \in [|A|]$.

\index{strong normalization!STLC}
The proof of strong normalization amounts to proving the following theorem:
\begin{theorem}[soundness of typing]
$ \inference{\Gamma|- t:A & \rho \in [|\Gamma|]}{[|t|]_\rho \in [|A|]} $
\end{theorem}
\begin{proof}
We prove this by induction on the typing derivation ($\Gamma|- t:A$).
\paragraph{}
For variables, it is trivial to show that
$ \inference{\Gamma |- x:A & \rho \in [|\Gamma|]}{[|x|]_\rho \in [|A|]} $.

Due to the \rulename{Var} rule, we know that $x:A \in \Gamma$.
So, $[|x|]_\rho =\rho(x)\in[|\Gamma(x)|] = [|A|]$.

\paragraph{}
For abstractions, we need to show that
$ \inference{\Gamma |- \l x.t : A -> B & \rho \in [|\Gamma|]}
             {[|\l x.t|]_\rho \in [|A -> B|]} $.

Since $[|\l x.t|]_\rho = \l x.[|t|]_\rho$ and
$[|A -> B|] = \{ t\in \SN \mid t~s\in[|B|]~\text{for all}~s\in[|A|] \}$,
what we need to show is equivalent to the following:
\[ \inference{\Gamma |- \l x.t : A -> B & \rho \in [|\Gamma|]}
             {\l x.[|t|]_\rho \in
                \{ t\in \SN \mid t~s\in[|B|] ~\text{for all}~ s\in[|A|] \} }
\]
By induction, we know that:
$ \inference{\Gamma,x:A |- t : B & \rho' \in [|\Gamma,x:A|]}
             {[|t|]_{\rho'} \in [|B|]} $.

Since this holds for all $\rho' \in [|\Gamma,x:A|]$, it also holds
for a particular $\rho'$, where $\rho' = \rho[x \mapsto s]$ for any $s \in [|A|]$.
So, $[|t|]_{\rho[x\mapsto s]} = ([|t|]_\rho)[s/x]\in[|B|]$ for any $s\in[|A|]$.
Since saturated sets are closed under normalizing weak head expansion,
$(\l x.[|t|]_\rho)~s \in[|B|]$ for any $s\in[|A|]$.
Therefore, $\l x.[|t|]_\rho$ is obviously in the set,
which we wanted it to be in, \ie,
$\l x.[|t|]_\rho\in\{t\in \SN \mid t~s\in[|B|] ~\text{for all}~ s\in[|A|]\}$.

\paragraph{}
For applications, we need to show that
$ \inference{\Gamma |- t~s : B & \rho\in[|\Gamma|]}{[|t~s|]_\rho \in [|B|]} $.

By induction we know that
$
\inference{\Gamma |- t : A -> B & \rho\in[|\Gamma|]}{[|t|]_\rho \in [|A -> B|]}
\qquad
\inference{\Gamma |- s : A & \rho\in[|\Gamma|]}{[|s|]_\rho \in [|A|]}
$.

Then, it is straightforward to see that $[|t~s|]_\rho\in[|B|]$
by definition of $[|A -> B|]$.\\
\end{proof}

\begin{corollary}[strong normalization]
        $\inference{\Gamma |- t : A}{t \in \SN}$
\end{corollary}
Once we have proved the soundness of typing with respect to interpretation,
it is easy to see that the STLC is strongly normalizing, even for open terms
(\ie, terms with free variables), by giving a trivial interpretation
such that $\rho(x) = x$ for all $x\in\dom(\Gamma)$. Note that
$[|t|]_\rho = t \in [|A|] \subset \SN$ under the trivial interpretation.


\begin{comment}
\subsection{Characteristics of the Church-style STLC}\label{sec:stlc:church}

This style of proof of strong normalization extends naturally
to the Church-style STLC as well, but we do not discuss the details
here. But, there are some interesting properties that hold in Church style,
but not in Curry style, which are worth discussing. Two of them are:
\begin{itemize}
\item \emph{Uniqueness of typing}
$\qquad\inference{\Gamma |- t : A & \Gamma |- t : A'}{A = A'} $

\item \emph{Type equivalence between well-typed $\beta$-equivalent terms}
\begin{align*}
& \inference{t =_{\beta} t' & \Gamma |- t : A & \Gamma |- t' : A'}{A = A'} \\
& \text{where $=_{\beta}$ is the reflexive symmetric transitive closure of $-->$.}
\end{align*}
\end{itemize}

In Church style, the variable ($x$) in an abstraction
($\l(x:A).t$) has a type annotation ($A$). Intuitively, we may think of
the abstraction ($\l(x:A).t$) as a function that expects an argument of
the type ($A$) specified by the type annotation.



\emph{Uniqueness of typing}, described in the first item above,
holds in the Church-style STLC.  More specifically, given
a well-formed typing context $\Gamma$ and a term $t$ as input,
if the term is well-typed, that is, $\Gamma |- t : A$ for some type $A$,
then that $A$ is the unique such type. We can prove this by induction on
the derivation of the typing judgment.
For variables, it trivially holds since the variables appearing in
the typing contexts are unique.
For abstractions, we use induction on the derivation.
To use the induction hypothesis we should make sure that
the typing context ($\Gamma,x:A$) and the term $t$ of the premise
is uniquely determined. It is easy to see that they are uniquely determined
since all the pieces appearing in the input (\ie, the typing context and
the term) of the premise (in particular, $\Gamma$, $A$, and $t$) are
part of the input (in particular the term $\l(x:A).t$) of the conclusion.
Therefore, by induction hypothesis, $B$ is uniquely determined, and,
as a consequence, $A -> B$ is uniquely determined.
For applications, it is easy to show by induction for each of the premises.
This proof describes the essence of the type reconstruction algorithm for
the Church-style STLC.


Proving \emph{type equivalence between well-typed $\beta$-equivalent terms},
described in the second item above, amounts to proving \emph{type equivalence
between terms before and after well-typed $\beta$-reducion
(or, $\beta$-expansion)},
described below:
\begin{align}
\inference{t --> t' & \Gamma |- t : A & \Gamma |- t' : A'}{A = A'}
        \label{eqn:welltypedarrow}
\end{align}
That is, when a well-typed term ($t$) reduces to
another well-typed term ($t'$) in single step ($t --> t'$),
the types of those two terms are identical ($A=A'$).
Since the claim above is symmetric, we can also say: when a well-typed term
($t'$) expands to another well-typed term ($t$) in single step ($t' <-- t$),
the types of those two terms are identical ($A'=A$).
Let us break down the claim (\ref{eqn:welltypedarrow}) above into two parts:
\[
\inference{t --> t' & \Gamma |- t : A}
          {\Gamma |- t' : A' ~~\text{implies}\quad A = A'} \qquad
        \begin{smallmatrix}
                \text{type equivalence between terms} \\
                \text{before and after well-typed $\beta$-reduction}
        \end{smallmatrix}
\]
\[
\inference{t --> t' & \Gamma |- t' : A'}
          {\Gamma |- t : A \quad\text{implies}\quad A = A'} \qquad
        \begin{smallmatrix}
                \text{type equivalence between terms} \\
                \text{before and after well-typed $\beta$-expansion}
        \end{smallmatrix}
\]~\vspace*{-3em}\\

We know that the former (type equivalence between terms before and after
well-typed $\beta$-reduction) holds because we already know
a stronger property, subject reduction, which we discussed earlier
(repeated below) that it is one of the properties that commonly hold
in both the Church style and the Curry style.
\begin{align*}
\inference{\Gamma |- t : A  & t --> t'}{\Gamma |- t' : A}
 &\qquad \text{subject reduction, or type preservation}
\end{align*}

\begin{figure}
\begin{singlespace}
\begin{tabular}{lp{7cm}}
$t <-- (\l(x:\iota).x)~t$ \qquad ($t$ is closed)&
relying on the property of the void type
that $\iota$ is not inhabited by any closed term
\\ ~ \\
$t <-- (\l(x:\iota -> \iota).t)~(\l(x:\iota).x~x)$ &
without relying on the property of $\iota$ : \par
apply a constant function to an already ill-typed term (self application)
\end{tabular}
\end{singlespace}
\caption{Examples of $\beta$-expansions to ill-typed terms}
\label{ill-typed_expand}
\hrule
\end{figure}

So, what is interesting about the Church style STLC, in contrast to
the Curry-style STLC, is the latter part of the claim that terms before and
after well-typed $\beta$-expansion always have the same type. Unlike
the former part of the claim on $\beta$-reduction, where a reduced term
is always well-typed (corollary of subject reduction), we really need
the condition that expanded term is also well-typed ($\Gamma |- t : A$).
We need this premise, because a well-typed term can be expanded to a ill-typed term.
In fact, we can always expand any well-typed term to an ill-typed term
(see Figure~\ref{ill-typed_expand}).

The proof of the claim (\ref{eqn:welltypedarrow}) is very simple.
Recall that we want to show that $A = A'$,
assuming $t --> t'$, $\Gamma |- t : A$ and $\Gamma |- t' : A'$.
We can prove it by using \emph{subject reduction} and
\emph{uniqueness of typing} as follows:
\[ \inference[(\emph{uniqueness of typing})]
        { \inference[(\emph{subject reduction})\!\!\!]
                {t --> t' & \Gamma |- t : A}
                {\Gamma |- t' : A \phantom{a_f}} 
        & \Gamma |- t' : A' }
        { A = A' }
\]

\subsection{Characteristics of the Curry-style STLC}\label{sec:stlc:curry}
In Curry style, there is no annotation on the variable in an abstraction.
Since the variable binding in the abstraction is no longer specified to be
a specific type, \emph{uniqueness of typing} does not hold,
as it does in Church style. For instance, the identity function ($\l x.x$)
could have any type of the form $A -> A$, such as:
\begin{quote}\vspace*{-1em}
\begin{singlespace}
$\Gamma |- \l x.x : \iota -> \iota$ \\
$\Gamma |- \l x.x : (\iota -> \iota) -> (\iota -> \iota)$ \\
$\Gamma |- \l x.x : (\iota -> (\iota -> \iota)) -> (\iota -> (\iota -> \iota))$ \\
$\Gamma |- \l x.x : ((\iota -> \iota) -> \iota) -> ((\iota -> \iota) -> \iota)$ \\
$\Gamma |- \l x.x : ((\iota -> \iota) -> (\iota -> \iota)) -> ((\iota -> \iota) -> (\iota -> \iota))$ \\
$~~~~ \vdots $
\end{singlespace}
\end{quote}
In Curry style, we read the typing judgment $\Gamma |- t : A$,  as
\begin{quote}
$t$ \emph{can have type} $A$, under the typing context $\Gamma$.
\end{quote}
In Church style, we read the typing judgment as
\begin{quote}
$t$ \emph{has {\bf the} type} $A$, under the typing context $\Gamma$.
\end{quote}
However, we do not consider the Curry-style STLC to be
a polymorphic type system since the typing for a variable
under a well-formed context is still unique. That is,
\[ \inference{\Gamma |- x : A & \Gamma |- x : A'}{A = A'} \]
In an environment $\Gamma$, a variable is assigned a unique {\em concrete}
type, not a type variable that can stand for other types.


Unlike Church style, type equivalence between a term and its $\beta$-reduct, or between
a term and one of its $\beta$-expansions, does not hold in Curry style.
This is expected since uniqueness of typing does not hold in Curry style.
Even when a well-typed term reduces to another well-typed term, it may be possible
to assign different types to each\footnote{Subject reduction still holds for
the Curry-style STLC since we can choose to give the same type to
the reduced term as the type before the reduction.}.

For example, consider the reduction $(\l x'.x')(\l x.x) --> \l x.x$.
We can give different types to the term before the reduction and the
term after the reduction. For example,
\[\setpremisesend{.1em} 
\inference[\sc App]
 {
   \inference[\sc Lam]
     { \inference[\sc Var]
         {x' : \iota -> \iota ~\in~ \cdot, x' : \iota -> \iota}
         {\cdot,x' : \iota -> \iota |- x': \iota -> \iota}
     }
     {\cdot |- \l x'.x' : (\iota -> \iota) -> (\iota -> \iota)}
 &
   \inference[\sc Lam]
     {\inference[\sc Var]{x:\iota ~\in~ \cdot,x:\iota}
                         {\cdot,x:\iota |- x:\iota} \phantom{x'}}
     {\cdot |- \l x.x : \iota -> \iota \phantom{(x')}}
 }
 {\cdot |- (\l x'.x')(\l x.x) : \iota -> \iota}
\]
\\
\[\qquad\qquad\qquad\quad
\inference[\sc Lam]
  {\inference[\sc Var]{x:\iota->\iota ~\in~ \cdot,x:\iota->\iota}
                      {\cdot,x:\iota->\iota |- x:\iota->\iota} }
  {\cdot |- \l x.x : (\iota -> \iota) -> (\iota -> \iota)}
\]

%% does the set of possible types get larger after reduction even when
%% the only ground type is the void type? I am not sure. So, I won't
%% elaborate on this.

%%% \paragraph{}
%%% \KYA{TODO there should be a story on type inference and principal types
%%%     to motivate the HM type system}
%%%%%%%%%%% maybe not needed
\end{comment}


\subsection{Motivations for polymorphic type systems}\label{sec:stlc:topoly}
A limitation of the STLC is that a variable ($x$) can be given only one
type binding ($x:A$) in a given context ($\Gamma$). Thus a variable
term can have only one type.

It is possible to give many typings for terms
other than variables in Curry style (\eg, the abstraction $\l x .x$ 
in the previous subsection), a type for variable ($x$) is uniquely determined
once the context ($\Gamma$) is determined. This becomes inconvenient
when we want to abstract over functions that can be given multiple types,
such as the identity function $(\l x.x)$. That is, when we have a variable
$x_\textsf{id}$ that stands for $(\l x.x)$ and have a context $\Gamma$
such that $x_\textsf{id}:A -> A \in \Gamma$ for some $A$, we cannot apply
this $x_\textsf{id}$ to arguments of differing types within the same context
$\Gamma$. Due to this limitation, most typed functional languages are based on
a polymorphic lambda calculus, which has richer notion of types than the STLC.
Polymorphic lambda calculi supports polymorphic types, such as
$\forall X.X -> X$ for the type of the identity function, which captures
the idea that the type variable $X$ can be instantiated to any type you need,
in each occurrence of the identity function ($x_\textsf{id}$).
We will introduce several well-known polymorphic lambda calculi and
discuss their properties in the following subsections.

