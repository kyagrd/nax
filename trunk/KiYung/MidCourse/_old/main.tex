\documentclass[a4paper,12pt]{book}
\usepackage{fullpage}
\usepackage{amssymb}
\usepackage[fleqn]{amsmath}
\usepackage{semantic}
\usepackage[backref]{hyperref}
\usepackage[sort&compress,square,comma,numbers,longnamesfirst]{natbib}
\usepackage[all]{xy}
\usepackage{xcolor}
\usepackage{url}

\title{Normalizing calculi with recursive types}
\author{Ki Yung Ahn}

\newcommand{\Fw}[0]{\ensuremath{F_\omega}}
\newcommand{\ie}[0]{{i.e.},\;}
\newcommand{\eg}[0]{{e.g.},\;}
\newcommand{\calT}[0]{\ensuremath{\mathcal{T}}}
\newcommand{\calC}[0]{\ensuremath{\mathcal{C}}}
\newcommand{\calB}[0]{\ensuremath{\mathcal{B}}}
\newcommand{\inl}[0]{\ensuremath{\mathsf{inl}}}
\newcommand{\inr}[0]{\ensuremath{\mathsf{inr}}}
\newcommand{\case}[0]{\ensuremath{\mathsf{case}}}
\newcommand{\fst}[0]{\ensuremath{\mathsf{fst}}}
\newcommand{\snd}[0]{\ensuremath{\mathsf{snd}}}
\newcommand{\unit}[0]{\ensuremath{\mathsf{unit}}}
\newcommand{\Unit}[0]{\ensuremath{\mathsf{Unit}}}
\newcommand{\Void}[0]{\ensuremath{\mathsf{Void}}}
\newcommand{\miter}[0]{\ensuremath{\mathsf{mit}}}
\newcommand{\mrec}[0]{\ensuremath{\mathsf{mrec}}}
\newcommand{\mcvit}[0]{\ensuremath{\mathsf{mcvit}}}
\newcommand{\mcvrec}[0]{\ensuremath{\mathsf{mcvrec}}}
\newcommand{\msfit}[0]{\ensuremath{\mathsf{msfit}}}
\newcommand{\msfcvit}[0]{\ensuremath{\mathsf{msfcvit}}}
\newcommand{\msfrec}[0]{\ensuremath{\mathsf{msfrec}}}
\newcommand{\msfcvrec}[0]{\ensuremath{\mathsf{msfcvrec}}}
\newcommand{\In}[0]{\ensuremath{\mathsf{in}}}
\newcommand{\out}[0]{\ensuremath{\mathsf{out}}}
\newcommand{\Inv}[0]{\ensuremath{\mathsf{inv}}}

\newcommand{\Prog}[0]{\ensuremath{\mathsf{Prog}}}
\newcommand{\Log}[0]{\ensuremath{\mathsf{Log}}}

\begin{document}

\maketitle
\tableofcontents

\part{Motivation}
\chapter{ABC}
Nax to allow polymorhpism, recursive types and normalization

The motivation behind the Nax language design arose in the context of
the Trellys project. The goal of the Trellys project is to design and
implement a dependently typed language, named Trellys, that is both
a functional programming language and a logical reasoning system.
More specifically, Trellys consists of two sublanguages:
a programmatic sublanguage (\Prog) and a logical sublanguge (\Log).
Or, we can view Trellys having two modes of type checking
($|-_{\!\!\!_\Prog}$ and $|-_{\!\!\!_\Log}$).
A Trellys expression may be considered as a program when it type checks
programatically ($\Gamma |-_{\!\!\!_\Prog} e : t$).
A Trellys expression may be considered as a proof when it type checks
logically ($\Gamma |-_{\!\!\!_\Log} e : t$).
We want Trellys to be a unified system, where programs in \Prog\ and
proofs in \Log\ can refer to each other. That is, in Trellys, we should
be able to construct logical proofs reasoning about properties of programs
(\emph{freedom of speech}), and, conversely, we should be able to write
programs computing over data containing proofs (\emph{certified programming}).

What are the minimal requirements for \Prog\ and \Log\, when designing
a language like Trellys?
For \Prog\, we have a pretty good idea from the beginning. Since \Prog\ is
intended to be a general purpose typed functional programming language,
it should support all the features commonly found in functional programming
languages like ML or Haskell (e.g., first class functions, recursive types,
polymorhpism). In addtion, we want \Prog\ to have dependent types, or at least
indexed types since we want to support \emph{certified programming}.
For \Log\, we have two minimal requirements.
Firstly, \Log\ must be normalizing.
Secondly, \Log\ must support all the types that \Prog\ supports.

\part{Recursive Types and Normalization}
\chapter{Recursive Types}
\section{iso-recursive types}
\section{equi-recursive types}
\section{polarity}
\section{}
\chapter{Normalization}
\chapter{TODO}
\chapter{Dependent types and Induction}

\part{Mendler-style Iteration and Recursion Combinators}

\part{Nax}

\section{$F_2$ (System F) with some extensions}

\paragraph{Syntax}
\begin{align*}
Dec  ::=&~ F\,\overline{X}\,X.\{\,\overline{C\,\overline{T}}\,\} \\
Decs ::=&~ \cdot \mid Dec;Decs
\\ ~ \\
T ::=&~ F\,\overline{T}\,T
 \mid   T -> T
 \mid   X 
 \mid   \forall X . T
 \mid   \mu(F\,\overline{T})
\\ ~ \\
M ::=&~ x \mid C
 \mid   \case\,M\,\{\,\overline{C\,\overline{x}.M}\,\}
 \mid   \lambda x . M
 \mid   M M
%% \mid   \Lambda X. M
%% \mid   M [T]
\\
 \mid&~ \In\,M
 \mid   \miter\,M
 \mid   \mrec\,M
 \mid   \mcvit\,M
 \mid   \mcvrec\,M
 \mid   \msfit\,M \\
 \mid&~ \out
 \mid   \Inv ~~~\text{-\,\!- these are transient objects cannot apppear in source code }
\\ ~ \\
Program ::=&~ Decs; M
\end{align*}

\paragraph{Typing rules}
\[ \inference[dec]
      {\Gamma, F:\star^n->\star->\star,
       \overline{C:\forall\overline{X}^n.\forall X.\overline{T}->F\,\overline{X}^n\,X}
       |- Decs \mid M : T}
      {\Gamma
       |- F\,\overline{X}^n\,X.\{\,\overline{C\,\overline{T}}\,\};Decs \mid M : T}
\]

\[ \inference[dec0]{\Gamma |- M : T}{\Gamma |- \cdot \mid M : T}
   ~~~~~~~~
   \inference[var]{(x:T) \in \Gamma}{\Gamma |- x:T}
   ~~~~~~~~
   \inference[ctor]{(C:T) \in \Gamma}{\Gamma |- C:T}
\]

\[ \inference[case]
      { \Gamma |- M : F\,\overline{T'} & Ctors(F) = \{ \overline{C_k} \}
      & \overline{\Gamma |- C_k : \overline{T_i} -> F\,\overline{T'}}
      & \overline{ \Gamma, \overline{x_i:T_i} |- M_k : T }
      }{\Gamma |- \case\,M\,\{\,\overline{C_k\,\overline{x_i}.M_k}\,\} : T}
\]

\[ \inference[abs]{\Gamma |- x |- M : T}
                  {\Gamma |- \lambda x .M : T' -> T}
   ~~~~~~~~
   \inference[app]{\Gamma |- M : T' -> T & \Gamma |- M' : T'}
                  {\Gamma |- M\,M' : T}
\]

\[ \inference[tabs]{\Gamma, X:\star |- M : T}
                   {\Gamma |- M : \forall X.T}
   ~~~~~~~~
   \inference[tapp]{\Gamma |- M : \forall X.T}
                   {\Gamma |- M : T[T'/X]}
\]

\[ \inference[in]{\Gamma |- M : F\,\overline{T}\,(\mu(F\,\overline{T}))}
                 {\Gamma |- \In\,M : \mu(F\,\overline{T})}
   ~~~~~~~~
   \inference[mit]{\Gamma, X:\star |- M : (X->T') -> F\,\overline{T}\,X -> T'}
                    {\Gamma |- \miter\,M : \mu(F\,\overline{T}) -> T'}
\]

\[
   \inference[mrec]{\Gamma, X:\star |- M : (X->\mu(F\,\overline{T})) -> (X->T') -> F\,\overline{T}\,X -> T'}
                   {\Gamma |- \mrec\,M : \mu(F\,\overline{T}) -> T'}
\]

\[ \inference[mcvit]
     { \Gamma, X:\star,
       |- M : (X->F\,\overline{T}\,X) -> (X -> T') -> F\,\overline{T}\,X -> T'
     \\ F \text{ is positive} }
     {\Gamma |- \mcvit\,M : \mu(F\,\overline{T}) -> T'}
\]

\[ \inference[mcvrec]
     { \Gamma, X:\star,
       |- M : (X->F\,\overline{T}\,X) -> (X->F\,\overline{T}\,X) -> (X -> T') -> F\,\overline{T}\,X -> T'
     \\ F \text{ is positive} }
     {\Gamma |- \mcvrec\,M : \mu(F\,\overline{T}) -> T'}
\]


\[ \inference[msfit]
     {\Gamma, X:\star |- M : (T'->X) -> (X->T') -> F\,\overline{T}\,X -> T'}
     {\Gamma |- \msfit\,M : \mu(F\,\overline{T}) -> T'}
\]

\[ \inference[msfit']
     {\Gamma, X':\star->\star,
      |- M : (T'->X'\,T') -> (X'\,T'->T') -> F\,\overline{T}\,(X'\,T') -> T'}
     {\Gamma |- \msfit\,M : \mu(F\,\overline{T}) -> T'}
\]


The typing rule msfit' will be safe (\ie normalizing) since it has an easy
embedding into \Fw. However, the type signature for $M$ revealed to the user
is a bit ugly, and we need to introduce a type variable $X'$ of kind
$\star->\star$, which is a bit outside $F_2$.
So, it would be nice to know whether the simplified rule msfit is enough
to guarantee normalization for \msfit\ as a language primitive.

We know that System F + msfit' is normalizing by embedding into Fw.
We want to show that System F + msfit is normalizing by embedding it
into System F + msfit'.

A semi formal proof illustrated in Haskell
\begin{verbatim}
type T' = ()

m' :: (T' -> x) -> (x -> T') -> f x -> T'
m' = undefined

m :: (T' -> x' T') -> (x' T' -> T') -> f (x' T') -> T'
m = m'
\end{verbatim}
Every $M$ of type that appears in msfit can be typed with the type of msfit'
as well!

There can be rules for \msfcvit\, \msfrec\ and \msfcvrec\ as well
(haven't written them down). \msfcvit\ will have the same requirement for
positivity (or, monotonicity) on the flat datatype $F$ as \mcvit does.
I am not yet sure how to formulate the recursion opererators \msfrec\ and
\msfcvrec\ since I don't have clear thoughts on the $\mu$ object injected
inside $M$ by the casting operator.



We need some basic side conditions for well-formed (or, well-kinded) types,
but omiterted here. Those conditions would be needed for $F$ things and $C$
things are well-defined and well-kinded. The will-kindedness condition is
pretty straightforward since we are dealing with extension of $F_2$:
(1) every type $T$,
    type variable $X$, and
    the type application $F\,\overline{T}\,X$
    is of kind $\star$
    (\ie except for $X'$ in the rule msfit,
          the identifier $F$ for flat non-recursive structure definition, and
          its partial application $F\,\overline{T}$), and
(2) $F$ does not appear in $\overline{T}$ in the rule dec.

\paragraph{Reduction rules}
\begin{align*}
(\lambda x.M) M' -->&~ M[M'/x] \\
%%% (\Lambda X:\star.M)[T] -->&~ M[T/X] \\
%%% C[T] -->&~ C\\
\miter\,M\,(\In\,M') -->&~ M\,(\miter\,M)\,M' \\
\mrec\,M\,(\In\,M') -->&~ M\,(\lambda x.x)\,(\mrec\,M)\,M' \\
\mcvit\,M\,(\In\,M') -->&~ M\,\out\,(\mcvit\,M)\,M' \\
\mcvrec\,M\,(\In\,M') -->&~ M\,(\lambda x.x)\,\out\,(\mcvrec\,M)\,M' \\
\out\,(\In\,M') -->&~ M' \\
\msfit\,M\,(\In\,M') -->&~ M\,\Inv\,(\msfit\,M)\,M' \\
\msfit\,M\,(\Inv\,M') -->&~ M'
\end{align*}
\[ \inference{M --> M'}{E[M] --> E[M']} \]
\[ E ::= ... \]

\newpage
\paragraph{Metatheory}

type safety should be easy to show



\section{\Fw}
TODO allow quantification over type constructors

\paragraph{Syntax}
\begin{align*}
Dec  ::=&~ F\,\overline{(X:K)}\,(X:K) : K.\{\,\overline{C\,\overline{T}}\,\} \\
Decs ::=&~ \cdot \mid Dec;Decs
\\ ~ \\
K ::=&~ \star \mid K \to K
\\ ~ \\
T ::=&~ F \mid T\,T
 \mid   T -> T
 \mid   X 
 \mid   \forall(X:K) . T
 \mid   \mu(F\overline{T})
\\ ~ \\
M ::=&~ x \mid C
 \mid   \case\,M\,\{\,\overline{C\,\overline{x}.M}\,\}
 \mid   \lambda(x:T). M
 \mid   M M
 \mid   \Lambda(X:K). M
 \mid   M [T] \\
 \mid&~ \In\,M
 \mid   \miter\,M
 \mid   \mcvit\,M
 \mid   \msfit\,M
\end{align*}

\section{dependent calculus}
extending it with true value dependencies

The lift datatype???

\section{Note on Abel and Matthes' CSL'04 paper}

Just trying factorial example (a simple example for $\kappa=\star$).

\begin{verbatim}
MRec : (forall r. (r->Mu f) -> (r -> a) -> f r -> a) -> Mu f -> a

data N r = X 
type Nat = Mu N
zero      = In Z
succ x xs = In (S x xs)

factorial = MRec ( \i: x -> Nat . \fac: x -> Nat . \n: N x .
                   case n of  Z   -> succ zero
                              S m -> i(succ m) * fac m )
\end{verbatim}


\paragraph{Other notes}

Comparison of iteration and recursion in Mendler style

Iteration: only have access to abstract recursive values and answers

Recursion: additional \emph{casting operator} from an abstract recursive type
to a concrete recursive type

Course of values is about how deep the recursive call be pushed in

\bibliographystyle{plainnat}
\bibliography{main}

\end{document}
