\chapter{System \Fi}\label{ch:fi}

\newcommand{\newFi}[1]{\colorbox{grey}{\ensuremath{#1}}}

It is well known that datatypes can be embedded into polymorphic lambda
calculi by means of functional encodings,
such as the Church and Boehm-Berarducci encodings~\cite{BoehmBerarducci}.
\index{Church encoding}

\index{regular datatypes}
In System~\textsf{F}, one can embed \emph{regular datatypes},
like homogeneous lists:
\[
\begin{array}{ll}
\text{Haskell:} & \texttt{data List a = Cons a (List a) | Nil} \\
\text{System \textsf{F}:}~& 
\texttt{{List}}\:\: A\:\:\triangleq\:\:
\forall X.
(A\to X\to X) \to X \to X ~~\; \\
&
\quad~~
\texttt{Cons} \triangleq \l w.\l x.\l y.\l z.\,y\;w\,(x\;y\;z),~
\texttt{Nil} \triangleq \l y.\l z.z
\end{array}
\]
In such regular datatypes, constructors have algebraic structure that
directly translates into polymorphic operations on abstract types as
encapsulated by universal quantification over types (of kind $*$).

\index{type constructor}
In the more expressive System \Fw, where one can abstract over
type constructors of any kind, one can encode more general
\emph{type-indexed datatypes} that go beyond the regular datatypes.
For example, one can embed powerlists with heterogeneous elements
in which an element of type \texttt{a} is followed by
an element of the product type \texttt{(a,a)}:
\[
\begin{array}{ll}
\text{Haskell:} & \texttt{data Powl a = 
        PCons a (Powl(a,a))
        | 
        PNil 
} \\
& \!\!\!\!\!\!\!\!\!
  \textcolor{gray}{\small\texttt{-- PCons 1 (PCons (2,3) (PCons ((3,4),(1,2)) PNil)) :: Powl Int}}\\
\text{System \Fw:}~& \texttt{{Powl}}\:\triangleq\:
\lambda A^{*}.\forall X^{*\to*}. (A\to X(A\times A)\to X A) \to X A \to XA
\end{array}
\]
Note the non-regular occurrence (\texttt{Powl(a,a)}) in the definition of
(\texttt{Powl a}), and the use of universal quantification over
higher-order kinds ($\forall X^{*\to*}$).
The term encodings for {\small\tt PCons} and {\small\tt PNil} are exactly
the same as the term encodings for {\small\tt Cons} and {\small\tt Nil},
but have different types.

\index{datatype!term-indexed}
What about term-indexed datatypes?  What extensions to System~\Fw\ are
needed to embed term-indices as well as type-indices?  Our answer is
System~\Fi.

\index{vector}
In a functional language supporting term-indexed datatypes, we envisage
that the classic example of homogeneous length-indexed lists would be
defined along the following lines (in Nax-like syntax):\begin{singlespace}
\begin{lstlisting}[basicstyle={\ttfamily\small},language=Haskell]
   data Nat = S Nat | Z 
   data Vec : * -> Nat -> * where
     VCons : a -> Vec a {i} -> Vec a {S i}
     VNil  : Vec a {Z}
\end{lstlisting}\end{singlespace}
Here the type constructor~{\tt Vec} is defined to admit parameterisation
by both a type parameter and a term index.\footnote{Recall,
	in Chapter \ref{ch:mendler}, we classify the arguments of
	type constructors either as parameters,
	which appear uniformly in the datatype definition
	(\eg, \texttt{a} in \texttt{Vec}, or as indices,
	which vary (\eg, \texttt{i}, \texttt{S i}, \texttt{Z}).
	Type arguments are sometimes used as parameters and sometimes
	used as as indices. Term arguments, on the other hand, are
	almost always used as indices, except for some degenerate cases
	(\eg, term-indexing by a unit value).}
For instance, the type (\texttt{\small Vec\,(List\;Nat)\,\{S(S\;Z)\}}) is
a vector whose elements are lists of natural numbers. By design, our syntax
directly reflects the difference between type arguments and term arguments
by enclosing the latter in curly braces. We also make this distinction
in System~\Fi, where it is useful within the type system to guarantee
the static nature of term-indexing.

The encoding of the vector datatype in System~\Fi\ is as follows:\vspace*{-3pt}
\begin{equation*}\label{FiVecType}
\texttt{{Vec}}
\triangleq
\begin{array}[t]{l}
\lambda A^{*}.\lambda
i^{\texttt{{Nat}}}.  \forall X^{\texttt{{Nat}}\to *}.
  (\forall j^{\texttt{{Nat}}}.A\to X\{j\}\to X\{\texttt{{S}}\; j\})
  \to X\{\texttt{{Z}}\}
    \to X\{i\}
\end{array}
\end{equation*}\vskip-1ex\noindent
where $\texttt{{Nat}}$, $\mathtt Z$, and $\mathtt S$ respectively encode
the natural number type and its two constructors,  zero and successor.
Again, the term encodings for {\small\tt VCons} and {\small\tt VNil} are exactly
the same as the encodings for {\small\tt Cons} and {\small\tt Nil},
but have different types.

Without going into the details of the formalism, which are given in the
next section, one sees that such a calculus incorporating term-indexing
structure needs four additional constructs (see \Fig{Fi} for the
highlighted extended syntax).
\begin{enumerate}
\item 
  Type-indexed kinding~($A\to\kappa$), as in $(\texttt{{Nat}\ensuremath{\to}*})$
  in the example above, where the compile-time nature of term-indexing
  will be reflected in the typing rules, enforcing that $A$ be a closed
  type~(rule~$(Ri)$ in \Fig{Fi2}).

\item 
  Term-index abstraction~$\lambda i^A.F$~(as
  $\lambda i^{\texttt{{Nat}}}.\cdots$ in the example above) for constructing
  (or introducing) term-indexed kinds (rule~$(\lambda i)$ in \Fig{Fi2}).

\item 
  Term-index application~$F\{s\}$ (as $X\{{\tt Z}\}$, $X\{j\}$, and
  $X\{\texttt{S}\;j\}$ in the example above) for destructing (or
  eliminating) term-indexed kinds, where the compile-time nature of
  indexing will be reflected in the typing rules, enforcing that the index be
  statically typed (rule~$(@i)$ in \Fig{Fi2}).

\item 
  Term-index polymorphism~$\forall i^A.B$~(as
  $\forall j^{\texttt{{Nat}}}.\cdots$ in the example above)
  where the compile-time nature of polymorphic term-indexing
  will be reflected in the typing rules enforcing that the variable~$i$
  be static of closed type~$A$~(rule~$(\forall Ii)$ in \Fig{Fi2}).
\end{enumerate}

As described above, System~\Fi\ maintains a clear-cut separation between
type-indexing and term-indexing.  This adds a level of abstraction
to System~\Fw\ and yields types that in addition to parametric polymorphism
also keep track of inductive invariants using term-indices. All term-index
information can be erased, since it is only used at compile-time.  
It is possible to project any well-typed System~\Fi\ term into
a well-typed System~\Fw\ term.
For instance, the erasure of the \Fi-type~\texttt{Vec}
is the \Fw-type~\texttt{List}.  This is established in
\S\ref{sec:fi:theory} and used to deduce the strong normalization of
System~\Fi.

\paragraph{}
A conference paper \cite{AhnSheFioPit13}, modified from
the contents of this Chapter, is published in TLCA 2013.

\section{System \Fi}\label{sec:fi:fi}
System \Fi\ is a higher-order polymorphic lambda calculus 
designed to extend System~\Fw\ by the inclusion of term indices.
The syntax and rules of System~\Fi\ are described in
Figures \ref{fig:Fi}, \ref{fig:Fi2} and~\ref{fig:eqFi}. 
The extensions new to System~\Fi, which are not originally part of System~\Fw,
are highlighted by \newFi{\text{grey boxes}}.  Eliding all the grey boxes from
Figures~\ref{fig:Fi}, \ref{fig:Fi2} and~\ref{fig:eqFi}, one obtains
a version of System~\Fw\ with Curry-style terms and the typing context
separated into two parts (type-level context $\Delta$ and
term-level context $\Gamma$). 

In this section, we first discuss the rational for our design
choices~(\S\ref{ssec:rationale}) and then introduce the new constructs of
System~\Fi\ %, which are not found in System~\Fw
(\S\ref{ssec:newFi}).


\subsection{Design of System~\Fi%Rationale for the design choices
} \label{ssec:rationale}
Terms in \Fi\ are Curry style. That is, term level abstractions are unannotated
($\lambda x.t$), and type generalizations ($\forall I$) and type instantiations
($\forall E$) are implicit at term level. A Curry-style calculus generally has
an advantage over its Church-style counterpart when reasoning about properties of
reduction. For instance, the Church-Rosser property naturally holds for 
$\beta$-, $\eta$-, and $\beta\eta$-reduction in the Curry style, but
may not hold in the Church style. This is due to the presence of annotations in
abstractions \cite{Miquel01}.\footnote{The Church-Rosser property,
in its strictest sense (\ie, $\alpha$-equivalence over terms),
generally does not hold in Church-style calculi , but may hold under
	certain approximations, such as modulo ignoring the annotations
	in abstractions.}

Type constructors, on the other hand, remain Church style in \Fi. That is, type level abstractions are
annotated by kinds ($\lambda X^\kappa.F$). Choosing type constructors
to be Church style makes the kind of
a type constructor visually explicit. The choice of style for type constructors
is not as crucial as the choice of style for terms, since the syntax and
kinding rules at type level are essentially a simply typed lambda calculus.
Annotating the type level abstractions with kinds makes kinds
explicit in the type syntax. Since \Fi\ is essentially an extension of \Fw\
with a new formation rule for kinds, making kinds explicit is a pedagogical
tool to emphasize the consequences of this new formation rule.
As a notational convention, we write
$A$ and $B$, instead of $F$ and $G$, where $A$ and $B$ to are expected
to be types (\ie, nullary type constructors) of kind $*$.

In a language with term indices, terms appear in types (e.g., the length index
$(n+m)$ in the type $\textit{Vec}\;\textit{Nat}\;\{n+m\}$).
Such terms contain variables. The binding sites of these variables matter.
In \Fi, we expect such variables to be statically bound. Dynamically bound
index variables would require a dependently typed calculus, such as
the calculus of constructions. To reflect this design choice,
typing contexts are separated into type level contexts ($\Delta$) and
term level contexts ($\Gamma$). Type level (static) variables ($X$ , $i$) are
bound in $\Delta$ and term (dynamic) variables ($x$) are bound in $\Gamma$.
Type level variables are either type constructor variables ($X$) or
term variables to be used as indices ($i$). As a notational convention,
we write $i$, instead of $x$, when term variables are to be used as indices
(\ie, introduced by either index abstraction or index polymorphism).

In contrast to our design choice, System \Fw\ is most often formalized using
a single context, which binds both type variables~($X$) and term 
variables~($x$). 
In such a formalization, the free type variables in the typing of
the term variable must be bound earlier in the context. For example,
if $X$ and $Y$ appear free in the type of $f$, they must appear earlier
in the single context ($\Gamma$) as below:
\[ \Gamma = \dots,X^{\mathtt*},\dots,Y^{\mathtt*},\dots,
		(f:\forall Z^{\mathtt*}.X -> Y -> Z),\dots \]
In such a formalization, the side condition ($X\notin\Gamma$)
in the $(\forall I)$ rule of Figure \ref{fig:Fi} is not necessary,
since such a condition is already a part of the well-formedness condition
for the context (\ie, $\Gamma,X^\kappa$ is well-formed when
$X\notin\FV(\Gamma)$). Thus, for \Fw, it is only a matter of taste
whether to formalize the system using a single context or two contexts,
since they are equivalent formalizations with comparable complexity.

However, in \Fi, we separate the context into two parts to distinguish
term variables used in types (which we call index variables, or indices,
and are bound as $\Delta,i^A$) from the ordinary use of term variables
(which are bound as $\Gamma,x : A$). The expectation is that indices
should have no effect on reduction at the term level.
Although it is imaginable to formalize \Fi\ with a single typing context
and distinguish index variables from ordinary term variables using
more general concepts (\eg, capability, modality), we think that splitting
the typing context into two parts is the simplest solution.

\begin{figure}\begin{singlespace}
	\small
\paragraph{Syntax:}
\begin{align*}
\!\!\!\!\!\!\!\!&\text{Sort}
 	& \square
	\\
\!\!\!\!\!\!\!\!&\text{Term Variables}
 	& x,i
\\
\!\!\!\!\!\!\!\!&\text{Type Constructor Variables}
 	& X
\\
\!\!\!\!\!\!\!\!&\text{Kinds}
 	& \kappa		&~ ::= ~ *
				\mid \kappa -> \kappa
				\mid \newFi{A -> \kappa}
\\
\!\!\!\!\!\!\!\!&\text{Type Constructors}
	& A,B,F,G		&~ ::= ~ X
				\mid A -> B \\ &&& ~\quad
				\mid \lambda X^\kappa.F
				\mid F\,G
				\mid \forall X^\kappa . B \\ &&& ~\quad
				\mid \newFi{\lambda i^A.F
				\mid F\,\{s\}
				\mid \forall i^A . B}
\\
\!\!\!\!\!\!\!\!&\text{Terms}
	& r,s,t			&~ ::= ~ x \mid \lambda x.t \mid r\;s
\\
\!\!\!\!\!\!\!\!&\text{Typing Contexts}
	& \Delta		&~ ::= ~ \cdot
				\mid \Delta, X^\kappa
				\mid \newFi{\Delta, i^A} \\
&	& \Gamma		&~ ::= ~ \cdot
				\mid \Gamma, x : A
\end{align*}
\paragraph{Reduction:} \fbox{$t \rightsquigarrow t'$}
\[ 
   \inference{}{(\lambda x.t)\,s \rightsquigarrow t[s/x]}
 ~~~~
   \inference{t \rightsquigarrow t'}{\lambda x.t \rightsquigarrow \lambda x.t'}
 ~~~~
   \inference{r \rightsquigarrow r'}{r\;s \rightsquigarrow r'\;s}
 ~~~~
   \inference{s \rightsquigarrow s'}{r\;s \rightsquigarrow r\;s'}
\]
~\\
\end{singlespace}
\caption{Syntax and Reduction rules of \Fi}
\label{fig:Fi}
\end{figure}

\begin{figure}\begin{singlespace}\small
\paragraph{Well-formed typing contexts:}
\[ \fbox{$|- \Delta$}
 ~~~~ ~~~~
   \inference{}{|- \cdot}
 ~~~~
   \inference{|- \Delta & |- \kappa:\square}{|- \Delta,X^\kappa}
      \big( X\notin\dom(\Delta) \big)
\]
\[ \qquad~\qquad~\qquad\quad
 ~~~~ \newFi{
   \inference{|- \Delta & \cdot |- A:*}{|- \Delta,i^A}
      \big( i\notin\dom(\Delta) \big) }
\]
\[ \fbox{$\Delta |- \Gamma$}
 ~~~~
   \inference{|- \Delta}{\Delta |- \cdot}
 ~~~~
   \inference{\Delta |- \Gamma & \Delta |- A:*}{
              \Delta |- \Gamma,x:A}
      \big( x\notin\dom(\Gamma)\cup\dom(\Delta) \big)
\]
~\\
\paragraph{Sorting:} \fbox{$|- \kappa : \square$}
\[
  \inference[($A$)]{}{|- *:\square}
 ~~~~
   \inference[($R$)]{|- \kappa:\square & |- \kappa':\square}{
                     |- \kappa -> \kappa' : \square}
 ~~~~
   \newFi{
   \inference[($Ri$)]{\cdot |- A:* & |- \kappa:\square}{
                      |- A -> \kappa : \square} }
\]
\paragraph{Kinding:} \fbox{$\Delta |- F : \kappa$}
$ \qquad
   \inference[($Var$)]{X^\kappa\in\Delta & |- \Delta}{
                      \Delta |- X : \kappa}
 ~~~~
   \inference[($->$)]{\Delta |- A : * & \Delta |- B : *}{
                      \Delta |- A -> B : * }
$
\[
  \inference[($\lambda$)]{|- \kappa:\square & \Delta,X^\kappa |- F : \kappa'}{
                          \Delta |- \lambda X^\kappa.F : \kappa -> \kappa'}
 ~~~~ \quad ~~
 \newFi{
  \inference[($\lambda i$)]{\cdot |- A:* & \Delta,i^A |- F : \kappa}{
			    \Delta |- \lambda i^A.F : A->\kappa} }
\]
\[
   \inference[($@$)]{ \Delta |- F : \kappa -> \kappa'
                    & \Delta |- G : \kappa }{
                     \Delta |- F\,G : \kappa'}
 ~~~~
 \newFi{
   \inference[($@i$)]{ \Delta |- F : A -> \kappa
                     & \Delta;\cdot |- s : A }{
		      \Delta |- F\,\{s\} : \kappa} }
\]
\[
   \inference[($\forall$)]{|- \kappa:\square & \Delta, X^\kappa |- B : *}{
                           \Delta |- \forall X^\kappa . B : *}
 ~~~~ \qquad
	\newFi{
   \inference[($\forall i$)]{\cdot |- A:* & \Delta, i^A |- B : *}{
                             \Delta |- \forall i^A . B : *} }
\]
\[ \newFi{
   \inference[($Conv$)]{ \Delta |- A : \kappa
                       & \Delta |- \kappa = \kappa' : \square }{
                        \Delta |- A : \kappa'} }
\]
~\\
\paragraph{Typing:} \fbox{$\Delta;\Gamma |- t : A$}
$ \qquad
 ~~~~
 \inference[($:$)]{(x:A) \in \Gamma & \Delta |- \Gamma}{
                   \Delta;\Gamma |- x:A}
 ~~~~ \newFi{
   \inference[($:i$)]{i^A \in \Delta & \Delta |- \Gamma}{
                      \Delta;\Gamma |- i:A} }
$
\[
   \inference[($->$$I$)]{\Delta |- A:* & \Delta;\Gamma,x:A |- t : B}{
                         \Delta;\Gamma |- \lambda x.t : A -> B}
 ~~~~ ~~~~
   \inference[($->$$E$)]{\Delta;\Gamma |- r : A -> B & \Delta;\Gamma |- s : A}{
                         \Delta;\Gamma |- r\;s : B}
\]
\[ \inference[($\forall I$)]{|- \kappa:\square
	                    & \Delta, X^\kappa;\Gamma |- t : B}{
                             \Delta;\Gamma |- t : \forall X^\kappa.B}
			    (X\notin\FV(\Gamma))
 ~~~~ ~~~~
   \inference[($\forall E$)]{ \Delta;\Gamma |- t : \forall X^\kappa.B
                            & \Delta |- G:\kappa }{
                             \Delta;\Gamma |- t : B[G/X]}
\]
\[ \newFi{
   \inference[($\forall I i$)]{\cdot |- A:* & \Delta, i^A;\Gamma |- t : B}{
                               \Delta;\Gamma |- t : \forall i^A.B}
   \left(\begin{matrix}
		i\notin\FV(t),\\
		i\notin\FV(\Gamma)\end{matrix}\right)
 ~~~~
   \inference[($\forall E i$)]{ \Delta;\Gamma |- t : \forall i^A.B
                              & \Delta;\cdot |- s:A}{
                               \Delta;\Gamma |- t : B[s/i]} }
\]
\[ \inference[($=$)]{\Delta;\Gamma |- t : A & \Delta |- A = B : *}{
                     \Delta;\Gamma |- t : B}
\]
~\\
\end{singlespace}
\caption{Sorting, Kinding, and Typing rules of \Fi}
\label{fig:Fi2}
\end{figure}

\begin{figure}\begin{singlespace}\small
\paragraph{Kind equality:} \fbox{$|- \kappa=\kappa' : \square$}
$ \quad
 ~~~~
   \inference{}{|- * = *:\square} $
\[
   \inference{ |- \kappa_1 = \kappa_1' : \square
             & |- \kappa_2 = \kappa_2' : \square }{
              |- \kappa_1 -> \kappa_2 = \kappa_1' -> \kappa_2' : \square}
 ~~~~ \newFi{
   \inference{\cdot |- A=A':* & |- \kappa=\kappa':\square}{
              |- A -> \kappa = A' -> \kappa' : \square} }
\]
\[ \inference{|- \kappa=\kappa':\square}{
              |- \kappa'=\kappa:\square}
 ~~~~
   \inference{ |- \kappa =\kappa' :\square
             & |- \kappa'=\kappa'':\square}{
              |- \kappa=\kappa'':\square}
\]
~
\paragraph{Type constructor equality:} \fbox{$\Delta |- F = F' : \kappa$}
\[
   \inference{\Delta,X^\kappa |- F:\kappa' & \Delta |- G:\kappa}{
              \Delta |- (\lambda X^\kappa.F)\,G = F[G/X]:\kappa'}
 ~~~~ \newFi{
   \inference{\Delta,i^A |- F:\kappa & \Delta;\cdot |- s:A}{
              \Delta |- (\lambda i^A.F)\,\{s\} = F[s/i]:\kappa} }
\]
\[ \inference{\Delta |- X:\kappa }{\Delta |- X=X:\kappa}
 ~~~~
   \inference{\Delta |- A=A':* & \Delta |- B=B':*}{\Delta |- A-> B=A'-> B':*}
\]
\[
 \inference{|- \kappa:\square & \Delta,X^\kappa |- F=F' : \kappa'}{
              \Delta |- \lambda X^\kappa.F=\lambda X^\kappa.F':\kappa-> \kappa'}
 ~~~~ ~
 \newFi{
   \inference{\cdot |- A:* & \Delta,i^A |- F=F' : \kappa}{
	      \Delta |- \lambda i^A.F=\lambda i^A.F' : A -> \kappa} }
\]
\[
   \inference{\Delta |- F=F':\kappa->\kappa' & \Delta |- G=G':\kappa}{
              \Delta |- F\,G = F'\,G' : \kappa'}
 ~~~~
 \newFi{
   \inference{\Delta |- F=F':A->\kappa & \Delta;\cdot |- s=s':A}{
	      k\Delta |- F\,\{s\} = F'\,\{s'\} : \kappa} }
\]
\[
   \inference{|- \kappa:\square & \Delta,X^\kappa |- B=B':*}{
              \Delta |- \forall X^\kappa.B=\forall X^\kappa.B':*}
 ~~~~ \quad
 \newFi{
   \inference{\cdot |- A:* & \Delta,i^A |- B=B':*}{
              \Delta |- \forall i^A.B=\forall i^A.B':*} }
\]
\[ \inference{\Delta |- F = F' : \kappa}{\Delta |- F' = F : \kappa}
 ~~~~
   \inference{\Delta |- F = F' : \kappa & \Delta |- F' = F'' : \kappa}{
              \Delta |- F = F'' : \kappa}
\]
~
\paragraph{Term equality:} \fbox{$\Delta;\Gamma |- t = t' : A$}
\[
   \inference{\Delta;\Gamma,x:A |- t:B & \Delta;\Gamma |- s:A}{
              \Delta;\Gamma |- (\lambda x.t)\,s=t[s/x] : B}
 ~~~~
   \inference{\Delta;\Gamma |- x:A}{\Delta;\Gamma |- x=x:A}
\]
\[ \inference{\Delta |- A:* & \Delta;\Gamma,x:A |- t=t':B}{
              \Delta;\Gamma |- \lambda x.t = \lambda x.t':B}
 ~~~~
   \inference{\Delta;\Gamma |- r=r':A-> B & \Delta;\Gamma |- s=s':A}{
              \Delta;\Gamma |- r\;s=r'\;s':B}
\]
\[ \inference{|- \kappa:\square & \Delta, X^\kappa;\Gamma |- t=t' : B}{
              \Delta;\Gamma |- t=t' : \forall X^\kappa.B}
	     (X\notin\FV(\Gamma))
 ~~~~ ~~~~
   \inference{ \Delta;\Gamma |- t=t' : \forall X^\kappa.B
             & \Delta |- G:\kappa }{
              \Delta;\Gamma |- t=t' : B[G/X]}
\]
\[ \newFi{
   \inference{\cdot |- A:* & \Delta, i^A;\Gamma |- t=t' : B}{
              \Delta;\Gamma |- t=t' : \forall i^A.B}
   \left(\begin{smallmatrix}
		i\notin\FV(t),\\
		i\notin\FV(t'),\\
		i\notin\FV(\Gamma)\end{smallmatrix}\right)
 ~~~~
   \inference{ \Delta;\Gamma |- t=t' : \forall i^A.B
             & \Delta;\cdot |- s:A}{
              \Delta;\Gamma |- t=t' : B[s/i]} }
\]
\[ \inference{\Delta;\Gamma |- t=t':A}{\Delta;\Gamma |- t'=t:A}
 ~~~~
   \inference{\Delta;\Gamma |- t=t':A & \Delta;\Gamma |- t'=t'':A}{
              \Delta;\Gamma |- t=t'':A}
\]
\end{singlespace}
\caption{Equality rules of \Fi}
\label{fig:eqFi}
\end{figure}

\subsection{%The constructs new to 
	System~\Fi\ compared to System~\Fw} \label{ssec:newFi}
We assume readers to be familiar with System~\Fw\
and focus on describing the new constructs of \Fi.  These appear in grey boxes.


\paragraph{Kinds.}
The key extension to \Fw\ is the addition of term-indexed arrow kinds of
the form \newFi{A -> \kappa}. This allows type constructors to have terms
as indices. The rest of the development of \Fi\ flows naturally from
this single extension.

\paragraph{Sorting.} \label{sorting}
The formation of indexed arrow kinds is
governed by the sorting rule \newFi{(Ri)}. The rule $(Ri)$ specifies that
an indexed arrow kind $A -> \kappa$ is well-sorted when $A$ has kind $*$
under the empty type level context ($\cdot$) and $\kappa$ is well-sorted.

Requiring the use of the empty context avoids dependent kinds (\ie, kinds depending on type level or value level
bindings). The type $A$ appearing in
the index arrow kind $A -> \kappa$ must be well-kinded under
the empty type level context ($\cdot$).
That is, $A$ should to be a closed type of kind $*$,
which does not contain any free type variables or index variables.
For example, $(\textit{List}\,X -> *)$ is not a well-sorted kind,
while $((\forall X^{*}\!.\,\textit{List}\,X) -> *)$ is a well-sorted kind.

\paragraph{Typing contexts.}
Typing contexts are split into two parts.
Type level contexts ($\Delta$) for type level (static) bindings,
and term level contexts ($\Gamma$) for term level (dynamic) bindings.
A new form of index variable binding ($i^A$) can appear in
type level contexts in addition to the traditional type variable bindings ($X^\kappa$).
There is only one form of term level binding ($x:A$) that appears in
term level contexts.

\paragraph{Well-formed typing contexts.}
A type level context $\Delta$ is well-formed if (1) it is either empty,
or (2) extended by a type variable binding $X^\kappa$ whose kind $\kappa$ is
well-sorted, or (3) extended by an index binding $i^A$ whose type $A$ is
well-kinded under the empty type level context at kind $*$.
This restriction is similar to the one that occurs in the sorting rule ($Ri$)
for term-indexed arrow kinds (see the paragraph {\bf\textit{Sorting}}).
The consequence of this is that, in typing contexts and in sorts,
$A$ must be a closed type (not a type constructor!) without free variables.

A term level context $\Gamma$ is well-formed under a type level context
$\Delta$ when it is either empty or extended by a term variable binding
$x:A$ whose type $A$ is well-kinded under $\Delta$.


\paragraph{Type constructors and their kinding rules.}
We extend the type constructor syntax by three constructs,
and extend the kinding rules accordingly for these new constructs.

\newFi{\lambda i^A.F} is the type level abstraction over an index
(or, index abstraction). Index abstractions introduce indexed arrow kinds
by the kinding rule \newFi{(\lambda i)}. Note, the use of the new form of context
extension, $i^A$, in the kinding rule ($\lambda i$).


\newFi{F\,\{s\}} is the type level index application. In contrast to
the ordinary type level application ($F\,G$) where the argument ($G$) is
a type constructor, the argument of an index application ($F\,\{s\}$) is
a term ($s$). We use the curly bracket notation around an index argument in a type to
emphasize the transition from ordinary type to term, and to emphasize
that $s$ is an index term, which is erasable. Index applications eliminate
indexed arrow kinds by the kinding rule \newFi{(@i)}. Note, we type check
the index term ($s$) under the current type level context paired with
the empty term level context ($\Delta;\cdot$) since we do not want
the index term ($s$) to depend on any term level bindings. Allowing such
a dependency would admit true dependent types.

\newFi{\forall i^A . B} is an index polymorphic type.
The formation of indexed polymorphic types is governed by
the kinding rule \newFi{\forall i}, which is very similar to
the formation rule ($\forall$) for ordinary polymorphic types.

In addition to the rules ($\lambda i$), ($@ i$), and ($\forall i$),
we need a conversion rule \newFi{(Conv)} at kind level. This is because
the new extension to the kind syntax $A -> \kappa$ involves types.
Since kind syntax involves types, we need more than simple structural
equality over kinds. The new equality over kinds is the usual structural equality
extended by type constructor equality when comparing indexed arrow kinds
(see \Fig{eqFi}).

\paragraph{Terms and their typing rules}
The term syntax is exactly the same as other Curry-style calculi.
We write $x$ for ordinary term variables introduced by
term level abstractions ($\lambda x.t$).
We write $i$ for index variables introduced by
index abstractions ($\lambda i^A.F$) and by
index polymorphic types ($\forall i^A.B$). As discussed earlier, the distinction between
$x$ and $i$ is for the convenience of readability.

Since \Fi\ has index polymorphic types ($\forall i^A . B$),
we need typing rules for index polymorphism:
\newFi{(\forall I i)} for index generalization
and \newFi{(\forall E i)} for index instantiation.

The index generalization rule ($\forall I i$) is similar to
the type generalization rule ($\forall I$), but generalizes over
index variables ($i$) rather than type consturctor variables ($X$).
The rule ($\forall I i$) has two side conditions
while the rule ($\forall I$) has only one.
The additional side condition $i\notin\FV(t)$ in the ($\forall I i$) rule
prevents terms from accessing the type level index variables introduced by
index polymorphism. Without this side condition, $\forall$-binder
would no longer behave polymorphically, but instead would behave as
a dependent function, which are usually denoted by the $\Pi$-binder in
dependent type theories. The rule ($\forall I$) for ordinary
type generalization does not need such additional side condition
because type variables cannot appear in the syntax of terms.
The side conditions on generalization rules for polymorphism is fairly standard
in dependently typed languages supporting distinctions between polymorphism
(or, erasable arguments) and dependent functions (\eg, IPTS\cite{LingerS08},
ICC\cite{Miquel01}).

The index instantiation rule ($\forall E i$) is similar to
the type instantiation rule ($\forall E i$), except that
we type check the index term $s$ to be instantiated for $i$
in the current type level context paired with the empty term level context
($\Delta;\cdot$) rather than the current term level context.
Since index terms are at type level, they should not depend on
term level bindings.

In addition to the rules ($\forall I i$) and ($\forall E i$) for
index polymorphism, we need an additional variable rule \newFi{(:i)}
to be able to access the index variables already in scope. Terms ($s$) used
at type level in index applications ($F\{s\}$) should be able to access
index variables already in scope. For example, $\lambda i^A.F\{i\}$ should be
well-kinded under a context where $F$ is well-kinded,
justified by the derivation in Figure \ref{fig:ivarexample}.

\begin{figure}
\[ \!\!\!\!\!\!\!\!\inference[($\lambda i$)]
      { \!\!\!\! \cdot |- A:* \!\!\!\!&
	\inference[($@i$)\!\!\!]{ \!\!\!\!\Delta, i^A |- F : A -> \kappa
                          & \!\!\!\!\inference[($:i$)\!\!\!]{\!\!\!\! i^A\in \Delta,i^A
                                              & \Delta |- \cdot \!\!\!\!}
                                              {\Delta,i^A;\cdot |- i:A\!\!\!\!}
                          }
                          {\Delta, i^A |- F\{i\} : \kappa} }
      {\Delta |- \lambda i^A.F\{i\} :A -> \kappa\!\!\!\!\!\!\!\!\!\!\!\!\!\!\!\!}
\]
\caption{Kinding derivation for an index abstraction}
\label{fig:ivarexample}
\end{figure}

 %% description of fi

\section{Embedding datatypes and Mendler-style iterators}\label{sec:fi:data}

System \Fi\ can express a rich collection of datatypes.
%% TODO cite some paper that does this with System Fw or System F if exist
First, we illustrate embeddings for both non-recursive and
recursive datatypes using Church encodings \cite{Church33} to define
data constructors (\S\ref{ssec:embedChurch}). Second, we illustrate
a more involved embedding for recursive datatypes based on two-level types
(\S\ref{ssec:embedTwoLevel}). Lastly, we discuss an encoding of equality over
term indices (\S\ref{Leibniz}).

\subsection{Embedding datatypes using Church-encoded terms}
\label{ssec:embedChurch}
\begin{figure}
\begin{singlespace}
\begin{align*}
&\!\!\!\!\!\!\mathtt{Bool} &=~& \forall X.X -> X -> X \\
&\!\!\!\!\!\!\mathtt{true}  &\!\!\!:~~& \texttt{Bool} ~~=~ \l x_1.\l x_2. x_1 \\
&\!\!\!\!\!\!\mathtt{false} &\!\!\!:~~& \texttt{Bool} ~~=~ \l x_1.\l x_2. x_2 \\
&\!\!\!\!\!\!\mathtt{elim_{Bool}} &\!\!\!:~~& \texttt{Bool} -> \forall X.X -> X -> X \\
&	&=~& \l x.\l x_1. \l x_2. x\;x_1\,x_2 \qquad
(\textbf{if}~x~\textbf{then}~x_1~\textbf{else}~x_2)
\end{align*}\vspace*{-19pt} \\ \vspace*{-4pt}
\rule{\linewidth}{.4pt}
\begin{align*}
&\!\!\!\!\!\!A_1\times A_2 &=~& \forall X. (A_1 -> A_2 -> X) -> X \\
&\!\!\!\!\!\!\mathtt{pair} &\!\!\!:~~&
	\forall A_1^{*}.\forall A_2^{*}.A_1\times A_2
	~~=~ \l x_1.\l x_2.\l x'.x'\,x_1\,x_2 \\
&\!\!\!\!\!\!\mathtt{elim_{(\times)}} &\!\!\!:~~&
	\forall A_1^{*}.\forall A_2^{*}.A_1\times A_2 ->
	\forall X. (A_1 -> A_2 -> X) -> X \\
	& &=~& \l x.\l x'.x\;x' \\
 &&&\!\!\!\!\!\!\!\!\text{(by passing appropriate values to $x'$, we get}\\
 &&&\!\!\!\!\texttt{fst} = \l x.x(\l x_1.\l x_2.x_1),~
            \texttt{snd} = \l x.x(\l x_1.\l x_2.x_2) ~)
\end{align*} \vspace*{-19pt} \\ \vspace*{-4pt}
\rule{\linewidth}{.4pt}
\begin{align*}
&\!\!\!\!\!\!A_1+A_2 &=~&\forall X^{*}. (A_1 -> X) -> (A_2 -> X) -> X \\
&\!\!\!\!\!\!\mathtt{inl} &\!\!\!:~~& \forall A_1^{*}.\forall A_2^{*}.A_1-> A_1+A_2
	~~=~ \l x. \l x_1. \l x_2 . x_1\,x \\
&\!\!\!\!\!\!\mathtt{inr} &\!\!\!:~~& \forall A_1^{*}.\forall A_2^{*}.A_2-> A_1+A_2
	~~=~ \l x. \l x_1. \l x_2 . x_2\,x \\
&\!\!\!\!\!\!\mathtt{elim_{(+)}} &\!\!\!:~~&
	\forall A_1^{*}.\forall A_2^{*}.(A_1+A_2) -> \\
	&&& \forall X^{*}. (A_1 -> X) -> (A_2 -> X) -> X \\
	& &=~& \l x.\l x_1. \l x_2. x\;x_1\,x_2 \\
	&&&			(\textbf{case}~x~\textbf{of}~
				\{\mathtt{inl}~x' -> x_1\;x';
				  \mathtt{inr}~x' -> x_2\;x'\})
\end{align*}~\vspace*{-10pt}
\end{singlespace}
\caption{Embedding non-recursive datatypes}
\label{fig:churchnonrec}
\end{figure}
\begin{figure}
\begin{singlespace}
\begin{align*}
&\!\!\!\!\!\!\mathtt{List} &\!\!\!\!\!=~& \l A^{*}.\forall X^{*}.(A-> X-> X)-> X-> X
	\\
&\!\!\!\!\!\!\texttt{cons} &\!\!\!\!\!:~~& \forall A^{*}.A-> \mathtt{List}\,A-> \mathtt{List}\,A \\
& & & \qquad~\qquad~\quad\, =~\l x_a.\l x.\l x_c.\l x_n. x_c\,x_a\,(x\;x_c\,x_n) \\
&\!\!\!\!\!\!\mathtt{nil} &\!\!\!\!\!:~~& \forall A^{*}.\texttt{List}\,A
~~=~ \l x_c.\l x_n.\l x_n \\
&\!\!\!\!\!\!\mathtt{elim_{List}} &\!\!\!\!:~~& \forall A^{*}.\texttt{List}\,A ->
	\forall X^{*}.(A -> X -> X) -> X -> X \\
& &\!\!\!\!\!=~& \l x.\l x_c. \l x_n.x\;x_c\,x_n\qquad
	\text{(\textit{foldr} $x_z$ $x_c$ $x~$ in Haskell)}
\end{align*}\vspace*{-19pt} \\ \vspace*{-4pt}
\rule{\linewidth}{.4pt}
\begin{align*}
&\!\!\!\!\!\!\mathtt{Powl} &\!\!\!\!\!=~& \l A^{*}.\\
&&&\forall X^{*-> *}.(A-> X(A\times A)-> XA)-> XA -> XA \\
&\!\!\!\!\!\!\texttt{pcons} &\!\!\!\!\!:~~& \forall A^{*}.A-> \mathtt{Powl}(A\times A)-> \mathtt{Powl}\,A \\
&&& \qquad~\qquad~\quad\, ~=~ \l x_a.\l x.\l x_c.\l x_n. x_c\,x_a\,(x\;x_c\,x_n) \\
&\!\!\!\!\!\!\mathtt{pnil} &\!\!\!\!\!:~~& \forall A^{*}.\texttt{Powl}\,A
~~~=~ \l x_c.\l x_n.\l x_n \\
&\!\!\!\!\!\!\mathtt{elim_{Powl}} &\!\!\!\!:~~& \forall A^{*}.\texttt{Powl}\,A -> \\
&&& \forall X^{*-> *}.(A -> X(A\times A) -> XA) -> XA -> XA \\
& &\!\!\!\!\!=~& \l x.\l x_c. \l x_n.x\;x_c\,x_n
\end{align*}\vspace*{-19pt} \\ \vspace*{-4pt}
\rule{\linewidth}{.4pt}
\begin{align*}
&\!\!\!\!\!\!\mathtt{Vec} &\!\!\!\!\!\!\!\!=~& \l A^{*}.\l i^{\mathtt{Nat}}.\\
&&&	\forall X^{\mathtt{Nat}-> *}.
	(\forall i^\mathtt{Nat}.A-> X\{i\}-> X\{\mathtt{S}\,i\}) ->  \\
&&& \qquad~\qquad X\{\texttt{Z}\} -> X\{i\} \\
 &\!\!\!\!\!\!\texttt{vcons} &\!\!\!\!\!\!\!\!:~& \forall A^{*}.\forall i^\mathtt{Nat}.A-> \mathtt{Vec}\,A\,\{i\}-> \mathtt{Vec}\,A\,\{\mathtt{S}\,i\} \\
&&&\;\qquad\qquad\quad =~ \l x_a.\l x.\l x_c.\l x_n. x_c\,x_a\,(x\;x_c\,x_n) \\
&\!\!\!\!\!\!\mathtt{vnil} &\!\!\!\!\!\!\!\!:~& \forall A^{*}.\texttt{Vec}\,A\,\{\mathtt{Z}\} 
~~~=~ \l x_c.\l x_n.x_n \\
&\!\!\!\!\!\!\mathtt{elim_{Vec}} &\!\!\!\!\!\!\!\!:~& \forall A^{*}.\forall i^\mathtt{Nat}.\texttt{Vec}\,A\,\{i\} -> \\
&&& \forall X^{\mathtt{Nat}-> *}.(\forall i^\mathtt{Nat}.A -> X\{i\} -> X\{\mathtt{S}\,i\}) -> \\
&&& \qquad~\qquad X\{\mathtt{Z}\} -> X\{i\} \\
& &\!\!\!\!\!=~& \l x.\l x_c. \l x_n.x\;x_c\,x_n
\end{align*} ~\vspace*{-14pt}
\end{singlespace}
\caption{Embedding recursive datatypes}
\label{fig:churchrec}
\end{figure}
\citet{Church33} invented an embedding of the natural numbers into
the untyped $\lambda$-calculus, which he used to argue
that the $\lambda$-calculus was expressive enough for the foundation of
logic and arithmetic. Church encoded the data constructors of natural numbers,
successor and zero, as higher-order functions,
$\mathtt{succ}=\l x.\l x_s.\l x_z.x_s(x\,x_s x_z)$ and
$\mathtt{zero}=\l x_s.\l x_z.x_z$.
The heart of the Church encoding is that a value is encoded as an elimination function.
The bound variables $x_s$ and $x_z$ (of both $\mathtt{succ}$ and $\mathtt{zero}$) stand for the operations needed to
eliminate the successor case and the zero case respectively. The Church encodings of
successor states: to eliminate $\mathtt{succ}\,x$, ``apply $x_s$
to the elimination of the predecessor $(x\,x_s x_z)$"; and,
to eliminate $\mathtt{zero}$, just ``return $x_z$".
Since values {\it are} elimination functions, the
eliminator can be defined as applying the value itself to the needed operations. One
for each of the data constructors. 
For instance, we can define an eliminator
for the natural numbers as $\mathtt{elim_{Nat}}=\l x.\l x_s.\l x_z.x\,x_s x_z$.
This is just an $\eta$-expansion of the identity function $\l x.x$.
The Church encoded natural numbers are typable in a polymorphic $\lambda$-calculi,
such as System \Fw, as follows:\vspace*{-2pt}
\begin{align*}
&\texttt{Nat} &=~& \forall X^{*}.(X -> X) -> X -> X \qquad\qquad\qquad\\
&\texttt{S} &\!\!\!:~~& \texttt{Nat} -> \texttt{Nat}
	~~ =~ \l x.\l x_s.\l x_z.x_s(x\,x_s x_z) \\
&\texttt{Z} &\!\!\!:~~& \texttt{Nat} \qquad\quad\,
	~~ =~ \l x_s.\l x_z.x_z \\
&\mathtt{elim_{Nat}} &\!\!\!:~~& \texttt{Nat} -> \forall X^{*}.(X -> X)-> X-> X \\
& &=~& \l x.\l x_s.\l x_z.x\,x_s x_z
\end{align*}~\vspace*{-13pt}

In a Similar fashion, other datatypes are also embeddable into
polymorphic $\lambda$-calculi.
Embeddings of some well-known non-recursive datatypes are illustrated
in Figure \ref{fig:churchnonrec}, and embeddings of the list-like
recursive datatypes, which we discussed as motivating examples
in the beginning of this chapter, are illustrated in Figure \ref{fig:churchrec}.
Note that the term encodings for the constructors and eliminators of
the list-like datatypes in Figure \ref{fig:churchrec} are exactly the same.
For instance, the term encodings for \texttt{nil}, \texttt{pnil}, and
\texttt{vnil} are all the same term: $\l x_s.\l x_z.x_z$. The nil and cons terms
capture the linear nature of lists, so they are the same for all list like structures.
But, the types differ, capturing different invariants about lists -- shape of the elements ({\tt Powl}), and
length of the list ({\tt Vec}).

\subsection{Embedding recursive datatypes as two-level types}
\label{ssec:embedTwoLevel}
We can divide a recursive datatype definition into two parts --
a recursive type operator and a base structure. The operator ``weaves" recursion
into the datatype definition, and the base structure describes
its shape (\ie, number of data constructors and their types).
One can program with two-level types in any functional language that supports
higher-order polymorphism\footnote{\aka\ higher-kinded polymorphism,
	or type-constructor polymorphism}, such as Haskell. 
In Figure \ref{fig:twoleveltypes}, we illustrate this by giving an example of a two level definition
for ordinary lists (all the other types in this paper have similar definitions).

The use of two-level types has been recognized as
a useful functional programming pearl \cite{Sheard04}, since two-level types
separate the two concerns of (1) recursion on recursive sub components
and (2) handling different cases (by pattern matching over the shape of the (non-recursive) base structure).
An advantage of such an approach, is that a single eliminator can be defined once for
all datatypes of the same kind. For example, the function $\mathtt{mit}_\kappa$ describes
Mendler-style iteration\footnote{An iteration is a principled recursion
	scheme guaranteed to terminate for any well-founded input.
	Also known as fold, or catamorphism.} for the recursive types
defined by $\mu_\kappa$. Although it is possible to write programs using two level datatypes
in a general purpose functional language, one could
not expect logical consistency in such systems.

\begin{figure}\vspace*{-8pt}
\begin{singlespace}
\begin{lstlisting}[basicstyle={\ttfamily\small},language=Haskell,mathescape]
newtype $\mu_{*}$ (f :: * -> *) = In$_{*}$ (f ($\mu_{*}$ f))

data ListF (a::*) (r::*) = Cons a r | Nil

type List a = $\mu_{*}$ (ListF a)
cons x xs = In$_{*}$ (Cons x xs)
nil       = In$_{*}$ Nil

mit$_{*}$ :: ($\forall$ r.(r->x) -> f r -> x) -> Mu0 f -> x
mit$_{*}$ phi (In$_{*}$ z) = phi (mit$_{*}$ phi) z

newtype $\mu_{(*-> *)}$ (f :: (*->*) -> (*->*)) (a::*)
  = In$_{(*-> *)}$ (f (Mu$_{(*-> *)}$ f)) a

data PowlF (r::*->*) (a::*) = PCons a (r(a,a)) | PNil

type Powl a = $\mu_{(*-> *)}$ PowlF a
pcons x xs = In$_{(*-> *)}$ (PCons x xs)
pnil       = In$_{(*-> *)}$ PNil

mit$_{(*-> *)}$ :: ($\forall$ r a.($\forall$a.r a->x a) -> f r a -> x a)
        -> $\mu_{(*-> *)}$ f a -> x a
mit$_{(*-> *)}$ phi (In$_{(*-> *)}$ z) = phi (mit$_{(*-> *)}$ phi) z

-- above is Haskell (with some GHC extensions)
-- below is Haskell-ish pseudocode

newtype $\mu_{(\mathtt{Nat}-> *)}$ (f::(Nat->*)->(Nat->*)) {n::Nat}
  = In$_{(\mathtt{Nat}-> *)}$ (f ($\mu_{(\mathtt{Nat}-> *)}$ f)) {n}

data VecF (a::*) (r::Nat->*) {n::Nat} where
  VCons :: a -> r n -> VecF a r {S n}
  VNil  :: VecF a r {Z}

type Vec a {n::Nat} = $\mu_{(\mathtt{Nat}-> *)}$ (VecF a) {n}
vcons x xs = In$_{(\mathtt{Nat}-> *)}$ (VCons x xs)
vnil       = In$_{(\mathtt{Nat}-> *)}$ VNil

mit$_{(\mathtt{Nat}-> *)}$::($\forall$ r n.($\forall$n.r{n}->x{n})->f r {n}->x{n})
        -> $\mu_{(\mathtt{Nat}-> *)}$ f {n} -> x{n}
mit$_{(\mathtt{Nat}-> *)}$ phi (In$_{(\mathtt{Nat}-> *)}$ z) = phi (mit$_{(\mathtt{Nat}-> *)}$ phi) z
\end{lstlisting}
\end{singlespace}
\caption{2-level types and their Mendler-style iterators in Haskell}
\label{fig:twoleveltypes}
\end{figure}

Interestingly, there exist embeddings of the recursive type operator
$\mu_\kappa$, its data constructor $\mathtt{In}_\kappa$, and
the Mendler-style iterator $\mathtt{mit}_\kappa$ for each kind $\kappa$
into the higher-order polymorphic $\lambda$-calculus \Fi, as illustrated
in Figure \ref{fig:mu}. In addition to illustrating the general form of
embedding $\mu_\kappa$, we also fully expand the embeddings for some
instances ($\mu_{*}$, $\mu_{*->*}$, $\mu_{\mathtt{Nat}->*}$), which are
used in Figure \ref{fig:twoleveltypes}. These embeddings support the embedding
of arbitrary type- and term-indexed recursive datatypes into System \Fi.
Thus we can reason about these datatypes in a logically consistent calculus.

However, it is important to note that there does not exist an embedding of the
arbitrary destruction (or, pattern matching away) of the $\mathtt{In}_\kappa$
constructor. It is known that combining arbitrary recursive datatypes with
the ability to destruct (or, unroll) their values
is powerful enough to define non-terminating computations in a type safe way,
leading to logical inconsistency. Some systems maintain consistency by restricting
which recursive datatypes can be defined, but allow arbitrary unrolling. In System
\Fi, we can define any datatype, but restrict unrolling to Mendler style operators
definable in \Fi. Such operators are quite expressive, capturing at least
iteration, primitive recursion, and courses of values recursion.

\afterpage{
\begin{landscape}
\begin{figure}
\begin{singlespace}
\begin{multline*} \text{notation:}\quad
   \boldsymbol{\l}\mathbb{I}^\kappa.F =
	\lambda I_1^{K_1}.\cdots.\lambda I_n^{K_n}.F \qquad
   \boldsymbol{\forall}\mathbb{I}^\kappa.B =
	\forall I_1^{K_1}.\cdots.\forall I_n^{K_n}.B \qquad
   F\mathbb{I} = F I_1 \cdots I_n \qquad
   F \stackrel{\kappa}{\pmb{\pmb{->}}} G =
	\boldsymbol{\forall}\mathbb{I}^\kappa.F\mathbb{I} -> G\mathbb{I} \\
\begin{array}{lll}
\text{where}
 	& \kappa = K_1 -> \cdots -> K_n -> * & \text{and} ~~~
 	\text{$I_i$ is an index variable ($i_i$) when $K_i$ is a type,}
 		\\
 	& \mathbb{I}\,=I_1,\;\dots\;\dots\;,\;I_n& \qquad~\qquad
 	\text{a type constructor variable ($X_i$) otherwise
		(\ie, $K_i=\kappa_i$).}
\end{array}
\end{multline*} ~ \vspace*{-5pt}
\hrule  \vspace*{-2pt}
\begin{align*}
&\mu_\kappa &\!\!\!\!\!~:~~& (\kappa -> \kappa) -> \kappa
  \qquad\qquad\qquad\qquad\quad
  = \l F^{\kappa -> \kappa}.\boldsymbol{\l}\mathbb{I}^\kappa.
  \forall X^\kappa.
  (\forall {X_r}^{\!\!\kappa}.
  	(X_r \karrow{\kappa} X) ->
	(F X_r \karrow{\kappa} X)) -> X\mathbb{I} \\
&\mu_{*} &\!\!\!\!\!~:~~& (* -> *) -> * 
 \qquad\qquad\qquad\qquad\quad~
 = \l F^{* -> *}.\phantom{\boldsymbol{\l}\mathbb{I}^\kappa.}
 \forall X^{*}.(\forall {X_r}^{\!\!*}.(X_r -> X) -> (F\,X_r -> X)) -> X \\
&\mu_{*-> *} &\!\!\!\!\!~:~~& ((* -> *) -> (* -> *)) -> (* -> *) \\
&            &\!\!\!\!\!=~&
  \l F^{(*-> *) -> (*-> *)}.\l X_1^{*}.
   \forall X^{* -> *}.(\forall {X_r}^{\!\!* -> *}.
   (\forall X_1^{*}.X_r X_1 -> X X_1) -> (\forall X_1^{*}.F\,X_r X_1 -> X X_1)) -> X X_1 \\
  &\mu_{\mathtt{Nat}-> *} &\!\!\!\!\!~:~~& ((\mathtt{Nat} -> *) -> (\mathtt{Nat} -> *)) -> (\mathtt{Nat} -> *) \\
&            &\!\!\!\!\!=~&
  \l F^{(\mathtt{Nat}-> *) -> (\mathtt{Nat}-> *)}.\l i_1^\mathtt{Nat}.
  \forall X^{\mathtt{Nat} -> *}.(\forall {X_r}^{\!\!\mathtt{Nat} -> *}.
  (\forall i_1^\mathtt{Nat}.X_r i_1 -> X i_1) -> (\forall i_1^\mathtt{Nat}.F\,X_r i_1 -> X i_1)) -> X i_1 \qquad\qquad
\end{align*}
\begin{align*}
\mathtt{In}_{\kappa} \,~\,&~~:~ \forall F^{\kappa-> \kappa}.
	F(\mu_\kappa F) \karrow{\kappa} \mu_\kappa F
&&=~ \l x_r. \l x_\varphi.x_\varphi\,(\mathtt{mit}_\kappa~x_\varphi)\,x_r
	\qquad~\qquad~\qquad~\qquad~\quad \\
\mathtt{mit}_\kappa &~~:~ \forall F^{\kappa-> \kappa}.\forall X^\kappa.
	(\forall {X_r}^{\!\!\kappa}.
	 (X_r \karrow{\kappa} X) ->
	 (F X_r \karrow{\kappa} X) ) ->
	(\mu_\kappa F \karrow{\kappa} X)
&&=~ \l x_\varphi.\l x_r.x_r\,x_\varphi
\end{align*} ~ \vspace*{-10pt}
\end{singlespace}
\caption{Embedding of the recursive operators ($\mu_\kappa$),
	their data constructors ($\mathtt{In}_\kappa$),
	and the Mendler-style iterators ($\mathtt{mit}_\kappa$) in \Fi.}
\label{fig:mu}
\end{figure}
\end{landscape}
} %% end afterpage

\begin{singlespace}
\begin{example}
The datatype of \mbox{$\lambda$-terms} in context 
\begin{verbatim}
data Lam ( C: Nat -> * ) { i: Nat } where
  LVar : C{i} -> Lam{i}
  LApp : Lam{i} -> Lam{i} -> Lam{i}
  LAbs : Lam{S i} -> Lam{i}
\end{verbatim}
is encoded as:
\[
\mathtt{Lam} \triangleq
\!\!\!
\begin{array}[t]{l}
\l C^{\mathtt{Nat}\to*}
\l i^\mathtt{Nat}.\,\forall X^{\mathtt{Nat}\to*}.
\\[1mm]
\quad
  (\forall j^\mathtt{Nat}.\,C\s j \to X\s j)
\\[1mm]
\quad\quad
 \to(\forall j^\mathtt{Nat}.\,X\s j \to X\s j \to X\s j)
\\[1mm]
\quad\quad\quad
\to(\forall j^\mathtt{Nat}.\,X\s{\mathtt S\, j} \to X\s j)
\\[1mm]
\quad\quad\quad\quad
  \to X\s i
\end{array}
\]
For a concrete representation one can consider
$\mathtt{Lam}\,\mathtt{Fin}$ where
\begin{verbatim}
data Fin { i: Nat } where
  FZ : Fin{S i}
  FS : Fin{i} -> Fin{S i}
\end{verbatim}
This is encoded as
\[
\mathtt{Fin}\triangleq
\!\!\!
\begin{array}[t]{l}
\l i^{\mathtt{Nat}}.\,\forall X^{\mathtt{Nat}\to*}.\,
(\forall j^\mathtt{Nat}.\, X\s{\mathtt S\, j})
	\to (\forall j^\mathtt{Nat}.\, X\s j\to X\s{\mathtt S\,j})
	\to X\s i
\end{array}
\]
\end{example}
\end{singlespace}


\subsection{Leibniz index equality}
\label{Leibniz}

The quantification over type-indexed kinding available in System~\Fi\
allows the definition of \emph{Leibniz-equality type} constructor
$\LEq_A: A\to A\to *$ on closed types~$A$, defined as follows:
\[ \LEq_A \triangleq
	\l i^A.\, \l j^A.\, \forall X^{A\to*}.\, X\{i\}\to X\{j\}
\]

%\[\begin{array}{c}
%\Delta;\cdot\vdash s=s':A\enspace\quad \Delta;\cdot\vdash t=t':A
%\\ \hline \\[-3mm]
%\Delta\vdash F_A \s s \s t = F_A\s{s'}\s{t'}:*
%\end{array}\] 
%and, as further basic properties, 
Observe that the following types are
inhabited: 
\[\!\!\!\!\begin{array}{l}
\text{\small(Reflexive)} 
~~\,
\forall i^A.\,\LEq_A\s{i}\s{i}
\\[1mm]
\text{\small(Transitive)} 
~~
\forall i^A.\,\forall j^A.\,\forall k^A.\,
  \LEq_A\s{i}\s{j}\to \LEq_A\s{j}\s{k}\to \LEq_A\s{i}\s{k}
\\[1mm]
\text{\small(Logical)}
\quad~\, \forall i^A.\,\forall j^A.\, 
\LEq_A\s{i}\s{j}\to \forall f^{A\to B}.\, \LEq_B\s{f\,i}\s{f\,j}
\\[1mm]
\qquad\qquad~~~
\forall f^{A\to B}.\,\forall g^{A\to B}.\, 
\LEq_{A\to B}\s{f}\s{g}\to 
\forall i^A.\, \LEq_B\s{f\,i}\s{g\,i}
\end{array}\]
In addition to the above,
one also has the inhabitation of the following type:\footnote{
	Intuitively, this is obvious since we can swap the order of
	consecutive universal quantification over indices.  That is,
	from $(\forall i^A.\,\forall j^A.\cdots)$
	to $(\forall j^A.\,\forall i^A.\cdots)$.}
\[\!\!\!\!\begin{array}{l}
\text{\small(Symmetric)} 
\quad
\forall i^A.\,\forall j^A.\,
  \LEq_A\s{i}\s{j}\to\LEq_A\s{j}\s{i}
\end{array}\qquad\qquad\qquad\qquad\qquad\]
Hence Leibniz equality is a congruence.

%This is not so for the type 
%\begin{equation}\label{NonSymmetry}
%\forall i^A.\,\forall j^A.\, \LEq_A\s{i}\s{j}\to\LEq_A\s{j}\s{i}
%\enspace.
%\end{equation}
%(Cf.~Example~\ref{PathologicalExampleContinued} in~\S\ref{sec:theory}.)

In applications, the types~$\LEq_A$ are useful in constraining the
term-indexing of datatypes. %as parameterised by coercions.  
A general such construction is given by the type constructors $\Ran_{A,B}:
(A\to B) \to (A\to*) \to B\to *$.  These are defined as 
\[
\Ran_{A,B}
\triangleq
\l f^{A\to B}.\,
  \l X^{A\to*}.\,
    \l j^B.
      \forall i^A.\,
        \LEq_B \s j \s{f\,i}
	  \to X\s i
\]
%for closed types~$A$ and $B$, 
and are in spirit right Kan extensions, a notion that is being extensively
used in programming,~\eg~\cite{AbeMatUus05,JohannGhani08}. %,Hinze}.  .  

%It follows that, %Here, 
%%for closed $t:A\to B$, $F:A\to*$, and $s:B$, a closed term $u:
%%(\Ext_{A,B}\ \s t\ F) \s s$ is a polymorphic function that, for every
%%closed $r:A$, given a generic coercion $\forall X^{B\to*}.\, X\s s \to
%%X\s{t\,r}$ provides output of type $F\s r$.  In particular, 
%the type
%$\forall f^{A\to B}.\,\forall X^{A\to*}.\,\forall i^A.\,
%(\Ext_{A,B}\ \s f\ X) \s{f\,i}\to X\s i$ is inhabited by 
%$\l f.\,f(\l x.\,x)$.
%%
%%(We note the interesting fact that the type 
%%$\forall X^{A\to*}.\,\forall j^A.\,\forall i^A.\, 
%%   X\s i \to (\Ext_{A,A}\ \s{\lambda x.\,x}\ X) \s i$
%%is inhabited by a retraction.)
%
One of their usefulness comes from the fact that the following type
is inhabited by a section
\[%\l h.\,\l y.\,\l g.\, h(g\, y):
\forall Y^{B\to*}\!.\,\forall X^{A\to*}\!.\,\forall f^{A\to B}\!.\,
\big(\forall i^A\!.\,Y\s{f\,i}\to X\s i\big)
\to
\big(\forall j^B\!.\,Y\s j\to (\Ran_{A,B}\s f X)\s j\big)
\]
This allows one to represent functions from input datatypes with
constrained indices as plain indexed functions, and vice versa.  For
instance, by means of the iterators of the previous section one can define
a vector tail function of type 
\[
\forall X^{*}.\,\forall j^{\mathtt{Nat}}.\,\mathtt{Vec}\, X\, \s j \to
\big(\Ran_{\mathtt{Nat},\mathtt{Nat}}\,\s{\mathtt S}(\mathtt{Vec}\,
X)\big)\s j 
\]
and retract it to one of type
\[
\forall X^{*}.\,\forall i^{\mathtt{Nat}}.\,\mathtt{Vec}\, X\, \s{\mathtt
S\, i} \to \mathtt{Vec}\, X\,\s i
\enspace.
\]
%
Analogously, one can use an iterator to define a single-variable
capture-avoiding substitution function of type
\[
\forall i^{\mathtt{Nat}}.\,
(\mathtt{Lam}\,\mathtt{Fin})\s i
\to
\big(\Ran_{\mathtt{Nat},\mathtt{Nat}}
\s{\mathtt S}
(\lambda j^{\mathtt{Nat}}.\,
\mathtt{Lam}\,\mathtt{Fin}\s j
\to
\mathtt{Lam}\,\mathtt{Fin}\s j)\big)
\s i
\]
and then retract it to one of type 
\[
\forall i^{\mathtt{Nat}}.\,
(\mathtt{Lam}\,\mathtt{Fin})\s{\mathtt S\, i}
\to
(\mathtt{Lam}\,\mathtt{Fin})\s{i}
\to
(\mathtt{Lam}\,\mathtt{Fin})\s{i}
\enspace.
\]

Type constructors ${\Lan_{A,B}:(A\to B)\to (A\to*)\to B\to *}$, which are
in spirit left Kan extensions, permit the encoding of functions of type
$(\forall i^{A}.\, F\s i\to G\s{t\,i})$, for ${F:A\to*}$, ${G:B\to*}$, and
${t:A\to B}$, as functions of type
\[ \forall j^{B}.\, (\Lan_{A,B}\s t F)\s j\to G\s{j}\; .\]
Left Kan extensions are dual to right Kan extensions, but
to define them as such one needs existential and product types.
In formalisms without them, these have to be encoded.
This can be done as follows: 
\[
\Lan_{A,B}
\triangleq
\l f^{A\to B}.\,
\l X^{A\to*}.\,
\l j^{B}.\, 
\forall Z^{*}.\,
  (\forall i^A.\, \LEq_B\s{f\,i}\s{j}\to X\s i\to Z)\to Z
\]
The type
\[
\forall X^{A\to*}.\,
\forall Y^{B\to*}.\,
\forall f^{A\to B}.\,
(\forall i^{A}.\, 
X\s i\to Y\s{f\,i})
\to
(\forall j^{B}.\, 
(\Lan_{A,B}\s f X)\s j\to Y\s{j})
\]
is thus inhabited by a section, providing a retractable coercion
between the two functional representations.

Left Kan extensions come with a canonical section of type
\[ \forall f^{A\to B}.\,\forall X^{A\to *}.\,\forall i^A.\, X\s i \to
(\Lan_{A,B}\s f X)\s{f\,i}\]
that, according to a reindexing function
$t:A\to B$, embeds an \mbox{$A$-indexed} type $F$ (at index $s$) into
the $B$-indexed type $\Lan_{A,B}\s t F$ (at index $t\,s$).  For instance, the
type constructor $\mathtt{Lan}_{A,A\times A}\s{\l x.\,\mathtt{pair}\,x\,x}$
embeds arrays of types into matrices along the diagonal; while the type
constructors $\mathtt{Lan}_{A\times A,A}\s{\mathtt{fst}}$ and
$\mathtt{Lan}_{A\times A,A}\s{\mathtt{snd}}$ respectively encapsulate matrices
of types as arrays by columns and by rows.

 %% embedding datatypes and |mcata| for term-indexed datatypes

\section{Metatheory}\label{sec:fi:theory}
The expectation is that System \Fi\ has all the nice properties of System \Fw,
yet is more expressive because of the addition of term-indexed types.

We show some basic well-formedness properties for
the judgments of \Fi\ in \S\ref{ssec:fi:wf}.
We prove erasure properties of \Fi, which captures the idea that indices are
erasable since they are irrelevant for reduction in \S\ref{ssec:fi:erasure}.
We show strong normalization, logical consistence, and subject reduction for
\Fi\ by reasoning about well-known calculi related to \Fi\ in \S\ref{ssec:fi:sn}.

\subsection{Well-formedness properties and substitution lemmas}
\label{ssec:fi:wf}
We want to show that kinding and typing derivations give
well-formed results under well-formed contexts. That is,
kinding derivations ($\Delta |- F : \kappa$) result in well-sorted kinds
($|- \kappa$) under well-formed type-level contexts ($|- \Delta$)
(Proposition \ref{prop:wfkind}), and
typing derivations ($\Delta;\Gamma |- t : A$) result in well-kinded types
($\Delta;\Gamma |- A:*$) under well-formed type and term-level contexts
(Proposition \ref{prop:wftype}).

\begin{proposition}
\label{prop:wfkind}
$ \inference*{ |- \Delta & \Delta |- F : \kappa}{
	\qquad |- \kappa:\square \quad} $
\end{proposition}

\begin{proposition}
\label{prop:wftype}
$ \inference*{ \Delta |- \Gamma & \Delta;\Gamma |- t : A}{
	\qquad \Delta |- A : * \qquad} $
\end{proposition}

We can prove these well-formedness properties
by induction over the judgment, and using 
the well-formness lemmas on equalities
(Lemmas~\ref{lem:wfeqkind}, \ref{lem:wfeqtype}, and \ref{lem:wfeqterm})
and the substitution lemma (Lemma~\ref{lem:subst}).
The proof for Propositions \ref{prop:wfkind} and \ref{prop:wftype}
are mutually inductive.  So, we prove these two propositions
at the same time, using a combined judgment $J$,
which is either a kinding judgment or a typing judgment
(\ie, $J ::= \Delta |- F : \kappa \mid \Delta;\Gamma |- t : A$).
See Appendix \ref{app:proofsFi} for the detailed proofs of the
two propositions above.

\begin{lemma}[kind equality is well-sorted]\label{lem:wfeqkind}
$ \inference{|- \kappa = \kappa':\square}
	{|- \kappa:\square \quad |- \kappa':\square} $
\end{lemma}
\begin{proof}
	By induction on the derivation of kind equality
	and using the sorting rules.
\end{proof}

\begin{lemma}[type constructor equality is well-kinded]\label{lem:wfeqtype}
\[ \inference{\Delta |- F = F':\kappa}
	{\Delta |- F:\kappa \quad \Delta |- F':\kappa}
\]
\end{lemma}
\begin{proof}
	By induction on the derivation of type constructor equality
	and using the kinding rules.
	Also use the type substitution lemma
	(Lemma~\ref{lem:subst}(\ref{lem:tysubst}))
	and the index substitution lemma
	(Lemma~\ref{lem:subst}(\ref{lem:ixsubst})).  
\end{proof}

\begin{lemma}[term equality is well-typed]\label{lem:wfeqterm}
\[ \inference{\Delta,\Gamma |- t = t':A}
	{\Delta,\Gamma |- t:A \quad \Delta,\Gamma |- t':A}
\]
\end{lemma}
\begin{proof}
	By induction on the derivation of term equality
	and using the typing rules.
	Also use the term substitution lemma
	(Lemma~\ref{lem:subst}(\ref{lem:tmsubst})).
\end{proof}

The proofs for the three lemmas above are straightforward
once we have dealt with the interesting cases for the equality rules
involving substitution. We can prove those interesting cases
by applying the substitution lemmas. The other cases fall into two
categories: firstly, the equality rules following the same structure of
the sorting, kinding, and typing rules; and secondly, the reflexive
rules and the transitive rules. The proof for the equality rules
following the same structure of the sorting, kinding, and typing rules
can be proved by induction and applying the corresponding
sorting, kinding, and typing rules. The proof for the reflexive rules
and the transitive rules can be proved simply by induction.

\begin{lemma}[substitution]\mbox{}\\[-3mm]
\label{lem:subst}
\begin{enumerate}
\item
%\begin{lemma}[type substitution]
\label{lem:tysubst}
\mbox{\rm (type substitution)}
$\inference{\Delta,X^\kappa |- F:\kappa' & \Delta |- G:\kappa}
	{\Delta |- F[G/X]:\kappa'} $
%\end{lemma}
\medskip

\item
%\begin{lemma}[index substitution]
\label{lem:ixsubst}
\mbox{\rm (index substitution)}
$ \inference{\Delta,i^A |- F:\kappa & \Delta;\cdot |- s:A}
	{\Delta |- F[s/i]:\kappa} $
%\end{lemma}
\medskip

\item
%\begin{lemma}[term substitution]
\label{lem:tmsubst}
\mbox{\rm (term substitution)}
$ \inference{\Delta;\Gamma,x:A |- t:B & \Delta;\Gamma |- s:A}
	{\Delta,\Gamma |- t[s/x]:B} $
%\end{lemma}
\end{enumerate}
\end{lemma}
The substitution lemma is fairly standard, comparable to substitution lemmas
in other well-known systems such as \Fw\ or ICC.

\subsection{Erasure properties}
\label{ssec:fi:erasure}

We define a meta-operation of index erasure that projects $\Fi$-types
to $\Fw$-types.

\begin{definition}[index erasure]\label{def:ierase}
\[ \fbox{$\kappa^\circ$}
 ~~~~ ~~
 *^\circ =
 *
 ~~~~ ~~
 (\kappa_1 -> \kappa_2)^\circ =
 {\kappa_1}^\circ -> {\kappa_2}^\circ
 ~~~~ ~~
 (A -> \kappa)^\circ =
 \kappa^\circ
\]
\[ \fbox{$F^\circ$}
 ~~~~
 X^\circ =
 X
 ~~~~ ~~~~
 (A -> B)^\circ =
 A^\circ -> B^\circ
\]
\[ \qquad
 (\lambda X^\kappa.F)^\circ =
 \lambda X^{\kappa^\circ}.F^\circ
 ~~~~ ~~~~
 (\lambda i^A.F)^\circ =
 F^\circ
\]
\[ \qquad
 (F\;G)^\circ =
 F^\circ\;G^\circ
 ~~~~ ~~~~ ~~~~ ~~~~ ~~
 (F\,\{s\})^\circ =
 F^\circ
\]
\[ \qquad
 (\forall X^\kappa . B)^\circ =
 \forall X^{\kappa^\circ} . B^\circ
 ~~~~ ~~~~
 (\forall i^A . B)^\circ =
 B^\circ
\]
\[ \fbox{$\Delta^\circ$}
 ~~~~
 \cdot^\circ = \cdot
 ~~~~ ~~
 (\Delta,X^\kappa)^\circ = \Delta^\circ,X^{\kappa^\circ}
 ~~~~ ~~
 (\Delta,i^A)^\circ = \Delta^\circ
\]
\[ \fbox{$\Gamma^\circ$}
 ~~~~
 \cdot^\circ = \cdot
 ~~~~ ~~~~
 (\Gamma,x:A)^\circ = \Gamma^\circ,x:A^\circ
\]
\end{definition}

\begin{example}\label{PathologicalExample}
The meta-operation of index erasure simply discards all indexing
information.  The effect of this on most datatypes is to project the
indexing invariants while retaining the type structure.  
%
This is clearly seen for the vector type constructor~$\mathtt{Vec}$ whose
index erasure is the list type constructor~$\mathtt{List}$,
see~\Fig{churchrec}.
%
One can however build pathological examples.  For instance, the
type $\mathtt P_A \triangleq \forall i^A.\,\forall j^A.\, \LEq_A\s i \s j$
has index erasure $\mathtt{Unit} \triangleq \forall X^\mathtt{*}.\,X\to
X$.
\end{example}

\begin{theorem}[index erasure on well-sorted kinds]
\label{thm:ierasesorting}
	$\inference{|- \kappa : \square}{|- \kappa^\circ : \square}$
\end{theorem}
\begin{proof}
	By induction on the sorting derivation.
\end{proof}
\begin{remark}
For any well-sorted kind $\kappa$ in \Fi,
$\kappa^\circ$ is a kind in \Fw.
\end{remark}

\begin{theorem}[index erasure on well-formed type level contexts]
\label{thm:ierasetyctx}
\[ \inference{|- \Delta}{|- \Delta^\circ} \]
\end{theorem}
\begin{proof}
	By induction on the derivation for well-formed type level context
	and using Theorem \ref{thm:ierasesorting}.
\end{proof}
\begin{remark}
For any well-formed type level context $\Delta$ in \Fi,
$\Delta^\circ$ is a well-formed type level context in \Fw.
\end{remark}

\begin{theorem}[index erasure on kind equality]\label{thm:ierasekindeq}
$ \inference{|- \kappa=\kappa':\square}
	{|- \kappa^\circ=\kappa'^\circ:\square}
$
\end{theorem}
\begin{proof}
	By induction on the kind equality judgement.
\end{proof}
\begin{remark}
For any well-sorted kind equality $|- \kappa=\kappa':\square$ in \Fi,
$|- \kappa^\circ=\kappa'^\circ:\square$ is a well-sorted kind equality in \Fw.
\end{remark}

The three theorems above on kinds are rather simple to prove since there is
no need to consider mutual recursion in the definition of kinds due to
the erasure operation on kinds. Recall that the erasure operation on kinds
discards the type ($A$) appearing in the index arrow type ($A -> \kappa$).
So, there is no need to consider the types appearing in kinds
and the index terms appearing in those types, after the erasure.\\

\begin{theorem}[index erasure on well-kinded type constructors]
\label{thm:ierasekinding}
\[ \inference{|- \Delta & \Delta |- F : \kappa}
		{\Delta^\circ |- F^\circ : \kappa^\circ}
\]
\end{theorem}
\begin{proof}
	By induction on the kinding derivation.
\begin{itemize}
\item[case] ($Var$)
	Use Theorem \ref{thm:ierasetyctx}.

\item[case] ($Conv$)
	By induction and using Theorem \ref{thm:ierasekindeq}.

\item[case] ($\lambda$)
	By induction and using Theorem \ref{thm:ierasesorting}.

\item[case] ($@$)
	By induction.

\item[case] ($\lambda i$)
	We need to show that
	$\Delta^\circ |- (\lambda i^A.F)^\circ : (A -> \kappa)^\circ$,
	which simplifies to $\Delta^\circ |- F^\circ : \kappa^\circ$
	by Definition \ref{def:ierase}.

	By induction, we know that
	$(\Delta,i^A)^\circ |- F^\circ : \kappa^\circ $,
	which simplifies $\Delta^\circ |- F^\circ : \kappa^\circ$
	by Definition \ref{def:ierase}.

\item[case] ($@ i$)
	We need to show that
	$\Delta^\circ |- (F\;\{s\})^\circ : \kappa^\circ$,
	which simplifies to $\Delta^\circ |- F^\circ : \kappa^\circ$
	by Definition \ref{def:ierase}.

	By induction we know that
	$\Delta^\circ |- F^\circ : (A -> \kappa)^\circ$,
	which simplifies to $\Delta^\circ |- F^\circ : \kappa^\circ$
	by Definition \ref{def:ierase}.

\item[case] ($->$)
	By induction.

\item[case] ($\forall$)
	We need to show that
	$\Delta^\circ |- (\forall X^\kappa.B)^\circ : *^\circ$,
	which simplifies to
	$\Delta^\circ |- \forall X^{\kappa^\circ}.B^\circ : *$
	by Definition \ref{def:ierase}.

	Using Theorem \ref{thm:ierasesorting}, we know that
	$|- \kappa^\circ : \square$.

	By induction we know that
	$(\Delta,X^\kappa)^\circ |- B^\circ : *^\circ$,
	which simplifies to
	$\Delta^\circ,X^{\kappa^\circ} |- B^\circ : *$
	by Definition \ref{def:ierase}.

	Using the kinding rule ($\forall$), we get exactly
	what we need to show:
	$\Delta^\circ |- \forall X^{\kappa^\circ}.B^\circ : *$.

\item[case] ($\forall i$)
	We need to show that
	$\Delta^\circ |- (\forall i^A.B)^\circ : *^\circ$,
	which simplifies to $\Delta^\circ |- B^\circ : *$
	by Definition \ref{def:ierase}.

	By induction we know that
	$(\Delta,i^A)^\circ |- B^\circ : *^\circ$,
	which simplifies $\Delta^\circ |- B^\circ : *$
	by Definition \ref{def:ierase}.
\end{itemize}\qedhere
\end{proof}

\begin{theorem}[index erasure on type constructor equality]
\label{thm:ierasetyconeq}
\[ \inference{\Delta |- F=F':\kappa}
		{\Delta^\circ |- F^\circ=F'^\circ:\kappa^\circ}
\]
\end{theorem}\begin{proof}
By induction on the derivation of type constructor equality.

Most of the cases are done by applying the induction hypothesis
and sometimes using Proposition \ref{prop:wfkind}.

The only interesting cases, which are worth elaborating on, are the
equality rules involving substitution.  There are two such rules.

\paragraph{}
  $\inference{\Delta,X^\kappa |- F:\kappa' & \Delta |- G:\kappa}
             {\Delta |- (\lambda X^\kappa.F)\,G = F[G/X]:\kappa'}$ \\

We need to show
$ \Delta^\circ |- ((\lambda X^\kappa.F)\,G)^\circ = (F[G/X])^\circ : \kappa'^\circ $,
which simplifies to 
$ \Delta^\circ |- (\lambda X^{\kappa^\circ}.F^\circ)\,G^\circ = (F[G/X])^\circ : \kappa'^\circ $
by Definition \ref{def:ierase}.

By induction, we know that $(\Delta,X^\kappa)^\circ |- F^\circ : \kappa'^\circ$,
which simplifies to $\Delta^\circ,X^{\kappa^\circ} |- F^\circ : \kappa'^\circ$
by Definition \ref{def:ierase}.

Using the kinding rule ($\lambda$), we get
$\Delta^\circ |- \lambda X^{\kappa^\circ}. F^\circ : \kappa^\circ -> \kappa'^\circ$.

Using the kinding rule ($@$), we get
$\Delta^\circ |- (\lambda X^{\kappa^\circ}. F^\circ) G^\circ : \kappa'^\circ$.

Using the very same equality rule of this case,\\ we get 
$\Delta^\circ |- (\lambda X^{\kappa^\circ} F^\circ) G^\circ =
F^\circ[G^\circ/X] : \kappa'^\circ$.

All we need to check is $(F[G/X])^\circ = F^\circ[G^\circ/X]$,
which is easy to see.

\paragraph{}
  $\inference{\Delta,i^A |- F:\kappa & \Delta;\cdot |- s:A}
             {\Delta |- (\lambda i^A.F)\,\{s\} = F[s/i]:\kappa}$ \\

By induction we know that $\Delta^\circ |- F^\circ : \kappa^\circ$.

The erasure of the left hand side of the equality is\\
$((\lambda i^A.F)\,\{s\})^\circ = (\lambda i^A.F)^\circ = F^\circ$.

All we need to show is $(F[s/i])^\circ = F^\circ$, which is obvious
since index variables can only occur in index terms and index terms
are always erased. Recall the index erasure over type constructors in
Definition \ref{def:ierase}; in particular, $(\lambda i^A.F)^\circ=F^\circ$,
$(F\{s\})^\circ=F^\circ$, and $(\forall i^A.B)^\circ=B^\circ$.
\end{proof}
\begin{remark}
For any well-kinded type constructor equality $\Delta |- F=F':\kappa$ in \Fi,
$\Delta^\circ|- F^\circ=F'^\circ:\kappa^\circ$ is
a well-kinded type constructor equality in \Fw.
\end{remark}

The proofs for the two theorems above on type constructors need not consider
mutual recursion in the definition of type constructors due to
the erasure operation. Recall that the erasure operation on type constructors
discards the index term ($s$) appearing in the index application $(F\;\{s\})$.
So, there is no need to consider the index terms appearing in the types after
the erasure.

\begin{theorem}[index erasure on well-formed term level contexts]
\label{thm:ierasetmctx}
\[ \inference{\Delta |- \Gamma}{\Delta^\circ |- \Gamma^\circ} \]
\end{theorem}
\begin{proof}
By induction on $\Gamma$.
\begin{itemize}
\item[case] ($\Gamma=\cdot$) It trivially holds.
\item[case] ($\Gamma = \Gamma',x:A$),
we know that  $\Delta |- \Gamma'$ and $\Delta |- A:*$
by the well-formedness rules
and that $\Delta^\circ |- \Gamma'^\circ$ by induction.

From $\Delta |- A:*$, we know that $\Delta^\circ |- A^\circ :*$
by Theorem \ref{thm:ierasekinding}.

We know that $\Delta^\circ |- \Gamma'^\circ,x:A^\circ$
from $\Delta^\circ |- \Gamma'^\circ$ and $\Delta^\circ |- A^\circ :*$
by the well-formedness rules.

Since $\Gamma'^\circ,x:A^\circ = (\Gamma',x:A)^\circ = \Gamma^\circ$
by definition, we know that $\Delta^\circ |- \Gamma^\circ$.
\end{itemize}\vspace*{-10pt}
\end{proof}

\begin{theorem}[index erasure on index-free well-typed terms]
\label{thm:ierasetypingifree}
\[ \inference{ \Delta |- \Gamma & \Delta;\Gamma |- t : A}
		{\Delta^\circ;\Gamma^\circ |- t : A^\circ}
		{\enspace(\dom(\Delta)\cap\FV(t) = \emptyset)}
\]
\end{theorem}
\begin{proof} By induction on the typing derivation.
	Interesting cases are the index related rules ($:i$), ($\forall Ii$),
	and ($\forall Ei$). Proofs for the other cases are straightforward
	by induction and applying other erasure theorems corresponding to
	the judgment forms.
\begin{itemize}
\item[case] ($:$)
	By Theorem \ref{thm:ierasetmctx}, we know that
	$\Delta^\circ|- \Gamma^\circ$ when $\Delta|- \Gamma$.
	By definition of erasure on term-level context, we know that
	$(x:A^\circ) \in \Gamma^\circ$ when $(x:A) \in \Gamma$.
\item[case] ($:i$)
	Vacuously true since $t$ does not contain any index variables
        (\ie, $\dom(\Delta)\cap\FV(t) = \emptyset$).
\item[case] ($->$$I$)
	By Theorem \ref{thm:ierasekinding}, we know that $\cdot |- A^\circ:*$.
	By induction, we know that
	$\Delta^\circ;\Gamma^\circ,x:A^\circ |- t^\circ : B^\circ$.
	Applying the ($->$$I$) rule to what we know, we have
	$\Delta^\circ;\Gamma^\circ |- \l x.t^\circ : A^\circ -> B^\circ$.
\item[case] ($->$$E$)
	Straightforward by induction.
\item[case] ($\forall I$)
	By Theorem \ref{thm:ierasesorting}, we know that
	$|- \kappa^\circ:\square$.
	By induction, we know that
	$\Delta^\circ,X^{\kappa^\circ};\Gamma^\circ |- t : B^\circ$.
	Applying the ($\forall I$) rule to what we know, we have
	$\Delta^\circ;\Gamma^\circ |- t : \forall X^{\kappa^\circ}.B^\circ$.
\item[case] ($\forall E$)
	By induction, we know that
	$\Delta^\circ;\Gamma^\circ |- t : \forall X^{\kappa^\circ}.B^\circ$.
	By Theorem \ref{thm:ierasekinding}, we know that
	$\Delta^\circ |- G^\circ : \kappa^\circ$.
	Applying the ($\forall E$) rule, we have
	$\Delta^\circ;\Gamma^\circ |- t : B^\circ[G^\circ / X]$.
\item[case] ($\forall Ii$)
	By Theorem \ref{thm:ierasekinding}, we know that $\cdot |- A^\circ:*$.
	By induction, we know that $\Delta^\circ;\Gamma^\circ |- t : B^\circ$,
	which is what we want since $(\forall i^A.B)^\circ = B^\circ$.
\item[case] ($\forall Ei$)
	By induction, we know that $\Delta^\circ;\Gamma^\circ |- t : B^\circ$,
	which is what we want since $(B[s/i])^\circ = B^\circ$.
\item[case] ($=$)
	By Theorem \ref{thm:ierasetyconeq} and induction.
\end{itemize}\qedhere
\end{proof}

\begin{example}\label{PathologicalExampleContinued}
The theorem yields that the pathological type~$\mathtt P_A$
of~Example~\ref{PathologicalExample} is not inhabited, as it is impossible
to have both $t:\mathtt P_A$ and $t:(\mathtt P_A)^\circ=\mathtt{Unit}$.
It follows as a corollary that the implication of
Theorem~\ref{thm:ierasetypingifree} does not admit a converse.

In this context for $A=\mathtt{Void}$, note that even though one has
%the open typing 
$i^\mathtt{Void};\cdot\vdash\lambda x.\,i:\forall
j^{\mathtt{Void}}.\,\forall X^{\mathtt{Void}\to*}.\, X\s i\to X\s j$, 
this open term %this derivation 
cannot be closed by rule~$(\forall Ii)$ because of its side
condition.  This is in stark contrast to what is possible in calculi with
full type dependency. In System \Fi, the index variables
in type level context~$\Delta$ cannot appear dynamically at term level.
Conversely, term variables in the term level context~$\Gamma$ cannot be
used for instantiation of index polymorphic types (rule $(\forall Ei)$).

%% Similar considerations to the above show that $\LEq_A$ is not symmetric,
%% in that the type {\small\rm(Symmetric)} in~\S\ref{Leibniz}
\end{example}

We introduce an index variable selection meta-operation that selects all
the index variable bindings from the type level context.

\begin{definition}[index variable selection]
\[ \cdot^\bullet = \cdot \qquad
	(\Delta,X^\kappa)^\bullet = \Delta^\bullet \qquad
	(\Delta,i^A)^\bullet = \Delta^\bullet,i:A
\]
\end{definition}

\begin{theorem}[index erasure on well-formed term level contexts
		prepended by index variable selection]
\label{thm:ierasetmctxivs}
\[ \inference{\Delta |- \Gamma}{\Delta^\circ |- (\Delta^\bullet,\Gamma)^\circ}
\]
\end{theorem}
\begin{proof}
Straightforward by Theorem \ref{thm:ierasetmctx} and the typing rule ($:i$).
\end{proof}

The following result is the appropriate version of
Theorem~\ref{thm:ierasetypingifree} without the side condition therein.

\begin{theorem}[index erasure on well-typed terms]
\label{thm:ierasetypingall}
\[ \inference{\Delta |- \Gamma & \Delta;\Gamma |- t : A}
		{\Delta^\circ;(\Delta^\bullet,\Gamma)^\circ |- t : A^\circ}
\]
\end{theorem}
\begin{proof}
	The proof is almost the same as that of
	Theorem~\ref{thm:ierasetypingifree}, except for the ($:i$) case.
	The proof for the rule~($:i$) case is easy
	since $(i:A) \in \Delta^\bullet$ when $i^A \in \Delta$ by definition of
	the index variable selection operation. The indices from $\Delta$
	being prepended to $\Gamma$ do not affect the proof for the other cases.
\end{proof}

%% \begin{theorem}[index erasure on term equality]
%% \[ \inference{\Delta;\Gamma |- t=t':A}
%%  	{\Delta^\circ;\Gamma^\circ |- t=t':A^\circ}
%% \]
%% \end{theorem}

\subsection{Strong normalization and logical consistency}
\label{ssec:fi:sn}
\index{strong normalization!System Fi@System \Fi}
Strong normalization is a corollary of the erasure property since we know that
System~\Fw\ is strongly normalizing.

Logical consistency is immediate since
System~\Fi\ is a strict subset of the \emph{restricted implicit calculus}
\cite{Miquel00}, which is in turn a restriction of ICC~\cite{Miquel01}.
Subject reduction is also immediate for the same reason.
%% \marginpar{\framebox{\bf\em State these results formally in a theorem?}}

We can also give a more direct proof of logical consistency by shoing that
the void type $\forall X^{*}.X$ is uninhabited in \Fi. By type erasure
no more terms inhabit \Fi-types than the corresponding \Fw-types.
Since we already know that the void type $\forall X^{*}.X$
is uninhabited in \Fw, it must be the case that the void type
is uninhabited in \Fi.

\begin{comment}
\subsection{No \texttt{Void} type instantiation from dynamic values}
\label{NoVoid}

There is an interesting difference between \Fi\ and a Curry-style
dependent calculus with implicit arguments such as ICC, regarding
the instantiation of uninhabited type. Consider the 
instantiation rule\marginpar{\framebox{\bf\em Instantiation!?}}
of ICC, shown below:
\[
\inference{\Gamma,x:A |- t : B}{\Gamma |- t : \forall x^A.B }~(x\notin\FV(t))
\]
When $A=\mathtt{Void}$ and $B=\forall i^\mathtt{Void}.\mathtt{NeverEverVoid}\{i\}$,
we can instantiate $i$ with $y$, according to the rule above,
provided that $(y:\mathtt{Void})\in\Gamma$. Note that, in ICC, it is possible to
instantiate a universally quantified term variable $x$ of an uninhabited type
from a possibly dynamic term $y$.

\marginpar{\framebox{\bf\em Sorry, I don't quite understand.  Needs to be
		improved.}}

In System~\Fi, one cannot instantiate $B$ with any of the term variables since
index instantiation cannot refer to the term-level context~($\Gamma$)
but only refers to the type-level context~($\Delta$) --
recall the ($\forall E i$) rule in Figure \ref{fig:fi}.
%\[
%\inference[($\forall E i$)]
%{ \Delta;\Gamma |- t : \forall i^A.B & \Delta;\cdot |- s:A }
%	                           {\Delta;\Gamma |- t : B[s/i]}
%\]
Note that it is still possible to instantiate uninhabited type from
index variables introduced at type level~(\ie, when $j^\mathtt{Void}\in\Delta$).
%%%However, such variables are only introduced within a pathological
%%%type constructor definition, as in Examples~\ref{PathologicalExample}
%%%and~\ref{PathologicalExampleContinued}.
\texttt{Void} type instantiation is localized inside type constructor
definition.  It is assured that function definitions at term level will
never cause \texttt{Void} type instantiation, even when some of the
function arguments have uninhabited type.



\begin{proposition}[anti-dependency on arrow kinds]
\[ \inference{ |- \Delta,X^\kappa
             & \Delta,X^\kappa |- F : \kappa' }
             { X\notin\FV(\kappa') }
\]
\end{proposition}
\begin{proof}
	By Proposition \ref{prop:wfkind}, $|- \kappa'$.
	Note that $|- \kappa'$ does not involve any type level context.

	Therefore, $X$ cannot appear free in $\kappa'$.
\end{proof}

\begin{proposition}[anti-dependency on indexed arrow kinds]
\[ \inference{ |- \Delta,i^A
             & \Delta,i^A |- F : \kappa }
             { i\notin\FV(\kappa) }
\]
\end{proposition}
\begin{proof}
	By Proposition \ref{prop:wfkind}, $|- \kappa'$.
	Note that $|- \kappa'$ does not involve any type level context.
	Therefore, $i$ cannot appear free in $\kappa'$.
\end{proof}

\begin{proposition}[anti-dependency on arrow types]
\[ \inference{ \Delta |- \Gamma,x:A
             & \Delta;\Gamma,x:A |- t : B }
             { x\notin\FV(B) }
\]
\end{proposition}
\begin{proof}
	By Proposition \ref{prop:wftype}, $\Delta |- B:*$.
	Note that $\Delta |- \kappa'$ does not involve any term level context.
	Therefore, $x$ cannot appear free in $B$.
\end{proof}


\begin{remark} Our system is more strong??? than anti-dependency on arrow types
TODO
\end{remark}

\end{comment}

 %% metatheory

%% \input{fi_relwork} related work

%% summary

