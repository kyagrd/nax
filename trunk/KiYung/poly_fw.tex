\section{System \Fw} \label{sec:fw}
System \Fw\ extends the type syntax of System \F\ with lambda types and
application types (see Figure \ref{fig:fw}). Lambda types ($\l X^\kappa.F$)
and application types ($F\;G$) at type level are analogous to lambda terms
and applications at term level. Type constructors are like functions, but
at type level. Type constructors are categorized by kinds, just as termes
are categorized by types. A type constructor of kind $\kappa -> \kappa'$
expects another type constructor of kind $\kappa$ as an argument to produce
yet another type constructor of kind $\kappa'$, just as a function of type
$A -> B$ expects another term of type $A$ as an argument to produce yet another
term of type $B$. Types of the star kind ($*$) are type constructors that
do not expect any arguments. Type constructors that expect an argument
are of arrow kinds ($\kappa -> \kappa'$).

We can think of System \F\ as a restriction of System \Fw\ where we only
allow types of the star kind ($*$). So, all the type variables appearing in
well-kinded types in System \F\ are of the star kind. Since there exist only
one kind ($*$) in System \F, the kinding rules of System \F\ only need to make
sure that type variables are bounded (\ie, member of $\Delta$). 
Since the kind structure of System \Fw\ is richer than System \F, we need to
keep track of the kind of the type variables in the kinding context ($\Delta$).
So, the kinding context is extended by a type variable annotated by its kind
($X^\kappa$). The kinding rules of System \Fw\ keep track of the kinds of
type constructors as well as making sure that the type variables are bounded.
The kinding rules \rulename{TVar}, \rulename{TArr}, \rulename{TAll}
are similar to the kinding rules of System \F. The kinding rules
\rulename{TLam} and \rulename{TApp} states when lambda types and
application types are well-kinded.

Typing rules of System \Fw\ are almost identical to the typing rules of
System \F, except for one new rule \rulename{Conv}. The \rulename{Conv} rule
supports conversion between equivalent types beyond $\alpha$-equivalence
(\ie, up to change of bound type variable names).
In STLC, types are equal when they are syntactically identical.
In System \F, types are equal when they are $\alpha$-equivalent. For example,
$\forall X.X$ and $\forall X'.X'$ are considered to be same types in System \F.
In System \Fw, we expect richer notion of equality, such as $\beta$-equivalence,
since the type syntax of System \Fw\ is analogous to a STLC at type level.
For instance, we want $(\lambda X^{*}.X) A = A$. TODO

%% syntax directed formalism???
%% http://pauillac.inria.fr/~herbelin/talks/cic.ps
%% there is a slide the refers to other paper and say that CIC is okay
%% since CIC is full. Is Fw also?

The complete syntax, kinding rules, and typing rules of System \Fw
are illustrated in Figure \ref{fig:fw}. The left column describes
the Church-Church-style System \F\ and the right column describes
the Curry-Church-style System \F. Since lambda types exist at type level
in System \Fw, we also have a choice of either Church style (kind annotations
on lambda types) or Curry style (no kind annotations on lambda types) for
the type syntax. The Church-Church-style System \Fw\ is a version System \Fw\ 
with the Church-style term syntax and the Church-style type syntax.
The Curry-Church-style System \Fw\ is a version of System \Fw\ 
with the Curry-style term syntax and the Church-style type syntax.
Another version of System \Fw, the Curry-Curry-style System \Fw,
with Curry-style term syntax and Curry style type syntax is described
in the left column of Figure \ref{fig:fw2}.

The reduction rules of System \Fw\ (Figure \ref{fig:redfw}) are
almost identical to the reduction rules of System \F\ since the term syntax of
System \Fw\ is almost identical to the term syntax of System \F.
The Church-style term syntax only differs from the term syntax of System \F\ 
that there is a kind annotation on the type variable appearing in
the type abstraction ($\L X^\kappa.t$). The Church-style term syntax is
exactly the same as the term syntax of System \F.

\begin{figure}
\begin{singlespace}
\begin{minipage}{.46\textwidth}
	\begin{center}Church-Church-style\end{center}\vspace*{-1em}
\def\baselinestretch{0}
\small
\begin{align*}
\textbf{term syntax} \\
t,s ::= &~ x               & \text{variable}    \\
      | &~ \l(x:A) . t     & \text{abstraction} \\
      | &~ t ~ s           & \text{application} \\
      | &~ \L X^\kappa . t & \text{type abstraction} \\
      | &~ t [G]           & \text{type application} \\
\textbf{type syntax} \\
F,G,A,B ::= &~ X                  & \text{variable type} \\
          | &~ A -> B             & \text{arrow type} \\
          | &~ \forall X^\kappa.B & \text{forall type}   \\
          | &~ \l X^\kappa.F      & \text{lambda type}   \\
          | &~ F ~ G              & \text{application type}   \\
\textbf{kind syntax} \\
\kappa ::= &~ \kappa -> \kappa' & \text{arrow kind} \\
         | &~ *                 & \text{star kind}   \\
\end{align*}
\[ \textbf{kinding rules} \quad \framebox{$ \Delta |- F:\kappa $} \]\vspace*{-1em}
\begin{align*}
& \inference[\sc TVar]{X^\kappa \in \Delta}{\Delta |- X:\kappa} \\
& \inference[\sc TArr]{\Delta |- A:* & \Delta |- B:*}{\Delta |- A -> B:*} \\
& \inference[\sc TAll]{\Delta,X^\kappa |- B:*}
		      {\Delta |- \forall X^\kappa.B:*} \\
& \inference[\sc TLam]{\Delta,X^\kappa |- F:\kappa'}
		      {\Delta |- \l X^\kappa.F:\kappa -> \kappa'} \\
& \inference[\sc TApp]{\Delta |- F : \kappa -> \kappa' & \Delta |- G : \kappa}
		      {\Delta |- F ~ G : \kappa'} \\
\end{align*}
\[ \textbf{typing rules} \quad \framebox{$ \Delta;\Gamma |- t : A $ } \]
\vspace*{-1em}
\begin{align*}
& \inference[\sc Var]{x:A \in \Gamma}{\Delta;\Gamma |- x:A} \\
& \inference[\sc Abs]{\Delta |- A:* & \Delta;\Gamma,x:A |- t : B}
		     {\Delta;\Gamma |- \l(x:A).t : A -> B} \\
& \inference[\sc App]{\Delta;\Gamma |- t : A -> B & \Delta;\Gamma |- s : A}
		     {\Delta;\Gamma |- t~s : B} \\
& \inference[\sc TyAbs]{\Delta,X^\kappa;\Gamma |- t : B}
		       {\Delta;\Gamma |- \L X^\kappa.t : \forall X^\kappa.B} \\
& \inference[\sc TyApp]{\Delta;\Gamma |- t : \forall X^\kappa.B & \Delta |- G:\kappa}
		       {\Delta;\Gamma |- t[G] : B[G/X]} \\
& \inference[\sc Conv]{\Delta;\Gamma |- t : A & \Delta |- A = A' : *}
		      {\Delta;\Gamma |- t : A'}
\end{align*}
\end{minipage}
\begin{minipage}{.46\textwidth}
	\begin{center}Curry-Church-style\end{center}\vspace*{-1em}
\def\baselinestretch{0}
\small
\begin{align*}
\textbf{term syntax} \\
t,s ::= &~ x           \\
      | &~ \l x    . t \\
      | &~ t ~ s       \\
      \phantom{|} &~ \\
      \phantom{|} &~ \\
\textbf{type syntax} \\
F,G,A,B ::= &~ X                  \\
          | &~ A -> B             \\
          | &~ \forall X^\kappa.B \\
          | &~ \l X^\kappa.F      \\
          | &~ F ~ G              \\
\textbf{kind syntax} \\
\kappa ::= &~ \kappa -> \kappa' \\
         | &~ *                 \\
\end{align*}
\[ \textbf{kinding rules} \quad \framebox{$ \Delta |- F:\kappa$}\]\vspace*{-1em}
\begin{align*}
& \inference[\sc TVar]{X^\kappa \in \Delta}{\Delta |- X:\kappa} \\
& \inference[\sc TArr]{\Delta |- A:* & \Delta |- B:*}{\Delta |- A -> B:*} \\
& \inference[\sc TAll]{\Delta,X^\kappa |- B:*}{\Delta |- \forall X^\kappa.B:*} \\
& \inference[\sc TLam]{\Delta,X^\kappa |- F:\kappa'}
		      {\Delta |- \l X^\kappa.F:\kappa -> \kappa'} \\
& \inference[\sc TApp]{\Delta |- F : \kappa -> \kappa' & \Delta |- G : \kappa}
		      {\Delta |- F ~ G : \kappa'} \\
\end{align*}
\[ \textbf{typing rules} \quad \framebox{$ \Delta;\Gamma |- t : A $ } \]
\vspace*{-1em}
\begin{align*}
& \inference[\sc Var]{x:A \in \Gamma}{\Delta;\Gamma |- x:A} \\
& \inference[\sc Abs]{\Delta |- A:* & \Delta;\Gamma,x:A |- t : B}
		     {\Delta;\Gamma |- \l x   .t : A -> B} \\
& \inference[\sc App]{\Delta;\Gamma |- t : A -> B & \Delta;\Gamma |- s : A}
		     {\Delta;\Gamma |- t~s : B} \\
& \inference[\sc TyAbs]{\Delta,X^\kappa;\Gamma |- t : B}
		       {\Delta;\Gamma |- t : \forall X^\kappa.B} \\
& \inference[\sc TyApp]{\Delta;\Gamma |- t : \forall X^\kappa.B & \Delta |- G:\kappa}
		       {\Delta;\Gamma |- t : B[G/X]} \\
& \inference[\sc Conv]{\Delta;\Gamma |- t : A & \Delta |- A = A' : *}
		      {\Delta;\Gamma |- t : A'}
\end{align*}
\end{minipage}
~\\
\caption{System \Fw\ in Church-Church-style and Curry-Church-style}
\label{fig:fw}
\end{singlespace}
\end{figure}

\begin{figure}
\begin{singlespace}
\begin{minipage}{.46\textwidth}
	\begin{center}Curry-Church-style\end{center}
\def\baselinestretch{0}
\small
\begin{align*}
\textbf{term syntax} \\
t,s ::= &~ x           & \text{variable}    \\
      | &~ \l x    . t & \text{abstraction} \\
      | &~ t ~ s       & \text{application} \\
\textbf{type syntax} \\
F,G,A,B ::= &~ X                  & \text{variable type}    \\
          | &~ A -> B             & \text{arrow type}       \\   
          | &~ \forall X^\kappa.B & \text{forall type}      \\
          | &~ \l X^\kappa.F      & \text{lambda type}      \\
          | &~ F ~ G              & \text{application type} \\
\textbf{kind syntax} \\
\kappa ::= &~ \kappa -> \kappa' & \text{arrow kind} \\
         | &~ *                 & \text{star kind}   \\
\end{align*}
\[ \textbf{kinding rules} \quad \framebox{$ \Delta |- F:\kappa $} \]\vspace*{-1em}
\begin{align*}
& \inference[\sc TVar]{X^\kappa \in \Delta}{\Delta |- X:\kappa} \\
& \inference[\sc TArr]{\Delta |- A:* & \Delta |- B:*}{\Delta |- A -> B:*} \\
& \inference[\sc TAll]{\Delta,X^\kappa |- B:*}{\Delta |- \forall X^\kappa.B:*} \\
& \inference[\sc TLam]{\Delta,X^\kappa |- F:\kappa'}
		      {\Delta |- \l X^\kappa.F:\kappa -> \kappa'} \\
& \inference[\sc TApp]{\Delta |- F : \kappa -> \kappa' & |- G : \kappa}
		      {\Delta |- F ~ G : \kappa'} \\
\end{align*}
\[ \textbf{typing rules} \quad \framebox{$ \Delta;\Gamma |- t : A $ } \]
\vspace*{-1em}
\begin{align*}
& \inference[\sc Var]{x:A \in \Gamma}{\Delta;\Gamma |- x:A} \\
& \inference[\sc Abs]{\Delta |- A:* & \Delta;\Gamma,x:A |- t : B}
		     {\Delta;\Gamma |- \l x   .t : A -> B} \\
& \inference[\sc App]{\Delta;\Gamma |- t : A -> B & \Delta;\Gamma |- s : A}
		     {\Delta;\Gamma |- t~s : B} \\
& \inference[\sc TyAbs]{\Delta,X^\kappa;\Gamma |- t : B}
		       {\Delta;\Gamma |- t : \forall X^\kappa.B} \\
& \inference[\sc TyApp]{\Delta;\Gamma |- t : \forall X^\kappa.B & \Delta |- G:\kappa}
		       {\Delta;\Gamma |- t : B[G/X]} \\
& \inference[\sc Conv]{\Delta;\Gamma |- t : A & \Delta |- A = A' : *}
		      {\Delta;\Gamma |- t : A'}
\end{align*}
\end{minipage}
\begin{minipage}{.46\textwidth}
	\begin{center}Curry-Curry-style\end{center}
\def\baselinestretch{0}
\small
\begin{align*}
\textbf{term syntax} \\
t,s ::= &~ x           \\
      | &~ \l x    . t \\
      | &~ t ~ s       \\
\textbf{type syntax} \\
F,G,A,B ::= &~ X                  \\
          | &~ A -> B             \\   
          | &~ \forall X.B \\
          | &~ \l X.F      \\
          | &~ F ~ G              \\
\textbf{kind syntax} \\
\kappa ::= &~ \kappa -> \kappa'  \\
         | &~ *                  \\
\end{align*}
\[ \textbf{kinding rules} \quad \framebox{$ \Delta |- F:\kappa $} \]\vspace*{-1em}
\begin{align*}
& \inference[\sc TVar]{X^\kappa \in \Delta}{\Delta |- X:\kappa} \\
& \inference[\sc TArr]{\Delta |- A:* & \Delta |- B:*}{\Delta |- A -> B:*} \\
& \inference[\sc TAll]{\Delta,X^\kappa |- B:*}{\Delta |- \forall X.B:*} \\
& \inference[\sc TLam]{\Delta,X^\kappa |- F:\kappa'}
		      {\Delta |- \l X.F:\kappa -> \kappa'} \\
& \inference[\sc TApp]{\Delta |- F : \kappa -> \kappa' & |- G : \kappa}
		      {\Delta |- F ~ G : \kappa'} \\
\end{align*}
\[ \textbf{typing rules} \quad \framebox{$ \Delta;\Gamma |- t : A $ } \]
\vspace*{-1em}
\begin{align*}
& \inference[\sc Var]{x:A \in \Gamma}{\Delta;\Gamma |- x:A} \\
& \inference[\sc Abs]{\Delta |- A:* & \Delta;\Gamma,x:A |- t : B}
		     {\Delta;\Gamma |- \l x   .t : A -> B} \\
& \inference[\sc App]{\Delta;\Gamma |- t : A -> B & \Delta;\Gamma |- s : A}
		     {\Delta;\Gamma |- t~s : B} \\
& \inference[\sc TyAbs]{\Delta,X^\kappa;\Gamma |- t : B}
		       {\Delta;\Gamma |- t : \forall X.B} \\
& \inference[\sc TyApp]{\Delta,X^\kappa;\Gamma |- t : B & \Delta |- G:\kappa}
		       {\Delta;\Gamma |- t : B[G/X]} \\
& \inference[\sc Conv]{\Delta;\Gamma |- t : A & \Delta |- A = A' : *}
		      {\Delta;\Gamma |- t : A'}
\end{align*}
\end{minipage}
~\\
\caption{System \Fw\ in Curry-Church-style and Curry-Curry-style}
\label{fig:fw2}
\end{singlespace}
\end{figure}

\begin{figure}
\paragraph{Reduction rules for the Church-$*$-style System \Fw}
\begin{align*}
& \inference[\sc RedBeta]{}{(\l(x:A).t)~s --> t[s/x]}
&&\inference[\sc RedTy]{}{(\L X   .t)[A] --> t[A/X]} \\
& \inference[\sc RedAbs]{t --> t'}{\l x   .t --> \l x   .t'}
&&\inference[\sc RedTyAbs]{t --> t'}{\L X^\kappa   .t --> \L X^\kappa   .t'} \\
& \inference[\sc RedApp1]{t --> t'}{t~s --> t'~s}
&&\inference[\sc RedTyApp]{t --> t'}{t[A] --> t'[A]} \\
& \inference[\sc RedApp2]{s --> s'}{t~s --> t~s'}
\end{align*}
\paragraph{Reduction rules for the Curry-$*$-style System \Fw}
\begin{align*}
& \inference[\sc RedBeta]{}{(\l x   .t)~s --> t[s/x]} \\
& \inference[\sc RedAbs]{t --> t'}{\l x   .t --> \l x   .t'} \\
& \inference[\sc RedApp1]{t --> t'}{t~s --> t'~s} \\
& \inference[\sc RedApp2]{s --> s'}{t~s --> t~s'}
\end{align*}
\caption{Reduction rules for System \Fw}
\label{fig:redfw}
\end{figure}

\begin{figure}
\[ \inference[\sc EqTVar]{X^\kappa \in \Delta}{\Delta |- X = X : \kappa}
\]

\caption{Type constructor equality rules for System \Fw}
\label{fig:eqtyfw}
\end{figure}

\paragraph{From Curry-Church-style to Curry-Curry-style}
We can also have a version of \Fw\ where type syntax is also in Curry style.
That is, the forall type ($\forall X.B$) and the lambda type ($\l X.B$)
do not have kind annotations. We call this version of \Fw, where both terms
and types are unannotated, the Curry-Curry-style \Fw.
In Figure \ref{fig:fw2}, the Curry-Curry-style \Fw\ (right) is laid out
side-by-side to the Curry-Church-style \Fw\ (left).

Since we changed the type syntax of forall types and lambda types in
the Curry-Curry-style \Fw, we need to adjust the we need to adjust the rules
involving forall types and lambda types. The rules we need to adjust are
the kinding rules \rulename{TAll} and \rulename{TLam}, and
the typing rules \rulename{TyAbs} and \rulename{TyApp}.

For the kinding rules \rulename{TAll} and \rulename{TLam}, all we need to do
is simply dropping the kind annotations appearing in the forall type and
the lambda type in each rule. In Figure \ref{fig:fw2}, you can see that
\rulename{TAll} and \rulename{TLam} in left (Curry-Church-style) and
right (Curry-Curry-style) are identical except for the kind annotations on
the forall type and the lambda type.

How should we adjust the typing rules \rulename{TyAbs} and \rulename{TyApp}
in the Curry-Curry-style \Fw? Our first attempt may be just dropping
the kind annotations (just as we did for the kinding rules) to adjust to
the changes to the type syntax as below:
\begin{align*}
& \inference[\sc TyAbs]{\Delta,X^\kappa;\Gamma |- t : B}
		       {\Delta;\Gamma |- t : \forall X.B}
& \inference[\sc TyApp]{\Delta;\Gamma |- t : \forall X.B & \Delta |- G:\kappa}
		       {\Delta;\Gamma |- t : B[G/X]}
\end{align*}
The \rulename{TyAbs} rule above is fine. However, the \rulename{TyApp} rule
above is problematic because it fails to require that $X$ must be of kind
$\kappa$ as well as $G$. To ensure that $X$ is of kind $\kappa$, we need to
adjust the \rulename{TyApp} rule as follows:
\begin{align*}
& \phantom{ \inference[\sc TyAbs]{\Delta,X^\kappa;\Gamma |- t : B}
                                 {\Delta;\Gamma |- t : \forall X.B} }
& \inference[\sc TyApp]{\Delta,X^\kappa;\Gamma |- t : B & \Delta |- G:\kappa}
			{\Delta;\Gamma |- t : B[G/X]}
\end{align*}
Note, the first premise ($\Delta,X^\kappa;\Gamma |- t : B$) of
the \rulename{TyApp} rule is exactly the same as the premise of
the \rulename{TyAbs} rule.

\paragraph{Church-Curry-style \Fw}
Although not illustrated in Figures \ref{fig:fw} and \ref{fig:fw2},
we can imagine yet another version of \Fw\ with annotated terms
and unannotated types -- namely, the Church-Curry-style \Fw.
I am not sure when one would need this style, though.



\subsection{Subject reduction and strong normalziation}\label{sec:f:srsn}
\subsection*{Subject reduction}
\begin{theorem}[subject reduction]
$\inference{\Delta;\Gamma |- t : A  & t --> t'}{\Delta;\Gamma |- t' : A}$
\end{theorem}


\subsection*{Strong normalization}
\begin{figure}
\begin{singlespace}
\begin{description}
\item[Interpretation of kinds] as pointwise generalization of $\SAT$
	\[ [| \kappa |] = \SAT_\kappa \]
\item[Interpretation of type constructors]
	as function spaces over saturated sets of normalizing terms
	whose free type variables are substituted according to
	the given type constructor valuation ($\xi$):
\begin{align*}
[| X |]_\xi      &= \xi(X) \\ 
[| A -> B |]_\xi &= [|A|]_\xi -> [|B|]_\xi \\
[| \forall X^\kappa . B |]_\xi
	&= \bigcap_{\mathcal{F}\in[|\kappa|]} [|B|]_{\xi[X\mapsto\mathcal{F}]}
		\qquad\qquad\qquad (X\notin\dom(\xi)) \\
[| \l X^\kappa . F |]_\xi
	&= \bbl(\mathcal{G} \in [|\kappa|]) . [|F|]_{\xi[X\mapsto\mathcal{G}]}
		\qquad\quad~ (X\notin\dom(\xi)) \\
[| F \; G |]_\xi &= [|F|]_\xi ( [|G|]_\xi )
\end{align*}
\item[Interpretation of kinding and typing contexts]
	as sets of type constructor valuations and term valuations
	($\xi$ and $\rho$):
\begin{align*}
[| \Delta        |] &= \{ \xi \in \dom(\Delta) -> \bigcup_{\kappa} [|\kappa|] \mid \xi(x)\in[|\Delta(x)|] ~\text{for all}~x\in\dom(\Delta) \} \\
[| \Delta;\Gamma |] &= \{ \xi;\rho \mid \xi\in[|\Delta|], \rho\in[|\Gamma|]_\xi \} \\
[| \Gamma        |]_\xi\ &= \{ \rho \in \dom(\Gamma) -> \SN \mid \rho(x)=[|\Gamma(x)|]_\xi ~\text{for all}~x\in\dom(\Gamma) \}
\end{align*}
\item[Interpretation of terms]
	as terms themselves whose free variables are substituted according to
	the given pair of type constructor and term valuations
	($\xi$;$\rho$):
\begin{align*}
[| x      |]_{\xi;\rho} &= \rho(x) \\
[| \l x.t |]_{\xi;\rho} &= \l x . [|t|]_{\xi;\rho} \qquad (x\notin\dom(\rho)) \\
[| t ~ s  |]_{\xi;\rho} &= [| t |]_{\xi;\rho} ~ [| s |]_{\xi;\rho}
\end{align*}
\end{description}
\caption[Interpretation of System \Fw\ for proving strong normalization]
	{Interpretation of type constructors, kinding and typing contexts,
		and terms of System \Fw\ for the proof of strong normalization}
\label{fig:interpFw}
\end{singlespace}
\end{figure}
For the proof of strong normalization of System \Fw, we use the same strategy
of interpreting types as saturated sets of normalizing terms as we did for
System \F. However, we need to generalize the interpretation of types to
the interpretation of type constructors. In the strong normalization proof
of System \F, we had a complete lattice $(\SAT,\subseteq)$.
We generalize this to an arbitrary kind $\kappa$ such that
$(\SAT_\kappa,\sqsubseteq_\kappa)$ from a complete lattice for any $\kappa$.
The set $\SAT_\kappa$ are generalization of $\SAT$ such that $\SAT_{*}=\SAT$
and $\SAT_{\kappa -> \kappa'} = \SAT_\kappa -> \SAT_{\kappa'}$
(\ie, functions from $\SAT_\kappa$ to $\SAT_\kappa'$).
The relation $\sqsubseteq_\kappa$ is a pointwise generalization of $\subseteq$
such that
\begin{align*}
\mathcal{A} \sqsubseteq_{*} \mathcal{A'} &= \mathcal{A} \subseteq \mathcal{A'}\\
\mathcal{F} \sqsubseteq_{\kappa -> \kappa'} \mathcal{F'} &=
	\mathcal{F}(\mathcal{G}) \sqsubseteq_{\kappa'} \mathcal{F'}(\mathcal{G})
	~\text{for all}~\mathcal{G}\in\SAT_\kappa
\end{align*}
Then, we can give interpretation of kind $\kappa$ as $\SAT_\kappa$.
That is, $[| \kappa |] = \SAT_\kappa$.
The interpretation of kinds, type constructors, contexts, and
terms of System \Fw\ are illustrated in Figure \ref{fig:interpFw}. 

We use the Curry-Church-style System \Fw\ to present the strong normalization
proof. It is more convenient to interpret terms in Curry style since
the Curry-style terms syntax is simpler than the Church-style term syntax.
It is more convenient to interpret type constructors in Curry style since
the kind annotation makes it clear how to interpret the bound type variable $X$
in forall types and lambda types (\ie, for $X^\kappa$ choose from $[|\kappa|]$).

The proof of strong normalization amounts to proving the following theorem:
\begin{theorem}[soundness of typing]
$ \inference{\Delta;\Gamma|- t:A & \xi;\rho \in [|\Delta;\Gamma|]}
	    {[|t|]_{\xi;\rho} \in [|A|]_\xi} $
\end{theorem}

\begin{corollary}[strong normalization]
	$\inference{\Delta;\Gamma |- t : A}{t \in \SN}$
\end{corollary}
\begin{proof}
We prove by induction on the typing derivation ($\Delta;\Gamma |- t:A$).

The cases for \rulename{Var}, \rulename{Abs}, and \rulename{App} are pretty
much the same as the strong normalization proof for System \F.
The cases for \rulename{TyAbs} and \rulename{TyApp} is almost the same
as the strong normalization proof for System \F, except that the type variable
can be of some kind $\kappa$ other than just the star kind.
We need to consider one more rule \rulename{Conv}, which is new in System \Fw.

\paragraph{Case (\rulename{TyAbs})}
We need to show that
$ \inference{\Delta;\Gamma |- t : \forall X.B & \xi;\rho\in[|\Delta;\Gamma|]}
	{[|t|]_{\xi;\rho} \in [|\forall X^\kappa.B|]_\xi} $

By induction, we know that
\[ \inference{\Delta,X^\kappa;\Gamma |- t:B & \xi';\rho'\in[|\Delta,X;\Gamma|]}
	{[|t|]_{\xi';\rho'} \in [|B|]_{\xi'}} ~
	(X\notin\FV(\Gamma))
\]
Since this holds for all $\xi';\rho' \in [|\Delta,X^\kappa;\Gamma|]$ where
$X\notin\FV(\Gamma)$, it also holds for particular subset such that
$\xi' = \xi[X\mapsto\mathcal{F}]$ and $\rho'=\rho$ for all $\mathcal{F}\in[|\kappa|]$.
That is,
\[ [|t|]_{\xi[X\mapsto\mathcal{F}];\rho} \in [|B|]_{\xi[X\mapsto\mathcal{F}]}
	\quad \text{for all}~\mathcal{F}\in[|\kappa|] \]
From $X\notin\FV(\Gamma)$, we know that
$[|t|]_{\xi[X\mapsto\mathcal{F}];\rho} = [|t|]_{\xi;\rho}$
because $\rho$ is independent of what $X$ maps to.
So, we know that
\[ [|t|]_{\xi;\rho} \in [|B|]_{\xi[X\mapsto\mathcal{F}]}
	\quad \text{for all}~\mathcal{F}\in[|\kappa|] \]
By set theoretic definition, this is exactly what we wanted to show:
\[ [|t|]_{\xi;\rho} \in
	\bigcap_{\mathcal{F}\in[|\kappa|]} [|B|]_{\xi[X\mapsto\mathcal{F}]}
	= [|\forall X^\kappa.B|]_\xi
\]

\paragraph{Case (\rulename{TyApp})}
We need to show that
$ \inference{\Delta;\Gamma |- t : B[G/X] & \xi;\rho\in[|\Delta;\Gamma|]}
	{[|t|]_{\xi;\rho} \in [|B[G/X]|]_\xi} $

By induction, we know that
$ \inference{\Delta;\Gamma |- t : \forall X^\kappa.B & \xi';\rho'\in[|\Delta;\Gamma|]}
	{[|t|]_{\xi';\rho'} \in [|\forall X^\kappa.B|]_{\xi'}}
$.

Since this holds for all $\xi';\rho' \in [|\Delta,\Gamma|]$,
it also holds for $\xi'=\xi$ and $\rho'=\rho$. Then, we are done:
$ [|t|]_{\xi;\rho} \in [|\forall X^\kappa.B|]_{\xi}
	= \bigcap_{\mathcal{G}\in[|\kappa|]} [|B|]_{\xi[X\mapsto\mathcal{G}]}
	\subseteq [|B|]_{\xi[X\mapsto[|G|]_\xi]} = [|B[G/X]|]_\xi
$.

\paragraph{Case (\rulename{Conv})}
We need to show that
$ \inference{\Delta;\Gamma |- t : A' & \xi;\rho\in[|\Delta;\Gamma|]}
	{[|t|]_{\xi;\rho} \in [|A'|]_\xi} $

By induction we know that 
$ \inference{\Delta;\Gamma |- t : A & \xi;\rho\in[|\Delta;\Gamma|]}
	{[|t|]_{\xi;\rho} \in [|A|]_\xi} $

If we can show that $[|A|]_\xi = [|A'|]_\xi$, we are done.

To show that $[|A|]_\xi = [|A'|]_\xi$, we use Lemma \ref{lem:fw:soundtyeq}.
\end{proof}

\begin{lemma}[soundness of type equality] \label{lem:fw:soundtyeq}
\[ \inference{\Delta |- F = F' : \kappa & \xi\in[|\Delta|]}
	{[|F|]_\xi = [|F'|]_\xi} \]
\end{lemma}
\begin{proof}
	TODO
\end{proof}

\begin{theorem}[soundness of kinding]
$ \inference{\Delta |- F : \kappa & \xi\in[|\Delta|]}{[|F|]_\xi\in[|\kappa|]} $
\end{theorem}
%% \begin{proof} TODO \end{proof}
TODO this seems to be true but do we need this anywhere???
