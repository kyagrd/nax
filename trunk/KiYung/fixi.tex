\chapter{System \Fixi}\label{ch:fixi}

It is known that primitive recursion is not embeddable in System \F\ since
induction is not derivable in second-order dependent type theory
\cite{Geuvers01}. For similar reasons, it is strongly believed that
primitive recursion is not embeddable in System \Fw. 
\citet{AbeMat04} designed \Fixw, an extension of \Fw\ with polarized kinds and
equi-recursive fixpoint type operator, in order to embed primitive recursion
over regular datatypes and type-indexed datatypes.
We present \Fixi, an extension of \Fixw\ with erasable term-indices,
to embed primitive recursion over term-indexed datatypes.

The organization of this chapter is analogous to the chapter on System \Fi\
(Chapter \ref{ch:fi}). Since the extensions from \Fixw\ to \Fixi\ is
basically the same as the extensions from \Fw\ to \Fi,
the description for \Fixi\ will be focused on the differences between \Fi.
Readers may refer back to Chapter \ref{ch:fi} for the details that are
in common with System~\Fi.

We describe syntax and typing rules (\S\ref{sec:fixi:def}),
illustrate embeddings of primitive recursion (\S\ref{sec:fixi:data})
and discuss embeddings of course-of-values primitive recursion
(\S\ref{sec:fixi:cv}) for term-indexed datatypes. Lastly, we
discuss the metatheory of \Fixi\ (\S\ref{sec:fixi:theory}).

\section{System \Fixi} \label{sec:fixi:def}
The syntax and rules of System~\Fi\ are described in
Figures \ref{fig:Fixi}, \ref{fig:Fixi2} and~\ref{fig:eqFixi}.
The extensions new to System~\Fixi, which are not original part of
either System~\Fw\ or System~\Fixw\ are highlighted by either
\dbox{dashed boxes} or \newFi{\text{grey boxes}}.

The extensions, which are not originally part of System~\Fixw, are highlighted
by \newFi{\text{grey boxes}}. Those extensions in grey boxes are to support
term indexing.  Eliding all the grey boxes from Figures~\ref{fig:Fixi},
\ref{fig:Fixi2} and~\ref{fig:eqFixi},
one obtains a version of System~\Fixw\ with typing contexts separated into
two parts.\footnote{The original description of \Fixw \cite{AbeMat04} has
one combined typing context.}

The extensions, which are not originally part of System \Fw\ but
present in System~\Fixw, are highlighted by \dbox{dashed boxes}.
Those extensions in dashed boxes are to support equi-recursive types. 
Eliding all the dashed boxes, as well as all the grey boxes,
from Figures~\ref{fig:Fixi}, \ref{fig:Fixi2} and~\ref{fig:eqFixi}, one obtains
the Curry-style System \Fw\ with typing contexts separated into two parts.


\begin{figure}\begin{singlespace}
	\small
\paragraph{Syntax:}
\begin{align*}
\!\!\!\!\!\!\!\!&\text{Sort}
 	& \square
	\\
\!\!\!\!\!\!\!\!&\text{Term Variables}
 	& x,i
\\
\!\!\!\!\!\!\!\!&\text{Type Constructor Variables}
 	& X
\\
\!\!\!\!\!\!\!\!&\text{\dbox{Polarities}}
	& \dbox{$p$} &~ ::= + \mid - \mid 0
\\
\!\!\!\!\!\!\!\!&\text{Kinds}
 	& \kappa		&~ ::= ~ *
				\mid \dbox{$p\kappa$} -> \kappa
				\mid \newFi{A -> \kappa}
\\
\!\!\!\!\!\!\!\!&\text{Type Constructors}
	& A,B,F,G		&~ ::= ~ X
				\mid A -> B
				\mid \dbox{$\fix F$} \\ &&& ~\quad
				\mid \lambda \dbox{$X^{p\kappa}$}.F
				\mid F\,G
				\mid \forall X^\kappa . B \\ &&& ~\quad
				\mid \newFi{\lambda i^A.F
				\mid F\,\{s\}
				\mid \forall i^A . B}
\\
\!\!\!\!\!\!\!\!&\text{Terms}
	& r,s,t			&~ ::= ~ x \mid \lambda x.t \mid r\;s
\\
\!\!\!\!\!\!\!\!&\text{Typing Contexts}
	& \Delta		&~ ::= ~ \cdot
				\mid \Delta,\dbox{$X^{p\kappa}$}
				\mid \newFi{\Delta, i^A} \\
&	& \Gamma		&~ ::= ~ \cdot
				\mid \Gamma, x : A
\end{align*}
\paragraph{Reduction:} \fbox{$t \rightsquigarrow t'$}
\[ 
   \inference{}{(\lambda x.t)\,s \rightsquigarrow t[s/x]}
 ~~~~
   \inference{t \rightsquigarrow t'}{\lambda x.t \rightsquigarrow \lambda x.t'}
 ~~~~
   \inference{r \rightsquigarrow r'}{r\;s \rightsquigarrow r'\;s}
 ~~~~
   \inference{s \rightsquigarrow s'}{r\;s \rightsquigarrow r\;s'}
\]
~\\
\end{singlespace}
\caption{Syntax and Reduction rules of \Fixi}
\label{fig:Fixi}
\end{figure}

\begin{figure}\begin{singlespace}\small
\paragraph{Well-formed typing contexts:}
\[ \fbox{$|- \Delta$}
 ~~~~
   \inference{}{|- \cdot}
 ~~~
   \inference{|- \Delta & |- \kappa:\square}{|- \Delta,\dbox{$X^{p\kappa}$}}
      \big( X\notin\dom(\Delta) \big)
 ~~~ \newFi{
   \inference{|- \Delta & \cdot |- A:*}{|- \Delta,i^A}
      \big( i\notin\dom(\Delta) \big) }
\]
$ \fbox{$\Delta |- \Gamma$}
 ~~~~
   \inference{|- \Delta}{\dbox{$0\Delta$} |- \cdot}
 ~~~~
   \inference{\Delta |- \Gamma & \Delta |- A:*}{
              \Delta |- \Gamma,x:A}
      \big( x\notin\dom(\Gamma) \big)
$ \vskip1ex ~
\paragraph{Sorting:} \fbox{$|- \kappa : \square$}
$ \quad
  \inference[($A$)]{}{|- *:\square}
 ~~~
   \inference[($R$)]{|- \kappa:\square & |- \kappa':\square}{
		     |- \dbox{$p\kappa$} -> \kappa' : \square}
 ~~~
   \newFi{
   \inference[($Ri$)]{\cdot |- A:* & |- \kappa:\square}{
                      |- A -> \kappa : \square} }
$
~\\
\paragraph{Kinding:} \fbox{$\Delta |- F : \kappa$}
$ \qquad
   \inference[($Var$)]{\dbox{$X^{p\kappa}$}\in\Delta & |- \Delta}{
 		       \Delta |- X : \kappa}
		\, \dbox{$(p\in \{+,0\})$} $
\[
   \inference[($->$)]{\dbox{$-\Delta$} |- A : * & \Delta |- B : *}{
                      \Delta |- A -> B : * }
 \qquad \qquad \dbox{
   \inference[($\fix$)]{\Delta |- F : +\kappa -> \kappa}{
		      \Delta |- \fix F : \kappa } }
\]
\[
  \inference[($\lambda$)]{ |- \kappa:\square
                         & \Delta,\dbox{$X^{p\kappa}$} |- F:\kappa'}{
  	\Delta |- \lambda \dbox{$X^{p\kappa}$}.F : \dbox{$p\kappa$} -> \kappa'}
 ~~~~ \qquad
 \newFi{
  \inference[($\lambda i$)]{\cdot |- A:* & \Delta,i^A |- F : \kappa}{
			    \Delta |- \lambda i^A.F : A->\kappa} }
\]
\[
  \inference[($@$)]{ \Delta |- F : \dbox{$p\kappa$} -> \kappa'
		    & \dbox{$p\Delta$} |- G : \kappa }{
                     \Delta |- F\,G : \kappa'}
 ~~~~
 \newFi{
   \inference[($@i$)]{ \Delta |- F : A -> \kappa
   		     & \dbox{$0\Delta$\!};\cdot |- s : A }{
		      \Delta |- F\,\{s\} : \kappa} }
\]
\[
   \inference[($\forall$)]{ |- \kappa:\square
   			  & \Delta,\dbox{$X^{0\kappa}$} |- B : *}{
                           \Delta |- \forall X^\kappa . B : *}
 ~~~~ \qquad ~\,
	\newFi{
   \inference[($\forall i$)]{\cdot |- A:* & \Delta, i^A |- B : *}{
                             \Delta |- \forall i^A . B : *} }
\]
\[ \newFi{
   \inference[($Conv$)]{ \Delta |- A : \kappa
                       & \Delta |- \kappa = \kappa' : \square }{
                        \Delta |- A : \kappa'} }
\]
~\\
\paragraph{Typing:} \fbox{$\Delta;\Gamma |- t : A$}
$ \qquad
 ~~~~
 \inference[($:$)]{(x:A) \in \Gamma & \Delta |- \Gamma}{
                   \Delta;\Gamma |- x:A}
 ~~~~ \newFi{
   \inference[($:i$)]{i^A \in \Delta & \Delta |- \Gamma}{
                      \Delta;\Gamma |- i:A} }
$
\[
   \inference[($->$$I$)]{\Delta |- A:* & \Delta;\Gamma,x:A |- t : B}{
                         \Delta;\Gamma |- \lambda x.t : A -> B}
 ~~~~ ~~~~
   \inference[($->$$E$)]{\Delta;\Gamma |- r : A -> B & \Delta;\Gamma |- s : A}{
                         \Delta;\Gamma |- r\;s : B}
\]
\[ \inference[($\forall I$)]{|- \kappa:\square
			    & \Delta,\dbox{$X^{0\kappa}$\!};\Gamma |- t : B}{
                             \Delta;\Gamma |- t : \forall X^\kappa.B}
			    (X\notin\FV(\Gamma))
 ~~~~ ~~~~
   \inference[($\forall E$)]{ \Delta;\Gamma |- t : \forall X^\kappa.B
                            & \Delta |- G:\kappa }{
                             \Delta;\Gamma |- t : B[G/X]}
\]
\[ \newFi{
   \inference[($\forall I i$)]{\cdot |- A:* & \Delta, i^A;\Gamma |- t : B}{
                               \Delta;\Gamma |- t : \forall i^A.B}
   \left(\begin{matrix}
		i\notin\FV(t),\\
		i\notin\FV(\Gamma)\end{matrix}\right)
 ~~~~
   \inference[($\forall E i$)]{ \Delta;\Gamma |- t : \forall i^A.B
                              & \Delta;\cdot |- s:A}{
                               \Delta;\Gamma |- t : B[s/i]} }
\]
\[ \inference[($=$)]{\Delta;\Gamma |- t : A & \Delta |- A = B : *}{
                     \Delta;\Gamma |- t : B}
\]
\end{singlespace}
\caption{Sorting, Kinding, and Typing rules of \Fixi}
\label{fig:Fixi2}
\end{figure}

\begin{figure}\begin{singlespace}\small
\paragraph{Kind equality:} \fbox{$|- \kappa=\kappa' : \square$}
$ \quad
 ~~~~
   \inference{}{|- * = *:\square} $
\[
   \inference{ |- \kappa_1 = \kappa_1' : \square
             & |- \kappa_2 = \kappa_2' : \square }{
   |- \dbox{$p\kappa_1$}-> \kappa_2 = \dbox{$p\kappa_1'$}-> \kappa_2' : \square}
 ~~~~ \newFi{
   \inference{\cdot |- A=A':* & |- \kappa=\kappa':\square}{
              |- A -> \kappa = A' -> \kappa' : \square} }
\]
\[ \inference{|- \kappa=\kappa':\square}{
              |- \kappa'=\kappa:\square}
 ~~~~
   \inference{ |- \kappa =\kappa' :\square
             & |- \kappa'=\kappa'':\square}{
              |- \kappa=\kappa'':\square}
\]
~
\paragraph{Type constructor equality:} \fbox{$\Delta |- F = F' : \kappa$}
$\qquad \dbox{$
  \inference{\Delta|- F:+\kappa-> \kappa}{\Delta|- \fix F=F(\fix F):\kappa}$ } $
\[
   \inference{ \Delta,\dbox{$X^{p\kappa}$} |- F:\kappa'
   	     & \dbox{$p\Delta$} |- G:\kappa}{
	      \Delta |- (\lambda X^{p\kappa}.F)\,G = F[G/X]:\kappa'}
 ~~~~ \newFi{
   \inference{ \Delta,i^A |- F:\kappa
             & \dbox{$0\Delta$\!};\cdot |- s:A}{
              \Delta |- (\lambda i^A.F)\,\{s\} = F[s/i]:\kappa} }
\]
\[ \inference{\Delta |- X:\kappa }{\Delta |- X=X:\kappa}
 ~~~~
   \inference{-\Delta |- A=A':* & \Delta |- B=B':*}{\Delta |- A-> B=A'-> B':*}
 ~~~~
   \inference{\Delta |- F=F' : +\kappa -> \kappa}{
	      \Delta |- \fix F = \fix F' : \kappa}
\]
\[
   \inference{ |- \kappa:\square
   	     & \Delta,\dbox{$X^{p\kappa}$} |- F=F' : \kappa'}{
   	\Delta |- \lambda \dbox{$X^{p\kappa}$}.F
		= \lambda \dbox{$X^{p\kappa}$}.F': \dbox{$\kappa$} -> \kappa'}
 ~~~~ ~
 \newFi{
   \inference{\cdot |- A:* & \Delta,i^A |- F=F' : \kappa}{
	      \Delta |- \lambda i^A.F=\lambda i^A.F' : A -> \kappa} }
\]
\[\!\!\!\!\!\!\!\!\!\!\!\!
   \inference{ \Delta |- F=F' : \dbox{$p\kappa$} -> \kappa'
   	     & \dbox{$p\Delta$} |- G=G':\kappa}{
              \Delta |- F\,G = F'\,G' : \kappa'}
 ~~~~
 \newFi{
   \inference{ \Delta |- F=F': A -> \kappa
             & \dbox{$0\Delta$\!};\cdot |- s=s':A}{
	      \Delta |- F\,\{s\} = F'\,\{s'\} : \kappa} }
\]
\[
   \inference{ |- \kappa:\square
   	     & \Delta,\dbox{$X^{0\kappa}$} |- B=B':*}{
              \Delta |- \forall X^\kappa.B=\forall X^\kappa.B':*}
 ~~~~ \quad
 \newFi{
   \inference{\cdot |- A:* & \Delta,i^A |- B=B':*}{
              \Delta |- \forall i^A.B=\forall i^A.B':*} }
\]
\[ \inference{\Delta |- F = F' : \kappa}{\Delta |- F' = F : \kappa}
 ~~~~
   \inference{\Delta |- F = F' : \kappa & \Delta |- F' = F'' : \kappa}{
              \Delta |- F = F'' : \kappa}
\]
~
\paragraph{Term equality:} \fbox{$\Delta;\Gamma |- t = t' : A$}
\[
   \inference{\Delta;\Gamma,x:A |- t:B & \Delta;\Gamma |- s:A}{
              \Delta;\Gamma |- (\lambda x.t)\,s=t[s/x] : B}
 ~~~~
   \inference{\Delta;\Gamma |- x:A}{\Delta;\Gamma |- x=x:A}
\]
\[ \inference{\Delta |- A:* & \Delta;\Gamma,x:A |- t=t':B}{
              \Delta;\Gamma |- \lambda x.t = \lambda x.t':B}
 ~~~~
   \inference{\Delta;\Gamma |- r=r':A-> B & \Delta;\Gamma |- s=s':A}{
              \Delta;\Gamma |- r\;s=r'\;s':B}
\]
\[ \inference{ |- \kappa:\square
	     & \Delta, \dbox{$X^{0\kappa}$\!};\Gamma |- t=t' : B}{
              \Delta;\Gamma |- t=t' : \forall X^\kappa.B}
	     (X\notin\FV(\Gamma))
 ~~~~ ~~~~
   \inference{ \Delta;\Gamma |- t=t' : \forall X^\kappa.B
             & \Delta |- G:\kappa }{
              \Delta;\Gamma |- t=t' : B[G/X]}
\]
\[ \newFi{
   \inference{\cdot |- A:* & \Delta, i^A;\Gamma |- t=t' : B}{
              \Delta;\Gamma |- t=t' : \forall i^A.B}
   \left(\begin{smallmatrix}
		i\notin\FV(t),\\
		i\notin\FV(t'),\\
		i\notin\FV(\Gamma)\end{smallmatrix}\right)
 ~~~~
   \inference{ \Delta;\Gamma |- t=t' : \forall i^A.B
             & \Delta;\cdot |- s:A}{
              \Delta;\Gamma |- t=t' : B[s/i]} }
\]
\[ \inference{\Delta;\Gamma |- t=t':A}{\Delta;\Gamma |- t'=t:A}
 ~~~~
   \inference{\Delta;\Gamma |- t=t':A & \Delta;\Gamma |- t'=t'':A}{
              \Delta;\Gamma |- t=t'':A}
\]
\end{singlespace}
\caption{Equality rules of \Fixi}
\label{fig:eqFixi}
\end{figure}

The grey-boxed extensions for term-indexing are essentially the same as
those grey-boxed extensions in System \Fi\ (\S\ref{sec:fi:fi}). So, we will
only focus our description on the dashed-box extensions regarding polarities
(\S\ref{ssec:fixi:def:polarity}) and equi-recursive types
(\S\ref{ssec:fixi:def:equirec}).

\subsection{Polarities} \label{ssec:fixi:def:polarity}
Polarities track how type constructor variables are used.
A polarity ($p$) is either covariant ($+$), contravariant ($-$), or
avariant ($0$). When a type variable is bound, its polarity is 
made explicit both at its binding site, and in the context.
The avariant polarity ($0$) means a variable can be used both
covariantly and contravariantly\footnote{the notation ``invariant'' is sometimes used
	(see \cite{AbeMat04}), but we think this notation is quite misleading,
	since $0$ means that the system \emph{does not care} about that
	variable's polarity,
	rather than maintaining some unchanging set of properties about the variable's polarities.}
We prefix a kind  by a polarity (\ie, $p\kappa$) to specify the variable's
kind and polarity. For example,
\[
\begin{array}{rlrl}
X_1^{+*},X_2^{-*} |- \!\!\! & X_2 -> X_1 : *
	& ~~~\text{justifies} & \l X_1^{+*}.\l X_2^{-*}.X_2 -> X_1 \\
X_1^{0*},X_2^{0*} |- \!\!\! & X_2 -> X_1 : *
	& ~~~\text{also justifies} & \l X_1^{0*}.\l X_2^{0*}.X_2 -> X_1 \\
X^{0*} |- \!\!\! & X\phantom{_.} -> X\phantom{_2} : *
	& ~~~\text{justifies} & \l X^{0*}.X -> X
\end{array}
\]
Note that we can replace $+$ and $-$ in the first example with $0$
as in the second example, since the variables of avariant polarity can be used
in any position, that is both in covariant position and contravariant position.
In the third example, the polarity of $X$ can be neither $+$ nor $-$, but must
must be $0$, since $X$ appears in both covariant and contravariant positions.

\paragraph{Syntax using polarized kinds:}
The kind syntax are polarized. That is, the domain kind ($\kappa$) of
an arrow kind ($p\kappa -> \kappa'$) must be prefixed by its polarity ($p$).
A type abstractions ($\l X^{p\kappa}.F$) in the type syntax are annotated by
polarity-prefixed kinds ($p\kappa$). Type constructor variables ($X$) bound
in the type-level contexts ($\Delta$) are annotated by polarity-prefixed kinds
($p\kappa$), likewise. Note the syntax for extending the type-level context
$\Delta,X^{p\kappa}$ in Figure~\ref{fig:Fixi}. The kinding rule ($\lambda$)
make use of all these three uses of polarized kinds -- in type abstractions,
in kind arrows, and in type-level contexts.

\paragraph{Polarity operation on type-level context ($p\Delta$):}
The kinding judgment $\Delta |- F : \kappa$ assumes that $F$ is in covariant
positions. This is why the ($Var$) rule requires the polarity of $X$ to be
either $+$ or $0$, but not $-$. To judge well-kindedness of type constructors
in contravariant positions (\eg, $A$ in $A -> B$), we should invert
the polarities of all the type constructor variables in the context.
This idea of inverting polarities in the context is captured by $-\Delta$
operation in the kinding rule ($->$). More generally, the well-kindedness $F$
expected to be used as $p$-polarity can be determined by the judgement
$p\Delta |- F : \kappa$, where $p\Delta$ operation is defined as:
\[
\begin{array}{lcl}
p~\cdot &=& \cdot \\
p(\Delta,X^{p'\kappa}) &=& p\Delta,X^{(pp')\kappa} \\
p(\Delta,i^A) &=& p\Delta,i^A
\end{array}
\quad\text{where $pp'$ is the usual sign product}~~
\begin{smallmatrix}
+ p' & = & p' \\
- +  & = & -  \\
- -  & = & +  \\
- 0  & = & 0  \\
0 p' & = & 0
\end{smallmatrix}
\]
Note the use of $p\Delta$ operation in the kinding rule ($@$) in order to
determine the well-kindedness of $G$ expected to be used as $p$-polarity
by the type constructor $F : p\kappa -> \kappa'$ being applied to $G$.

\paragraph{Where polarities are irrelevant (or, avariant):}
Polarities are irrelevant (or, avariant) for universally quantified variables,
indices, and in the typing rules. This is because the sole purpose of tracking 
polarities in \Fixi\ is to make sure that we only take the equi-recursive
fixpoint over covariant type constructors, as in the kinding rule ($\mu$).
Note that we can only take fixpoints over type constructors of covariant
arrow kinds whose domain and codomain coincides ($+\kappa -> \kappa$).
We can never take fixpoints over forall types (or, universal quantification)
and type constructor expecting an index because they are not of arrow kinds.
Forall types are always of kind $*$ and type constructors expecting an index
are of arrow kinds ($A -> \kappa$). So, we give universally quantified variables
avariant polarity ($X^{0\kappa}$ in the ($\forall$) rule) and nullify polarities
when type checking indices ($0\Delta$ in the ($@i$) rule). For similar reasons,
we assume that type-level contexts are nullified in the typing rules;
note $0\Delta$ in the well-formedness condition for $\Delta |- \Gamma$
in Figure~\ref{fig:Fixi2}. That is, we always type check under nullified
type-level context for all terms in general, as well as indices appearing
in type applications. As a result, the typing rules of \Fixi\ have no dashed-box
extensions except for $X^{0\kappa}$ in the generalization rule ($\forall$$I$)
where we introduce a universally quantified type constructor variable.

\subsection{The Equi-recursive type operator $\fix$}
\label{ssec:fixi:def:equirec}
\Fixi\ provides the equi-recursive type operator $\fix$.
The kinding rule ($\fix$) in Figure~\ref{fig:Fixi2} is similar to
the ($\mu$) rule of System \Fi\ (see Figure~\ref{fig:Fi2} in \S\ref{sec:fi:fi}),
but requires base structure $F$ to be covariant (or, positive), \ie,
$F : +\kappa -> \kappa$. This restriction on the polarity of $F$ is due to
the equi-recursive nature of $\fix$, \ie, $\fix F=F(\fix F)$, which is
described by the first type constructor equality rule inside a dashed box
in Figure~\ref{fig:eqFixi}. Recall that adding equi-recursive types
without restricting the polarity of base structure breaks strong normalization.
\KYA{TODO should reference intro chapter section which is not written yet}

Note that there are no explicit term syntax that guides the conversion between
$\fix F$ and $F(\fix F)$, unlike in iso-recursive systems where $\In$ and
$\mathsf{unIn}$ are term syntax that explicitly guide rolling
(from $\mu F$ to $F(\mu F)$) and unrolling (from $F(\mu F)$ to $\mu F$).
Since $\fix F=F(\fix F)$ is given definitionally (\ie, by the equality rule
definition), the type constructor conversion rule ($Conv$) can silently
roll (from $F(\fix F)$ to $\fix F$) and unroll (from $F(\fix F)$ to $\fix F$)
the recursive types, just as it can silently $\beta$-convert type constructors.

In the following section, we will review how iso-recursive type operator
$\mu_\kappa$ and its $\In_\kappa$ constructor, which is well-behaved
(\ie, strongly normalize) for base structures of arbitrary polarity,
can be embedded into \Fixi\ being defined in terms of the equi-recursive
type operator $\fix$, which is only well-behaved for covariant base structures.

 %% \label{sec:fixi:def}

\section{Embedding datatypes and primitive recursion}
\label{sec:fixi:data}
Embedding for primitive recursion over datatypes of arbitrary polarities into
\Fixi\ was discovered by \citet{AbeMat04}. We review these embeddings
in the context of \Fixi.

The embeddings of non-recursive datatypes in Figure~\ref{fig:fixiNonRecData}
are exactly the same as in \Fi\ (see \S\ref{sec:fi:data}), other than tracking
polarities of the type constructor variables. That is, we use the usual
impredicative encodings for non-recursive datatypes such as void, unit, pairs,
sums, and existential types. The examples in Figure \ref{fig:fixiNonRecData}
are mostly from \citet{AbeMat04}, except for the last example of $\exists_A$,
an existential type over term-indices of type $A$.
\begin{figure}
\begin{singlespace}
\begin{align*}
\bot &~\triangleq~ \forall X^{*}.X
	&:\;& *\\
\textsf{Unit} &~\triangleq~ \forall X.\lambda X^{0*}. X
	&:\;& * \\
\times &~\triangleq~
	\l X_1^{+*}.\l X_2^{+*}.\forall X^{*}.(X_1 -> X_2 -> X) -> X
	&:\;& +* -> +* -> * \\
+ &~\triangleq~
	\lambda X_1^{+*}.\l X_2^{+*}.\forall X^{*}.(X_1 -> X) -> (X_2 -> X) -> X
	&:\;& +* -> +* -> * \\
\exists_\kappa &~\triangleq~
	\l X_{\!F}^{0\kappa -> *}.\forall X^{*}.
		(\forall X_1^\kappa.X_{\!F}\,X_1 -> X) -> X
	&:\;& +(0\kappa -> *) -> * \\
\exists_A &~\triangleq~
	\l X_{\!F}^{A -> *}.\forall X^{*}.
	(\forall i^A.X_{\!F}\{i\} -> X) -> X
	&:\;& +(A -> *) -> *
\end{align*}
\caption{Embeddings of some well-known non-recursive datatypes in \Fixi}
\label{fig:fixiNonRecData}
\end{singlespace}
\end{figure}

Embedding recursive datatypes and their Mendler-style primitive recursion
amounts to embedding $\mu_\kappa$, $\In_\kappa$, and $\MPr_\kappa$ described
in \S\ref{sec:mpr}. Figure~\ref{fig:embedMPr} illustrates the embeddings
discovered by \citet{AbeMat04}, reformatted using our conventions (see
Figure~\ref{fig:mu} in \S\ref{ssec:embedTwoLevel}) and taking term-indices
into consideration. To confirm the correctness of these embeddings,
we should check that (1) the embeddings are well-kinded and well-typed
and that (2) the primitive recursion behaves well
(\ie, $\MPr_\kappa\;s\;(\In_\kappa) -->+ s\;\textit{id}\;(\MPr_\kappa\;s)\;t$).
From the term encodings of $\MPr_\kappa$ and $\In_\kappa$, it is obvious that
the reduction of primitive recursion behaves well. Thus, we only need to check
that $\mu_\kappa$ is well-kinded and $\MPr_\kappa$ and $\In_\kappa$ are
well-typed.
\afterpage{ %%%%%%%%%%%%%%%%%%%%%%% begin afterpage
\begin{landscape}
\begin{figure}
\begin{singlespace}
\begin{multline*} \text{notation:}\quad
   \boldsymbol{\l}\mathbb{I}^\kappa.F =
	\lambda I_1^{K_1}.\cdots.\lambda I_n^{K_n}.F \qquad
   \boldsymbol{\forall}\mathbb{I}^\kappa.B =
	\forall I_1^{K_1}.\cdots.\forall I_n^{K_n}.B \qquad
   F\mathbb{I} = F I_1 \cdots I_n \qquad
   F \stackrel{\kappa}{\pmb{\pmb{->}}} G =
	\boldsymbol{\forall}\mathbb{I}^\kappa.F\mathbb{I} -> G\mathbb{I} \\
\begin{array}{lll}
\text{where}
 	& \kappa = K_1 -> \cdots -> K_n -> * & \text{and} ~~~
 	\text{$I_i$ is an index variable ($i_i$) when $K_i$ is a type,}
 		\\
 	& \mathbb{I}\,=I_1,\;\dots\;\dots\;,\;I_n& \qquad~\qquad
		\text{a type constructor variable ($X_i$) otherwise
			(\ie, $K_n=p_i\kappa_i$).}
\end{array}
\end{multline*} ~ \vspace*{-5pt}
\hrule  \vspace*{-2pt}
\begin{align*}
\mu_\kappa &\;:~ 0(0\kappa -> \kappa) -> \kappa \\
\mu_\kappa &\triangleq
\l X_{\!F}^{0(0\kappa -> \kappa)}.\fix(\Phi_\kappa\,X_{\!F})\\
\Phi_\kappa &\;:~ 0(0\kappa -> \kappa) -> +\kappa -> \kappa \\
\Phi_\kappa &\triangleq \l X_{\!F}^{0(0\kappa -> \kappa)}.
\l X_c^{+\kappa}.\boldsymbol{\l}\mathbb{I}^\kappa.
\forall X^\kappa.
(\forall X_r^\kappa. (X_r \karrow{\kappa} X_c)
		-> (X_r \karrow{\kappa} X)
		-> (X_{\!F}\,X_r \karrow{\kappa} X) ) -> X\,\mathbb{I}\\
~\\
\MPr_\kappa &\;:~
	\forall X_{\!F}^{0(0\kappa-> \kappa)}.\forall X^\kappa.
	(\forall {X_r}^{\!\!\kappa}.
	 (X_r \karrow{\kappa} \mu_\kappa X_{\!F}) ->
	 (X_r \karrow{\kappa} X) ->
	 (X_{\!F}\,X_r \karrow{\kappa} X) ) ->
	 (\mu_\kappa X_{\!F} \karrow{\kappa} X) \\
\MPr_\kappa &\triangleq \l s.\l r.r\;s \\
~\\
\In_\kappa &\;:~ \forall X_{\!F}^{0(0\kappa-> \kappa)}.
		X_{\!F}(\mu_\kappa X_{\!F}) \karrow{\kappa} \mu_\kappa X_{\!F}\\
\In_\kappa &\triangleq \l t.\l s.s\;\textit{id}\;(\MPr_\kappa\;s)\;t \\
\textit{id} &\triangleq \l x.x
\end{align*}
\end{singlespace}
\caption{Embedding of the recursive type operators ($\mu_\kappa$),
	their data constructors ($\In_\kappa$),
	and the Mendler-style primitive recursors ($\MPr_\kappa$) in \Fixi.}
\label{fig:embedMPr}
\end{figure}

\begin{figure}
\begin{singlespace}
\[
\text{The type of $r$ can be expanded by the defintion of $\fix$
	and the equi-recursive equality rule on $\fix$ as follows:}\]
	\vskip-6ex
\begin{multline*}\,
\mu_\kappa X_{\!F}\,\mathbb{I} = \fix(\Phi_\kappa X_{\!F})\mathbb{I}
= \Phi_\kappa X_{\!F}(\fix(\Phi_\kappa X_{\!F}))\mathbb{I}
= \Phi_\kappa X_{\!F}(\mu_\kappa X_{\!F}) \mathbb{I} \\
= \forall X^\kappa.\underbrace{
	(\forall X_r^\kappa.
		(X_r \karrow{\kappa} \mu_\kappa X_{\!F}) ->
		(X_r \karrow{\kappa} X) ->
		(X_{\!F}\,X_r \karrow{\kappa} X) )}_\text{exactly matches with
						the type of {\,\small$s$}}
	-> X\,\mathbb{I} \,
\end{multline*}
\[
\inference{
	\inference{
	X_{\!F}^{0(0\kappa-> \kappa)}, X^{0\kappa},
	\mathbb{I}^\kappa
	; \;
	s: (\forall {X_r}^{\!\!\kappa}.
	 (X_r \karrow{\kappa} \mu_\kappa X_{\!F}) ->
	 (X_r \karrow{\kappa} X) ->
	 (X_{\!F}\,X_r \karrow{\kappa} X) ),
	r: \mu_\kappa X_{\!F}\,\mathbb{I}
	|- r\;s : X\,\mathbb{I}
	}{
	X_{\!F}^{0(0\kappa-> \kappa)}, X^{0\kappa} ; \;
	s: (\forall {X_r}^{\!\!\kappa}.
	 (X_r \karrow{\kappa} \mu_\kappa X_{\!F}) ->
	 (X_r \karrow{\kappa} X) ->
	 (X_{\!F}\,X_r \karrow{\kappa} X) )
	|- \l s.r\;s : \mu_\kappa X_{\!F} \karrow{\kappa} X
	}
}{
	\cdot;\cdot |- \l s.\l r.r\;s :
	\forall X_{\!F}^{0\kappa-> \kappa}.\forall X^\kappa.
	(\forall {X_r}^{\!\!\kappa}.
	 (X_r \karrow{\kappa} \mu_\kappa X_{\!F}) ->
	 (X_r \karrow{\kappa} X) ->
	 (X_{\!F}\,X_r \karrow{\kappa} X) ) ->
	 (\mu_\kappa X_{\!F} \karrow{\kappa} X) 
}
\]
\end{singlespace} \vskip-1.5ex
\caption{Well-typedness of the $\MPr$ embedding in \Fixi}
\label{fig:embedMPrJustify}
\end{figure}

\begin{figure}
\begin{singlespace}~\qquad
Let ~ $\Delta =
	X_{\!F}^{0(0\kappa-> \kappa)}, \mathbb{I}^\kappa,
	X^{0\kappa}$ ~ and ~
$\Gamma =
	t: X_{\!F}(\mu_\kappa X_{\!F})\mathbb{I},
	s: (\forall X_r^\kappa.
		(X_r \karrow{\kappa} \mu_\kappa X_{\!F}) ->
		(X_r \karrow{\kappa} X) ->
		(X_{\!F}\,X_r \karrow{\kappa} X) )$.
\[
\inference{
	\inference{
		\inference{
			\inference{
				{\begin{array}{llll}
			\Delta;\Gamma |- s \;:
			(\mu_\kappa X_{\!F} \karrow{\kappa} \mu_\kappa X_{\!F})
			~ -> &
			(\mu_\kappa X_{\!F} \karrow{\kappa} X)
			~ -> &
			(X_{\!F}(\mu_\kappa X_{\!F}) \karrow{\kappa} X)
			& \qquad \text{\small(by instantiating $X_r$
						with $\mu_\kappa X_{\!F})$}
				\\
			\Delta;\Gamma |- \textit{id} :
			(\mu_\kappa X_{\!F} \karrow{\kappa} \mu_\kappa X_{\!F})
				\\
			\Delta;\Gamma |- (\MPr_\kappa\; s) ~~ : &
			(\mu_\kappa X_{\!F} \karrow{\kappa} X)
				\\
			\Delta;\Gamma |- t \qquad\qquad\, : & &
			X_{\!F}(\mu_\kappa X_{\!F})\mathbb{I}
				\end{array}}
			}{
	X_{\!F}^{0(0\kappa-> \kappa)}, \mathbb{I}^\kappa,
	X^{0\kappa}
	; \;
	t: X_{\!F}(\mu_\kappa X_{\!F})\mathbb{I},
	s: (\forall X_r^\kappa.
		(X_r \karrow{\kappa} \mu_\kappa X_{\!F}) ->
		(X_r \karrow{\kappa} X) ->
		(X_{\!F}\,X_r \karrow{\kappa} X) )
	|- s\;\textit{id}\;(\MPr_\kappa\;s)\;t : X\,\mathbb{I}
			}
		}{
	X_{\!F}^{0(0\kappa-> \kappa)}, \mathbb{I}^\kappa ; \;
	t: X_{\!F}(\mu_\kappa X_{\!F})\mathbb{I}
	|- \l s.s\;\textit{id}\;(\MPr_\kappa\;s)\;t :
	\forall X^\kappa.
	(\forall X_r^\kappa.
		(X_r \karrow{\kappa} \mu_\kappa X_{\!F}) ->
		(X_r \karrow{\kappa} X) ->
		(X_{\!F}\,X_r \karrow{\kappa} X) ) -> X\,\mathbb{I}
		}
	}{
	X_{\!F}^{0(0\kappa-> \kappa)}, \mathbb{I}^\kappa ; \;
	t: X_{\!F}(\mu_\kappa X_{\!F})\mathbb{I}
	|- \l s.s\;\textit{id}\;(\MPr_\kappa\;s)\;t :
		\mu_\kappa X_{\!F}\,\mathbb{I} \quad
		\qquad \text{\small(We can expand the type into above
				as in Figure \ref{fig:embedMPrJustify})}
	}
}{
	\cdot;\cdot |- \l t.\l s.s\;\textit{id}\;(\MPr_\kappa\;s)\;t :
	\forall X_{\!F}^{\kappa-> \kappa}.
		X_{\!F}(\mu_\kappa X_{\!F}) \karrow{\kappa} \mu_\kappa X_{\!F}
}
\]
\end{singlespace} \vskip-2.5ex
\caption{Well-typedness of the $\In$ embedding in \Fixi}
\label{fig:embedInJustify}
\end{figure}

\end{landscape}
} %%%%%%%%%%%%%%%%%%%%%%% end of afterpage

Note that the polarities appearing in the embedding of $\mu_\kappa$ are all
$0$. The embedding from a non-polarize kind $\kappa$ into
a polarize kinds $\ulcorner\kappa\urcorner$ can be defined as:
\[ \ulcorner * \urcorner = * \qquad
\ulcorner \kappa_1 -> \kappa_2 \urcorner =
0\ulcorner\kappa_1\urcorner -> \ulcorner\kappa_2\urcorner \qquad
\ulcorner A -> \kappa \urcorner = A -> \ulcorner \kappa \urcorner.
\]

It is easy to see that the embedding of the non-polarized recursive
type operator $\mu_\kappa : 0(0\kappa -> \kappa) -> \kappa$
is well-kinded, provided that
$\Phi_\kappa : 0(0\kappa -> \kappa) -> +\kappa -> \kappa$
is well kinded. Note that $\Phi_\kappa$ turns an avariant type constructor
($0\kappa -> \kappa$) into a positive type constructor
($+\kappa -> \kappa$). From the definition of $\Phi_\kappa$, we only need 
to check that $(X_r \karrow{\kappa} X_c)$, $(X_r \karrow{\kappa} X)$,
$X_F\,X_r \karrow{\kappa} X$ and $X\;\mathbb{I}$ are of kind $*$
under the context $ X_{\!F}^{0(0\kappa -> \kappa)},
		X_c^{+\kappa}, \mathbb{I}^\kappa, X^{0\kappa}$,
which is not difficult to see.

Well-typedness of $\MPr_\kappa$ and $\In_\kappa$ are justified in
Figures~\ref{fig:embedMPrJustify} and \ref{fig:embedInJustify}


 %% \label{sec:fixi:data}

\section{Embedding course-of-values primitive recursion}
\label{sec:fixi:cv}

\lstset{language=Haskell,
	basicstyle=\ttfamily\small,
%	keywordstyle=\color{ta4chameleon},
%	emph={List,Int,Bool},
	commentstyle=\color{grey},
	literate =
		{forall}{{$\forall$}}1
%		{|}{{$\mid\;\,$}}1
%		{=}{{\textcolor{ta3chocolate}{$=\,\;$}}}1
		{::}{{$:\!\,:$}}1
		{->}{{$\to$}}1
		{=>}{{$\Rightarrow$}}1
		{\\}{{$\lambda$}}1
	}

\afterpage{ %%%%%%%%%%%%%%%%%%%%%%% begin afterpage
\begin{landscape}
\begin{figure}
\begin{singlespace}
\begin{align*}
\mu^{+}_\kappa &\;:~ 0(+\kappa -> \kappa) -> \kappa \\
\mu^{+}_\kappa &\triangleq
\l X_{\!F}^{0(+\kappa -> \kappa)}.\fix(\Phi^{+}_\kappa\,X_{\!F})\\
\Phi^{+}_\kappa &\;:~ 0(+\kappa -> \kappa) -> +\kappa -> \kappa \\
\Phi^{+}_\kappa &\triangleq \l X_{\!F}^{0(+\kappa -> \kappa)}.
\l X_c^{+\kappa}.\boldsymbol{\l}\mathbb{I}^\kappa.
\forall X^\kappa.
(\forall X_r^\kappa. (X_r \karrow{\kappa} X_{\!F}\,X_r)
		-> (X_r \karrow{\kappa} X_c)
		-> (X_r \karrow{\kappa} X)
		-> (X_{\!F}\,X_r \karrow{\kappa} X) ) -> X\,\mathbb{I}\\
~\\
\McvPr_\kappa &\;:~
	\forall X_{\!F}^{+\kappa-> \kappa}.\forall X^\kappa.
	(\forall {X_r}^{\!\!\kappa}.
	 (X_r \karrow{\kappa} X_{\!F}\,X_r) ->
	 (X_r \karrow{\kappa} \mu^{+}_\kappa X_{\!F}) ->
	 (X_r \karrow{\kappa} X) ->
	 (X_{\!F}\,X_r \karrow{\kappa} X) ) ->
	 (\mu^{+}_\kappa X_{\!F} \karrow{\kappa} X) \\
\McvPr_\kappa &\triangleq \l s.\l r.r\;s\\
~\\
\In_F &\;:~ F(\mu^{+}_\kappa F) \karrow{\kappa} \mu^{+}_\kappa F\\
\In_F &\triangleq
	\l t. \l s. s~\unIn_F\;\textit{id}\;\,(\McvPr_\kappa\;s)\;\,t \\
\end{align*}\vskip -2.5ex
Provided that there exists
$\unIn_{F} : \mu^{+}_\kappa F \karrow{\kappa} F(\mu^{+}_\kappa F)$
for the base structure $F:+\kappa -> \kappa$, 
such that $\unIn_F(\In_F\;t) -->+ t$
where the reduction is constant regardless of $t$
(steps may vary between each base structure $F$ though).
\end{singlespace} \vskip-3.5ex
\[\text{See Figure \ref{fig:unInExamples} for
embeddings of unrollers ($\unIn_F$) for
some well-known positive base structures ($F$).}
\]
\caption{Embedding of the recursive type operators ($\mu^{+}_\kappa$),
	the Mendler-style course-of-values primitive recursors
	($\McvPr_\kappa$), and the roller ($\In_F$) in \Fixi,
        provided that the embedding of $\unIn_F$ exists}
\label{fig:embedMcvPr}
\end{figure}

\begin{figure}
\[\!\!\!\!\!\!\!
\begin{array}{llcll}
	& \text{\textbf{Regular datatypes}} \\
N &\!\!\!\triangleq \l X^{+*}.X + \textsf{Unit} &\qquad&
\unIn_N &\!\!\!\triangleq \McvPr_{*} (\l\_.\l\textit{cast}.\l\_.
	\l x. x\;(\texttt{InL}\circ\textit{cast})\;\texttt{InR})
	\\
L &\!\!\!\triangleq \l X_a^{+*}.\l X^{+*}.(X_a\times X) + \textsf{Unit} &&
\unIn_{(LA)} &\!\!\!\triangleq \McvPr_{*} (\l\_.\l\textit{cast}.\l\_.
	\l x. x\;(\texttt{InL}\circ(\textit{id}\times cast))\;\texttt{InR})
	\\
R &\!\!\!\triangleq \l X_a^{+*}.\l X^{+*}.(X_a\times \texttt{List}\,X) -> X &&
\unIn_{(RA)} &\!\!\!\triangleq \McvPr_{*} (\l\_.\l\textit{cast}.\l\_.
	\l x. x\;(\textit{id}\times \textit{fmap}_\texttt{List}\;cast) )
	\\
& \text{\textbf{Type-indexed datatypes}} \phantom{G^{G^{G^{G^{G^G}}}}}\\
P &\!\!\!\triangleq \l X^{+* -> *}.\l X_a^{+*}.
	X_a \times X(X_a \times X_a) + \textsf{Unit} &&
\unIn_P &\!\!\!\triangleq \McvPr_{+* -> *} (\l\_.\l\textit{cast}.\l\_.
	\l x. x \;(\texttt{InL}\circ(\textit{id}\times\textit{cast}))
		\;\texttt{InR})
	\\
B &\!\!\!\triangleq \l X^{+* -> *}.\l X_a^{+*}.
	X_a \times X(X\,X_a) + \textsf{Unit} &&
\unIn_B &\!\!\!\triangleq \McvPr_{+* -> *} (\l\_.\l\textit{cast}.\l\_.
 	\l x. x \;(\texttt{InL}\circ
 		(\textit{id}\times
 			(\textit{cast}\circ\textit{fmap}\;\textit{cast})))
 		\;\texttt{InR})
	\\
	& \text{\textbf{Term-indexed datatypes}} \phantom{G^{G^{G^{G^{G^G}}}}}
\end{array}
\]\vskip-4.5ex
\[
V \triangleq \l X_a^{*}.\l X^{\texttt{Nat} -> *}.\l i^{\texttt{Nat}}.
(\exists j^\texttt{Nat}.((i=\texttt{succ}\,j) \times X_a \times X\{j\})) +
(i=\texttt{zero})
\]
\[
\begin{array}{lll}
\texttt{VCons} &\!\!\!\triangleq \l x_a.\l x.
	\texttt{InL}(\mathtt{Ex_{Nat}}(\mathtt{Eq_{\,Nat}},x_a,x))
& : \;
\forall X_a^{*}. \forall X^{\texttt{Nat} -> *}. \forall i^\texttt{Nat}.
	X_a -> X\,\{i\} -> V\,X_a\,X\,\{\texttt{succ}\,i\}
	\\
\texttt{VNil} &\!\!\!\triangleq \texttt{InR}~\mathtt{Eq_{\,Nat}}
& : \;
\forall X_a^{*}. \forall X^{\texttt{Nat} -> *}. V\,X_a\,X\,\{\texttt{zero}\}
\end{array}
\]
\[
\unIn_{(V\,A)} \triangleq \McvPr_{\texttt{Nat} -> *}(\l\_.\l\textit{cast}.\l\_.
\l x. x \;(\texttt{InL}\circ
		(\textit{id}\times\textit{id}\times\textit{cast}))
	\;\texttt{InR})
\]
The notation $\exists j^A.B$ is a shorthand for $\exists_A(\l j^A.B)$
where $\exists_A$ is defined in Figure~\ref{fig:fixiNonRecData}.\\
$\mathtt{Ex_{A}} : \forall F^{A -> *}.\exists_A F$ and
$\mathtt{Eq_{A}} : \forall i^A.\forall j^A.(i=j)$ are
the data constructors of the existential type and the equality type.
\[\text{
See Figure \ref{fig:embedMcvPr} for the embedding of the Mendler-style
course-of-values primitive recursor ($\McvPr_\kappa$)}
\]
\caption{Embeddings of unroller ($\unIn_F$)
	for some well-known positive base structures ($F$).}
\label{fig:unInExamples}
\end{figure}

\end{landscape}
} %%%%%%%%%%%%%%%%%%%%%%% end of afterpage

%%%%%%%%%%%% TODO move this to somewhere else
To add a new Mendler-style combinator, we need to address several issues:
\begin{itemize}
\item First, we need to add an appropriate type-level fixpoint operator
	(\eg, $\mu_\kappa$ for primitive recursion)
that is used to build recursive types. This type-level operation needs to
capture not only the tying of the recursive knot, but also the compatible
structure needed to encode the new Mendler operation.
\item Second, we need to specify the behavior of the new Mendler operation.
	This means discovering the characteristic equations
	that the operation should obey
	(\eg, Haskell definition of Mendler-style combinators
	in Chapter~\ref{ch:mendler}).
\item Third, we need to find an embedding that preserves
	the characteristic equations in the host calculus
	(\eg, embedding of $In_\kappa$ and $\MPr_\kappa$).
\item In practice, the second and third steps are intimately entwined as
the equations and embedding are carefully designed
(using a Church-style encoding) to achieve the desired result
(\eg, $In_\kappa$ is defined in terms of
	the Mendler-style combinator $\MPr_\kappa$).
\end{itemize}

To embed Mendler-style course of values recursion, we can follow the steps above
just ad we did for Mendler-style primitive recursion. In addition, we need
to embed an unroller $\unIn_F$, which is the key operation to support
course-of-values recursion. Recall the use of \textit{out} in the Fibonacci
number example (Figure~\ref{fig:fib}) and the Lucas number example
(Figure~\ref{fig:lucas}) in Section~\ref{ssec:tourHist0}.

\subsection{The general form for the embedding of course-of-values recursion}
Figure~\ref{fig:embedMcvPr} and illustrates the embedding of
a new iso-recursive type operator ($\mu^{+}_\kappa$),
the Mendler-style course-of-values primitive recursor ($\McvPr_\kappa$),
and the roller $\In_F$ in \Fixi. Since the embedding of $\In_F$ uses $\unIn_F$,
we also need embed unroller $\unIn_F$ in order to complete the embedding of
the roller $\In_F$. Embeddings of $\unIn_F$ are possible for a fairly large
class of positive base structures. Figure~\ref{fig:unInExamples} illustrates
some of those unrollers. But, it may not be possible to give an embedding
of the unroller for some base structures. Recall that not all base structures
can have well-defined course-of-values recursion that grantee termination
(see Figure~\ref{fig:LoopHisto} in Section~\ref{ssec:tourHist0}).

The embeddings of $\mu^{+}_\kappa$, $\McvPr_\kappa$, and $\In_F$
for course-of-values recursion (see Figure~\ref{fig:embedMcvPr}) are
very similar to the embeddings of $\mu_\kappa$, $\MPr_\kappa$, and $\In_\kappa$
for primitive recursion (see Figure~\ref{fig:embedMPr}), which we discussed
earlier in the previous section. The definition of
$\McvPr_\kappa \triangleq \l s.\l r.r\,s$ is exactly the same as
the definition of $\MPr_\kappa$, but only differs in its type signature.
The definition of $\In_F$ is similar to $\In_\kappa$ but uses the additional
$\unIn_F$, which implements the abstract unroller. So, the last piece
of the puzzle for embedding Mendler-style course-of-values recursion is
the embedding of $\unIn_F$.

In the following subsections, we will elaborate on how to embed unrollers
($\unIn_F$) through examples (\S\ref{sec:fixi:cv:unInExamples}), derive
uniform embeddings of the unrollers generalizing from those examples, and
discuss whether the embeddings of unrollers satisfy their desired properties.

\subsection{Embedding unrollers}
\label{sec:fixi:cv:unInExamples}
Embeddings of unrollers for some well-known positive datatypes
are illustrated in Figure~\ref{fig:unInExamples}. The general idea is to
use $\McvPr_\kappa$ to define $\unIn_F$ for the base structure
$F:+\kappa -> \kappa$ without using the abstract recursive call operation
but only using the abstract cast operation in order be define constant
time unrollers. To define an unroller, we map non-recursive
components ($X_a$) as they are using \textit{id} and map abstract recursive
components ($X_r$) to concrete recursive components
($\mu^{+}_\kappa F$) using the abstract \textit{cast} operation provided
by the $\McvPr_\kappa$ combinator. We can embed unrollers
for regular datatypes such as natural numbers (the base $N$) and lists
(the base $L$), type-indexed datatypes such as powerlists (the base $P$),
and term-indexed datatypes such as vectors (the base $V$) in this way.
The notation for combining functions for tuples are defined as
$(f \times g) \triangleq \l x.(f\,x,g\,x)$, and $(f\times g\times h)$
are defined similarly for triples.

\paragraph{Embeding unrollers for regular datatypes:}
The embeddings of $\unIn_N$ and $\unIn_{(L A)}$ are self explanatory.
The embeddings of $\unIn_{(R A)}$ relies on the map function for lists,
since the rose tree is indirect datatype where recursive subcomponents
appear inside the list ($\texttt{List}\,X_r$).
The $\textit{fmap}_\texttt{List}$ function applies \textit{cast} to each of
the the abstract recursive subcomponents of type $X_r$ inside $F\;X_r$
values into a concrete recursive type $\mu_{*}^{+}F$ in order to obtain
$F(\mu_{*}^{+}F)$ values.

\paragraph{Embeding unrollers for nested datatypes} are no more complicated
than embedding unrollers for regular datatypes. For instance, the embedding
of $\unIn_P$ for powerlists is almost identical to the embedding of $\unIn_L$
for regular lists except the use of $\McvPr_{* -> *}$ instead of $\McvPr_{*}$.
This is because the cast operation provided by $\McvPr_{* -> *}$ is polymorphic
over the type index:
$\textit{cast}:\forall X_i. X_r X_i -> \mu^{+}_{* -> *} X_F X_i$.
Since unrollers preserve indices, there is no extra work to be done
other than applying the \textit{cast}. In the embeding of $\unIn_P$,
the $\texttt{cast}$ function performs an index preserving cast from
an abstract recursive type $X_r (X_a\times X_a)$ to the concrete powerlist type 
$\mu^{+}P (X_a\times X_a)$.

Embedding unrollers for \emph{truly nested datatypes} \cite{AbeMatUus05},
such as bushes, are similar to embedding unrollers for indirect regular
recursive types. Truly nested datatypes are recursive datatypes whose
indices may involve themselves. Truly nested datatypes are similar to
indirect recursive types in the sense that a bunch of recursive components
are contained in certain data structures -- in case of truly nested datatypes
those data structures are exactly the nested datatypes themselves.
Assuming that the nested datatype has a notion of monotone map, we can
use \textit{fmap} to push down the \textit{cast} to the inner structure
and then \textit{cast} the outer structure. Note the use of
$(\textit{cast} \circ \textit{fmap}\;\textit{cast})$ in the embedding of
the unroller $\unIn_R$ for bushes.

\paragraph{Embeding unrollers for indexed datatypes} are no more complicated
than embedding unrollers for regular datatypes, as well. To embed unrollers
for term-indexed datatypes, we would often need existential types
(Figure~\ref{fig:fixiNonRecData}) and equality types. We can encode
equality types in \Fixi\ as a Leibniz equality over indices,
\ie, $(i=j) \triangleq \forall F^{A -> *}.F\{i\} -> F\{j\}$,
as discussed in \S\ref{Leibniz}. These extra encodings for maintaining
term-indices does not terribly complicate the embeddings of unrollers,
since unrollers are index preserving. The embedding of $\unIn_{(V A)}$ for
length indexed lists are almost identical to the embedding of $\unIn_{(L A)}$
for regular lists, except that it has one more \textit{id}. The first
\textit{id} appearing in $(\textit{id}\times\textit{id}\times\textit{cast})$ is
for the index equality, to leave the index equality unchanged.

\paragraph{}
Giving different definitions of $\unIn_F$ for each different $F$ as illustrated
in Figure~\ref{fig:unInExamples} looks too ad-hoc. So, we will discuss how to
generalize the embeddings of the $\unIn$ operations, assuming that $F$ has
a notion of a monotone map (\eg, \textit{fmap} for $F:+* ->*$) in the following
subsection. Later on, in Section~\ref{ssec:fixi:theory:cv}, we will
reason about what are the conditions for $F$ to have a notion of a monotone map.

\begin{figure}
\begin{singlespace}
\begin{lstlisting}
newtype Mu0 f = In0 { unIn0 :: f(Mu0 f) }

mcvpr0 :: Functor f => (forall r. (r -> f r) ->
                           (r -> Mu0 f) ->
                           (r -> a) ->
                           (f r -> a) )
       -> Mu0 f -> a
mcvpr0 phi = phi unIn0 id (mcvpr0 phi) . unIn0

newtype Mu1 f i = In1 { unIn1 :: f(Mu1 f)i }

mcvpr1 :: Functor1 f =>
         (forall r i'. Functor r => (forall i. r i -> f r i) ->
                              (forall i. r i -> Mu1 f i) ->
                              (forall i. r i -> a i) ->
                              (f r i' -> a i') )
       -> Mu1 f i -> a i
mcvpr1 phi = phi unIn1 id (mcvpr1 phi) . unIn1

class Functor1 h where
  fmap1'  :: Functor f => (forall i j. (i -> j) -> f i -> g j)
                     -> (a -> b) -> h f a -> h g b
  -- fmap1' h = fmap1 (h id)

  fmap1 :: Functor f => (forall i. f i -> g i)
                     -> (a -> b) -> h f a -> h g b
  fmap1 h = fmap1' (\f -> h . fmap f)

instance (Functor1 h, Functor f) => Functor (h f) where
  fmap f = fmap1 id
        -- fmap1' (\f -> id . fmap f)

instance Functor (f (Mu1 f)) => Functor (Mu1 f) where
  fmap f = In1 . fmap f . unIn1
\end{lstlisting}
\end{singlespace}
\caption{$\mu_{*}$, $\McvPr_{*}$, and $\mu_{* -> *}$, $\McvPr_{* -> *}$
	transcribed into Haskell}
\label{fig:HaskellFunctor1}
\end{figure}

\begin{figure}
\begin{lstlisting}
data N r   = S r   | Z  deriving Functor
type Nat = Mu0 N

unInN :: Mu0 N -> N(Mu0 N)
unInN = mcvpr0 (\ _ cast _ ->  fmap cast)

data L a r = C a r | N  deriving Functor
type List a = Mu0 (L a)

unInL :: Mu0(L a) -> (L a) (Mu0(L a))
unInL = mcvpr0 (\ _ cast _ ->  fmap cast)

data R a r = F a [r]    deriving Functor -- relies on (Functor [])
type Rose a = Mu0 (R a)

unInR :: Mu0(R a) -> (R a) (Mu0(R a))
unInR = mcvpr0 (\ _ cast _ ->  fmap cast)
\end{lstlisting}
\caption{Embeddings of $\unIn_N$, $\unIn_{(L A)}$, $\unIn_{(R A)}$
	transcribed into Haskell}
\label{fig:HaskellunInRegular}
\end{figure}

\begin{figure}
\begin{singlespace}
\begin{lstlisting}
data P r i = PC i (r (i,i)) | PN
type Powl i = Mu1 P i

instance Functor1 P where
  fmap1' h f (PC x a) = PC (f x) (h (\(i,j) -> (f i,f j)) a)
  fmap1' _ _ PN = PN

unInP :: Mu1 P i -> P(Mu1 P) i
unInP = mcvpr1 (\ _ cast _ -> fmap1 cast id)
  -- mcvpr1 phi where
  --   phi _ cast _ (PC x xs) = PC x (cast xs)
  --   phi _ cast _ PN = PN

data B r i = BC i (r (r i)) | BN
type Bush i = Mu1 B i

instance Functor1 B where
  fmap1' h f (BC x a) = BC (f x) (h (h f) a)
  fmap1' _ _ BN = BN

unInB :: Mu1 B i -> B (Mu1 B) i
unInB = mcvpr1 (\ _ cast _ -> fmap1 cast id)
  -- mcvpr1 phi where
  --   phi _ cast _ (BC x xs) = BC x (cast (fmap cast xs))
  --   phi _ cast _ BN = BN
\end{lstlisting}
\end{singlespace}
\caption{Embedding of $\unIn_P$ and $\unIn_B$ transcribed into Haskell}
\label{fig:HaskellunInNested}
\end{figure}

\begin{figure}
\begin{singlespace}
\begin{lstlisting}
class FunctorI1 (h :: (* -> *) -> * -> *) where
  fmapI1 :: (forall i . f i -> g i) -> h f j -> h g j

mcvprI1 :: FunctorI1 f =>
          (forall r j. (forall i. r i -> f r i) ->
                  (forall i. r i -> Mu1 f i) ->
                  (forall i. r i -> a i) ->
                  (f r j -> a j) )
       -> Mu1 f i' -> a i'
mcvprI1 phi = phi unIn1 id (mcvprI1 phi) . unIn1


data Succ n
data Zero

data V a r i where
  VC :: a -> r n -> V a r (Succ n)
  VN :: V a r Zero

instance FunctorI1 (V a) where
  fmapI1 h (VC x a) = VC x (h a)
  fmapI1 _ VN = VN

unInV :: Mu1 (V a) i -> (V a) (Mu1 (V a)) i
unInV = mcvprI1 (\_ cast _ -> fmapI1 cast)


instance FunctorI1 P where
  fmapI1 h (PC x a) = PC x (h a)
  fmapI1 _ PN = PN

unInP' :: Mu1 P a -> P (Mu1 P) a
unInP' = mcvprI1 (\_ cast _ -> fmapI1 cast)
\end{lstlisting}
\end{singlespace}
\caption{Embedding of $\unIn_{(V A)}$ and
	yet another embedding of $\unIn_P$\\ transcribed into Haskell}
\label{fig:HaskellunInIndexed}
\end{figure}

\subsection{Deriving uniform embeddings of the unrollers}
To derive uniform embeddings of the unrollers, we transcribed the embeddings
of the unrollers appearing in Figure~\ref{fig:unInExamples} into Haskell
and observed common patterns among them. These results are summarized
in Figures~\ref{fig:HaskellFunctor1}, \ref{fig:HaskellunInRegular},
\ref{fig:HaskellunInNested}, and \ref{fig:HaskellunInIndexed}.
This excercise of Haskell transcription not only helped us derive
uniform embeddings of the unrollers, but also helped us recognize
the conditions that the base structures should satisfy in order to
have an embedding of its unroller in \Fixi.

Haskell transcriptions of the unroller embeddings for regular datatypes are
given in Figure~\ref{fig:HaskellunInRegular} (Haskell definitions of
\texttt{Mu0} and \texttt{mcvpr0} are given in Figure~\ref{fig:HaskellFunctor1}).
Note that the definitions of these unrollers are uniform:
\lstinline$mvcvp0 (\ _ cast _ ->  fmap cast)$.
This uniform definition relies on the existence of \textit{fmap} over
the base structures -- note the \lstinline$deriving Functor$
in the data declarations. In Section~\ref{ssec:fixi:theory:cv},
we will show that \textit{fmap} exists for any $F:+* -> *$ in \Fixi.
So, we can derive a uniform embedding for the unroller $\unIn_{*}$
for any base $F:+* -> *$ as follows:
\begin{align*}
\unIn_{*} &: \forall X_F^{+* -> *}.\mu_{*}^{+}X_F -> X_F(\mu_{*}^{+}X_F) \\
\unIn_{*} &\triangleq \McvPr_{*} (\lambda\_.\lambda \textit{cast}.\lambda\_.
					\textit{fmap}_{X_F} ~\textit{cast})
\end{align*}

Haskell transcriptions of the unroller embeddings for nested datatypes are
given in Figure~\ref{fig:HaskellunInNested}. (Haskell definitions of
\texttt{Mu1}, \texttt{mcvpr1}, \texttt{fmap1'} are given
in Figure~\ref{fig:HaskellFunctor1}). Note that the definitions of
the unrollers \texttt{unInP} for powerlists and \texttt{unInB} for bushes
are uniform: \lstinline$mvcvp1 (\ _ cast _ ->  fmap1 cast id)$.
The function \texttt{fmap1} is the rank 2 monotone map for type constructors
of kind $+(+* -> *) -> (+* -> *)$, which analogous to rank 1 monotone map
\texttt{fmap} for type constructors of kind $+* -> *$. Or, you can think of 
the \texttt{Functor1} class as a bifunctor over type constructors of kind
$+(+* -> *) -> +* -> *$, whose first argument
($+\underline{(+* -> *)} -> +* -> *$) is a type constructor
and second argument ($+(+* -> *) -> +\underline{*} -> *$) is a type.

Lastly, a Haskell transcription of the unroller embedding for
the length indexed list datatype, which is a term-indexed datatype,
is given in Figure~\ref{fig:HaskellunInIndexed}. To give an embedding of
$\unIn_{(V A)}$, we defined another version of Mendler-style course-of-values
recursion combinator \texttt{mcvprI1}, which is similar to \texttt{mcvpr1}
in Figure~\ref{fig:HaskellFunctor1} but requires the base structure
to be an instance of the \texttt{FunctorI1} class rather than an instance of
the \texttt{Functor1} class. The \texttt{FunctorI1} class is more simple than
the \texttt{Functor1} class since \texttt{FunctorI1} requires only the first
argument to be monotone while \texttt{Functor1} requires both the first and
the second argument to be monotone -- this is evident when you compare
the types of thier member functions \texttt{fmap1} and \texttt{fmapI1}
side-by-side:
\begin{lstlisting}
    fmap1  :: (Functor f, Functor1 h) =>
              (forall i. f i -> g i) -> (a -> b) -> h f a -> h g b
    fmapI1 :: FunctorI1 h =>
              (forall i. f i -> g i)           -> h f a -> h g a
\end{lstlisting}
Note that there fmapI1 does not have the extra \texttt{Functor}
requirement since it does not require the second argument type
to be monotone. The function \texttt{fmapI1} only transforms the first type
constructor argument and preserves the second argument (which may be
a type index or a term index), while \texttt{fmap1} is able to transform
both the first and the second arugment. For term-indexed datatypes,
\texttt{fmapI1} is enough to construct embeddings of the unrollers,
since there is no need to transform indices -- recall that term-indices
do not appear at value level, so no values to transform in the first place.
And even if we had such values, we do not need to transform them anyway
since unrollers are index preserving by definition.
Thus, in the type signature of \texttt{mcvprI1}, we need not reqire
the abstract recursive structure \texttt{r} to be a \texttt{Functor},
unlike in the type signature of \texttt{mcvpr1} where we did require
\texttt{Functor r}. The embedding of $\unIn_{(V A)}$ has the shape
you would expect: \lstinline$mcvprI1 (\_ cast _ ->  fmapI1 cast)$.

An interesting observation is that we can give yet another
Haskell transcription of $\unIn_P$ via in terms of \texttt{mcvprI1}
(see \texttt{unInP'} in Figure~\ref{fig:HaskellunInIndexed}),
rather than in terms of \texttt{mcvpr1}. This is because we did not
really need the ability to transform type indices to embed unrollers
for powerlists. However, for truly nested datatypes such as bushes,
this alternative is not possible. Recall that we do needed to
transform indices from abstract recursive type to concrete recursive type
to embed $\unIn_B$, because bushes were indexed by its own structure.
In summary, unroller embeddings for indexed datatypes, regardless of
term-indexed or type-indexed, without requiring indices to be monotone
unless the datatype is truley nested. For truly nested datatype,
we do have to require indices to be monotone as well as the recursive
type constructor itself, in order to embed their unrollers. We can have
a good approximation of this idea by the type system due to the polirzed kinds
in \Fixi. We conjecture that all base structures of kind $+(+* -> *) -> +* -> *$
are instances of \texttt{Functor1}, while all base structures of kind
$+(0* -> *) -> 0* -> *$ are instances of \texttt{Functor1}. We will
discuss this more formally in Section~\ref{ssec:fixi:theory:cv}.

\subsection{Properties of the unrollers}
We expect two properties to hold for the embeddings of the unrollers.
First, $\unIn_F$ must be a left identity of $\In_\kappa$.
That is, $\unIn_F(\In_\kappa t) -->+ t$ for any term $t$.
Second, $\unIn_F$ should be a constant time operation regardless of its
argument being supplied. That is, $\unIn_F(\In_\kappa t) -->+ t$ takes
constant steps independent of $t$ (but may vary between differrent $F$s).
One bad news is that some embeddings of the unrollers illustrated in
Figure~\ref{fig:unInExamples} are not constant time. However, the good news is
that we can safely optimized them into constant time functions due to
the metatheoretic property of $\McvPr_\kappa$ and $\textit{fmap}$:
\begin{itemize}
\item The \textit{cast} operation of $\McvPr_\kappa$
	is implemened as the identity function \textit{id}.
\item $(\textit{fmap}_F\;\textit{id})\;t -->+ t\;$ for any $\;t : F A$.
	This property generalizes to monotone maps of higher kinds.
	For instance,
      $(\textit{fmap1}_H\;\textit{id}\;\textit{id})\;t -->+ t$
      for any $t : H\,F\,A$ (see Figure~\ref{fig:HaskellFunctor1}
      for the definition of \textit{fmap1}).
\end{itemize}

For instance, $\unIn_{(R A)}$ for the rose tree datatype, which is an
\emph{indirect recursive datatype}, are not constant time. The map function
for lists $\textit{fmap}_\textit{List}$ appearing in the definition
of $\unIn_{(R A)}$ is obviously not a constant time function. That is,
we traverse the list inside a rose tree to to \textit{cast} each element of
the list. Thus, $\unIn_{(R A)}$ is linear to the length of the list apparing
in the rose tree. We can safely optimize $\unIn_{(R A)}$ into a constant time
operation by optimizing $(\textit{fmap}_\texttt{List}\;\textit{cast})$
into the identity function $\textit{id}$. This optimization is safe due to
the property of \textit{cast} being implemented by \textit{id}
and the property of $(\textit{fmap}\;\textit{id})$ being
equivalent to $\textit{id}$.
However, this does not mean we have a constant time embedding of $\unIn_{(R A)}$
within \Fixi, since the optimized term is not type correct.
The identity function $\textit{id}\triangleq \l x.x$ cannot be given
the same type as $(\textit{fmap}_\texttt{List}\;\textit{id}) :
	\texttt{List}(X_r\,X_a) -> \texttt{List}(\mu^{+}_\kappa R\,X_a)$.

For similar reasons, $\unIn_B$ for the bush datatype, which is a
\emph{truly nested datatype}, is not constant time either
due to the use of $\textit{fmap}\;\textit{cast} :
			(X_r(X_r\,X_a)) -> (X_r(\mu^{+}_{* -> *}B\,X_a))$,
which traverses the outer $X_r$ structure (an abstract bush) to cast each
elelemt from $(X_r\,X_a)$ to $(\mu^{+}_{* -> *}B\,X_a)$. But this time,
there is yet another subtlety before worrying about the embedding of
$\unIn_B$ not being constant time. Note, we boldly assumed that
the abstract recursive type $X_r$ has an \textit{fmap} operation
(specified by \texttt{{\bf\ttfamily Functor} r} in the Haskell transcription).
Previously, in the embedding of $\unIn_R$ for the rose tree, we relied on
a property of a specific instance of \textit{fmap} for \texttt{List},
which is a well known type to have \textit{fmap} and indeed equipped with
it desired property. In case of $\unIn_B$, we cannot assume anything but
the kind of $X_r : +* -> *$ since it is abstract. So, we should rely on
a more general property that \textit{fmap} is well-defined for any
type constructors of kind $+* -> *$ in \Fixi. We will discuss this
general property of \textit{fmap} in Section~\ref{ssec:fixi:theory:cv}.

Lastly, we even further optimize the unrollers since we observed
in the previous subsection that all unroller embeddings have uniform shape of
\[
\McvPr_\kappa (\l _.\l cast.\l _.
		\textit{fmap??}~\textit{cast}~\textit{id}\dots\textit{id}))
\]
% Embeddings of unrollers for term-indexed datatypes (e.g. vectors)
% are intuitively more simple than type-indexed datatypes (e.g. powerlists,
% bushes). Due to the erasure property, existence of unroll
% We will discuss the conditions when we can further NO, it's the other way
% around

%% \KYA{ TODO cite Monotone Inductive and Coinductive Constructors of Rank 2  - Matthes CSL paper }
%% 
%% \KYA{ TODO state a theorem about this in metatheory and
%% 	cite some previous work on this maybe? }
%% \KYA{ TODO P do not need Functor requirement but $+* -> *$ kind already
%%         implies that so it will be true anyway}
%% \KYA{ TODO someone must have proved free theorems for higher-kinded cases? }


%% Apart from the limitations of constant-time undefinability of $\unIn_F$
%% discussed above, the embeddings illustrated in Figure~\ref{fig:unInExamples}
%% are not in spirit of Mendler-style. Note that the embeddings of $\unIn_F$ are
%% polytypic (different term encodings for each different $F$) rather than
%% polymorphic (one uniform term encoding whose type is polymorphic over $F$).
%% Recall that the key advantage of Mendler-style comes from being polymorhpic.
%% 
%% Fortunately, there does exists more proper Mendler-style embeddings
%% of the course-of-values combinators over arbitrary positive datatypes
%% using both iteration and coiteration schemes \cite{TODO}. Since coiteration
%% is embeddable in \Fi\ and co(-primitive-)recursion is embeddable in \Fixi,
%% the result directly applies without extending our calculi. However,
%% to our knowledge, course-of-values combinators over higher-kinded
%% type constructors (\ie, type constructors other than kind $*$) has not been
%% well studied enough, even in that setting of using both iteration/recursion
%% and coiteration/corecursion. That is, course-of-values combinators for
%% regular indirect recursive datatypes are very likely to be embeddable in
%% a calculus like \Fi\ or \Fixi\ directly applying the known results, but
%% we may need further investigation to assure ourselves for the behavior of
%% course-of-values combinators over higher-kinded datatypes.
%% 
%% We leave the search for embeddings for arbitrary positive datatypes,
%% including indirectly recursive datatypes and truly nested datatypes,
%% as future work, since coiteration and corecursion are out of the scope of
%% this dissertation.



 %% \label{sec:fixi:cv}

\section{Metatheory} \label{sec:fixi:theory}

\subsection{TODO Strong normalization} \label{ssec:fixi:theory:sn}
We can prove strong normalization of \Fixi\ by erasing term-indices in \Fixi\ 
types into \Fixw\ types. Since \Fixw\ is strongly normalizing \cite{AbeMat04},
the existence of a index erasure that maps a valid typing judgment on a term
in \Fixi\ to a valid typing judgment on the same term in \Fixw\ implies
strong normalization of \Fixi.

The definition of the index erasure operation and the proofs for
the related theorems are almost exactly the same as their counterparts
in System \Fi\ (see \S\ref{sec:fi:theory}). So, we simply illustrate
the definition and just give a very brief sketch of the proofs for the
theorems.

We define a meta-operation of index erasure that projects $\Fixi$ types
to $\Fixw$ types.
\begin{definition}[index erasure]\label{def:Fixierase}
\[ \fbox{$\kappa^\circ$}
 ~~~~ ~~
 *^\circ =
 *
 ~~~~ ~~
 (p\kappa_1 -> \kappa_2)^\circ =
 p{\kappa_1}^\circ -> {\kappa_2}^\circ
 ~~~~ ~~
 (A -> \kappa)^\circ =
 \kappa^\circ
\]
\[ \fbox{$F^\circ$}
 ~~~~
 X^\circ =
 X
 ~~~~ ~~~~
 (A -> B)^\circ =
 A^\circ -> B^\circ
 ~~~~ ~~~~
 (\mu F)^\circ =
 \mu F^\circ
\]
\[ \qquad
 (\lambda X^{p\kappa}.F)^\circ =
 \lambda X^{p\kappa^\circ}.F^\circ
 ~~~~ ~~~~
 (\lambda i^A.F)^\circ =
 F^\circ
\]
\[ \qquad
 (F\;G)^\circ =
 F^\circ\;G^\circ
 ~~~~ ~~~~ ~~~~ ~~~~ ~~
 (F\,\{s\})^\circ =
 F^\circ
\]
\[ \qquad
 (\forall X^\kappa . B)^\circ =
 \forall X^{\kappa^\circ} . B^\circ
 ~~~~ ~~~~
 (\forall i^A . B)^\circ =
 B^\circ
\]
\[ \fbox{$\Delta^\circ$}
 ~~~~
 \cdot^\circ = \cdot
 ~~~~ ~~
 (\Delta,X^{p\kappa})^\circ = \Delta^\circ,X^{p\kappa^\circ}
 ~~~~ ~~
 (\Delta,i^A)^\circ = \Delta^\circ
\]
\[ \fbox{$\Gamma^\circ$}
 ~~~~
 \cdot^\circ = \cdot
 ~~~~ ~~~~
 (\Gamma,x:A)^\circ = \Gamma^\circ,x:A^\circ
\]
\end{definition}

\begin{theorem}[index erasure on well-sorted kinds]
\label{thm:Fixierasesorting}
	$\inference{|- \kappa : \square}{|- \kappa^\circ : \square}$
\end{theorem}

\begin{theorem}[index erasure on well-formed type level contexts]
\label{thm:Fixierasetyctx}
\[ \inference{|- \Delta}{|- \Delta^\circ} \]
\end{theorem}

\begin{theorem}[index erasure on kind equality]\label{thm:Fixierasekindeq}
$ \inference{|- \kappa=\kappa':\square}
	{|- \kappa^\circ=\kappa'^\circ:\square}
$
\end{theorem}

\begin{theorem}[index erasure on well-kinded type constructors]
\label{thm:Fixierasekinding}
\[ \inference{|- \Delta & \Delta |- F : \kappa}
		{\Delta^\circ |- F^\circ : \kappa^\circ}
\]
\end{theorem}
\begin{theorem}[index erasure on type constructor equality]
\[ \inference{\Delta |- F=F':\kappa}
		{\Delta^\circ |- F^\circ=F'^\circ:\kappa^\circ}
\]
\label{thm:Fixierasetyconeq}
\end{theorem}

\begin{theorem}[index erasure on well-formed term level contexts]
\label{thm:Fixierasetmctx}
\[ \inference{\Delta |- \Gamma}{\Delta^\circ |- \Gamma^\circ} \]
\end{theorem}

\begin{theorem}[index erasure on index-free well-typed terms]
\label{thm:Fixierasetypingifree}
\[ \inference{ \Delta |- \Gamma & \Delta;\Gamma |- t : A}
		{\Delta^\circ;\Gamma^\circ |- t : A^\circ}
		{\enspace(\dom(\Delta)\cap\FV(t) = \emptyset)}
\]
\end{theorem}


We introduce an index variable selection meta-operation that selects all
the index variable bindings from the type level context.
\begin{definition}[index variable selection]
\[ \cdot^\bullet = \cdot \qquad
	(\Delta,X^{p\kappa})^\bullet = \Delta^\bullet \qquad
	(\Delta,i^A)^\bullet = \Delta^\bullet,i:A
\]
\end{definition}

\begin{theorem}[index erasure on well-formed term level contexts
		prepended by index variable selection]
\label{thm:Fixierasetmctxivs}
\[ \inference{\Delta |- \Gamma}{\Delta^\circ |- (\Delta^\bullet,\Gamma)^\circ}
\]
\end{theorem}

\begin{theorem}[index erasure on well-typed terms]
\label{thm:Fixierasetypingall}
\[ \inference{\Delta |- \Gamma & \Delta;\Gamma |- t : A}
		{\Delta^\circ;(\Delta^\bullet,\Gamma)^\circ |- t : A^\circ}
\]
\end{theorem}


\KYA{TODO has there been any studies on the logical consistencies of
	implicit calculus + equi-recursive type?}



\subsection{Conditions for well-behaved course-of-values recursion}
\label{ssec:fixi:theory:cv}

\begin{proposition}\label{prop:fixi:fmap}
For any $F : +* -> *$, there exists
\[ \textit{fmap}_F : \forall X^{*}.\forall Y^{*}.(X -> Y) -> F\;X -> F\;Y \]
\end{proposition}

start with a more easy one
\begin{proposition}
There exists
$\textit{fmap}_F : \forall X^{*}.\forall Y^{*}.(X -> Y) -> F\;X -> F\;Y$
for any $F : +* -> *$ such that
\begin{itemize}
	\item all free variables of $F$, including $X$ and $Y$,
		are introduced by universal quantification
		and those variabels are of kind $*$,
	\item all the bound variabels within $F$ are
		introduced by universal quantification
		and those variables are of kind $*$, and
	\item $F : +* -> *$ is in normal form, that is, $F$ can neither be
		an application ($F'\,G$) nor index application ($F'\{s\}$).
\end{itemize}
\end{proposition}
\begin{proof}
	We can derive $\textit{fmap}_F$ from the structure of $F$.
	Due to the kind of $F$ and the fact that $F$ is in normal form
	it must be a in the form of $\l Z^{+*}.B$.
\begin{itemize}
\item[case]($Z\notin \FV(B)$)
	$ \textit{fmap}_{(\l Z^{+*}. B)} = \l \_ . \l x. x $

	Since $F\;X = F\;Y = B$, we just return the identity function on $B$.

\item[case]($F \triangleq \l Z^{+*}. Z$)
	$ \textit{fmap}_{(\l Z^{+*}. Z)} = \l z . z $

	Since $F\;X = X$ and $F\;X = Y$,
	we return the function $z:X -> Y$ itself.

\item[case]($F \triangleq \l Z^{+*}. \forall X_1^{*} .B_1$)
	$\textit{fmap}_{(\l Z^{+*}. \forall X_1^{*} .B_1)}
	= \textit{fmap}_{(\l Z^{+*}.B_1)}$

\item[case]($F \triangleq \l Z^{+*}. A -> B_1$)

	When $Z\notin\FV(A)$,
	$\textit{fmap}_{(\l Z^{+*}.A -> B_1)}
	= \l z.\l y. \l x. \textit{fmap}_{(\l Z^{+*}.B_1)} \; z \; (y \; x)$

	When $A \triangleq \forall X_1^{*}.A_1$,
	$\textit{fmap}_{(\l Z^{+*}.(\forall X_1^{*}.A_1) -> B_1)}
	= \textit{fmap}_{(\l Z^{+*}.A_1 -> B_1)}$

	\begin{singlespace}
	When $A \triangleq A_1 -> \cdots -> A_n -> \forall X_2^{*}.B_2'$,
	\vspace{-1.5ex}
	\[\textit{fmap}_{(\l Z^{+*}.(A_1 -> \cdots -> A_n -> \forall X_2^{*}.B_2') -> B_1)}
	= \textit{fmap}_{(\l Z^{+*}.(A_1 -> \cdots -> A_n -> B_2') -> B_1)} \]

	When $A \triangleq A_1 -> \cdots -> A_n -> B_2$,
	where $B_2$ is not an arrow type
	\vspace{-1.5ex}
	\begin{align*}
	  & \textit{fmap}_{(\l Z^{+*}.(A_1 -> \cdots -> A_n -> B_2) -> B_1)} \\
	=~& \l z.\l y. \l x.\;
	\textit{fmap}_{(\l Z^{+*}.B_1)} \; z \;
		(y \; (\l x_1.\ldots\l x_n. ~
		   x  &\!\!\!\!(\textit{fmap}_{(\l Z^{+*}.A_1)} ~ z ~ x_1)& \\
		     &&\!\!\!\!\vdots \qquad\qquad& \\
		     &&\!\!\!\!(\textit{fmap}_{(\l Z^{+*}.A_n)} ~ z ~ x_n)&
		\,) \,)
	\end{align*}
	\end{singlespace}

\end{itemize}
\end{proof}

TODO to give an idea that the derived fmaps are type correct in the above
proof give a example in Haskell accepted by GHC for each case
(already have it almost done put it in the repository)

What if there are type constructor variables?
Should still work, let's write down the rules

\begin{proposition}\label{prop:fixi:fmapFree}
If $\textit{fmap}_F:\forall X^{*}.\forall Y^{*}.(X -> Y) -> F\;X -> F\;Y$
exists, then
\begin{align*}
\textit{fmap}_F~\textit{id} &~=~ \textit{id} \\
\textit{fmap}_F~\textit{f} \;\circ\; \textit{fmap}_F~\textit{g}
&~=~ \textit{fmap}_F~(f\circ g)
\end{align*}
\end{proposition}\noindent
This is a well-known parametricity theorem on maps any instance of the type
$\forall X^{*}.\forall Y^{*}.(X -> Y) -> F\;X -> F\;Y$ satisfies
the two equations above. However, for the purpose of defining $\McvPr_{*}$,
we only need to know that there exists one such $\textit{fmap}_F$. That is,
\begin{proposition}\label{prop:fixi:fmapHom}
For any $F : +* -> *$, there exists

$\textit{fmap}_F:\forall X^{*}.\forall Y^{*}.(X -> Y) -> F\;X -> F\;Y$
such that
\begin{align*}
\textit{fmap}_F~\textit{id} &~=~ \textit{id} \\
\textit{fmap}_F~\textit{f} \;\circ\; \textit{fmap}_F~\textit{g}
&~=~ \textit{fmap}_F~(f\circ g)
\end{align*}
\end{proposition}
\begin{proof}
	You can check that each case
	in the proof of Proposition \ref{prop:fixi:fmap}
	satisfies the two equations above.
\end{proof}

\begin{proposition} For any $F : +* -> *$, there exists
$\unIn_F : \mu^{+}{*} F -> F(\mu^{+}{*} F)$ such that
$\unIn_F (\In_F\;t) -->+ t$.
\end{proposition}
\begin{proof}
Since we know that $\textit{fmap}_F$ exists by Proposition~\ref{prop:fixi:fmap},
we can define
\[ \unIn_F = \McvPr_{*}\;
            (\l\_.\l\textit{cast}.\l\_.\l x.\textit{fmap}_F\;\textit{cast}\;x)
\]

From Proposition~\ref{prop:fixi:fmapFree}, we know that
$\textit{fmap}_F\;\textit{id}\;x -->+ x$.
Thus,
\[ \unIn_F (\In_F\;t) -->+ \textit{fmap}_F\;\textit{id}\;t -->+ t \]
\end{proof}

\begin{align*}
A \rrarrow_{*} B &~\triangleq~ A -> B \\
F \rrarrow^{p\kappa -> \kappa'} G &~\triangleq~
	\forall X^\kappa.\forall Y^\kappa.
		(X \rrarrow_\kappa Y) -> F X \rrarrow_\kappa F Y \\
F \rrarrow^{A -> \kappa} G &~\triangleq~
	\forall i^A.\forall f^{A->A}. F\{i\} \rrarrow_\kappa F\{f\;i\}
\end{align*}

\[
\textsf{mon}_\kappa
  = \l X^{0\kappa}.X \rrarrow^\kappa X
\]

$\textsf{mon}_{+* -> *} F$ is the type of $\textit{fmap}_F$
where $F : +* -> *$.

$\textsf{mon}_{+(p* -> *)->(p* -> *)} F$ is the type of $\textit{fmap1}_F$
where $F : +(p* -> *)->(p* -> *)$.

if $\textsf{mon}_{+(p* -> *)->(p* -> *)}$ is inhabited
then $\unIn_F$ for any $F : +(p* -> *)->(p* -> *)$?

what about general case? if $\textsf{mon}_\kappa$ is inhabited
then $\unIn_F$ for any $F : \kappa$?

 %% \label{sec:fixi:theory}

