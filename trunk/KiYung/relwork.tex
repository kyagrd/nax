\chapter{Related work}\label{ch:relwork}
In this chapter, we discuss additional related work,
not discussed in the introduction or the related work sections
of \S\ref{sec:intro} and  \S\ref{sec:related}.
We discuss five categories of related work:
Mendler-style co-recursion schemes over co-data (\S\ref{sec:relwork:co}),
Mendler-style recursion schemes over multiple values (\S\ref{sec:relwork:mult}),
dependently-typed Mendler-style induction (\S\ref{sec:relwork:dep}), 
the use of sized-types to explain the termination of Mendler-style
recursion schemes (\S\ref{sec:relwork:sized}), and the comparison of
our Mendler-style approach to logical frameworks (\S\ref{sec:relwork:LF}).


\input{relwork_mcoit} %% sec:relwork:co

\input{relwork_mmult} %% sec:relwork:mult

\section{Mendler-style induction}
\label{sec:relwork:dep}
The dependently-typed version of primitive recursion is called induction.
We formulate Mendler-style induction over regular datatypes as follows.
\vspace*{-2em}
\begin{singlespace}
\[\begin{array}{ll}
\textbf{mind}_{*} \, :
& \!\!\forall (F:* -> *) (A: \mu_{*}F -> *). \\
& ~~ \big(\forall(r:*).\;(cast : r -> \mu_{*}F) \\
& ~\qquad\quad -> (call : (x:r) -> A\,(cast~x)) \\
& ~\qquad\quad -> (y: F\;r) -> A\,(\In_{*}(\textit{fmap}_{F}\;cast\;y)) \big) 
-> (z:\mu_{*}\,f) -> A\;z
\end{array}
\]
\[
\textbf{mind}_{*} ~ \varphi ~ (\In_{*}\;x)
  ~=~ \varphi~~\textit{id}~~(\textbf{mind}_{*}~\varphi)~~x \]
\end{singlespace}\noindent
The definition of Mendler-style induction $\textbf{mind}$ shows that induction
is essentially the same as the Mendler-style primitive recursion $\MPr$,
except that the type signature involves dependent types.
Note, the final answer type $(A\;z)$ is dependent on
the recursive argument $z:\mu_{*}F$.
Since $A: \mu_{*}F -> *$ expects a concrete recursive value,
we use $cast$ in the type signature of the $\varphi$ function
to cast $(x:r)$ and $(y:F\,r)$ into $\mu_{*}F$ values, so that
they can be passed to $A$.
In the type signature of $\textbf{mind}$, $cast$ comes before $call$
because the type signature of $call$ depends on $cast$.
When defining $\MPr$, $cast$ and $call$ can come in any order
since there is no dependency in the type signature of $\MPr$.

One important thing to notice about $\textbf{mind}_{*}$ is that
it is well-defined only over positive $F$, because we relied on
the existence of $\textit{fmap}_F$ to write its type signature.
It is an open question whether one can formulate a Mendler-style induction
that works for negative datatypes.

The idea behind $\textbf{mind}_{*}$ comes from
the discussion with Tarmo Uustalu. He described this on a whiteboard
when I met with him at the University of Cambridge in Fall 2011.

In the future work section (\S\ref{sec:futwork:mprsi}), we introduce 
even another Mendler-style recursion scheme, which is useful for
mixed-variant datatypes. The work of a Mendler stylist is never done.



\section{Type-based termination and sized types}\label{sec:relwork:sized}
\emph{Type-based termination} (coined by \citet{BartheFGPU04}) stands for
approaches that integrate termination into type checking, as opposed to
syntactic approaches that reason about termination over untyped term structures.
The Mendler-style approach is, of course, type-based.  In fact, the idea of
type-based termination was inspired by \citet{Mendler87,Mendler91}.
In the Mendler style, we know that well-typed functions defined using
Mendler-style recursion schemes always terminate.  This guarantee flows
from the design of the recursion scheme, where the use of higher-rank 
polymorphic types in the abstract operations enforce the invariants
necessary for termination.

\citet{abel06phd,Abel12talkFICS} summarizes the advantages of
type-based termination as follows:
\textbf{communication} (programmers can think using types),
\textbf{certification} (types are machine checkable certificates),
\textbf{a simple theoretical justification}
        (no additional complication for termination other than type checking),
\textbf{orthogonality} (only small parts of the language are affected,
        \eg, principled recursion schemes instead of general recursion),
\textbf{robustness} (type system extensions are less likely to
                        disrupt termination checking),
\textbf{compositionality}\footnote{This is not listed in Abel's thesis,
                                but comes from his invited talk in FICS 2012.}
        (one needs only types, not the code, for checking the termination), and
\textbf{higher-order functions and higher-kinded datatypes}
        (works well even for higher-order functions and non-regular datatypes,
        as a consequence of compositionality).
In his dissertation \cite{abel06phd} (Section 4.4) on sized types,
Abel views the Mendler-style approach as enforcing size restrictions
using higher-rank polymorphism as follows:
\begin{itemize}
\item The abstract recursive type $r$ in Mendler style corresponds to
        $\mu^\alpha F$ in his sized-type system (System \Fwhat),
        where the sized type
        for the value being passed in corresponds to $\mu^{\alpha+1} F$.
\item The concrete recursive type $\mu F$ in Mendler style corresponds to
        $\mu^\infty F$ since there is no size restriction.
\item By subtyping, a type with a smaller size index can be cast to
        the same type with a larger size index.
\end{itemize}
This view is based on the same intuition we discussed in
Chapter \ref{ch:mendler}. Mendler-style recursion schemes terminate, for
positive datatypes -- because $r$-values are direct subcomponents
of the value being eliminated. They are always smaller
than the value being passed in. Types enforce that recursive calls
are only well typed, when applied to smaller subcomponents.

Abel's System \Fwhat\ can express primitive recursion quite naturally
using subtyping. The casting operation $(r -> \mu F)$ in Mendler-style
primitive recursion corresponds to an implicit conversion by subtyping
from $\mu^\alpha F$ to $\mu^\infty F$ because $\alpha \leq \infty$.

System \Fwhat\ \cite{abel06phd} is closely related to
System \Fixw\ \cite{AbeMat04}. Both of these systems are base on
equi-recursive fixpoint types over positive base structures.
Both of these systems are able to embed (or simulate) Mendler-style
primitive recursion (which is based on iso-recursive types) via
the encoding \cite{Geu92} of arbitrary base structures into
positive base structures. In \S\ref{sec:fixi:data}, we rely on
the same encoding, denoted by $\Phi$, when embedding \MPr\ into System \Fixi.

Abel's sized-type approach provides good intuitions why 
certain recursion schemes terminate over positive datatypes.
But, it does not give a good intuition of whether or not
those recursion schemes would terminate for negative datatypes,
unless there is an encoding that can translate negative datatypes into
positive datatypes. For primitive recursion, this is possible (as we
mentioned above). However, for our recursion scheme \MsfIt, which is
especially useful over negative datatypes, we do not know of an encoding
that can map the inverse augmented fixpoints into positive fixpoints.
So, it is not clear whether Abel's the sized type approach based on
positive equi-recursive fixpoint types can provide a good intuition
for the termination behavior of \MsfIt.  In \ref{sec:futwork:mprsi},
we will discuss another Mendler-style recursion scheme (\mprsi), which
is also useful over negative datatypes and has a termination property
(not proved yet) based on the size of the index in the datatype.

\section{Logical Frameworks based on the $\lambda\Pi$-calculus}
\label{sec:relwork:LF}
A ``logical framework'', in a broad sense, refers to
any system that serves as ``a meta-language for
the formalization of deductive systems'' \cite{Pfe02LFintro}.
In a more narrow sense, logical frameworks are systems closely related to
to the Edinburgh Logical Framework (LF or ELF) \cite{Harper87}, which
use the $\lambda\Pi$-calculus as its specification language.
In this section, we discuss \emph{logical frameworks} in this more narrow sense.

The $\lambda\Pi$-calculus (\aka\ $\lambda\mathbf{P}$) is one of
corners in Barendregt's $\lambda$-cube \cite{Barendregt91} which is
adjacent to the simply-typed lambda calculus (STLC, or, $\lambda\!\!\rightarrow$).
The $\lambda\Pi$-calculus extends the STLC with dependent types,
but without polymorphism or functions from types to types (type operators).
The syntax of $\lambda\Pi$, extended with constants ($c$), is describe below:
\[
\begin{array}{lll}
\text{Kinds}         & K   & ::= \texttt{type} \mid \Pi x:A.K
        \\
\text{Type Families} & A,B & ::= c \mid \Pi x:A.B \mid \lambda x:A.B \mid AM
        \\
\text{Objects}       & M,N & ::= c \mid x \mid \l x:A.M \mid MN
\end{array}
\]
In logical frameworks, you can introduce new constants naming types and objects.
These are intended to represent datatypes such as natural numbers, lists, and
they may even involve higher-order abstract syntax. However, these constants
are merely syntactic descriptions, not necessary tied to any specific 
semantics or logical interpretations. That is, introducing constants
does not automatically supply any recursion schemes or
induction principles, as is done in functional languages or proof assistants
that support new datatypes as a feature. Each logical framework
supports its own meta-logic to give meanings to the logic (or, the language)
specified by introducing such constants. The choice of meta-logic can be either
relational (like a logic programming language),
functional (like a functional programming language), or something else.

Logical frameworks are very flexible for describing many
different logical systems (i.e. formalizing a language),
by using a two-layered approach of a minimal specification language ($\lambda\Pi$)
and a meta-logic. However, this two-layered approach is not ideal
as a programming system. One can model arbitrary
programming languages, giving them semantics
in the logical framework. But, the programming capability of the 
specifcation language and the meta-logic is limited.
In the remainder of this section, we discuss Twelf, whose meta-logic
is relational, and, Beluga and Delphin, whose meta-logics are functional.

\paragraph{}
Twelf\footnote{\url{http://twelf.org/}} is the most widely used
logical framework. In Twelf, you can define abstract syntax for datatypes
by introducing constants for types and objects involving those types.
For example, you can define natural numbers as follows.\footnote{
        Twelf examples are adopted from Boyland's Twelf Library on Github.\\
        $~~~~~~~$
        \url{https://github.com/boyland/twelf-library}}\vspace*{-2em}
\begin{singlespace}
\begin{verbatim}
  nat : type.       %%% define a type constant
  z : nat.          %%% define a constant for zero
  s : nat -> nat.   %%% define a constant for sucessor
\end{verbatim}
\end{singlespace}\noindent
At this point, the constants \texttt{z} and \texttt{s} are just typed syntax.
Introduction of constants is not associated with any semantics for
the constants, unlike the natively supported
inductive datatypes in Coq or Agda. So, there are no restrictions
on how these constants my be used, such as the positivity constraint on
inductive datatypes in Coq or Agda.
You can give meanings to the natural number constants by defining
inductive relations over them. For example, we can define addition
as a ternary relation over natural numbers, as follows:\vspace*{-2em}
\begin{singlespace}
\begin{verbatim}
  plus : nat -> nat -> nat -> type.
  plus/z : plus z Y Y.
  plus/s : plus (s X) Y (s Z)
        <- plus X Y Z.
\end{verbatim}
\end{singlespace}\noindent
If you are familiar with Prolog, you will notice that the
right-hand sides (after the colon) of \verb|plus/z| and \verb|plus/s|
look like a Prolog program defining addition. Twelf's meta-logic
is typed first-order relational logic.  At the type level, 
Twelf predicates are like pure Prolog programs with type-checking.
All computational issues,
such as termination checking, are present at the level of these
relational definitions (as opposed to the introduction of new constants).
Twelf has a termination checker
for inductive relations  (external to the type checker)
based on lexicographic subterm ordering over untyped terms.
In addition to the type signatures of the relations, one can optionally specify
input/output modes for each of their arguments, if necessary, to guide
the termination checker to consider only the input arguments
for termination.\footnote{There are various directives to guide
        checking input/output modes, coverage, and termination in Twelf.
        For further information, see the  documentations from its homepage.}

One cannot write higher-order relations natively in Twelf
because Twelf's meta-logic is first-order, not higher-order.
To write a program using higher-order functions in Twelf, one has to model
one's own object language that supports higher-order functions, and program
within that object language, rather than programming in twelf's meta-logic.
We summarize the steps necessary to program using higher-order functions in Twelf:
\begin{itemize}
\item[(1)] Define an object language syntax
        (as you define the syntax \texttt{z} and \texttt{s} for natural numbers)
        with bindings (this is done by HOAS), applications, and whatever
        you need to express higher-order functions.
\item[(2)] Define the evaluation semantics of your object language using
        inductive relations (\ie, write an evaluation relation for
        your object language in a Prolog-like way).
\item[(3)] Write programs in the object language by putting
        together pieces of the syntax you defined in (1).
\item[(4)] Finally, you can evaluate your program by reasoning based on
        the evaluation relation defined in (2).
\end{itemize}
This is clearly not ideal if what you wanted was just to ``program''
with higher-order functions in a type-safe way, possibly with some
termination guarantees. You don't always want to reason about
the meta-theory of the object language in general.

\paragraph{}
Beluga \cite{Pie10} is similar to Twelf, but it is closer to
a functional language since the inductive definitions are functional,
rather than relational. Beluga supports higher-order functions, unlike Twelf.
One can write a natural number addition function in Beluga 
as follows:\footnote{
        Adopted from the Beluga tutorial.
        \url{http://complogic.cs.mcgill.ca/beluga/} }\vspace*{-2em}
\begin{singlespace}
\begin{verbatim}
  rec add : [. nat ] -> [. nat ] -> [. nat ]
    = fn x => fn y =>
      case x of
      | [. z ]   => y
      | [. s N ] => let [. R ] = add [. N ] y in [. s R ]
      ;
\end{verbatim}
\end{singlespace}\noindent
Types like \verb|{nat}| and \verb|nat -> nat| are called
representation-level types. So, objects like \verb|z| and \verb|s|
are called representation-level objects.
Types like \verb|[. nat ]| and \verb|[. nat ] -> [. nat ]| are called
computation-level types.
Note, the new representation-level variable binding \verb|R| in
the second case branch of the \verb|add| function definition.
One cannot write \verb|[. s (add [. N ] y) ]| because \verb|s|
expects a representation-level object as its argument.
In Twelf-style logical frameworks, representation-level types are
inhabited only by $\eta$-long $\beta$-normal representation-level objects,
which do not include application forms of computational-level objects.

More generally, computation-level types can have the form \verb|[g . t]|
where \verb|g| is a context object and \verb|t| is a representation-level type.
One of the Beluga's unique features is supporting pattern matching over
computational objects with contexts, and also coverage checking of those
patterns. Computational types with the empty context, of the form
\verb|[. t]|, are inhabited by closed values, which do not involve
any free (representation-level) variables.
Since Beluga has explicit access to context objects, we think
it can express what \MsfIt\ can express, and in addition, it can
also express what \textbf{openit} (\S\ref{sec:openit}) can express.

One can also write higher-order functions (\eg, map)\footnote{
        A map function over natural number lists is
        given in the Beluga tutorial. }
in Beluga almost the same way one does in functional programming languages,
except perhaps, for the tedious representation-level bindings 
(\eg, \verb|R| in the \verb|add| function above).
In regards to higher-order functions, Beluga is in a much better position
than Twelf. Recall that, in Twelf, one needs to model a whole new
functional language by describing its semantics with inductive relations
in order to express higher-order functions.

Termination is not type based in Beluga either. Like Twelf, it needs
an external termination (or totality) checker, but its prototype
implementation currently lacks such a checker.
We suspect one of the reasons why the Beluga implementation does not implement
a termination checker yet is due to the difficulty of checking termination
of higher-order functions. The syntactic approaches to termination,
used by logical frameworks based on first-order meta-logic,
may fail to check termination for many higher-order functions.

\paragraph{}
Delphin \cite{pos08phd} has goals similar to Beluga,
supporting functional programming rather than relational reasoning.
For example, the addition function over natural numbers can be defined
in Delphin as follows.\vspace{-2em}
\begin{singlespace}
\begin{verbatim}
fun plus : <nat> -> <nat> -> <nat> 
  = fn <z>   <M> => <M>
    |  <s N> <M> => let val <x> = plus <N> <M> in <s x> end
    ;
\end{verbatim}
\end{singlespace}\noindent
Although both Beluga and Delphin support similar features with similar syntax,
but their theoretical foundations differ \cite{Pie10} on how they treat contexts.
Delphin cannot distinguish open values from closed values as is done in Beluga,
since Delpin does not explicitly manage contexts.
\citet{Pie10} also points out that Delphin tries to reuse Twelf's infrastructure
as much as possible. For instance, the termination checker of Delphin is
based on lexicographic subterm ordering, which is also the case in Twelf.

Although Delphin and Beluga do support higher-order functions,
they do not support polymorphism. That is, you can only write
monomorphic (possibly dependently typed) functions. Recall that,
in $\lambda\Pi$, you can only index type families by terms, not types.
Indexing by types would support polymorphism.
This is inconvienent for programming higher-order functions,
because many higher-order functions are polymorphic in nature -- users
need to duplicate their definitions for at each differenet type needed.
%% One may work around this limiation by using type-representation objects
%% -- define constant objects that represent types and pass them around instead
%% of passing types.

