\documentclass[11pt,twocolumn]{article}

\usepackage{amsfonts,euler}
\usepackage{MnSymbol}

\usepackage[all]{xy}
\UseComputerModernTips
\xyoption{knot}

\usepackage[a4paper,includehead,includefoot,headheight=13.6pt,
  headsep=4mm,top=1cm,bottom=1cm,left=2cm,right=2cm]{geometry}

\usepackage{float}
\floatstyle{ruled}
\restylefloat{figure}

\usepackage{dashrule}

\usepackage{multirow}

\usepackage{color}
\definecolor{grey}{gray}{.85}

\setcounter{secnumdepth}{4}
\renewcommand{\theparagraph}{\arabic{subsection}.\arabic{subsubsection}.\arabic{paragraph}}

%%% BEGIN macros
\newenvironment{myitemize}
  {\begin{list}{$\bullet$}
  {\setlength{\topsep}{1pt}
   \setlength{\partopsep}{1pt}
   \setlength{\itemsep}{0pt}
   \setlength{\parsep}{0pt}
   \setlength{\leftmargin}{1em}
   \setlength{\labelwidth}{.5em}}}
  {\end{list}}
\newenvironment{myindentitemize}
  {\begin{list}{$-$}
  {\setlength{\topsep}{2pt}
   \setlength{\partopsep}{2pt}
   \setlength{\itemsep}{2.5pt}
   \setlength{\parsep}{2.5pt}
   \setlength{\leftmargin}{2.125em}
   \setlength{\labelwidth}{1.625em}}}
  {\end{list}}
\newenvironment{myquote}
  {\begin{list}{}
  {\setlength{\topsep}{2pt}
   \setlength{\partopsep}{2pt}
   \setlength{\itemsep}{2.5pt}
   \setlength{\parsep}{2.5pt}
   \setlength{\rightmargin}{1em}
   \setlength{\leftmargin}{1em}
   \setlength{\labelwidth}{.5em}}}
  {\end{list}}
\newenvironment{mybigitemize}
  {\begin{list}{$\bullet$}
  {\setlength{\topsep}{2pt}
   \setlength{\partopsep}{2pt}
   \setlength{\itemsep}{2.5pt}
   \setlength{\parsep}{2.5pt}
   \setlength{\leftmargin}{1em}
   \setlength{\labelwidth}{.5em}}}
  {\end{list}}
\newenvironment{btritemize}
  {\begin{list}{\btr}
  {\setlength{\topsep}{2pt}
   \setlength{\partopsep}{2pt}
   \setlength{\itemsep}{2.5pt}
   \setlength{\parsep}{2.5pt}
   \setlength{\leftmargin}{1em}
   \setlength{\labelwidth}{.5em}}}
  {\end{list}}
\newcounter{CC}
\newenvironment{resenumerate}
  {\begin{list}{[\textbf{\arabic{CC}]}}
  {\usecounter{CC}
   \setlength{\topsep}{2pt}
   \setlength{\partopsep}{2pt}
   \setlength{\itemsep}{2.5pt}
   \setlength{\parsep}{2.5pt}
   \setlength{\leftmargin}{1.65em}
   \setlength{\labelwidth}{1.15em}
 }}
  {\end{list}}

\newcommand{\mysf}{\small\sf}
\newcommand{\mytextsf}[1]{\textsf{\small #1}}
\newcommand{\erc}{{\small\sf MaStrPLan}}
\newcommand{\ERC}{Mathematical Structures for Type Theories,\\[-.5mm] Logical
  Systems, and Programming Languages}

\newcommand{\hide}[1]{}
\newcommand{\hidebib}[1]{}%{#1}

\newcommand{\hl}[1]{#1}%{\emph{#1}}

\newcommand{\eg}{\emph{eg.}}
\newcommand{\Eg}{\emph{Eg.}}
\newcommand{\vs}{\emph{vs.}}
\newcommand{\ie}{\emph{ie.}}
\newcommand{\viz}{\emph{viz.}}
\newcommand{\etc}{\emph{etc.}}
\newcommand{\etal}{\emph{et al.}}
\newcommand{\cf}{\emph{cf.}}

\newcommand{\lcalculus}{\mbox{$\lambda$-calculus}}
\newcommand{\SysL}{$L$}%{$\boldsymbol L$}
\newcommand{\SystemF}{\mbox{System~$F$}}
\newcommand{\SystemFi}{\mbox{System~$F_i$}}
\newcommand{\SystemFomega}{\mbox{System~$F_\omega$}}
\newcommand{\LC}{\mbox{$LC$}}

\newcommand{\btr}{$\blacktriangleright$}

\newcommand{\reqpsize}{8.113395cm}%{\columnwidth}

\newcommand{\req}[2]{\begin{center}\colorbox{grey}{\begin{minipage}{\reqpsize} 
  \mytextsf{Research question}\hfill$^{\mbox{\scriptsize see (#1) }}$\\[-5.5mm]
  \begin{btritemize}
  \item #2
  \end{btritemize}
\end{minipage}}\end{center}}

\newcommand{\reqs}[2]{\begin{center}\colorbox{grey}{\begin{minipage}{\reqpsize}
  \mytextsf{Research questions}\hfill$^{\mbox{\scriptsize see (#1) }}$\\[-5.5mm]
  \begin{btritemize}
  \item #2
  \end{btritemize}
\end{minipage}}\end{center}}

\newcommand{\rep}[2]{\begin{center}\colorbox{grey}{\begin{minipage}{\reqpsize}
  \mytextsf{Research pathway}\hfill$^{\mbox{\scriptsize see (#1) }}$\\[-5.5mm]
  \begin{btritemize}
  \item #2
  \end{btritemize}
\end{minipage}}\end{center}}

\newcommand{\Set}{{\boldsymbol{\mathscr S}}}
\newcommand{\scat}[1]{\mathbb{#1}}
\newcommand{\cat}[1]{\mathscr{#1}}
\newcommand{\op}{\circ}
\newcommand{\Id}{\mathrm{Id}}
\newcommand{\Di}{\mathrm{Di}}

\newcommand{\qed}{\Box}
\newcommand{\logo}
{\mbox{\tiny$\xymatrix@C=3pt@R=4.5pt{
    & \qed \ar@{-}[d] 
    \ar@/_.5em/@{-}[ddl]<-.25em>
    \ar@/^.5em/@{-}[ddr]<.25em>
    & 
    \\
    & \ar@{-}[dl] \qed \ar@{-}[dr] & 
    \\
    \ar@/_.5em/@{-}[rr]<.125em>
    \qed & & \qed
  }$}}
%%% END macros

\usepackage{fancyhdr}
\pagestyle{fancy}
\lhead{\emph{Marcelo Fiore}}
\chead{Part~B1}
\rhead{\erc}
\lfoot{}
\cfoot{\thepage}
\rfoot{}

\usepackage[compact]{titlesec}

\usepackage{compactbib}

\usepackage{setspace}
\setstretch{.9}

\usepackage{times}

\makeatletter
\renewcommand\@biblabel[1]{#1}
\makeatother

\renewcommand{\thesection}{\arabic{section}}
\renewcommand{\thesubsection}{\arabic{subsection}}

\def\contentsname{\large Contents\\[-7.5mm]}
\usepackage{minitoc}
\tightmtctrue

\usepackage[draft]{hyperref}
\hypersetup{pdftitle={\ERC},pdfauthor={Marcelo Fiore}}

\newcommand{\obullet}{\mbox{$\raisebox{-.85mm}{\huge$\circ$}\hspace*{-2.6mm}\bullet$}}

\renewcommand{\ostar}{\mbox{$\raisebox{-.85mm}{\huge$\circ$}\hspace*{-2.75mm}\star$}}

\setlength{\intextsep}{5pt}

\begin{document}
\setlength{\abovedisplayskip}{5pt}
\setlength{\belowdisplayskip}{5pt}

\setcounter{page}{0}
\thispagestyle{plain}
\twocolumn[\begin{@twocolumnfalse}
{\large European Research Council}

\vspace*{7.5mm}
\begin{center}\Large
ERC Advanced Grant\\
Research Proposal\\
(Part B1)
\end{center}

\vspace*{2.5mm}

\begin{center}\bfseries\LARGE
Mathematical Structures for Type Theories,\\
Logical Systems, and Programming Languages\\
\vspace*{10mm}
\Large\bfseries\sf 
MaStrPLan\\
  \logo
\end{center}

\vspace*{2.5mm}

\begin{center}\large
Principal Investigator: Marcelo Fiore\\
\vspace*{1mm}
Host Institution: University of Cambridge\\
\vspace*{1mm}
Proposal duration: 60 months
\end{center}

\vspace*{5mm}

\renewcommand{\abstractname}{\bfseries\large Summary}
\begin{abstract}\normalsize
Headed by Principal Investigator Marcelo Fiore, the project builds an
international group of world-leading experts in theoretical and applied
computer science.  This team will be dedicated to combined research in
Category Theory, Mathematical Logic, Programming Theory, and Type Theory.
Specifically, to understand and develop formal languages and mathematical
models for computer programming and mathematical proof. 

{\color{red}TO BE FINISHED}

We aim to produce several artifacts: a framework to analyse and synthesise
type theories; mathematical theories and models for computational phenomena;
formal calculi for proof and/or computation; and experimental high-level
programming languages.
\end{abstract}

\end{@twocolumnfalse}]

\clearpage
\chead{Part\,B1(a)}
\twocolumn[\begin{@twocolumnfalse}
	\hfill{\bfseries\Large 
      Extended Synopsis of the scientific proposal
  }\hfill\null\\
\end{@twocolumnfalse}]

\subsection{Proposal}
\label{Proposal}

\paragraph*{General aim.}

Headed by Principal Investigator Marcelo Fiore, the project builds an
international group of world-leading experts in theoretical and applied
computer science.  This team will be dedicated to combined research in
Category Theory, Mathematical Logic, Programming Theory, and Type Theory.
Specifically, to understand and develop formal languages and mathematical
models for computer programming and mathematical proof. 

\paragraph*{Working thesis.}

The research is driven by the thesis that
\begin{myquote}
\item
Languages for computer programming and languages for mathematical proof are of
the same character.
\end{myquote}
This view was first developed in the 1960s and is now central to research in
programming languages and constructive mathematics.

An aspect of the thesis that is fundamental to us here is typically referred
to as the Proofs-as-Programs or Propositions-as-Types correspondence.  It can
be impressionistically presented as follows
\[
  \mbox{proof : Proposition} 
  \enspace \approx \enspace 
  \mbox{program : Type} 
\]
and is intuitively explained along these lines: to give a constructive proof
of a proposition is to construct a program that certifies the statement, while
to type a program is to establish a property of that program.

Research stemming from the Propositions-as-Types correspondence has been very
fruitful.  Many of its mathematical theories have been implemented as computer
systems, and used academically and industrially.  Cases in point are:
programming languages for high-assurance code and proof assistants for
computer-aided verification; examples are Haskell, %~\cite{Haskell} 
an industrial-strength functional programming language, and Coq, %~\cite{Coq} 
a proof assistant in which bodies of mathematics are being verified.

\paragraph*{The future.}

The transfer of mathematical theories to computer systems has required
both technical/theoretical and technological/applied progress.  In turn,
this has necessarily demanded increasingly specialised, and to some extent
fragmentary, research.  Importantly, there is now a consolidated body of
work and expertise.  How will the field continue to advance?  By
considering new directions and challenges, of course.  But, it is our firm
view that, %as in the early days of the subject, 
it will be essential to do so generating questions and tackling problems
from the perspectives of a variety of research areas.  This is %indeed 
what we put forward here.

\paragraph*{Our philosophy.}

Two disciplines are clearly involved in investigating the Proofs-as-Programs 
correspondence: Mathematical Logic and Programming Theory.  This is only half
of the picture.  The full picture also involves Type Theory and Category
Theory.  Figure~\ref{ResearchAreas} 
\begin{figure}[h]
\caption{Research areas and interactions.}
\vspace*{2mm}
\begin{center}
\hspace*{.5mm}
\xymatrix@R=25pt@C=15pt{
& 
\raisebox{7mm}{\fbox{\txt{\small Mathematical\\\small Logic}}}
\ar@/_1em/@{<->}[ddl]<-1em>|-
  {\txt{\scriptsize Propositions\\ \raisebox{1mm}{\scriptsize as Types}}}
\ar@/^1em/@{<->}[ddr]<1em>|-
  {\txt{\scriptsize Proofs as\\\raisebox{1mm}{\scriptsize Programs}}} 
& 
\\
& 
\ar@{<->}[dl]|-
  {\txt{\scriptsize Internal\\\raisebox{1mm}{\scriptsize Languages}}}
\ar@{<->}[dr]|-
  {\txt{\scriptsize Denotational\\\raisebox{1mm}{\scriptsize Semantics}}} 
\ar@{<->}[u]|-
{\txt{\scriptsize Mathematical\\\raisebox{1mm}{\scriptsize Models}}}
\fbox{\txt{\small Category\\\small Theory}}
& 
\\
  \fbox{\txt{\small Type\\\quad\small Theory\quad\null}}
\ar@/_1em/@{<->}[rr]|-
  {\txt{\scriptsize Formal\\\raisebox{1mm}{\scriptsize Languages}}}
& & 
\fbox{\txt{\small Programming\\\small Theory}}
}
\end{center}
\vspace*{-2mm}
\label{ResearchAreas}
\end{figure}
gives a schematic view of the interactions between these four research areas
in this context.  Each of them is complementary to the others.  Together, as a
unified whole, they have shaped fields of computer science and mathematics.
It is within this framework that we propose cross-cutting research in Category
Theory, Mathematical Logic, Programming Theory, and Type Theory.  We contend
that an approach neglecting any one of them is to the detriment of the others;
missing the depth and richness of the subject and, crucially, missing
opportunities for research and development.  

\paragraph*{Main goals.}

The motivations for our research proposal are specifically as follows. 
\begin{myitemize}
\item[\raisebox{.75mm}{\tiny$\bigstar$}]\hspace*{-2mm}
  To identify the next-generation framework of Type Theory relevant to
  computer science and mathematics.
\item[\raisebox{.75mm}{\tiny$\bigstar$}]\hspace*{-2mm}
  To exercise and test Category Theory both as a unifying and as a
  foundational mathematical language.
\item[\raisebox{.75mm}{\tiny$\bigstar$}]\hspace*{-2mm}
  To broaden the current role of Mathematical Logic in computation.
\item[\raisebox{.75mm}{\tiny$\bigstar$}]\hspace*{-2mm}
  To design, implement, and experiment within Programming Theory, looking
  into the languages of the future.
\end{myitemize}

We particularly aim to contribute with: a framework to analyse and synthesise
type theories; mathematical theories and models for varieties of computational
phenomena; formal calculi for proof and/or computation; and experimental
high-level programming languages.

\paragraph*{Research programme.}

The research programme has been precisely conceived to target the goals
above.  It is organised in four strands as outlined below.
\begin{myitemize}
\item[{\bfseries 1\enspace Foundations:}]\mbox{}\enspace\thinspace 
  \emph{A comprehensive research programme on the metamathematics of type
    theories.}

  \vspace*{1mm}
  We will address the fundamental question of what type theories are, which
  will also lead us to rethink them.  Our approach aims at an algebraic
  framework that will generalise to type theories all aspects of our current
  understanding of algebraic theories.  The scale at which this will be
  attempted is unprecedented, systematically exploring a wide spectrum of
  key type-theoretic features.
  \vspace*{1mm}

\item[{\bfseries 2\enspace Models:}]\mbox{}\enspace\thinspace
  \emph{Study of mathematical models for type theories and logical systems.}

  \vspace*{1mm}
  We will tackle semantic problems at the forefront of current understanding.
  %
  In the context of type theories, we will conduct investigations
  concerning a main problem in the area: to reconcile extensional equality
  and intensional identity (\ie~the mathematical and computational notions
  of sameness).  
  %
  In the context of logical systems, we will develop both calculi suggested by
  mathematical models and models for so far intractable calculi,
  \eg~encompassing aspects of resource management and computational effects.  
  \vspace*{1mm}%%%HACK!!!
  
\item[{\bfseries 3\enspace Calculi:}]\mbox{}\enspace\thinspace
  \emph{Development of formalisms of deduction as internal languages of
    mathematical models.}
  
  \vspace*{1mm}
  We will aim at evolving the term and type structure of current type
  theories, which in essence has not changed since the late 1960s.  Pursuing
  the view that evolution will come from mathematical input, we will
  research type theories as formal languages of (higher-dimensional)
  categorical structures.  
  \vspace*{1mm}

\item[{\bfseries 4\enspace Programming:}]\mbox{}\enspace\thinspace
  \emph{Design and implementation of novel computational languages.}

  \vspace*{1mm}
  Guided by mathematical theories and pragmatics, we will look into the
  engineering of principles and concepts for programming, exporting them to
  language designs and implementations.  We will specifically consider
  experimental languages, \eg~supporting indexed data,
  computational effects, and metaprogramming.  
\end{myitemize}

\paragraph*{The team.}

The research team will be led by Principal Investigator~(PI) \emph{Marcelo
Fiore} (%Computer Laboratory, 
University of Cambridge): 
    a leading expert in applications of category theory to computer science,
    including abstract algebra, concurrency theory, programming-language
    semantics, sheaf theory, and type theory.  
He will be backed up by an international group of three Senior Visiting
Researchers~(SVRs): \emph{Pierre-Louis Curien} (%Laboratoire PPS, 
Univerist\'e Paris Diderot - Paris~7): 
  a leading expert in programming-language semantics, with contributions into
  category theory, logic, %rewriting theory, 
  and type theory; 
  \emph{Peter \mbox{Dybjer}} (%Department of Computer Science, 
Chalmers University of Technology): 
  a leading expert in type theory, including its connections with category
  theory and logic, and its implementation in proof assistants; and 
  \emph{Tim Sheard} (%Department of Computer Science, 
Portland State University): 
  a leading expert in pro\-gram\-ming-language design and implementation,
  encompassing functional and metaprogramming systems. 
%
The PI is to collaborate with the SVRs by mutual research visits, email, and
conference calls.
%
The team is completed by three outstanding Research Associates~(RAs) to be
based at the Computer Laboratory (University of Cambridge) under the PI.  They
are
  \emph{Ki Yung Ahn} (%Department of Computer Science, 
Portland State University): 
  a PhD student of Sheard writing up his dissertation on the design and
  implementation of a language for indexed programming;
  \emph{Nicola \mbox{Gambino}} (%Dipartimento di Matematica e Informatica, 
Universit\`a degli Studi di Palermo):  
  an established researcher in the areas of type theory and category theory,
  with whom the PI has already collaborated; and 
  \emph{Guillaume Munch-Maccagnoni} (%Laboratoire PPS, 
Univerist\'e Paris Diderot - Paris~7): 
  a PhD student of Curien writing up his dissertation on the
  Propositions-as-Types correspondence in classical logic.
Currently, the two junior RAs lack future funding.  All RAs would be ready
to join the project from the start.

\subsection{Origins and influences}
\label{Origins}

We sketch the background of our scientific framework, 
touching upon %mentioning 
developments that led to the holistic conception of
Figure\,\ref{ResearchAreas}.


\paragraph*{Logic and computation.}

Computer science was born as a branch of mathematics, specifically
mathematical logic.  Its inception was Hilbert's %~\cite{?}
Decision Problem asking whether there is an algorithmic procedure for
deciding mathematical statements.  Negative answers were provided
independently by Church %~\cite{Church1936} 
and Turing, %~\cite{Turing},
giving birth to the mathematical theory of computation.  Their approaches
founded different branches of theoretical computer science.  
The %On one hand, the 
line of development starting with Church's {\lcalculus} %~\cite{?} 
is concerned with prototypical computational languages that are used to study
high-level programming languages.  
%On the other hand, 
Turing's machines %~\cite{?} 
are the most widely used model for analysing computational complexity.

\paragraph*{Type theory and logic.}
\label{SectionATypeTheoryAndLogicParagraph}

Type Theory, as we %now 
know it, arose from Church's typing of
his \lcalculus, that yielded the 
Simply-Typed Lambda Calculus. %~\cite{Church1940}.
This was a natural step under the naive interpretation of
\mbox{$\lambda$-abstraction} as defining functions.  In this
context,
Howard %~\cite{Howard1969}
made a fundamental discovery relating Mathematical Logic to Type Theory.
This is the Prop\-o\-si\-tions-as-Types correspondence, referred to
in Section~\ref{Proposal}.

In another %very 
important direction, Church introduced 
an axiomatization of Higher-Order Logic, the 
Simple Theory of Types, %~\cite{Church1940},
within a Simply-Typed Lambda Calculus.  In doing so, he adopted a
radically new perspective, shifting the status of the Simply-Typed Lambda
Calculus from that of a `language' to a `metalanguage'; \ie~a language in
which it is possible to represent and work with other languages. 

\paragraph*{Category theory, logic, and type theory.}

The connection between Category Theory, Logic, and Type Theory was
initiated by Lawvere %~\cite{LawvereAinF} 
and Lambek. %~\cite{LambekI}. 
%
Lawvere's insight was to understand logical connectives and type constructors
as categorical structures arising from adjoint functors.  It is now
well-understood that the Simply-Typed Lambda Calculus with products provides
an internal language for Cartesian Closed Categories; \viz~it is the
calculus of all such models.  This view extends the Propositions-as-Types
correspondence to a trinity consisting of the Simply-Typed Lambda Calculus
with products, Cartesian Closed Categories, and Propositional
Intuitionistic Logic.

\paragraph*{Type theory and programming.}

The Simply-Typed Lambda Calculus is a prototypical functional programming
language.  In this light, the problem of type inference became of central
practical importance.  Milner %~\cite{Milner1978} 
solved it for the Simply-Typed Lambda Calculus with parametric polymorphism.  
 
Polymorphism in programming was introduced by Strachey %~\cite{Strachey1967}.
and formalised by Reynolds %~\cite{Reynolds} 
in his Polymorphic Typed Lambda Calculus.
Strikingly, this system had already been proposed by
Girard %~\cite{GirardSystemF} 
as the type-theoretic counterpart of Second-Order Propositional Logic. 
In fact, he %Girard
had also introduced {\SystemFomega}, the type-theoretic counterpart of
Higher-Order Propositional Logic.  
This is the system %{\SystemFomega} 
at the core of the Haskell programming language. 

These type systems aim at logical foundations and thus only introduce
terminating computations.  As for Turing-complete languages, 
%motivated by model-theoretic studies of the {\lcalculus} 
Scott %~\cite{ScottTCS}
introduced an extension of Typed Combinatory Logic with a fixpoint combinator. 
Plotkin %~\cite{PlotkinLCF} 
studied the calculus as a programming language, enriching type
theory with concerns from programming; \viz~the distinction between
call-by-value and call-by-name. %~\cite{PlotkinCBVCBN}.

\subsection{Questions and pathways for research}

We turn attention to topics of current research, presenting the basis for
the research methodology and plan outlined in the next section.

\paragraph*{Foundations.}

Type theories are ubiquitous in computer science and mathematics.  However,
the following fundamental question remains open:
\req{\ref{AlgebraicTypeTheoryParagraph}}
{What is a type theory?}

\paragraph*{Dependent types.}

Dependent type theory was introduced by de~Bruijn %~\cite{deBruijn} 
as an extension of simple type theory that allows types to be indexed by
other types.  Such objects abound in computer science and mathematics.  
Two essential constructions on dependent judgements are: dependent sums and
dependent product types.  %(see~\eg~\cite{Jacobs}).  
Under the Propositions-as-Types correspondence, they amount to existential and
universal quantification.  A crucial problem 
is to:
\rep{\ref{IntensionalTypeTheoryParagraph}}
  {Investigate equality and identity in dependent type theory.}

The passage from sum and product types to their dependent versions required new
type theories.  Categorical models suggest generalisations and the following: 
\rep{\ref{CalculiSubsection}}
  {Develop type theories from mathematical models.}

\paragraph*{Mathematical universes.}

The construction of new mathematical universes of discourse from old ones is
basic in semantics.  It is thus natural to ask:
\reqs{\ref{MethodologyMathematicalUniversesParagraph}}
  {What mechanisms are there for changing from a type theory to another one
    as universes of discourse? 
%
%    Can this be done while maintaining the relevant computational properties
%    and then incorporated into mechanical proof assistants?
}
%
One technique to do so is the presheaf construction.  

\paragraph*{Indexed programming.}

Discrete %In %their most basic %elementary discrete form, 
pre\-sheaves can be found in programming languages as
indexed data\-types, programming with which will be referred to as indexed
programming.  

Indexed programming developed from the practical needs of supporting data
structures with strong invariants and the experimentation with
dependently-typed programming languages. %~\cite{Cayenne,Epigram}
These two views pull the design of languages for indexed programming in
opposite directions.  One is lead to:
\rep{\ref{IndexedProgrammingParagraph}} 
  {Develop foundational type theories for indexed datatypes.  
   % 
   Design and implement indexed programming languages from these and
   pragmatics.}

\paragraph*{Resources, effects, modalities.}

Girard %~\cite{GirardLinearLogic} 
and Moggi %~\cite{MoggiLambdaC} 
proposed two important analyses of computation, respectively for resources and
effects.  The former in the contexts of logic and proof theory; the latter in
that of denotational semantics and category theory.  Both had tremendous
impact in programming-language theory.  

The predominant model-theoretic view of resources and effects is as
categorical structures arising at opposite sides of an adjunction.  This is
not so in all models and the following remains unanswered.
%
\req{\ref{PolarisationParagraph}}
  {How can resources and effects be reconciled and unified?}
%
The answer seems to involve the logical notion of
polarisation. %~\cite{Andreoli}. 
This further enriches the computational scenario with eager \vs~lazy modes of
computation and data structures.  We incorporate this into the following.
%
\rep{\ref{PolarisationParagraph}\,\&\,\ref{ProgrammingEffectsParagraph}}
  {Study and develop the theory of resource management, computational
    effects, and polarisation.  Percolate this down into the design of
    programming languages.}

\paragraph*{Sequent calculi.}

A theme orthogonal to the logics under consideration is whether they are
specified in natural deduction or sequent calculus style. 

Work on the Propositions-as-Types correspondence has focussed on natural
deduction systems, mainly because there is as yet no established syntax for
sequent calculi.  Two questions at the core of this situation are:
%
\reqs{\ref{ProgrammingEffectsParagraph}}
  {What is the categorical algebra of classical sequent calculi? %}
%
%\req{\ref{ProgrammingEffectsParagraph}}
  %{
  What can the proof theory of sequent calculi do for programming?}

\subsection{Research methodology and plan}

We expand the research programme outlined in Section\,\ref{Proposal}, giving
just a taste of the full scientific proposal.

\subsubsection{Foundations}
\label{Foundations}

\paragraph{Algebraic Type Theory.}
\label{AlgebraicTypeTheoryParagraph}

Many-sorted algebra provides the most basic notion of type theory, with no
binding operators and no type constructors.  Its modern understanding is
through an Algebraic Trinity of three interrelated perspectives: Equational
Logic, Categorical Algebra, and Universal Algebra.
%(see~\cite{Birkhoff,LawvereThesis,Linton}).
Each of these approaches gives an important different viewpoint of the
subject.

We will \hl{investigate} an analogous development for varieties of Type
Theory.  To make substantial prog\-ress, we will systematically explore a broad
spectrum of key features present in type theories; to wit, term and type
variable binding with simple, polymorphic, and dependent typing, in
combination with the orthogonal notion of linearity.  

This research rests on fundamental work, old and recent, of
Fiore~\etal\
%~\cite{FiorePlotkinTuri,FioreLICS08,FioreHurLMCS,FioreSecOrdEqLog,FioreMahmoud,FioreICALP2012}.
Two salient \hl{goals} are: To develop a mathematical algebraic framework
for the semantics of language phrases, whereby free models universally
characterise the abstract syntax of the language; and to synthesise
metalanguages for type theories in the form of formal systems for equational
deduction, that are sound and complete for the model theory.  The outcome of
this work will be new unified metamathematical foundations for Type Theory.

\subsubsection{Models}
\label{Models}

\paragraph{Equality and identity in dependent type theory.}
\label{IntensionalTypeTheoryParagraph}

Central to a type theory is its associated equational theory, or judgemental
equality.  In Pure Intensional Type Theory, this is generated by rules of
computation.  Extensional Type Theory refers to extensions of Pure Intensional
Type Theory with extensionality features.  These may be given by weak rules of
extensionality or by a strong rule of equality reflection (of an intensional
notion of equality internal to the type theory as a judgemental equality).  

Conceptually, one may regard a type theory as a deduction system extended with
type constructors.  As such, a type theory is to be thought not just as a
single universe of discourse, but rather as a variety of universes of
discourse.  From this perspective, it is important to understand the
interaction of modularly extending type theories.  A key notion is that of
conservative extension.  Informally, an extension of a type theory is
conservative when it does not modify the judgemental equality of the extended
type theory.

We will pursue \hl{research} on the so far unexplored problem of
conservativity in dependent type theory.  We will \hl{aim} at being
comprehensive, targeting Intensional and Weak/Strong Extensional type theories
with dependent sums, dependent products, inductive types, and universes.  Our
\hl{goal} is to deepen the current understanding of equality/extensionality
and identity/intensionality in type theory via a model-theoretic framework for
conservative extension results.  

The full scientific proposal expands on the above in connection to an
interesting recent approach to extensionality in Martin-L\"of Type Theory,
\viz~Voevodsky's Univalence Axiom. %~\cite{UnivalentProgramme}. 
Those considerations \hl{aim} to investigate a main problem in type
theory: To reconcile Intensional and Extensional Type Theory in a
computational framework.

\paragraph{Mathematical universes.}
\label{MethodologyMathematicalUniversesParagraph}

Our \hl{research} activity centered on exploring new models will investigate
pre\-sheaf, orthogonality, sheaf, glueing, and forcing constructions on type
theories.  Topos Theory %~\cite{Elephant} 
will be used as a main source of inspiration and guidance.  Our \hl{goal}
is to build a mathematical theory of constructions on type theories that
produce new type theories from old ones, together with an interpretation
(or compilation) of the latter into the former.

\paragraph{Polarised logic.}
\label{PolarisationParagraph}

A logical system (or type theory) with explicit polarisation classifies
logical connectives (or type constructors) as either being positive or
negative; with these two worlds being dual to each other (categorically, by
adjunction).  Examples are the Girard's logic~{\LC} %~\cite{GirardLC} 
and Levy's Call-By-Push-Value calculus. %~\cite{LevyCBPV}.

A distinguishing aspect of polarised calculi is that they make call-by-value
(or eager) and call-by-name (or lazy) modes of computation explicit, allowing
for their combination in a framework with eager and lazy data structures.

Our investigations in this context will \hl{aim} at a model-theoretic study of
a rich variety of calculi encompassing aspects of resource management
(categorically, comonadic structure) %~\cite{Seely}) 
and computational effects (categorically, monadic 
structure) %~\cite{MoggiLambdaC})
as informed by polarisation.

Details of the models to be considered are in the full scientific proposal.
Here let us just mention that they subsume 
%
Girard's Intuitionistic Linear Logic; %~\cite{GirardLinearLogic};
%
Levy's Call-By-Push-Value; %~\cite{LevyCBPV};
%
Egger, M{\o}gelberg, and Simpson's Enriched Effect Calculus; %~\cite{EEC}
and
%
Melli\`es and Tabareau's Tensor Logic. %~\cite{TensorLogic}
%
In fact, we will go far beyond tackling the open \hl{problem} of developing a
model theory for Girard's~\LC.

Importantly, our investigations will target so-called direct-style languages
and models; so that they can directly feed into the proposed research of
Section~\ref{ProgrammingEffectsParagraph}.

\paragraph{Modal logics.}
\label{ModalLogicsParagraph}

Resource comonads and effect monads are logical modalities.  Polarisation in
the general context of modal logic has not been considered yet, and we will
complement the previous studies with the \hl{investigation} of Polarised
Modal Logics.  Our \hl{goal} is to develop the Propositions-as-Types
correspondence for modal logics.  We are specially interested in grounding it
through categorical models, that have been scarcely investigated.

In connection to Section~\ref{MetaprogrammingParagraph}, we will 
consider: 
%
Borghuis' Modal Pure Type Systems; %~\cite{ModalPTS};
%
Artemov's Logic of Proofs %~\cite{ArtemovLP} 
and related systems;
%(\eg~\cite{AltArtemov,ArtemovIemhoff}); 
%Daniyar S Shamkanov: Strong Normalization and Confluence for Reflexive
%  Combinatory Logic 
%
Mendler's Multimodal $CK$; %~\cite{MendlerMMCK};
and
%
Park and Im's Calculus $S_\Delta$. %~\cite{ParkIm}.

\subsubsection{Calculi}
\label{CalculiSubsection}

\paragraph*{Categorical Type Theory.}

We propose two lines of \hl{research} aimed at developing type-theoretic
formalisms as languages for categorical structures.

\paragraph{Generalised Type Theory.}
\label{GeneralisedTypeTheoryParagraph}

We will consider Lawvere's Generalised Logical Calculus, %~\cite{LawvereMetric},
synthesised by him from the model of presheaves in enriched category theory.
Our \hl{goal} will be to develop a type theory providing a
Propositions-as-Types interpretation of the Generalised Logical Calculus.
Success will yield a system where mathematical calculations in enriched
category theory can be established formally.  The problem of automating it
will be investigated.

The central novel constructions of the Generalised Logical Calculus are
so-called coends and ends.  Intuitively, these are quotients of dependent sums
and restrictions of dependent products.  Taking this view seriously, we will
\hl{aim} to extend the above to a fully-fledged, possibly
higher-dimensional, dependent type theory.  

\paragraph{Directed Type Theory.}
\label{DirectedTypeTheoryParagraph}

Homotopy Type Theory (HoTT) refers to a body of work at the boundary between
Homotopy Theory and Dependent Type Theory through Martin-L\"of's Identity
Types. %~\cite{ProgMLTT}.  
In a related vein, our \hl{aim} is to discover a notion of Directed Type
connecting Higher-Dimensional Category Theory and Dependent Type Theory on
which to establish a body of work on Directed Type Theory (DiTT).  Our
\hl{motivation} for this comes from computation.  Indeed, while Identity
Types establish the intentional equality of elements in a type, Directed Types
would instead represent the ways in which an element of a type may evolve to
another one in a possibly irreversible manner.  The full scientific proposal
includes preliminary thoughts for development in this direction.

\subsubsection{Programming}
\label{Programming}

We will design and implement programming languages from first principles and
pragmatics, following this up by their subsequent test, use, and distribution.
The research directions below will be considered as units in their own right,
and in relation to each other.

\paragraph{Indexed programming.}
\label{IndexedProgrammingParagraph}

Our main \hl{goal} here is to design/implement a language for programming
indexed data structures that will inform the design/implementation of the next
generation of programming languages.  To this end, we will pursue
\hl{research} in a bottom-up fashion, an approach that in this context has not
been explored systematically, extending functional programming languages with
indexing structure as prescribed by foundational calculi.

Under the Propositions-as-Types paradigm, we \hl{aim} at a language
supporting the activity of proving properties as a by-product of programming,
rather than that of extracting programs from the activity of proving
properties.  A strong pragmatic reason for this is that our uppermost interest
is in a language for programmers. 

An important \hl{challenge} to be faced is to design language constructs
with minimal type annotation that support maximal type inference.  Indeed,
this is a main open problem in dependently-typed programming language theory.

\paragraph{Effects.}
\label{ProgrammingEffectsParagraph}

Our overall \hl{aim} is to consider the model-theoretic investigations of
Section~\ref{PolarisationParagraph} from a proof-theoretic viewpoint and
percolate them down to programming language theory.

A main \hl{novelty} of our approach is that the proof theory to be
considered is based on sequent calculus, rather than the traditional line
followed so far in programming-language foundations based on sequent-style
natural deduction.  This requires the use and further investigation of an
emerging formalism, currently referred to as the calculus or system~\SysL,
that is proving robust in applications.

The research will require theoretical and applied work.  Theoretically, our
\hl{goal} will be to establish a Propositions-as-Types trinity between
classical sequent calculi and computational languages with effects and rich
type structure through adjoint interpretations.  Practically, we will need to
address a crucial programming-language \hl{question}:  What is the
programming paradigm stemming from sequent calculi in general, and the
calculus~\SysL\ in particular? 

As for polarisation, we believe that programmers will be able to intuitively
assimilate the eager {\vs}~lazy modes of computation and data structures
underlying it.  But, pragmatically, we will need to \hl{investigate} the
kind of programming style and language that will easily allow a programmer to
code polarisation.

The full scientific proposal contains further considerations on \hl{research}
possibilities for system~\SysL\ in programming and for effects encompassing
control.

\paragraph{Metaprogramming.}
\label{MetaprogrammingParagraph}

\mbox{}\newline
{\color{red}MISSING}

\subsection{Conclusion}

{\color{red}MISSING}

\hidebib{\setstretch{0}\footnotesize
\begin{thebibliography}{00}
\bibitem{UnivalentProgramme}
S.\,Awodey, T.\,Coquand, and V.\,Voevodsky (2012)
\newblock Univalent Foundations of Mathematics.
%\newblock IAS School of Mathematics program, Princeton.
\newblock IAS Princeton.
%\url{http://www.math.ias.edu/sp/univalent}.  

\bibitem{Birkhoff}
G.\,Birkhoff (1935).
\newblock On the structure of abstract algebras.
\newblock {\em P.\ Camb.\ Philos.\ Soc.}, 31:433--454. 

\bibitem{Church1936}
A.\,Church (1936).
\newblock An unsolvable problem of elementary number theory.
%\newblock \emph{American Journal of Mathematics}, 58:345--363. 
\newblock \emph{American Journal of Mathematics} 58.

\bibitem{Church1940}
A.\,Church (1940).
\newblock A formulation of the simple theory of types.
\newblock {\em J.\,Symbolic Logic}, 5:56--68. 

\bibitem{FioreLICS08}
M.\,Fiore (2008).
\newblock Second-order and dependently-sorted abstract syntax. 
\newblock In \emph{LICS'08}, pp.\,57--68. 

\bibitem{FioreICALP2012}
M.\,Fiore (2012).   
\newblock Discrete Generalised Polynomial Functors.  
\newblock In \emph{ICALP'12}, LNCS 7392, pp\,214--226.

\bibitem{FioreSecOrdEqLog}
M.\,Fiore and C.-K.\,Hur (2010).   
\newblock Second-order equational logic.  
\newblock In \emph{CSL'10}, LNCS 6247, pp\,320--335. 

\bibitem{FioreHurLMCS}
M.\,Fiore and C.-K.\,Hur (2011).   
\newblock On the mathematical synthesis of equational logics.
\newblock \emph{LMCS}-7(3:12).

\bibitem{FioreMahmoud}
M.\,Fiore and O.\,Mahmoud (2010).   
\newblock Second-order algebraic theories.  
\newblock In \emph{MFCS'10}, LNCS 6281, pp\,368--380. 

\bibitem{FiorePlotkinTuri}
M.\,Fiore, G.\,Plotkin and D.\,Turi (1999). 
\newblock Abstract syntax and variable binding.
\newblock In \emph{LICS'99}.

\bibitem{GirardSystemF}
J.-Y.\,Girard (1972).
\newblock \emph{Interpr\'{e}tation Fonctionnelle et \'{E}li\-mi\-na\-tion des
  Coupures de l'Arithm\'{e}tique d'Ordre Sup\'{e}rieur}.
\newblock Th\`{e}se de doctorat d'\'{e}tat, Universit\'{e} Paris VII. 

\bibitem{GirardLinearLogic}
J.-Y.\,Girard (1987).
\newblock Linear logic.
%\newblock \emph{Theoretical Computer Science}, 50:1--101.
\newblock \emph{TCS}, 50:1--101.

\bibitem{Howard1969}
W.\,Howard (1969).
\newblock The formulae-as-types notion of construction.
\newblock In \emph{\cite{ToHBCurry}}, pp\,479--490. 

\bibitem{LambekI}
J.\,Lambek (1968).
\newblock Deductive systems and categories I.
\newblock \emph{J.\ Math.\ Systems Theory}, 2:278--318.

\bibitem{LawvereThesis}
F.\,W.\,Lawvere (1963). %\,\&\,1968).
\newblock Functorial Semantics of Algebraic Theories. %and Some Algebraic
  %Problems in the context of Functorial Semantics of Algebraic Theories.
\newblock \emph{Reprints in TAC} 5, 2004.

\bibitem{LawvereAinF}
F.\,W.\,Lawvere (1969).
\newblock Adjointness in foundations.
\newblock \emph{Reprints in TAC} 16, 2006.

\bibitem{LawvereMetric}
F.\,W.\,Lawvere (1973).
\newblock Metric spaces, generalized logic and closed categories.
\newblock \emph{Reprints in TAC} 1, 2002.

\bibitem{PlotkinCBVCBN}
G.\,Plotkin (1975).
\newblock Call-by-name, call-by-value and the \mbox{$\lambda$-calculus}.
\newblock \emph{Theoretical Computer Science}, 1:125--159.

\bibitem{PlotkinLCF}
G.\,Plotkin (1977).
\newblock LCF considered as a programming language.
\newblock \emph{Theoretical Computer Science}, 5:223--255.

\bibitem{Milner1978}
R.\,Milner (1978).
\newblock A Theory of Type Polymorphism in Programming.
%\newblock \emph{Journal of Computer and System Sciences}, 17:348--375.
\newblock \emph{J.\ of Computer and System Sciences}, 17:348--375.

\bibitem{MoggiLambdaC}
E.\,Moggi (1991).
\newblock Notions of computation and monads. 
\newblock \emph{Information And Computation}, 93(1).

\bibitem{ProgMLTT}
B.\,Nordstr\"om, K.\,Petersson, and J.\,Smith (1990).
\newblock Programming in Martin-L\"of Type Theory: An Introduction.
%\newblock Oxford University Press.
\newblock OUP.

\bibitem{Reynolds}
J.\,Reynolds (1983).
\newblock Types, abstraction and parametric polymorphism.
\newblock \emph{Information Processing} 83, pp\,513--523.

\bibitem{ScottTCS}
D.\,Scott (1969).
\newblock A type-theoretical alternative to ISWIM, CUCH, OWHY
%\newblock In \emph{Theoretical Computer Science}, 121(1--2):411--440,
%1993.
\newblock In \emph{TCS}, 121(1--2):411--440, 1993.

\bibitem{ToHBCurry}
J.\,Seldin and J.\,R.\,Hindley (1980).
\newblock \emph{To H.B.\,Curry: Essays on Combinatory Logic, Lambda
  Calculus and Formalism}.
%\newblock Academic Press. 

\bibitem{Strachey1967}
C.\,Strachey (1967).
\newblock Fundamental Concepts in Programming Languages.  
%\newblock In \emph{Higher-Order and Symbolic Computation}, 13:11--49,
%2000.
\newblock In \emph{HOSC}, 13:11--49, 2000.

\bibitem{Turing}
A.\,Turing (1937). 
\newblock On computable numbers, with an application to the
  Entscheidungsproblem. 
%\newblock \emph{Proc.\ of the London Mathematical Society}, 42:230--265.
\newblock \emph{Proc.\ LMS}, 42:230--265.

\end{thebibliography}}

\clearpage
\setcounter{page}{1}
\chead{Part\,B1(b)}
\twocolumn[\begin{@twocolumnfalse}
	\hfill{\bfseries\Large 
    Curriculum Vitae
  }\hfill\null\\
\end{@twocolumnfalse}]

\subsection*{\S\enspace\thinspace Personal data}

\begin{myitemize}
\item
\textbf{\em Name}: 
\begin{tabular}[t]{l}
  Marcelo Pablo Fiore.
\end{tabular}

\item
\textbf{\em Nationality}: 
\begin{tabular}[t]{l}
  Argentinian/Italian.
\end{tabular}

\item
\textbf{\em Date of birth}: 
\begin{tabular}[t]{l}
  June 29, 1966.
\end{tabular}

\item
\textbf{\em Marital status}: 
\begin{tabular}[t]{l}
  Married, with two children (8\,\&\,3).
\end{tabular}

\item
\textbf{\em Address}: 
\begin{tabular}[t]{l}
  University of Cambridge, 
  Computer\\ Laboratory, 15 JJ Thomson
  Avenue,\\ Cambridge CB3 0FD, UK.
\end{tabular}

\item
\textbf{\em Web}: 
\begin{tabular}[t]{l}\small
  \url{www.cl.cam.ac.uk/~mpf23}
\end{tabular}

\item
\textbf{\em Email}: 
\begin{tabular}[t]{l}\small
  \texttt{Marcelo.Fiore@cl.cam.ac.uk}
\end{tabular}

\item
\textbf{\em Phone}: 
\begin{tabular}[t]{l}
  +44 (0)1223 334622.
\end{tabular}
\end{myitemize}

\subsection*{\S\enspace\thinspace University appointments}

\begin{myitemize}
\item
\textbf{\em Professor} in \emph{Mathematical Foundations of Computer
Science},  Computer Laboratory, University of Cambridge.  Since October
2011.

\item
\textbf{\em Reader} in \emph{Mathematical Foundations of Computer
Science}, Computer Laboratory, University of Cambridge.  October 2005 to
September 2011.

\item
\textbf{\em University Lecturer}, Computer Laboratory, University of
Cambridge.  October 2000 to September 2005.

\item
\textbf{\em Lecturer} in Computer Science, School of Cognitive and
Computing Sciences, University of Sussex.  April 1998 to September 2000. 

\item
\textbf{\em Research Fellow}, Laboratory for Foundations of Computer
Science, Department of Computer Science, University of Edinburgh.  January
1994 to March 1998.
\end{myitemize}

\subsection*{\S\enspace\thinspace University education}

\begin{myitemize}
\item
{\em Postgraduate}: \textbf{\em PhD in Computer Science}, University of
Edinburgh, June 1994.  \emph{Thesis}: Axiomatic Domain Theory in
Categories of Partial Maps.  \emph{Supervisor}: Prof Gordon Plotkin
(University of Edinburgh).  \emph{Examiner}: Prof Dana Scott (Carnegie
Mellon University)

\item
{\em Graduate}: \textbf{\em Licentiate in Informatics}, Escuela Superior
Latino Americana de Inform\'atica, Universidad Nacional de Luj\'an, Buenos
Aires (Argentina), December 1989.  \emph{Thesis}: On-line Algorithms for
Traversal of Dynamic Directed Hypergraphs with Application to
Satisfiability of Horn Formulae.  \emph{Supervisor}: Prof Giorgio
Ausiello, Uniersit\`a di Roma ``La Sapienze'', Italy.
\end{myitemize}

\subsection*{\S\enspace\thinspace Awards}

\begin{myitemize}
\item
\textbf{\em 10-Year Most Influential PPDP Paper Award} for the article
\emph{Semantic Analysis of Normalisation by Evaluation for Typed Lambda
Calculus} in the 2002 International Symposium on Principles and Practice
of Declarative Programming.

\item
\textbf{\em Distinguished Dissertation in Computer Science} for 1995.
\emph{Axiomatic Domain Theory in Categories of Partial Maps}.  Doctoral
thesis selected by the Conference of Professors and Heads of Computing in
conjunction with the British Computer Society.
\end{myitemize}

\subsection*{\S\enspace\thinspace Postgraduate supervision}
\vspace*{-1.9mm}
\begin{myitemize}
\item[]\small\ at the Computer Laboratory, University of Cambridge
\end{myitemize}

\begin{myitemize}
\item
  Marco Ferreira Devesas Campos.  Second year of PhD in Computer Science.

\item
  Ola Mahmoud, 2011 PhD in Computer Science.  \emph{Thesis}: Second-Order
  Algebraic Theories.

\item
  Marco Ferreira Devesas Campos, 2011 MPhil in Advanced Computer Science.
  \emph{Research essay}: Generalizing Bigraphs to DAG-like Place Graphs.

\item
  Eirik Tsarpalis, 2010 Certificate of Postgraduate Studies.

\item
  Maciej Wos, 2010 MPhil in Advanced Computer Science. \emph{Research essay}:
  Generic Programming with Fixed-Points for Nested Datatypes.

\item
  Chung-Kil Hur, 2010 PhD in Computer Science.  \emph{Thesis}: Categorical
  Equational Systems: Algebraic Models and Equational Reasoning.

\item
  Samuel Staton, 2007 PhD in Computer Science.  \emph{Thesis}:
  Name-Passing Process Calculi: Operational Models and Structural
  Operational Semantics.
\end{myitemize}

\subsection*{\S\enspace\thinspace Mentoring}
\vspace*{-2mm}
\begin{myitemize}
\item[]\small\ at the Computer Laboratory, University of Cambridge
\end{myitemize}

\begin{myitemize}
\item
Bartosz Klin.  EPSRC Postdoctoral Research Fellow in Theoretical Computer
Science.  October 2008 to September 2011.  

\item
Johan Glimming.  Visiting Research Fellow funded by the Swedish Research
Council VR.  October 2008 to September 2010.  

\item
Samuel Staton.  EPSRC Postdoctoral Research Fellow in Theoretical Computer
Science.  June 2007 to May 2010.  
\end{myitemize}

\subsection*{\S\enspace\thinspace Examining}

\begin{myitemize}
\item
\textbf{\em External} committee member for the PhD theses of: 
%
Wei Chen (University of Nottingham, 2012);
%
Julianna Zsido (Universit\'e de Nice Sophia Antipolis, 2010); 
%
Stephane Gimenez (Universit\'e Paris Diderot - Paris~7, 2009); 
%
Vincenzo Ciancia (Dipartamento di Informatica, Universit\`a di Pisa,
2008); 
%
Joachim de Lataillade (Universit\'e Paris Diderot - Paris~7, 2007); 
%
Emmanuel Beffara (Universit\'e Paris Diderot - Paris~7, 2005); 
%
Pierre Hyvernat (Institut Math\'ematique de Luminy, 2005);
%
Bartosz Klin (Aarhus University, 2004);
%
Vincent Balat~(Universit\'e Paris Diderot - Paris~7, 2002);
%
Krzysztof Worytkiewicz~(\'Ecole Polytechnique F\'ed\'erale de Lausanne,
2000).

\item
\textbf{\em Internal} committee member for the PhD theses of: 
%
Bjarki Holm (2011); 
%
Ranald Clouston (2009); 
%
Paul Hunter (2007); 
%
Lucy Brace-Evans (2007).

\item
Committee member for the Professorship habilitation of
%
Davide Sangiorgi~(INRIA Sophia Antipolis, 2002).

\item
Committee member for the MPhil thesis of
%
Henrik Enstr{\o}m (Aarhus University, 1998).

\item
Computer Science Tripos Examiner for Part IB, Part II, Part II (General),
and Diploma in 2006/07, 2007/08, and 2008/09 (Chair).  
\pagebreak
Computer Laboratory, University of Cambridge.  
\end{myitemize}

\vspace*{-1.25mm}
\subsection*{\S\enspace\thinspace Funding}
\vspace*{-1mm}
\begin{myitemize}
\item
  EPSRC grant submission (under consideration).
  \emph{Efficient Extraction of Feasible Programs}.  \textbf{\em Principal
    Investigator}, with co-investigators Timothy Griffin and Glynn Winskel
  (University of Cambridge).  Joint submission with Dr Mart\'{\i}n
  Escard\'o (University of Birmingham), Paulo Oliva (Queen Mary University
  of London), Ulrich Berger and Monika Seisenberger (Swansea University).
  Travel funding for 24 months.

\item
  \textbf{\em Visiting grant} from the Gunma University Foundation for
  Science and Technology.  Department of Computer Science, Gunma
  University.  April 3--12, 2012.

\item
  \textbf{\em Research in Paris grant} from the \emph{programme d'accueil
    des chercheurs \'etrangers de la Ville de Paris}, Laboratoire PPS,
  Universit\'e Paris Diderot - Paris~7.  October--December 2011.

\item
  \emph{Domain theory for concurrency---New categorical foundations}.
  \textbf{\em Co-investigator} of Glynn Winskel.  EPSRC Grant
  GR/T22049/01.  July 2005 to June 2008.

\item
  \emph{Mathematical models for functional and concurrent computation}.
  \textbf{\em EPSRC Advanced Research Fellowship}. October 2000 to
  September 2005.  
\end{myitemize}

\vspace*{-1.25mm}
\subsection*{\S\enspace\thinspace Invited research visits}
\vspace*{-1mm}

\begin{myitemize}
\item
  Short-Term Scholar invitation to the Univalent Foundations of Mathematics
  Program at the School of Mathematics, Institute for Advanced Study,
  Princeton. 

\item
  Assistant Prof Makoto Hamana at the Department of Computer Science,
  Gunma University.  In April 2012.

\item
  Prof Peter Dybjer at the Department of Computer Science and Engineering,
  Chalmers University of Technology.  In September 2008. 

\item
  Prof Anders Kock at the Department of Mathematical Sciences, Aarhus
  University.  In August 2007. 

\item
  Prof Mart\'{\i}n Abadi at the Department of Computer Science, University
  of California at Santa Cruz.  November 2003. 

\item
  Dr Vincent Danos at Laboratoire PPS, Universit\'e Paris Diderot -
  Paris~7.  June 2002.

\item
  Dr Paul-Andr\'e Melli\`es at Laboratoire PPS, Universit\'e Paris Diderot
  - Paris~7.  June 2001.

\item
  Mart\'{i}n Abadi at Bell Labs Research, Lucent Technologies, Palo Alto.
  In October--November 1999. 

\item
  Prof Gordon Plotkin at LFCS, Department of Computer Science, University
  of Edinburgh.  In November--December 1998.

\item
  Prof Marta Bunge at Department of Mathematics and Statistics, McGill
  University.  September and October 1998.

\item
  Prof Glynn Winskel at Department of Computer Science, Aarhus University,
  Denmark.  In August--September 1997.
\end{myitemize}

\vspace*{-1mm}
\subsection*{\S\enspace\thinspace Hosted research visitors}
\vspace*{-2mm}
\begin{myitemize}
\item[]\small\ at the Computer Laboratory, University of Cambridge
\end{myitemize}
\pagebreak

\noindent
PhD Student Ki Yung Ahn 
(Portland State University) in September 2012; 
%
Dr Michael Warren (IAS School of Mathematics, Princeton) in May 2012; 
%
Associate Prof Iliano Cervesato (Carnegie Mellon University - Qatar
Campus) in July 2011; 
%
PhD Student Guillaume Munch-Maccagnoni, (Universit\'e Paris Diderot -
Paris~7) March–May 2011 and in October 2012;
%
Prof Pierre-Louis Curien (Universit\'e Paris Diderot - Paris~7)
April--June 2009, April--June 2010, and in October 2012; 
%
Assistant Prof Makoto Hamana (Gunma University) in September 2009, March
2010 and October 2012;
%
Ichiro Hasuo (RIMS, Kyoto University) in November 2009;
%
Mat\'{\i}as Menni (Lifia, Universidad Nacional de La Plata) in July 2007
and April 2009;
%
Prof Franck van Breugel (York University, Toronto) sabbatical leave in
2004--05.

\vspace*{-0mm}
\subsection*{\S\enspace\thinspace Peer-reviewing}

\begin{myitemize}
\item
  \textbf{\em Programme committees}: CT 2013, ICALP 2013, Wollic 2012,
  MFPS XXVIII, MFCS 2012, URC 2010, SOS 2009, FOSSACS 2009, CSL 08, ICALP
  2008, MFPS XXIII (Chair), LICS 2007, SOFSEM 07, MSFP’06, CT 2006, MFPS
  XXI, ICALP’05, TCS 2004, CTCS 2004, CMCIM’03, WAIT’2003 (Co-chair),
  LICS’2003, FOSSACS’2002, MFPS XVII, LICS’2000, WAIT’98.

\item
  Reviewer for national and international conferences and journals.

\item
  Reviewer for the Uruguayan National Agency for Research and Innovation
  (ANII).

\item
  EPSRC Computer Science Panel Member.  February~1, 2005.
\end{myitemize}

\vspace*{-0mm}
\subsection*{\S\enspace\thinspace Memberships}

\begin{myitemize}
\item
  Fellow of \textbf{\em Christ's College}, University of Cambridge.  Since
  October 2001.

\item
  \textbf{\em EPSRC Peer Review College} member. October 2000 to September
  2010.

\item
  Editorial Board member of \textbf{\em Applied Categorical Structures}.
  Since October 2004.

\item
  Editorial Board member of the \textbf{\em Bulletin of Symbolic Logic}
  reviews for the \emph{Association for Symbolic Logic}.  January 2009 to
  December 2011.
\end{myitemize}

\vspace*{-0mm}
\subsection*{\S\enspace\thinspace Teaching}
\begin{myitemize}
\item Sundry undergraduate and postgraduate lecturing at the Computer
  Laboratory, University of Cambridge.  
\item Director of Studies for Computer Science in Christ's College
  Cambridge.  October 2001 to September 2011.  
\item Sundry undergraduate lecturing and tutorials at the School of
  Cognitive and Computing Sciences, University of Sussex. 
\item Sundry postgraduate lecturing at the Laboratory for Foundations of
  Computer Science, University of Edinburgh.
\end{myitemize}

\subsection*{\S\enspace\thinspace Administration}

\begin{myitemize}
\item
Sundry committees at the University of Cambridge and Christ's College
Cambridge.
\end{myitemize}

\clearpage
\setcounter{page}{1}
\chead{Part\,B1(c)}
\twocolumn[\begin{@twocolumnfalse}
	\hfill{\bfseries\Large 
    10-Year Track Record
  }\hfill\null\\
\end{@twocolumnfalse}]

%\subsection*{\S\enspace\thinspace Personal statement}

My research is in theoretical computer science, and by its own nature develops
in the context of close specialised collaboration.  My work stems from the
interaction between two approaches: abstract and concrete. At the abstract
level, I build mathematical theories. At the concrete level, theories are
applied to specific problems. This duality has yielded deep results and
provided research breadth, giving fundamental contributions to a variety of
fields and finding connections among them; \eg~domain theory, category theory,
concurrency theory, sheaf theory, type theory, logic, algebra.  Indeed,
note that I publish in computer science and mathematics venues, and I have
been invited speaker at both computer science and mathematics meetings.
My \mbox{h-index} is 21, according to Google scholar.  

\subsection*{\S\enspace\thinspace Award}

\begin{mybigitemize}
\item%[$\star$]
\textbf{\em 10-Year Most Influential PPDP Paper Award} for the article
\emph{Semantic Analysis of Normalisation by Evaluation for Typed Lambda
Calculus} in the 2002 International Symposium on Principles and Practice
of Declarative Programming.
\end{mybigitemize}

\subsection*{\S\enspace\thinspace Selected %peer-reviewed 
  publications}
\vspace*{-1.125mm}
\begin{myitemize}
\item[]\small\ 
  Number of citations provided according to Google
  \\\mbox{} 
  scholar.  Conference publications further selected for
  \\\mbox{} 
  journal proceedings have been circled.  Representative 
  \\\mbox{} 
  publications have been starred.
\end{myitemize}

\paragraph*{Research monograph}

\begin{mybigitemize}
\item[$\star$] 
  M.\,Fiore (2004). \emph{Axiomatic Domain Theory in Categories of Partial
    Maps}.  Distinguished Dissertations in Computer Science, No.\,14.
  Paperback Edition, Cambridge University
  Press.\mbox{}\hfill{\small[111~citations]}
\end{mybigitemize}

\paragraph*{Journals}

\begin{mybigitemize}
\item[$\obullet$]
  M.\,Fiore and C.-K.\,Hur (2011).  On the mathematical synthesis of
  equational logics.   In \emph{Selected Papers of the Conference
    ``Typed Lambda Calculi and Applications 2009''}.  \emph{Logical
    Methods in Computer Science}, Volume 7, ISSUE 3, PAPER
  12.\\\mbox{}\hfill{\small[19 (conference version) citations]}

%\item
%  M.\,Fiore and T.\,Leinster (2010).  An abstract characterization of
%  Thompson’s group F.  \emph{Semigroup Forum}, Volume 80, Number 2,
%  325--340.\\\mbox{}\hfill{\small[1~citation]}

\item[$\star$]
  M.\,Fiore and S.\,Staton (2009).  A congruence rule format for
  name-passing process calculi.  In \emph{Special Issue on Structural
    Operational Semantics (SOS)}. \emph{Information and Computation},
  207(2):209-236.\\\mbox{}\hfill{\small[26 (conference version) + 5~citations]}

\item[$\obullet$]
  M.\,Fiore and C.-K.\,Hur (2008).  On the construction of free
  algebras for equational systems.  In \emph{Special Issue for the
    Thirty-fourth International Colloquium on Automata, Languages and
    Programming (ICALP'07).  Theoretical Computer Science},
  410:1704--1729.\\\mbox{}\hfill{\small[15 (conference version) + 11 citations]}

\item[$\star$]
  M.\,Fiore, N.\,Gambino, M.\,Hyland, and G.\,Winskel (2008).  The
  cartesian closed bicategory of generalised species of structures. 
  \emph{Journal of the London Mathematical Society},
  77:203-220.\mbox{}\hfill{\small[19~citations]}

%\item
%  G.L.\,Cattani and M.\,Fiore (2007).  The bicategory-theoretic
%  solution of recursive domain equations.  In \emph{Computation,
%    Meaning, and Logic: Articles dedicated to Gordon Plotkin, Electronic Notes
%    in Theoretical Computer Science}, Volume 172,
%  pp\,203--222.\\\mbox{}\hfill{\small[0~citations]}

\item[$\obullet$]
  M.\,Fiore and S.\,Staton (2006).  
  Comparing operational models of
  name-passing process calculi.  \emph{Information and Computation},
  Vol\,204, Issue 4, pp\,524--560.\\\mbox{}\hfill{\small[15 (conference version)
    + 29~citations]}

\item[$\obullet$]
  M.\,Fiore, R.\,Di Cosmo, and V.\,Balat (2006).  Remarks on
  isomorphisms in typed lambda calculi with empty and sum types. 
  \emph{Annals of Pure and Applied Logic}, Vol\,141, Issues 1--2,
  pp\,35--50.\\\mbox{}\hfill{\small[19 (conference version) + 11 citations]}

%\item
%  M.\,Fiore and M.\,Menni (2005).  Reflective Kleisli subcategories
%  of the category of Eilenberg-Moore algebras for factorization monads.
%   In \emph{Proceedings of the International Category Theory
%    Conference (CT2004), Theory and Applications of Categories}, Vol\,15,
%  No\,2, pp\,40--65.\\\mbox{}\hfill{\small[3~citations]}

\item
  M.\,Fiore and T.\,Leinster (2005).  Objects of categories as
  complex numbers.  \emph{Advances in Mathematics},
  190(2):264--277.\mbox{}\hfill{\small[11~citations]} 

\item
  M.\,Fiore and T.\,Leinster (2004).  An Objective Representation of
  the Gaussian Integers.  \emph{Journal of Symbolic Computation},
  37(6): 707--716.\mbox{}\hfill{\small[5~citations]} 

\item[$\ostar$]
  M.\,Fiore, E.\,Moggi, and D.\,Sangiorgi (2002).  A fully-abstract
  model for the pi-calculus.  Information and Computation,
  179:76--117.\\\mbox{}\hfill{\small[95 (conference version) + 30 citations]}
\end{mybigitemize}

\paragraph*{Conferences}

\begin{mybigitemize}
\item[$\star$]
  M.\,Fiore (2012).  Discrete Generalised Polynomial Functors.  In
  \emph{Automata, Languages, and Programming Conference (ICALP'12)}.
  Volume 7392 of Lecture Notes in Computer Science,
  pp.\,214--226.\\\mbox{}\hfill{\small[0~citations]}

%\item
%  M.\,Hamana and M.\,Fiore (2011).  A foundation for GADTs and
%  Inductive Families: Dependent polynomial functor approach.  In
%  \emph{ACM SIGPLAN Seventh Workshop on Generic Programming (WGP'11)},
%  pp.\,59--70.  ACM Press.\\\mbox{}\hfill{\small[0~citations]}

\item[$\obullet$]
  M.\,Fiore and O.\,Mahmoud (2010).  Second-order algebraic theories.
   In \emph{Proceedings of the Thirty-fifth International Symposium
    on Mathematical Foundations of Computer Science (MFCS'10)}.  
  Volume 6281 of Lecture Notes in Computer Science,
  pp\,368--380.\mbox{}\hfill{\small[7~citations]}

\item[$\ostar$]
  M.\,Fiore and C.-K.\,Hur (2010).  Second-order equational logic. In
  \emph{Proceedings of the Nineteenth EACSL Annual Conference on Computer
    Science Logic (CSL'10)}.  Volume 6247 of Lecture Notes in
  Computer Science, pp\,320--335.\mbox{}\hfill{\small[11~citations]}

%\item
%  M.\,Fiore and S.\,Staton (2010).  Positive structural operational
%  semantics and monotone distributive laws.  \emph{In Short
%    Contributions for the Tenth International Workshop on Coalgebraic Methods
%    in Computer Science (CMCS 2010)}.  CWI Technical report SEN-1004,
%  pp\,8--9.\\\mbox{}\hfill{\small[0~citations]}

%%\item
%%  M.\,Fiore and C.-K.\,Hur (2008).   Term equational systems and
%%  logics.  In \emph{Proceedings of the Twenty-fourth Conference on
%%    the Mathematical Foundations of Programming Semantics (MFPS XXIV)}.
%%   Electronic Notes in Theoretical Computer Science, volume 218,
%%  pp\,171--192.\\\mbox{}\hfill{\small[19~citations]}

\item[$\star$]
  M.\,Fiore (2008).  Second-order and dependently-sorted abstract
  syntax.  In \emph{Logic in Computer Science Conference (LICS'08)},
  pp.\,57--68.  IEEE, Computer Society
  Press.\mbox{}\hfill{\small[22~citations]}

%%\item
%%  M.\,Fiore and C.-K.\,Hur (2007).  Equational systems and free
%%  constructions.  In \emph{International Colloquium on Automata,
%%    Language and Programming (ICALP 2007)}. Volume 4596 of Lecture Notes in
%%  Computer Science, pp\,607--619.\\\mbox{}\hfill{\small[15~citations]}

\item[$\obullet$]
  M.\,Fiore (2007).  Differential structure in models of
  multiplicative biadditive intuitionistic linear logic.  In
  \emph{Typed Lambda Calculi and Applications (TLCA'07)}.  Volume
  4583 of Lecture Notes in Computer Science,
  pp\,163--177.\mbox{}\hfill{\small[14~citations]}

%%\item
%%  M.\,Fiore and S.\,Staton (2006).  A congruence rule format for
%%  name-passing process calculi from mathematical structural operational
%%  semantics.  In \emph{Twenty-first Logic in Computer Science
%%    Conference (LICS'06)}, pp\,49--58.  IEEE, Computer Society
%%  Press.\\\mbox{}\hfill{\small[26~citations]}

%%\item
%%  M.\,Fiore and S.\,Staton (2004).  Comparing operational models of
%%  name-passing process calculi.  In \emph{Proceedings of the 7th
%%  Coalgebraic Methods in Computer Science Workshop (CMCS'04)}. 
%%  Volume 106 of Electronic Notes in Theoretical Computer Science, pages
%%  91--104.  Elsevier.\mbox{}\hfill{\small[29 citations]}

\item
  M.\,Fiore (2004).  Isomorphisms of generic recursive polynomial
  types.   In \emph{Thirty-first Symposium on Principles of
    Programming Languages (POPL'04)}, pp\,64--76.  ACM
  Press.\mbox{}\hfill{\small[22~citations]}

\item[$\star$]
  V.\,Balat, R.\,Di Cosmo, and M.\,Fiore (2004).  Extensional
  normalisation and type-directed partial evaluation for typed lambda
  calculus with sums.   
  In \emph{Thirty-first Symposium on Principles
    of Programming Languages (POPL'04)}, pp\,64--76.  ACM
  Press.\mbox{}\hfill{\small[47~citations]}

\item[$\star$]
  M.\,Fiore (2002).  Semantic analysis of normalisation by evaluation
  for typed lambda calculus.  In \emph{Fourth Principles and Practice
    of Declarative Programming Conference (PPDP'02)}, pages 26--37. 
  ACM Press.\mbox{}\hfill{\small[45~citations]}

%%\item
%%  M.\,Fiore, R.\,Di Cosmo, and V.\,Balat (2002).  Remarks on
%%  isomorphisms in typed lambda calculi with empty and sum types.  In
%%  \emph{Seventeenth Logic in Computer Science Conference (LICS'02)},
%%  pp\,147--156.   IEEE, Computer Society
%%  Press.\\\mbox{}\hfill{\small[19~citations]}
\end{mybigitemize}

\paragraph*{Invited}

\begin{mybigitemize}
\item[$\star$]
  M.\,Fiore (2005).  Mathematical models of computational and combinatorial
  structures.  {\em Foundations of Software Science and Computation Structures
    (FOSSACS'05)}.  Volume 3441 of Lecture Notes in Computer Science,
  pp\,25--46.\mbox{}\hfill{\small[18~citations]}
\end{mybigitemize}

\subsection*{\S\enspace\thinspace Invited addresses}
\vspace*{-1.125mm}
\begin{myitemize}
\item[]\small\ 
  Salient invitations have been starred.
\end{myitemize}


\begin{mybigitemize}
\item[$\star$]
  Indexing Structures in Programming Language Semantics and Design.
  \emph{Fourteenth International Symposium on Principles and Practice of
    Declarative Programming~(PPDP'12)}, 
  %10-Year Most Influential PPDP Paper Award talk, 
  Leuven (Belgium), September~2012.

\item
  Lie in Logic.  \emph{S\'{e}minaire CHoCoLa}, \'Ecole Normale Sup\'erieure de
  Lyon (France), May~2012.

\item[$\star$]
  Second-order algebra and generalised polynomial functors.  \emph{Logic and
    Interactions 2012}, Centre International de Rencontres Math\'{e}matiques,
  Luminy (France), February~2012.

\item
  Estructuras matem\'{a}ticas en lenguajes de programaci\'{o}n.
  \emph{French-Argentinean Laboratoire Internationale Associ\'e LIA INFINIS},
  %inaugural meeting, 
  Departamento de Computaci\'{o}n, Facultad de Ciencias Exactas y Naturales -
  Universidad de Buenos Aires (Argentina), December~2011.

\item
  On Higher-Order Algebra.   \emph{Third Scottish Category Theory Seminar},
  University of Strathclyde, Glasgow (Scotland), December~2010.

\item
  Algebraic simple type theory.  \emph{Curry-Howard and Concurrency Theory
    (CHoCo) S\'eminaire}, l'Ecole Normale Sup\'erieure de Lyon (France),
  December 2009.

\item[$\star$]
  Mathematical synthesis of equational deduction systems.  \emph{Ninth
    International Conference on Typed Lambda Calculi and Applications
    (TLCA'09)}, Brasilia (Brazil), June 2009.  

\item
  Algebraic Type Theory.  \emph{Eighty-eighth Peripatetic Seminar on Sheaves
    and Logic (88th PSSL), Celebrating the 60th birthdays of Martin Hyland and
    Peter Johnstone}, University of Cambridge (UK), April~2009.

\item[$\star$]
  Algebraic theories and equational logics.  \emph{Twenty-fourth Conference on
    the Mathematical Foundations of Programming Semantics (MFPS~XXIV)},
  University of Pennsylvania~(USA), May~2008.  

\item
  Second-order and dependently-sorted algebraic theories.  \emph{Workshop on
    Categorical and Homotopical Structures in Proof Theory}, Centre de Recerca
  Matem\`atica, %~(CRM), 
  Barcelona~(Spain), February 2008.

\item
  An axiomatics and a combinatorial model of creation/annihilation operators
  and differential structure.  \emph{Categorical Quantum Logic
    Workshop (CQL)}, %Computing Laboratory, 
  University of Oxford (UK), August 2007.

\item
  A mathematical theory of substitution and its applications to syntax and
  semantics.  \emph{Coalgebraic Logic Workshop (CoL)}, %Computing Laboratory,
  University of Oxford (UK), August 2007.

\item[$\star$]
  Towards a mathematical theory of substitution.  \emph{Annual International
    Conference on Category Theory (CT'07)}, Carvoeiro, Algarve (Portugal),
  June 2007.
 
\item[$\star$]
  A mathematical theory of substitution and its applications to syntax and
  semantics.  \emph{Workshop on Mathematical Theories of Abstraction,
    Substitution and Naming in Computer Science}, International Centre for
  Mathematical Sciences, %(ICMS), 
  Edinburgh (Scotland), May~2007.
 
\item[$\star$]
  Analytic functors and domain theory.  \emph{Symposium for Gordon Plotkin},
  Laboratory for Foundations of Computer Science, %~(LFCS), 
  University of Edinburgh (Scotland), September~2006.
 
\item[$\star$]
  On the structure of substitution. \emph{Twenty-second Mathematical
    Foundations of Programming Semantics Conference (MFPS XXII)}, %DISI,
  University of Genova (Italy), May~2006.
 
\item
  Isomorphisms on generic mutually recursive polynomial types.  \emph{Second
    International Workshop on Isomorphisms of Types}, Universit\'e de Toulouse
  (France), October 2005.

\item[$\star$]
  Mathematical models of computational and combinatorial structures.  
  \emph{Foundations of Software Science and Computation Structures
  (FOSSACS'05)} for European Joint Conferences on Theory and Practice of
  Software (ETAPS'05), Edinburgh~(UK), April 2005.  

\item[$\star$]
  Computational semantics.  \emph{Winter School on Semantics and
    Applications}, Instituto de Computaci\'on, Facultad de Ingenier\'{\i}a,
  Universidad de la Rep\'ublica, Montevideo~(Uruguay), July 2003.

\item
  A semantic framework for name and value passing process calculi.
  \emph{Workshop on Concurrency and Mobility, Logic and Foundations of
    Computation}, Fields Institute Summer School, University of Ottawa
  (Canada), June 2003.

\item[$\star$]
  Imaginary types.  \emph{Nineteenth Conference on the Mathematical
    Foundations of Programming Semantics (MFPS~XIX)}, Montr\'eal~(Canada),
  March~2003.

\item
  Foundational theories of combinatorial type structures.  \emph{Birmingham,
    Leicester, and Nottingham Christmas Theory Day}, Department of Computer
  Science, University of Birmingham~(UK), December~2002.
\end{mybigitemize}
\vspace*{-6mm}
\end{document}
