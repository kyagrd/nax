\documentclass[11pt,twocolumn]{article}

\usepackage{amsfonts,euler}
\usepackage{MnSymbol}

\usepackage[all]{xy}
\UseComputerModernTips
\xyoption{knot}

\usepackage[a4paper,includehead,includefoot,headheight=13.6pt,
  headsep=4mm,top=1cm,bottom=1cm,left=2cm,right=2cm]{geometry}

\usepackage{float}
\floatstyle{ruled}
\restylefloat{figure}

\usepackage{dashrule}

\usepackage{multirow}

\usepackage{color}
\definecolor{grey}{gray}{.85}

\setcounter{secnumdepth}{3}
\newcommand{\myparagraph}[1]{\paragraph*{\em #1}}
\newcommand{\mynumparagraph}[1]{\paragraph{\em #1}}

%%% BEGIN macros
\newenvironment{myitemize}
  {\begin{list}{$\bullet$}
  {\setlength{\topsep}{1pt}
   \setlength{\partopsep}{1pt}
   \setlength{\itemsep}{0pt}
   \setlength{\parsep}{0pt}
   \setlength{\leftmargin}{1em}
   \setlength{\labelwidth}{.5em}}}
  {\end{list}}
\newenvironment{myindentitemize}
  {\begin{list}{$-$}
  {\setlength{\topsep}{2pt}
   \setlength{\partopsep}{2pt}
   \setlength{\itemsep}{2.5pt}
   \setlength{\parsep}{2.5pt}
   \setlength{\leftmargin}{2.125em}
   \setlength{\labelwidth}{1.625em}}}
  {\end{list}}
\newenvironment{myquote}
  {\begin{list}{}
  {\setlength{\topsep}{2pt}
   \setlength{\partopsep}{2pt}
   \setlength{\itemsep}{2.5pt}
   \setlength{\parsep}{2.5pt}
   \setlength{\rightmargin}{1em}
   \setlength{\leftmargin}{1em}
   \setlength{\labelwidth}{.5em}}}
  {\end{list}}
\newenvironment{mybigitemize}
  {\begin{list}{$\bullet$}
  {\setlength{\topsep}{2pt}
   \setlength{\partopsep}{2pt}
   \setlength{\itemsep}{2.5pt}
   \setlength{\parsep}{2.5pt}
   \setlength{\leftmargin}{1em}
   \setlength{\labelwidth}{.5em}}}
  {\end{list}}
\newenvironment{btritemize}
  {\begin{list}{\btr}
  {\setlength{\topsep}{2pt}
   \setlength{\partopsep}{2pt}
   \setlength{\itemsep}{2.5pt}
   \setlength{\parsep}{2.5pt}
   \setlength{\leftmargin}{1em}
   \setlength{\labelwidth}{.5em}}}
  {\end{list}}
\newcounter{CC}
\newenvironment{resenumerate}
  {\begin{list}{[\textbf{\arabic{CC}]}}
  {\usecounter{CC}
   \setlength{\topsep}{2pt}
   \setlength{\partopsep}{2pt}
   \setlength{\itemsep}{2.5pt}
   \setlength{\parsep}{2.5pt}
   \setlength{\leftmargin}{1.65em}
   \setlength{\labelwidth}{1.15em}
 }}
  {\end{list}}

\newcommand{\mysf}{\small\sf}
\newcommand{\mytextsf}[1]{\textsf{\small #1}}
\newcommand{\erc}{{\small\sf MaStrPLan}}
\newcommand{\ERC}{Mathematical Structures for Type Theories,\\[-.5mm] Logical
  Systems, and Programming Languages}

\newcommand{\hide}[1]{}
\newcommand{\note}[1]{\begin{quote}{\color{blue}$\leadsto$ \bf\em
      #1}\end{quote}}
\newcommand{\hidenote}{\hide}

\newcommand{\pref}[1]{\,(\ref{#1})}
\newcommand{\itemref}[1]{\textbf{[\ref{#1}]}}

\newcommand{\eg}{\emph{eg.}}
\newcommand{\Eg}{\emph{Eg.}}
\newcommand{\vs}{\emph{vs.}}
\newcommand{\ie}{\emph{ie.}}
\newcommand{\viz}{\emph{viz.}}
\newcommand{\etc}{\emph{etc.}}
\newcommand{\etal}{\emph{et al.}}
\newcommand{\cf}{\emph{cf.}}

\newcommand{\lcalculus}{\mbox{$\lambda$-calculus}}
\newcommand{\SystemL}{\mbox{System~$L$}}
\newcommand{\SystemF}{\mbox{System~$F$}}
\newcommand{\SystemFi}{\mbox{System~$F_i$}}
\newcommand{\SystemFomega}{\mbox{System~$F_\omega$}}
\newcommand{\LC}{\mbox{$LC$}}

\newcommand{\btr}{$\blacktriangleright$}

\newcommand{\reqpsize}{8.113395cm}%{\columnwidth}

\newcommand{\req}[2]{\begin{center}\colorbox{grey}{\begin{minipage}{\reqpsize} 
  \mytextsf{Research question}\hfill$^{\mbox{\scriptsize #1 }}$\\[-5.5mm]
  \begin{btritemize}
  \item #2
  \end{btritemize}
\end{minipage}}\end{center}}

\newcommand{\reqs}[2]{\begin{center}\colorbox{grey}{\begin{minipage}{\reqpsize}
  \mytextsf{Research questions}\hfill$^{\mbox{\scriptsize #1 }}$\\[-5.5mm]
  \begin{btritemize}
  \item #2
  \end{btritemize}
\end{minipage}}\end{center}}
\newcommand{\rep}[2]{\begin{center}\colorbox{grey}{\begin{minipage}{\reqpsize}
  \mytextsf{Research pathway}\hfill$^{\mbox{\scriptsize #1 }}$\\[-5.5mm]
  \begin{btritemize}
  \item #2
  \end{btritemize}
\end{minipage}}\end{center}}

\newcommand{\Set}{{\boldsymbol{\mathscr S}}}
\newcommand{\scat}[1]{\mathbb{#1}}
\newcommand{\cat}[1]{\mathscr{#1}}
\newcommand{\op}{\circ}
\newcommand{\Id}{\mathrm{Id}}
\newcommand{\Di}{\mathrm{Di}}

\newcommand{\qed}{\Box}
\newcommand{\logo}
{\mbox{\tiny$\xymatrix@C=3pt@R=4.5pt{
    & \qed \ar@{-}[d] 
    \ar@/_.5em/@{-}[ddl]<-.25em>
    \ar@/^.5em/@{-}[ddr]<.25em>
    & 
    \\
    & \ar@{-}[dl] \qed \ar@{-}[dr] & 
    \\
    \ar@/_.5em/@{-}[rr]<.125em>
    \qed & & \qed
  }$}}
%%% END macros

\usepackage{fancyhdr}
\pagestyle{fancy}
\lhead{Marcelo Fiore}
\chead{Part~B1}
\rhead{\erc}
\lfoot{}
\cfoot{\thepage}
\rfoot{}

\usepackage[compact]{titlesec}

\usepackage{compactbib}

\usepackage{setspace}
\setstretch{.9}

\usepackage{times}

\makeatletter
\renewcommand\@biblabel[1]{#1}
\makeatother

\renewcommand{\thesection}{\arabic{section}}
\renewcommand{\thesubsection}{\arabic{subsection}}

\def\contentsname{\large Contents\\[-7.5mm]}
\usepackage{minitoc}
\tightmtctrue

\usepackage[draft]{hyperref}
\hypersetup{pdftitle={\ERC},pdfauthor={Marcelo Fiore}}

\newcommand{\obullet}{\mbox{$\raisebox{-.85mm}{\huge$\circ$}\hspace*{-2.6mm}\bullet$}}

\renewcommand{\ostar}{\mbox{$\raisebox{-.85mm}{\huge$\circ$}\hspace*{-2.75mm}\star$}}

\setlength{\intextsep}{5pt}

\begin{document}
\setlength{\abovedisplayskip}{5pt}
\setlength{\belowdisplayskip}{5pt}

\setcounter{page}{0}
\thispagestyle{plain}
\twocolumn[\begin{@twocolumnfalse}
{\large European Research Council}

\vspace*{7.5mm}
\begin{center}\Large
ERC Advanced Grant\\
Research Proposal\\
(Part B1)
\end{center}

\vspace*{2.5mm}

\begin{center}\bfseries\LARGE
Mathematical Structures for Type Theories,\\
Logical Systems, and Programming Languages\\
\vspace*{10mm}
\Large\bfseries\sf 
MaStrPLan\\
  \logo
\end{center}

\vspace*{2.5mm}

\begin{center}\large
Principal Investigator: Marcelo Fiore\\
\vspace*{1mm}
Host Institution: University of Cambridge\\
\vspace*{1mm}
Proposal duration: 60 months
\end{center}

\vspace*{5mm}

\renewcommand{\abstractname}{\bfseries\large Summary}
\begin{abstract}\normalsize
Headed by Principal Investigator Marcelo Fiore, {\erc} assembles an
international team of world-leading experts in theoretical and applied
computer science.  The general aim of the project is to build a group
dedicated to combined research in Category Theory, Mathematical Logic,
Programming Theory, and Type Theory.  Specifically, to understand and develop
formal languages and mathematical models for computer programming and
mathematical proof. 

{\color{red}FINISH}

We particularly aim to contribute with: a framework to analyse and synthesise
type theories; mathematical theories and models for computational phenomena;
formal calculi for proof and/or computation; and experimental high-level
programming languages.
\end{abstract}

\end{@twocolumnfalse}]

\clearpage
\chead{Part\,B1(a)}
\twocolumn[\begin{@twocolumnfalse}
	\hfill{\bfseries\Large 
      Extended Synopsis of the scientific proposal
  }\hfill\null\\
\end{@twocolumnfalse}]

\subsection{Proposal}

\paragraph*{General aim.}

Headed by Principal Investigator Marcelo Fiore, {\erc} assembles an
international team of world-leading experts in theoretical and applied
computer science.  The general aim of the project is to build a group
dedicated to combined research in Category Theory, Mathematical Logic,
Programming Theory, and Type Theory.  Specifically, to understand and
develop formal languages and mathematical models for computer programming
and mathematical proof. 

\paragraph*{Working thesis.}

The research is driven by the thesis that
\begin{myquote}
\item
Languages 
for computer programming
and languages 
for mathematical proof 
are of the same character.
\end{myquote}
This view %conception 
%%is of course not original to us.  It 
started developing in the 1960s %, with proponents from computer science
%(\eg~McCarthy, Dijkstra, Hoare, Strachey) and from mathematics (\eg~de Bruijn,
%Scott, Martin-L\"of); 
and is now central to research in programming languages and constructive
mathematics.

An aspect of the thesis that is fundamental to us here is usually referred
to as the Proofs-as-Programs or Propositions-as-Types correspondence.
%There is an aspect of the thesis that is rooted in the constructive
%approach to mathematical practice and is fundamental to us here.  This 
It can be impressionistically %roughly 
presented %as a correspondence 
as follows
\[%\begin{equation}\label{PATproportion}%\xymatrix@C5pt{
%  \frac{\txt{program}}{\txt{Type}} 
%  & \mbox{\Large$\approx$} & 
%  \frac{\txt{proof}}{\txt{Proposition}}
%}
%
  \mbox{proof : Proposition} 
  \enspace \approx \enspace 
  \mbox{program : Type} 
%\end{equation}
\]
and one intuitively argues for it along these lines: %the lines below:
%\begin{myitemize} 
%\item 
  to give a constructive proof of a proposition 
  %(\eg~that every number is smaller than a prime number) 
  is to construct a program that certifies the statement, 
  %(\eg~one that on inputing a number, outputs some prime number bigger
  %than it);
  while
%\item 
  to type a program is to establish a property of it.
  %(\eg~that certain input/output behaviour or invariant is maintained). 
%\end{myitemize} 

Research stemming form the Propositions-as-Types correspondence has been
very fruitful.  Various %Many 
of its mathematical theories have been implemented as computer systems, and
used academically and industrially.  Cases in point are: programming languages
for high-assurance code and proof assistants for computer-aided verification;
examples of which are
Haskell, %~\cite{Haskell} 
an industrial-strength functional programming language, 
and Coq, %~\cite{Coq} 
a proof assistant in which bodies of mathematics are being verified.

\paragraph*{The future.}

The developments of transfer from mathematical theories to computer
systems have required technical/theoretical and technological/applied
progress.  This has necessarily demanded increasingly specialised, and to
some extent fragmentary, research.  
Importantly, %though, 
the field has reached a first plateau; encompassing a substantial body of
work and expertise.  
How %But, how
will it continue to advance?  By considering new directions and
challenges, of course.  But, as in the very beginning of the subject, it is
our firm view that it will be essential to do so generating questions and
tackling problems from the perspectives of a variety of research areas.  It is
both of these aspects that we put forward here.

\paragraph*{Our philosophy.}

Two disciplines are clearly involved in investigating the
Proofs-as-Programs 
correspondence: %~(\ref{PATproportion}) above:
Mathematical Logic and Programming Theory.  
%Much research, as the one proposed here, is geared towards finding a middle
%ground that can accommodate the concerns of both, and expand their 
%realms. %domains.
%
This is however only half of the picture.  The full strength of the
correspondence involves also the disciplines of Type Theory and Category
Theory.  
%Their role is to enrich~(\ref{PATproportion}) as follows 
%\[\xymatrix@C5pt@R2.5pt{
%  & \frac{\txt{\small construction}}{\txt{\small Type}} 
%    \ar@{}[rd]|-{\rotatebox[origin=c]{-45}{\Large$\approx$}}
%  & 
%    \\
%    \frac{\txt{\small program}}{\txt{\small Type}} 
%    \ar@{}[ru]|-{\rotatebox[origin=c]{-135}{\Large$\approx$}}
%    & & 
%    \frac{\txt{\small proof}}{\txt{\small Proposition}}
%  \\
%  & \frac{\txt{\small morphism}}{\txt{\small Object}} 
%  \ar@{}[ru]|-{\rotatebox[origin=c]{-135}{\Large$\approx$}}
%  & 
%}\]
Figure\,\ref{ResearchAreas} 
\begin{figure}[h]
\caption{Research areas and interactions.}
\vspace*{2mm}
\begin{center}
\hspace*{.5mm}
\xymatrix@R=25pt@C=15pt{
& 
\raisebox{7mm}{\fbox{\txt{\small Mathematical\\\small Logic}}}
\ar@/_1em/@{<->}[ddl]<-1em>|-
  {\txt{\scriptsize Propositions\\ \raisebox{1mm}{\scriptsize as Types}}}
\ar@/^1em/@{<->}[ddr]<1em>|-
  {\txt{\scriptsize Proofs as\\\raisebox{1mm}{\scriptsize Programs}}} 
& 
\\
& 
\ar@{<->}[dl]|-
  {\txt{\scriptsize Internal\\\raisebox{1mm}{\scriptsize Languages}}}
\ar@{<->}[dr]|-
  {\txt{\scriptsize Denotational\\\raisebox{1mm}{\scriptsize Semantics}}} 
\ar@{<->}[u]|-
{\txt{\scriptsize Mathematical\\\raisebox{1mm}{\scriptsize Models}}}
\fbox{\txt{\small Category\\\small Theory}}
& 
\\
  \fbox{\txt{\small Type\\\quad\small Theory\quad\null}}
\ar@/_1em/@{<->}[rr]|-
  {\txt{\scriptsize Formal\\\raisebox{1mm}{\scriptsize Languages}}}
& & 
\fbox{\txt{\small Programming\\\small Theory}}
}
\end{center}
\vspace*{-2mm}
\label{ResearchAreas}
\end{figure}
gives a schematic view of the interactions between these four research areas
in this context.  Each of them is complementary to the others.  Together, as a
unified whole, they have shaped fields of computer science and mathematics.  
\hide{
Indeed, consider for instance that: 
\begin{myitemize}
\item
  the categorical interpretation of quantifiers as
  adjoints~\cite{LawvereAinF} informed the development of the
  type-theoretic dependent sums and dependent products~\cite{ScottCV};
\item
  type theories are the foundation of programming-language typing
  systems~\cite{Pierce};
\item
  the control operators of programming languages are key to the
  constructive interpretation of classical proofs~\cite{Griffin}; and 
\item
  model-theoretic studies of the polymorphic lambda
  calculus~\cite{GirardSystemF,Reynolds} led to remarkable small complete
  categories~\cite{Hyland}.
\end{myitemize}
}
%
It is within this framework that we propose cross-cutting research in Category
Theory, Mathematical Logic, Programming Theory, and Type Theory.  
We %Furthermore, we 
contend that an approach neglecting any one of them is to the detriment of
the others; missing the depth and richness of the subject and, crucially,
missing opportunities for research and development.  

\paragraph{Main goals.}

The motivations for our research proposal are specifically as follows. 
\begin{myitemize}
\item[\raisebox{.75mm}{\tiny$\bigstar$}]\hspace*{-2mm}
  To search for %elucidate 
  %what we hope to be 
  the next-generation framework of Type Theory
  relevant to computer science and mathematics.
\item[\raisebox{.75mm}{\tiny$\bigstar$}]\hspace*{-2mm}
  To exercise and test Category Theory both as a unifying and as a
  foundational mathematical language.
\item[\raisebox{.75mm}{\tiny$\bigstar$}]\hspace*{-2mm}
  To broaden the current role of Mathematical Logic in computation.
\item[\raisebox{.75mm}{\tiny$\bigstar$}]\hspace*{-2mm}
  %To develop Programming Theory, designing, implementing, and experimenting
  %with languages of the future.
  To design, implement, and experiment within Programming Theory, looking
  into the languages of the future.
\end{myitemize}

We particularly aim to contribute with: a framework to analyse and synthesise
type theories; mathematical theories and models for computational phenomena;
formal calculi for proof and/or computation; and experimental high-level
programming languages.

\paragraph{Research programme.}

The research programme has been precisely conceived to target the goals
above.  It is organised in four strands as outlined below.
\begin{myitemize}
\item[{\bfseries 1\enspace Foundations:}]\mbox{}\enspace\thinspace 
  \emph{A comprehensive research programme on the metamathematics of type
    theories.}
  %--- addressing the problem of what type theories are.

  \vspace*{1mm}
  We will address the fundamental question of what type theories are,
  which will also lead us to rethink them.  
  Our approach 
  %is novel and ambitious.  It 
  aims at an algebraic framework that will generalise to type theories all
  aspects of our current understanding of algebraic
  theories. %, from the perspectives of categorical algebra,
  %equational logic, and universal algebra. 
  %%(see the Algebraic Trinity of Figure~\ref{} on page~\pageref{}).  
  The scale at which we will be attempting this is unprecedented,
  systematically exploring a wide spectrum of key type-theoretic features.
  \vspace*{1mm}

\item[{\bfseries 2\enspace Models:}]\mbox{}\enspace\thinspace
  \emph{Study of mathematical models for type theories and logical systems.}
  %--- targeting semantic problems at the forefront of current understanding. 

  \vspace*{1mm}
  We will tackle semantic problems at the forefront of current understanding.
  %
  In the context of type theories, we will conduct investigations
  concerning a main problem in the area: to reconcile extensional equality
  and intensional identity (which, roughly, respectively correspond to the
  mathematical and computational notions of sameness).  
  %
  In the context of logical systems, we will develop both %novel 
  calculi suggested by mathematical models and %new 
  models for so far intractable calculi, encompassing aspects of resource
  management and computational effects.  
  %A distinguishing novelty of our approach is that it will be informed by the
  %logical notion of polarisation.
  \vspace*{-3mm}%%%HACK!!!
  
\item[{\bfseries 3\enspace Calculi:}]\mbox{}\enspace\thinspace
  \emph{Development of formalisms of deduction as internal languages of
    mathematical models.}
  %--- researching type theories that go beyond current logical frameworks. 
  
  \vspace*{1mm}
  We will aim at evolving the term and type structure of current type
  theories, which in essence has not changed since the late 1960s.  Pursuing
  the view that further evolution will come from mathematical input, we will
  research type theories as formal languages of (higher-dimensional)
  categorical structures.  
  \vspace*{1mm}

\item[{\bfseries 4\enspace Programming:}]\mbox{}\enspace\thinspace
  \emph{Design and implementation of novel computational languages.}
  %--- %looking into the 
  %engineering %of first 
  %principles and concepts from mathematical theories.

  \vspace*{1mm}
  Guided by mathematical theories and pragmatics, we will look into the
  engineering of principles and concepts for programming, exporting them to
  language designs and implementations.  We will specifically consider
  experimental languages, supporting indexed data structures, computational
  effects, and metaprogramming.  
\end{myitemize}

\paragraph{The team.}

The team to pursue the research programme will be led by 
%\begin{myitemize}
%  \item 
    Principal Investigator~(PI) \emph{Marcelo Fiore} 
    (Computer Laboratory, University of Cambridge): 
    a %world-
    leading expert in applications of category theory to computer science,
    including abstract algebra, concurrency theory, programming-language
    semantics, sheaf theory, and type theory.  
%\end{myitemize}
%
He will be 
backed up %supported 
by an international group of three Senior Visiting Researchers~(SVRs):
%(that are to visit the PI for one month per year): 
%\begin{myitemize}
%\item 
  \emph{Pierre-Louis Curien} 
  (Laboratoire PPS, Univerist\'e Paris Diderot - Paris~7): 
  a %world-
  leading expert in programming-language semantics, with contributions
  into category theory, logic, %rewriting theory, 
  and type theory; 
%\item
  \emph{Peter Dybjer} 
  (Department of Computer Science, Chalmers University of Technology): 
  a %world-
  leading expert in type theory, including its connections with category
  theory and logic, and its implementation in proof assistants; and 
%\item
  \emph{Tim Sheard} 
  (Department of Computer Science, Portland State University): 
  a %world-
  leading expert in pro\-gram\-ming-language design and implementation,
  encompassing functional and metaprogramming systems. 
%\end{myitemize}
%
The PI is to collaborate with the SVRs by mutual research visits, email, and
conference calls.
%
The team is completed by three outstanding Research Associates~(RAs) to be
based at the Computer Laboratory, University of Cambridge, under the PI.  They
are
%\begin{myitemize}
%\item
  \emph{Ki Yung Ahn}
  (Department of Computer Science, Portland State University): 
  a PhD student of %Tim 
  Sheard currently writing up his dissertation on the design and
  implementation of a language for indexed programming;
%\item
  \emph{Nicola Gambino}
  (Dipartimento di Matematica e Informatica, Universit\`a degli Studi di
  Palermo):  
  an established researcher in the areas of type theory and category theory,
  with whom the PI has already collaborated; and 
%\item
  \emph{Guillaume Munch-Maccagnoni}
  (Laboratoire PPS, Univerist\'e Paris Diderot - Paris~7): 
  a PhD student of %Pierre-Louis 
  Curien writing up his dissertation on the Propositions-as-Types
  correspondence in the context of classical logic.
%\end{myitemize}
Currently, the two junior RAs lack future funding.  All RAs would be ready
to join the project from the start.

\hide{
Figure\,\ref{ercTeam} gives a schematic presentation of the team that matches
the research areas as rendered in Figure\,\ref{ResearchAreas}.  
\begin{figure}[h]
\caption{{\erc} team}
\vspace*{2mm}
\begin{center}
\hspace*{.125mm}
\xymatrix@R=10pt@C=32.5pt{
& 
\raisebox{5mm}
  {\fbox{\txt{\small Pierre-Louis\\ \small Curien}}}
\ar@/_1em/@{<->}[dddl]<-1em>
\ar@/^1em/@{<->}[dddr]<1em> 
& 
\\
\\
& 
  {\fbox{\txt{\small Marcelo\\ \small Fiore}}}
\ar@{<->}[dl]|-
  {\txt{\scriptsize Nicola\\ \raisebox{1mm}{\scriptsize Gambino}}}
\ar@{<->}[dr]|-
  {\txt{\scriptsize Ki Yung\\ \raisebox{1mm}{\scriptsize Ahn}}}
\ar@{<->}[uu]|(.475)
  {\txt{\scriptsize Guillaume\\ \raisebox{1mm}{\scriptsize
        Munch-Maccagnoni}}}
& 
\\
\raisebox{0mm}{\fbox{\txt{\small Peter\\ \small\ Dybjer\ \ }}}
\ar@/_1em/@{<->}[rr]
& & 
\raisebox{0mm}{\fbox{\txt{\small Tim\\ \small\ Sheard\ \ }}}
}
\end{center}
\vspace*{-2mm}
\label{ercTeam}
\end{figure}
%Details %Further details 
%are deferred to Section\pref{ResourcesSection}. 
}

\subsection{Origins and influences}
\label{Origins}

This section sketches the origins and influences that led to the holistic
conception of Figure\,\ref{ResearchAreas}. 
%, roughly expanding through 1935 to 1985.
It is important to appreciate and understand our scientific framework.

\hide{
Section\pref{StateOfTheArtSubsection} describes the current state of the
field and raises questions for research.  %Having set the scene,
Section\pref{ObjectivesSubsection} presents the research objectives we
wish to pursue and the team that we have assembled to reach them.  
%Both of these points are further expounded upon in
%Sections\,(\ref{MethodologySection}\,\&\,\ref{ResourcesSection}).
}

\paragraph{Logic and computation.}

%It is worth recalling that 
Computer science was born as a branch of mathematics, specifically
mathematical logic. %, even before the first electronic computers were built.
Its inception %The inception of computer science 
was Hilbert's %~\cite{?}
1928 Decision Problem, %Entscheidungsproblem (decision problem),
asking whether there is an algorithmic procedure for deciding mathematical
statements.  Negative answers were provided independently by
Church %~\cite{Church1936} 
in 1936 and 
Turing %~\cite{Turing} 
in 1937,
giving birth to the mathematical theory of computation.  Their completely
different approaches gave rise to different branches of theoretical
computer science that still persist today.  
The %On one hand, the 
line
of development starting with Church's {\lcalculus} %~\cite{?} 
concerns itself with prototypical computational languages that are
used to study high-level programming languages.  
%On the other hand, Turing's machines %~\cite{?} 
%are the most widely 
%used model for analysing computational complexity.
%
This %Church's 
view is central to this proposal.  
\hide{
The {\lcalculus} is a
deceptively simple formal system.  Its syntax consists of three types of
phrases: variables, applications, and abstraction.  Church's seminal
innovation was in introducing the latter one, 
%the other two being already familiar form the realm of
%algebra~\cite{Birkhoff}, 
which in modern terminology is referred to as a binding operator, a notion
that goes beyond the operators of universal algebra. %~\cite{Birkhoff}.
Binding operators are an integral part of all high-level programming
languages.  
%Their mathematical theory is subtle %, see~\ref{}, 
%because they define syntax up to the consistent renaming of bound
%variables (technically referred to as \mbox{$\alpha$-equivalence}).
%This subtlety is made evident by the anecdotal fact that the first definitions
%of substitution were formally flawed.  
%%These important topics are core to our proposed research in
%%Section\pref{AlgebraicTypeTheoryParagraph}.
}

%Incidentally, the notion of variable binding was initially introduced much
%earlier, by Frege~\cite{Frege1879}, in the context of mathematical logic.  
%He used binders to define a formal symbolic
%system axiomatising not only the propositional connectives of Boolean
%logic~\cite{Boole} but also, for the first time, the quantifiers of
%predicate logic.

%The %In modern notation, the 
%{\lcalculus} syntax is given by
%  \[\begin{array}{rcll}
%    s , t & ::= & & \mbox{($\lambda$-terms)}\\
%      & \mid & x & \quad\mbox{(variables)}\\
%      & \mid & (t)s & \quad\mbox{(application)}\\
%      & \mid & \lambda x.\,t & \quad\mbox{(abstraction)}
%  \end{array}\]
%The system has only one rule of computation:
%  \begin{equation}\label{BetaReduction}\begin{array}{rl}
%(\beta) & (\lambda x.\,t)s \,\longrightarrow\, t[s/x]
%  \end{array}\end{equation}
%by which the result of computing a so-called redex $(\lambda x.\,t)s$ is
%the \mbox{$\lambda$-term} $t[s/x]$ denoting the result of substituting the
%free occurrences of $x$ in $t$ by $s$.  Non-terminating behaviour arises from
%self application.

\paragraph{Type theory and logic.}
\label{SectionATypeTheoryAndLogicParagraph}

The concept of `type' was conceived to solve a foundational
problem.  In 1903, %~\cite{Frege1903}, building on~\cite{Frege1879}, 
Frege proposed
a logical system as a foundation for mathematics.
%including arithmetic.  
In his now famous paradox, 
Russell %~\cite{Russell1902}
observed that one of Frege's axioms led to inconsistency.  
%He was led to this contradiction by related contradictions he found while
%developing his account of set theory.  
To overcome such foundational problems, 
he %Russell 
introduced the `doctrine of types'. 
% --- typed systems can avoid these inconsistencies.

Type Theory, as we now know it, arose from the integration by Church
of types into his \lcalculus, yielding the %so-called 
Simply-Typed Lambda Calculus. %~\cite{Church1940}.  
This was a natural step, particularly if one held the naive interpretation of
\mbox{$\lambda$-abstraction} as defining functions.  
%The type theory known as the Simply-Typed Lambda Calculus has a set of types
%consisting of basic ones
%closed under a function-type constructor. %:
%\[\begin{array}{rcll}
%    \sigma,\tau & ::= & & \mbox{(simple types)}\\
%    & \mid & \theta & \quad\mbox{(basic types)}\\
%    & \mid & \sigma\to\tau & \quad\mbox{(function types)}
%  \end{array}\]
%In the Simply-Typed Lambda
%Calculus, \mbox{$\lambda$-terms}~$t$ are classified by types~$\tau$, in
%contexts $\Gamma$ (assigning types to variables), for which the notation
%$\Gamma\vdash t:\tau$
%is commonly used.  This is done according to syntax-directed rules. 
% that in the mathematical vernacular are presented as follows
%  \[
%  \begin{array}{c}
%    \\ \hline
%    x_1:\tau_1,\ldots,x_n:\tau_n\vdash x_i:\tau_i
%  \end{array}
%  \enspace(1\leq i\leq n)
%\]
%  \[\begin{array}{c}
%    \Gamma\vdash t:\sigma\to\tau
%    \quad 
%    \Gamma\vdash s:\sigma
%    \\ \hline
%    \Gamma\vdash (t)s:\tau
%\end{array}\]
%\vspace*{-2mm}
%  \begin{equation}\label{LambdaAbstraction} 
%  \begin{array}{c}
%    \Gamma,x:\sigma\vdash t:\tau
%    \\ \hline
%    \Gamma\vdash\lambda x.\,t:\sigma\to\tau
%  \end{array}
%\end{equation}
%\vspace*{-2mm}

A fundamental discovery relating Mathematical Logic to Type Theory was
made by Curry and by Howard.
%in two slightly different, though related, logical
%contexts (technically, Hilbert-style deduction %~\cite{Hilbert} 
%and Gentzen's natural deduction in sequent form). %~\cite{Gentzen1935}).
This is the Propositions-as-Types correspondence; roughly establishing a
correspondence between terms having types and proofs proving propositions.
%(\cf~(\ref{PATproportion})).  
%For the Simply-Typed Lambda Calculus, the correspondence
%can be illustrated by noting that the erasure of term information in 
%the typing rules (given above) 
%yields the deduction rules
%\[\begin{array}{c}
%  \begin{array}{c}
%    \\ \hline
%    \tau_1,\ldots,\tau_n\vdash \tau_i
%  \end{array}
%  \enspace(1\leq i\leq n)
%  \\[4mm]
%  \begin{array}{c}
%    \Gamma\vdash \sigma\to\tau
%    \quad 
%    \Gamma\vdash\sigma 
%    \\ \hline
%    \Gamma\vdash\tau
%  \end{array}
%  \enspace\quad
%  \begin{array}{c}
%    \Gamma,\sigma\vdash\tau
%    \\ \hline
%    \Gamma\vdash\sigma\to\tau
%  \end{array}
%\end{array}
%\]
%derivable judgements $x_1:\tau_1,\ldots,x_n:\tau_n\vdash t:\tau$ yields the
%valid judgements $\tau_1,\ldots,\tau_n\vdash\tau$ 
%of Intuitionistic %Minimal 
%Logic. %~\cite{?}.

In another very important direction Church introduced a Simple Theory of
Types %~\cite{Church1940} 
as an axiomatization of Higher-Order Logic.
This axiomatization was formalised within a Simply-Typed Lambda Calculus.
%with basic types for both individuals and propositions, further enriched with
%constants for the logical connectives.  
In doing so, he adopted a radically new perspective, shifting the status
of the Simply-Typed Lambda Calculus from that of a `language' to a
`metalanguage'. %;~\ie, a language in which it is possible to represent and
%work with other languages. 
%In this particular case for simple type theories.  
\hide{
  When regarded as a metalanguage, the Simply-Typed
Lambda Calculus is considered as an equational theory. %, with the
%computational \mbox{$\beta$-rule}~(\ref{BetaReduction}) stated as an
%equation together with the extensionality equation
%  \[\begin{array}{rll}
%(\eta) & \lambda x.\,(t)x = t 
%& \mbox{, where $x$ is not free in $t$}
%  \end{array}\]
%\vspace*{-4mm}
}

\hide{
The processes of abstracting from languages to metalanguages has become a
common activity in computer science, and plays an important role in our
proposed investigations.  
}

\paragraph{Category theory, logic, and type theory.}

\hide{
The theory of categories was introduced by Eilenberg and Mac
Lane~\cite{EilenbergMacLane}.  A category is a mathematical structure
consisting of objects and morphisms.  Morphisms are classified
by pairs of objects. The notation $f:A\to B$ stipulates that $A$ and $B$ are
objects, and that $f$ is a morphism with domain $A$ and codomain~$B$.  
Categories come equipped with an associative law that composes pairs of
morphisms 
%as on the left below
%\begin{equation}\label{Category}
%  \begin{array}{c}
%  f:A\to B\quad g:B\to C
%  \\ \hline
%  g\,f:A\to C
%\end{array}
%\quad\qquad
%1_A:A\to A
%\end{equation}
together with, for every object, an identity morphism 
%as on the right above
that is a neutral element for composition.  
%Categories were defined to introduce functors, a notion of morphism between
%categories, which in turn was defined to introduce natural transformations, a
%notion of morphism between functors.

In understanding categories, it is helpful to have in mind
that they have two kinds of uses: in the
large and in the small.  In the large, categories are seen as mathematical
universes of discourse (within which mathematical constructions take place,
typically by means of universal properties); like the categories of: sets and
functions, algebraic structures and homomorphisms, spaces and continuous
functions.  In the small, categories are seen as mathematical objects
themselves; like a set, a monoid, a preorder, and a graph, all of which can
be suitably regarded as a category.
}

Category Theory analyses mathematical structure by isolating the principles
for which mathematical theories work.  
%Because of this attention to essentials, it has had considerable success
%in unifying ideas from many areas of mathematics. 
%Today it an indispensable tool in abstract algebra, algebraic geometry,
%mathematical logic, mathematical physics, topology, and theoretical computer
%science, as well as a growing research area in its own right.
%
The connection between Category Theory, Logic, and Type Theory was
initiated by Lawvere %~\cite{LawvereAinF} 
and Lambek. %~\cite{LambekI}, 
%both
%of whom recognised logical systems and type theories as categories with
%structure.  
%Here we are concerned with the former, postponing the latter to the following
%section.

Lawvere's insight was to understand logical connectives and type
constructors as categorical structures arising from 
%universal properties in the form of 
adjoint functors. %, a notion introduced by Kan~\cite{Kan}
%motivated by homology theory.  
%For instance, the categories with structure corresponding to the Simply-Typed
%Lambda Calculus with products are Cartesian Closed Categories.  These have
%categorical products and exponentials 
%, respectively denoted $\times$ and $\Rightarrow$, 
%defined by adjoint situations.
%establishing natural bijective correspondences between
%morphisms as follows:
%  \[
%  \begin{array}{c}
%    C\to A \enspace,\enspace C\to B
%    \\ \hline\hline
%    C\to (A\times B)
%  \end{array}
%  \quad\qquad
%  \begin{array}{c}
%    (C\times A)\to B
%    \\ \hline\hline
%    C\to (A\Rightarrow B)
%  \end{array}
%\]
%The required naturality condition amounts to the computational $\beta$
%laws and the extensionality $\eta$ laws.
%
It
%The similarity between the bijective correspondence on the right above and the
%typing rule for \mbox{$\lambda$-abstraction}~(\ref{LambdaAbstraction}) is not
%casual, and it 
is now well-understood that the Simply-Typed Lambda Calculus with products
provides an internal language for Cartesian Closed Categories; namely, it is
the calculus of all such models.  This view further enriches the
Propositions-as-Types correspondence as 
in the trinity below %in Figure\,\ref{PAT}.
%\begin{figure}[h]
%\caption{Propositions-as-Types Trinity}
%\vspace*{2mm}
%\[
\begin{equation}\label{PATtrinity}
\hspace*{-3.75mm}
\begin{array}{c}
\xymatrix@C=0pt@R=17.5pt{
    & \ar@{<->}[dl]_(.5){\null\hspace*{-15mm}\txt{\scriptsize Internal\\
        \raisebox{1mm}{\scriptsize Language}}} 
    \ar@{<->}[dr]^(.55){\null\hspace*{1mm}\txt{\scriptsize Mathematical\\
        \raisebox{1mm}{\scriptsize Models}}}
    \txt{\small Cartesian Closed\\ \small Categories} & \\
    \txt{\small Simply-Typed\\ \small Lambda Calculus\\ \small with products
    }
    \ar@{<->}[rr]_-{\txt{\scriptsize
        Propositions\\\raisebox{1mm}{\scriptsize as Types}}} & & 
    \txt{Propositional\\ 
      Intuitionistic\\ Logic}
  }
\end{array}
\vspace*{-1mm}
\end{equation}
%\]
%\vspace*{-2mm}
%\label{PAT}
%\end{figure}
%and is the standard with respect to which analogous developments in the
%area are measured.  

\paragraph{Type theory and programming.}

The Simply-Typed Lambda Calculus 
%, regarded as a language rather than as a metalanguage, 
is a prototypical functional programming language.  The change of
perspective from Type Theory to Programming Theory is not straightforward
and comes with considerations that enrich both subjects.  

In the context of programming languages, Milner %~\cite{Milner1978}
understood early on that while a programming language should come with a
type discipline to classify programs according to their type invariants,
programmers would be better served if type annotations were inferred
automatically.  
%
The problem of type inference 
%(by which given a program one wishes to compute the best possible type for
%it) 
became of central practical importance.  
%For
%Combinatory Logic %\footnote{Combinatory Logic is an important symbolic
%  formalism introduced by Sch\"onfinkel~\cite{Schonfinkel} closely related
%  to the {\lcalculus}, but based on algebraic combinators, that enjoys the
%  Propositions-as-Types correspondence with respect to the Hilbert-style
%  deduction system for Intuitionistic %Minimal 
%  Logic.} 
%this problem was solved by Hindley. %~\cite{Hindley1969}.  
%Independently, 
Milner %~\cite{Milner1978} 
solved it %in a context more relevant to programming; namely, 
for the Simply-Typed Lambda Calculus with parametric polymorphism.  
%The algorithm is known as the Hindley-Milner type inference method.  
%The approach is very robust. 
%, extending broadly to many related systems.
 
Polymorphism in programming refers to languages that support abstraction
mechanisms by which a program %(function or procedure) 
can be used with a variety of types.  The concept was introduced by 
Strachey. %~\cite{Strachey1967},
%who further classified the phenomena into ad-hoc polymorphism and parametric
%polymorphism.  The former is also referred to as overloading and accounts for
%uses of a program with different types (like integers and reals) by means of
%different algorithms.  The latter indicates the use of a uniform program for
%all types (like a duplicator program making copies of its input).
%
Reynolds %~\cite{Reynolds} 
formalised the programming intuition by introducing
the Polymorphic Typed Lambda Calculus. 
%; an extension of the Simply-Typed Lambda
%Calculus with polymorphic types.  
%Roughly, these are abstract parametric types
%whose programs can be used for all instances of the parameter.  
Strikingly,
this system had already been proposed %several years earlier 
by
Girard %~\cite{GirardSystemF} 
under the name of {\SystemF}, as the
type-theoretic counterpart of Second-Order Propositional Logic. 
%in the context of the Propositions-as-Types correspondence.  
Further, Girard had
also introduced {\SystemFomega}; the type-theoretic counterpart of
Higher-Order Propositional Logic.  
%These logical systems widely extend the
%expressiveness of the Simply-Typed Lambda Calculus, notably by the
%possibility of encoding inductive data types. %~\cite{BoehmBerarducci}.
{\SystemFomega} lies at the core of the Haskell programming
language. %~\cite{EqProofICFP12}.

The above type systems aim at providing logical foundations.  When viewed
as programming languages, they can only introduce terminating
computations.  There are various ways in which one can extend them to
Turing
complete computational languages.  In this direction, 
%and motivated by model-theoretic studies of the \lcalculus, 
Scott %~\cite{ScottTCS}
introduced an extension of Typed Combinatory Logic with a fixpoint
combinator for general recursion.  Plotkin %~\cite{PlotkinLCF} 
studied it as a programming language
in the context of the %, shifting from Typed Combinatory Logic to
Simply-Typed Lambda Calculus, and %In doing so, he 
opened up further possible distinctions in the study of type theories for
computation; namely, the consideration of equational theories for
different function call mechanisms: by value or by name,
%~\cite{PlotkinCBVCBN}, 
as in ML or Haskell.  

%The influence of mathematical models on type theories, logical systems, and
%programming languages %as advocated by Scott~\cite{?} 
%plays a central role throughout the proposal.

\subsection{Questions and pathways for research}

We now %Having introduced the scientific framework, we 
turn attention to topics of active research, identifying questions and
pathways for research.  In connection to them, our research plan is
outlined in the next section.

\setcounter{paragraph}{0}
\paragraph{Foundations.}

We have already mentioned several type theories, and we will mention a few
more in the sequel.  However, the following fundamental question remains
open:
\req{(\ref{AlgebraicTypeTheoryParagraph})}
{What is a type theory?}

\hide{
Section\pref{AlgebraicTypeTheoryParagraph} aims at a comprehensive
mathematical answer, that will also serve as a framework for our other
type-theoretic developments.  
\hide{
  We regard this as a step towards the related open question:
\req{}{What is a programming language?}
that will be kept in the background of our investigations.
}
}

\hide{%%% BEGIN hide
\paragraph{Logical wiring.}
\label{LogicalWiringSubsection}

Lambek~\cite{LambekI} recognised the similarity between the basic
structure of a category, given by identities and
composition~(see\pref{Category}), and the wiring of logical deduction
systems, specifically Gentzen's sequent calculi~\cite{Gentzen1935}, given
by the axiom and cut rules.  Intuitionistic sequents led
Lambek~\cite{LambekII} to axiomatize their algebra under the concept of
multicategory.  This correspondence is roughly as outlined below (where the
vector notation $\vec{\,\cdot\,}$ stands for finite sequences of objects): 
\begin{center}\begin{tabular}{|c|c|}\hline
    \begin{tabular}{c}
      \small Intuitionistic\\[-1mm] \small Sequent Calculus
  \end{tabular}
  &
  \begin{tabular}{c}
    \small
    Multicategories
  \end{tabular}
  \\ \hline\hline
  \begin{tabular}{l}
    \hspace*{-15mm}
    \small 
    (Axiom)
    \\[-3mm]
    $\begin{array}{c}\\ \hline P \vdash P\end{array}$
  \end{tabular}
  & 
  \begin{tabular}{l}
    \small
    \hspace*{-7mm}
    (Identity)
  \\[1mm]
  $(A)\stackrel{1_A}\longrightarrow A$
  \end{tabular}
  \\ \hline
  \begin{tabular}{l} 
    \hspace*{-2mm}(Cut) 
    \\[1mm]
    $\begin{array}{c}
    \Gamma\vdash P\quad\Gamma_1,P,\Gamma_2\vdash Q
    \\ \hline 
    \Gamma_1,\Gamma,\Gamma_2\vdash Q
    \end{array}$
  \end{tabular}
  &
  \begin{tabular}{l}
    \small
    \hspace*{-3mm}
  (Multicomposition)
  \\%[1mm]
    \hspace*{-2mm}
    $\begin{array}{c}
    \vec Y \stackrel f\longrightarrow A
    \\
    \vec X,A,\vec Z \stackrel g\longrightarrow B
    \\ \hline\\[-4mm]
    \vec X, \vec Y,\vec Z
    \xymatrix@C=25pt{\ar[r]^-{g\, _{\vec
          X}\hspace*{-.0125mm}\circ\hspace*{-.25mm}_{\vec Z}f}&} 
    B
    \end{array}$
    \hspace*{-2mm}
  \end{tabular}
  \\ \hline
\end{tabular}\end{center}
The analogous for classical sequents was done by Szabo~\cite{Szabo} with the
introduction of polycategories.

The structure of multicategory appeared independently in a very different
mathematical context, the work of May~\cite{May} on homotopical algebra, under
the name of Operad.  Operads are now central to studies in higher-dimensional
algebra (see~\eg~\cite{Leinster,LodayVallette}).  

\rep{(\ref{WiringStructureParagraph})}
{Investigate interactions between logical sequent calculi and algebraic operad
  theory.}
}%%% END hide

\paragraph{Dependent types.}
\label{DependentTypesParagraph}

Dependent type theory is a formalism introduced by
de~Bruijn %~\cite{deBruijn} 
that extends simple type theories by allowing types to be indexed 
%(or parameterised by) 
other types.  Such objects abound in computer science and mathematics.  
%For instance, in
%combinatorics one is interested in the type of permutations $\mathfrak
%S(n)$ on $n$ elements, as the index (or parameter) $n$ ranges over the
%natural numbers $\mathbb N$.  In modern notation, this is presented by a
%judgement of the form
%  \[
%  n:\mathbb N \vdash \mathfrak S(n) \mbox{ \mysf type}
%\]
There are two fundamental constructions on %such 
dependent judgements: %as follows
%\[
%  \begin{array}{c}
%  i: I \vdash T(i) \mbox{ \mysf type}
%  \\ \hline
%  \vdash \Sigma\,{i:\!I}.\, T(i) \mbox{ \mysf type}
%  \end{array}
%  \qquad
%  \begin{array}{c}
%  i: I \vdash T(i) \mbox{ \mysf type}
%  \\ \hline
%  \vdash \Pi\,{i:\!I}.\, T(i) \mbox{ \mysf type}
%  \end{array}
%\]
%respectively known as 
dependent sums and dependent product types.  %(see~\eg~\cite{Jacobs}).  
They generalise the product and function types of Simply-Typed Lambda
Calculus and, under the Propositions-as-Types correspondence, amount to
existential and universal quantification.  
%We omit discussing the syntax of terms for these types.  As for their
%equational theory, we only mention that dependent sums may be weak or strong
%and that dependent products may be intensional or extensional.
%%---see \eg~\cite{?}.

A fundamental problem in the area is to:
\rep{(\ref{IntensionalTypeTheoryParagraph})}
  {Investigate equality and identity in dependent type theory.}
%This lays at the core of our proposed investigations in
%Section\pref{IntensionalTypeTheoryParagraph}.

The passage from sum and product types to their dependent versions
required new type theories.  Categorical models suggest further
generalisations 
and the following %These are the subject of
%Sections\,(\ref{GeneralisedTypeTheoryParagraph}\,\&\,\ref{DirectedTypeTheoryParagraph})
%Section\pref{CalculiSubsection}
%under the following. 
\rep{(\ref{CalculiSubsection})}
  {Develop type theories from mathematical models.}

\paragraph{Mathematical universes.}
\label{MathematicalUniversesParagraph}

The ability to construct new mathematical universes of discourse from old ones
is a fundamental part of the technical toolkit of researchers in semantics, be
it either in logic or computation.

One technique to do so is to enrich the semantic universe with a `mode of
variation' by means of a so-called presheaf construction.  
%In its basic form, starting from the universe of sets and functions~$\Set$ one
%considers the universe~$\Set^{\scat C}$ of so-called presheaves consisting of
%the \mbox{$\scat C$-variable} sets for a small category~$\scat C$.  
%The importance of this passage is that the kind of variation embodied in the
%parameter small category translates to new, often surprising, internal
%structure in the universe of presheaves.  

%The presheaf construction was introduced by Grothendieck together with a
%more sophisticated and important refinement of it known as the sheaf
%construction~\cite{SGA4} (in the context of the Weil conjectures in
%cohomology theory).  Sheaves are a central object of study in the area of
%mathematics known as Topos Theory~\cite{Elephant}; a topos being a
%universe of discourse for Higher-Order Intuitionistic
%Logic~\cite{LambekScott}.  

The use of presheaf models in computer science applications has been
prominent;~\eg~in 
concurrency theory, %~\cite{?}, 
domain theory, %~\cite{?}, 
lambda calculus, %~\cite{?}, 
programming-language theory, %~\cite{?}, 
and type theory. %~\cite{?}.  
%
%In this view, 
%In view of the many possible applications, 
It is thus %it is 
natural to ask:
\reqs{(\ref{MethodologyMathematicalUniversesParagraph})}
  {Which mechanisms are there for changing from a type theory to another one
    as universes of discourse? 
%
    Can this be done while maintaining the relevant computational properties
    and then incorporated into mechanical proof assistants?}

%This question should not only be considered from the topos-theoretic viewpoint
%mentioned above; but also from other approaches, notably that of the related
%forcing technique of Cohen~\cite{Cohen} (introduced by him to prove the
%independence of both the axiom of choice and the continuum hypothesis from
%Zermelo-Fraenkel Set Theory) and its elaboration by Scott and Solovay as
%Boolean-valued models~\cite{ScottSolovay}.

%Recent work in this direction has been done by
%Coquand~\cite{CoquandForcingInTT} and Jaber, Tabareau, and
%Sozeau~\cite{TypeTheoryWithForcing}.\footnote{Note
%  that~\cite{TypeTheoryWithForcing}, despite its title, is about internalising
%  the presheaf construction on partial orders.}

\paragraph{Indexed programming.}
\label{IndexedProgrammingIntro}

In their most elementary discrete form, presheaves can be found in programming
languages as indexed datatypes; programming with which will be %generically
referred to as indexed programming.  

Indexed programming developed from two main influences:
%(none to do with presheaves though): 
the practical needs of supporting data structures able to
maintain strong data invariants, 
%, like nested %~\cite{?} 
%and generalised algebraic %~\cite{?}
%datatypes (GADTs) in functional programming~\cite{Omega,Haskell}; 
and the experimentation with dependently-typed programming
languages %~\cite{Cayenne,Epigram} 
as a by-product of dependent type theory.  These two views somehow pull in
opposite directions and, as such, lead to conceptually different
languages.  One is lead to investigate the following.
\rep{(\ref{IndexedProgrammingParagraph})} 
  {Develop foundational type theories for indexed datatypes.  
   % 
   Design and %subsequently 
   implement indexed programming languages from these and
   %programming-language 
   pragmatics.}

\paragraph{Resources, effects, modalities.}
\label{ResourcesEffectsModalitiesParagraph}

In the late 1980s, two important analyses of computation, respectively for
resources and effects, were proposed by Girard %~\cite{GirardLinearLogic} 
and by Moggi. %~\cite{MoggiLambdaC}.  
The former was in the contexts of logic and proof theory; the latter in that
of denotational semantics and category theory.  Both %, however, have 
had tremendous impact in programming-language theory.  
%The resource analysis in the form of Linear Logic calculi; the effect
%analysis in the form of Computational \mbox{$\lambda$-calculi}.

\hide{
From the viewpoint of categorical models, the resource and effect management
structures are respectively seen as comonadic and monadic structures.
Comonads and monads being dual categorical concepts that arose in the context
of cohomology theory in the 1960s~\cite{BeckThesis}.

A classical basic result of category theory establishes a strong
correspondence between (co)monads and adjoint functors
(see~\eg~\cite{MacLane}).  One aspect of this is that every adjunction
  \begin{equation}\label{ResourceEffectAdjunction}
  \xymatrix@C=40pt{
    \cat D \ar@/^.75em/[r]|(.625){\mbox{$\,G\,$}}\ar@{}[r]|-\swvdash&
    \ar@/^.75em/[l]|(.625){\mbox{$\,F\,$}} \cat C }
\end{equation}
with $F$ and $G$ respectively left and right adjoint to each other, gives
rise (by composition) to a comonad on $\cat D$ and a monad on $\cat C$.  This
viewpoint gave impetus to further analyses based on the more primitive notion
of adjunction.  

Models of Linear Logic are founded on the theory of
monoidal categories~\cite[Chapter~VII.1]{MacLane}.  For them, one requests
that the adjunction be monoidal with respect to linear structure on $\cat D$
and cartesian (or multiplicative) structure on $\cat C$
(see~\eg~\cite{MelliesCMLL}).  On the other hand, models of Computational
\mbox{$\lambda$-calculi} rely on enriched category theory~\cite{KellyBook}.
Here, the structure is roughly given by an enriched adjunction with $\cat C$
cartesian and $\cat D$ with powers (or exponentials) relative to $\cat C$.
}

\hide{
In the context of Linear Logic, examples are the mixed linear/cartesian models
and calculi of Benton and Wadler, %~\cite{BentonWadler} 
and of Barber and Plotkin. %~\cite{BarberPlotkin}.  
In the context of effect calculi, examples are the Call-By-Push-Value of
Levy. %~\cite{LevyCBPV} 
and the Enriched Effect Calculus of 
Egger, M{\o}gelberg, and Simpson. %~\cite{EEC}.  
}
The predominant model-theoretic view of resources and effects is as
structures arising at opposite sides of an adjunction.  This is not so in
all models and the following question remains unanswered.
%
\req{\pref{PolarisationParagraph}}
  {How can resources and effects be reconciled and unified?}
%
To answer it, a broader view of the subject involving the orthogonal
notion of polarisation seems to be needed.  
This %From the programming-language viewpoint, this 
further enriches the computational picture with the ability of
distinguishing between eager \vs~lazy modes of computation and data
structures.  We incorporate this into the following.
%
\rep{(\ref{PolarisationParagraph}\,\&\,\ref{ProgrammingEffectsParagraph})}
  {Study and develop the theory of resource management, computational
    effects, and polarisation.  Percolate this down into the design of
    programming languages.}

\hide{
From the logical point of view, resource and effect structure are so-called
modal operators, and some of the literature has indeed considered them as
such.  %(see \eg~\cite{Kobayashi}).  %BiermanDePaiva
The field of modal logics is broad, with many subfields of specialised logics
motivated by computation, linguistics, and philosophy.  While a class of modal
logics known as temporal logics have played a prominent role, and been very
successful, in the area of computer aided verification; the impact of modal
logics on programming languages has been peripheral.  It is thus natural and
important to reconsider them in this context. 
%
\rep{(\ref{ModalLogicsParagraph}\,\&\,\ref{MetaprogrammingParagraph})}
  {Investigate the Propositions-as-Types 
    correspondence %Trinity 
    for modal logics, and apply it to programming languages.}
We stress that we are specially interested in modalities for computation with
reflection.
}

\paragraph{Sequent calculi.}
\label{SequentCalculiParagraph}

A theme that runs orthogonal to the logics under consideration is whether
they are specified in natural deduction or sequent calculus
style. %(see \eg~\cite{vonPlato}).

Most of the work on the Propositions-as-Types correspondence has been done for
natural deduction systems. %; especially in relation to programming
%language theory, where the logical syntax matches that of functional
%languages.  
%
On the other hand, there is as yet no established syntax for sequent-style
calculi.  
%A question at the core of this situation is:
Two questions at the core of this situation are:
%
\reqs{(\ref{ProgrammingEffectsParagraph})}
  {What is the categorical algebra of classical sequent calculi? %}
%
\hide{
Nevertheless, a syntactic formalism that is proving robust in applications
seems to be emerging.  This will be referred to as the calculus,
formalism, or system~\SysL; and underlies the ones developed
in~\eg~\cite{CurienHerbelin,Wadler,Munch,CurienMunch}.

The main novel features of the formalism~\SysL, and its philosophy, are an
intrinsic symmetry (corresponding to the computation roles of program and
environment) with syntax that reflects an adjoint situation and is closely
connected to abstract machines (which are in fact internal to the
calculus).  From this perspective we ask:
}
%
%\req{(\ref{ProgrammingEffectsParagraph})}
  %{
%\item 
  What can the proof theory of sequent calculi do for programming?}

\subsection{Research methodology and plan}

{\color{red}FINISH}

\clearpage
\setcounter{page}{1}
\chead{Part\,B1(b)}
\twocolumn[\begin{@twocolumnfalse}
	\hfill{\bfseries\Large 
    Curriculum Vitae
  }\hfill\null\\
\end{@twocolumnfalse}]

\subsection*{\S\enspace\thinspace Personal data}

\begin{myitemize}
\item
\textbf{\em Name}: 
\begin{tabular}[t]{l}
  Marcelo Pablo Fiore.
\end{tabular}

\item
\textbf{\em Nationality}: 
\begin{tabular}[t]{l}
  Argentinian/Italian.
\end{tabular}

\item
\textbf{\em Date of birth}: 
\begin{tabular}[t]{l}
  June 29, 1966.
\end{tabular}

\item
\textbf{\em Marital status}: 
\begin{tabular}[t]{l}
  Married, with two children (8\,\&\,3).
\end{tabular}

\item
\textbf{\em Address}: 
\begin{tabular}[t]{l}
  University of Cambridge, 
  Computer\\ Laboratory, 15 JJ Thomson
  Avenue,\\ Cambridge CB3 0FD, UK.
\end{tabular}

\item
\textbf{\em Web}: 
\begin{tabular}[t]{l}\small
  \url{www.cl.cam.ac.uk/~mpf23}
\end{tabular}

\item
\textbf{\em Email}: 
\begin{tabular}[t]{l}\small
  \texttt{Marcelo.Fiore@cl.cam.ac.uk}
\end{tabular}

\item
\textbf{\em Phone}: 
\begin{tabular}[t]{l}
  +44 (0)1223 334622.
\end{tabular}
\end{myitemize}

\subsection*{\S\enspace\thinspace University appointments}

\begin{myitemize}
\item
\textbf{\em Professor} in \emph{Mathematical Foundations of Computer
Science},  Computer Laboratory, University of Cambridge.  Since October
2011.

\item
\textbf{\em Reader} in \emph{Mathematical Foundations of Computer
Science}, Computer Laboratory, University of Cambridge.  October 2005 to
September 2011.

\item
\textbf{\em University Lecturer}, Computer Laboratory, University of
Cambridge.  October 2000 to September 2005.

\item
\textbf{\em Lecturer} in Computer Science, School of Cognitive and
Computing Sciences, University of Sussex.  April 1998 to September 2000. 

\item
\textbf{\em Research Fellow}, Laboratory for Foundations of Computer
Science, Department of Computer Science, University of Edinburgh.  January
1994 to March 1998.
\end{myitemize}

\subsection*{\S\enspace\thinspace University education}

\begin{myitemize}
\item
{\em Postgraduate}: \textbf{\em PhD in Computer Science}, University of
Edinburgh, June 1994.  \emph{Thesis}: Axiomatic Domain Theory in
Categories of Partial Maps.  \emph{Supervisor}: Prof Gordon Plotkin
(University of Edinburgh).  \emph{Examiner}: Prof Dana Scott (Carnegie
Mellon University)

\item
{\em Graduate}: \textbf{\em Licentiate in Informatics}, Escuela Superior
Latino Americana de Inform\'atica, Universidad Nacional de Luj\'an, Buenos
Aires (Argentina), December 1989.  \emph{Thesis}: On-line Algorithms for
Traversal of Dynamic Directed Hypergraphs with Application to
Satisfiability of Horn Formulae.  \emph{Supervisor}: Prof Giorgio
Ausiello, Uniersit\`a di Roma ``La Sapienze'', Italy.
\end{myitemize}

\subsection*{\S\enspace\thinspace Awards}

\begin{myitemize}
\item
\textbf{\em 10-Year Most Influential PPDP Paper Award} for the article
\emph{Semantic Analysis of Normalisation by Evaluation for Typed Lambda
Calculus} in the 2002 International Symposium on Principles and Practice
of Declarative Programming.

\item
\textbf{\em Distinguished Dissertation in Computer Science} for 1995.
\emph{Axiomatic Domain Theory in Categories of Partial Maps}.  Doctoral
thesis selected by the Conference of Professors and Heads of Computing in
conjunction with the British Computer Society.
\end{myitemize}

\subsection*{\S\enspace\thinspace Postgraduate supervision}
\vspace*{-1.9mm}
\begin{myitemize}
\item[]\small\ at the Computer Laboratory, University of Cambridge
\end{myitemize}

\begin{myitemize}
\item
  Marco Ferreira Devesas Campos.  Second year of PhD in Computer Science.

\item
  Ola Mahmoud, 2011 PhD in Computer Science.  \emph{Thesis}: Second-Order
  Algebraic Theories.

\item
  Marco Ferreira Devesas Campos, 2011 MPhil in Advanced Computer Science.
  \emph{Research essay}: Generalizing Bigraphs to DAG-like Place Graphs.

\item
  Eirik Tsarpalis, 2010 Certificate of Postgraduate Studies.

\item
  Maciej Wos, 2010 MPhil in Advanced Computer Science. \emph{Research essay}:
  Generic Programming with Fixed-Points for Nested Datatypes.

\item
  Chung-Kil Hur, 2010 PhD in Computer Science.  \emph{Thesis}: Categorical
  Equational Systems: Algebraic Models and Equational Reasoning.

\item
  Samuel Staton, 2007 PhD in Computer Science.  \emph{Thesis}:
  Name-Passing Process Calculi: Operational Models and Structural
  Operational Semantics.
\end{myitemize}

\subsection*{\S\enspace\thinspace Mentoring}
\vspace*{-2mm}
\begin{myitemize}
\item[]\small\ at the Computer Laboratory, University of Cambridge
\end{myitemize}

\begin{myitemize}
\item
Bartosz Klin.  EPSRC Postdoctoral Research Fellow in Theoretical Computer
Science.  October 2008 to September 2011.  

\item
Johan Glimming.  Visiting Research Fellow funded by the Swedish Research
Council VR.  October 2008 to September 2010.  

\item
Samuel Staton.  EPSRC Postdoctoral Research Fellow in Theoretical Computer
Science.  June 2007 to May 2010.  
\end{myitemize}

\subsection*{\S\enspace\thinspace Examining}

\begin{myitemize}
\item
\textbf{\em External} committee member for the PhD theses of: 
%
Wei Chen (University of Nottingham, 2012);
%
Julianna Zsido (Universit\'e de Nice Sophia Antipolis, 2010); 
%
Stephane Gimenez (Universit\'e Paris Diderot - Paris~7, 2009); 
%
Vincenzo Ciancia (Dipartamento di Informatica, Universit\`a di Pisa,
2008); 
%
Joachim de Lataillade (Universit\'e Paris Diderot - Paris~7, 2007); 
%
Emmanuel Beffara (Universit\'e Paris Diderot - Paris~7, 2005); 
%
Pierre Hyvernat (Institut Math\'ematique de Luminy, 2005);
%
Bartosz Klin (Aarhus University, 2004);
%
Vincent Balat~(Universit\'e Paris Diderot - Paris~7, 2002);
%
Krzysztof Worytkiewicz~(\'Ecole Polytechnique F\'ed\'erale de Lausanne,
2000).

\item
\textbf{\em Internal} committee member for the PhD theses of: 
%
Bjarki Holm (2011); 
%
Ranald Clouston (2009); 
%
Paul Hunter (2007); 
%
Lucy Brace-Evans (2007).

\item
Committee member for the Professorship habilitation of
%
Davide Sangiorgi~(INRIA Sophia Antipolis, 2002).

\item
Committee member for the MPhil thesis of
%
Henrik Enstr{\o}m (Aarhus University, 1998).

\item
Computer Science Tripos Examiner for Part IB, Part II, Part II (General),
and Diploma in 2006/07, 2007/08, and 2008/09 (Chair).  
\pagebreak
Computer Laboratory, University of Cambridge.  
\end{myitemize}

\vspace*{-1.25mm}
\subsection*{\S\enspace\thinspace Funding}
\vspace*{-1mm}
\begin{myitemize}
\item
  EPSRC grant submission (under consideration).
  \emph{Efficient Extraction of Feasible Programs}.  \textbf{\em Principal
    Investigator}, with co-investigators Timothy Griffin and Glynn Winskel
  (University of Cambridge).  Joint submission with Dr Mart\'{\i}n
  Escard\'o (University of Birmingham), Paulo Oliva (Queen Mary University
  of London), Ulrich Berger and Monika Seisenberger (Swansea University).
  Travel funding for 24 months.

\item
  \textbf{\em Visiting grant} from the Gunma University Foundation for
  Science and Technology.  Department of Computer Science, Gunma
  University.  April 3--12, 2012.

\item
  \textbf{\em Research in Paris grant} from the \emph{programme d'accueil
    des chercheurs \'etrangers de la Ville de Paris}, Laboratoire PPS,
  Universit\'e Paris Diderot - Paris~7.  October--December 2011.

\item
  \emph{Domain theory for concurrency---New categorical foundations}.
  \textbf{\em Co-investigator} of Glynn Winskel.  EPSRC Grant
  GR/T22049/01.  July 2005 to June 2008.

\item
  \emph{Mathematical models for functional and concurrent computation}.
  \textbf{\em EPSRC Advanced Research Fellowship}. October 2000 to
  September 2005.  
\end{myitemize}

\vspace*{-1.25mm}
\subsection*{\S\enspace\thinspace Invited research visits}
\vspace*{-1mm}

\begin{myitemize}
\item
  Short-Term Scholar invitation to the Univalent Foundations of Mathematics
  Program at the School of Mathematics, Institute for Advanced Study,
  Princeton. 

\item
  Assistant Prof Makoto Hamana at the Department of Computer Science,
  Gunma University.  In April 2012.

\item
  Prof Peter Dybjer at the Department of Computer Science and Engineering,
  Chalmers University of Technology.  In September 2008. 

\item
  Prof Anders Kock at the Department of Mathematical Sciences, Aarhus
  University.  In August 2007. 

\item
  Prof Mart\'{\i}n Abadi at the Department of Computer Science, University
  of California at Santa Cruz.  November 2003. 

\item
  Dr Vincent Danos at Laboratoire PPS, Universit\'e Paris Diderot -
  Paris~7.  June 2002.

\item
  Dr Paul-Andr\'e Melli\`es at Laboratoire PPS, Universit\'e Paris Diderot
  - Paris~7.  June 2001.

\item
  Mart\'{i}n Abadi at Bell Labs Research, Lucent Technologies, Palo Alto.
  In October--November 1999. 

\item
  Prof Gordon Plotkin at LFCS, Department of Computer Science, University
  of Edinburgh.  In November--December 1998.

\item
  Prof Marta Bunge at Department of Mathematics and Statistics, McGill
  University.  September and October 1998.

\item
  Prof Glynn Winskel at Department of Computer Science, Aarhus University,
  Denmark.  In August--September 1997.
\end{myitemize}

\vspace*{-1mm}
\subsection*{\S\enspace\thinspace Hosted research visitors}
\vspace*{-2mm}
\begin{myitemize}
\item[]\small\ at the Computer Laboratory, University of Cambridge
\end{myitemize}
\pagebreak

\noindent
PhD Student Ki Yung Ahn 
(Portland State University) in September 2012; 
%
Dr Michael Warren (IAS School of Mathematics, Princeton) in May 2012; 
%
Associate Prof Iliano Cervesato (Carnegie Mellon University - Qatar
Campus) in July 2011; 
%
PhD Student Guillaume Munch-Maccagnoni, (Universit\'e Paris Diderot -
Paris~7) March–May 2011 and in October 2012;
%
Prof Pierre-Louis Curien (Universit\'e Paris Diderot - Paris~7)
April--June 2009, April--June 2010, and in October 2012; 
%
Assistant Prof Makoto Hamana (Gunma University) in September 2009, March
2010 and October 2012;
%
Ichiro Hasuo (RIMS, Kyoto University) in November 2009;
%
Mat\'{\i}as Menni (Lifia, Universidad Nacional de La Plata) in July 2007
and April 2009;
%
Prof Franck van Breugel (York University, Toronto) sabbatical leave in
2004--05.

\vspace*{-0mm}
\subsection*{\S\enspace\thinspace Peer-reviewing}

\begin{myitemize}
\item
  \textbf{\em Programme committees}: CT 2013, ICALP 2013, Wollic 2012,
  MFPS XXVIII, MFCS 2012, URC 2010, SOS 2009, FOSSACS 2009, CSL 08, ICALP
  2008, MFPS XXIII (Chair), LICS 2007, SOFSEM 07, MSFP’06, CT 2006, MFPS
  XXI, ICALP’05, TCS 2004, CTCS 2004, CMCIM’03, WAIT’2003 (Co-chair),
  LICS’2003, FOSSACS’2002, MFPS XVII, LICS’2000, WAIT’98.

\item
  Reviewer for national and international conferences and journals.

\item
  Reviewer for the Uruguayan National Agency for Research and Innovation
  (ANII).

\item
  EPSRC Computer Science Panel Member.  February~1, 2005.
\end{myitemize}

\vspace*{-0mm}
\subsection*{\S\enspace\thinspace Memberships}

\begin{myitemize}
\item
  Fellow of \textbf{\em Christ's College}, University of Cambridge.  Since
  October 2001.

\item
  \textbf{\em EPSRC Peer Review College} member. October 2000 to September
  2010.

\item
  Editorial Board member of \textbf{\em Applied Categorical Structures}.
  Since October 2004.

\item
  Editorial Board member of the \textbf{\em Bulletin of Symbolic Logic}
  reviews for the \emph{Association for Symbolic Logic}.  January 2009 to
  December 2011.
\end{myitemize}

\vspace*{-0mm}
\subsection*{\S\enspace\thinspace Teaching}
\begin{myitemize}
\item Sundry undergraduate and postgraduate lecturing at the Computer
  Laboratory, University of Cambridge.  
\item Director of Studies for Computer Science in Christ's College
  Cambridge.  October 2001 to September 2011.  
\item Sundry undergraduate lecturing and tutorials at the School of
  Cognitive and Computing Sciences, University of Sussex. 
\item Sundry postgraduate lecturing at the Laboratory for Foundations of
  Computer Science, University of Edinburgh.
\end{myitemize}

\subsection*{\S\enspace\thinspace Administration}

\begin{myitemize}
\item
Sundry committees at the University of Cambridge and Christ's College
Cambridge.
\end{myitemize}

\clearpage
\setcounter{page}{1}
\chead{Part\,B1(c)}
\twocolumn[\begin{@twocolumnfalse}
	\hfill{\bfseries\Large 
    10-Year Track Record
  }\hfill\null\\
\end{@twocolumnfalse}]

%\subsection*{\S\enspace\thinspace Personal statement}

My research is in theoretical computer science, and by its own nature develops
in the context of close specialised collaboration.  My work stems from the
interaction between two approaches: abstract and concrete. At the abstract
level, I build mathematical theories. At the concrete level, theories are
applied to specific problems. This duality has yielded deep results and
provided research breadth, giving fundamental contributions to a variety of
fields and finding connections among them; \eg~domain theory, category theory,
concurrency theory, sheaf theory, type theory, logic, algebra.  Indeed,
note that I publish in computer science and mathematics venues, and I have
been invited speaker at both computer science and mathematics meetings.
My \mbox{h-index} is 21, according to Google scholar.  

\subsection*{\S\enspace\thinspace Award}

\begin{mybigitemize}
\item%[$\star$]
\textbf{\em 10-Year Most Influential PPDP Paper Award} for the article
\emph{Semantic Analysis of Normalisation by Evaluation for Typed Lambda
Calculus} in the 2002 International Symposium on Principles and Practice
of Declarative Programming.
\end{mybigitemize}

\subsection*{\S\enspace\thinspace Selected %peer-reviewed 
  publications}
\vspace*{-1.125mm}
\begin{myitemize}
\item[]\small\ 
  Number of citations provided according to Google
  \\\mbox{} 
  scholar.  Conference publications further selected for
  \\\mbox{} 
  journal proceedings have been circled.  Representative 
  \\\mbox{} 
  publications have been starred.
\end{myitemize}

\paragraph*{Research monograph}

\begin{mybigitemize}
\item[$\star$] 
  M.\,Fiore (2004). \emph{Axiomatic Domain Theory in Categories of Partial
    Maps}.  Distinguished Dissertations in Computer Science, No.\,14.
  Paperback Edition, Cambridge University
  Press.\mbox{}\hfill{\small[111~citations]}
\end{mybigitemize}

\paragraph*{Journals}

\begin{mybigitemize}
\item[$\obullet$]
  M.\,Fiore and C.-K.\,Hur (2011).  On the mathematical synthesis of
  equational logics.   In \emph{Selected Papers of the Conference
    ``Typed Lambda Calculi and Applications 2009''}.  \emph{Logical
    Methods in Computer Science}, Volume 7, ISSUE 3, PAPER
  12.\\\mbox{}\hfill{\small[19 (conference version) citations]}

%\item
%  M.\,Fiore and T.\,Leinster (2010).  An abstract characterization of
%  Thompson’s group F.  \emph{Semigroup Forum}, Volume 80, Number 2,
%  325--340.\\\mbox{}\hfill{\small[1~citation]}

\item[$\star$]
  M.\,Fiore and S.\,Staton (2009).  A congruence rule format for
  name-passing process calculi.  In \emph{Special Issue on Structural
    Operational Semantics (SOS)}. \emph{Information and Computation},
  207(2):209-236.\\\mbox{}\hfill{\small[26 (conference version) + 5~citations]}

\item[$\obullet$]
  M.\,Fiore and C.-K.\,Hur (2008).  On the construction of free
  algebras for equational systems.  In \emph{Special Issue for the
    Thirty-fourth International Colloquium on Automata, Languages and
    Programming (ICALP'07).  Theoretical Computer Science},
  410:1704--1729.\\\mbox{}\hfill{\small[15 (conference version) + 11 citations]}

\item[$\star$]
  M.\,Fiore, N.\,Gambino, M.\,Hyland, and G.\,Winskel (2008).  The
  cartesian closed bicategory of generalised species of structures. 
  \emph{Journal of the London Mathematical Society},
  77:203-220.\mbox{}\hfill{\small[19~citations]}

%\item
%  G.L.\,Cattani and M.\,Fiore (2007).  The bicategory-theoretic
%  solution of recursive domain equations.  In \emph{Computation,
%    Meaning, and Logic: Articles dedicated to Gordon Plotkin, Electronic Notes
%    in Theoretical Computer Science}, Volume 172,
%  pp\,203--222.\\\mbox{}\hfill{\small[0~citations]}

\item[$\obullet$]
  M.\,Fiore and S.\,Staton (2006).  
  Comparing operational models of
  name-passing process calculi.  \emph{Information and Computation},
  Vol\,204, Issue 4, pp\,524--560.\\\mbox{}\hfill{\small[15 (conference version)
    + 29~citations]}

\item[$\obullet$]
  M.\,Fiore, R.\,Di Cosmo, and V.\,Balat (2006).  Remarks on
  isomorphisms in typed lambda calculi with empty and sum types. 
  \emph{Annals of Pure and Applied Logic}, Vol\,141, Issues 1--2,
  pp\,35--50.\\\mbox{}\hfill{\small[19 (conference version) + 11 citations]}

%\item
%  M.\,Fiore and M.\,Menni (2005).  Reflective Kleisli subcategories
%  of the category of Eilenberg-Moore algebras for factorization monads.
%   In \emph{Proceedings of the International Category Theory
%    Conference (CT2004), Theory and Applications of Categories}, Vol\,15,
%  No\,2, pp\,40--65.\\\mbox{}\hfill{\small[3~citations]}

\item
  M.\,Fiore and T.\,Leinster (2005).  Objects of categories as
  complex numbers.  \emph{Advances in Mathematics},
  190(2):264--277.\mbox{}\hfill{\small[11~citations]} 

\item
  M.\,Fiore and T.\,Leinster (2004).  An Objective Representation of
  the Gaussian Integers.  \emph{Journal of Symbolic Computation},
  37(6): 707--716.\mbox{}\hfill{\small[5~citations]} 

\item[$\ostar$]
  M.\,Fiore, E.\,Moggi, and D.\,Sangiorgi (2002).  A fully-abstract
  model for the pi-calculus.  Information and Computation,
  179:76--117.\\\mbox{}\hfill{\small[95 (conference version) + 30 citations]}
\end{mybigitemize}

\paragraph*{Conferences}

\begin{mybigitemize}
\item[$\star$]
  M.\,Fiore (2012).  Discrete Generalised Polynomial Functors.  In
  \emph{Automata, Languages, and Programming Conference (ICALP'12)}.
  Volume 7392 of Lecture Notes in Computer Science,
  pp.\,214--226.\\\mbox{}\hfill{\small[0~citations]}

%\item
%  M.\,Hamana and M.\,Fiore (2011).  A foundation for GADTs and
%  Inductive Families: Dependent polynomial functor approach.  In
%  \emph{ACM SIGPLAN Seventh Workshop on Generic Programming (WGP'11)},
%  pp.\,59--70.  ACM Press.\\\mbox{}\hfill{\small[0~citations]}

\item[$\obullet$]
  M.\,Fiore and O.\,Mahmoud (2010).  Second-order algebraic theories.
   In \emph{Proceedings of the Thirty-fifth International Symposium
    on Mathematical Foundations of Computer Science (MFCS'10)}.  
  Volume 6281 of Lecture Notes in Computer Science,
  pp\,368--380.\mbox{}\hfill{\small[7~citations]}

\item[$\ostar$]
  M.\,Fiore and C.-K.\,Hur (2010).  Second-order equational logic. In
  \emph{Proceedings of the Nineteenth EACSL Annual Conference on Computer
    Science Logic (CSL'10)}.  Volume 6247 of Lecture Notes in
  Computer Science, pp\,320--335.\mbox{}\hfill{\small[11~citations]}

%\item
%  M.\,Fiore and S.\,Staton (2010).  Positive structural operational
%  semantics and monotone distributive laws.  \emph{In Short
%    Contributions for the Tenth International Workshop on Coalgebraic Methods
%    in Computer Science (CMCS 2010)}.  CWI Technical report SEN-1004,
%  pp\,8--9.\\\mbox{}\hfill{\small[0~citations]}

%%\item
%%  M.\,Fiore and C.-K.\,Hur (2008).   Term equational systems and
%%  logics.  In \emph{Proceedings of the Twenty-fourth Conference on
%%    the Mathematical Foundations of Programming Semantics (MFPS XXIV)}.
%%   Electronic Notes in Theoretical Computer Science, volume 218,
%%  pp\,171--192.\\\mbox{}\hfill{\small[19~citations]}

\item[$\star$]
  M.\,Fiore (2008).  Second-order and dependently-sorted abstract
  syntax.  In \emph{Logic in Computer Science Conference (LICS'08)},
  pp.\,57--68.  IEEE, Computer Society
  Press.\mbox{}\hfill{\small[22~citations]}

%%\item
%%  M.\,Fiore and C.-K.\,Hur (2007).  Equational systems and free
%%  constructions.  In \emph{International Colloquium on Automata,
%%    Language and Programming (ICALP 2007)}. Volume 4596 of Lecture Notes in
%%  Computer Science, pp\,607--619.\\\mbox{}\hfill{\small[15~citations]}

\item[$\obullet$]
  M.\,Fiore (2007).  Differential structure in models of
  multiplicative biadditive intuitionistic linear logic.  In
  \emph{Typed Lambda Calculi and Applications (TLCA'07)}.  Volume
  4583 of Lecture Notes in Computer Science,
  pp\,163--177.\mbox{}\hfill{\small[14~citations]}

%%\item
%%  M.\,Fiore and S.\,Staton (2006).  A congruence rule format for
%%  name-passing process calculi from mathematical structural operational
%%  semantics.  In \emph{Twenty-first Logic in Computer Science
%%    Conference (LICS'06)}, pp\,49--58.  IEEE, Computer Society
%%  Press.\\\mbox{}\hfill{\small[26~citations]}

%%\item
%%  M.\,Fiore and S.\,Staton (2004).  Comparing operational models of
%%  name-passing process calculi.  In \emph{Proceedings of the 7th
%%  Coalgebraic Methods in Computer Science Workshop (CMCS'04)}. 
%%  Volume 106 of Electronic Notes in Theoretical Computer Science, pages
%%  91--104.  Elsevier.\mbox{}\hfill{\small[29 citations]}

\item
  M.\,Fiore (2004).  Isomorphisms of generic recursive polynomial
  types.   In \emph{Thirty-first Symposium on Principles of
    Programming Languages (POPL'04)}, pp\,64--76.  ACM
  Press.\mbox{}\hfill{\small[22~citations]}

\item[$\star$]
  V.\,Balat, R.\,Di Cosmo, and M.\,Fiore (2004).  Extensional
  normalisation and type-directed partial evaluation for typed lambda
  calculus with sums.   
  In \emph{Thirty-first Symposium on Principles
    of Programming Languages (POPL'04)}, pp\,64--76.  ACM
  Press.\mbox{}\hfill{\small[47~citations]}

\item[$\star$]
  M.\,Fiore (2002).  Semantic analysis of normalisation by evaluation
  for typed lambda calculus.  In \emph{Fourth Principles and Practice
    of Declarative Programming Conference (PPDP'02)}, pages 26--37. 
  ACM Press.\mbox{}\hfill{\small[45~citations]}

%%\item
%%  M.\,Fiore, R.\,Di Cosmo, and V.\,Balat (2002).  Remarks on
%%  isomorphisms in typed lambda calculi with empty and sum types.  In
%%  \emph{Seventeenth Logic in Computer Science Conference (LICS'02)},
%%  pp\,147--156.   IEEE, Computer Society
%%  Press.\\\mbox{}\hfill{\small[19~citations]}
\end{mybigitemize}

\paragraph*{Invited}

\begin{mybigitemize}
\item[$\star$]
  M.\,Fiore (2005).  Mathematical models of computational and combinatorial
  structures.  {\em Foundations of Software Science and Computation Structures
    (FOSSACS'05)}.  Volume 3441 of Lecture Notes in Computer Science,
  pp\,25--46.\mbox{}\hfill{\small[18~citations]}
\end{mybigitemize}

\subsection*{\S\enspace\thinspace Invited addresses}
\vspace*{-1.125mm}
\begin{myitemize}
\item[]\small\ 
  Salient invitations have been starred.
\end{myitemize}


\begin{mybigitemize}
\item[$\star$]
  Indexing Structures in Programming Language Semantics and Design.
  \emph{Fourteenth International Symposium on Principles and Practice of
    Declarative Programming~(PPDP'12)}, 
  %10-Year Most Influential PPDP Paper Award talk, 
  Leuven (Belgium), September~2012.

\item
  Lie in Logic.  \emph{S\'{e}minaire CHoCoLa}, \'Ecole Normale Sup\'erieure de
  Lyon (France), May~2012.

\item[$\star$]
  Second-order algebra and generalised polynomial functors.  \emph{Logic and
    Interactions 2012}, Centre International de Rencontres Math\'{e}matiques,
  Luminy (France), February~2012.

\item
  Estructuras matem\'{a}ticas en lenguajes de programaci\'{o}n.
  \emph{French-Argentinean Laboratoire Internationale Associ\'e LIA INFINIS},
  %inaugural meeting, 
  Departamento de Computaci\'{o}n, Facultad de Ciencias Exactas y Naturales -
  Universidad de Buenos Aires (Argentina), December~2011.

\item
  On Higher-Order Algebra.   \emph{Third Scottish Category Theory Seminar},
  University of Strathclyde, Glasgow (Scotland), December~2010.

\item
  Algebraic simple type theory.  \emph{Curry-Howard and Concurrency Theory
    (CHoCo) S\'eminaire}, l'Ecole Normale Sup\'erieure de Lyon (France),
  December 2009.

\item[$\star$]
  Mathematical synthesis of equational deduction systems.  \emph{Ninth
    International Conference on Typed Lambda Calculi and Applications
    (TLCA'09)}, Brasilia (Brazil), June 2009.  

\item
  Algebraic Type Theory.  \emph{Eighty-eighth Peripatetic Seminar on Sheaves
    and Logic (88th PSSL), Celebrating the 60th birthdays of Martin Hyland and
    Peter Johnstone}, University of Cambridge (UK), April~2009.

\item[$\star$]
  Algebraic theories and equational logics.  \emph{Twenty-fourth Conference on
    the Mathematical Foundations of Programming Semantics (MFPS~XXIV)},
  University of Pennsylvania~(USA), May~2008.  

\item
  Second-order and dependently-sorted algebraic theories.  \emph{Workshop on
    Categorical and Homotopical Structures in Proof Theory}, Centre de Recerca
  Matem\`atica, %~(CRM), 
  Barcelona~(Spain), February 2008.

\item
  An axiomatics and a combinatorial model of creation/annihilation operators
  and differential structure.  \emph{Categorical Quantum Logic
    Workshop (CQL)}, %Computing Laboratory, 
  University of Oxford (UK), August 2007.

\item
  A mathematical theory of substitution and its applications to syntax and
  semantics.  \emph{Coalgebraic Logic Workshop (CoL)}, %Computing Laboratory,
  University of Oxford (UK), August 2007.

\item[$\star$]
  Towards a mathematical theory of substitution.  \emph{Annual International
    Conference on Category Theory (CT'07)}, Carvoeiro, Algarve (Portugal),
  June 2007.
 
\item[$\star$]
  A mathematical theory of substitution and its applications to syntax and
  semantics.  \emph{Workshop on Mathematical Theories of Abstraction,
    Substitution and Naming in Computer Science}, International Centre for
  Mathematical Sciences, %(ICMS), 
  Edinburgh (Scotland), May~2007.
 
\item[$\star$]
  Analytic functors and domain theory.  \emph{Symposium for Gordon Plotkin},
  Laboratory for Foundations of Computer Science, %~(LFCS), 
  University of Edinburgh (Scotland), September~2006.
 
\item[$\star$]
  On the structure of substitution. \emph{Twenty-second Mathematical
    Foundations of Programming Semantics Conference (MFPS XXII)}, %DISI,
  University of Genova (Italy), May~2006.
 
\item
  Isomorphisms on generic mutually recursive polynomial types.  \emph{Second
    International Workshop on Isomorphisms of Types}, Universit\'e de Toulouse
  (France), October 2005.

\item[$\star$]
  Mathematical models of computational and combinatorial structures.  
  \emph{Foundations of Software Science and Computation Structures
  (FOSSACS'05)} for European Joint Conferences on Theory and Practice of
  Software (ETAPS'05), Edinburgh~(UK), April 2005.  

\item[$\star$]
  Computational semantics.  \emph{Winter School on Semantics and
    Applications}, Instituto de Computaci\'on, Facultad de Ingenier\'{\i}a,
  Universidad de la Rep\'ublica, Montevideo~(Uruguay), July 2003.

\item
  A semantic framework for name and value passing process calculi.
  \emph{Workshop on Concurrency and Mobility, Logic and Foundations of
    Computation}, Fields Institute Summer School, University of Ottawa
  (Canada), June 2003.

\item[$\star$]
  Imaginary types.  \emph{Nineteenth Conference on the Mathematical
    Foundations of Programming Semantics (MFPS~XIX)}, Montr\'eal~(Canada),
  March~2003.

\item
  Foundational theories of combinatorial type structures.  \emph{Birmingham,
    Leicester, and Nottingham Christmas Theory Day}, Department of Computer
  Science, University of Birmingham~(UK), December~2002.
\end{mybigitemize}
\vspace*{-6mm}
\end{document}
